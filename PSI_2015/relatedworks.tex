\section{Related Works}

Our parsing algorithm is based on a RNGLR-algorithm presented by Elizabeth Scott 
and Adrian Johnstone in~\cite{RNGLR}. In order to better understand the paper, a reader 
should be familiar to its principles of work, so we briefly describe RNGLR-algorithm 
in this section.  Also we point out differences between our approach and existing
tools which operate with regular approximation of string-embedded language since
we use such type of approximation as input for our algorithm.

% Abstract parsing -- Doh

\subsection{Regular Approximation of Sting-Embedded Language}
Some tools are aimed to build high quality regular approximation. For example, 
Stranger~\cite{Stranger} which use forward reachability analysis to compute 
over-approximation of all string values for program. Further analysis in Stranger 
is based on patterns detection in approximation or generation finite subset of 
strings for analyzing with standalone tools. Implementation of our algorithm may 
use such tools as input generator.

Paper~\cite{JSA} presents Java String Analyzer (JSA) � tool for static syntax 
correctness checking of embedded SQL statements.  This tool build regular approximation 
with Mohri-Nederhof~\cite{MohriNederhof} algorithm and then check its inclusion into reference grammar 
without parsing and forest construction.
 
Our algorithm is inspired by Alvor~\cite{Alvor} which apply GLR-based technique 
for syntax correctness checking of regular approximation. Key difference of our 
algorithm is building of parse forest finite representation. 

\subsection{RNGLR}
Generalized LR parsing algorithm was presented by Masaru Tomita~\cite{Tomita}
as a solution for natural language processing and was intended to handle ambiguous
context-free grammars. Ambiguities of grammar produce Shift/Reduce and 
Reduce/Reduce conflicts. The algorithm uses parser tables similar to classical LR-tables,
each cell of which can contain multiple actions in case of conflicts. The general approach of 
the algorithm is to carry out all possible actions in these situations. 

However, Tomita's algorithm failed to process general context-free grammars.  
Elizabeth Scott and Adrian Johnstone presented Right-Nulled Generalized LR algorithm~\cite{RNGLR}
which extends and corrects Tomita's GLR parsing methods by
specific way of handling \emph{right nullable} rules (i.e. rules of the form 
$\mathrm{A} \rightarrow \alpha \beta$, where $\beta$ reduces to the empty string). 
That is, not only reductions for items $\mathrm{A} \rightarrow \alpha \cdot$ are 
applied, but also for the items of the form  $\mathrm{A} \rightarrow \alpha \cdot 
\beta$, where $\beta \Rightarrow \epsilon$. Thus, reduction length -- the number of 
symbols to be reduced to a nonterminal -- may be less than or equal to the length 
of righthand side of the rule. There are also possible reductions of 0-length, 
also called as $\epsilon$-reductions, corresponding to items of the form $\mathrm{A} 
\rightarrow \cdot$. 

To represent the set of stacks produced during conflict processing efficiently,
RNGLR algorithm uses Graph Structured Stack. GSS is an ordered graph, 
vertices of which corresponds to elements of classical stack and edges link sequential 
elements together. Each vertex can have multiple incoming edges and by means of 
it be shared between several stacks. Vertex is a pair $(s, l)$, where $s$ is a 
parser state and $l$ is a level -- position in an input string. Vertices in GSS 
are unique and there is no multiple edges. GSS construction routine is illustrated with 
\emph{addVertex} and \emph{addEdge} functions in Algorithm~\ref{rnglr}.

RNGLR-algorithm reads an input from left to right, one token at a time, and 
constructs levels of GSS sequentially for each position in the input. In the 
main loop of the algorithm for each token from the input, firstly, all possible 
reductions are applied (see \emph{reduce} function in Algorithm~\ref{rnglr}), and then the next token 
is shifted (see \emph{push} function in Algorithm~\ref{rnglr}).
\begin{algorithm}[!ht]
\begin{algorithmic}[1]
\caption{RNGLR algorithm}
\label{rnglr}
  
\Function{addVertex}{$level, state$}
  \If{GSS does not contain vertex $v = (level, state)$}
    \State{add new vertex $v = (level, state)$ to GSS}
    \State{calculate the set of shifts by $v$ and the next token and add them to $\mathcal{Q}$}
    \State{calculate the set of zero-reductions by $v$ and the next token and add them to $\mathcal{R}$}
  \EndIf
  \State{\Return{$v$}}
\EndFunction

\Function{addEdge}{$v_{h}, level_{t}, state_{t}, isZeroReduction$}
  \State{$v_{t} \gets$ \Call{addVertex}{$level_{t}, state_{t}$}}
  \If{GSS does not contain edge from $v_{t}$ to $v_{h}$}
    \State{add new edge from $v_{t}$ to $v_{h}$ to GSS}
    \If{not $isZeroReduction$}
      \State{calculate the set of reductions by $v$ and the next token and add them to $\mathcal{R}$}
    \EndIf
  \EndIf
\EndFunction

\Function{reduce}{}
  \While{$\mathcal{R}$ is not empty}
    \State{$(v, N, l) \gets \mathcal{R}.Dequeue()$}
    \State{find the set $\mathcal{X}$ of vertices reachable from $v$ along the path of length $(l-1)$, or length $0$ if $l=0$}
    \ForAll{$v_{h} = (level_{h}, state_{h})$ in $\mathcal{X}$}
      \State{$state_{t} \gets$ calculate new state by $state_{h}$ and nonterminal $N$}
      \State{\Call{addEdge}{$v_{h}, v.level, state_{tail}, (l=0)$}}
    \EndFor
  \EndWhile
\EndFunction

\Function{push}{}
  \State{$\mathcal{Q^{'}} \gets$ copy $\mathcal{Q}$}
  \While{$\mathcal{Q^{'}}$ is not empty}
    \State{$(v, state) \gets \mathcal{Q}.Dequeue()$}
    \State{\Call{addEdge}{$v, v.level + 1, state, false$}}
  \EndWhile
\EndFunction

\end{algorithmic}
\end{algorithm}
