\documentclass{beamer}
\usepackage{beamerthemesplit}
\usetheme{SPbGU}
%{CambridgeUS}
% Выпишем часть возможных стилей, некоторые из них могут содержать
% дополнительные опции
% Darmstadt, Ilmenau, CambridgeUS, default, Bergen, Madrid, AnnArbor,Pittsburg, Rochester,
% Antiles, Montpellier, Berkley, Berlin
\usepackage{pdfpages}
\usepackage{amsmath}
\usepackage{cmap} % for serchable pdf's
\usepackage[T2A]{fontenc} 
\usepackage[utf8]{inputenc}
\usepackage[english,russian]{babel}
\usepackage{indentfirst}
\usepackage{amsmath}
\usepackage{tikz}
\usepackage{graphicx}
\usetikzlibrary{shapes,arrows}
% Если у вас есть логотип вашей кафедры, факультета или университета, то
% его можно включить в презентацию.

%\usefoottemplate{\vbox{}}%  \tinycolouredline{structure!25}% {\color{white}\textbf{\insertshortauthor\hfill% \insertshortinstitute}}% \tinycolouredline{structure}% {\color{white}\textbf{\insertshorttitle}\hfill}% }}

%\logo{\includegraphics[width=1cm]{SPbGU_Logo.png}}

%[GLR-анализатор]
\title[РСР]{Разработка средств реинжиниринга}
\subtitle[курсовая]{Курсовая работа}
\institute[СПбГУ]{
Санкт-Петербургский государственный университет \\
Математико-Механический факультет \\
Кафедра системного программирования }



\author[С.В. Григорьев]{студент 361 гр. С.В. Григорьев \\
  \and  
  {\bfseries Научный руководитель:} к.ф.-м.н. А.С. Лукичёв \\ 
  \and
  {\bfseries Рецензент:} д.ф.-м.н., проф. Б.К. Мартыненко  
}

\date{10 декабря 2011г.}

\begin{document}
{

\begin{frame}
\begin{center}
{\includegraphics[width=1cm]{SPbGU_Logo.png}}
\end{center}
\titlepage
\end{frame}
}

\begin{frame}
    \transwipe[direction=90]
    \frametitle{Введение/Область применения}
    О том, как это всё важно и нужно
\end{frame}

\begin{frame}
    \transwipe[direction=90]
    \frametitle{Задача}
    Разработать генератор синтаксических анализаторов
    \begin{itemize}
        \item Работа с произвольными контекстно-свободными грамматиками
        \item Поддержка EBNF-грамматик
        \item Поддержка s-атрибутных грамматик
    \end{itemize}
\end{frame}

\begin{frame}
    \transwipe[direction=90]
    \frametitle{Алгоритм}
        \author[12]{adads}
        \institute[СПбГУ]{asdasdasdasd}
     GLR-анализатор предназначен для работы с произвольной {\bfseries{\underline {(в том числе неоднозначной!)}}} КС  грамматикой
    \begin{itemize}
        \item Для однозначных грамматик работает за линейное время
    \item {$O(n^{3})$ в худшем случае}
    \end{itemize}
    Рассмотренные подходы:
    \begin{itemize}
            \item Алгоритм Эрли             
            \item Алгоритм Томиты               
            \item Рекурсивно-восходящий алгоритм                
    \end{itemize}                   
\end{frame}

\begin{frame}
    \transwipe[direction=90]
    \frametitle{EBNF-грамматики}
    Конструкции регулярных выражений в правых частях правил
  \\
    Пример грамматики: 
        \begin{itemize}
            \item $S \rightarrow A(+A)*$
            \item $A \rightarrow a$
        \end{itemize}
    Преобразованная грамматика:
  \begin{itemize}
        \item $S \rightarrow AB$
        \item $A \rightarrow a$
        \item $B \rightarrow +AB$       
        \item $B \rightarrow \varepsilon$
    \end{itemize}
    Входная цепочка: a+a

\end{frame}

\begin{frame}[t]
    \transwipe[direction=90]
    \frametitle{EBNF-грамматики}
   
\end{frame}

\begin{frame}
    \transwipe[direction=90]
    \frametitle{Вычисление атрибутов}
  
\end{frame}

\begin{frame}
    \transwipe[direction=90]
    \frametitle{Вычисление атрибутов}
  \begin{itemize}
   \item
    Конечный автомат с помеченными переходами:\\
   \item
    Трасса автомата: \\
    $[(SeqS,1);$\\             
    $\phantom \qquad(ClsS,1);$\\
    $\phantom \qquad \qquad \qquad (SeqS,2); (LeafS,4); \ 'a'; (LeafE,4); ... (SeqE,2);$ \\ 
    $\phantom \qquad (ClsE,1); $\\
    $\phantom \qquad (LeafS,4); \ 'c'; (LeafE,4);$\\
    $(SeqE,1)]$
   \item
    По трассе строится дерево разбора
  \end{itemize}
\end{frame}

\begin{frame}
    \transwipe[direction=90]
    \frametitle{Реализация}
    \begin{itemize}
        \item Язык реализации -- F\#
    \item Фронтенд -- YARD -- разработка кафедры системного программирования
    \end{itemize}   
\end{frame}


\begin{frame}
    \transwipe[direction=90]
    \frametitle{Результаты}
    \begin{itemize}
    \item Реализован прототип GLR-анализатора
    \begin{itemize}
        \item По однозначной LR-грамматике строится анализатор с линейной сложностью
        \item По неоднозначной грамматике строится анализатор, возвращающий все возможные деревья вывода для данной входной цепочки
        \item Реализует поддержку EBNF-грамматик без их преобразования
        \item Реализует вычисление s-атрибутов
    \end{itemize}
    \item  Участие в конкурсе-конференции "Технологии Microsoft в теории и практике программирования"
    \begin{itemize}
        \item Тезисы опубликованы в сборнике материалов конкурса-конференции   
    \end{itemize}   
    \end{itemize}   
\end{frame}

\end{document}
