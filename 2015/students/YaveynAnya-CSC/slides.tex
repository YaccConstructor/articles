\documentclass[18pt,pdf,hyperref={unicode}]{beamer}
\usepackage[english,russian]{babel}

\usepackage[T2A]{fontenc}
\usepackage[utf8]{inputenc}
\usepackage{graphicx}
\usepackage{slashbox}
\usepackage{multirow}
\usepackage{xcolor,colortbl}
\usepackage{listings}

\usepackage{paratype}
\renewcommand*\familydefault{\sfdefault}

\usepackage{tikz}
\usetikzlibrary{snakes,arrows,shapes}
\usepackage{amsmath}

\usetikzlibrary{automata}


\newcommand{\E}{\ensuremath{\mathbb{E}}}
\newcommand{\D}{\ensuremath{\mathbb{D}}}
\renewcommand{\P}{\ensuremath{\mathbb{P}}}
\newcommand{\COV}{\ensuremath{\mathbb{C}\mathrm{ov}}}
\newcommand{\N}{\ensuremath{\mathrm{N}}}
\newcommand{\X}{\ensuremath{\mathbf{X}}}
\newcommand{\Y}{\ensuremath{\mathbf{Y}}}
\newcommand{\CHI}{\ensuremath{\mathcal{X}}}
\newcommand{\LAW}{\ensuremath{\mathcal{L}}}

\newcommand{\mom}[1]{\ensuremath{m_{#1}}}
\newcommand{\hmom}[1]{\ensuremath{\widehat{m}_{#1}}}

\newcommand{\summ}[2]{\ensuremath{\sum\limits_{#1}^{#2}}}
\newcommand{\SL}{\ensuremath{\sigma_{lim}}}
\newcommand{\HSL}{\ensuremath{\widehat{\sigma}_{lim}}}
\newcommand{\SLT}{\ensuremath{\sigma_{lim}^2}}
\newcommand{\HSLT}{\ensuremath{\widehat{\sigma}_{lim}^2}}
\newcommand{\RH}{\ensuremath{\rho(\X_n, \Y_n)}}
\newcommand{\HRH}{\ensuremath{\widehat{\rho}(\X_n, \Y_n)}}
\newcommand{\SE}{\ensuremath{\mathrm{\mathbf{SE}}}}
\newcommand{\SEM}{\ensuremath{\mathrm{\mathbf{SEM}}}}
\newcommand{\SEN}{\ensuremath{\mathrm{\mathbf{SE(N)}}}}
\newcommand{\SEMN}{\ensuremath{\mathrm{\mathbf{SEM(N)}}}}

\usepackage[customcolors]{hf-tikz}
\definecolor{mlisting}{RGB}{237,237,237}
\definecolor{mblue}{RGB}{180,180,255}
\definecolor{mred}{RGB}{255,180,180}

\newcommand{\listing}[1]{\texttt{\colorbox{mlisting}{#1}}}
\newcommand{\foottt}[1]{\texttt{\footnotesize {#1}}}
\newcommand{\smalltt}[1]{\texttt{\small {#1}}}
\newcommand{\blue}[1]{\textcolor{blue}{#1}}
\newcommand{\red}[1]{\textcolor{red}{#1}}

\usetheme{Berlin}
\usecolortheme{seahorse}

\parindent=0mm
% \parskip=.10in

\setbeamertemplate{headline}{}

\beamertemplatenavigationsymbolsempty

\setbeamertemplate{footline}
{
  \leavevmode
  \hbox{%
  \begin{beamercolorbox}[wd=.38\paperwidth,ht=2.25ex,dp=1ex,center]{author in head/foot}%
    \usebeamerfont{author in head/foot}\insertshortauthor\ \ %
  \end{beamercolorbox}%
  \begin{beamercolorbox}[wd=.38\paperwidth,ht=2.25ex,dp=1ex,left]{title in head/foot}%
    \usebeamerfont{title in head/foot}\ \insertshorttitle
  \end{beamercolorbox}%
  \begin{beamercolorbox}[wd=.24\paperwidth,ht=2.25ex,dp=1ex,right]{date in head/foot}%
    \usebeamerfont{date in head/foot}\hspace*{2em}
    \insertframenumber{} / \inserttotalframenumber\hspace*{2ex}
  \end{beamercolorbox}}%
  \vskip0pt%
}


\title[Строковые операции]{\textbf{Поддержка строковых операций в лексическом анализе динамически формируемого кода}}
\subtitle{}
\author[Анна Явейн]{
  Анна Явейн \\ \vskip 15pt
  {
    Куратор: Екатерина Вербицкая
  }
}
\date[декабрь 2015]{2015}


\begin{document}


\frame{\titlepage}

\makeatletter
\begin{frame}
  \frametitle{YaccConstructor}
  \begin{itemize}
    \item Динамический SQL \\
      \vskip15pt
      \foottt{
        \blue{IF} @X = @Y \\
          \hskip17pt\blue{SET} @TBL = \red{' table1 '} \\
        \blue{ELSE} \\
          \hskip17pt\blue{SET} @TBL = \red{' table2 '} \\
        \blue{SET}\,@S\,=\,\red{'SELECT\,x\,FROM'}\,+\,@TBL\,+\,\red{'WHERE ISNULL(n,0)\,>\,1'} \\
        \blue{EXECUTE} (@S) \\
      }
      \vskip15pt
    \item Встроенный SQL \\
      \vskip15pt
      \foottt{
        \blue{SqlCommand} myCommand = new \blue{SqlCommand}(\\
          \hskip17pt\red{''SELECT $\ast$ FROM table WHERE Column = @Param2''},\\
            \hskip17ptmyConnection);\\
        myCommand.Parameters.Add(myParam2);
      }
  \end{itemize}
\end{frame}
\makeatother

\begin{frame}
\frametitle{Лексический анализ и строковые операции}
На входе:
  \smalltt{
    \hskip60pt\blue{string} res = \red{``''}; \\
    \hskip60pt\hskip49pt\blue{for} (\blue{int} i = 0; i $<$ l; $++$i) \\
      \hskip60pt\hskip49pt\hskip25pt res = \red{``()''} + res; \\
    \hskip60pt\hskip49pt\blue{do\_something\_with}(res); \\
  }\vskip20pt
Множество
  \vskip-7pt
    \hskip110pt\smalltt{\{\red{``''}, \red{``()''}, \dots, \red{``()''}\textsuperscript{N}\}} \\
\vskip-7pt значений:
  \vskip25pt
Аппроксимация:
  \vskip19pt
    \hskip111pt\smalltt{(\red{``()''})*}
  \vskip-31pt
  \hskip200pt
{\small
  \begin{tikzpicture}[->,>=stealth',shorten >=1pt,auto,node distance=1.4cm, semithick, initial text=]
    \tikzstyle{every state}=[fill=red!60,draw=none,text=white, minimum size = 10pt]
    \tikzstyle{accepting}=[fill=blue!60,draw=none,text=white, minimum size = 10pt]

    \node[state,initial,accepting]       (A)              {};
    \node[state]                         (B) [below of=A] {};

    \path (B) edge    [bend left=40]          node {\smalltt{\red{``)''}}} (A)
          (A) edge    [bend left=40]          node {\smalltt{\red{``(''}}} (B);
  \end{tikzpicture}
}
\end{frame}


\begin{frame}<1-4>[label=past]
\frametitle{Что было}
\begin{itemize}
  \item<1-> Конкатенация.
  \smallskip
  \item<2-> Объединение.
  \smallskip
  \item<3-> Пересечение.
  \smallskip
  \item<4-> Избыточная аппроксимация для \listing{replace}.
\end{itemize}
\end{frame}


\begin{frame}
\frametitle{Зачем нужен replace}

На входе:
\smalltt{
  \hskip60pt \blue{var} col = \red{``a\_b\_c''}; \\
  \hskip60pt\hskip49pt replace(col, \red{``\_''}, \red{`` ''}); \\
  \hskip60pt\hskip49pt db.execute(\red{``select''} + col + \\
      \hskip60pt\hskip175pt \red{``from tbl''});
}
\vskip40pt
\smalltt{
  \red{``select a\_b\_c from tbl''} -- identifier \\\vskip25pt
  \red{``select a b\_c from tbl''} -- identifier + alias \\\vskip25pt
  \red{``select a b c from tbl''} -- error
}
\end{frame}


\begin{frame}
\frametitle{Избыточная аппроксимация для replace}
\begin{block}{\vskip-1pt\texttt{replace(F, P, T) = R}\\\vskip3pt}
  \smallskip
  $\omega \in$ \texttt{R} тогда и только тогда, когда
  \smallskip
  \begin{itemize}
    \item $\exists\, \omega^\prime \in$ \texttt{F}:
        $\;\;\omega^\prime = \omega_1x_1\omega_2x_2\dots\omega_kx_k\omega_{k+1}, x_i \in$ \texttt{P}.
    \item $\omega_i$ не содежат подстроки из \texttt{P}.
    \item $\omega = \omega_1c_1\omega_2c_2\dots\omega_kc_k\omega_{k+1},\; c_i \in$ \texttt{T}.
  \end{itemize}
  \smallskip
\end{block}
\end{frame}

\begin{frame}
\frametitle{Избыточная аппроксимация для replace}
\bigskip

\texttt{F}:
\vskip-19pt
\hskip60pt
{\small
  \begin{tikzpicture}[->,>=stealth',shorten >=1pt,auto,node distance=1.4cm, semithick, initial text=]
    \tikzstyle{every state}=[fill=red!60,draw=none,text=white, minimum size = 10pt]
    \tikzstyle{accepting}=[fill=blue!60,draw=none,text=white, minimum size = 10pt]

    \node[initial,state]           (A)              {};
    \node[state]                   (B) [right of=A] {};
    \node[state]                   (C) [right of=B] {};
    \node[state]                   (D) [right of=C] {};
    \node[state,accepting]         (E) [right of=D] {};

    \path (A) edge              node {b} (B)
          (B) edge              node {a} (C)
          (C) edge              node {a} (D)
          (D) edge              node {b} (E);
  \end{tikzpicture}
  \bigskip
  \bigskip
}

\texttt{P}:
\vskip-17pt
\hskip60pt
{\small
  \begin{tikzpicture}[->,>=stealth',shorten >=1pt,auto,node distance=1.4cm, semithick, initial text=]
    \tikzstyle{every state}=[fill=red!60,draw=none,text=white, minimum size = 10pt]
    \tikzstyle{accepting}=[fill=blue!60,draw=none,text=white, minimum size = 10pt]

    \node[initial,state]           (A)              {};
    \node[state,accepting]         (B) [right of=A] {};

    \path (A) edge                node {a} (B)
          (B) edge  [loop right]  node {a} (B);
  \end{tikzpicture}
  \bigskip
  \bigskip
}

\texttt{T}:
\vskip-17pt
\hskip60pt
{\small
  \begin{tikzpicture}[->,>=stealth',shorten >=1pt,auto,node distance=1.4cm, semithick, initial text=]
    \tikzstyle{every state}=[fill=red!60,draw=none,text=white, minimum size = 10pt]
    \tikzstyle{accepting}=[fill=blue!60,draw=none,text=white, minimum size = 10pt]

    \node[initial,state]           (A)              {};
    \node[state,accepting]         (B) [right of=A] {};

    \path (A) edge                node {m} (B);
  \end{tikzpicture}
  \bigskip
  \bigskip
  \bigskip
  \bigskip
}

\texttt{R}:
\vskip-19pt
\hskip60pt
{\small
  \begin{tikzpicture}[->,>=stealth',shorten >=1pt,auto,node distance=1.4cm, semithick, initial text=]
    \tikzstyle{every state}=[fill=red!60,draw=none,text=white, minimum size = 10pt]
    \tikzstyle{accepting}=[fill=blue!60,draw=none,text=white, minimum size = 10pt]

    \node[initial,state]           (A)              {};
    \node[state]                   (B) [right of=A] {};
    \node[state]                   (C) [right of=B] {};
    \node[state]                   (D) [right of=C] {};
    \node[state,accepting]         (E) [right of=D] {};

    \path (A) edge              node {b} (B)
          (B) edge              node {m} (C)
          (C) edge              node {m} (D)
          (D) edge              node {b} (E)
          (B) edge [bend right=70] node {m} (D);
  \end{tikzpicture}
}
\end{frame}


\begin{frame}<1>[label=present]
\frametitle{Возможные сужения аппроксимации}
\begin{itemize}
  \item<1-> Жадная (\textbf{greedy}) семантика.
  \smallskip
  \item<2-> Ленивая (\textbf{reluctant}) семантика.
  \bigskip
  \smallskip
  \item<3-> Самое левое совпадение (\textbf{leftmost}).
  \bigskip
  \smallskip
  \item<4-> Замена только первого вхождения (\textbf{replaceFirst}).
\end{itemize}
\end{frame}


\begin{frame}
\frametitle{Greedy replace}
\bigskip

\texttt{F}:
\vskip-19pt
\hskip60pt
{\small
  \begin{tikzpicture}[->,>=stealth',shorten >=1pt,auto,node distance=1.4cm, semithick, initial text=]
    \tikzstyle{every state}=[fill=red!60,draw=none,text=white, minimum size = 10pt]
    \tikzstyle{accepting}=[fill=blue!60,draw=none,text=white, minimum size = 10pt]

    \node[initial,state]           (A)              {};
    \node[state]                   (B) [right of=A] {};
    \node[state]                   (C) [right of=B] {};
    \node[state]                   (D) [right of=C] {};
    \node[state,accepting]         (E) [right of=D] {};

    \path (A) edge              node {b} (B)
          (B) edge              node {a} (C)
          (C) edge              node {a} (D)
          (D) edge              node {b} (E);
  \end{tikzpicture}
  \bigskip
  \bigskip
}

\texttt{P}:
\vskip-17pt
\hskip60pt
{\small
  \begin{tikzpicture}[->,>=stealth',shorten >=1pt,auto,node distance=1.4cm, semithick, initial text=]
    \tikzstyle{every state}=[fill=red!60,draw=none,text=white, minimum size = 10pt]
    \tikzstyle{accepting}=[fill=blue!60,draw=none,text=white, minimum size = 10pt]

    \node[initial,state]           (A)              {};
    \node[state,accepting]         (B) [right of=A] {};

    \path (A) edge                node {a} (B)
          (B) edge  [loop right]  node {a} (B);
  \end{tikzpicture}
  \bigskip
  \bigskip
}

\texttt{T}:
\vskip-17pt
\hskip60pt
{\small
  \begin{tikzpicture}[->,>=stealth',shorten >=1pt,auto,node distance=1.4cm, semithick, initial text=]
    \tikzstyle{every state}=[fill=red!60,draw=none,text=white, minimum size = 10pt]
    \tikzstyle{accepting}=[fill=blue!60,draw=none,text=white, minimum size = 10pt]

    \node[initial,state]           (A)              {};
    \node[state,accepting]         (B) [right of=A] {};

    \path (A) edge                node {m} (B);
  \end{tikzpicture}
  \bigskip
  \bigskip
  \bigskip
  \bigskip
}

\texttt{R}:\hfill greedy \hspace{10pt}
\vskip-19pt
\hskip60pt
{\small
  \begin{tikzpicture}[->,>=stealth',shorten >=1pt,auto,node distance=1.4cm, semithick, initial text=]
    \tikzstyle{every state}=[fill=red!60,draw=none,text=white, minimum size = 10pt]
    \tikzstyle{accepting}=[fill=blue!60,draw=none,text=white, minimum size = 10pt]

    \node[initial,state]           (A)              {};
    \node[state]                   (B) [right of=A] {};
    \node[state, fill=white]       (C) [right of=B] {};
    \node[state]                   (D) [right of=C] {};
    \node[state,accepting]         (E) [right of=D] {};

    \path (A) edge              node {b} (B)
          (D) edge              node {b} (E)
          (B) edge [bend right=70] node {m} (D);
  \end{tikzpicture}
}
\end{frame}


\againframe<2>{present}


\begin{frame}
\frametitle{Reluctant replace}
\bigskip

\texttt{F}:
\vskip-19pt
\hskip60pt
{\small
  \begin{tikzpicture}[->,>=stealth',shorten >=1pt,auto,node distance=1.4cm, semithick, initial text=]
    \tikzstyle{every state}=[fill=red!60,draw=none,text=white, minimum size = 10pt]
    \tikzstyle{accepting}=[fill=blue!60,draw=none,text=white, minimum size = 10pt]

    \node[initial,state]           (A)              {};
    \node[state]                   (B) [right of=A] {};
    \node[state]                   (C) [right of=B] {};
    \node[state]                   (D) [right of=C] {};
    \node[state,accepting]         (E) [right of=D] {};

    \path (A) edge              node {b} (B)
          (B) edge              node {a} (C)
          (C) edge              node {a} (D)
          (D) edge              node {b} (E);
  \end{tikzpicture}
  \bigskip
  \bigskip
}

\texttt{P}:
\vskip-17pt
\hskip60pt
{\small
  \begin{tikzpicture}[->,>=stealth',shorten >=1pt,auto,node distance=1.4cm, semithick, initial text=]
    \tikzstyle{every state}=[fill=red!60,draw=none,text=white, minimum size = 10pt]
    \tikzstyle{accepting}=[fill=blue!60,draw=none,text=white, minimum size = 10pt]

    \node[initial,state]           (A)              {};
    \node[state,accepting]         (B) [right of=A] {};

    \path (A) edge                node {a} (B)
          (B) edge  [loop right]  node {a} (B);
  \end{tikzpicture}
  \bigskip
  \bigskip
}

\texttt{T}:
\vskip-17pt
\hskip60pt
{\small
  \begin{tikzpicture}[->,>=stealth',shorten >=1pt,auto,node distance=1.4cm, semithick, initial text=]
    \tikzstyle{every state}=[fill=red!60,draw=none,text=white, minimum size = 10pt]
    \tikzstyle{accepting}=[fill=blue!60,draw=none,text=white, minimum size = 10pt]

    \node[initial,state]           (A)              {};
    \node[state,accepting]         (B) [right of=A] {};

    \path (A) edge                node {m} (B);
  \end{tikzpicture}
  \bigskip
  \bigskip
  \bigskip
  \bigskip
}

\texttt{R}:\hfill reluctant \hspace{7pt}
\vskip-19pt
\hskip60pt
{\small
  \begin{tikzpicture}[->,>=stealth',shorten >=1pt,auto,node distance=1.4cm, semithick, initial text=]
    \tikzstyle{every state}=[fill=red!60,draw=none,text=white, minimum size = 10pt]
    \tikzstyle{accepting}=[fill=blue!60,draw=none,text=white, minimum size = 10pt]

    \node[initial,state]           (A)              {};
    \node[state]                   (B) [right of=A] {};
    \node[state]                   (C) [right of=B] {};
    \node[state]                   (D) [right of=C] {};
    \node[state,accepting]         (E) [right of=D] {};

    \path (A) edge              node {b} (B)
          (B) edge              node {m} (C)
          (C) edge              node {m} (D)
          (D) edge              node {b} (E);
  \end{tikzpicture}
}
\end{frame}


\againframe<3>{present}


\begin{frame}
\frametitle{Leftmost replace}
\smallskip

\texttt{F}:
\vskip-19pt
\hskip60pt
{\small
  \begin{tikzpicture}[->,>=stealth',shorten >=1pt,auto,node distance=1.4cm, semithick, initial text=]
    \tikzstyle{every state}=[fill=red!60,draw=none,text=white, minimum size = 10pt]
    \tikzstyle{accepting}=[fill=blue!60,draw=none,text=white, minimum size = 10pt]

    \node[initial,state]           (A)              {};
    \node[state]                   (B) [right of=A] {};
    \node[state]                   (C) [right of=B] {};
    \node[state,accepting]         (D) [right of=C] {};

    \path (A) edge              node {a} (B)
          (B) edge              node {b} (C)
          (C) edge              node {b} (D);
  \end{tikzpicture}
  \bigskip
  \bigskip
}

\texttt{P}:
\vskip-17pt
\hskip60pt
{\small
  \begin{tikzpicture}[->,>=stealth',shorten >=1pt,auto,node distance=1.4cm, semithick, initial text=]
    \tikzstyle{every state}=[fill=red!60,draw=none,text=white, minimum size = 10pt]
    \tikzstyle{accepting}=[fill=blue!60,draw=none,text=white, minimum size = 10pt]

    \node[initial,state]           (A)              {};
    \node[state]                   (B) [right of=A] {};
    \node[state,accepting]         (C) [right of=B] {};

    \path (A) edge                 node {a} (B)
          (B) edge                 node {b} (C)
          (A) edge [bend right=70] node {b} (C)
          (C) edge [loop right]    node {b} (C);
  \end{tikzpicture}
  \bigskip
  \bigskip
  \bigskip
}


\texttt{R}:
\vskip-19pt
\hskip60pt
{\small
  \begin{tikzpicture}[->,>=stealth',shorten >=1pt,auto,node distance=1.4cm, semithick, initial text=]
    \tikzstyle{every state}=[fill=red!60,draw=none,text=white, minimum size = 10pt]
    \tikzstyle{accepting}=[fill=blue!60,draw=none,text=white, minimum size = 10pt]

    \node[initial,state]           (A)              {};
    \node[state]                   (B) [right of=A] {};
    \node[state,accepting]         (C) [right of=B] {};
    \node[state,accepting]         (D) [right of=C] {};
    \node[state,accepting]         (E) [below of=B] {};
    \node[state,accepting]         (F) [below of=C] {};

    \path (A) edge              node {a} (B)
          (B) edge              node {m} (C)
          (C) edge              node {m} (D)
          (A) edge [bend right=45] node {m} (E)
          (E) edge              node {m} (F);
  \end{tikzpicture}
  \bigskip
  \bigskip
}

\texttt{R}:\hfill leftmost \hspace{10pt}
\vskip-19pt
\hskip60pt
{\small
  \begin{tikzpicture}[->,>=stealth',shorten >=1pt,auto,node distance=1.4cm, semithick, initial text=]
    \tikzstyle{every state}=[fill=red!60,draw=none,text=white, minimum size = 10pt]
    \tikzstyle{accepting}=[fill=blue!60,draw=none,text=white, minimum size = 10pt]

    \node[state, initial]          (B)              {};
    \node[state,accepting]         (C) [right of=B] {};
    \node[state,accepting]         (D) [right of=C] {};

    \path (B) edge              node {m} (C)
          (C) edge              node {m} (D);
  \end{tikzpicture}
}
\end{frame}


\begin{frame}
\frametitle{Leftmost replace}
\smallskip

\texttt{F}:
\vskip-19pt
\hskip60pt
{\small
  \begin{tikzpicture}[->,>=stealth',shorten >=1pt,auto,node distance=1.4cm, semithick, initial text=]
    \tikzstyle{every state}=[fill=red!60,draw=none,text=white, minimum size = 10pt]
    \tikzstyle{accepting}=[fill=blue!60,draw=none,text=white, minimum size = 10pt]

    \node[initial,state]           (A)              {};
    \node[state]                   (B) [right of=A] {};
    \node[state]                   (C) [right of=B] {};
    \node[state,accepting]         (D) [right of=C] {};

    \path (A) edge              node {a} (B)
          (B) edge              node {b} (C)
          (C) edge              node {b} (D);
  \end{tikzpicture}
  \bigskip
  \bigskip
}


\texttt{P}:
\vskip-17pt
\hskip60pt
{\small
  \begin{tikzpicture}[->,>=stealth',shorten >=1pt,auto,node distance=1.4cm, semithick, initial text=]
    \tikzstyle{every state}=[fill=red!60,draw=none,text=white, minimum size = 10pt]
    \tikzstyle{accepting}=[fill=blue!60,draw=none,text=white, minimum size = 10pt]

    \node[initial,state]           (A)              {};
    \node[state]         (B) [right of=A] {};
    \node[state,accepting]         (C) [right of=B] {};

    \path (A) edge                 node {a} (B)
          (B) edge                 node {b} (C)
          (A) edge [bend right=70] node {b} (C)
          (C) edge [loop right]    node {b} (C);
  \end{tikzpicture}
  \bigskip
  \bigskip
  \bigskip
}

\texttt{R}:\hfill leftmost + greedy \hspace{10pt}
\vskip-19pt
\hskip60pt
{\small
  \begin{tikzpicture}[->,>=stealth',shorten >=1pt,auto,node distance=1.4cm, semithick, initial text=]
    \tikzstyle{every state}=[fill=red!60,draw=none,text=white, minimum size = 10pt]
    \tikzstyle{accepting}=[fill=blue!60,draw=none,text=white, minimum size = 10pt]

    \node[state, initial]          (B)              {};
    \node[state,accepting]         (C) [right of=B] {};

    \path (B) edge              node {m} (C);
  \end{tikzpicture}
  \bigskip
  \bigskip
}

\texttt{R}:\hfill leftmost + reluctant \hspace{10pt}
\vskip-19pt
\hskip60pt
{\small
  \begin{tikzpicture}[->,>=stealth',shorten >=1pt,auto,node distance=1.4cm, semithick, initial text=]
    \tikzstyle{every state}=[fill=red!60,draw=none,text=white, minimum size = 10pt]
    \tikzstyle{accepting}=[fill=blue!60,draw=none,text=white, minimum size = 10pt]

    \node[state, initial]          (B)              {};
    \node[state]                   (C) [right of=B] {};
    \node[state,accepting]         (D) [right of=C] {};

    \path (B) edge              node {m} (C)
          (C) edge              node {m} (D);
  \end{tikzpicture}
}
\end{frame}


\againframe<4>{present}


\begin{frame}
\frametitle{Что дальше?}

\texttt{F}:\hfill \colorbox{mred}{aba}\,a \hspace{130pt}
\bigskip
\bigskip

\texttt{P}:\hfill \colorbox{mred}{aba} | \colorbox{mblue}{abaa} \hspace{117pt}
\bigskip
\bigskip

\texttt{T}:\hfill m \hspace{140pt}
\bigskip
\bigskip
\bigskip
\bigskip


\texttt{R}:\hfill \colorbox{mred}{m}\,a \hspace{105pt} (PCRE)
\bigskip
\bigskip

\texttt{R}:\hfill \colorbox{mblue}{m} \hspace{100pt} (C++ ?!)


\end{frame}


\begin{frame}
\frametitle{Трудности и полученные знания}
\begin{itemize}
  \item Новый язык: F\#.
  \smallskip
  \item Регулярные выражения.
  \smallskip
  \item Автоматы и трансдьюсеры.
  \bigskip
  \bigskip
  \item Академический характер задачи.
  \smallskip
  \item Найденные исследования оказались неприменимы в моем случае.
\end{itemize}
\end{frame}


\begin{frame}
Спасибо за внимание.
\bigskip
\bigskip
\bigskip
\bigskip
\bigskip
\bigskip

Репозиторий на GitHub:
\medskip

\hskip20pt\small{\texttt{YaccConstructor/QuickGraph}}
\bigskip

\hskip20pt\small{\texttt{anya.yaveyn@yandex.ru}}
\end{frame}


\end{document}
