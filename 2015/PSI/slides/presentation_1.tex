\documentclass{beamer}
\usepackage{beamerthemesplit}
\usepackage{wrapfig}
\usetheme{SPbGU}
\usepackage{pdfpages}
\usepackage{amsmath}
\usepackage{cmap} 
\usepackage[T2A]{fontenc} 
\usepackage[utf8]{inputenc}
\usepackage[english,russian]{babel}
\usepackage{indentfirst}
\usepackage{amsmath}
\usepackage{tikz}
\usepackage{multirow}
\usepackage[noend]{algpseudocode}
\usepackage{algorithm}
\usepackage{algorithmicx}
\usetikzlibrary{shapes,arrows}
\usepackage{fancyvrb}
\newtheorem{rutheorem}{Theorem}
\newtheorem{ruproof}{Доказательство}
\newtheorem{rudefinition}{Определение}
\newtheorem{rulemma}{Лемма}
\beamertemplatenavigationsymbolsempty

\title[]{Relaxed Parsing of Regular Approximations of String-Embedded Languages}
%\subtitle[]{В рамках проекта лаборатории JetBrains}
\institute[SPbSU]{
Saint Petersburg State University \\
JetBrains Programming Languages and Tools Lab }

\author[Ekaterina Verbitskaia]{Ekaterina Verbitskaia
% \\
%  \and  
%    {\bfseries Научный руководитель:} ст.пр. С.В. Григорьев \\ 
%  \and
%    {\bfseries Рецензент:} программист "ИнтеллиДжей Лабс" А.А. Бреслав 
}

\date{26/08/2015}

\definecolor{orange}{RGB}{179,36,31}

\begin{document}
{


\begin{frame}[fragile]
\transwipe[direction=90]
\frametitle{Static analysis of string-embedded code: the scheme}
\begin{tabular}{p{5cm}|p{6cm}}
Code: hotspot is marked & Possible values
\\
\begin{minipage}{6cm}
  \begin{Verbatim}[commandchars=\\\{\}]

\textcolor{blue}{string} res = \textcolor{orange}{""};
\textcolor{blue}{for}(i = 0; i < l; i++) \{
    res = \textcolor{orange}{"()"} + res;
\}
\textcolor{red}{use(res);}   

  \end{Verbatim}
\end{minipage}
& 
\begin{minipage}{6cm}
  \begin{Verbatim}[commandchars=\\\{\}]

\{\textcolor{orange}{""},
 \textcolor{orange}{"()"},
 \textcolor{orange}{"()()"},
 ...
 \textcolor{orange}{"()"}^l,
\}
  \end{Verbatim}
\end{minipage}
\\ \hline 
Regular approximation & Finite automaton
\\
\begin{minipage}{6cm}
  \begin{Verbatim}[commandchars=\\\{\}]
(\textcolor{orange}{"()"})*
  \end{Verbatim} 
\end{minipage}
&
\\
&
f
\\
\end{tabular}
\end{frame}

\end{document}
