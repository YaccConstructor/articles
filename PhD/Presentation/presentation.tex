\documentclass{beamer}
\usepackage{beamerthemesplit}
\usetheme{SPbGU}
%{CambridgeUS}
% Выпишем часть возможных стилей, некоторые из них могут содержать
% дополнительные опции
% Darmstadt, Ilmenau, CambridgeUS, default, Bergen, Madrid, AnnArbor,Pittsburg, Rochester,
% Antiles, Montpellier, Berkley, Berlin
\usepackage{pdfpages}
\usepackage{amsmath}
\usepackage{cmap} % for serchable pdf's
\usepackage[T2A]{fontenc} 
\usepackage[utf8]{inputenc}
\usepackage[english,russian]{babel}
\usepackage{indentfirst}
\usepackage{amsmath}
\usepackage{dot2texi}
\usepackage{tikz}
\usepackage{graphicx}
\usetikzlibrary{shapes,arrows}
% Если у вас есть логотип вашей кафедры, факультета или университета, то
% его можно включить в презентацию.

%\usefoottemplate{\vbox{}}%  \tinycolouredline{structure!25}% {\color{white}\textbf{\insertshortauthor\hfill% \insertshortinstitute}}% \tinycolouredline{structure}% {\color{white}\textbf{\insertshorttitle}\hfill}% }}

%\logo{\includegraphics[width=1cm]{SPbGU_Logo.png}}

%[GLR-анализатор]
\title[]{Cинтаксический анализ динамически формируемых строковых выражений}
%\subtitle[студроект]{Студенческий проект}
\institute[СПбГУ]{
Санкт-Петербургский государственный университет \\
Математико-Механический факультет \\
Кафедра системного программирования }
%[Лукичёв А.С. Григорьев С.В.]


\author[Семён]{Григорьев Семён Вячеславович \\
  \and  
  {\bfseries Научный руководитель:} кандидат физико-математических наук, доцент Д.В. Кознов \\ 
%  \and
%  {\bfseries Рецензент:} д.ф.-м.н., проф. Б.К. Мартыненко  
}

\date{2015г.}

\begin{document}
{

\begin{frame}
\begin{center}
{\includegraphics[width=1cm]{SPbGU_Logo.png}}
\end{center}
\titlepage
\end{frame}
}

\begin{frame}
	\transwipe[direction=90]
	\frametitle{Динамически формируемые строковые выражения: пример}
\end{frame}

\begin{frame}
	\transwipe[direction=90]
	\frametitle{Динамически формируемые строковые выражения: проблемы}
\end{frame}

\begin{frame}
	\transwipe[direction=90]
	\frametitle{Обзор}
\end{frame}

\begin{frame}
	\transwipe[direction=90]
	\frametitle{Цели работы}
	\begin{itemize}
		\item Создать подхода к статическому синтаксическому анализу динамически формируемых выражений, позволяющего обрабатывать произвольную регулярную аппроксимацию без потери точности.
	        \item Разработать технологию для создания инструментов статического анализа встроенных текстовых языков, реализующей данный алгоритм.
	\end{itemize}
\end{frame}

\begin{frame}
	\transwipe[direction=90]
	\frametitle{Положения, выносимые на защиту}
        \begin{enumerate}
            \item Разработан алгоритм синтаксического анализа динамически формируемых выражений, позволяющий обрабатывать произвольную регулярную аппроксимацию множества значений выражения в точке выполнения, реализующий эффективное управление стеком и гарантирующий конечность представления леса вывода. Доказана завершаемость и корректность предложенного алгоритма.
            \item Создана архитектура инструментария для разработки программных средств синтаксического анализа динамически формируемых строковых выражений.
            \item Разработана методика анализа динамически формируемых строковых выражений в проектах по реинжинирингу информационных систем.  
\end{enumerate}
\end{frame}

\begin{frame}
	\transwipe[direction=90]
	\frametitle{Научная новизна}
        На текущий момент существует несколько подходов к анализу динамически формируемых строковых выражений. Некоторые из них, такие как JSA, предназначены только для проверки корректности выражений, основанной на решении задачи о включении одного языка в другой. Выполнение более сложных видов анализа, трансформаций или построения леса разбора не предполагается. В работах А. Бреслава и Кюнг-Гу Дох (Kyung-Goo Doh) рассматривается применение механизмов синтаксического анализа для работы с динамически формируемыми выражениями, однако не решается вопрос эффективного представления результатов разбора. Предложенный в диссертации алгоритм предназначен для синтаксического анализа динамически формируемых выражений и построения компактной структуры данных, содержащей для всех корректных значений выражения их деревья вывода. Это позволяет как проверять корректность анализируемых выражений, так и проводить более сложные виды анализа, используя деревья вывода, хранящиеся в построенной структуре данных.
 
        Большинство существовавших готовых инструментов для анализа динамически формируемых строковых выражений (JSA, PHPSA, Alvor и т.д.), как правило, предназначены для решения конкретных задач в рамках конкретных языков. Решение новых задач или поддержка других языков с помощью этих инструментов затруднено ввиду ограничений, накладываемых архитектурой и возможностями используемого алгоритма анализа. В рамках работы предложена архитектура инструментального средства, включающего предложенный алгоритм и позволяющего упростить создание новых инструментов для анализа динамически формируемых выражений на любых языках программирования (различные диалекты SQL, HTML, JSON и т.д.).

\end{frame}

\begin{frame}
	\transwipe[direction=90]
	\frametitle{Архитектура: постановка задачи}
\end{frame}

\begin{frame}
	\transwipe[direction=90]
	\frametitle{Архитектура}
\end{frame}

\begin{frame}
	\transwipe[direction=90]
	\frametitle{Алгоритм: постановка задачи}
\end{frame}

\begin{frame}
	\transwipe[direction=90]
	\frametitle{Алгоритм: описание}
\end{frame}

\begin{frame}
	\transwipe[direction=90]
	\frametitle{Алгоритм: завершаемость}
	\begin{theorem}


	\end{theorem}
\end{frame}

\begin{frame}
	\transwipe[direction=90]
	\frametitle{Алгоритм: корректность}
	\begin{theorem}


	\end{theorem}

\end{frame}

\begin{frame}
	\transwipe[direction=90]
	\frametitle{Алгоритм: пример работы}
\end{frame}

\begin{frame}[t]
	\transwipe[direction=90]
	\frametitle{Методология}
\end{frame}

\begin{frame}[t]
	\transwipe[direction=90]
	\frametitle{Апробация}
\end{frame}

\begin{frame}
	\transwipe[direction=90]
	\frametitle{Публикации}
  \begin{itemize}
	  \item  а
	  \item  и
  \end{itemize}	
\end{frame}

\end{document}
