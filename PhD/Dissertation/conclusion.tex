\chapter*{Заключение}                       % Заголовок
\addcontentsline{toc}{chapter}{Заключение}  % Добавляем его в оглавление

В ходе выполнения исследования получены следующие основные результаты.

\begin{enumerate}
    \item Разработан алгоритм синтаксического анализа динамически формируемых выражений, позволяющий обрабатывать произвольную регулярную аппроксимацию множества значений выражения в точке выполнения, реализующий 
    эффективное управление стеком и гарантирующий конечность представления леса вывода. Доказана завершаемость и корректность предложенного алгоритма при анализе регулярной аппроксимации, представимой в виде произвольного конечного автомата без $\varepsilon$-переходов.
    \item Создана архитектура инструментария для разработки программных средств статического анализа динамически формируемых строковых выражений.
    \item Разработан метод реинжиниринга встроенного программного кода в проектах по реинжинирингу информационных систем. Данный метод применён в проекте компании ЗАО ``Ланит-Терком'' по переносу информационной системы с MS-SQL Server на Oracle Server, для чего реализованы соответствующие программные компоненты.
\end{enumerate}

Кроме того, реализован инструментальный пакет для разработки средств статического анализа динамически формируемых выражений. На его основе реализован плагин для ReSharper. Код опубликован на сервисе GitHub: \\ \url{https://github.com/YaccConstructor/YaccConstructor} под лицензией Apache License Version 2.0, автор работал под учётной записью с именем gsvgit.

В дальнейшем планируется развитие платформы и плагина. На уровне платформы необходимо реализовать механизмы, требующиеся для трансформаций кода на встроенных языках. Механизмы трансформации встроенных языков требуются для проведения миграции с одной СУБД на другую~\cite{Syrcose} или для миграции на новые технологии, например, LINQ. Эта задача связана с двумя проблемами: возможностью проведения нетривиальных трансформаций и доказательство корректности трансформаций. Планируется реализация проверки корректности типов. Для SQL это должна быть как проверка типов внутри запроса, так и проверка того, что тип возвращаемого запросом результата соответствует типу хост-переменной, выделенной для сохранения результата в основном коде.

Более общей проблемой, подлежащей дальнейшему исследованию, является возможность выполнения семантических действий непосредственно над SPPF. Это необходимо для рефакторинга, улучшения качества трансляции, автоматизации перехода на более надёжные средства метапрограммирования~\cite{JSStagedMetaProgramming,JSStagedMetaProgrammingFull}.

Важной задачей является теоретическая оценка сложности предложенного алгоритма синтаксического анализа. В известных работах не приводится строгих оценок подобных алгоритмов, поэтому данная задача является самостоятельным исследованием.

С целью обобщения предложенного подхода к синтаксическому анализу, а также для получения лучшей производительности и возможностей для более качественной диагностики ошибок, планируется переход на алгоритм обобщённого LL-анализа (GLL)~\cite{GLL,AbstractGLL}. Планируется исследовать возможность улучшения предложенного алгоритма при переходе на другие алгоритмы обобщённого LR-анализа~\cite{GeneralisedlrBIG}, например, такие как BRNGLR~\cite{BRNGLR} и RIGLR~\cite{RIGLR}.

Кроме того, важной задачей является реализация диагностики ошибок, решение которогй для обобщённого восходящего анализа активно исследуется~\cite{RNGLRSyntaxerror1,RNGLRSyntaxerror2, RNGLRSyntaxerror3, RNGLRSyntaxerror4}. Адоптация предложенных решений для пименеия в пердставленном алгоритме требует отдельной работы.

\clearpage
