\subsection*{\Large Общая характеристика работы}
\fontsize{14pt}{15pt}\selectfont
\subsubsection*{\large{Актуальность темы}}

Взаимодействие приложений, например, с базами данных, часто реализуется с помощью встроенных языков: приложение, созданное на одном языке, генерирует код на другом языке, и передаёт этот код на выполнение в соответствующее окружение. Примерами могут служить динамические SQL-запросы к базам данных в Java-коде или формирование HTML-страниц в PHP-приложениях. Генерируемый код собирается из строк таким образом, чтобы в момент выполнения результирующая строка представляла собой корректное выражение на соответствующем языке. Такой подход весьма гибок, так как позволяет использовать для формирования выражений различные строковые операции (replace, substring и т.д.) и получать части кода из различных источников (например, учитывать текстовый ввод пользователя, что часто используется для задания фильтров при конструировании SQL-запросов). Кроме того, использование динамически формируемых строковых выражений избавлено от дополнительных накладных расходов, присущих, например, таким технологиям, как ORM, что позволяет достичь высокой производительности. 

Однако динамически формируемые выражения часто конструируются посредством конкатенации в циклах, ветках условных операторов или рекурсивных процедурах, что приводит к получению множества различных значений для каждого выражения в момент выполнения. При этом фрагменты кода на встроенных языках воспринимаются компилятором исходного языка как простые строки, не подлежащие дополнительному анализу. Невозможность статической проверки корректности формируемого выражения приводит к высокой вероятности возникновения ошибок во время выполнения программы. В худшем случае такая ошибка не приведет к прекращению работы приложения, что указало бы на проблемы, однако целостность данных при этом может оказаться нарушена. Более того, использование динамически формируемых выражений затрудняет как разработку информационных систем, так и реижиниринг уже созданных. Для реинжинирига важно иметь возможность изучать систему и модифицировать её, сохраняя функциональность. Однако, например, при наличии в коде приложения встроенного SQL нельзя, не проанализировав все динамически формируемые выражения, точно ответить на вопрос о том, с какими элементами базы данных не взаимодействует система, и  удалить их. При переносе такой системы на другую СУБД необходимо гарантировать, что для всех динамически формируемых выражений значение в момент выполнения будет корректным кодом на языке новой СУБД. С другой стороны, распространённой практикой при написании кода является использование интегрированных сред разработки, производящих подсветку синтаксиса и автодополнение, сигнализирующих о синтаксических ошибках, предоставляющих возможность проводить рефакторинг кода. Такая функциональность значительно упрощают процесс разработки и отладки приложений и полезна не только для основного языка, но и для встроенных языков. Для решения таких задач необходимы инструменты, проводящие анализ множества выражений, которые могут быть получены на этапе исполнения из строковых выражений исходного языка.  

Проблема анализа встроенных языков активно исследуется. Большинство работ используют анализ регулярного множества (регулярной аппроксимации), приближающего множество значений динамически формируемого выражения. Как правило, рассматривается вопрос корректности генерируемых выражений или ищутся фрагменты кода, уязвимые для SQL-инъекций. Сильная специализация таких решений не позволяет применять их для других задач. В исследованиях Кюнг-Гу Дох (Kyung-Goo Doh) предлагается комбинация анализа потока данных и синтаксического анализа на основе LR-алгоритма и поднимается вопрос о семантическом анализе встроенных языков. Предлагается использовать классический для LR-анализа механизм атрибутных грамматик, однако, опускается вопрос ресурсоёмкости данного подхода при сложных видах анализа. В работах А. Бреслава рассматривается подход, основанный на построении регулярной аппроксимации множества возможных значений и последующем анализе с использованием обобщённого LR-алгоритма, что кроме расширения класса поддерживаемых языков даёт дополнительные преимущества при переиспользовании структур данных, характерных для обобщённого анализа. Однако, эффективное хранение результатов разбора и оптимизация управления стеком разбора не рассмотрены. Существует также ряд инструментов для работы с динамически формируемыми выражениями: Alvor и IntelliLang, предоставляющие поддержку встроенных языков в интегрированных средах разработки, JSA и PHPSA, позволяющие искать ошибки в выражениях на встроенных языках, SQLWays, поддерживающий трансформацию выражений на встроенных языках, SAFELI --- инструмент статического анализа, предназначенный для определения возможности SQL-инъекций в Web-приложениях и некоторые другие. Однако, эти инструменты либо не поддерживают часто встречающиеся на практике сложные способы формирования выражений, либо имеют существенные ограничения по функциональности: не поддерживают сложные способы формирования строковых выражений, решают только одну узкую задачу (проверка корректности, поиск уязвимых конструкций) и т.д. 

\subsubsection*{\large{Цель диссертационной работы}}

Целью данной работы является создание подхода к статическому синтаксическому анализу динамически формируемых выражений, позволяющего обрабатывать произвольную регулярную аппроксимацию без потери точности, а также разработать технологию для создания инструментов статического анализа встроенных текстовых языков, реализующей данный алгоритм.

\subsubsection*{\large{Результаты, выносимые на защиту}}
\begin{enumerate}
 \item Разработан алгоритм синтаксического анализа динамически формируемых выражений, позволяющий обрабатывать произвольную регулярную аппроксимацию множества значений выражения в точке выполнения, реализующий эффективное управление стеком и гарантирующий конечность представления леса вывода. Доказана завершаемость и корректность предложенного алгоритма при анализе регулярной аппроксимации, представимой в виде произвольного конечного автомата без эпсилон-переходов.
% \item Для хранения деревьев вывода бесконечного множества цепочек в памяти конечного размера, использована структура данных для компактного представления леса разбора, позволяющая восстанавливать деревья вывода.
 \item Создана архитектура инструментария для разработки программных средств синтаксического анализа динамически формируемых строковых выражений.
 \item Разработана методика анализа динамически формируемых строковых выражений в проектах по реинжинирингу информационных систем.  

\end{enumerate}

\subsubsection*{\large{Методы исследования}}

В работе используется алгоритм обобщённого восходящего синтаксического анализа RNGLR, созданный Элизабет Скотт (Elizabeth Scott) и Адриан Джонстон (Adrian Johnstone) из университета Royal Holloway (Великобритания). Для компактного хранения леса вывода использовалась структура Shared Packed Parse Forest (SPPF), которую предложил Ян Рекерс (Jan Rekers, University of Amsterdam).

	Доказательство завершаемости и корректности предложенного алгоритма проводилось с применением теории формальных языков, теории графов и теории сложности алгоритмов. Приближение множества значений динамически формируемого выражения строилось в виде регулярного множества, описываемого с помощью конечного автомата.

	Апробация созданного подхода проводилась в рамках промышленного проекта компании ЗАО “Ланит-Терком” (Россия) по переносу хранимого кода, содержащего большое количество динамического SQL, с MS SQL Server на Oracle Server. Предложенный в работе алгоритм апробирован в рамках инфраструктуры проекта ReSharper компании ООО “Интеллиджей Лабс” (Россия).

\subsubsection*{\large{Научная новизна}}

На текущий момент существует несколько подходов к анализу динамически формируемых строковых выражений. Некоторые из них, такие как JSA, предназначены только для проверки корректности выражений, основанной на решении задачи о включении одного языка в другой. Выполнение более сложных видов анализа, трансформаций или построения леса разбора не предполагается. В работах А. Бреслава и Кюнг-Гу Дох (Kyung-Goo Doh) рассматривается применение механизмов синтаксического анализа для работы с динамически формируемыми выражениями, однако не решается вопрос эффективного представления результатов разбора. Предложенный в диссертации алгоритм предназначен для синтаксического анализа динамически формируемых выражений и построения компактной структуры данных, содержащей для всех корректных значений выражения их деревья вывода. Это позволяет как проверять корректность анализируемых выражений, так и проводить более сложные виды анализов, используя деревья вывода, хранящиеся в построенной структуре данных.

Большинство существовавших готовых инструментов для анализа динамически формируемых строковых выражений (JSA, PHPSA, Alvor и т.д.), как правило, предназначены для решения конкретных задач в рамках конкретных языков. Решение новых задач или поддержка других языков с помощью этих инструментов затруднено ввиду ограничений, накладываемых архитектурой и возможностями используемого алгоритма анализа. В рамках работы предложена архитектура, учитывающая возможности предложенного алгоритма и позволяющая упростить создание новых инструментов для анализа динамически формируемых выражений.

\subsubsection*{\large{Практическая ценность}}

На основе полученных в работе научных результатов был разработан инструментарий (SDK), предназначенный для создания средств статического анализа динамически формируемых выражений. В данный инструментарий входят следующие компоненты: генератор лексических анализаторов, генератор синтаксических анализаторов, библиотеки времени выполнения, реализующие соответствующие алгоритмы анализа, набор интерфейсов и вспомогательных функций для реализации конечного инструмента. Набор генераторов позволяет по описанию лексики и синтаксиса языка строить синтаксический и лексический анализатор, обрабатывающий аппроксимацию множества значений динамически формируемого выражения на соответствующем языке, представленную в виде произвольного конечного автомата. Устранение эпсилон-переходов, необходимое для корректной работы синтаксического анализа, происходит на этапе лексического анализа.

Данный инструментарий позволяет автоматизировать создание лексических и синтаксических анализаторов при разработке программных средств, использующих регулярную аппроксимацию для приближения множества значений динамически формируемых выражений. Инструментарий может использоваться  для решения задач реижинирига --- изучения и инвентаризации систем, поиска ошибок в исходном коде, автоматизации трансформации выражений на встроенных языках. Также данный инструментарий может использоваться при реализации поддержки встроенных языков в интегрированных средах разработки.

Разработанная методика обработки динамического SQL основана на использовании инструментария в качестве генератора для создания лексического и синтаксического анализатора для динамически формируемых выражений по соответствующим спецификациям. В случае динамического SQL могут быть переиспользованы ранее разработанные спецификации. Построение регулярной аппроксимации выделяется в отдельный шаг и производится с помощью анализов, реализованных для обработки основного кода. После завершения синтаксического разбора, анализ леса проводится в основном с помощью тех же методов, что и анализ основного кода, что достигается за счёт идентичности структур деревьев. Данная методика может быть переиспользована для работы с произвольными встроенными текстовыми языками.


\subsubsection*{\large{Апробация работы и публикации}}

С использованием разработанного инструментария было реализовано расширение к инструменту ReSharper (компания JetBrains), предоставляющее поддержку встроенного T-SQL в проектах на языке программирования C\# в среде разработки Microsoft Visual Studio. Была реализована следующая функциональность: статическая проверка корректности выражений и подсветка ошибок, подсветка синтаксиса и подсветка парных элементов (\url{https://github.com/YaccConstructor/YaccConstructor}).

	Так же была проведена апробация результатов работы на промышленном проекте по переносу хранимого SQL-кода с MS-SQL Server 2005 на Oraclе 11gR2 (ЗАО ``Ланит-Терком''). Исходная система состояла из 850 хранимых процедур и содержала более 3000 динамических запросов на 2,7 млн. строк хранимого кода. Более половины динамических запросов были сложными и формировались с использованием от 7 до 212 операторов. При этом, среднее количество операторов для формирование запроса ---  40. Реализованный механизм позволил корректно автоматически обработать примерно 45\% запросов и существенно упростил ручную доработку системы. 

Основные результаты работы были доложены на ряде научно-практических конференциях: SECR-2012, SECR-2013, SECR-2014, TMPA-2014, Parsing@SLE-2013, Рабочий семинар “Наукоемкое программное обеспечение” при конференции PSI-2014. Доклад на SECR-2014 награждён премией Бертрана Мейера за лучшую исследовательскую работу в области программной инженерии. Разработка инструментальных средств на основе предложенного алгоритма была поддержана Фондом содействия развитию малых форм предприятий в технической сфере (программа УМНИК).
Результаты диссертации изложены в 6 научных работах из которых 3~\cite{1,2,3} опубликованы в журналах из списка ВАК. Работы~\cite{1, 2, 4, 5, 6} написаны в соавторстве. 

В~\cite{1} С. Григорьеву принадлежит реализация ядра платформы YaccConstructor. В~\cite{2} и~\cite{5} Григорьеву С. принадлежит постановка задачи, формулирование требований к разрабатываемым инструментальным средствам. В~\cite{4} автору принадлежит идея, описание и реализация анализа встроенных языков на основе RNGLR алгоритма.  В~\cite{6} Григорьеву С. принадлежит реализация инструментальных средств, проведение замеров, работа над текстом.



%\underline{\textbf{Объем и структура работы.}} Диссертация состоит из~введения, четырех глав, заключения и~приложения. Полный объем диссертации \textbf{ХХХ}~страниц текста с~\textbf{ХХ}~рисунками и~5~таблицами. Список литературы содержит \textbf{ХХX}~наименование.

%\newpage
%\subsection*{\Large Содержание работы}
%Во \underline{\textbf{введении}} обосновывается актуальность исследований, проводимых в рамках данной диссертационной работы, приводится обзор научной литературы по изучаемой проблеме, формулируется цель, ставятся задачи работы, сформулированы научная новизна и практическая значимость представляемой работы.

%\underline{\textbf{Первая глава}} посвящена ...

% картинку можно добавить так:
%\begin{figure}[h] 
%  \center
%  \includegraphics [scale=0.27] {latex}
%  \caption{Подпись к картинке.} 
%  \label{img:latex}
%\end{figure}

%Формулы в строку без номера добавляются так:
%$$
%  \lambda_{T_s} = K_x\frac{d{x}}{d{T_s}}, \qquad
%  \lambda_{q_s} = K_x\frac{d{x}}{d{q_s}},
%$$

%\underline{\textbf{Вторая глава}} посвящена исследованию 

%\underline{\textbf{Третья глава}} посвящена исследованию 

%В \underline{\textbf{четвертой главе}} приведено описание 

%В \underline{\textbf{заключении}} приведены основные результаты работы, которые заключаются в следующем:
%\begin{enumerate}
% \item Результат номер один.
% \item Результат номер два.
% \item Результат номер три.
%% и так далее, если нужно
%\end{enumerate}


\newpage
\renewcommand{\refname}{\Large Публикации автора по теме диссертации}
%\nocite{*}
%\bibliography{biblio}

\begin{thebibliography}{99}

\bibitem{1} Кириленко Я.А., Григорьев С. В., Авдюхин Д. А. Разработка синтаксических анализаторов в проектах по автоматизированному реинжинирингу информационных систем.  Научно-технические ведомости Санкт-Петербургского государственного политехнического университета информатика, телекоммуникации, управление. Т. 3, N 174, 2013. C. 94 --- 98.
\bibitem{2} Григорьев С. В., Вербицкая Е. А., Полубелова М. И., Иванов А. В., Мавчун Е. В. Инструментальная поддержка встроенных языков в интегрированных средах разработки. Моделирование и анализ информационных систем. Т. 21, N 6, 2014. С. 131---143.
\bibitem{3}Григорьев С.В. Алгоритм синтаксического анализа динамически формируемых выражений. Моделирование и анализ информационных систем. Т. 21, N 6, 2014. С. 131---143.

\bibitem{4}Semen Grigorev, Iakov Kirilenko. GLR-based abstract parsing. In Proceedings of the 9th Central \& Eastern European Software Engineering Conference in Russia (CEE-SECR ’13). 2013. ACM, New York, NY, USA. 1-9 p.
\bibitem{5}Semen Grigorev, Ekaterina Verbitskaia, Andrei Ivanov, Marina Polubelova, Ekaterina Mavchun. String-embedded language support in integrated development environment. In Proceedings of the 10th Central and Eastern European Software Engineering Conference in Russia (CEE-SECR '14). 2014. ACM, New York, NY, USA. 1-11 p.
\bibitem{6}Semen Grigorev, Iakov Kirilenko. From Abstract Parsing to Abstract Translation. Proceedings of the Spring/Summer Young Researchers' Colloquium on Software Engineering. 2014. Saint Petersburg, Russia. 1-5 p.

\end{thebibliography}
