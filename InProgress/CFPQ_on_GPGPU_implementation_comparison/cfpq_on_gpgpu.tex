%4 or 8 pages
%
% The first command in your LaTeX source must be the \documentclass command.
\documentclass[sigconf]{acmart}

\usepackage{algpseudocode}
\usepackage{algorithm}
\usepackage{algorithmicx}
\usepackage{verbatim}
\usepackage{mathtools}
\usepackage{multirow}
\usepackage{subcaption}

\usepackage{tikz}
\usetikzlibrary{automata,positioning}


%
% defining the \BibTeX command - from Oren Patashnik's original BibTeX documentation.
\def\BibTeX{{\rm B\kern-.05em{\sc i\kern-.025em b}\kern-.08emT\kern-.1667em\lower.7ex\hbox{E}\kern-.125emX}}

% Rights management information.
% This information is sent to you when you complete the rights form.
% These commands have SAMPLE values in them; it is your responsibility as an author to replace
% the commands and values with those provided to you when you complete the rights form.
%
% These commands are for a PROCEEDINGS abstract or paper.
\copyrightyear{2018}
\acmYear{2018}
\setcopyright{acmlicensed}
\acmConference[GRADES-NDA 2019]{GRADES-NDA 2019: the 2nd Joint International Workshop on Graph Data Management Experiences \& Systems (GRADES) and Network Data Analytics (NDA) 2019}{June 30, 2019}{Amsterdam, Netherlands}
%!!\acmBooktitle{Woodstock '18: ACM Symposium on Neural Gaze Detection, June 03--05, 2018, Woodstock, NY}
\acmPrice{15.00}
\acmDOI{10.1145/1122445.1122456}
\acmISBN{978-1-4503-9999-9/18/06}

%
% These commands are for a JOURNAL article.
%\setcopyright{acmcopyright}
%\acmJournal{TOG}
%\acmYear{2018}\acmVolume{37}\acmNumber{4}\acmArticle{111}\acmMonth{8}
%\acmDOI{10.1145/1122445.1122456}

%
% Submission ID.
% Use this when submitting an article to a sponsored event. You'll receive a unique submission ID from the organizers
% of the event, and this ID should be used as the parameter to this command.
%\acmSubmissionID{123-A56-BU3}

%
% The majority of ACM publications use numbered citations and references. If you are preparing content for an event
% sponsored by ACM SIGGRAPH, you must use the "author year" style of citations and references. Uncommenting
% the next command will enable that style.
%\citestyle{acmauthoryear}

\newcommand{\ltz}{$<0.001$}

%
% end of the preamble, start of the body of the document source.
\begin{document}

%
% The "title" command has an optional parameter, allowing the author to define a "short title" to be used in page headers.
\title[Evaluation of the CFPQ Algorithm Based on Matrix Multiplication]{Evaluation of the Context-Free Path Querying Algorithm Based~on~Matrix Multiplication}

%\and JetBrains Research, Universitetskaya emb., 7-9-11/5A \\ St.Petersburg, Russia \\


%
% The "author" command and its associated commands are used to define the authors and their affiliations.
% Of note is the shared affiliation of the first two authors, and the "authornote" and "authornotemark" commands
% used to denote shared contribution to the research.
\author{Nikita Mishin}
\email{mishinnikitam@gmail.com}
\author{Iaroslav Sokolov}
\email{sokolov.yas@gmail.com}
\author{Egor Spirin}
\email{egor@spirin.tech}
\affiliation{%
  \institution{Saint Petersburg State University}
  \streetaddress{7/9 Universitetskaya nab.}
  \city{St. Petersburg}
  \country{Russia}
  \postcode{199034}
}


\author{Vladimir Kutuev}
\email{vladimir.kutuev@gmail.com}
\author{Egor Nemchinov}
\email{nemchegor@gmail.com}
\author{Sergey Gorbatyuk}
\email{sergeygorbatyuk171@gmail.com}
\affiliation{%
  \institution{Saint Petersburg State University}
  \streetaddress{7/9 Universitetskaya nab.}
  \city{St. Petersburg}
  \country{Russia}
  \postcode{199034}
}


\author{Semyon Grigorev}
\email{s.v.grigoriev@spbu.ru}
\email{semen.grigorev@jetbrains.com}
\orcid{0000-0002-7966-0698}
\affiliation{
  \institution{Saint Petersburg State University}
  \streetaddress{7/9 Universitetskaya nab.}
  \city{St. Petersburg}
  \country{Russia}
  \postcode{199034}
}
\affiliation{
  \institution{JetBrains Research}
  \streetaddress{Universitetskaya emb., 7-9-11/5A}
  \city{St. Petersburg}
  \country{Russia}
  \postcode{199034}
}





%
% By default, the full list of authors will be used in the page headers. Often, this list is too long, and will overlap
% other information printed in the page headers. This command allows the author to define a more concise list
% of authors' names for this purpose.
\renewcommand{\shortauthors}{Mishin and Sokolov, et al.}

%
% The abstract is a short summary of the work to be presented in the article.
\begin{abstract}
Recently proposed matrix multiplication based algorithm for context-free path querying (CFPQ) offloads the most performance-critical parts onto boolean matrices multiplication. Thus, it is possible to utilize modern parallel hardware and software to achieve high performance of CFPQ easily. In this work, we provide results of empirical performance comparison of different implementations of this algorithm on both real data and synthetic data for the worst cases.  
 
%We collect and publish a dataset which includes both real data and syntatica data for worst cases.
%We provide a number of implementations of matrix-based CFPQ algorithm which utilize GPGPUs and high-perfprmance libraries for BMM, and compare performance of them.
%Our evaluation shows that !!! more then 10 times faster then even by naive !!!
\end{abstract}

%
% The code below is generated by the tool at http://dl.acm.org/ccs.cfm.
% Please copy and paste the code instead of the example below.
%
\begin{CCSXML}
<ccs2012>
<concept>
<concept_id>10002951.10002952.10003197.10010825</concept_id>
<concept_desc>Information systems~Query languages for non-relational engines</concept_desc>
<concept_significance>500</concept_significance>
</concept>
<concept>
<concept_id>10003752.10003766.10003771</concept_id>
<concept_desc>Theory of computation~Grammars and context-free languages</concept_desc>
<concept_significance>500</concept_significance>
</concept>
<concept>
<concept_id>10010147.10010169.10010170.10010174</concept_id>
<concept_desc>Computing methodologies~Massively parallel algorithms</concept_desc>
<concept_significance>500</concept_significance>
</concept>
<concept>
<concept_id>10003752.10003753.10003761.10003762</concept_id>
<concept_desc>Theory of computation~Parallel computing models</concept_desc>
<concept_significance>300</concept_significance>
</concept>
<concept>
<concept_id>10010520.10010521.10010528.10010534</concept_id>
<concept_desc>Computer systems organization~Single instruction, multiple data</concept_desc>
<concept_significance>300</concept_significance>
</concept>
</ccs2012>
\end{CCSXML}

\ccsdesc[500]{Information systems~Query languages for non-relational engines}
\ccsdesc[500]{Theory of computation~Grammars and context-free languages}
\ccsdesc[500]{Computing methodologies~Massively parallel algorithms}
\ccsdesc[300]{Theory of computation~Parallel computing models}
\ccsdesc[300]{Computer systems organization~Single instruction, multiple data}
%
% Keywords. The author(s) should pick words that accurately describe the work being
% presented. Separate the keywords with commas.
\keywords{Context-free path querying, transitive closure, graph databases, context-free grammar, GPGPU, CUDA, matrix multiplication, boolean matrix}


%
% This command processes the author and affiliation and title information and builds
% the first part of the formatted document.
\maketitle

\section{Introduction}

Scalable high-performance graph analysis is an actual challenge.
There is a big number of ways to attack this challenge~\cite{Coimbra2021} and the first promising idea is to utilize general-purpose graphic processing units (GPGPU).
Such existing solutions, as CuSha~\cite{10.1145/2600212.2600227} and Gunrock~\cite{7967137} show that utilization of GPUs can improve the performance of graph analysis, moreover it is shown that solutions may be scaled to multi-GPU systems.
But low flexibility and high complexity of API are problems of these solutions.

The second promising thing which provides a user-friendly API for high-performance graph analysis algorithms creation is a GraphBLAS API~\cite{7761646} which provides linear algebra based building blocks to create graph analysis algorithms.
The idea of GraphBLAS is based on a well-known fact that linear algebra operations can be efficiently implemented on parallel hardware.
Along with that, a graph can be natively represented using matrices: adjacency matrix, incidence matrix, etc.
While reference CPU-based implementation of GraphBLAS, SuiteSparse:GraphBLAS~\cite{10.1145/3322125}, demonstrates good performance in real-world tasks, GPU-based implementation is challenging.

One of the challenges in this way is that real data are often sparse, thus underlying matrices and vectors are also sparse, and, as a result, classical dense data structures and respective algorithms are inefficient. 
So, it is necessary to use advanced data structures and procedures to implement sparse linear algebra, but the efficient implementation of them on GPU is hard due to the irregularity of workload and data access patterns.
Though such well-known libraries as cuSPARSE show that sparse linear algebra operations can be efficiently implemented for GPGPU, it is not so trivial to implement GraphBLAS on GPGPU. 
First of all, it requires \textit{generic} sparse linear algebra, thus it is impossible just to reuse existing libraries which are almost all specified for operations over floats.
The second problem is specific optimizations, such as masking fusion, which can not be natively implemented on top of existing kernels.
Nevertheless, there is a number of implementations of GraphBLAS on GPGPU, such as GraphBLAST~\cite{yang2019graphblast}, GBTL~\cite{7529957}, which show that GPGPUs utilization can improve the performance of GraphBLAS-based graph analysis solutions.
But these solutions are not portable because they are based on Nvidia Cuda stack.
Moreover, the scalability problem is not solved: all these solutions support only single-GPU, not multi-GPU computations.

To provide portable GPU implementation of GraphBLAS API we developed a \textit{SPLA} library\footnote{Source code available at: \url{https://github.com/JetBrains-Research/spla}}.
This library utilizes OpenCL for GPGPU computing to be portable across devices of different vendors.
Moreover, it is initially designed to utilize multiple GPGPUs to be scalable.
To sum up, the contribution of this work is the following.
\begin{itemize}
    \item Design of portable GPU GraphBLAS implementation proposed. The design involves the utilization of multiple GPUS. Additionally, the proposed design is aimed to simplify library tuning and wrappers for different high-level platforms and languages creation. 
    \item Subset of GraphBLAS API, including such operations as masking, matrix-matrix multiplication, matrix-matrix e-wise addition, is implemented. The current implementation is limited by COO and CSR matrix representation format and uses basic algorithms for some operations, but work in progress and more data formats will be supported and advanced algorithms will be implemented in the future.
    \item Preliminary evaluation on such algorithms as breadth-first search (BFS) and triangles counting (TC), and real-world graphs shows portability across different vendors and promising performance: for some problems Spla is comparable with GraphBLAST. Surprisingly, for some problems, the proposed solution on embedded Intel graphic card shows better performance than SuiteSparse:GraphBLAS on the respective CPU. At the same time, the evaluation shows that further optimization is required.
\end{itemize} 
\section{Matrix-based Algorithm for CFPQ}

Formal description of matrix-based algorithm.

\subsection{Архитектура решения}
\paragraph{}

Архитектура предлагаемого решение приведена на рисунке \ref{fig:arch}.
\begin{figure}[h!]
\centering
\includegraphics[scale=0.5]{pictures/dia3.png}
\caption{Архитектура предлагаемой библиотеки GraphBLAS-sharp}
\label{fig:arch}
\end{figure}

В рамках данной работы были реализованы компоненты библиотеки GraphBLAS-sharp, описанные ниже.
\begin{itemize}
    \item \textbf{Модули коллекций Matrix и Vector}. Модули содержат абстрактные классы для представления матрицы и вектора, реализованные в виде размеченного объединения, а также конкретные форматы их хранения.
    \item \textbf{Модули операций Matrix и Vector}. Модули содержат определение операций стандарта GraphBLAS для работы с абстрактными коллекциями.
    \item \textbf{Модуль AlgebraicStructures}. Модуль содержит реализацию не\-ко\-то\-рых алгебраических структур, таких как моноид и полукольцо.
    \item \textbf{Модуль Predefined}. Модуль содержит реализацию представителей классов алгебраических структур для встроенных типов данных. Так, например, в модуле реализованно стандартное булево полукольцо, стандартное арифметическое полукольцо над встроенными типами данных, а также тропическое полукольцо.
    \item \textbf{Модуль GraphblasEvaluation}. Модуль содержит вычислительное выражение GraphblasEvaluation. Оно выполняет 2 функции --- во-первых, являясь по своей сути монадой Reader, позволяет выставлять глобальные и локальные параметры вычисления, а во-вторых, оно скрывает использование интерфейса для работы с OpenCL, который предоставляет библиотека Brahma.Fsharp.
    \item \textbf{Модуль MtxReader}. Модуль предназначен для импорта матриц из файлов в формате \textit{mtx}.
    \item \textbf{Модуль Algorithms}. Модуль содержит небольшой набор классических алгоритмов на графах. В нем реализованы следующие алгоритмы: 
    \begin{itemize}
        \item алгоритм поиска в ширину из единственной вершины
        \item алгоритм поиска кратчайшего пути из единственной вершины
        \item алгоритм подсчета числа треугольников в графе
        \item алгоритм вычисления меры центральности вершины
    \end{itemize}
    \item \textbf{Модуль Backend}. Модуль содержит реализацию соответствующих операций для конкретных форматов хранения матрицы и вектора. В рамках данной работы были реализованны следующие операции:
    \begin{itemize}
        \item умножение матрицы в CSR формате на разреженный вектор
        \item транспонирование матрицы в CSR формате
        \item создание разреженного вектора из списка элементов
    \end{itemize}
\end{itemize}

\subsection{Реализация коллекций}
\paragraph{}
Особенностью стандарта GraphBLAS является то, что описываемые в нем объекты абстрактны --- за реализацию внутреннего хранения объектов ответственен разработчик решения. 

В качестве форматов хранения матрицы используются 2 формата --- CSR и COO.
Матрицы, которые выражают графы, обычно сильно разреженны, поэтому использовать плотную матрицу в качестве внутренней реализации нецелесообразно.
CSR формат, по сравнению, например, с диагональным (DIA) или ELLPACK (ELL) форматом, лучше подходит для хранения произвольных разреженных матриц, так как не требует от матрицы соответствия определенному паттерну для эффективного хранения. По сравнению с COO он занимает $2 \cdot nnz + n + 1$ вместо $3\cdot nnz$ памяти (здесь и далее $nnz$ --- число ненулевых элементов матрицы, $n$ --- число строк матрицы, $m$ --- число столбцов) и предоставляет доступ к произвольной строке за $O(1)$, что является существенным преимущество при реализации параллельных алгоритмов. В то же время для хранения сверхразреженных матриц, у которых $nnz > (m\cdot(n-1)-1)/2$ лучше подходит формат COO.

\subsection{Реализация операций}
\subsubsection{Умножение матрицы в CSR формате на разреженный вектор}
\paragraph{}
Для реализации операции умножения матрицы на вектор были рассмотрены несколько алгоритмов. В статье за авторством W. Liu and B. Vinter\cite{esc} приводится алгоритм умножения разреженных матриц в CSR формате. Ключевой операцией в алгоритмах такого рода является операция слияния строк промежуточной матрицы. Так как все строки имеют разное число ненулевых элементов, то для данной операции особо важно выбрать эффективный метод балансирования нагрузки. В данной статье предлагается разделить все строки промежуточной матрицы на 38 групп в зависимости числа ненулевых элементов в них и для каждой группы использовать один из четырех алгоритмов слияния. Однако, как показывает последние работы в данной области\cite{hash}, существуют более производительные и простые в решения. Так, статья за авторством Y. Nagasaka и др.\cite{hash}, которая также описывает алгоритм умножения разреженных матриц в CSR формате, предлагает для слияния строк использовать хеш-таблицу. По словам автора, этот алгоритм выигрывает у предыдущего и по времени работы, и по количеству потребляемой памяти. Однако для обновления данных в хеш-таблице алгоритм использует атомарные операции, а атомарные операции над произвольными типами на данный момент не поддерживаются библиотекой Brahma.FSharp, поэтому реализовать данный алгоритм оказалось невозможно. Алгоритм, описывающий умножение матрицы в CSR формате на разреженный вектор, за авторством Y. Tao и др.\cite{atomic} также предполагает использование атомарных операций, что делает невозможным его реализацию. Другой метод авторов Yang, C., Wang, Y., и Owens, J. D\cite{mxv_bfs} описывает умножения матрицы в CSR формате на разреженный вектор в контексте применения его в алгоритмах, подобных алгоритму поиска в ширину, а при оценке эффективности работы использует допущение о том, что вектор имеет небольшое число ненулевых элементов. Несмотря на то, что именно этот алгоритм реализован в аналогичной библиотеке GraphBLAST, было решено отказаться от него в пользу альтернативного варианта. В предлагаемой работе был реализован следующий алгоритм.
\begin{enumerate}
    \item Каждая строка матрицы умножается на разреженный вектор.
    \begin{itemize}
        \item Каждая строка обрабатывается 1 рабочей группой.
        \item Каждый поток в рабочей группе обрабатывает 1 элемент левого вектора с шагом, равным размеру рабочей группы.
        \item Значение с нужным индексом в правом векторе ищется бинарным поиском.
    \end{itemize}
    \item Полученный вектор промежуточных значений фильтруется от нулевых элементов с использованием префиксной суммы.
\end{enumerate}

\subsubsection{Транспонирование матрицы в CSR формате}
\paragraph{}
Для транспонирования матрицы в CSR формате был реализован наивный алгоритм, предполагающий конвертацию матрицы в формат COO.
\begin{enumerate}
    \item Матрица в CSR формате конвертируется в формат COO.
    \item Индексы элементов сортируются битонической сортировкой так, чтобы соответствовать транспонированной матрице.
    \item Матрица в COO формате конвертируется обратно в формат CSR.
\end{enumerate}

\subsection{Тестирование и непрерывная интеграция}
\paragraph{}
Использование высокоуровневого языка для реализации стандарта GraphBLAS, в том числе, облегчает тестирование. В GraphBLAS внутреннее поведение операции зависит от объектов управления и от того, как именно реализованы абстрактные объекты коллекций, причем с ростом числа используемых реализаций число вариантов операции, которые нужно тестировать, растет экспоненциально. В реализациях на С++, в которых для тестирования применялся Boost Test Framework, отдельно тестируется каждый вариант и, как правило, только на одном типе данных. 

В данной работе для тестирования используется библиотека Expecto\footnote{Репозиторий библиотеки Expecto: \url{https://github.com/haf/expecto}. Дата посещения: 01.06.2021}. За счет интеграции с библиотекой FsCheck\footnote{Репозиторий библиотеки FsCheck: \url{https://github.com/fscheck/FsCheck}. Дата посещения: 01.06.2021}, она поддерживает property-based тестирование, что позволяет автоматически генерировать наборы данных для тестов. Кроме того, она позволяет писать комбинаторные параметрические тесты, благодаря чему можно легко проверить все возможные варианты операции.

Возможность исполнения на CPU программ, которые используют OpenCL, позволила легко настроить тестирование библиотеки в сервисах непрерывной интеграции, таких как AppVeyor и GitHub Actions. 
\section{Dataset description}

In our evaluation we use \textbf{[RDF]} dataset which contains the following parts.
\begin{itemize}
\item The real-world data RDFs provided in CFPQ\_Data dataset\footnote{CFPQ\_Data dataset GitHub repository: \url{https://github.com/JetBrains-Research/CFPQ_Data}. Access date: 12.11.2019.} from~\cite{Mishin:2019:ECP:3327964.3328503}.
\item Geospecies (RDF which contains information about biological hierrarchy\footnote{\url{https://old.datahub.io/dataset/geospecies}. Access date: 12.11.2019.} and same generation query over \textit{broaderTransitive} relation) is provided in~\cite{Kuijpers:2019:ESC:3335783.3335791} and integrated in our evaluation with CFPQ\_Data.
\item It was shown in~\cite{Mishin:2019:ECP:3327964.3328503} that matrix-based algorithm is performant enough to handle bigger RDFs than those used in the initial datasets, such as~\cite{RDF}.
So, we add a number of big RDFs to CFPQ\_Data and use them in our evaluation.
New RDFs: \textit{go-hierarchy, go, enzime, core, pathways} are from UniProt database\footnote{Protein sequences data base: \url{https://www.uniprot.org/}. RDFs with data are avalable here: \url{ftp://ftp.uniprot.org/pub/databases/uniprot/current_release/rdf}. Access date: 12.11.2019}, and \textit{eclass-514en} is from eClassOWL project\footnote{eClassOWL project: \url{http://www.heppnetz.de/projects/eclassowl/}. eclass-514en file is available here: \url{http://www.ebusiness-unibw.org/ontologies/eclass/5.1.4/eclass_514en.owl}. Access date: 12.11.2019.}.
\end{itemize}

The variants of the \textit{same generation query}~\cite{FndDB} are used in almost all cases because it is an important example of real-world queries that are context-free but not regular.
So, variations of the same generation query are used in our evaluation.
All queries are added to the CFPQ\_Data dataset.

We use two queries over \textit{subClassOf} and \textit{type} relations.
The first query is the grammar $G_1$:
\[
 \begin{array}{lcl}
   s  \rightarrow \textit{subClassOf}^{\ -1} \ s \ \textit{subClassOf}   & \quad & s  \rightarrow \textit{type}^{\ -1} \ s \ \textit{type}     \\
   s  \rightarrow \textit{subClassOf}^{\ -1} \ \textit{subClassOf}       & \quad & s  \rightarrow  \textit{type}^{\ -1}  \ \textit{type}

 \end{array}
 \]
The second one is the grammar $G_2$: \[s \rightarrow \textit{subClassOf}^{\ -1} \ s \ \textit{subClassOf} \mid  \textit{subClassOf}\]

For geospecies we use same-generation queries from the original paper $geo$: \[s \rightarrow \textit{broaderTransitive} \ s \ \textit{broaderTransitive}^{\ -1} \]
\[s \rightarrow \textit{broaderTransitive}  \ \textit{broaderTransitive}^{\ -1} \]


The properties of the RDFs from the dataset are given in tables \ref{tbl:propRDF} and \ref{tbl:propGeo}.

{\setlength{\tabcolsep}{0.4em}
	\begin{table*}[h]
		\caption{RDFs properties}
		\label{tbl:propRDF}
		\rowcolors{2}{}{lightgray}
		\begin{tabular}{| l | c | c | c | c |}
			\hline
			Name                  & \#V    & \#E     & \#type &\#subClassOf \\
			\hline
			\hline
			atom-primitive				& 291		& 685		& 138	& 122	\\
			univ-bench					& 179		& 413		& 84		& 36		\\
			travel						& 131		& 397		& 90		& 30		\\
			skos							& 144		& 323		& 70		& 1		\\
			people\_pets					& 337		& 834		& 161	& 33		\\
			generations					& 129		& 351		& 78		& 0		\\
			foaf							& 256		& 815		& 174	& 10		\\
			biomedical-mesure-primitive	& 341		& 711		& 130	& 122	\\
			funding						& 778		& 1480		& 304	& 90               \\
			pizza						& 671		& 2604		& 365	& 259              \\
			wine							& 733		& 2450		& 485	& 126              \\
			core							& 1323		& 8684		& 1412	& 178              \\
			pathways						& 6238		& 37196		& 3118 	& 3117             \\
			go-hierarchy					& 45007		& 1960436	& 0		& 490109           \\
			enzyme						& 48815		& 219390		& 14989	& 8163             \\
			eclass\_514en				& 239111		& 1047454	& 72517	& 90962            \\
			go							& 272770		& 1068622	& 58483	& 90512            \\
			\hline
		\end{tabular}
	\end{table*}
}


{\setlength{\tabcolsep}{0.4em}
	\begin{table*}[h]
		\caption{Geospecies properties}
		\label{tbl:propGeo}
		\rowcolors{1}{}{lightgray}
		\begin{tabular}{| c | c | c |}
			\hline
			\#V    & \#E     & \#broaderTransitive \\
			\hline
			\hline
			450609 & 2311461 & 20867 \\
			\hline
		\end{tabular}
	\end{table*}
}

\section{Evaluation}

This section describes the methodology and answers the following research questions.

\begin{enumerate}
    \item Does fusion via distillation give any benefits at the software and hardware levels?
    \item What are the properties of the generated hardware?
    \item Does the generated hardware outperform software implementations?
\end{enumerate}

\subsection{Methodology}

Our focus is on creating a basis for future research and experiments, thus we make our experiments as much reproducible as possible\footnote{\url{https://github.com/sedwards-lab/fhw/tree/sparse-linear-algebra-distillation/examples/QTreeBenchmarks/diploma/verilog-bool-no-nnz-inlined} (online; accessed:
2022-06-07) Here one could find all the results. For each benchmark all statistics are specified: matrix names, their sizes, collected metrics for both hardware and software benchmarks.}. We benchmarked the following list of chained functions. The choice is prompted by the current state of the distiller: at the moment, it does not successfully distill matrix multiplication. However, the functions are still practical enough, for example, chained addition could be seen in Luby's maximal independent set algorithm and clearly describe the applicability of the proposed approach.

\begin{itemize}
    \item \mintinline{Haskell}{mAdd (==False) (||) (mAdd (==False) (||) m1 m2) m3}
    \item \mintinline{Haskell}{mask (mAdd (== False) (||) m2 m3) (m1)}
    \item \mintinline{Haskell}{map (==Zero) (to_nat) (mAdd (==False) (||) m1 m2}
    \item \mintinline{Haskell}{map (==Zero) (to_nat) (kron (==False) (&&) m1 m2}
\end{itemize}

Above, \mintinline{Haskell}{Zero} and \texttt{to\_nat} are corresponding definitions for Peano arithmetics, since the \texttt{.pot} language does not have any primitives. For the same reason, we operated with boolean matrices. Such functions could be abstracted with free variables and then instantiated in the emitted Haskell code. However, to get maximum from distillation, we provided all the information about the functions. 

For these functions, we compared the execution time of distilled and not distilled hardware generated circuits, execution time of original and distilled Haskell code and reference \textit{Suite Sparse}\footnote{\url{https://github.com/DrTimothyAldenDavis/GraphBLAS} (online; accessed:
2022-06-07), Suite Sparse library sources.}\textsuperscript{,}\footnote{The library also uses different variations of coordinate formats (opaque to the user) and not a quadtree representation.} variants of these functions in C\texttt{++}. Note that SuiteSparse does not support recursive data types, thus only the first two function chains were implemented in SuiteSparse (since Peano number is essentially a linked list). We did not replace Peano numbers with integers, so our experiments could be interpreted easier. For hardware experiments we collected execution time and the number of memory writes and reads, to access how well fusion is performed. For software experiments we only measured the execution time. Also note that we measured only the time, required to execute the lines above, not including any IO, required to get and evaluate function arguments. But in hardware benchmarks we measured the time required to pass arguments into the circuit's memory, because such IO is inevitable. It is tricky to make such measures in Haskell due to laziness, thus the programs were compiled with \texttt{--fno-full-laziness} to turn off memoization. Also all the arguments were forced to normal form via \texttt{force} and \texttt{evaluate}. Haskell programs were compiled\footnote{GHC 8.10.4.} with \texttt{-O2 --fno-full-laziness} and Suite Sparse was compiled with default flags and linked as a shared library to C\texttt{++} code.

We took matrices from SuiteSparse matrix collection with sizes ranging from \texttt{64x64} to \texttt{512x512}. We limited ourselves with such sizes due to the fact that this is the maximum sizes that fit into \texttt{bram} with $2^{16}$ address space. Such number of \texttt{bram} blocks is available only on really expensive FPGA boards, thus in practice sizes would be smaller to achieve better utilization. Once again, it models the situation when data fits into the cache, since \texttt{bram} in our circuits will operate as a cache in real application.

\subsection{Experiments}

Table~\ref{tab:bench_results} shows the results of all execution time benchmarks. To evaluate execution time for hardware simulation, implementation stage was performed to assess the maximum frequency of FPGA device used for synthesis and implementation, and the number of execution cycles was multiplied by the number of nanoseconds a clock cycle takes. The frequencies were equal within the same benchamark set, i.e., frequency was not affected by distillation. We used \texttt{xcu250figd2104-2L} device\footnote{\url{https://www.xilinx.com/products/boards-and-kits/alveo/u250.html}  (online; accessed:
2022-06-07)} for synthesis and implementation stages. It is not really a casual and affordable chip, but it contains enough \texttt{bram} for our evaluation to see scalability. In the table a median across several benchmarks is shown. 

As it could be seen, distillation steadily increases performance: up to 2x speedup for hardware simulation and up to 3x for software benchmarks. The results are maintained within the borders of the corresponding confidence interval and the borders are not shown for brevity. Hardware speedup is lower due to the different execution semantics, dataflow is not reduction-based and distillation is a reduction-based transformation. Note that generated hardware appears to be less performant than both Haskell and C\texttt{++}, which a bit contradicts the results from~\cite{oldfhw}. For hardware benchmarks \texttt{time (IO)} shows the execution time including the time needed to transfer the data though the arguments, \texttt{time (no IO)} does not include it in its turn. It could be seen that not all the benchmarks are computationally extensive enough to cover memory transferring costs, but for more complex examples the ratio would be better. Since we basically transfer the matrices node by node from a file in the testbench, we have probably the lowest possible latency, and in practice it would be higher if reading from DDR, however the bandwidth could be increased. Noticeably, running times for \texttt{mMaskAdd} for C\texttt{++} and distilled Haskell are similar, which shows that fusion really provides some extra performance: SuiteSparse at the moment does not implement any fusion.

Table~\ref{tab:mem_results} summarizes the ratios between distilled and not distilled hardware circuits memory reads and writes. Since in general case distillation removes extra pattern matching, essentially it saves memory reads and writes. The eventual number of memory reads and writes is implementation dependent, thus the table shows what share of speedup is prompted by saving memory operations. Distillation successfully reduces the number of memory accesses, about 15\% on average. \texttt{mMapKron} has a bit higher ratio due to the fact that \texttt{Nat} numbers require additional memory accesses, since the type is recursive. It could also be seen that a major part of speedups is attributed to saved memory accesses. 

Finally, table~\ref{tab:resource_util} shows device resources utilization ratios between distilled and not distilled hardware circuits and frequencies. Columns are device primitives: registers, lookup tables, \texttt{bram} blocks or multiplexers. Utilization for both types of circuits is below 1\% of available resources on the device, except for the memory. Memory blocks utilization is about 30\% (since we choose larger \texttt{brams} to store larger matrices). Apart from that, distilled circuits could have both higher and lower utilization. Since the hardware generation is primarily syntax-directed it follows from the distilled program structure. For example, distillation might glue two recursive functions into one (in that case, memory utilization would be lower, because each cluster of mutually recursive functions possesses its own heap) or make more recursive functions than in the original program. The frequencies are the same, however, they possibly could be made better with more intelligent buffer allocation.

\subsection{Discussion}
Answering the research questions above.

\begin{enumerate}
    \item Fusion gives significant benefits, however at the hardware level the benefits are a bit smaller since hardware semantics is not reduction based. The benefits at the hardware level are mostly determined by the reduced number of memory accesses (each access takes 2 clock cycles). Notably, distilled Haskell implementation of \texttt{mMaskAdd} has similar performance with C\texttt{++}. 
    \item Device utilization is low, but such circuits could be copied on the same device to provide better utilization and higher parallelism. Resource utilization might be both better and worse after distillation, depending on the transformed program itself since translation is syntax-directed. Frequency could be increased by more intelligent buffering strategy.
    \item Although we use low-latency design with \texttt{bram}s that take 2 clock cycles per request and transfer matrices from files, which does not have any latency in simulation, we get slower execution time than Haskell and C\texttt{++} counterparts. It could be partly due to excessive buffering performed by FHW at the moment. Also there is no pipelining for recursive calls, i.e. only one set
of function argument tokens are allowed to enter a tail-recursive function call until a result is finally generated. Further CPS transformation hinders parallelization, which could be made more explicit with SSA. Some other optimizations exist that may significantly influence the performance. Also, since device utilization is about 1\%, such circuits could be copied on one device and provide more parallelism. A more detailed discussion could be found at~\cite{Edwards2019FHWP}.
\end{enumerate}

Distillation clearly showed its applicability to optimization of sparse linear algebra routines and notably it still could be combined with other techniques, like rewrite rules to achieve better results. High-level synthesis has a room for improvements by increasing pipelining, parallelism and frequency and the generated hardware could be improved from usability perspective: a support for arbitrary sized matrices is desirable. Thus we will focus on these directions. Probably a better solution would be to embed \texttt{.pot} language into e.g. Haskell to leverage its type system (to be able to use some rewrite rules as well), and add support for primitive types and parallel primitives to be able to conduct a more scalable comparison with SuiteSparse (since SuiteSparse is multithreaded). For such embedding different execution models could be implemented, including hardware synthesis, for which SSA form of GRIN looks promising, as well as extra optimizations shipped with GRIN. For hardware synthesis, an interesting direction is achieving predictable results in hardware from certain modifications in software. This property partly holds for the current approach, since the translation is syntax- directed. More information on this could be found at~\cite{predict}.

\pagebreak

\begin{table}[t]
\scriptsize
\centering
\caption*{mAddAdd}
\begin{tabular}{|c|c|c|c|c|c|c|c|c|c|} 
\hline
\rowcolor{LightBlue}
\multicolumn{3}{|c|}{Matrices dimensions} & Haskell & Haskell (distilled) & \multicolumn{2}{c|}{FHW} & \multicolumn{2}{c|}{FHW (distilled)} & {C\texttt{++}}\\
% \rowcolor{LightBlue}
\hline
m1 & m2 & m3 & time & time & time (no IO) & time (IO) & time (no IO) & time (IO) & time \\ 
\hline
64 & 64 & 64 & 29 us & 20 us & 76 us & 170 us & 64 us & 158 us & 14 us\\ 
128 & 128 & 128 & 94 & 79 & 146 & 476 & 134 & 469 & 30 \\
256 & 256 & 256 & 123 & 103 & 202 &  681 & 168 & 662 & 44\\
512 & 512 & 512 & 219 & 143 & 474 & 1192 & 375 & 1093 & 49\\
\hline
\end{tabular}

\caption*{mMaskAdd}
\begin{tabular}{|c|c|c|c|c|c|c|c|c|c|} 
\hline
\rowcolor{LightBlue}
\multicolumn{3}{|c|}{Matrices dimensions} & Haskell & Haskell (distilled) & \multicolumn{2}{c|}{FHW} & \multicolumn{2}{c|}{FHW (distilled)} & {C\texttt{++}}\\
% \rowcolor{LightBlue}
\hline
m1 & m2 & m3 & time & time & time (no IO) & time (IO) & time (no IO) & time (IO) & time \\ 
\hline
64 & 64 & 64 & 10 us & 7 us & 64 us & 133 us & 46 us & 111 us & 18 us\\ 
128 & 128 & 128 & 38 & 30 & 118 & 322 & 75 & 292 & 33 \\
256 & 256 & 256 & 48 & 42 & 168 &  498 & 104 & 456 & 46\\
512 & 512 & 512 & 126 & 76 & 400 & 762 & 300 & 729 & 65\\
\hline
\end{tabular}

\caption*{mMapAdd}
\begin{tabular}{|c|c|c|c|c|c|c|c|c|c|} 
\hline
\rowcolor{LightBlue}
\multicolumn{3}{|c|}{Matrices dimensions} & Haskell & Haskell (distilled) & \multicolumn{2}{c|}{FHW} & \multicolumn{2}{c|}{FHW (distilled)} & {C\texttt{++}}\\
% \rowcolor{LightBlue}
\hline
m1 & m2 & m3 & time & time & time (no IO) & time (IO) & time (no IO) & time (IO) & time \\ 
\hline
64 & 64 & --- & 45 us & 37 us & 189 us & 253 us & 137 us & 202 us & ---\\ 
128 & 128 & --- & 162 & 105 & 524 & 685 & 397 & 579 & --- \\
256 & 256 & --- & 312 & 216 & 1047 &  1360 & 680 & 986 & ---\\
512 & 512 & --- & 436 & 273 & 1346 & 1776 & 900 & 1330 & ---\\
\hline
\end{tabular}

\caption*{mMapKron}
\begin{tabular}{|c|c|c|c|c|c|c|c|c|c|} 
\hline
\rowcolor{LightBlue}
\multicolumn{3}{|c|}{Matrices dimensions} & Haskell & Haskell (distilled) & \multicolumn{2}{c|}{FHW} & \multicolumn{2}{c|}{FHW (distilled)} & {C\texttt{++}}\\
% \rowcolor{LightBlue}
\hline
m1 & m2 & m3 & time & time & time (no IO) & time (IO) & time (no IO) & time (IO) & time \\ 
\hline
2 & 64 & --- & 64 us & 36 us & 212 us & 242 us & 94 us & 125 us & ---\\ 
2 & 128 & --- & 137 & 68 & 434 & 502 & 199 & 266 & --- \\
2 & 256 & --- & 364 & 126 & 1004 &  1188 & 449 & 636 & ---\\
4 & 128 & --- & 302 & 94 & 694 & 763 & 330 & 401 & ---\\
\hline
\end{tabular}



\caption{Execution time}
\label{tab:bench_results}

\end{table}
\begin{table}[h]
\scriptsize
\begin{minipage}{0.45\linewidth}
\centering
\caption*{mAddAdd}
\begin{tabular}{|c|c|c|c|c|c|c|} 
\hline
\rowcolor{LightBlue}
\multicolumn{3}{|c|}{Matrices dimensions} & \multicolumn{2}{c|}{Ratio ($\frac{FHW}{FHW_{distilled}}$)}\\
% \rowcolor{LightBlue}
\hline
m1 & m2 & m3 & writes & reads\\ 
\hline
64 & 64 & 64 & 1.10 & 1.15\\ 
128 & 128 & 128 & 1.02 & 1.05\\
256 & 256 & 256 & 1.03 & 1.06\\
512 & 512 & 512 & 1.10 & 1.16\\
\hline
\end{tabular}
\end{minipage}
\begin{minipage}{0.45\linewidth}
\centering
\caption*{mMaskAdd}
\begin{tabular}{|c|c|c|c|c|c|c|} 
\hline
\rowcolor{LightBlue}
\multicolumn{3}{|c|}{Matrices dimensions} & \multicolumn{2}{c|}{Ratio ($\frac{FHW}{FHW_{distilled}}$)}\\
% \rowcolor{LightBlue}
\hline
m1 & m2 & m3 & writes & reads\\ 
\hline
64 & 64 & 64 & 1.13 & 1.26\\ 
128 & 128 & 128 & 1.06 & 1.11\\
256 & 256 & 256 & 1.08 & 1.09\\
512 & 512 & 512 & 1.10 & 1.16\\
\hline
\end{tabular}
\end{minipage}
\begin{minipage}{0.45\linewidth}
\centering
\caption*{mMapAdd}
\begin{tabular}{|c|c|c|c|c|c|c|} 
\hline
\rowcolor{LightBlue}
\multicolumn{3}{|c|}{Matrices dimensions} & \multicolumn{2}{c|}{Ratio ($\frac{FHW}{FHW_{distilled}}$)}\\
% \rowcolor{LightBlue}
\hline
m1 & m2 & m3 & writes & reads\\ 
\hline
64 & 64 & --- & 1.10 & 1.21\\ 
128 & 128 & --- & 1.07 & 1.14\\
256 & 256 & --- & 1.07 & 1.19\\
512 & 512 & --- & 1.10 & 1.21\\
\hline
\end{tabular}
\end{minipage}
\hfill
\begin{minipage}{0.45\linewidth}
\centering
\caption*{mMapKron}
\begin{tabular}{|c|c|c|c|c|c|c|} 
\hline
\rowcolor{LightBlue}
\multicolumn{3}{|c|}{Matrices dimensions} & \multicolumn{2}{c|}{Ratio ($\frac{FHW}{FHW_{distilled}}$)}\\
% \rowcolor{LightBlue}
\hline
m1 & m2 & m3 & writes & reads\\ 
\hline
2 & 64 & --- & 1.71 & 1.88\\ 
2 & 128 & --- & 1.72 & 1.87\\
2 & 256 & --- & 1.65 & 1.83\\
4 & 128 & --- & 1.81 & 1.91\\
\hline
\end{tabular}
\end{minipage}

\caption{Memory accesses}
\label{tab:mem_results}
\end{table}

\begin{table}[h]
\scriptsize
\centering
\begin{tabular}{|l|c|c|c|c|c|c|c|c|c|} 
\hline
\rowcolor{LightBlue}

{Benchmark} & \multicolumn{8}{c|}{Ratio (${\frac{FHW}{FHW_{distilled}}}$)} & {Frequency}\\
\hline
{} & FDRE & LUT3 & LUT6 & LUT5 & LUT4 & LUT2 & RAMB36E2 & MUXF7 & {} \\
% \rowcolor{LightBlue}
\hline
mAddAdd & 0.3 & 0.3 & 0.3 & 0.5 & 0.3 & 0.3 & 0.5 & 0.5 & 200 MHz\\ 
mMaskAdd & 0.5 & 0.5 & 0.7 & 0.4 & 0.7 & 0.5 & 0.7 & 0.6 & 200 MHz\\
mMapAdd & 1 & 0.9 & 0.9 & 1.2 & 1 & 1.1 & 1.1 & 1.2 & 200 MHz\\
mMapKron & 1.5 & 1.5 & 1.3 & 2 & 2 & 1.8 & 1.4 & 1.7 & 200 MHz\\
\hline
\end{tabular}
\caption{Resource utilization}
\label{tab:resource_util}
\end{table}
\pagebreak

\section{Conclusion and Future Work}

We present !!!

Our evaluation shows that !!!

First direction for future research is a more detailed CFPQ algorithms investigation.
We should do More evaluation on sparse matrices on GPGPUs.

Also it is nesessary to implement and evaluate solutions for graphs which is not fit in RAM.
There is a set of technics for huge matrices multiplication.
Is it possible to dopt it for CFPQ

Another direcion is a dataset improvement.
More data.
More grammars/queries.


\begin{acks}
The research was supported by the Russian Science Foundation grant 18-11-00100 and a grant from JetBrains Research.
\end{acks}

%
% The next two lines define the bibliography style to be used, and the bibliography file.
\bibliographystyle{ACM-Reference-Format}
\bibliography{cfpq_on_gpgpu}

%
% If your work has an appendix, this is the place to put it.
%\appendix

%\section{Research Methods}

%\subsection{Part One}


\end{document}
