\section{Conclusion and Future Work}

We provide a number of implementations of the matrix-based algorithm for context-free path querying, collect a dataset for evaluation and provide results of evaluation of our implementation on the collected dataset.
Our evaluation shows that GPGPU utilization for boolean matrices multiplication can significantly increase the performance of CFPQs evaluation, but requires more research on implementation details.

The first direction for future research is a more detailed CFPQ algorithms investigation.
We should do more evaluation on sparse matrices on GPGPUs and investigate technics for high-performance GPGPU code creation.
Also, it is necessary to implement and evaluate solutions for graphs which do not fit in RAM, and for big queries which disallow to allocate all required matrices on single GPGPU.
We hope that it is possible to utilize existing technics for huge matrices multiplication for this problem.

Another direction is dataset improvement.
First of all, it is necessary to collect more data, and more grammars/queries.
Especially it would be important to add to the dataset more real-world graphs and more real-world queries.
Secondly, it is necessary to discuss and fix the data format to be able to evaluate different algorithms.
We beleave that it is necessary to create a public dataset for CFPQ algorithms evaluation, and collaboration with the community is required.

%log cfg. datalog -> cfg or to boolean/conjunctive
