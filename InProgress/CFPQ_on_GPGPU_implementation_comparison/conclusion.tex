\section{Conclusion and Future Work}

We provide a number of implementations of matrix-based algorithm for context-free path querying, collect a dataset for evaluation and provide results of evaluation of our implementation on collected dataset.
Our evaluation shows that GPGPU utilization for boolean matrices multiplication can significantely increase perfprmance of CFPQs evaluation, but requires more research on implementattion details.

First direction for future research is a more detailed CFPQ algorithms investigation.
We should do more evaluation on sparse matrices on GPGPUs and investigate technics for high-performance GPGPU code creation.
Also it is nesessary to implement and evaluate solutions for graphs which is not fit in RAM, and for big queryes which disallow to allocate all requred matrices on single GPGPU.
We hope that it is possible to utilize existing technics for huge matrices multiplication for this problem.

Another direcion is a dataset improvement.
First of all, it is necessary to collect more data, and more grammars/queries.
Especially it would be interesting to add to dataset more real graphs and more real queryes.
Secondly, it is nesessary to discuss and fix data format to be able to evaluate different algorithms.
We think that it is necessary to create public dataset for CFPQ algorithms evaluation, and collaboration with community is requred.

%log cfg. datalog -> cfg or to boolean/conjunctive

