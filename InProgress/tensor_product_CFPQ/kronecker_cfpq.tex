\section{Kronecker Product Based CFPQ Algorithm}

The idea of the algorithm is based on generalisation of the finite-state machine intersection for a recursive automata, created from input grammar, and an input graph. The result of the intersection is evaluated as a Kronecker product of the corresponding adjacency matrices for automata and graph. To solve reachability problem it is enough to represent intersection result as a Boolean matrix, what simplifies algorithm implementation and allows to express it in terms of basic matrix operations. Listing~\ref{tensor:cfpq}. shows main steps of the solution.

As an input algorithm accepts context-free grammar $G=(\Sigma,N,P$) and graph $\mathcal{G}=(V,E,L)$. All sets in the grammar and graph are supposed to be of finite size. Recursive automata $R$ is created from $G$. The process of the creation is out of the scope of this article. $M_1$ and $M_2$ are the adjacency matrices for automata $R$ and graph $\mathcal{G}$ correspondingly. Cell values of this matrices could be represented as sets of elements from $L \cup N \cup \Sigma$.

Then the $\varepsilon$ symbol is explicitly added for matrices $M_1$ and $M_2$. For each state of the $R$ algorithm adds transitions from state $i$ to state $i$ via $\varepsilon$ if $i$ is initial and final state for some non-terminal. This info is queried via $getNonterminals()$ function call. Also for each vertex $i$ of the graph $\mathcal{G}$ the edge from $i$ to $i$ with label $\varepsilon$ is added. Here the rule is implied: each vertex of the graph $\mathcal{G}$ is reachable by itself through $\varepsilon$-transition. 

The algorithm is executed while matrix $M_2$ is changing. For each iteration Kronecker product of matrices $M_1$ and $M_2$ is evaluated. The result is saved in $M_3$ as a Boolean matrix. For given $M_3$ evaluated $C_3$ matrix via $transitiveClosure()$ function call. The $M_3$ could be interpreted as an adjacency matrix for an oriented graph without labels, used to evaluate transitive closure in terms of classical graph definition of this operation. Then the algorithm iterates over cells of the $C_3$. For pair of indices $(i,j)$ computes $s$ and $f$ - initial and final states in recursive automata $R$ which relate to the concrete $C_3[i,j]$ of the tensor matrix. Function $hasPathForNonterminals()$ checks whether for given $s$ and $f$ states automata has at least one non-terminal path. If the conditional statement is $true$ then algorithm adds non-terminals of that path via $getNonterminals()$ to the concrete cell of the adjacency matrix $M_2$ of the graph.

As an result the algorithm returns updated matrix $M_2$ which contains initial graph $\mathcal{G}$ data and non-terminals from $N$. If a cell $M_2[i,j]$ for any valid indices $i$ and $j$ contains symbol $S \in N$, therefore, vertex $j$ is reachable from vertex $i$ in grammar $G$ for non-terminal $S$.

\begin{algorithm}[h]
\floatname{algorithm}{Listing}
\begin{algorithmic}[1]
\caption{Kronecker product based CFPQ}
\label{tensor:cfpq}
\Function{contextFreePathQuerying}{G, $\mathcal{G}$}
    % Input data preparation
    \State{$R \gets$ Recursive automata for $G$}
    \State{$M_1 \gets$ Adjacency matrix for $R$}
    \State{$M_2 \gets$ Adjacency matrix for $\mathcal{G}$}
    % Eps-transition handling for graph
    \For{$i \in 0..dim(M_1)-1$} 
        \If{$|\textit{getNonterminals}(R,i,i)| > 0$}
            \Comment{Has at least one non-terminal}
            \State{$M_2[i,i] \gets M_2[i,i] \cup \{ \varepsilon \}$}
        \EndIf
    \EndFor
    % Eps-transition handling for graph
    \For{$i \in 0..dim(M_2)-1$} 
        \State{$M_3[i,i] \gets M_3[i,i] \cup \{ \varepsilon \}$}
    \EndFor
    \While{Matrix $M_2$ is changing}
        \State{$M_3 \gets M_1 \otimes M_2$}
        \Comment{Evaluate tensor product}
        \State{$C_3 \gets \textit{transitiveClosure}(M_3)$}
        \State{$n \gets$ dim($M_3)$}
        \Comment{Matrix $M_3$ size = $n \times n$}
        % Add non-terminals (possibly new)
        \For{$i \in 0..n-1$}
           \For{$j \in 0..n-1$}
                \If{$C_3[i,j]$}
                    \State{$s, f \gets \textit{getStates}(C_3,i,j)$}
                    \If{$\textit{hasPathForNonterminals}(R,s,f)$}
                        \State{$x, y \gets \textit{getCoordinates}(C_3,i,j)$}
                        \State{$M_2[x,y] \gets M_2[x,y] \cup \textit{getNonterminals}(R,s,f)$}
                    \EndIf
                \EndIf
           \EndFor
        \EndFor
    \EndWhile
\State \Return $M_2$
\EndFunction
\end{algorithmic}
\end{algorithm}

Since the Kronecker product evaluated in the fixed order, such as $M_1 \otimes M_2$, the functions $getStates$ and $getCoordinates$ could be implemented as shown in Listing~\ref{tensor:cfpq:help}. This implementation appeals to the blocked structure of the matrix $C_3$, where each block corresponds to some automata and graph edge.

\begin{algorithm}[h]
\floatname{algorithm}{Listing}
\begin{algorithmic}[1]
\caption{Help functions for Kronecker product based CFPQ}
\label{tensor:cfpq:help}
\Function{getStates}{$C, i, j$}
    \State{$r \gets dim(M_1)$}
    \Comment{$M_1$ is adjacency matrix for automata $R$}
    \State \Return{$\left\lfloor{i / r}\right\rfloor, \left\lfloor{j / r}\right\rfloor$}
\EndFunction
\Function{getCoordinates}{$C, i, j$}
    \State{$n \gets dim(M_2)$}
    \Comment{$M_2$ is adjacency matrix for graph $\mathcal{G}$}
    \State \Return{$i \bmod n, j \bmod n$}
\EndFunction
\end{algorithmic}
\end{algorithm}

The proposed algorithm terminates in finite number of steps. Since functions, used in the algorithm, terminate in finite number of steps, it is enough to proof that the $while$ loop takes finite number of iterations. The matrix $M_2$ is updated only by possible addition of the new non-terminals in the line \textbf{20}. The number of elements in the cell of matrix $M_2$ only grows and cannot exceed $|L \cup N \cup \Sigma|$. Therefore, the $while$ loop terminates in finite number of iterations, which no more than $|L \cup N \cup \Sigma| * |V|^2$ in case when only 1 symbol is added for only one vertex per iteration.

\subsection{Example}

This section is intended to provide step-by-step demonstration of the proposed algorithm. As an example consider the following query, theoretical worst case for CFPQ time complexity, proposed by Hellings~\cite{Hellings}: graph $\mathcal{G}$ presented in Figure~\ref{input:graph} and  context-free grammar $G=(\Sigma,N,P)$ for a language $\{a^n b^n | n \geq 1\}$ where:

\begin{itemize}
    \item Set of terminals $\Sigma = \{a, b\}$.
    \item Set of non-terminals $V = \{ S \}$.
    \item Set of production rules $P = \{ S \to a S b, S \to a b\}$.
\end{itemize}

Since the proposed algorithm processes grammar in form of recursive automata, we first provide automata $R$ in Figure~\ref{input:automata}. The initial state of the automata is $(0)$, the final state is $(3)$. The notation $\{S\}$ denotes here that non-terminal $S$ could be derived in automata path from vertex $(0)$ to $(3)$.

\begin{figure}[h]
     \begin{tikzpicture}[shorten >=1pt,auto]
           \node[state, initial] (q_0)   {$0 \{S\}$};
           \node[state] (q_1) [right=of q_0] {$1$};
           \node[state] (q_2) [right=of q_1] {$2$};
           \node[state, accepting] (q_3) [right=of q_2] {$3\{S\}$};
            \path[->]
            (q_0) edge node {a} (q_1)
            (q_1) edge node {S} (q_2)
            (q_2) edge node {b} (q_3)
            (q_1) edge [bend left, above]  node {b} (q_3);
    \end{tikzpicture}
    \centering
    \caption{The recursive automata $R$ of grammar $G$ for example query}
    \label{input:automata}
\end{figure}

\begin{figure}[h]
        \centering
        \begin{tikzpicture}[shorten >=1pt,auto]
           \node[state] (q_0)                      {$0$};
           \node[state] (q_1) [above right=of q_0] {$1$};
           \node[state] (q_2) [right=of q_0]       {$2$};
           \node[state] (q_3) [right=of q_2]       {$3$};
            \path[->]
            (q_0) edge  node {a} (q_1)
            (q_1) edge  node {a} (q_2)
            (q_2) edge  node {a} (q_0)
            (q_2) edge[bend left, above]  node {b} (q_3)
            (q_3) edge[bend left, below]  node {b} (q_2);
        \end{tikzpicture}
        \caption{The input graph $\mathcal{G}$ for example query}
        \label{input:graph}
\end{figure}

Adjacency matrices $M_1$ and $M_2$ for automata $R$ and graph $\mathcal{G}$ respectively are initialised as follows:
    $$
    M_1 =
    \begin{pmatrix}
    . & . & \{a\} & .     \\
    . & . & \{S\} & \{b\} \\
    . & . & . & \{b\}     \\
    . & . & . & .
    \end{pmatrix}
    ,~~~~~
    M_2^0 =
    \begin{pmatrix}
    . & \{a\} & . & .     \\
    . & . & \{a\} & .     \\
    \{a\} & . & . & \{b\} \\
    . & . & \{b\} & .
    \end{pmatrix}.
    $$

After all the data is initialised in lines \textbf{2--4}, the algorithm handles $\varepsilon$-case. Because automata $R$ does not have $\varepsilon$-transitions and $\varepsilon$-word is not included in grammar $G$ language lines \textbf{5-9} of the algorithm do not affect on the input data.

Then the algorithm enters while loop and iterates as long as matrix $M_2$ is changing. We provide step-by-step evaluation of matrices $M_3$, $C_3$ and updating of matrix $M_2$. All the matrices are denoted with upper index of the current loop iteration. The first loop iteration is indexed as 1.

For the first while loop iteration the tensor product $M_3^1 = M_1 \otimes M_2^0$ and transitive closure $C_3^1$ are evaluated as follows:

%\begin{figure}
    {\renewcommand{\arraystretch}{0.6}
    %\centering
    $$
    M_3^1 =
    \left(
    \begin{array}{c c c c | c c c c | c c c c | c c c c }
    . & . & . & .  &  . & 1 & . & .  &  . & . & . & .  &  . & . & . & .   \\
    . & . & . & .  &  . & . & 1 & .  &  . & . & . & .  &  . & . & . & .   \\
    . & . & . & .  &  1 & . & . & .  &  . & . & . & .  &  . & . & . & .   \\
    . & . & . & .  &  . & . & . & .  &  . & . & . & .  &  . & . & . & .   \\
    \hline
    . & . & . & .  &  . & . & . & .  &  . & . & . & .  &  . & . & . & .   \\
    . & . & . & .  &  . & . & . & .  &  . & . & . & .  &  . & . & . & .   \\
    . & . & . & .  &  . & . & . & .  &  . & . & . & .  &  . & . & . & 1   \\
    . & . & . & .  &  . & . & . & .  &  . & . & . & .  &  . & . & 1 & .   \\
    \hline
    . & . & . & .  &  . & . & . & .  &  . & . & . & .  &  . & . & . & .   \\
    . & . & . & .  &  . & . & . & .  &  . & . & . & .  &  . & . & . & .   \\
    . & . & . & .  &  . & . & . & .  &  . & . & . & .  &  . & . & . & 1   \\
    . & . & . & .  &  . & . & . & .  &  . & . & . & .  &  . & . & 1 & . \\
    \hline
    . & . & . & .  &  . & . & . & .  &  . & . & . & .  &  . & . & . & .   \\
    . & . & . & .  &  . & . & . & .  &  . & . & . & .  &  . & . & . & .   \\
    . & . & . & .  &  . & . & . & .  &  . & . & . & .  &  . & . & . & .   \\
    . & . & . & .  &  . & . & . & .  &  . & . & . & .  &  . & . & . & .
    \end{array}
    \right)
    ,~~~~~~
    C_3^1 =
    \left(
    \begin{array}{c c c c | c c c c | c c c c | c c c c }
    . & . & . & .  &  . & 1 & . & .  &  . & . & . & .  &  . & . & . & . \\
    . & . & . & .  &  . & . & 1 & .  &  . & . & . & .  &  . & . & . & \cellcolor{lightgray}\textbf{1} \\
    . & . & . & .  &  1 & . & . & .  &  . & . & . & .  &  . & . & . & . \\
    . & . & . & .  &  . & . & . & .  &  . & . & . & .  &  . & . & . & . \\
    \hline
    . & . & . & .  &  . & . & . & .  &  . & . & . & .  &  . & . & . & . \\
    . & . & . & .  &  . & . & . & .  &  . & . & . & .  &  . & . & . & . \\
    . & . & . & .  &  . & . & . & .  &  . & . & . & .  &  . & . & . & 1 \\
    . & . & . & .  &  . & . & . & .  &  . & . & . & .  &  . & . & 1 & . \\
    \hline
    . & . & . & .  &  . & . & . & .  &  . & . & . & .  &  . & . & . & . \\
    . & . & . & .  &  . & . & . & .  &  . & . & . & .  &  . & . & . & . \\
    . & . & . & .  &  . & . & . & .  &  . & . & . & .  &  . & . & . & 1 \\
    . & . & . & .  &  . & . & . & .  &  . & . & . & .  &  . & . & 1 & . \\
    \hline
    . & . & . & .  &  . & . & . & .  &  . & . & . & .  &  . & . & . & . \\
    . & . & . & .  &  . & . & . & .  &  . & . & . & .  &  . & . & . & . \\
    . & . & . & .  &  . & . & . & .  &  . & . & . & .  &  . & . & . & . \\
    . & . & . & .  &  . & . & . & .  &  . & . & . & .  &  . & . & . & .
    \end{array}
    \right).
    $$
    }
    %\caption{The first iteration tensor product and transitive closure evaluation for example query}
    %\label{example:iteration1eval}
%\end{figure}

Note here that the dimension $n$ of the matrix $M_3$ equals 16, and this value is constant in time of the algorithm execution.

After the transitive closure evaluation matrix $C_3^1$ cell $(1,15)$ contains non-zero value. It means that vertex with index $15$ is accessible from vertex with index $1$ in a graph, represented by adjacency matrix $M_3^1$.

Then the algorithm lines \textbf{14--20} are executed. In that section algorithm adds non-terminals to the graph matrix $M_2^1$. Because this step is additive we are only interested in newly appeared values in matrix $C_3^1$ such as value $C_3^1[1,15]$.

For the value $C_3^1[1,15]$:
\begin{itemize}
    \item Indices of the automata vertices $s = 0$ and $f = 3$, because value $C_3^1[1,15]$ located in upper right matrix block $(0,3)$.
    \item Indices of the graph vertices $x = 1$ and $y = 3$ are evaluated as
    value $C_3^1[1,15]$ indices relatively to its block $(0,3)$.
    \item Function call $hasPathForNonterminals()$ returns \textbf{true} since the automata $R$ has path for non-terminal $S$ from vertex $0$ to $3$.
    \item Function call $getNonterminals()$ returns $\{S\}$ since this is the only non-terminal which could be derived in path from vertex $0$ to $3$.
\end{itemize}{}

\begin{figure}[h]
    \begin{subfigure}[]{0.5\textwidth}
    \centering
    $$
    M_2^1 =
    \begin{pmatrix}
    . & \{a\} & . & .     \\
    . & . & \{a\} & \{S\} \\
    \{a\} & . & . & \{b\} \\
    . & . & \{b\} & .
    \end{pmatrix}
    $$
    \end{subfigure}
    \begin{subfigure}[]{0.4\textwidth}
    \centering
    \begin{tikzpicture}[shorten >=1pt,auto]
           \node[state] (q_0)                      {$0$};
           \node[state] (q_1) [above right=of q_0] {$1$};
           \node[state] (q_2) [right=of q_0]       {$2$};
           \node[state] (q_3) [right=of q_2]       {$3$};
            \path[->]
            (q_0) edge  node {a} (q_1)
            (q_1) edge  node {a} (q_2)
            (q_1) edge[bend left, above]  node {\textbf{S}} (q_3)
            (q_2) edge  node {a} (q_0)
            (q_2) edge[bend left, above]  node {b} (q_3)
            (q_3) edge[bend left, below]  node {b} (q_2);
    \end{tikzpicture}
    \end{subfigure}
    \caption{The updated matrix $M_2^1$ and graph $\mathcal{G}$ after first loop iteration for example query}
    \label{example:iteration1res}
\end{figure}

After the first loop iteration matrix symbol $S$ is added to the cell $M_2^1[1,3]$. It is relevant data, because initial graph has path $1 \to 2 \to 3$ which could be derived for $S$. The updated matrix and graph are depicted in Figure~\ref{example:iteration1res}.

For the second loop iteration matrices $M_3^2$ and $C_3^2$ are evaluated as listed in Figure~\ref{example:iteration2eval}. For this iteration in the matrix $C_3^2$ appeared new non-zero values in cells with indices $[0,11]$, $[0,14]$ and $[5,14]$. Because only the cell value with index $[0,14]$ corresponds to the automata path with not empty non-terminal set $\{S\}$ its data affects adjacency matrix $M_2$. The updated matrix and graph $\mathcal{G}$ are depicted in Figure~\ref{example:iteration2res}.

\begin{figure}
    \renewcommand{\arraystretch}{0.6}
    \centering
    $$
    M_3^2 =
    \left(
    \begin{array}{c c c c | c c c c | c c c c | c c c c }
    . & . & . & .  &  . & 1 & . & .  &  . & . & . & .  &  . & . & . & .   \\
    . & . & . & .  &  . & . & 1 & .  &  . & . & . & .  &  . & . & . & .   \\
    . & . & . & .  &  1 & . & . & .  &  . & . & . & .  &  . & . & . & .   \\
    . & . & . & .  &  . & . & . & .  &  . & . & . & .  &  . & . & . & .   \\
    \hline
    . & . & . & .  &  . & . & . & .  &  . & . & . & .           &  . & . & . & .   \\
    . & . & . & .  &  . & . & . & .  &  . & . & . & \textbf{1}  &  . & . & . & .   \\
    . & . & . & .  &  . & . & . & .  &  . & . & . & .           &  . & . & . & 1 \\
    . & . & . & .  &  . & . & . & .  &  . & . & . & .           &  . & . & 1 & . \\
    \hline
    . & . & . & .  &  . & . & . & .  &  . & . & . & .  &  . & . & . & .   \\
    . & . & . & .  &  . & . & . & .  &  . & . & . & .  &  . & . & . & .   \\
    . & . & . & .  &  . & . & . & .  &  . & . & . & .  &  . & . & . & 1 \\
    . & . & . & .  &  . & . & . & .  &  . & . & . & .  &  . & . & 1 & . \\
    \hline
    . & . & . & .  &  . & . & . & .  &  . & . & . & .  &  . & . & . & .   \\
    . & . & . & .  &  . & . & . & .  &  . & . & . & .  &  . & . & . & .   \\
    . & . & . & .  &  . & . & . & .  &  . & . & . & .  &  . & . & . & .   \\
    . & . & . & .  &  . & . & . & .  &  . & . & . & .  &  . & . & . & .
    \end{array}
    \right)
    C_3^2 =
    \left(
    \begin{array}{c c c c | c c c c | c c c c | c c c c }
    . & . & . & .  &  . & 1 & . & .  &  . & . & . & \textbf{1}  &  . & . & \textbf{1} & . \\
    . & . & . & .  &  . & . & 1 & .  &  . & . & . & .  &  . & . & . & 1 \\
    . & . & . & .  &  1 & . & . & .  &  . & . & . & .  &  . & . & . & . \\
    . & . & . & .  &  . & . & . & .  &  . & . & . & .  &  . & . & . & . \\
    \hline
    . & . & . & .  &  . & . & . & .  &  . & . & . & .  &  . & . & . & . \\
    . & . & . & .  &  . & . & . & .  &  . & . & . & 1  &  . & . & \textbf{1} & . \\
    . & . & . & .  &  . & . & . & .  &  . & . & . & .  &  . & . & . & 1 \\
    . & . & . & .  &  . & . & . & .  &  . & . & . & .  &  . & . & 1 & . \\
    \hline
    . & . & . & .  &  . & . & . & .  &  . & . & . & .  &  . & . & . & . \\
    . & . & . & .  &  . & . & . & .  &  . & . & . & .  &  . & . & . & . \\
    . & . & . & .  &  . & . & . & .  &  . & . & . & .  &  . & . & . & 1 \\
    . & . & . & .  &  . & . & . & .  &  . & . & . & .  &  . & . & 1 & . \\
    \hline
    . & . & . & .  &  . & . & . & .  &  . & . & . & .  &  . & . & . & . \\
    . & . & . & .  &  . & . & . & .  &  . & . & . & .  &  . & . & . & . \\
    . & . & . & .  &  . & . & . & .  &  . & . & . & .  &  . & . & . & . \\
    . & . & . & .  &  . & . & . & .  &  . & . & . & .  &  . & . & . & .
    \end{array}
    \right)
    $$
    \caption{The second iteration tensor product and transitive closure evaluation for example query}
    \label{example:iteration2eval}
\end{figure}

\begin{figure}
    \begin{subfigure}[]{0.5\textwidth}
    \centering
    $$
    M_2^2 =
    \begin{pmatrix}
    .     & \{a\} & \{S\} & .     \\
    .     & .     & \{a\} & \{S\} \\
    \{a\} & .     & .     & \{b\} \\
    .     & .     & \{b\} & .
    \end{pmatrix}
    $$
    \end{subfigure}
    \begin{subfigure}[]{0.4\textwidth}
    \centering
    \begin{tikzpicture}[shorten >=1pt,auto]
           \node[state] (q_0)                      {$0$};
           \node[state] (q_1) [above right=of q_0] {$1$};
           \node[state] (q_2) [right=of q_0]       {$2$};
           \node[state] (q_3) [right=of q_2]       {$3$};
            \path[->]
            (q_0) edge  node {a} (q_1)
            (q_1) edge  node {a} (q_2)
            (q_1) edge[bend left, above]  node {S} (q_3)
            (q_2) edge  node {a} (q_0)
            (q_0) edge[bend right, below]  node {\textbf{S}} (q_2)
            (q_2) edge[bend left, above]  node {b} (q_3)
            (q_3) edge[bend left, below]  node {b} (q_2);
    \end{tikzpicture}
    \end{subfigure}
    \caption{The updated matrix $M_2^2$ and graph $\mathcal{G}$ after second loop iteration for example query}
    \label{example:iteration2res}
\end{figure}

The remaining matrices $C_3$ and $M_2$ for the algorithm main loop execution are listed in the Figure~\ref{example:iteration3to6eval} and Figure~\ref{example:iteration3to6res} correspondingly. For the sake of simplicity evaluated matrices $M_3$ are not included because its computation is a straightforward process. The last loop iteration is $7$. Although the matrix $M_2^6$ is updated with new non-terminal $S$ for the cell $[2,2]$ after transitive closure evaluation the new values to the matrix $M_2$ is not added. Therefore matrix $M_2$ has stopped changing and the algorithm is successfully finished.

For the example query algorithm takes $7$ iterations for the $while-loop$. The updated graph $\mathcal{G}$ is depicted in the Figure~\ref{example:result}.

\begin{figure}
    \renewcommand{\arraystretch}{0.6}
    \centering
    $$
    C_3^3 =
    \left(
    \begin{array}{c c c c | c c c c | c c c c | c c c c }
    . & . & . & .  &  . & 1 & . & .  &  . & . & . & 1  &  . & . & 1 & . \\
    . & . & . & .  &  . & . & 1 & .  &  . & . & . & .  &  . & . & . & 1 \\
    . & . & . & .  &  1 & . & . & .  &  . & . & \cellcolor{lightgray}\textbf{1} & .  &  . & . & . & \cellcolor{lightgray}\textbf{1} \\
    . & . & . & .  &  . & . & . & .  &  . & . & . & .  &  . & . & . & . \\
    \hline
    . & . & . & .  &  . & . & . & .  &  . & . & 1 & .  &  . & . & . & \textbf{1} \\
    . & . & . & .  &  . & . & . & .  &  . & . & . & 1  &  . & . & 1 & . \\
    . & . & . & .  &  . & . & . & .  &  . & . & . & .  &  . & . & . & 1 \\
    . & . & . & .  &  . & . & . & .  &  . & . & . & .  &  . & . & 1 & . \\
    \hline
    . & . & . & .  &  . & . & . & .  &  . & . & . & .  &  . & . & . & . \\
    . & . & . & .  &  . & . & . & .  &  . & . & . & .  &  . & . & . & . \\
    . & . & . & .  &  . & . & . & .  &  . & . & . & .  &  . & . & . & 1 \\
    . & . & . & .  &  . & . & . & .  &  . & . & . & .  &  . & . & 1 & . \\
    \hline
    . & . & . & .  &  . & . & . & .  &  . & . & . & .  &  . & . & . & . \\
    . & . & . & .  &  . & . & . & .  &  . & . & . & .  &  . & . & . & . \\
    . & . & . & .  &  . & . & . & .  &  . & . & . & .  &  . & . & . & . \\
    . & . & . & .  &  . & . & . & .  &  . & . & . & .  &  . & . & . & .
    \end{array}
    \right)
    C_3^4 =
    \left(
    \begin{array}{c c c c | c c c c | c c c c | c c c c }
    . & . & . & .  &  . & 1 & . & .  &  . & . & . & 1  &  . & . & 1 & . \\
    . & . & . & .  &  . & . & 1 & .  &  . & . & . & \textbf{1}  &  . & . & \textbf{1} & 1 \\
    . & . & . & .  &  1 & . & . & .  &  . & . & 1 & .  &  . & . & . & 1 \\
    . & . & . & .  &  . & . & . & .  &  . & . & . & .  &  . & . & . & . \\
    \hline
    . & . & . & .  &  . & . & . & .  &  . & . & 1 & .  &  . & . & . & 1 \\
    . & . & . & .  &  . & . & . & .  &  . & . & . & 1  &  . & . & 1 & . \\
    . & . & . & .  &  . & . & . & .  &  . & . & . & 1  &  . & . & \textbf{1} & 1 \\
    . & . & . & .  &  . & . & . & .  &  . & . & . & .  &  . & . & 1 & . \\
    \hline
    . & . & . & .  &  . & . & . & .  &  . & . & . & .  &  . & . & . & . \\
    . & . & . & .  &  . & . & . & .  &  . & . & . & .  &  . & . & . & . \\
    . & . & . & .  &  . & . & . & .  &  . & . & . & .  &  . & . & . & 1 \\
    . & . & . & .  &  . & . & . & .  &  . & . & . & .  &  . & . & 1 & . \\
    \hline
    . & . & . & .  &  . & . & . & .  &  . & . & . & .  &  . & . & . & . \\
    . & . & . & .  &  . & . & . & .  &  . & . & . & .  &  . & . & . & . \\
    . & . & . & .  &  . & . & . & .  &  . & . & . & .  &  . & . & . & . \\
    . & . & . & .  &  . & . & . & .  &  . & . & . & .  &  . & . & . & .
    \end{array}
    \right)
    $$
    $$
    C_3^5 =
    \left(
    \begin{array}{c c c c | c c c c | c c c c | c c c c }
    . & . & . & .  &  . & 1 & . & .  &  . & . & \textbf{1} & 1  &  . & . & 1 & \textbf{1} \\
    . & . & . & .  &  . & . & 1 & .  &  . & . & . & 1  &  . & . & 1 & 1 \\
    . & . & . & .  &  1 & . & . & .  &  . & . & 1 & .  &  . & . & . & 1 \\
    . & . & . & .  &  . & . & . & .  &  . & . & . & .  &  . & . & . & . \\
    \hline
    . & . & . & .  &  . & . & . & .  &  . & . & 1 & .  &  . & . & . & 1 \\
    . & . & . & .  &  . & . & . & .  &  . & . & 1 & 1  &  . & . & 1 & \textbf{1} \\
    . & . & . & .  &  . & . & . & .  &  . & . & . & 1  &  . & . & 1 & 1 \\
    . & . & . & .  &  . & . & . & .  &  . & . & . & .  &  . & . & 1 & . \\
    \hline
    . & . & . & .  &  . & . & . & .  &  . & . & . & .  &  . & . & . & . \\
    . & . & . & .  &  . & . & . & .  &  . & . & . & .  &  . & . & . & . \\
    . & . & . & .  &  . & . & . & .  &  . & . & . & .  &  . & . & . & 1 \\
    . & . & . & .  &  . & . & . & .  &  . & . & . & .  &  . & . & 1 & . \\
    \hline
    . & . & . & .  &  . & . & . & .  &  . & . & . & .  &  . & . & . & . \\
    . & . & . & .  &  . & . & . & .  &  . & . & . & .  &  . & . & . & . \\
    . & . & . & .  &  . & . & . & .  &  . & . & . & .  &  . & . & . & . \\
    . & . & . & .  &  . & . & . & .  &  . & . & . & .  &  . & . & . & .
    \end{array}
    \right)
    C_3^6 =
    \left(
    \begin{array}{c c c c | c c c c | c c c c | c c c c }
    . & . & . & .  &  . & 1 & . & .  &  . & . & 1 & 1  &  . & . & 1 & 1 \\
    . & . & . & .  &  . & . & 1 & .  &  . & . & . & 1  &  . & . & 1 & 1 \\
    . & . & . & .  &  1 & . & . & .  &  . & . & 1 & \textbf{1}  &  . & . & \textbf{1} & 1 \\
    . & . & . & .  &  . & . & . & .  &  . & . & . & .  &  . & . & . & . \\
    \hline
    . & . & . & .  &  . & . & . & .  &  . & . & 1 & 1  &  . & . & \textbf{1} & 1 \\
    . & . & . & .  &  . & . & . & .  &  . & . & 1 & 1  &  . & . & 1 & 1 \\
    . & . & . & .  &  . & . & . & .  &  . & . & . & 1  &  . & . & 1 & 1 \\
    . & . & . & .  &  . & . & . & .  &  . & . & . & .  &  . & . & 1 & . \\
    \hline
    . & . & . & .  &  . & . & . & .  &  . & . & . & .  &  . & . & . & . \\
    . & . & . & .  &  . & . & . & .  &  . & . & . & .  &  . & . & . & . \\
    . & . & . & .  &  . & . & . & .  &  . & . & . & .  &  . & . & . & 1 \\
    . & . & . & .  &  . & . & . & .  &  . & . & . & .  &  . & . & 1 & . \\
    \hline
    . & . & . & .  &  . & . & . & .  &  . & . & . & .  &  . & . & . & . \\
    . & . & . & .  &  . & . & . & .  &  . & . & . & .  &  . & . & . & . \\
    . & . & . & .  &  . & . & . & .  &  . & . & . & .  &  . & . & . & . \\
    . & . & . & .  &  . & . & . & .  &  . & . & . & .  &  . & . & . & .
    \end{array}
    \right)
    $$
    \caption{Transitive closure for $3-6$ loop iterations for example query}
    \label{example:iteration3to6eval}
\end{figure}{}

\begin{figure}
    \centering
    $$
    M_2^3 =
    \begin{pmatrix}
    .     & \{a\} & \{S\} & .       \\
    .     & .     & \{a\} & \{S\}   \\
    \{a\} & .     & .     & \{b,S\} \\
    .     & .     & \{b\} & .
    \end{pmatrix}
    M_2^4 =
    \begin{pmatrix}
    .     & \{a\} & \{S\}   & .       \\
    .     & .     & \{a,S\} & \{S\}   \\
    \{a\} & .     & .       & \{b,S\} \\
    .     & .     & \{b\}   & .
    \end{pmatrix}
    $$
    $$
    M_2^5 =
    \begin{pmatrix}
    .     & \{a\} & \{S\}   & \{S\}   \\
    .     & .     & \{a,S\} & \{S\}   \\
    \{a\} & .     & .       & \{b,S\} \\
    .     & .     & \{b\}   & .
    \end{pmatrix}
    M_2^6 =
    \begin{pmatrix}
    .     & \{a\} & \{S\}   & \{S\}   \\
    .     & .     & \{a,S\} & \{S\}   \\
    \{a\} & .     & \{S\}   & \{b,S\} \\
    .     & .     & \{b\}   & .
    \end{pmatrix}
    $$
    \caption{The updated matrix $M_2$ for $3-6$ loop iterations for example query}
    \label{example:iteration3to6res}
\end{figure}{}

\begin{figure}
    \begin{center}
        \begin{tikzpicture}[shorten >=1pt,auto]
        \node[state] (q_0)                      {$0$};
        \node[state] (q_1) [above right=of q_0] {$1$};
        \node[state] (q_2) [right=of q_0]       {$2$};
        \node[state] (q_3) [right=of q_2]       {$3$};
          \path[->]
            (q_0) edge  node {a} (q_1)
            (q_1) edge  node {a,S} (q_2)
            (q_2) edge[bend right, above]  node {a} (q_0)
            (q_2) edge[loop right]  node {S} (q_2)
            (q_1) edge[bend left, above]  node {S} (q_3)
            (q_0) edge[bend right, above]  node {S} (q_2)
            (q_2) edge[bend left, above]  node {b,S} (q_3)
            (q_0) edge[bend right, below]  node {S} (q_3)
            (q_3) edge[bend left, below]  node {b} (q_2);
    \end{tikzpicture}
    \end{center}{}
    \caption{The result graph $\mathcal{G}$ for example query}
    \label{example:result}
\end{figure}{}