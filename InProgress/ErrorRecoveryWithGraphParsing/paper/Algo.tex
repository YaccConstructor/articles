\section{Error Recovery Algorithm}


Additional edges with error markers goes forward and with all tokens, goes in the its start vertex
(as a result we have loops).

\begin{tikzpicture}[shorten >=1pt,node distance=2cm,on grid,auto]
\node[state] (q_1)   {$1$};
\node[state] (q_2) [above=of q_1] {$2$};
\node[state] (q_3) [above right=of q_1, below right=of q_2] {$0$};
\node[state] (q_4) [right=of q_3] {$3$};
\path[->]
(q_1) edge  node {A} (q_2)
(q_2) edge  node {A} (q_3)
(q_3) edge  node {A} (q_1)
(q_3) edge[bend left, above]  node {B} (q_4)
(q_4) edge[bend left, below]  node {B} (q_3);
\end{tikzpicture}


Number of edges may be optimized by filtering with FIRST/REST and other functions

Select the best tree from SPPF after parsing finish.

Priority queue for descriptors.
How to choose priority function? --- ordered tuples!

Priority is a number of additional edges (not from the original input) in processed prefix.
Suffix length.
