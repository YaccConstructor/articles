% This is "sig-alternate.tex" V1.9 April 2009
% This file should be compiled with V2.4 of "sig-alternate.cls" April 2009
%
% This example file demonstrates the use of the 'sig-alternate.cls'
% V2.4 LaTeX2e document class file. It is for those submitting
% articles to ACM Conference Proceedings WHO DO NOT WISH TO
% STRICTLY ADHERE TO THE SIGS (PUBS-BOARD-ENDORSED) STYLE.
% The 'sig-alternate.cls' file will produce a similar-looking,
% albeit, 'tighter' paper resulting in, invariably, fewer pages.
%
% ----------------------------------------------------------------------------------------------------------------
% This .tex file (and associated .cls V2.4) produces:
%       1) The Permission Statement
%       2) The Conference (location) Info information
%       3) The Copyright Line with ACM data
%       4) NO page numbers
%
% as against the acm_proc_article-sp.cls file which
% DOES NOT produce 1) thru' 3) above.
%
% Using 'sig-alternate.cls' you have control, however, from within
% the source .tex file, over both the CopyrightYear
% (defaulted to 200X) and the ACM Copyright Data
% (defaulted to X-XXXXX-XX-X/XX/XX).
% e.g.
% \CopyrightYear{2007} will cause 2007 to appear in the copyright line.
% \crdata{0-12345-67-8/90/12} will cause 0-12345-67-8/90/12 to appear in the copyright line.
%
% ---------------------------------------------------------------------------------------------------------------
% This .tex source is an example which *does* use
% the .bib file (from which the .bbl file % is produced).
% REMEMBER HOWEVER: After having produced the .bbl file,
% and prior to final submission, you *NEED* to 'insert'
% your .bbl file into your source .tex file so as to provide
% ONE 'self-contained' source file.
%
% ================= IF YOU HAVE QUESTIONS =======================
% Questions regarding the SIGS styles, SIGS policies and
% procedures, Conferences etc. should be sent to
% Adrienne Griscti (griscti@acm.org)
%
% Technical questions _only_ to
% Gerald Murray (murray@hq.acm.org)
% ===============================================================
%
% For tracking purposes - this is V1.9 - April 2009

\documentclass{sig-alternate-05-2015}
  \pdfpagewidth=8.5truein
  \pdfpageheight=11truein

\usepackage{verbatim}
\usepackage{graphicx}
\usepackage{subcaption}
\usepackage{hyperref}
\usepackage{listings}
\usepackage{courier}
\usepackage{epstopdf}

\usepackage{tikz}
\usetikzlibrary{automata,positioning}


% \lstset{language=[Sharp]C}
\lstset{numbers=left,xleftmargin=3em,numberstyle=\footnotesize\ttfamily,captionpos=b}
\lstset{basicstyle=\footnotesize\ttfamily}

\begin{document}
\setcopyright{acmlicensed}
%
% --- Author Metadata here ---
% \conferenceinfo{SAC'15}{April 13-17, 2015, Salamanca, Spain.}
% \CopyrightYear{2015} % Allows default copyright year (2002) to be over-ridden - IF NEED BE.
% \crdata{978-1-4503-3196-8/15/04}  % Allows default copyright data (X-XXXXX-XX-X/XX/XX) to be over-ridden.
% --- End of Author Metadata ---

\title{Syntax Error Recovery via Context-Free Language Reachability}
% \subtitle{[Extended Abstract]
% \titlenote{A full version of this paper is available as
% \textit{Author's Guide to Preparing ACM SIG Proceedings Using
% \LaTeX$2_\epsilon$\ and BibTeX} at
% \texttt{www.acm.org/eaddress.htm}}}
%
% You need the command \numberofauthors to handle the 'placement
% and alignment' of the authors beneath the title.
%
% For aesthetic reasons, we recommend 'three authors at a time'
% i.e. three 'name/affiliation blocks' be placed beneath the title.
%
% NOTE: You are NOT restricted in how many 'rows' of
% "name/affiliations" may appear. We just ask that you restrict
% the number of 'columns' to three.
%
% Because of the available 'opening page real-estate'
% we ask you to refrain from putting more than six authors
% (two rows with three columns) beneath the article title.
% More than six makes the first-page appear very cluttered indeed.
%
% Use the \alignauthor commands to handle the names
% and affiliations for an 'aesthetic maximum' of six authors.
% Add names, affiliations, addresses for
% the seventh etc. author(s) as the argument for the
% \additionalauthors command.
% These 'additional authors' will be output/set for you
% without further effort on your part as the last section in
% the body of your article BEFORE References or any Appendices.

\numberofauthors{3} %  in this sample file, there are a *total*
% of EIGHT authors. SIX appear on the 'first-page' (for formatting
% reasons) and the remaining two appear in the \additionalauthors section.
%
\author{
% You can go ahead and credit any number of authors here,
% e.g. one 'row of three' or two rows (consisting of one row of three
% and a second row of one, two or three).
%
% The command \alignauthor (no curly braces needed) should
% precede each author name, affiliation/snail-mail address and
% e-mail address. Additionally, tag each line of
% affiliation/address with \affaddr, and tag the
% e-mail address with \email.
%
% 1st. author
% 1st. author
\alignauthor Semyon Grigorev\\
       \affaddr{St. Petersburg State University}\\
       \affaddr{198504, Universitetsky prospekt 28}\\
       \affaddr{Peterhof, St. Petersburg, Russia}\\
       \email{rsdpisuy@gmail.com}
}
% There's nothing stopping you putting the seventh, eighth, etc.
% author on the opening page (as the 'third row') but we ask,
% for aesthetic reasons that you place these 'additional authors'
% in the \additional authors block, viz.
% \additionalauthors{Additional authors: John Smith (The
% Th{\o}rv{\"a}ld Group, email: {\texttt{jsmith@affiliation.org}})
% and Julius P.~Kumquat (The Kumquat Consortium, email:
% {\texttt{jpkumquat@consortium.net}}).}
%\date{27 July 2018}
% Just remember to make sure that the TOTAL number of authors
% is the number that will appear on the first page PLUS the
% number that will appear in the \additionalauthors section.

\maketitle

\begin{abstract}

Syntax error recovery is an important feature of such development tools as IDEs and language services.
In practice, it is important to find the best recovery: minimal in the number of transformations of the erroneous string.
We formalize error recovery in terms of Context-Free Path Querying (CFPQ): a problem of searching for paths in a graph which form sentences of a context-free language.
Considering the erroneous string to be a linear graph with edges marked with the individual tokens of the string, we can add weighted edges which represent individual transformations (insertions, deletions, replacements).
Then the best recovery is a path in this graph of minimal weight that forms a sentence of the programming language, which is what CFPQ is capable of finding.
There are practical parsing algorithms for CFPQ, including the GLL-based algorithm which constructs a parse forest (SPPF).
We believe that using the proposed approach, it is possible to implement better practical error recovery.

%Is it possibe to create effective syntax error recovery algorithm by using

\end{abstract}

\printccsdesc

\keywords{Syntax analysis, parsing, error recovery, CFL reachability, Generalized parsing, GLL}

\section{Introduction}

Context-Free Path Querying (CFPQ) is an actively developed area in graph database analysis.
CFPQ is also used for static code analysis~\cite{Reps,10.1145/193173.195287,Zheng}, RDF querying~\cite{10.1007/978-3-319-46523-4_38,MEDEIROS201975}, biological data analysis~\cite{cfpqBio}.

Most of research is focused on developping algorithms for CFPQ evaluation~\cite{hellingsRelational,ward2008distributed,cfpqBio,MEDEIROS201975,Azimov:2018:CPQ:3210259.3210264,Grigorev:2017:CPQ:3166094.3166104}, whereas specification languages for context-free queries are not investigated enough.
Best to our knowledge, only one extension for Sparql supports context-free constarints: cfSPARQL~\cite{10.1007/978-3-319-46523-4_38}.
There is also a proposal for CFPQ as a part of Cypher\footnote{Proposal with path pattern syntax for openCypher: \url{https://github.com/thobe/openCypher/blob/rpq/cip/1.accepted/CIP2017-02-06-Path-Patterns.adoc}.
It is shown that context-free constraints can be expressed with the proposed syntax. Access date: 30.03.2020} language, but there is no implementation for it yet.
We believe that more research should be conducted on the specification languages fo context-free constraints in graph querying.

It is worth noting that graph analysis is often only a part of a more complex system, usually implemented in a general-purpose language.
Since a graph query language is unsuitable to implement a whole system, there should be means of integration of them into general-purpose programming languages.
There are many ways to integrate them ranging from creating graph queries from string values of a general-purpose language to implementing a special embedded domain specific language, and even more sophisticated.

Although simple, the string manipulating approach does not provide a developper with any safety guarantees.
There is no way to ensure that a string generated by an application is a valid query or, in case it is not, to provide any feedback.
This makes string manipulating technique error prone, the code --- unclear and hard to maintain.

Safety of an embedded DSL entirely depends on its implementation.
Some general-purpose languages with powerfull type systems (such as \haskell{}, \ocaml{} or \scala{}) or the ones supporting hygienic macros (such as \scheme{} or \rust{}) facilitate creating safe and reliable DSLs.
Still, they typically lack full support of a development environment: it may be harder to debug queries or issues with composability may arise.

There is a general trend towards imposing more restricting type systems on programming languages.
Among many others are typing annotations for \python{} and \typescript{} code and nullability checks in \kotlin{}.
Typing graphs and query languages improves  readability and simplifies maintainance~\cite{10.1145/2076623.2076653}.

Parser combinators are the answer to the integration of parsing into a general-purpose programming language.
Recursive descend parsers are encoded as functions of the host language, while grammar constructions such as sequencing and choice are implemented as higher-order functions.
This idea was first introduced in~\cite{burge} and further developped in numerous works.
Notable development is monadic parser combinators~\cite{hutton1996monadic}.
In this approach, one can not only parse the input, but simultaneously run semantics calculation if parsing succeeds.
Paper~\cite{izmaylova2016practical} proposed the first monadic parser combinator library which solves the long-standing problem of inability to handle ambiguous and left-recursive grammars.
A library for graph querying was developped~\cite{10.1145/3241653.3241655} based on this work.
The core idea is to use generalized parser combinators as both a way to formulate a query and to execute it.
This approach inherits benefits of combinatory parsing: ease of code reuse, type safety guaranteed by the host language and, since the parser is simply a function, the integrated development support.

Besides integration, it can compute both the single-source and all pairs semantics, as well as execute user actions.
The~single-source semantics is relevant to many real-world applications, including manual data analysis.
It also may be less time-intensive, since on average it needs to expore only a subgraph of the input graph.
Many querying algorithms are only capable to compute all pairs reachability which makes them unsuitable for some applications.

In this paper we make the following contributions.
\begin{itemize}
  \item We demonstrate how to use combinatory-based graph querying on example.
  \item We illustrate such features of the approach as type-safety, flexibility (composability and generics), IDE support and computing user-defined actions.
  \item We evaluate single-source context-free path querying on some real-world RDFs.
  \begin{itemize}
    \item Based on our evaluation, the most common case in RDF context-free querying is when the number of paths in the answer set is big, but they are small.
    \item We demonstrate that the single-source CFPQ can feasibly be used to evaluate such queries.
    \item We conclude that there is a need for a further detailed analysis of both theoretical time and space complexity of single-source CFPQ.
  \end{itemize}
\end{itemize}

\section{Error Recovery Algorithm}



Additional edges with error markers goes forward and with all tokens, goes in the its start vertex 
(as a result we have loops).
Number of edges may be optimized by filtering with FIRST/REST and other functions

Priority queue for descriptors. 
Priority is a number of additional edges (not from the original input) in processed prefix.
\section{Evaluation}

The goal of this evaluation is to assess the performance scaling of Spla on Vortex.
Due to limitations in atomic operation support within the RTL implementation, all experiments were performed using the SimX functional simulator.

\subsection{Environment}

Initial testing revealed issues with floating-point operations, which produced incorrect results for some hardware configurations.
Consequently, we limited subsequent experiments to Breadth-First Search (BFS) and Triangle Counting (TC), excluding Single-Source Shortest Path (SSSP) and PageRank.
To keep simulation times manageable, we used a single graph from the SuiteSparse matrix collection\footnote{A diverse collection of sparse matrices from various domains: \url{http://sparse.tamu.edu/}}: soc-Epinions1, with 75~888 vertices and 508~837 edges.


We conducted two series of experiments.
The first varies the number of warps and threads per warp while keeping the number of clusters and cores fixed (at 2 and 4, respectively), with the goal of selecting the best core configuration while preserving multi-core execution to account for cache effects.
The second series, using the best configuration identified in the first step, varies the number of clusters and cores per cluster to assess scaling at the core and cluster levels.
Cache sizes were set to their default values: 16 KB for $L_1$, 1 MB for $L_2$, and 2 MB for $L_3$.

We use the number of cycles reported by SimX as a performance metric.
For multi-core configurations, we report the maximum cycle count across all cores.
During the experiments, we encountered unexpected behavior in SimX that led to out-of-memory exceptions. 
Therefore, some data points are missing from the graphs below.

\subsection{Results}

In figures~\ref{fig:tc_threads_warps} and~\ref{fig:bfs_threads_warps}

\begin{figure}
    \begin{center}
        \includegraphics[width=0.49\textwidth]{pictures/TC_threads_warps.pdf}
    \end{center}
    \caption{Scaling analysis of triangle counting for varying numbers of warps and threads per warp}
    \label{fig:tc_threads_warps}
\end{figure}

\begin{figure}
    \begin{center}
        \includegraphics[width=0.49\textwidth]{pictures/BFS_threads_warps.pdf}
    \end{center}
    \caption{Scaling analysis of BFS for varying numbers of warps and threads per warp}
    \label{fig:bfs_threads_warps}
\end{figure}

Best configuration for BFS is 2 warps, 8 threads per warp (16 threads total). 
Best configuration for TC is 4 warps, 16 threads per warp (64 threads total).


\begin{figure}
    \begin{center}
        \includegraphics[width=0.49\textwidth]{pictures/BFS_cores_clusters.pdf}
    \end{center}
    \caption{Scaling analysis of BFS for varying numbers of clusters and cores per cluster}
    \label{fig:bfs_cores_clusters}
\end{figure}


Edges per core on cycle. Compare with Spla on other GPUs.

\subsection{Scaling limitations analysis}

%sum(scoreboard stalls * lsu_percent) / sum(instr) * 100
To analyze the reasons for limited scaling as the number of threads increases, we measured the average utilization of the ALU and LSU, in terms of stall cycles, for the best BFS configuration.
The results are presented in Fig.~\ref{fig:bfs_alu_stalls} and Fig.~\ref{fig:bfs_lsu_stalls}, respectively.
The data indicate that the LSU is the performance bottleneck within the core.

The same bottleneck was observed in the scaling analysis across clusters and cores.
Whether increasing cache sizes can alleviate this problem remains a question for future research.
We anticipate that careful cache size tuning may help identify a more efficient configuration.

\begin{figure}
    \begin{center}
        \includegraphics[width=0.49\textwidth]{pictures/BFS_alu.pdf}
    \end{center}
    \caption{ALU stalls on BFS for the best configuration}
    \label{fig:bfs_alu_stalls}
\end{figure}

\begin{figure}
    \begin{center}
        \includegraphics[width=0.49\textwidth]{pictures/BFS_lsu.pdf}
    \end{center}
    \caption{LSU stalls on BFS for the best configuration}
    \label{fig:bfs_lsu_stalls}
\end{figure}
\section{Conclusion and Future Work}

Platform presented.

Education. Metaprogramming, translators development, GPGPU programming, etc.

Graph parsing.

Geterogenious porgramming generalization. Hopac is better then MBP~\footnote{\url{https://vasily-kirichenko.github.io/fsharpblog/actors}}.

Research: Automatic memory management.

Data to code translation (automata can be translated into code instead of data structures in memory)

Other technical improvements: IDE support, type provider improvements, new OpenCL standard support, runtime extension, etc.

\bibliographystyle{abbrv}
\bibliography{Main}

\balancecolumns

\end{document}
