%%
%% This is file `sample-sigconf.tex',
%% generated with the docstrip utility.
%%
%% The original source files were:
%%
%% samples.dtx  (with options: `sigconf')
%% 
%% IMPORTANT NOTICE:
%% 
%% For the copyright see the source file.
%% 
%% Any modified versions of this file must be renamed
%% with new filenames distinct from sample-sigconf.tex.
%% 
%% For distribution of the original source see the terms
%% for copying and modification in the file samples.dtx.
%% 
%% This generated file may be distributed as long as the
%% original source files, as listed above, are part of the
%% same distribution. (The sources need not necessarily be
%% in the same archive or directory.)
%%
%% The first command in your LaTeX source must be the \documentclass command.
\documentclass[sigconf]{acmart}
%% NOTE that a single column version may be required for 
%% submission and peer review. This can be done by changing
%% the \doucmentclass[...]{acmart} in this template to 
%% \documentclass[manuscript,screen]{acmart}
%% 
%% To ensure 100% compatibility, please check the white list of
%% approved LaTeX packages to be used with the Master Article Template at
%% https://www.acm.org/publications/taps/whitelist-of-latex-packages 
%% before creating your document. The white list page provides 
%% information on how to submit additional LaTeX packages for 
%% review and adoption.
%% Fonts used in the template cannot be substituted; margin 
%% adjustments are not allowed.
%%
%%


\usepackage{colortbl}
\usepackage{xcolor}
\usepackage{multirow}

%% Rights management information.  This information is sent to you
%% when you complete the rights form.  These commands have SAMPLE
%% values in them; it is your responsibility as an author to replace
%% the commands and values with those provided to you when you
%% complete the rights form.
\setcopyright{acmcopyright}
\copyrightyear{2018}
\acmYear{2018}
\acmDOI{10.1145/1122445.1122456}

%% These commands are for a PROCEEDINGS abstract or paper.
\acmConference[Woodstock '18]{Woodstock '18: ACM Symposium on Neural
  Gaze Detection}{June 03--05, 2018}{Woodstock, NY}
\acmBooktitle{Woodstock '18: ACM Symposium on Neural Gaze Detection,
  June 03--05, 2018, Woodstock, NY}
\acmPrice{15.00}
\acmISBN{978-1-4503-XXXX-X/18/06}


%%
%% Submission ID.
%% Use this when submitting an article to a sponsored event. You'll
%% receive a unique submission ID from the organizers
%% of the event, and this ID should be used as the parameter to this command.
%%\acmSubmissionID{123-A56-BU3}

%%
%% The majority of ACM publications use numbered citations and
%% references.  The command \citestyle{authoryear} switches to the
%% "author year" style.
%%
%% If you are preparing content for an event
%% sponsored by ACM SIGGRAPH, you must use the "author year" style of
%% citations and references.
%% Uncommenting
%% the next command will enable that style.
%%\citestyle{acmauthoryear}

%%
%% end of the preamble, start of the body of the document source.
\begin{document}

%%
%% The "title" command has an optional parameter,
%% allowing the author to define a "short title" to be used in page headers.
\title{GLL-based Context-Free Path Querying for Neo4j}

%%
%% The "author" command and its associated commands are used to define
%% the authors and their affiliations.
%% Of note is the shared affiliation of the first two authors, and the
%% "authornote" and "authornotemark" commands
%% used to denote shared contribution to the research.
\author{Vlada Pogozhelskaya}
\affiliation{%
  \institution{Saint Petersburg State University}
  \streetaddress{7/9 Universitetskaya nab.}
  \city{St. Petersburg}
  \country{Russia}
  \postcode{199034}
}
\affiliation{
		\institution{JetBrains Research}
		\streetaddress{Primorskiy prospekt 68-70, Building 1}
		\city{St. Petersburg}
		\country{Russia}
		\postcode{199034}
	}
\email{pogozhelskaya@gmail.com}
%\orcid{1234-5678-9012}


\author{Anna Vlasova}
\affiliation{
		\institution{JetBrains Research}
		\streetaddress{Primorskiy prospekt 68-70, Building 1}
		\city{St. Petersburg}
		\country{Russia}
		\postcode{199034}
	}
\email{anna.vlasova@jetbrains.com}
%\orcid{1234-5678-9012}

\author{Semyon Grigorev}
\affiliation{
		\institution{Saint Petersburg State University}
		\streetaddress{7/9 Universitetskaya nab.}
		\city{St. Petersburg}
		\country{Russia}
		\postcode{199034}
	}
	\affiliation{
		\institution{JetBrains Research}
		\streetaddress{Primorskiy prospekt 68-70, Building 1}
		\city{St. Petersburg}
		\country{Russia}
		\postcode{199034}
	}
\email{s.v.grigoriev@spbu.ru}
\email{semyon.grigorev@jetbrains.com}
\orcid{0000-0002-7966-0698}


%%
%% By default, the full list of authors will be used in the page
%% headers. Often, this list is too long, and will overlap
%% other information printed in the page headers. This command allows
%% the author to define a more concise list
%% of authors' names for this purpose.
\renewcommand{\shortauthors}{Pogozhelskaya, Vlasova, Grigorev.}

%%
%% The abstract is a short summary of the work to be presented in the
%% article.
\begin{abstract}
  We show that a fast GLL parsing algorithm provides \textit{context-free path querying} for Neo4j graph database with reasonable performance. The proposed solution solves both the \textit{reachability} and the \textit{all paths} problems for the \textit{all pairs} and the \textit{multiple sources} cases. The evaluation on real-world graphs demonstrates that our solution is more than 25 times faster than the previous solution for Neo4j and is comparable, in some cases,  with the linear algebra based solution for RedisGraph.
\end{abstract}

%%
%% The code below is generated by the tool at http://dl.acm.org/ccs.cfm.
%% Please copy and paste the code instead of the example below.
%%
\begin{CCSXML}
		<ccs2012>
		<concept>
			<concept_id>10002951.10002952.10003197.10010825</concept_id>
			<concept_desc>Information systems~Query languages for non-relational engines</concept_desc>
			<concept_significance>500</concept_significance>
		</concept>
		<concept>
			<concept_id>10003752.10003766.10003771</concept_id>
			<concept_desc>Theory of computation~Grammars and context-free languages</concept_desc>
			<concept_significance>500</concept_significance>
		</concept>
		<concept>
			<concept_id>10002950.10003624.10003633.10003640</concept_id>
			<concept_desc>Mathematics of computing~Paths and connectivity problems</concept_desc>
			<concept_significance>300</concept_significance>
		</concept>
		<concept>
			<concept_id>10002951.10002952.10002953.10010146</concept_id>
			<concept_desc>Information systems~Graph-based database models</concept_desc>
			<concept_significance>500</concept_significance>
		</concept>
		</ccs2012>
\end{CCSXML}

    \ccsdesc[500]{Information systems~Graph-based database models}
	\ccsdesc[500]{Information systems~Query languages for non-relational engines}
	\ccsdesc[500]{Theory of computation~Grammars and context-free languages}
    \ccsdesc[300]{Mathematics of computing~Paths and connectivity problems}

%%
%% Keywords. The author(s) should pick words that accurately describe
%% the work being presented. Separate the keywords with commas.
\keywords{Graph database, context-free path querying, CFPQ, reachability problem, all paths problem, generalized LL, GLL}

%%
%% This command processes the author and affiliation and title
%% information and builds the first part of the formatted document.
\maketitle

\section{Introduction}

Scalable high-performance graph analysis is an actual challenge.
There is a big number of ways to attack this challenge~\cite{Coimbra2021} and the first promising idea is to utilize general-purpose graphic processing units (GPGPU).
Such existing solutions, as CuSha~\cite{10.1145/2600212.2600227} and Gunrock~\cite{7967137} show that utilization of GPUs can improve the performance of graph analysis, moreover it is shown that solutions may be scaled to multi-GPU systems.
But low flexibility and high complexity of API are problems of these solutions.

The second promising thing which provides a user-friendly API for high-performance graph analysis algorithms creation is a GraphBLAS API~\cite{7761646} which provides linear algebra based building blocks to create graph analysis algorithms.
The idea of GraphBLAS is based on a well-known fact that linear algebra operations can be efficiently implemented on parallel hardware.
Along with that, a graph can be natively represented using matrices: adjacency matrix, incidence matrix, etc.
While reference CPU-based implementation of GraphBLAS, SuiteSparse:GraphBLAS~\cite{10.1145/3322125}, demonstrates good performance in real-world tasks, GPU-based implementation is challenging.

One of the challenges in this way is that real data are often sparse, thus underlying matrices and vectors are also sparse, and, as a result, classical dense data structures and respective algorithms are inefficient. 
So, it is necessary to use advanced data structures and procedures to implement sparse linear algebra, but the efficient implementation of them on GPU is hard due to the irregularity of workload and data access patterns.
Though such well-known libraries as cuSPARSE show that sparse linear algebra operations can be efficiently implemented for GPGPU, it is not so trivial to implement GraphBLAS on GPGPU. 
First of all, it requires \textit{generic} sparse linear algebra, thus it is impossible just to reuse existing libraries which are almost all specified for operations over floats.
The second problem is specific optimizations, such as masking fusion, which can not be natively implemented on top of existing kernels.
Nevertheless, there is a number of implementations of GraphBLAS on GPGPU, such as GraphBLAST~\cite{yang2019graphblast}, GBTL~\cite{7529957}, which show that GPGPUs utilization can improve the performance of GraphBLAS-based graph analysis solutions.
But these solutions are not portable because they are based on Nvidia Cuda stack.
Moreover, the scalability problem is not solved: all these solutions support only single-GPU, not multi-GPU computations.

To provide portable GPU implementation of GraphBLAS API we developed a \textit{SPLA} library\footnote{Source code available at: \url{https://github.com/JetBrains-Research/spla}}.
This library utilizes OpenCL for GPGPU computing to be portable across devices of different vendors.
Moreover, it is initially designed to utilize multiple GPGPUs to be scalable.
To sum up, the contribution of this work is the following.
\begin{itemize}
    \item Design of portable GPU GraphBLAS implementation proposed. The design involves the utilization of multiple GPUS. Additionally, the proposed design is aimed to simplify library tuning and wrappers for different high-level platforms and languages creation. 
    \item Subset of GraphBLAS API, including such operations as masking, matrix-matrix multiplication, matrix-matrix e-wise addition, is implemented. The current implementation is limited by COO and CSR matrix representation format and uses basic algorithms for some operations, but work in progress and more data formats will be supported and advanced algorithms will be implemented in the future.
    \item Preliminary evaluation on such algorithms as breadth-first search (BFS) and triangles counting (TC), and real-world graphs shows portability across different vendors and promising performance: for some problems Spla is comparable with GraphBLAST. Surprisingly, for some problems, the proposed solution on embedded Intel graphic card shows better performance than SuiteSparse:GraphBLAS on the respective CPU. At the same time, the evaluation shows that further optimization is required.
\end{itemize} 
\section{Preliminaries}

In this section we introduce common definitions in graph theory and formal language theory which will be used in this paper. 
Also, we provide brief description of Azimov's algorithm which is used as a base of our solution.

\subsection{Graphs}

In this work we use edge-labelled digraph as a data model and define it as follows.
\begin{definition} \emph{Labeled directed graph} is a triple $D = (V,E,\sigma)$, where
\begin{itemize}
    \item $V$ is a set of vertices
    \item $E$ is a set of edges
    \item $\sigma \subseteq \Sigma$ is a set of labels, and a set of edges $E\subseteq V\times \sigma \times V$
\end{itemize}
\end{definition}

An example of the graph is presented in figure~\ref{fig:example_input_graph}.

\begin{figure}[h]
    \centering        
    \begin{tikzpicture}[shorten >=1pt,auto]
       \node[state] (q_0)                      {$0$};
       \node[state] (q_1) [above right=of q_0] {$1$};
       \node[state] (q_2) [right=of q_0]       {$2$};
       \node[state] (q_3) [right=of q_2]       {$3$};
        \path[->]
        (q_0) edge  node {a} (q_1)
        (q_1) edge  node {a} (q_2)
        (q_2) edge  node {a} (q_0)
        (q_2) edge[bend left, above]  node {b} (q_3)
        (q_3) edge[bend left, below]  node {b} (q_2);
    \end{tikzpicture}
    \caption{The example of input graph $\mathcal{G}$}
    \label{fig:example_input_graph}
\end{figure}

We use adjacency matrix decomposed to a set of a boolean matrix as a representation of the graph.
\begin{definition}
An adjacency matrix $M$ of the graph $\mathcal{G}=$ is a square $|V|\times|V|$ matrix, such that $M[i,j] = \{l \mid e = (i,l,j) \in E\}$.
\end{definition}

Adjacency matrix $M$ of the graph $\mathcal{G}$ is

$$
    M =
    \begin{pmatrix}
    . & \{a\} & . & .     \\
    . & . & \{a\} & .     \\
    \{a\} & . & . & \{b\} \\
    . & . & \{b\} & .
    \end{pmatrix}.
$$

\begin{definition}

Boolean decomposition of adjacency matrix $M$ of graph $\mathcal{G}=$ is set of Boolean matrix $$\mathcal{M} = \{M^l \mid l \in L, M^l[i,j]=1 \iff l \in M[i,j]\}.$$

\end{definition}

Matrix $M$ can be represented as a set of two Boolean matrices $M^a$ and $M^b$ where
\begin{align}
M^{a} =
\begin{pmatrix}
    . & 1 & . & .   \\
    . & . & 1 & .   \\
    1 & . & . & .   \\
    . & . & . & .  
\end{pmatrix}, 
M^{b} =
\begin{pmatrix}      
    . & . & . & .   \\
    . & . & . & .   \\
    . & . & . & 1   \\
    . & . & 1 & . 
\end{pmatrix} \label{eq:boolean_decomposition_of_graph}
\end{align}
\subsection{Languages}

\begin{definition}\emph{Context-free grammar} is a 4-tuple $G=(N, \Sigma, R, S)$, where 
\begin{itemize}
    \item $N$ is a set of nonterminals
    \item $\Sigma$ is a set of terminals
    \item $R$ is a finite set of productions of the followings form: $A \to \alpha, ~A \in N,~ \alpha \in (N \cup \Sigma)^*$
    \item $S$ - a starting nonterminal
\end{itemize}
\end{definition}

\begin{definition} \emph{Context-free language} is a language generated by a context-free grammar:
\begin{align*}
     L(G) = \{w \in \Sigma^* \mid S \Rightarrow^* w \} 
\end{align*}
Where $S \Rightarrow^* w$  denotes that a string $w$ can be generated from a starting non-terminal $S$ using some sequence of production rules from $P$.
\end{definition}

\begin{definition} Context-free grammar $G = (N, \Sigma, R, S)$ is said to be in \emph{Chomsky normal form} if all productions in $R$ are of the form:
    \begin{itemize}
        \item $A \rightarrow BC,~A,~B,~C \in N$
        \item  $A \rightarrow a,~A \in N,~a \in \Sigma$
        \item $S \rightarrow \varepsilon,~\varepsilon$ is an empty string
    \end{itemize}
\end{definition}
Note that every context-free grammar can be transformed into an equivalent one in Chomsky Normal Form. 
\begin{definition} Context-free grammar $G = (N, \Sigma, P, S)$ is said to be in \emph{Weak Chomsky normal form} if all productions in $P$ are of the form:
    \begin{itemize}
        \item $A \rightarrow BC,~A,~B,~C \in N$
        \item  $A \rightarrow a,~A \in N,~a \in \Sigma$
        \item $A \rightarrow \varepsilon,~A \in N$
    \end{itemize}
\end{definition}
In other words, weak Chomsky normal form differs from Chomsky normal Form in the followings:
\begin{itemize}
    \item $\varepsilon$ can be derived from any non-terminal
    \item $S$ can be at a right part of productions
\end{itemize}
    
    
For example, let's consider the following context-free grammar, which generates the language $L(G) = \{A^nB^n, n \in \mathbb{N}\}$:
$G=(N, \Sigma, P, S), ~N=\{S\},~\Sigma=\{A,B\}$ and productions: 
\begin{align*}
S \rightarrow AB \\
S \rightarrow ASB\\
S \rightarrow \varepsilon
\end{align*}
After transformation to Chomsky Normal Form the resulting grammar:
\begin{align*}
S \rightarrow AB \\
S \rightarrow AC \\
C \rightarrow SB \\
S \rightarrow \varepsilon
\end{align*}

This productions itself are the grammar that has the same result as original grammar.

We use a context-free grammar in the weak Chomsky Normal Form without a starting non-terminal, which will be specified in the path queries for the graph. It should be noted that we omit the rules of the form $A \rightarrow \varepsilon$ for the reason that they correspond to trivial paths, which are more convenient to consider separately.

\begin{definition}\emph{Context-free relation} is a relation $R_A \subseteq V \times V$ for graph $G = (V, E)$, context-free grammar $G = (N,~\Sigma,~P)$ and fixed non-terminal $A$:
\begin{align*}
     R_A = \{(n, m) \mid \exists n \pi m~(l(\pi) \in L(G_A))\}
\end{align*}
\end{definition}

 Now, the definition for \emph{multiple-source (single-source) context-free path querying problem} can be formulated in the introduced notation as follows. For the given graph $G = (V, E)$, context-free grammar $G=(N, \Sigma, P)$ and set of source vertices $Src$ we need to find all context-free relations $R_A$ for any $A \in Src$. 
 
\subsection{Matrix-Based Algorithm}
Let $D = (V, E)$ be the input graph and $G = (N, \Sigma, P)$ be the input grammar. For the context-free path query evaluation, we need to provide context-free relations \mbox{$R_A \subseteq V \times V$} for every \mbox{$A \in N$}.
The matrix-based algorithm for CFPQ can be expressed in terms of operations over Boolean matrices (see listing~\ref{alg:algo0}) which is an advantage for implementation.
{\footnotesize
\begin{algorithm}
\begin{algorithmic}[1]
\caption{Context-free path querying algorithm}
\label{alg:algo0}
\Function{evalCFPQ}{$D=(V,E), G=(N,\Sigma,P)$}
    \State{$n \gets$ |V|}
    \State{$T \gets \{T^{A_i} \mid A_i \in N, T^{A_i}$ is a matrix $n \times n$, $T^{A_i}_{k,l} \gets$ \texttt{false}\} }
    \ForAll{$(i,x,j) \in E$, $A_k \mid A_k \to x \in P$}
        %\Comment{Matrices initialization}
        %\For{$A_k \mid A_k \to x \in P$}
          {$T^{A_k}_{i,j} \gets \texttt{true}$}
        %\EndFor
    \EndFor
    \ForAll{$A_k \mid A_k \to \varepsilon \in P$}
        \ForAll{$i \in \{0,\ldots ,n-1\}$}
            {$T^{A_k}_{i,i} \gets \texttt{true}$}
        \EndFor
    \EndFor

    \While{any matrix in $T$ is changing}
        %\Comment{Transitive c	losure calculation}
        \For{$A_i \to A_j A_k \in P$}
          { $T^{A_i} \gets T^{A_i} + (T^{A_j} \times T^{A_k})$ } 
        \EndFor
    \EndWhile
\State \Return $T$
\EndFunction
\end{algorithmic}
\end{algorithm}
}

This CFPQ algorithm allows efficiently apply GPGPU techniques, but it solves all-pairs problem and takes unreasonable amount of memory in scenarios in which we want to find paths from a relatively small set of vertices, since it calculates a lot of redundant information.  
\section{Solution Description}

\subsection{Design Principles}

SPLA library is designed the way to maximize potential library performance, simplify its implementation and extensions, and to provided the end-user verbose, but effective interface allowing customization and precise control over operations execution. These ideas are captured in the following principles.

\begin{itemize}
    \item \textit{DAG-based expressions}. User constructs a computational expression from basic nodes and uses oriented edges to describe data dependencies between these nodes. 
    \item \textit{Automated hybrid-storage format}. Library uses internally specialized preprocessing to format data and automate its sharing between computational nodes.
    \item \textit{Automated multi-GPU scheduling}. Computational work is automatically scheduled between available devices for execution. Scheduling order, dependencies, and granularity are defined from DAG expression, submitted by a user.
    \item \textit{Customization of primitive types and operations}. Underlying primitives types and functions can be customized by user. The customization process does not require library re-compilation. 
    \item \textit{Exportable interface}. The library has a C++ interface with an automated reference-counting and with no-templates usage. It can be wrapped by C99 compatible API and exported to other languages, for example, in a form of a Python package.
\end{itemize}

\subsection{Architecture Overview}

Library general execution architecture is depicted in Fig.~\ref{fig:architecture}. As an input library accepts expression composed in the form of a DAG.
Nodes represent fundamental operations, such as matrix-matrix multiplication. 
Links describe dependencies between nodes.
Expression execution is \textit{asynchronous}. 
User can block and wait until its completion, or without blocking probe the expression until it is either \textit{completed} or \textit{aborted}. 

Expression is transformed into a task graph. 
The task graph is submitted for execution to the task manager. 
Each task is processed by specialized \textit{NodeProcessor}, capable of processing a particular node type.
Each task, when executed, is split dynamically into a set of parallel sub-tasks. 
Each sub-task is processed by specialized \textit{Algorithm}, which is capable of processing input blocks of matrices or vectors in particular storage formats with a specific set of options. \textit{NodePorcessor} and \textit{Algorithm} are selected at runtime from a registry using properties and arguments of the expression. 
Thus, it allows precise processing and optimization of edge-cases.

The granularity level of sub-tasks is defined by the structure of underlying processed primitives. 
The target devise for execution is automatically assigned for the sub-task based on expression and node parameters. 
Currently, the fixed and uniform distribution for assignment is supported.

\begin{figure}[t]
\includegraphics[width=0.99\linewidth]{figures/library_architecture.png}
\caption{Library expression processing architecture.}
\label{fig:architecture}
\end{figure}
    
\subsection{Containers}

Library provides general \textit{M-by-N Matrix}, \textit{N Vector} and \textit{Scalar} primitives.
Underlying primitives types are specified by \textit{Type} object. 
Primitives are stored in hybrid storage in a form of two- or one- dimensional blocks' grid form matrices and vectors respectively. 
Each block is empty (not stored) or stores some data in any format. Blocks are immutable, they can be safely shared across computational devices.

Currently, COO/CSR blocks for matrices and COO/Dense blocks for vectors a are supported. Format choice is motivated by its simplicity and ease of implementation. 
Other formats, such as CSC, DCSR, ELL, etc. can be added to the library by the implementation of formats conversion or by the specialization of \textit{Algorithm} for a specific format.

\subsection{Algebraic Operations}

Library supports all commonly used linear algebra operations, such as \textit{mxm}, \textit{vxm}, \textit{eadd}, \textit{reduce}, \textit{transpose}. 
Other operations are coming soon since the library is still in development.
Interface of operations is designed \textit{similar} to GraphBLAS. 
It supports \textit{masking}, \textit{accum} of the result, \textit{add} and/or \textit{mult} user-functions specification, and \textit{descriptor} object for additional operation tweaking.

\subsection{Implementation Details}

Library uses OpenCL 1.2 API as underlying compute API. 
Boost Compute~\cite{10.1145/2909437.2909454:boost:compute} is utilized as a high-level library on top of the OpenCL functionality. 
It provides thread-safe kernel caching, meta-kernel programming, and a set of basic parallel primitives such as \textit{device vector}, \textit{sort}, \textit{reduce}, \textit{scan}, etc. which was extended further to meet this project requirements.
Taskflow~\cite{Huang2022TaskflowAL} is used as a tasking library. It supports task-dependencies and dynamic tasking, utilized in order to create and execute sub-tasks. 

User-defined \textit{Types} are represented as POD-structures and handled by the library as fixed-size sequences of bytes.
User-defined \textit{Functions} are effectively textual strings with OpenCL code, injected into generalized meta-kernels.
Library has a number of predefined types, such as \textit{signed/unsigned integers}, \textit{floating point} types, and a set of common operations, such as \textit{arithmetic}, \textit{logic}, \textit{first/second}, etc.

For particular SpVSpM implementation ESC algorithm~\cite{10.1145/2699470:esc:algo} is employed. 
Masked SpGEMM is based on Yang et al. work ~\cite{yang2019graphblast} solution. 
Tiled GPU merge path~\cite{inproceedings:gpu_merge_path} algorithm utilized for element-wise addition and masking implementation.
The code is generalized and written in a form of meta-kernels, so actual functions for elements reduction or multiplication are injected later.
Kernel compilation is done on demand if no previously cached entry is present.
\section{Evaluation}

This section describes the methodology and answers the following research questions.

\begin{enumerate}
    \item Does fusion via distillation give any benefits at the software and hardware levels?
    \item What are the properties of the generated hardware?
    \item Does the generated hardware outperform software implementations?
\end{enumerate}

\subsection{Methodology}

Our focus is on creating a basis for future research and experiments, thus we make our experiments as much reproducible as possible\footnote{\url{https://github.com/sedwards-lab/fhw/tree/sparse-linear-algebra-distillation/examples/QTreeBenchmarks/diploma/verilog-bool-no-nnz-inlined} (online; accessed:
2022-06-07) Here one could find all the results. For each benchmark all statistics are specified: matrix names, their sizes, collected metrics for both hardware and software benchmarks.}. We benchmarked the following list of chained functions. The choice is prompted by the current state of the distiller: at the moment, it does not successfully distill matrix multiplication. However, the functions are still practical enough, for example, chained addition could be seen in Luby's maximal independent set algorithm and clearly describe the applicability of the proposed approach.

\begin{itemize}
    \item \mintinline{Haskell}{mAdd (==False) (||) (mAdd (==False) (||) m1 m2) m3}
    \item \mintinline{Haskell}{mask (mAdd (== False) (||) m2 m3) (m1)}
    \item \mintinline{Haskell}{map (==Zero) (to_nat) (mAdd (==False) (||) m1 m2}
    \item \mintinline{Haskell}{map (==Zero) (to_nat) (kron (==False) (&&) m1 m2}
\end{itemize}

Above, \mintinline{Haskell}{Zero} and \texttt{to\_nat} are corresponding definitions for Peano arithmetics, since the \texttt{.pot} language does not have any primitives. For the same reason, we operated with boolean matrices. Such functions could be abstracted with free variables and then instantiated in the emitted Haskell code. However, to get maximum from distillation, we provided all the information about the functions. 

For these functions, we compared the execution time of distilled and not distilled hardware generated circuits, execution time of original and distilled Haskell code and reference \textit{Suite Sparse}\footnote{\url{https://github.com/DrTimothyAldenDavis/GraphBLAS} (online; accessed:
2022-06-07), Suite Sparse library sources.}\textsuperscript{,}\footnote{The library also uses different variations of coordinate formats (opaque to the user) and not a quadtree representation.} variants of these functions in C\texttt{++}. Note that SuiteSparse does not support recursive data types, thus only the first two function chains were implemented in SuiteSparse (since Peano number is essentially a linked list). We did not replace Peano numbers with integers, so our experiments could be interpreted easier. For hardware experiments we collected execution time and the number of memory writes and reads, to access how well fusion is performed. For software experiments we only measured the execution time. Also note that we measured only the time, required to execute the lines above, not including any IO, required to get and evaluate function arguments. But in hardware benchmarks we measured the time required to pass arguments into the circuit's memory, because such IO is inevitable. It is tricky to make such measures in Haskell due to laziness, thus the programs were compiled with \texttt{--fno-full-laziness} to turn off memoization. Also all the arguments were forced to normal form via \texttt{force} and \texttt{evaluate}. Haskell programs were compiled\footnote{GHC 8.10.4.} with \texttt{-O2 --fno-full-laziness} and Suite Sparse was compiled with default flags and linked as a shared library to C\texttt{++} code.

We took matrices from SuiteSparse matrix collection with sizes ranging from \texttt{64x64} to \texttt{512x512}. We limited ourselves with such sizes due to the fact that this is the maximum sizes that fit into \texttt{bram} with $2^{16}$ address space. Such number of \texttt{bram} blocks is available only on really expensive FPGA boards, thus in practice sizes would be smaller to achieve better utilization. Once again, it models the situation when data fits into the cache, since \texttt{bram} in our circuits will operate as a cache in real application.

\subsection{Experiments}

Table~\ref{tab:bench_results} shows the results of all execution time benchmarks. To evaluate execution time for hardware simulation, implementation stage was performed to assess the maximum frequency of FPGA device used for synthesis and implementation, and the number of execution cycles was multiplied by the number of nanoseconds a clock cycle takes. The frequencies were equal within the same benchamark set, i.e., frequency was not affected by distillation. We used \texttt{xcu250figd2104-2L} device\footnote{\url{https://www.xilinx.com/products/boards-and-kits/alveo/u250.html}  (online; accessed:
2022-06-07)} for synthesis and implementation stages. It is not really a casual and affordable chip, but it contains enough \texttt{bram} for our evaluation to see scalability. In the table a median across several benchmarks is shown. 

As it could be seen, distillation steadily increases performance: up to 2x speedup for hardware simulation and up to 3x for software benchmarks. The results are maintained within the borders of the corresponding confidence interval and the borders are not shown for brevity. Hardware speedup is lower due to the different execution semantics, dataflow is not reduction-based and distillation is a reduction-based transformation. Note that generated hardware appears to be less performant than both Haskell and C\texttt{++}, which a bit contradicts the results from~\cite{oldfhw}. For hardware benchmarks \texttt{time (IO)} shows the execution time including the time needed to transfer the data though the arguments, \texttt{time (no IO)} does not include it in its turn. It could be seen that not all the benchmarks are computationally extensive enough to cover memory transferring costs, but for more complex examples the ratio would be better. Since we basically transfer the matrices node by node from a file in the testbench, we have probably the lowest possible latency, and in practice it would be higher if reading from DDR, however the bandwidth could be increased. Noticeably, running times for \texttt{mMaskAdd} for C\texttt{++} and distilled Haskell are similar, which shows that fusion really provides some extra performance: SuiteSparse at the moment does not implement any fusion.

Table~\ref{tab:mem_results} summarizes the ratios between distilled and not distilled hardware circuits memory reads and writes. Since in general case distillation removes extra pattern matching, essentially it saves memory reads and writes. The eventual number of memory reads and writes is implementation dependent, thus the table shows what share of speedup is prompted by saving memory operations. Distillation successfully reduces the number of memory accesses, about 15\% on average. \texttt{mMapKron} has a bit higher ratio due to the fact that \texttt{Nat} numbers require additional memory accesses, since the type is recursive. It could also be seen that a major part of speedups is attributed to saved memory accesses. 

Finally, table~\ref{tab:resource_util} shows device resources utilization ratios between distilled and not distilled hardware circuits and frequencies. Columns are device primitives: registers, lookup tables, \texttt{bram} blocks or multiplexers. Utilization for both types of circuits is below 1\% of available resources on the device, except for the memory. Memory blocks utilization is about 30\% (since we choose larger \texttt{brams} to store larger matrices). Apart from that, distilled circuits could have both higher and lower utilization. Since the hardware generation is primarily syntax-directed it follows from the distilled program structure. For example, distillation might glue two recursive functions into one (in that case, memory utilization would be lower, because each cluster of mutually recursive functions possesses its own heap) or make more recursive functions than in the original program. The frequencies are the same, however, they possibly could be made better with more intelligent buffer allocation.

\subsection{Discussion}
Answering the research questions above.

\begin{enumerate}
    \item Fusion gives significant benefits, however at the hardware level the benefits are a bit smaller since hardware semantics is not reduction based. The benefits at the hardware level are mostly determined by the reduced number of memory accesses (each access takes 2 clock cycles). Notably, distilled Haskell implementation of \texttt{mMaskAdd} has similar performance with C\texttt{++}. 
    \item Device utilization is low, but such circuits could be copied on the same device to provide better utilization and higher parallelism. Resource utilization might be both better and worse after distillation, depending on the transformed program itself since translation is syntax-directed. Frequency could be increased by more intelligent buffering strategy.
    \item Although we use low-latency design with \texttt{bram}s that take 2 clock cycles per request and transfer matrices from files, which does not have any latency in simulation, we get slower execution time than Haskell and C\texttt{++} counterparts. It could be partly due to excessive buffering performed by FHW at the moment. Also there is no pipelining for recursive calls, i.e. only one set
of function argument tokens are allowed to enter a tail-recursive function call until a result is finally generated. Further CPS transformation hinders parallelization, which could be made more explicit with SSA. Some other optimizations exist that may significantly influence the performance. Also, since device utilization is about 1\%, such circuits could be copied on one device and provide more parallelism. A more detailed discussion could be found at~\cite{Edwards2019FHWP}.
\end{enumerate}

Distillation clearly showed its applicability to optimization of sparse linear algebra routines and notably it still could be combined with other techniques, like rewrite rules to achieve better results. High-level synthesis has a room for improvements by increasing pipelining, parallelism and frequency and the generated hardware could be improved from usability perspective: a support for arbitrary sized matrices is desirable. Thus we will focus on these directions. Probably a better solution would be to embed \texttt{.pot} language into e.g. Haskell to leverage its type system (to be able to use some rewrite rules as well), and add support for primitive types and parallel primitives to be able to conduct a more scalable comparison with SuiteSparse (since SuiteSparse is multithreaded). For such embedding different execution models could be implemented, including hardware synthesis, for which SSA form of GRIN looks promising, as well as extra optimizations shipped with GRIN. For hardware synthesis, an interesting direction is achieving predictable results in hardware from certain modifications in software. This property partly holds for the current approach, since the translation is syntax- directed. More information on this could be found at~\cite{predict}.

\pagebreak

\begin{table}[t]
\scriptsize
\centering
\caption*{mAddAdd}
\begin{tabular}{|c|c|c|c|c|c|c|c|c|c|} 
\hline
\rowcolor{LightBlue}
\multicolumn{3}{|c|}{Matrices dimensions} & Haskell & Haskell (distilled) & \multicolumn{2}{c|}{FHW} & \multicolumn{2}{c|}{FHW (distilled)} & {C\texttt{++}}\\
% \rowcolor{LightBlue}
\hline
m1 & m2 & m3 & time & time & time (no IO) & time (IO) & time (no IO) & time (IO) & time \\ 
\hline
64 & 64 & 64 & 29 us & 20 us & 76 us & 170 us & 64 us & 158 us & 14 us\\ 
128 & 128 & 128 & 94 & 79 & 146 & 476 & 134 & 469 & 30 \\
256 & 256 & 256 & 123 & 103 & 202 &  681 & 168 & 662 & 44\\
512 & 512 & 512 & 219 & 143 & 474 & 1192 & 375 & 1093 & 49\\
\hline
\end{tabular}

\caption*{mMaskAdd}
\begin{tabular}{|c|c|c|c|c|c|c|c|c|c|} 
\hline
\rowcolor{LightBlue}
\multicolumn{3}{|c|}{Matrices dimensions} & Haskell & Haskell (distilled) & \multicolumn{2}{c|}{FHW} & \multicolumn{2}{c|}{FHW (distilled)} & {C\texttt{++}}\\
% \rowcolor{LightBlue}
\hline
m1 & m2 & m3 & time & time & time (no IO) & time (IO) & time (no IO) & time (IO) & time \\ 
\hline
64 & 64 & 64 & 10 us & 7 us & 64 us & 133 us & 46 us & 111 us & 18 us\\ 
128 & 128 & 128 & 38 & 30 & 118 & 322 & 75 & 292 & 33 \\
256 & 256 & 256 & 48 & 42 & 168 &  498 & 104 & 456 & 46\\
512 & 512 & 512 & 126 & 76 & 400 & 762 & 300 & 729 & 65\\
\hline
\end{tabular}

\caption*{mMapAdd}
\begin{tabular}{|c|c|c|c|c|c|c|c|c|c|} 
\hline
\rowcolor{LightBlue}
\multicolumn{3}{|c|}{Matrices dimensions} & Haskell & Haskell (distilled) & \multicolumn{2}{c|}{FHW} & \multicolumn{2}{c|}{FHW (distilled)} & {C\texttt{++}}\\
% \rowcolor{LightBlue}
\hline
m1 & m2 & m3 & time & time & time (no IO) & time (IO) & time (no IO) & time (IO) & time \\ 
\hline
64 & 64 & --- & 45 us & 37 us & 189 us & 253 us & 137 us & 202 us & ---\\ 
128 & 128 & --- & 162 & 105 & 524 & 685 & 397 & 579 & --- \\
256 & 256 & --- & 312 & 216 & 1047 &  1360 & 680 & 986 & ---\\
512 & 512 & --- & 436 & 273 & 1346 & 1776 & 900 & 1330 & ---\\
\hline
\end{tabular}

\caption*{mMapKron}
\begin{tabular}{|c|c|c|c|c|c|c|c|c|c|} 
\hline
\rowcolor{LightBlue}
\multicolumn{3}{|c|}{Matrices dimensions} & Haskell & Haskell (distilled) & \multicolumn{2}{c|}{FHW} & \multicolumn{2}{c|}{FHW (distilled)} & {C\texttt{++}}\\
% \rowcolor{LightBlue}
\hline
m1 & m2 & m3 & time & time & time (no IO) & time (IO) & time (no IO) & time (IO) & time \\ 
\hline
2 & 64 & --- & 64 us & 36 us & 212 us & 242 us & 94 us & 125 us & ---\\ 
2 & 128 & --- & 137 & 68 & 434 & 502 & 199 & 266 & --- \\
2 & 256 & --- & 364 & 126 & 1004 &  1188 & 449 & 636 & ---\\
4 & 128 & --- & 302 & 94 & 694 & 763 & 330 & 401 & ---\\
\hline
\end{tabular}



\caption{Execution time}
\label{tab:bench_results}

\end{table}
\begin{table}[h]
\scriptsize
\begin{minipage}{0.45\linewidth}
\centering
\caption*{mAddAdd}
\begin{tabular}{|c|c|c|c|c|c|c|} 
\hline
\rowcolor{LightBlue}
\multicolumn{3}{|c|}{Matrices dimensions} & \multicolumn{2}{c|}{Ratio ($\frac{FHW}{FHW_{distilled}}$)}\\
% \rowcolor{LightBlue}
\hline
m1 & m2 & m3 & writes & reads\\ 
\hline
64 & 64 & 64 & 1.10 & 1.15\\ 
128 & 128 & 128 & 1.02 & 1.05\\
256 & 256 & 256 & 1.03 & 1.06\\
512 & 512 & 512 & 1.10 & 1.16\\
\hline
\end{tabular}
\end{minipage}
\begin{minipage}{0.45\linewidth}
\centering
\caption*{mMaskAdd}
\begin{tabular}{|c|c|c|c|c|c|c|} 
\hline
\rowcolor{LightBlue}
\multicolumn{3}{|c|}{Matrices dimensions} & \multicolumn{2}{c|}{Ratio ($\frac{FHW}{FHW_{distilled}}$)}\\
% \rowcolor{LightBlue}
\hline
m1 & m2 & m3 & writes & reads\\ 
\hline
64 & 64 & 64 & 1.13 & 1.26\\ 
128 & 128 & 128 & 1.06 & 1.11\\
256 & 256 & 256 & 1.08 & 1.09\\
512 & 512 & 512 & 1.10 & 1.16\\
\hline
\end{tabular}
\end{minipage}
\begin{minipage}{0.45\linewidth}
\centering
\caption*{mMapAdd}
\begin{tabular}{|c|c|c|c|c|c|c|} 
\hline
\rowcolor{LightBlue}
\multicolumn{3}{|c|}{Matrices dimensions} & \multicolumn{2}{c|}{Ratio ($\frac{FHW}{FHW_{distilled}}$)}\\
% \rowcolor{LightBlue}
\hline
m1 & m2 & m3 & writes & reads\\ 
\hline
64 & 64 & --- & 1.10 & 1.21\\ 
128 & 128 & --- & 1.07 & 1.14\\
256 & 256 & --- & 1.07 & 1.19\\
512 & 512 & --- & 1.10 & 1.21\\
\hline
\end{tabular}
\end{minipage}
\hfill
\begin{minipage}{0.45\linewidth}
\centering
\caption*{mMapKron}
\begin{tabular}{|c|c|c|c|c|c|c|} 
\hline
\rowcolor{LightBlue}
\multicolumn{3}{|c|}{Matrices dimensions} & \multicolumn{2}{c|}{Ratio ($\frac{FHW}{FHW_{distilled}}$)}\\
% \rowcolor{LightBlue}
\hline
m1 & m2 & m3 & writes & reads\\ 
\hline
2 & 64 & --- & 1.71 & 1.88\\ 
2 & 128 & --- & 1.72 & 1.87\\
2 & 256 & --- & 1.65 & 1.83\\
4 & 128 & --- & 1.81 & 1.91\\
\hline
\end{tabular}
\end{minipage}

\caption{Memory accesses}
\label{tab:mem_results}
\end{table}

\begin{table}[h]
\scriptsize
\centering
\begin{tabular}{|l|c|c|c|c|c|c|c|c|c|} 
\hline
\rowcolor{LightBlue}

{Benchmark} & \multicolumn{8}{c|}{Ratio (${\frac{FHW}{FHW_{distilled}}}$)} & {Frequency}\\
\hline
{} & FDRE & LUT3 & LUT6 & LUT5 & LUT4 & LUT2 & RAMB36E2 & MUXF7 & {} \\
% \rowcolor{LightBlue}
\hline
mAddAdd & 0.3 & 0.3 & 0.3 & 0.5 & 0.3 & 0.3 & 0.5 & 0.5 & 200 MHz\\ 
mMaskAdd & 0.5 & 0.5 & 0.7 & 0.4 & 0.7 & 0.5 & 0.7 & 0.6 & 200 MHz\\
mMapAdd & 1 & 0.9 & 0.9 & 1.2 & 1 & 1.1 & 1.1 & 1.2 & 200 MHz\\
mMapKron & 1.5 & 1.5 & 1.3 & 2 & 2 & 1.8 & 1.4 & 1.7 & 200 MHz\\
\hline
\end{tabular}
\caption{Resource utilization}
\label{tab:resource_util}
\end{table}
\pagebreak

\section{Conclusion and Future Work}

We present !!!

Our evaluation shows that !!!

First direction for future research is a more detailed CFPQ algorithms investigation.
We should do More evaluation on sparse matrices on GPGPUs.

Also it is nesessary to implement and evaluate solutions for graphs which is not fit in RAM.
There is a set of technics for huge matrices multiplication.
Is it possible to dopt it for CFPQ

Another direcion is a dataset improvement.
More data.
More grammars/queries.


%%
%% The acknowledgments section is defined using the "acks" environment
%% (and NOT an unnumbered section). This ensures the proper
%% identification of the section in the article metadata, and the
%% consistent spelling of the heading.
\begin{acks}
The research was supported by the Russian Science Foundation,
grant №18-11-00100.

We thank Adrian Johnstone for his pointing out the Generilized LL algorithm in our discussion at Parsing@SLE--2013 which gave us the motivation to develop the presented solution.

%We thank  George Fletcher for his discussion of evaluation of different CFPQ algorithms for Neo4j.

%We thank Tobias Lindaaker, 

\end{acks}

%%
%% The next two lines define the bibliography style to be used, and
%% the bibliography file.
\bibliographystyle{ACM-Reference-Format}
\bibliography{gll4graph}

%%
%% If your work has an appendix, this is the place to put it.
%\appendix

%\section{Research Methods}

\end{document}
\endinput
%%
%% End of file `sample-sigconf.tex'.
