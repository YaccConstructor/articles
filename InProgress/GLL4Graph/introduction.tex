\section{Introduction}

Context-free path querying (CFPQ) allows one to use context-free grammars to specify constraints on paths in edge-labeled graphs. Context-free constraints are more expressive than the regular ones (RPQ) and can be used for graph analysis in such areas as bioinformatics (hierarchy analysis~\cite{10.1007/978-3-319-46523-4_38}, similarity queries~\cite{SubgraphQueriesbyContextfreeGrammars}), data provenance analysis~\cite{8731467}, static code analysis~\cite{Zheng, 10.1145/373243.360208}. Although a lot of algorithms for CFPQ were proposed, poor performance on real-world graphs and bad integration with real-world graph databases and graph analysis systems still are problems which hinder the adoption of CFPQ.

The problem with the performance of CFPQ algorithms in real-world scenarios was pointed out by Jochem Kuijpers~\cite{Kuijpers:2019:ESC:3335783.3335791} as a result of an attempt to extend Neo4j graph database with CFPQ. Several algorithms, based on such techniques as LR parsing algorithm~\cite{10.1007/978-3-319-91662-0_17}, dynamic programming~\cite{hellingsRelational}, LL parsing algorithm~\cite{10.1145/3167132.3167265}, linear algebra~\cite{Azimov:2018:CPQ:3210259.3210264}, were implemented using Neo4j as a graph storage and evaluated. None of them was performant enough to be used in real-world applications.   

Since Jochem Kuijpers pointed out the performance problem, it was shown that linear algebra based CFPQ algorithms, which operate over the adjacency matrix of the input graph and utilize parallel algorithms for linear algebraic operations, demonstrate good performance~\cite{10.1145/3398682.3399163}. Moreover, the matrix-based CFPQ algorithm is a base for the first full-stack support of CFPQ by extending RedisGrpah graph database~\cite{10.1145/3398682.3399163}.  

However adjacency matrix is not the only possible format for graph representation, and data format conversion may take a lot of time, thus it is not applicable in some cases. As a result, the development of a performant CFPQ algorithm for graph representations not based on matrices and its integration with real-world graph databases is still an open problem. Moreover, while the \textit{all pairs context-free constrained reachability} is widely discussed in the community, such practical cases as the \textit{all paths} queries and the \textit{multiple sources} queries are not studied enough. Additionally, to the best of our knowledge, ways to provide parallel solutions based not on linear-algebra-oriented algorithms are still not investigated. In the multi- and many-core world and the big data era, it is important to provide a parallel solution for CFPQ.

It was shown that the generalized LL (GLL) parsing algorithm can be naturally generalized to CFPQ algorithm~\cite{Grigorev:2017:CPQ:3166094.3166104}. Moreover, this provides a natural solution not only for the \textit{reachability} problem but also for the \textit{all paths} problem. At the same time, there exists a high-performance GLL parsing algorithm~\cite{10.1007/978-3-662-46663-6_5} and its implementation in Iguana project\footnote{Iguana parsing framework: \url{https://iguana-parser.github.io/}. Accessed: 12.11.2021.}. In this paper, we generalize the parsing algorithm implemented in Iguana to get a high-performance CFPQ algorithm and integrate it with the Neo4j graph database. So, we make the following contributions in the paper.
\begin{itemize}
    \item We provide an implementation of the GLL-based CFPQ algorithm. Implementation is based on the high-performance GLL parsing algorithm, which we modified to handle graphs. We provide modifications for both the \textit{reachability} and the \textit{all paths} cases. 
    \item We integrate the implemented algorithm with Neo4j using Native Java API. Currently, we use Neo4j as a graph storage and do not provide the query language support. Implementing a query language extension amounts to a lot of additional effort and is a part of future work.
    \item We evaluate the proposed solution on several real-world graphs for both the \textit{all pairs} and the \textit{multiple sources} scenarios. Our evaluation shows that the proposed solution is more than 25 times faster than the previous solution for Neo4j and is comparable, in some cases, with the linear algebra based solution for RedisGraph. Moreover, performance of the \textit{all paths} problem for \textit{multiple sources} CFPQ is reasonable for relatively small start sets.
\end{itemize}