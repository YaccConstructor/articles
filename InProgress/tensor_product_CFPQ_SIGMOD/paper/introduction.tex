\section{Introduction}


Language-constrained path querying~\cite{barrett2000formal} is one of techniques for graph navigation querying.
This technique allows one to use formal languages as constraints on paths in edge-labeled graphs: path satisfies constraints if labels along it form a word from the specified language.

The utilization of regular languages as constraints, or \textit{Regular Path Querying} (RPQ), is most well-studied and widely spread.
Different aspects of RPQs are actively studied in graph databases~\cite{10.1145/2463664.2465216, 10.1145/3104031,10.1145/2850413}, and support of regular constraints is implemented in such popular query languages as PGQL~\cite{10.1145/2960414.2960421}, or SPARQL\footnote{Specification of regular constraints in SPARQL propert paths: \url{https://www.w3.org/TR/sparql11-property-paths/}. Access date: 07.07.2020.}~\cite{10.1007/978-3-319-25007-6_1} (property paths).
Even that, improvement of RPQ algorithms efficiency on huge graphs is an actual problem nowadays, and new algorithms and solutions are being created~\cite{Wang2019,10.1145/2949689.2949711}.

At the same time, utilization of more powerful languages, namely context-free languages, gain popularity in the last few years. 
\textit{Context-Free Path Querying} problem (CFPQ) was introduced by Mihalis Yannakakis in 1990 in~\cite{Yannakakis}.
A number of different algoritms was proposed since that time, 
Context-free is more specific, but actively developing last years.


To make it usable... Integration with graph DB.
But recently, in~\cite{Kuijpers:2019:ESC:3335783.3335791} Jochem Kuijpers et al. show that state-of-the-art CFPQ algorithms are not performant enough to be used in practice.
This fact motivates to finde new algorithms for CFPQ.

Linear algebra, GraphBLAS~\cite{7761646}, !!!! is a right way.

CFPQ as a separated algorithms. 
Matrix is the fastest. 

Recently, an algortithm was proposed. 
In this work we improve it, blah-blah-blah

Integration with query languages. 
The problem. We cannot separate regular and context-free queries in general case.  

Moreover, grammar transformation for matrix-based (the fastest existing algorithm) is required, !!!

Subcubic CFPQ. Long-time open problem. 
The best known result for general case is an $O(n^3/\log{n})$ algorithm of Swarat Chaudhuri~\cite{10.1145/1328438.1328460}.
Also it is shown by Chattergee that !!!
For 1-Dyck language : Phillip Bradford~\cite{8249039}. Cannot be generalized to arbitary CFPQs. 
We find a way

We make the following contributions in this paper.
\begin{enumerate}
	\item We rethink and improve tensor-product-based algorithm for CFPQ. First of all, we reduce this algorithm to operations over Boolean matrices. All paths semantics. Previous matrix-based solution only single path. For both regualr and context-free path queries. Correctness and time complexity.
	\item We demonstrate interconnection between CFPQ and incremental transitive closure. Conjecture on sublinear incremental transitive closure and subcubic CFPQ. We show that incremental transitive closure is a bottleneck on the way to get subcubic CFPQ algorithm.
	\item By using existing results we show how to get slightly subcubic algorithm for general case, and subcubic combinatorial algorithm for partial cases. This cretarion is output-sensitive, so it is not practical, but open a theoretical way to find more subclass with subcubic complexity.
	\item We implement the described algorithm and evaluate it on real-world data. RPQ, CFPQ. Results show that !!!
\end{enumerate}