\section{Related Work}

Language constrained path querying is widely used in graph databases, static code analysis, and other areas.
Both, RPQ and CFPQ (known as CFL reachability problem in static code analysis) are actively studied in the recent years.

There is a huge number of theoretical research on RPQ and its specific cases.
RPQ with single-path semantics was investigated from the theoretical point of view by~\cite{barrett2000formal}.
In order to research practical limits and restrictions of RPQ, a number of high-performance RPQ algorithms were provided.
For example, the derivative-based solution provided by~\cite{10.1145/2949689.2949711}, which is implemented on top of the Pregel-based system, or the solution by~\cite{10.1007/978-3-642-31235-9_12}.
But only a limited number of practical solutions provide the ability to restore paths of interest.
A recent work by~\cite{Wang2019} provides a Pregel-based provenance-aware RPQ algorithm which utilizes a Glushkov's construction~\citep{Glushkov1961}.
There is a lack of research of the applicability of linear algebra-based RPQ algorithms with paths-providing semantics.

On the other hand, many CFPQ algorithms with various properties were proposed recently.
They employ the ideas of different parsing algorithms, such as CYK in works by~\cite{hellingsRelational} and~\cite{8249039}, (G)LR and (G)LL in works by~\cite{10.1007/978-3-319-41579-6_22},~\cite{Grigorev:2017:CPQ:3166094.3166104},~\cite{10.1007/978-3-319-91662-0_17},~\cite{Medeiros:2018:EEC:3167132.3167265}.
Unfortunately, none of them has better than cubic time complexity in terms of the input graph size.
The algorithm by~\cite{Azimov:2018:CPQ:3210259.3210264} is, best to our knowledge, the first algorithm for CFPQ which is based on linear algebra.
It was shown by~\cite{10.1145/3398682.3399163} that this algorithm can be applied to real-world graph analysis problems, while~\cite{Kuijpers:2019:ESC:3335783.3335791} show that other state-of-the-art CFPQ algorithms are not performant enough to handle real-world graphs.

It is important in both RPQ and CFPQ to be able to restore paths of interest.
Some of the mentioned algorithms can solve only the reachability problem, while it may be important to provide at least one path which satisfies the query.
While~\cite{10.1145/3398682.3399163} provide the first CFPQ algorithm with single path semantics based on linear algebra,~\cite{HellSinglePath} provides the first theoretical investigation of this problem.
He also provides an overview of the related works and shows that the problem is related to the string generation problem and respective results from the formal language theory.
He concludes that both theoretical and empirical investigation of CFPQ with single-path and all-path semantics are at the early stage.
We agree with this point of view, and we only demonstrate the applicability of our solution to paths extraction and do not investigate its properties in details.

While CFPQ on $n$-node graph has a relatively straightforward $O(n^3)$ time algorithm, it is a long-standing open problem whether there exists a truly  subcubic $O(n^{3-\varepsilon})$ algorithm for this problem.
The question on the existence of a subcubic CFPQ algorithm was stated by~\cite{Yannakakis}.
A bit later~\cite{10.5555/271338.271343} proposed the CFL reachability as a framework for interprocedural static code analysis.
\cite{10.1145/258994.259006} gave a dynamic programming formulation of the problem which runs in $O(n^3)$ time.
The problem of the cubic bottleneck of context-free language reachability is also discussed by~\cite{10.5555/788019.788876} and~\cite{10.1145/258994.259006}.
The slightly subcubic algorithm with $O(n^3/\log{n})$ time complexity was provided by~\cite{10.1145/1328438.1328460}.
This result is inspired by recursive state machine reachability.
The first truly subcubic algorithm with $O(n^\omega polylog(n))$ time complexity ($\omega$ is the best exponent for matrix multiplication) for an arbitrary graph and 1-Dyck language was provided by~\cite{8249039}, and~\cite{pavlogiannis2020finegrained}.
Other partial cases were investigated by~\cite{10.1145/3158118},~\cite{zhang2020conditional}.

Employing linear algebra is a promising way to high-performance graph analysis.
There are many works which formulate specific graph algorithms in terms of linear algebra, for example, such algorithms as for computing transitive closure and all-pairs shortest paths.
Recently this direction was summarized in GrpahBLAS API~\citep{7761646} which provides building blocks to develop a graph analysis algorithm in terms of linear algebra.
There is a number of implementations of this API, such as SuiteSparse:GraphBLAS~\citep{10.1145/3322125} or CombBLAS~\citep{10.1177/1094342011403516}.
Approaches to evaluate different classes of queries in different systems based on linear algebra are being actively researched.
This approach demonstrates significant performance improvement when applied for SPARQL queries evaluation~\citep{10.1145/3302424.3303962,DBLP:journals/corr/MetzlerM15a} and for Datalog queries evaluation~\citep{sato_2017}.
Finally, RedisGraph~\citep{8778293}, a linear-algebra powered graph database, was created and it was shown that in some scenarios it outperforms many other graph databases.