\subsection{Paths Extraction Algorithm}
After the index has been created, one can enumerate all paths between specified vertices.
The index stores information about all reachable pairs for all nonterminals.
Thus, the most natural way to use this index is to query paths between the specified vertices derivable from the specified nonterminal.
\begin{algorithm}[h]
\floatname{algorithm}{Listing}
\begin{algorithmic}[1]
\footnotesize
\caption{Paths extraction algorithm}
\label{tensor:pathsExtraction}
\State{$C_3 \gets $ result of index creation algorithm: final transitive closure}
\State{$\mathcal{M}_1 \gets $  the set of adjacency matrices of the input RSM}
\State{$\mathcal{M}_2 \gets $ the set of adjacency matrices of the final graph}
\State{$R \gets$ Recursive automata for the input RSM}

\Function{getPaths}{$v_s, v_f, N$}
\State{$q_N^0 \gets$ Start state of automata for $N$}
\State{$F_N \gets$ Final states of automata for $N$}
\State{$paths\gets \bigcup\limits_{f \in F_N} \Call{genPaths}{(q_N^0,v_s),(f,v_f)}$}
\State{$resultPaths \gets \emptyset$}
\For{$path \in paths$}
	\State{$currentPaths \gets \emptyset$}
	\For{$((s_i, v_i), (s_j, v_j)) \in path$}
	    \State{\begin{minipage}[t]{0.2\textwidth}
	    		\vspace{-13pt}
	    		\begin{align*}
	    			currentSubPaths \gets & \{(v_i,t,v_j) \mid M_2^t[s_i, s_j] \wedge M_1^t[v_i, v_j]\}\\
	    			& \cup \ \bigcup_{\{N \mid M_2^N[s_i, s_j]\}}\Call{getPaths}{v_i, v_j, N}
	    		\end{align*}
	    	\end{minipage}
	    }
		\State{$currentPaths \gets currentPaths \cdot currentSubPaths$}
		\Comment{Concatenation of paths}
	\EndFor
	\State{$resultPaths \gets resultPaths \cup currentPaths$}
\EndFor

\State \Return $resultPaths$
\EndFunction

% note: the first index in the pair is the state of the RSM
% note: the second index in the pair is the vertex of the graph

\Function{genPaths}{$(s_i,v_i), (s_j,v_j)$}
    \State{$q \gets \text{vector of zeros with size } dim(\mathcal{M}_2)$}
    \Comment{Vector for indicating the current vertex}
    \State{$q[s_i |R| + v_i] \gets 1$}
    \State{$resultPaths \gets \emptyset$}
    \State{$supposedPaths \gets \{([~], q)\}$}
    \Comment{Set of pairs: path and the vector for current vertex}
    \For{$i \in 1..dim(\mathcal{M}_2)$}
     	\For{$(path, q) \in supposedPaths$}
     		\If{$q[s_j |R| + v_j] = 1$}
     			\State{$resultPaths \gets resultPaths \cup path$}
     		\EndIf
     		
     		\State{$q \gets q \cdot (\mathcal{M}_2)^T$}
     		\Comment{Boolean vector-matrix multiplication}
     		
     		\State{$\text{Remove } (path, q)\text{ from } supposedPaths$}
     		\For{$j \text{ such that } q[j] = 1$}
     			\State{$pathNew \gets path$}
     			\If{$path \text{ is empty path}$}
     				\State{$pathNew \gets pathNew \cdot [((s_i, v_i), (j / |R|, j \bmod |R|))]$}
     				\Comment{Edge adding}
     			\Else
     				\State{$(s_k, v_k) \gets \text{ the last vertex of } path$}
     				\State{$pathNew \gets pathNew \cdot [((s_k, v_k), (j / |R|, j \bmod |R|))]$}
     				\Comment{Edge adding}
     			\EndIf
     			\State{$qNew \gets \text{vector of zeros with size } dim(\mathcal{M}_2)$}
     			\State{$qNew[j] \gets 1$}
     			\State{$\text{Add } (pathNew, qNew)\text{ to } supposedPaths$}
     		\EndFor
     	\EndFor   
    \EndFor
    \State \Return $resultPaths$
\EndFunction
\end{algorithmic}
\end{algorithm}

To do so, we provide a function \textsc{getPaths}($v_s, v_f, N$), where $v_s$ is a start vertex of the graph, $v_f$ --- the final vertex, and $N$ is a nonterminal.
Implementation of this function is presented in Listing~\ref{tensor:pathsExtraction}.

Paths extraction is implemented as three mutually recursive functions.
The entry point is \textsc{getPaths}($v_s, v_f, N$).
This function returns a set of the paths between $v_s$ and $v_f$ such that the word formed by a path is derivable from the nonterminal $N$.

To compute such paths, it is necessary to compute paths from vertices of the form $(q_N^s,v_s)$ to vertices of the form $(q_N^f, v_f)$ in the result of transitive closure, where $q_N^s$ is an initial state of RSM for $N$ and $q_N^f$ is a final state.
The function \textsc{getPathsInner}$((s_i,v_i),(s_j,v_j))$ is used to do it.
This function finds all possible vertices $(s_k,v_k)$  which split a path from $(s_i,v_i)$ to $(s_j,v_j)$ into two subpaths.
After that, function \textsc{getSubpaths}$((s_i,v_i),(s_j,v_j),(s_k,v_k))$ computes the corresponding subpaths.
Each subpath may be at least a single edge.
If single-edge subpath is labeled by terminal then corresponding edge should be added to the result else (label is nonterminal) \textsc{getPaths} should be used to restore paths.
If subpath is longer then one edge, \textsc{getPaths} should be used to restore paths. 

It is assumed that the sets are computed lazily, so as to ensure termination in the case of an infinite number of paths.
We also do not check paths for duplication manually, since they are assumed to be represented as sets.