\subsection{The algorithm}

\begin{lemma}
    Let $\mathcal{G} = (V,E,L)$ be a graph and $G = (\Sigma, N, P)$ be a grammar.
    Let $\mathcal{G}_k = (V,E_k,L \cup N)$ be graph and $M_k$ its adjacency
    matrix of the execution some iteration $k \geq 0$ of the algorithm \ref{?}.
    Then for each edge $e = (m,S,n) \in E_k$, where $S \in N$,
    the following statement holds: $\exists m\pi n: S \to_{G} l(\pi)$.
\end{lemma}

\begin{proof}{(Proof by induction)}

    \textbf{Basis:} For $k = 0$ and the statement of the lemma holds, since
    $M_0 = M$, $M$ where is adjacency matrix of the graph $G$ , $L \cap N$
    is empty. Non-terminals, which allow to derive $\varepsilon$-word, are also added,
    since each vertex of the graph is reachable by itself through an
    $\varepsilon$-transition.
    
    \textbf{Inductive step:} Assume that the statement of the lemma hold for any
    $k \geq (p - 1)$ and show that it also holds for $k = p$, where $p \geq 1$.
    
    For the algorithm iteration $p$ the Kronecker product $K_p$ and transitive
    closure $C_p$ are evaluated as described in the algorithm. By the properties
    of this operations, some edge $e = ((s,x),(f,y))$ exists in the oriented
    graph, represented by adjacency matrix $C_p$, if and only if $\exists s
    \pi_1 f$ in the RSM graph, represented by matrix $M_r$, and $\exists x \pi y$
    in graph, represented by $M_{p-1}$. Concatenated symbols along the path
    $\pi_1$ form some derivation string, composed from terminals and
    non-terminals, included in the graph by inductive assumption. 
    
    Therefore, if $s$ an $f$ are initial and final states of some box $B$ of the
    RSM, new edge between vertices $x$ and $y$ with the respective non-terminal
    $S_B$ will be added to the matrix $M_p$ and this completes the proof of the
    lemma.
    
\end{proof}

\begin{lemma}
    Let $\mathcal{G} = (V,E,L)$ be a graph and  $G = (\Sigma, N, P)$ be a grammar. 
    Let $\mathcal{G}_k = (V,E_k,L \cup N)$ be graph and $M_k$ its adjacency
    matrix of the execution some iteration $k \geq 1$ of the algorithm \ref{?}. 
    For any path $m \pi n$ in graph $\mathcal{G}$ with word
    $l = l(\pi): $ $S \to_G l$, if $h \leq k$, where $h$ is a derivation 
    tree height, then $\exists e = (m,S,n): e \in E_k$.
\end{lemma}

\begin{proof}{(Proof by induction)}

    \textbf{Basis:} Show that statement of the lemma holds for the $k = 1$. Matrix
    $M$ and edges of the graph $\mathcal{G}$ contains only labels from $L$. 
    Since the derivation tree of height $h = 1$ contains only one non-terminal $S$ as 
    a top node and only symbols from $\Sigma$ as leafs, for all paths, which form a 
    word with derivation tree of the height $h = 1$, the corresponding top nonterminals will
    be added to the $M_1$ via algorithm first iteration. Non-terminals, which allow 
    to derive $\varepsilon$-word, are also added via algorithm preprocessing step. 
    Thus, the lemma statement holds for the $k = 1$.

    \textbf{Inductive step:} Assume that the statement of the lemma hold for any
    $k \geq (p - 1)$ and show that it also holds for $k = p$, where $p \geq 2$.
    
    For the algorithm iteration $p$ the Kronecker product $K_p$ and transitive
    closure $C_p$ are evaluated as described in the algorithm. By the properties
    of this operations, some edge $e = ((s,x),(f,y))$ exists in the oriented
    graph, represented by adjacency matrix $C_p$, if and only if $\exists s
    \pi_1 f$ in the RSM graph, represented by matrix $M_r$, and $\exists x \pi y$
    in graph, represented by $M_{p-1}$. 
    
    Suppose, that exists derivation tree $T$ of height $h = p$ with the top non-terminal
    $S$ for the path $x \pi y$. The tree $T$ is formed as
    $S \to a_1 .. a_d, d \geq 1$ where $\forall i \in [1..d]$ $a_i$ is sub-tree of
    height $h_i \leq p - 1$ for the sub-path $x_i \pi_i y_i$. By inductive hypothesis,
    there exists path $\pi_i$ for each derivation sub-tree, that  
    $x = x_1 \pi_1 x_2 .. x_{d} \pi_{d} x_{d+1} = y$ and concatenation of these paths
    forms $x \pi y$, and the top non-terminals of this sub-trees are included in the
    matrix $M_{p - 1}$. 
    
    Therefore, vertices $ x_i ~\forall i \in [1..d]$ form path in the graph, 
    represented by matrix $M_{p-1}$, with complete set of labels.
    Thus, new edge between vertices $x$ and $y$ with the respective 
    non-terminal $S$ will be added to the matrix $M_p$ and this completes 
    the proof of the lemma.

\end{proof}

\begin{theorem}
    Let $\mathcal{G} = (V,E,L)$ be a graph and  $G = (\Sigma, N, P)$ be a grammar.
    Let $\mathcal{G}_R = (V, E_R, L)$  be a result graph for the execution 
    of the algorithm \ref{??}. Then $e = (m, S, n) \in E_R$, where $S \in N$ 
    if and only if $\exists m \pi n: S \to_G l(\pi)$. 
\end{theorem}{}

\begin{proof}
    Todo.
\end{proof}{}

\begin{theorem}{}
    Let $\mathcal{G} = (V,E,L)$ be a graph and $G = (\Sigma, N, P)$ be a grammar.
    The algorithm \ref{?} terminates in finite number of steps.
\end{theorem}

\begin{proof}
    Todo.
\end{proof}{}