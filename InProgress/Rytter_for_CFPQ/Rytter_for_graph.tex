\section {Rytter Algorithm for Graph Input} 

Main idea is to adopt algorithm from~\cite{Rytter} for CFPQ.
It should be possible to perform adaptation for arbitrary CFPQ, but we are interested in case of Dyck queries because it should simplify complexity estimation.

We introduce an example and try to explain key steps.
As far as example for graph and query introduced in the previous section is too big, we use another input data.

Let the input grammar is 
\begin{align*}
S & \rightarrow a \ S \ b \ 
\\
S & \rightarrow a \ b
\end{align*}

The input grammar in CNF is 
\begin{align*}
S   & \rightarrow A \ S_1
\\
S_1 & \rightarrow S \ B
\\
S   & \rightarrow A \ B
\\
A   & \rightarrow a
\\
B   & \rightarrow b
\end{align*}


Let the input graph is:
\\
\begin{tikzpicture}[shorten >=1pt,node distance=2cm,on grid,auto] 
   \node[state] (q_1)   {$1$}; 
   \node[state] (q_2) [above right=of q_1] {$2$}; 
   \node[state] (q_3) [right=of q_2] {$3$}; 
   \node[state] (q_4) [right=of q_3] {$4$};
    \path[->] 
    (q_1) edge  node {a} (q_2)          
    (q_2) edge  node {a} (q_3)
    (q_3) edge  node {a} (q_1)
    (q_3) edge[bend left, above]  node {b} (q_4)
    (q_4) edge[bend left, below]  node {b} (q_3);
\end{tikzpicture}

  
We use the same notation and the semiring as proposed by Rytter in~\cite{Rytter}.
The \emph{IMPLIED} relation for our example is presented in figure~\ref{implied}.
Furher we will write $(N_1, N_2)$ instead of $(N_1,i,j) \Rightarrow (N_2,k,l)$ when positions specification are not important in the context.

\begin{centering}
\begin{figure*}
\begin{align*} % S -> A B
(B,2,3)  & \Rightarrow (S,1,3)  & (B,2,4)  & \Rightarrow (S,1,4)  & (B,2,2)  & \Rightarrow (S,1,2) & (B,2,1)  & \Rightarrow (S,1,1)
\\
(B,3,4)  & \Rightarrow (S,2,4)  & (B,3,3)  & \Rightarrow (S,2,3)  & (B,3,2)  & \Rightarrow (S,2,2) & (B,3,1)  & \Rightarrow (S,2,1)
\\
(B,1,2)  & \Rightarrow (S,3,2)  & (B,1,3)  & \Rightarrow (S,3,3)  & (B,1,4)  & \Rightarrow (S,3,4) & (B,1,1)  & \Rightarrow (S,3,1)
\\ %S -> A S1
(S_1,2,3)  & \Rightarrow (S,1,3) & (S_1,2,4)  & \Rightarrow (S,1,4) & (S_1,2,2)  & \Rightarrow (S,1,2) & (S_1,2,1)  & \Rightarrow (S,1,1)
\\
(S_1,3,4)  & \Rightarrow (S,2,4)  & (S_1,3,3)  & \Rightarrow (S,2,3)  & (S_1,3,2)  & \Rightarrow (S,2,2) & (S_1,3,1)  & \Rightarrow (S,2,1)
\\
(S_1,1,2)  & \Rightarrow (S,3,2)  &  (S_1,1,3) & \Rightarrow (S,3,3)  & (S_1,1,4)  & \Rightarrow (S,3,4)  & (S_1,1,1)  & \Rightarrow (S,3,1)
\\ % S -> A B
(A,2,3)  & \Rightarrow (S,2,4)    &  (A,1,3) & \Rightarrow (S,1,4)    & (A,3,3)  & \Rightarrow (S,3,4)    & (A,4,3)  & \Rightarrow (S,4,4)
\\
(A,3,4)  & \Rightarrow (S,3,3)    &  (A,4,4) & \Rightarrow (S,4,3)    & (A,2,4)  & \Rightarrow (S,2,3)    & (A,1,4)  & \Rightarrow (S,1,3)
\\ % S1 -> S B
(S,2,3)  & \Rightarrow (S_1,2,4)  &  (S,1,3) & \Rightarrow (S_1,1,4)  & (S,3,3)  & \Rightarrow (S_1,3,4)  & (S,4,3)  & \Rightarrow (S_1,4,4)
\\
(S,3,4)  & \Rightarrow (S_1,3,3)  &  (S,4,4) & \Rightarrow (S_1,4,3)  & (S,2,4)  & \Rightarrow (S_1,2,3)  & (S,1,4)  & \Rightarrow (S_1,1,3)
\end{align*}

\caption{\emph{IMPLIED} relation for our example}
\label{implied}
\end{figure*}
\end{centering}

Initial grid graph is presented in fig~\ref{initialGrid}.
It can be constructed by the similar way as presented in~\cite{Rytter} and can by stored in two $n \times \ n$ matrix where $n$ is a number of vertices in input graph.

\begin{figure*}  

\begin{tikzpicture}[shorten >=1pt,node distance=4.5cm,on grid,auto] 
   \node[state] (q_1)   {$1$}; 
   \node[state] (q_2) [right=of q_1] {\tiny{$2$}}; 
   \node[state] (q_3) [right=of q_2] {\tiny{$3$ (B)}}; 
   \node[state] (q_4) [right=of q_3] {\tiny{$4$}};

   \node[state] (q_5) [below=of q_1] {\tiny{$5$ (A)}};
   \node[state] (q_6) [right=of q_5] {\tiny{$6$}};
   \node[state] (q_7) [right=of q_6] {\tiny{$7$}};
   \node[state] (q_8) [right=of q_7] {\tiny{$8$ (B)}};
  
   \node[state] (q_9) [below=of q_5] {\tiny{$9$}};
   \node[state] (q_10) [right=of q_9] {\tiny{$10$}};
   \node[state] (q_11) [right=of q_10] {\tiny{$11$ (A)}};
   \node[state] (q_12) [right=of q_11] {\tiny{$12$}};
   
   \node[state] (q_13) [below=of q_9] {\tiny{$13$}};
   \node[state] (q_14) [right=of q_13] {\tiny{$14$ (A)}};
   \node[state] (q_15) [right=of q_14] {\tiny{$15$}};
   \node[state] (q_16) [right=of q_15] {\tiny{$16$}};

    \path[->] 
     
     (q_11) edge  node {\tiny{$\{(B,S);(S_1,S)\}$}} (q_15)
     (q_12) edge  node {\tiny{$\{(B,S);(S_1,S)\}$}} (q_16)
     (q_10) edge  node {\tiny{$\{(B,S);(S_1,S)\}$}} (q_14)
     (q_9) edge  node {\tiny{$\{(B,S);(S_1,S)\}$}} (q_13)

     (q_8) edge  node {\tiny{$\{(B,S);(S_1,S)\}$}} (q_12)
     (q_7) edge  node {\tiny{$\{(B,S);(S_1,S)\}$}} (q_11)
     (q_6) edge  node {\tiny{$\{(B,S);(S_1,S)\}$}} (q_10)
     (q_5) edge  node {\tiny{$\{(B,S);(S_1,S)\}$}} (q_9)

     (q_14) edge[bend left]  node {\tiny{$\{(B,S);(S_1,S)\}$}} (q_6)
     (q_13) edge[bend left]  node {\tiny{$\{(B,S);(S_1,S)\}$}} (q_5)
     (q_15) edge[bend left]  node {\tiny{$\{(B,S);(S_1,S)\}$}} (q_7)
     (q_16) edge[bend left]  node {\tiny{$\{(B,S);(S_1,S)\}$}} (q_8)

     (q_11) edge[bend left]  node {\tiny{$\{(A,S);(S,S_1)\}$}} (q_12)
     (q_15) edge[bend left]  node {\tiny{$\{(A,S);(S,S_1)\}$}} (q_16)
     (q_7) edge[bend left]  node {\tiny{$\{(A,S);(S,S_1)\}$}} (q_8)
     (q_3) edge[bend left]  node {\tiny{$\{(A,S);(S,S_1)\}$}} (q_4)

     (q_8) edge[bend left]  node {\tiny{$\{(A,S);(S,S_1)\}$}} (q_7)
     (q_4) edge[bend left]  node {\tiny{$\{(A,S);(S,S_1)\}$}} (q_3)
     (q_12) edge[bend left]  node {\tiny{$\{(A,S);(S,S_1)\}$}} (q_11)
     (q_16) edge[bend left]  node {\tiny{$\{(A,S);(S,S_1)\}$}} (q_15)

     
     ;
\end{tikzpicture}
\caption{Initial grid graph}
\label{initialGrid}
\end{figure*}  

We should introduce the identity set \emph{id} such that:
\begin{itemize}
\item $\text{\emph{id}} \times A = A \times \text{\emph{id}} = A$
\item $\text{\emph{id}} \times \text{\emph{id}} = \text{\emph{id}}$
\end{itemize}

This set may be constructed as follows: $\text{\emph{id}} = \{ (N_i, N_i)  | N_i \in N\}$.

In order to compute transitive closure in logarithmic time we add self-loop with weight $\text{\emph{id}}$ to each vertex.
Result is graph $\mathcal{G}$ which is presented in fig.~\ref{selfLoops}.

\begin{figure*}  
\begin{tikzpicture}[shorten >=1pt,node distance=4.5cm,on grid,auto] 
   \node[state] (q_1)   {$1$}; 
   \node[state] (q_2) [right=of q_1] {\tiny{$2$}}; 
   \node[state] (q_3) [right=of q_2] {\tiny{$3$ (B)}}; 
   \node[state] (q_4) [right=of q_3] {\tiny{$4$}};

   \node[state] (q_5) [below=of q_1] {\tiny{$5$ (A)}};
   \node[state] (q_6) [right=of q_5] {\tiny{$6$}};
   \node[state] (q_7) [right=of q_6] {\tiny{$7$}};
   \node[state] (q_8) [right=of q_7] {\tiny{$8$ (B)}};
  
   \node[state] (q_9) [below=of q_5] {\tiny{$9$}};
   \node[state] (q_10) [right=of q_9] {\tiny{$10$}};
   \node[state] (q_11) [right=of q_10] {\tiny{$11$ (A)}};
   \node[state] (q_12) [right=of q_11] {\tiny{$12$}};
   
   \node[state] (q_13) [below=of q_9] {\tiny{$13$}};
   \node[state] (q_14) [right=of q_13] {\tiny{$14$ (A)}};
   \node[state] (q_15) [right=of q_14] {\tiny{$15$}};
   \node[state] (q_16) [right=of q_15] {\tiny{$16$}};

    \path[->] 
     
     (q_11) edge  node {\tiny{$\{(B,S);(S_1,S)\}$}} (q_15)
     (q_12) edge  node {\tiny{$\{(B,S);(S_1,S)\}$}} (q_16)
     (q_10) edge  node {\tiny{$\{(B,S);(S_1,S)\}$}} (q_14)
     (q_9) edge  node {\tiny{$\{(B,S);(S_1,S)\}$}} (q_13)

     (q_8) edge  node {\tiny{$\{(B,S);(S_1,S)\}$}} (q_12)
     (q_7) edge  node {\tiny{$\{(B,S);(S_1,S)\}$}} (q_11)
     (q_6) edge  node {\tiny{$\{(B,S);(S_1,S)\}$}} (q_10)
     (q_5) edge  node {\tiny{$\{(B,S);(S_1,S)\}$}} (q_9)

     (q_14) edge[bend left]  node {\tiny{$\{(B,S);(S_1,S)\}$}} (q_6)
     (q_13) edge[bend left]  node {\tiny{$\{(B,S);(S_1,S)\}$}} (q_5)
     (q_15) edge[bend left]  node {\tiny{$\{(B,S);(S_1,S)\}$}} (q_7)
     (q_16) edge[bend left]  node {\tiny{$\{(B,S);(S_1,S)\}$}} (q_8)

     (q_11) edge[bend left]  node {\tiny{$\{(A,S);(S,S_1)\}$}} (q_12)
     (q_15) edge[bend left]  node {\tiny{$\{(A,S);(S,S_1)\}$}} (q_16)
     (q_7) edge[bend left]  node {\tiny{$\{(A,S);(S,S_1)\}$}} (q_8)
     (q_3) edge[bend left]  node {\tiny{$\{(A,S);(S,S_1)\}$}} (q_4)

     (q_8) edge[bend left]  node {\tiny{$\{(A,S);(S,S_1)\}$}} (q_7)
     (q_4) edge[bend left]  node {\tiny{$\{(A,S);(S,S_1)\}$}} (q_3)
     (q_12) edge[bend left]  node {\tiny{$\{(A,S);(S,S_1)\}$}} (q_11)
     (q_16) edge[bend left]  node {\tiny{$\{(A,S);(S,S_1)\}$}} (q_15)

     (q_1) edge[loop] node {\tiny{$\text{\emph{id}}$}} (q_1)
     (q_2) edge[loop] node {\tiny{$\text{\emph{id}}$}} (q_2)
     (q_3) edge[loop] node {\tiny{$\text{\emph{id}}$}} (q_3)
     (q_4) edge[loop] node {\tiny{$\text{\emph{id}}$}} (q_4)
     (q_5) edge[loop] node {\tiny{$\text{\emph{id}}$}} (q_5)
     (q_6) edge[loop] node {\tiny{$\text{\emph{id}}$}} (q_6)
     (q_7) edge[loop] node {\tiny{$\text{\emph{id}}$}} (q_7)
     (q_8) edge[loop] node {\tiny{$\text{\emph{id}}$}} (q_8)
     (q_9) edge[loop] node {\tiny{$\text{\emph{id}}$}} (q_9)
     (q_10) edge[loop] node {\tiny{$\text{\emph{id}}$}} (q_10)
     (q_11) edge[loop] node {\tiny{$\text{\emph{id}}$}} (q_11)
     (q_12) edge[loop] node {\tiny{$\text{\emph{id}}$}} (q_12)
     (q_13) edge[loop] node {\tiny{$\text{\emph{id}}$}} (q_13)
     (q_14) edge[loop] node {\tiny{$\text{\emph{id}}$}} (q_14)
     (q_15) edge[loop] node {\tiny{$\text{\emph{id}}$}} (q_15)
     (q_16) edge[loop] node {\tiny{$\text{\emph{id}}$}} (q_16)
     

     
     ;
\end{tikzpicture}
\caption{Grid with self-loops}
\label{selfLoops}

\end{figure*}  

Now we can do some observations.
\begin{itemize}
\item Graph $\mathcal{G}$ is pretty similar to Rytter's grid graph (except cycles which have special structure and satisfy strongly congruence restriction) and can be represented as two matrices of size $n \times n$: $\mathcal{G}_H$ and $\mathcal{G}_V$ for horizontal and vertical edges respectively. We use the same representation as Rytter.
Note that self-loops should be duplicated and stored in both matrices.
\item We can compute transitive closure of $\mathcal{G}_H$ and $\mathcal{G}_V$ in $\widetilde{O}(BMM(n))$ by using standard techniques fro transitive closure calculation.
Let $\mathcal{G}_H'$ is a closure of $\mathcal{G}_H$ and $\mathcal{G}_V'$ is a closure of $\mathcal{G}_V$.
\item Our goal is find valid nonterminals for each vertex in $\mathcal{G}$. 
We can do it iteratively: we can check validity of nonterminals in final vertices of all paths from $\mathcal{G}_H'$ (or $\mathcal{G}_V'$) by multiplication on matrix $X: X[i,j] = \{(N_l,N_l)| N_l \text{ is known to be valid in } \mathcal{G}(i,j)\}$.
Formally we can define next block as one step of iteration.
\begin{itemize}
\item $X = X + X * \mathcal{G}_H'$
\item $X = X + X * \mathcal{G}_V'$
\item Update \emph{IMPLIED} relation and $\mathcal{G}$
\end{itemize}

This iteration process all paths with at most one new ``zig-zag''.

\item We should repeat previous step until all path of required length not processed.
\item As far as our query is a Dyck query we (hope that we) can use the technique from~\cite{bradford2018} for estimation of iteration numbers.
We can not use it ``as is'' but we can see, that structure of paths in $\mathcal{G}$ is related to ``Pyramids and Valleys'' structure from~\cite{bradford2018}. 
\end{itemize}


