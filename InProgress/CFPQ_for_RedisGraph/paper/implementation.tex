\section{Implementation}

We showed that CFPQ can be naturally reduced to linear algebra.
Linear algebra for graph problems is an actively developed area.
One of the most important results is a GraphBLAS API which provides a way to operate over matrices and vectors over user-defined semirings.

Previous works show~\cite{Mishin:2019:ECP:3327964.3328503, Azimov:2018:CPQ:3210259.3210264} that existing linear algebra libraries utilization is the right way to get high-performance CFPQ implementation with minimal effort.
But neither of these works do provide an evaluation with data storage, only pure time of algorithm execution was measured.

We provide a number of implementations of the matrix-based CFPQ algorithm.
All of them a based on RedisGraph~---~we use RedisGraph as storage and implement CFPQ as an extension by using a provided mechanism.
We choose this database because it uses sparse adjacency matrices for graphs representation which is an appropriate representation for the matrix-based algorithm.
Note that currently, we do not provide full integration with querying mechanism: one cannot use Cypher, which uses in RedisGraph as a query language.
Instead, query should be provided explicitly as a file with grammar in Chomsky normal form.
This is enough to evaluate querying algorithms, but in the future we should improve integration to make our solution applicable.

\textbf{CPU-based implementation (RG\_CPU)} uses SuteSparse implementation of GraphBLAS, which is used in RedisGraph, and a predefined boolean semiring.
Thus we avoid data format problems: we use native RedisGraph representation of the adjacency matrix in our algorithm.

\textbf{GPGPU-based implementation} has two versions.
The first one (\textbf{RG\_M4RI}) uses the Method of Four Russians implemented in~\cite{Mishin:2019:ECP:3327964.3328503}, and the second one (\textbf{RG\_CUSP}) utilizes a modified CUSP library for matrix operations.
Both implementations require matrix format conversion.
