\section{Dataset description}

In our evaluation we use combined dataset which contains the following parts.
\begin{itemize}
\item CFPQ\_Data dataset which provided in\footnote{CFPQ\_Data data set GitHub repository: \url{https://github.com/JetBrains-Research/CFPQ_Data}. Access date: 12.11.2019.}~\cite{Mishin:2019:ECP:3327964.3328503} and contains both syntethic and real-world graphs and queryes.
Real-world data includes RDFs, syntatic cases include theoretical worst-case and random graphs.
\item Dataset which provided in~\cite{Kuijpers:2019:ESC:3335783.3335791}. We integrate both Geospecies --- RDF which contains information about bioligical hierrarchy\footnote{\url{https://old.datahub.io/dataset/geospecies}. Access date: 12.11.2019.} and same generation query over \textit{broaderTransitive} relation, and Synthetic --- the set of graph generated by using the Barab\'asi-Albert model of scale-free networks and same generation quey, data into CFPQ\_Data and use it in our evaluation.
\item In~\cite{Mishin:2019:ECP:3327964.3328503} was shown that matrix-based algorithm is performad enough to handle bigger RDFs than used in initial data sets, such as~\cite{RDF}.
So, we add a number of big RDFs to CFPQ\_Data and use them in our evaluation.
New RDFs: \textit{go-hierarchy, go, enzime, core, pathways} --- parts of UniProt database\footnote{Protein sequences data base: \url{https://www.uniprot.org/}. RDFs with data are avalable here: \url{ftp://ftp.uniprot.org/pub/databases/uniprot/current_release/rdf}. Access date: 12.11.2019}, and \textit{eclass-514en} that given from eClassOWL project\footnote{eClassOWL project: \url{http://www.heppnetz.de/projects/eclassowl/}. eclass-514en file is available here: \url{http://www.ebusiness-unibw.org/ontologies/eclass/5.1.4/eclass_514en.owl}. Access date: 12.11.2019.}.
\end{itemize}

The variants of the \textit{same generation query}~\cite{FndDB} is used in almost all cases because it is an important example of real-world queries that are context-free but not regular.
So, variations of the same generation query is used in out evaluaton.
All queryes are added to the CFPQ\_Data data set.

For RDFs qurying we use two queryes over \textit{subClassOf} and \textit{type} relations.
The first query is the grammar $G_1$:
\[
 \begin{array}{lcl}
   s  \rightarrow \textit{subClassOf}^{\ -1} \ s \ \textit{subClassOf}   & \quad & s  \rightarrow \textit{type}^{\ -1} \ s \ \textit{type}     \\
   s  \rightarrow \textit{subClassOf}^{\ -1} \ \textit{subClassOf}       & \quad & s  \rightarrow  \textit{type}^{\ -1}  \ \textit{type}

 \end{array}
 \]
The second one is the grammar $G_2$: $$s \rightarrow \textit{subClassOf}^{\ -1} \ s \ \textit{subClassOf} \mid  \textit{subClassOf}.$$

%\textbf{[RDF]} The set of the real-world RDF files (ontologies) from~\cite{RDF} and two variants of the same generation query which describes hierarchy analysis.
%
%\textbf{[Worst]} The theoretical worst case for CFPQ time complexity proposed by Hellings~\cite{hellingsPathQuerying}: the graph is two cycles of coprime %lengths with a single common vertex.
%The first cycle is labeled by the open bracket and the second cycle is labeled by the close bracket.
%Query is a grammar for the $A^nB^n$ language.
%The example of such graph and grammar is presented in figure:
%
%        \[
%         \begin{array}{l}
%           s \rightarrow A \ s \ B \\
%           s \rightarrow A \ B
%         \end{array}
%         \]
%
%
%\textbf{[Full]} The case when the input graph is sparse, but the result is a full graph.
%Such a case may be hard for sparse matrices representations.
%As an input graph, we use a cycle, all edges of which are labeled by the same token.
%As a query we use two grammars which describe the sequence of tokens of arbitrary length: the simple ambiguous grammar $G_2$: $s \rightarrow  s \ s \ | \ %A$,  and the highly ambiguous grammar $G_3$: $s \rightarrow s \ s \ s \ | \ A$.
%
%\textbf{[Sparse]} Sparse graphs from~\cite{fan2018scaling} are generated by the GTgraph graph generator, and emulate realistic sparse data.
%Names of these graphs have the form \texttt{Gn-p}, where \texttt{n} corresponds to the total number of vertices, and \texttt{p} is the probability that %some pair of vertices is connected.
%The query is the same generation query represented by the grammar $G_1$ (figure~\ref{fig:grammar_example}).
%
