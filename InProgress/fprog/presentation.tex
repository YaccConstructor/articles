\documentclass[xcolor=table]{beamer}
\usepackage{beamerthemesplit}
\usepackage{wrapfig}
\usetheme{SPbGU}
\usepackage{pdfpages}
\usepackage{amsmath}
\usepackage{cmap} 
\usepackage[T2A]{fontenc} 
\usepackage[utf8]{inputenc}
\usepackage[english,russian]{babel}
\usepackage{indentfirst}
\usepackage{tikz}
\usepackage{multirow}
\usepackage[noend]{algpseudocode}
\usepackage{algorithm}
\usepackage{algorithmicx}
\usetikzlibrary{shapes,arrows}
%usepackage{fancyvrb}
%\usepackage{minted}
%\usepackage{verbments}


\newtheorem{rutheorem}{Теорема}
\newtheorem{ruproof}{Доказательство}
\newtheorem{rudefinition}{Определение}
\newtheorem{rulemma}{Лемма}
\beamertemplatenavigationsymbolsempty

\title[Brahma.FSharp]{Brahma.FSharp как средство ``прозрачного'' использования GPGPU в программах на F\#}
%\subtitle[YaccConstructor]{Parsing techniques for graph analysis}
% То, что в квадратных скобках, отображается в левом нижнем углу. 
\institute[СПбГУ]{
JetBrains Research, лаборатория языковых инструментов \\
Санкт-Петербургский государственный университет
}

% То, что в квадратных скобках, отображается в левом нижнем углу.
\author[Семён Григорьев]{Семён Григорьев}

\date{30.11.2017}

\definecolor{orange}{RGB}{179,36,31}

\begin{document}
{
\begin{frame}[fragile]
  \begin{tabular}{p{2.5cm} p{6.5cm} p{2cm}}
   \begin{center}
      \includegraphics[height=1.5cm]{pictures/JBLogo3.pdf}
    \end{center}
    &
    \begin{center}
      \includegraphics[height=1.5cm]{pictures/fprog_logo.png}
    \end{center} 
    &
    \begin{center}
      \includegraphics[height=1.5cm]{pictures/SPbGU_Logo.png}
    \end{center}
  \end{tabular}
  \titlepage
\end{frame}
}


\begin{frame}[fragile]
  \transwipe[direction=90]
  \frametitle{План}
  \begin{itemize}
  \item GPGPU в F\#: но зачем?
    \begin{itemize}
        \item Есть ли будущее у таких решений?
        \item Где их можно применять?
    \end{itemize}
  \item Почему именно так, а не инече?
      \begin{itemize}
        \item Уместен ли такой подход?
        \item Может можно проще?
      \end{itemize}
   \item Что под капотом у Brahma.FSharp?
  \end{itemize}
\end{frame}

\begin{frame}[fragile]
  \transwipe[direction=90]
  \frametitle{Зачем GPGPU?}
  \begin{itemize}
  \item Обработка больших объёмов данных ``регклярным'' способом
    \begin{itemize}
        \item Динамический параллелизм...
        \item Анализ сетей (социальных, интернет и т.д.)
    \end{itemize}
    \end{itemize}
\end{frame}



\begin{frame}[fragile]
  \transwipe[direction=90]
  \frametitle{Зачем GPGPU?}
  \begin{itemize}
  \item Обработка больших объёмов данных ``регклярным'' способом
    \begin{itemize}
        \item Динамический параллелизм...
        \item Анализ сетей (социальных, интернет и т.д.)
    \end{itemize}
  \item Почему именно так, а не инече?
      \begin{itemize}
        \item ``Надёжность''
        \item ``Прозрачность''/гомогенность
      \end{itemize}
   \item Что под капотом у Brahma.FSharp?
  \end{itemize}
\end{frame}


\begin{frame}[fragile]
  \transwipe[direction=90]
  \frametitle{Массовый параллелизм}
  \begin{itemize}
  \item FSCL
  \item Alea GPU
  \end{itemize}
\end{frame}

\begin{frame}[fragile]
  \transwipe[direction=90]
  \frametitle{Доступ к GPGPU из .NET}
  \begin{itemize}
  \item Средства программирования видеопроцессоров в .NET
  \begin{itemize}
     \item Alea GPU (CUDA)
     \item Brahma.FSharp (OpenCL)
     \item FSCL (OpenCL)
  \end{itemize}
  \item Средства запуска CUDA-кода из ЯВУ:
  \begin{itemize}
     \item CUSP
     \item ManagedCuda
  \end{itemize}
  \item ``Низкоуровневые драйвера''
  \begin{itemize}
     \item OpenCL.NET (\url{http://openclnet.codeplex.com/})
     \item CUDA.NET
  \end{itemize}
  \end{itemize}
\end{frame}

\begin{frame}[fragile]
  \transwipe[direction=90]
  \frametitle{Провайдеры типов}
  \begin {itemize}
  \item Функция построения типа по пользовательскому контексту
  \item Преимущества перед кодогенерацией
  \begin {itemize}
   \item Интеграция с пользовательским контекстом
   \item Статическая типизация
   \item Вспомогательная информация доступна в процессе разработки (работает автодополнение и т.д.)
  \end {itemize}

  \item Недостатки
  \begin {itemize}
    \item Высокая сложность тестирования
    \item Высокая сложность отладки
  \end {itemize}
\end {itemize}
\end{frame}

\begin{frame}[fragile]
  \transwipe[direction=90]
  \frametitle{Цитирование кода (Code quotation)}
\end{frame}




\begin{frame}
  \transwipe[direction=90]
  \frametitle{Итоги}

\begin{itemize} 
\item Есть ли будущее у такого подхода?
\begin{itemize} 
  \item Какие альтернативы?
  \item Нужна ли гомогенность?
  \item ...
\end{itemize}
\item Какие потенциальные области применения?
\item Не слишком ли сложный механизм для рядового пользователя?
\end{itemize}

\end{frame}

            
\begin{frame}
\transwipe[direction=90]
\frametitle{Контакты}
\begin{itemize}
  \item Почта: \url{semen.grigorev@jetbrains.com}
  \item Проект на GitHub: \url{https://github.com/YaccConstructor/Brahma.FSharp}
\end{itemize}
\end{frame}
\end{document}
