\section{Conclusion}

In this paper we propose a number of multiple-source modifications of Azimov's CFPQ algorithm.
Evaluation of the proposed modifications on the real-world examples shows that queries results caching is not useful in evaluated scenarios and the na\"{i}ve implementation is a best choice for integration with rel-world graph database.  
Finally, we provide the full-stack support of CFPQ.
For our solution we implement corresponding Cypher extension as a part of \texttt{libcypher-parser}, integrate the proposed algorithm into RedisGraph, and extend RedisGraph execution plan builder to support extended Cypher queries.
We demonstrate, that our solution is applicable for real-world graph analysis.

In the future, it is necessary to provide formal translation of Cypher to linear algebra, or find a maximal subset of Cypher which can be translated to linear algebra.
There is a number of work on a subset of SPARQL to linear algebra translation, such as~\cite{10.14778/3229863.3236239,10.1007/978-3-642-34002-4_36,10.1145/3302424.3303962,DBLP:journals/corr/MetzlerM15a}.
But most of them practical-oriented and do not provide full theoretical basis to translate querying language to linear algebra.
Other of them are discuss only partial cases and should be extended to cover real-world query languages. 
Deep investigation of this topic helps one to realize limits and restrictions of linear algebra utilization for graph databases.
Moreover, it helps to improve existing solutions.

We show that evaluation of regular queries is possible in practice by using CFPQ algorithm, as far as regular queries is a partial case of the context-free one.
But it seems, that the proposed solution is not optimal. 
For real-world solutions it is important to provide an optimal unified algorithm for both RPQ and CFPQ.
One of possible way to solve this problem is to use tensor-based algorithm~\cite{10.1007/978-3-030-54832-2_6}.

Another important task is to compare non-linear-algebra-based approaches to multiple-source CFPQ with the proposed solution. 
In~\cite{Kuijpers:2019:ESC:3335783.3335791} Jochem Kuijpers et al. show that all-pairs CFPQ algorithms implemented in Neo4j demonstrate unreasonable performance on real-world data.
At the same time, Arseniy Terekhov et.al. shows that matrix-based all-pairs CFPQ algorithm implemented in appropriate linear algebra based graph database (RedisGraph) demonstrates good performance.
But in the case of multiple-source scenario, when a number of start vertices is relatively small, non-linear-algebra-based solutions can be better, because such solutions naturally handle small required subgraph.
Thus detailed investigation and comparison of other approaches to evaluate multiple-source CFPQ is required in the future.