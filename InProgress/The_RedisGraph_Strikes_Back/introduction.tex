\section{Introduction}

CFPQ is a way to use grammars.

CFPQ is widely spread and gain popularity last years.

Application for real-world data analysis is a problem.
First of all, the performance problems. Jochem Kuijpers~\cite{Kuijpers:2019:ESC:3335783.3335791}.
There are no full-stack solutions, only separated algorithms, 
For example, recently graph segmentation in data provenance analysis was reduced to CFPQ~\cite{!!!}, but authors faced the problem during the evaluation of the proposed approach: no one graph database support CFPQ.
Thus, it is necessary to provide !!!

In~\cite{Azimov:2018:CPQ:3210259.3210264} Rustam Azimov propose a matrix-based algorithm for CFPQ.
This algoritm is one of promissing way to provide appropriate solution for real-world daata analysis: it was shown that !!!
But the proposed algorithm

All-pairs is a classical problem.
What about single-source and multiple-source?
We propose a matrix-based multiple-source CFPQ algorithm.

Also, we provide full-stack support of CFPQ.
We implement a Cypher query language extension\footnote{!!!} that allows one to express context-free constraints, and extend the RedisGraph graph database to support this extension.
In our knowledge, it is the first full-stack implementation of CFPQ.

The following contribution.
\begin{enumerate}
	\item Multiple-source matrix-based CFPQ algorithm. 
	Single-source as a partial case.
	\item Evaluateion of two versions of this algorithm.
	\item RedisGraph extending to provide full-stack support of CFPQ.
\end{enumerate}