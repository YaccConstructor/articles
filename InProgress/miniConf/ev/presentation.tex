\documentclass{beamer}
\usepackage{beamerthemesplit}
\usepackage{wrapfig}
\usetheme{SPbGU}
\usepackage{pdfpages}
\usepackage{amsmath}
\usepackage{cmap} 
\usepackage[T2A]{fontenc} 
\usepackage[utf8]{inputenc}
\usepackage[english,russian]{babel}
\usepackage{indentfirst}
\usepackage{amsmath}
\usepackage{tikz}
\usepackage{multirow}
\usepackage[noend]{algpseudocode}
\usepackage{algorithm}
\usepackage{algorithmicx}
\usepackage{listings}
\usepackage{pifont}% http://ctan.org/pkg/pifont
\usepackage{xcolor}
\usepackage{mdframed}
\usepackage{multicol}
\newcommand{\cmark}{\ding{51}}%
\newcommand{\xmark}{\ding{55}}%

\usepackage{epstopdf}
\usepackage{forest}
\usetikzlibrary{shapes,arrows}
\usepackage{fancyvrb}
\newtheorem{rutheorem}{Theorem}
\beamertemplatenavigationsymbolsempty
\setbeamertemplate{itemize items}[circle]

\lstdefinelanguage{ocaml}{
%keywords={fresh, let, begin, end, in, match, type, and, fun, function, try, with, when, class, 
%object, method, of, rec, %repeat, 
%until, while, not, do, done, as, val, inherit, 
%new, module, sig, deriving, datatype, struct, if, then, else, open, private, virtual, ostap},
sensitive=true,
basicstyle=\small,
commentstyle=\small\itshape\ttfamily,
keywordstyle=\ttfamily\underbar,
identifierstyle=\ttfamily,
basewidth={0.5em,0.5em},
columns=fixed,
fontadjust=true,
literate={->}{{$\;\;\to\;\;$}}1
         {===}{{$\equiv$}}1
         {&&&}{{$\wedge$}}1
         {|||}{{$\vee$}}1
         {fresh}{{$\exists$}}1,
morecomment=[s]{(*}{*)}
}

\lstset{
basicstyle=\small,
identifierstyle=\ttfamily,
keywordstyle=\bfseries,
commentstyle=\scriptsize\rmfamily,
basewidth={0.5em,0.5em},
fontadjust=true,
%escapechar=~,
language=ocaml,
mathescape=true,
moredelim=[is][\bfseries]{[*}{*]}
}

\definecolor{SadRed}{RGB}{255, 180, 180}
\definecolor{MehYellow}{RGB}{255, 255, 180}
\definecolor{HappyGreen}{RGB}{180, 255, 180}

\newmdenv[backgroundcolor=SadRed, innertopmargin=8,innerbottommargin=1, linecolor=SadRed]{badcode}

\newmdenv[backgroundcolor=MehYellow, innertopmargin=8,innerbottommargin=1, linecolor=MehYellow]{mehcode}

\newmdenv[backgroundcolor=HappyGreen, innertopmargin=8,innerbottommargin=1, linecolor=HappyGreen]{goodcode}

\title[]{Суперкомпиляция для miniKanren}
\institute[]{
Лаборатория языковых инструментов JetBrains
\\
Санкт-Петербургский государственный университет }

\author[Екатерина Вербицкая]{Екатерина Вербицкая}

\date{15 декабря 2018}

\definecolor{orange}{RGB}{179,36,31}

\begin{document}
{

\begin{frame}
  \begin{columns} 
    \begin{column}{4.1cm}
      \begin{center} 
        {\includegraphics[width=1.5cm]{pics/jb.png}} 
      \end{center}
    \end{column}
    \begin{column}{4.1cm}
      \begin{center} 
        {\includegraphics[width=1.5cm]{pics/SPbGU_Logo.png}} 
      \end{center}
    \end{column}
  \end{columns}

  \titlepage
\end{frame}
}

\begin{frame}[fragile]
  \transwipe[direction=90]
  \frametitle{Реляционное программирование}
  Программа --- отношение 
  
\begin{center}
$append^{o} \subseteq [A] \times [A] \times [A] $ 

$$
\begin{array}{lrcl}
append^{o} & = & \{ & ([\,], [\,], [\,]); \\
           &   &    & ([0], [\,], [0]); \\ 
           &   &    & ([1], [\,], [1]); \\
           &   &    & \dots \\
           &   &    & ([\,], [0], [0]); \\
           &   &    & \dots \\
           &   &    & ([4], [2], [4, 2]); \\
           &   &    & \dots \\
           &   &    & ([4, 2], [13], [4, 2, 13]); \\
           &   &    & \dots \\
           &   & \}
\end{array}
$$ 
\end{center}
\end{frame}

\begin{frame}[fragile]
  \transwipe[direction=90]
  \frametitle{Пример программы на miniKanren}
\begin{center}
\begin{minipage}{4.5cm}
\begin{lstlisting}[frame=single]  
appendo x y z = 
  (x === [] &&& z === y)
  ||| (fresh h t r
       ( x === h : t 
       &&& z === h : r 
       &&& appendo t y r))
\end{lstlisting}
\end{minipage}
\end{center}
\end{frame}


\begin{frame}[fragile]
  \transwipe[direction=90]
  \frametitle{Вычисление в реляционном программировании}
  
$$ 
\begin{array}{lccccccl}
append^o & [1] & [2, 3] & q         &\rightarrow& \{ &  [1, 2, 3] & \} \\
append^o & [1] & q      & [1, 2, 3] &\rightarrow& \{ & [2, 3] & \} \\
append^o & q   & [1]    & [2, 3]    &\rightarrow& \{ & & \}  \\
\end{array}
$$

\end{frame}

\begin{frame}[fragile]
  \transwipe[direction=90]
  \frametitle{Вычисление в реляционном программировании}

$$ 
\begin{array}{rccl}
append^o \, q \, p \, [1, 2] &\rightarrow& \{ &                \\
                    &           &    & ([\,], [1,2]), \\
                    &           &    & ([1], [2]), \\
                    &           &    & ([1,2], [\,]) \\
                    &           & \} \\
                    &           & \\ 
append^o \, q \, q \, [2, 4, 2, 4] &\rightarrow& \{ &  [2, 4] \quad \} \\
                    &           & \\ 
append^o \, q \, p \, r      &\rightarrow& \{ &  \\
                    &           &    &([\,], \ \alpha, \ \alpha), \\
                    &           &    & ([\alpha], \  \beta, \ (\alpha:\beta)) \\
                    &           &    & ((\alpha : \beta),\  \gamma, \ (\alpha:(\beta:\gamma))) \\
                    &           &    & \dots \\
                    &           &    \}  \\
\end{array}
$$

\end{frame}


\begin{frame}[fragile]
  \transwipe[direction=90]
  \frametitle{Направление вычислений}

\begin{center}
  $foo^o \subseteq A \times B$
\end{center}

\begin{itemize}
  \item $foo^o \ \alpha \ q : A \rightarrow [B]$
  \item $foo^o \ q \ \beta  : B \rightarrow [A]$ --- в ``обратном'' направлении
  \item $foo^o \ q \ p      : () \rightarrow [(A \times B)] $ 
\end{itemize}  
  
\pause  

\bigskip

Время вычисления в разных направлениях часто отличается

\pause 

$$
factorize \ num = mult^o \ [p, q] \ num 
$$
\end{frame}

\begin{frame}[fragile]
  \transwipe[direction=90]
  \frametitle{Цель}

Улучшать производительность реляционных программ, не уменьшая декларативности подхода

\begin{itemize}
  \item Для заданного направления
  \item Учитывая частично определенные входные данные
\end{itemize}
\end{frame}

\begin{frame}[fragile]
  \transwipe[direction=90]
  \frametitle{Оптимизации: известные данные}
\begin{columns}[t]
\begin{column}{7cm}
\begin{badcode}
\begin{lstlisting}[frame=single]      
foo p q &&& repeat x p
\end{lstlisting}

\begin{lstlisting}[frame=single]  
foo $\subseteq$ [A] $\times$ [B]
foo x y = 
  (x === [] &&& [*heavy*] y)
  ||| (fresh h t (x === h : t &&& light y))
      
repeat $\subseteq$ A $\times$ [A]
repeat x xs = 
  fresh r (xs === x : r &&& repeat x r)
\end{lstlisting}
\end{badcode}
\end{column}

\pause

\begin{column}{4cm}
\begin{goodcode}
\begin{lstlisting}[frame=single]  
foo_r q
\end{lstlisting}

\begin{lstlisting}[frame=single]  
foo_r $\subseteq$ [A] $\times$ [B]
foo_r y = light y
\end{lstlisting}
\end{goodcode}
\end{column}
\end{columns}
\end{frame}

\begin{frame}[fragile]
  \transwipe[direction=90]
  \frametitle{Оптимизации: промежуточные структуры данных}

\begin{columns}[t]
  \begin{column}{5.3cm}
\begin{badcode}
\begin{lstlisting}[frame=single]      
map f p [*q*] &&& map g [*q*] r
\end{lstlisting}

\begin{lstlisting}[frame=single]  
map f $\subseteq$ [A] $\times$ [B]
map f x y = 
 (x === [] &&& y === [])
 ||| (fresh h t r ( x === h : t 
            &&& y === f h : r
            &&& map f t r))
\end{lstlisting}
\end{badcode}
\end{column}

\pause

\begin{column}{6.1cm}
\begin{goodcode}
\begin{lstlisting}[frame=single]      
map_fg f g p r
\end{lstlisting}

\begin{lstlisting}[frame=single]  
map_fg f g $\subseteq$ [A] $\times$ [B]
map_fg f g x y = 
 (x === [] &&& y === [])
 ||| (fresh h t r ( x === h : t 
            &&& y === [*g (f h)*] : r
            &&& map_fg f g t r))
\end{lstlisting}
\end{goodcode}
\end{column}
\end{columns}
\end{frame}

\begin{frame}[fragile]
  \transwipe[direction=90]
  \frametitle{Оптимизации: группировка вычислений}
\vspace{-15pt}

\begin{columns}[t]
\begin{column}{5.8cm}
\begin{badcode}  
\begin{lstlisting}[frame=single]      
sum [*p*] s &&& len [*p*] l
\end{lstlisting}

\begin{lstlisting}[frame=single]  
sum $\subseteq$ [A] $\times$ Int
sum x s = (x === [] &&& s === 0)
          ||| (fresh h t r
              ( x === h : t 
              &&& sum t r
              &&& s = r + h))
 
len $\subseteq$ [A] $\times$ Int
len x l = (x === [] &&& l === 0)
          ||| (fresh h t m 
              ( x === h : t 
              &&& len t m
              &&& l = m + 1 ))
\end{lstlisting}
\end{badcode}
\end{column}

\pause

\begin{column}{5.55cm}
\begin{goodcode}
\begin{lstlisting}[frame=single]      
sum_len p s l
\end{lstlisting}

\begin{lstlisting}[frame=single]  
sum_len $\subseteq$ [A] $\times$ Int $\times$ Int
sum_len x [*s l*] = 
  (x === [] &&& s === 0 &&& l === 0)
  ||| (fresh h t r m
      ( x === h : t 
      &&& sum_len t r m
      &&& [*s = r + h*]
      &&& [*l = m + 1*]))
\end{lstlisting}
\end{goodcode}
\end{column}
\end{columns}
\end{frame}


\begin{frame}[fragile]
  \transwipe[direction=90]
  \frametitle{За счет чего можно улучшать производительность}
\begin{itemize}
  \item Статически вычислить все, что можно: специализация
  \begin{itemize}
    \item Если известны некоторые аргументы
    \item Если известно направление вычисления
  \end{itemize}
  \item Не делать одну и ту же работу дважды
  \begin{itemize}
    \item Избегать промежуточные значения: deforestation
    \item Группировать вычисления, выполняемые во время обхода одной структуры данных: tupling
  \end{itemize}
\end{itemize}

\bigskip

Все это может делать конъюнктивная частичная дедукция (суперкомпиляция) для логических языков 
\end{frame}

\begin{frame}[fragile]
  \transwipe[direction=90]
  \frametitle{Суперкомпиляция для miniKanren: дерево процессов}
  \begin{columns}
    \begin{column}{3.1cm}
\begin{minipage}[t]{3.1cm}
\begin{lstlisting}[frame=single]  
foo $\subseteq$ [A] $\times$ [B]
foo x y =
 (x === [] &&& bar y)
 ||| (fresh h t 
     ( x === h : t 
     &&& baz h y))
\end{lstlisting}
\end{minipage}
    \end{column}
    \begin{column}{7cm}
\vspace{-30pt}
   
    \begin{flushright}
\begin{forest}
  [$<>$ foo x y[{$<x \rightarrow [\,]>$ bar y}[\dots]][$<x \rightarrow (h:t)>$ baz h y[\dots]]]]
\end{forest}
\end{flushright}
    \end{column}
\end{columns}

\bigskip 

\begin{center}
В какой момент завершать символические вычисления? 
\end{center}
\end{frame}

\begin{frame}[fragile]
  \transwipe[direction=90]
  \frametitle{Дерево процессов: применение отношения}
\begin{center}
\begin{forest}
  [\dots[$<s>$ foo x y[\dots]]]
\end{forest}
\end{center}

\begin{itemize}
  \item Если цель (с точностью до подстановки) встречалась раньше, перестаем исследовать эту ветвь
  \item Если цель \emph{похожа} на какого-то ее предка, попробовать ее \emph{абстрагировать} и продолжить строить дерево
  \item Если цель ни на что не похожа, продолжаем строить дерево для применения отношения
\end{itemize}
\end{frame}

\begin{frame}[fragile]
  \transwipe[direction=90]
  \frametitle{Дерево процессов: конъюнкция}
\begin{columns}[t]
\begin{column}{6cm}
\begin{center}
\begin{forest}
  [\dots[$<s>$ g $\wedge$ (t $\equiv$ u) $\wedge$ h[$<s'>$ g $\wedge$ h[\dots]]]]
\end{forest}
\end{center}
\end{column}

\begin{column}{6cm}
\begin{center}

\begin{forest}
  [\dots[$<s>$ g $\wedge$ (t $\equiv$ u) $\wedge$ h[\xmark]]]
\end{forest}
\end{center}
\end{column}
\end{columns}
  
\begin{itemize}
  \item Все унификации проталкиваем вверх и вычисляем в подстановки
  \item Остается конъюнкция применений отношений
  \begin{itemize}
    \item Проверяем на совпадение с предками
    \item Проверяем на \emph{похожесть} с предками
    \item Иначе разворачиваем применения отношений
  \end{itemize}
\end{itemize}

Ошибка: слишком агрессивное разворачивание вызовов
\end{frame}

\begin{frame}[fragile]
  \transwipe[direction=90]
  \frametitle{Трудности}
\begin{itemize}
  \item Определить, что значит, что цели \emph{похожи}
  \item Определить, как \emph{абстрагировать}
  \item Определить, как учитывать связи между переменными
  \item Проверка на похожесть и абстракция связаны между собой существенно теснее, чем в функциональных языках
\end{itemize}

\bigskip

Ошибка: считать, что решение для функциональных языков применимо для трансформации реляционных программ
\end{frame}

\begin{frame}[fragile,t]
  \transwipe[direction=90]
  \frametitle{Примеры, которые уже работают}
\begin{center}
\begin{minipage}[c]{9.5cm}
\begin{lstlisting}[frame=single]  
appendo x y z = 
  (x === [] &&& y === z)
  ||| (fresh h t r (x === h : t &&& z === h : r &&& appendo t y r))
\end{lstlisting}
\end{minipage}
\end{center}

\begin{columns}[t]
\begin{column}{3.4cm}
\begin{badcode}
\begin{lstlisting}[frame=single]  
appendo [] y z
\end{lstlisting}
\end{badcode}
\end{column}

\pause

\begin{column}{5.5cm}
\begin{goodcode}
\begin{lstlisting}[frame=single]  
empty_appendo y z
\end{lstlisting}

\begin{lstlisting}[frame=single]  
empty_appendo y z = y === z
\end{lstlisting}
\end{goodcode}
\end{column}
\end{columns}
\end{frame}

\begin{frame}[fragile,t]
  \transwipe[direction=90]
  \frametitle{Примеры, которые уже работают}
\begin{center}
\begin{minipage}[c]{9.5cm}
\begin{lstlisting}[frame=single]  
appendo x y z = 
  (x === [] &&& y === z)
  ||| (fresh h t r (x === h : t &&& z === h : r &&& appendo t y r))
\end{lstlisting}
\end{minipage}
\end{center}

\begin{columns}[t]
\begin{column}{4.3cm}
\begin{badcode}
\begin{lstlisting}[frame=single]  
appendo [1,2,3] y z
\end{lstlisting}
\end{badcode}
\end{column}

\pause

\begin{column}{4.8cm}
\begin{goodcode}
\begin{lstlisting}[frame=single]  
app_123 y z 
\end{lstlisting}

\begin{lstlisting}[frame=single]  
app_123 y z = 
  z === 1 : (2 : (3 : y))
\end{lstlisting}
\end{goodcode}
\end{column}
\end{columns}
\end{frame}

\begin{frame}[fragile,t]
  \transwipe[direction=90]
  \frametitle{Примеры, которые уже работают}
\begin{center}
\begin{minipage}[c]{9.5cm}
\begin{lstlisting}[frame=single]  
appendo x y z = 
  (x === [] &&& y === z)
  ||| (fresh h t r (x === h : t &&& z === h : r &&& appendo t y r))
\end{lstlisting}
\end{minipage}
\end{center}

\begin{columns}[t]
\begin{column}{4.3cm}
\begin{badcode}
\begin{lstlisting}[frame=single]  
appendo x [1,2,3] z
\end{lstlisting}
\end{badcode}
\end{column}

\pause

\begin{column}{5.8cm}
\begin{mehcode}
\begin{lstlisting}[frame=single]  
app_123 x z 
\end{lstlisting}

\begin{lstlisting}[frame=single]  
app_123 x z = 
  (x === [] &&& z === [1,2,3])
  ||| (fresh h t r ( x === h : t 
             &&& z === h : r 
             &&& app_123 t r)) 
\end{lstlisting}
\end{mehcode}
\end{column}
\end{columns}
\end{frame}

\begin{frame}[fragile,t]
  \transwipe[direction=90]
  \frametitle{Примеры, которые уже работают}
\begin{center}
\begin{minipage}[c]{9.5cm}
\begin{lstlisting}[frame=single]  
appendo x y z = 
  (x === [] &&& y === z)
  ||| (fresh h t r (x === h : t &&& z === h : r &&& appendo t y r))
\end{lstlisting}
\end{minipage}
\end{center}

\vspace{-0.5cm}

\begin{columns}[t]
\begin{column}{4cm}
\begin{badcode}
\begin{lstlisting}[frame=single]  
appendo x y [*i*] 
 &&&  appendo [*i*] z r
\end{lstlisting}
\end{badcode}
\end{column}

\pause

\begin{column}{6cm}
\begin{goodcode}
\begin{lstlisting}[frame=single]  
double_app x y z r
\end{lstlisting}

\begin{lstlisting}[frame=single]  
double_app x y z r = fresh h t s 
  (x === [] &&& y === [] &&& z === r)
  ||| ( x === [] 
    &&& y === h : t 
    &&& r === h : s 
    &&& appendo t z s)
  ||| ( x === h : t 
    &&& r === h : s 
    &&& double_app t y z s)
\end{lstlisting}
\end{goodcode}
\end{column}
\end{columns}
\end{frame}

\begin{frame}[fragile]
  \transwipe[direction=90]
  \frametitle{Текущее положение дел}
\begin{itemize}
  \item Попробовали реализовать несколько версий суперкомпиляции
  \begin{itemize}
    \item Некоторые программы успешно преобразовываются
    \item Некоторые преобразовываются с ухудшением производительности
    \item На некоторых суперкомпиляция не завершается
  \end{itemize}
  \item Стало понятно, \emph{что} мы делали неправильно
  \item Решено \emph{буквально} адаптировать для miniKanren решения для логического программирования 
\end{itemize}
\end{frame}


\begin{frame}[fragile]
  \transwipe[direction=90]
  \frametitle{Дальнейшие планы}
\begin{itemize}
  \item Доведение до работоспособности конъюнктивной частичной дедукции
  \item Поддержка отношений высшего порядка
  \item ``Негативная'' суперкомпиляция
  \item Адаптация более мощных техник символьных вычислений
\end{itemize}
\end{frame}


\end{document}