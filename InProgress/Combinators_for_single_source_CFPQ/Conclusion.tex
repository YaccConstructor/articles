\section{Conclusion and Future Work}

We show that single-source context-free path queries can be evaluated in reasonable time for real-world graphs. 
Also, we demonstrate that the combinators-based approach to CFPQ is very flexible and powerful.

We demonstrate a combinator-based approach implemented in Meerkat.Graph Scala library, but this approach can be implemented in almost any high-level programing language.
While combinators are a very powerful way to specify context-free queries, it may seem hard to understand for many users.
There are other algorithms for context-free path queries which should be applicable for single-source path querying (GLL-based~\cite{Grigorev:2017:CPQ:3166094.3166104, MEDEIROS201975} or GLR-based~\cite{10.1007/978-3-319-41579-6_22, 10.1007/978-3-319-91662-0_17}) and we hope that they can be integrated with the existing graph database in a more convenient way.
But it is necessary more research in this direction.

We should investigate wore datasets to detect other shapes of query results.
For example, we should investigate the behavior of single-source querying in the case when a number of resulting paths is small, but paths are relatively long.
And the first question is which data analysis tasks lead to this scenario.
Also, we should provide detailed theoretical analysis of single-source CFPQ.

One of the important directions of future research is to optimize the performance of the proposed solution.
We hope that it is possible to reduce the utilization of graph size-dependent structures and thus make query execution time depends only on the size of the result.
One of the possible solutions is deep integration with Neo4j infrastructure to utilize cache system.

Another direction is combinators library improvement.
First of all, it is necessary to make combinators syntax more user-friendly.
Also, it is necessary to create a set of query templates (see same-generation template).