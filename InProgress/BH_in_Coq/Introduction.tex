\section{Introduction}

Formal language theory has deep connection with different areas such as ststic code analysis~\cite{!!!}, graph database querying~\cite{hellingsRelational, hellingsPathQuerying}, formal verification~\cite{!!!}, and others.
One of the most friquent scenarion is formulate problem in terms of languages intersection.
For example in verification one can use one languge as a model of program and another language for undesirable behaviors (for example from program specification).
In case when intersection of these two languags is not empty one can conclude that program is incorrect, so we are interested in languages intersection emptiness problem desidability.
But in some case we want to build constructive representation of intersection. 
For example, when we use languages intersection as a model for qury execution: language which prodused by intersection is a qury result and we want to have ability to process it.

Thus emptiness of intersection and constructive representation of intersection are useful for different applications.
The simplest case is linear---regular and we have a big number of works on sertified regular expressions~\cite{!!!}.
Regular---regular and regualr queryies in Coq~\cite{!!!}.
Next and one on wost comprehansive cases with desidable problem of emptyness problem: regular---context-free.
It is actual for parsing, program analysis, graph analysis~\cite{!!!}.
Constructive result is important: paths, etc

It is the well-known fact tat context-free languages are closed under intersection with regular languages. 
Theoretical result is Bar-Hillel theorem~\cite{bar1961formal} which provides construction for resulting language description.


Some of these aplications require sertifications. 
For verification is evident.
For databases, for example, it may be necessary to reason on sequrity aspects and, thus, we should create certified solutions for query executing.
So, mechanization of BH theorem may be ysefull step for...
On the other hand, mechanization (formalization) is important and many work done on formal languages theory mechanization. 
Parsing algorithms and reasoning about other problems on languages intersection.

Short overview of current results.
Many different parts of formal languages are mechanized. 
Smolka.
Algorithms and basic results.

Our work is a first step: mechanization of theoretical results.
The main contribution of this paper may be summarized as follows.
\begin{itemize}
\item We provide constructive proof of the Bar-Hillel theorem in Coq.
\item We generalize Smolka's CFL results: terminals is abstract types....
\item All code are publised on GitHub: \url{https://github.com/YaccConstructor/YC_in_Coq}.
\end{itemize}
