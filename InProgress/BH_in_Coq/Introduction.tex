\section{Introduction}

Different on languages intersection is a one of fundamental problem in formal languages theory.
Many different problems: Emptiness of intersection, closure under intersection, constructing of intersection 

It is the well-known fact tat context-free languages are closed under intersection with regular languages. 
Theoretical result is Bar-Hillel~\cite{bar1961formal} theorem which provide construction for resulting language description.

Language intersection problem is a foundation in many areas.
Parsing, program analysis, graph analysis~\cite{hellingsRelational, hellingsPathQuerying}.
Method proposed by Hellings is B-H theorem. 
All-path semantics.
Foundation in some areas: graphs, code analysis, etc.
Bar-Hillel theorem is a main on .

Mechanization (formalization) is important and many work done on formal languages theory mechanization. 
Parsing algorithms and reasoning about other problems on languages intersection.

Short overview of current results.
Many different parts of formal languages are mechanized. 
Algorithms and basic results.

The main contribution of this paper may be summarized as follows.
\begin{itemize}
\item We provide constructive proof of the Bar-Hillel theorem in Coq.
\item We generalize Smolka's CFL results: terminals is abstract types....
\item ...
\end{itemize}
