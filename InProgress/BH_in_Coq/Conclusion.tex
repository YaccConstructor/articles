\section{Conclusion}

We present mechanized proof of Bar-Hillel theorem on closure of context-free languages under intersection with regular one.
Also we generalize results of Smolka and !!! : generalized terminal alphabet. 
It makes More flexible and ease for reusing.
All results are published at GitHub.

One of direction of future research is mechanization of practical algorithms which are just implementation of Bar-Hillel theorem.
For example, context-free path querying algorithm, based on CYK~\cite{Hellings,RDF} or even on GLL~\cite{scott2010gll} parsing algorithm~\cite{grigorev2016context}.
Final target here is certified CFPQ alike Regular by Vardi

Yet another direction is mechanization of other problems on language intersection which can be useful for applications. For example, intersection of two context-free grammars one of which describes finite language~\cite{nederhof2002parsing, nederhof2004language}. It may be useful for compressed data processing.

