\section{Conclusion}

We present mechanized in Coq proof of Bar-Hillel theorem on closure of context-free languages under intersection with regular one.
By this we increase mechanized part of formal language theory and provide a base for reasoning about many applicative algorithms which are based on languages intersection.
Also we generalize results of Smolka and !!! : generalized terminal alphabet. 
It makes previously existing results more flexible and ease for reusing.
All results are published at GitHub and equipped with automatically generated documentation.

One of direction of future research is mechanization of practical algorithms which are just implementation of Bar-Hillel theorem.
For example, context-free path querying algorithm, based on CYK~\cite{Hellings,RDF} or even on GLL~\cite{scott2010gll} parsing algorithm~\cite{grigorev2016context}.
Final target here is certified algorithm for context-free constrained path querying for graph database.

Yet another direction is mechanization of other problems on language intersection which can be useful for applications.
For example, intersection of two context-free grammars one of which describes finite language~\cite{nederhof2002parsing, nederhof2004language}.
It may be useful for compressed data processing~\cite{!!!}.

Finally, thmthnk about integration with other results, generalization and zero-costs reusing (Firsov).

