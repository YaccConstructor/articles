Static analyses are important part of modern development tools. 
Generally speaking, they take care of verifying correctness of some program's behaviour freeing a programmer from this duty.
By used scope of program, an analyze can be classified as intraprocedural or interprocedural, i.e. as those which make decisions based on only one current procedure or based on the whole program respectively.
And interprocedural analyses, in theory, can be more precise due to amount of available information.

\begin{figure}[h]
	\includegraphics[width=\linewidth]{pictures/{SampleCode.dia}.png}
	\caption{Sample code}
	\label{fig:SampleCode}
\end{figure}

For example, let's consider the listing~\ref{fig:SampleCode}.
It is known that method \textit{Sink} is vulnerable to invalid arguments.
The method \textit{Filter} validates its argument and ensures that if some data is returned then it is definitely valid.
And the field \textit{Source} is known as potentially tainted.
So, the problem is to find out whether data from \textit{Source} reaches \textit{Sink} bypassing \textit{Filter}.

Generally speaking, this problem is a special case of label-flow analysis, so there are several approaches to solving such problems.
One of them is CFL-reachability. (TODO: ADV: performance, DISADV: expressive power, non-obvious structure, CITATIONS)
Another is abstract interpretation. (TODO: vice versa)
We propose to combine these two approaches to achieve acceptable performance and expressive power.
I.e. the program is translated into a graph as it is in CFL-r, but constraints that specifiy what paths is needed to be accepted are set by pushdown automaton which transition relation simulates the semantics of original program.
Let's take a closer look at these two components that define an analysis in conjunction.
