\documentclass[sigconf]{acmart}

\usepackage{calc}
\usepackage{amsmath}
\usepackage{mathtools}
\usepackage{textcomp}

\begin{document}

\title[IDEs-Friendly Interprocedural Analyser]{IDEs-Friendly Interprocedural Analyser}

\author{Ilya Nozhkin}
\affiliation{
  \institution{Saint Petersburg State University}
  \streetaddress{7/9 Universitetskaya nab.}
  \city{St. Petersburg}
  \country{Russia}
  \postcode{199034}
}
\email{nozhkin.ii@gmail.com}

\author{Semyon Grigorev}
\orcid{0000-0002-7966-0698}
\affiliation{
  \institution{Saint Petersburg State University}
  \streetaddress{7/9 Universitetskaya nab.}
  \city{St. Petersburg}
  \country{Russia}
  \postcode{199034}
}
\affiliation{
  \institution{JetBrains Research}
  \streetaddress{Universitetskaya nab., 7-9-11/5A}
  \city{St. Petersburg}
  \country{Russia}
  \postcode{199034}
}
\email{s.v.grigoriev@spbu.ru}
\email{semen.grigorev@jetbrains.com}


\begin{abstract}
We propose a extensible framework for interprocedural static code analysis implementation.
Our solution is based on CFL-reachability: analysis is formulated in terms of context-free constrainted reachability in interprocedural graph.
Extensible architecture allows one to implement new analysis and integrate it into your favorite IDE or static code analysis tool.
To demnstrate abilities of our solution, we implement the plugin whish provides basic taint analysis and variable flow analysis upon ReSharper infrastructure.
We demonstarte its applicability for real-world problems.
\end{abstract}


\begin{CCSXML}
<ccs2012>
<concept>
<concept_id>10011007.10010940.10010992.10010998.10011000</concept_id>
<concept_desc>Software and its engineering~Automated static analysis</concept_desc>
<concept_significance>500</concept_significance>
</concept>
<concept>
<concept_id>10011007.10011006.10011066.10011069</concept_id>
<concept_desc>Software and its engineering~Integrated and visual development environments</concept_desc>
<concept_significance>500</concept_significance>
</concept>
<concept>
<concept_id>10011007.10011074.10011099.10011102</concept_id>
<concept_desc>Software and its engineering~Software defect analysis</concept_desc>
<concept_significance>500</concept_significance>
</concept>
<concept>
<concept_id>10011007.10010940.10011003.10011114</concept_id>
<concept_desc>Software and its engineering~Software safety</concept_desc>
<concept_significance>300</concept_significance>
</concept>
<concept>
<concept_id>10003752.10003766.10003771</concept_id>
<concept_desc>Theory of computation~Grammars and context-free languages</concept_desc>
<concept_significance>300</concept_significance>
</concept>
</ccs2012>
\end{CCSXML}

\ccsdesc[500]{Software and its engineering~Automated static analysis}
\ccsdesc[500]{Software and its engineering~Integrated and visual development environments}
\ccsdesc[500]{Software and its engineering~Software defect analysis}
\ccsdesc[300]{Software and its engineering~Software safety}
\ccsdesc[300]{Theory of computation~Grammars and context-free languages}

\keywords{Static code analysis, interprocedural analysis, CFL-reachability, taint analysis, IDE, plugin, context-free languages, PDA}

%\acmBadgeR{artifacts_available}

\maketitle

\section{Introduction}
\section{Introduction}

Context-Free Path Querying (CFPQ) is an actively developed area in graph database analysis.
CFPQ is also used for static code analysis~\cite{Reps,10.1145/193173.195287,Zheng}, RDF querying~\cite{10.1007/978-3-319-46523-4_38,MEDEIROS201975}, biological data analysis~\cite{cfpqBio}.

Most of research is focused on developping algorithms for CFPQ evaluation~\cite{hellingsRelational,ward2008distributed,cfpqBio,MEDEIROS201975,Azimov:2018:CPQ:3210259.3210264,Grigorev:2017:CPQ:3166094.3166104}, whereas specification languages for context-free queries are not investigated enough.
Best to our knowledge, only one extension for Sparql supports context-free constarints: cfSPARQL~\cite{10.1007/978-3-319-46523-4_38}.
There is also a proposal for CFPQ as a part of Cypher\footnote{Proposal with path pattern syntax for openCypher: \url{https://github.com/thobe/openCypher/blob/rpq/cip/1.accepted/CIP2017-02-06-Path-Patterns.adoc}.
It is shown that context-free constraints can be expressed with the proposed syntax. Access date: 30.03.2020} language, but there is no implementation for it yet.
We believe that more research should be conducted on the specification languages fo context-free constraints in graph querying.

It is worth noting that graph analysis is often only a part of a more complex system, usually implemented in a general-purpose language.
Since a graph query language is unsuitable to implement a whole system, there should be means of integration of them into general-purpose programming languages.
There are many ways to integrate them ranging from creating graph queries from string values of a general-purpose language to implementing a special embedded domain specific language, and even more sophisticated.

Although simple, the string manipulating approach does not provide a developper with any safety guarantees.
There is no way to ensure that a string generated by an application is a valid query or, in case it is not, to provide any feedback.
This makes string manipulating technique error prone, the code --- unclear and hard to maintain.

Safety of an embedded DSL entirely depends on its implementation.
Some general-purpose languages with powerfull type systems (such as \haskell{}, \ocaml{} or \scala{}) or the ones supporting hygienic macros (such as \scheme{} or \rust{}) facilitate creating safe and reliable DSLs.
Still, they typically lack full support of a development environment: it may be harder to debug queries or issues with composability may arise.

There is a general trend towards imposing more restricting type systems on programming languages.
Among many others are typing annotations for \python{} and \typescript{} code and nullability checks in \kotlin{}.
Typing graphs and query languages improves  readability and simplifies maintainance~\cite{10.1145/2076623.2076653}.

Parser combinators are the answer to the integration of parsing into a general-purpose programming language.
Recursive descend parsers are encoded as functions of the host language, while grammar constructions such as sequencing and choice are implemented as higher-order functions.
This idea was first introduced in~\cite{burge} and further developped in numerous works.
Notable development is monadic parser combinators~\cite{hutton1996monadic}.
In this approach, one can not only parse the input, but simultaneously run semantics calculation if parsing succeeds.
Paper~\cite{izmaylova2016practical} proposed the first monadic parser combinator library which solves the long-standing problem of inability to handle ambiguous and left-recursive grammars.
A library for graph querying was developped~\cite{10.1145/3241653.3241655} based on this work.
The core idea is to use generalized parser combinators as both a way to formulate a query and to execute it.
This approach inherits benefits of combinatory parsing: ease of code reuse, type safety guaranteed by the host language and, since the parser is simply a function, the integrated development support.

Besides integration, it can compute both the single-source and all pairs semantics, as well as execute user actions.
The~single-source semantics is relevant to many real-world applications, including manual data analysis.
It also may be less time-intensive, since on average it needs to expore only a subgraph of the input graph.
Many querying algorithms are only capable to compute all pairs reachability which makes them unsuitable for some applications.

In this paper we make the following contributions.
\begin{itemize}
  \item We demonstrate how to use combinatory-based graph querying on example.
  \item We illustrate such features of the approach as type-safety, flexibility (composability and generics), IDE support and computing user-defined actions.
  \item We evaluate single-source context-free path querying on some real-world RDFs.
  \begin{itemize}
    \item Based on our evaluation, the most common case in RDF context-free querying is when the number of paths in the answer set is big, but they are small.
    \item We demonstrate that the single-source CFPQ can feasibly be used to evaluate such queries.
    \item We conclude that there is a need for a further detailed analysis of both theoretical time and space complexity of single-source CFPQ.
  \end{itemize}
\end{itemize}


\section{Analysis definition}
\subsection{Graph extraction}

The graph that is explored during analysis is an aggregate of control-flow graphs of each method.
The one that corresponds to our example is shown at figure (TODO: REF2)

TODO: REF2: GRAPH

It has the following structure.
Each node represents a position in a program.
Each edge contains an operation that reflects the one from the source code.

\subsection{PDA construction}


\section{Solution}
Using the described idea, we have developed the framework which makes it possible to implement interprocedural static code analysis based on CFL-reachability approach.
Our solution is an extensible infrastructure which is responsible for extracting graphs from the source code, aggregating them and their metadata into one database and finding paths in this database accepted by PDAs representing different analyses.
Logically, it is divided into two separate entities which are shown in fig.~\ref{fig:SolutionStructure}.

\begin{figure}[h]
	\includegraphics[width=\linewidth]{pictures/{SolutionStructure.dia}.png}
	\caption{Solution structure}
	\label{fig:SolutionStructure}
\end{figure}

The first entity, the core of the solution, is a backend implemented as a remote service running in a separate process and interacting with the frontend using a socket-based protocol.
Architecturally, it is also divided into two subsystems.
First of them is a database which provides the continuous incremental updating of the graph and its metadata, and supports dumping to a disk and further loading in the beggining of next session.
The second one is responsible for execution of analyses.
It contains an implementation of the resolver improving the one which is provided by IDE by adding dynamic invocations resolving such as lambdas propagation, the algorithm of PDAs running and the first extension point making the adding new analyses possible.
The set of analyses contained in the backend can be extended by adding a new PDA as an implementation of the appropriate generic abstract class.
Furhter, it is possible to run this new anaysis using existing internal algorithm of PDA simulation and get any finite subset of paths in the graph which are accepted by the PDA.

The second entity is a frontend that is also divided into two subsystems.
First of them is a graph extractor which parses source code, extracts graphs and metadata from it and sends collected data to the backend.
The second one is a results interpreter which receives the set of paths in the graph each of which leads to an error, maps it to the source code and translates it to a human-readable format.
For example, it can highlight pieces of code contained in the received paths.

Since a frontend is completely separate from the backend and the only requirement for it is to follow the communication protocol, the frontend can be considered as the second extension point.
I.e. it is possible to replace the currently implemented frontend with any other implementation having the same functions as the original one including graphs extraction and results processing.
The current implementation is also open to modifications which add support for new types of analysis or any other features which requires interaction with IDE.

The protocol itself is based on request-response pattern where the frontend acts as a master and the backend acts as a slave.
I.e. the frontend informs the backend if there are some changes in the source code and asks it to update the database according to them.
When there is a need to get the results of the analysis, for example, when the IDE performs the code highliglighting, the frontend asks the backend for found issues, maps results to the source code and highlights corresponding lines.


\section{Evaluation}
\section{Evaluation}

The goal of this evaluation is to assess the performance scaling of Spla on Vortex.
Due to limitations in atomic operation support within the RTL implementation, all experiments were performed using the SimX functional simulator.

\subsection{Environment}

Initial testing revealed issues with floating-point operations, which produced incorrect results for some hardware configurations.
Consequently, we limited subsequent experiments to Breadth-First Search (BFS) and Triangle Counting (TC), excluding Single-Source Shortest Path (SSSP) and PageRank.
To keep simulation times manageable, we used a single graph from the SuiteSparse matrix collection\footnote{A diverse collection of sparse matrices from various domains: \url{http://sparse.tamu.edu/}}: soc-Epinions1, with 75~888 vertices and 508~837 edges.


We conducted two series of experiments.
The first varies the number of warps and threads per warp while keeping the number of clusters and cores fixed (at 2 and 4, respectively), with the goal of selecting the best core configuration while preserving multi-core execution to account for cache effects.
The second series, using the best configuration identified in the first step, varies the number of clusters and cores per cluster to assess scaling at the core and cluster levels.
Cache sizes were set to their default values: 16 KB for $L_1$, 1 MB for $L_2$, and 2 MB for $L_3$.

We use the number of cycles reported by SimX as a performance metric.
For multi-core configurations, we report the maximum cycle count across all cores.
During the experiments, we encountered unexpected behavior in SimX that led to out-of-memory exceptions. 
Therefore, some data points are missing from the graphs below.

\subsection{Results}

In figures~\ref{fig:tc_threads_warps} and~\ref{fig:bfs_threads_warps}

\begin{figure}
    \begin{center}
        \includegraphics[width=0.49\textwidth]{pictures/TC_threads_warps.pdf}
    \end{center}
    \caption{Scaling analysis of triangle counting for varying numbers of warps and threads per warp}
    \label{fig:tc_threads_warps}
\end{figure}

\begin{figure}
    \begin{center}
        \includegraphics[width=0.49\textwidth]{pictures/BFS_threads_warps.pdf}
    \end{center}
    \caption{Scaling analysis of BFS for varying numbers of warps and threads per warp}
    \label{fig:bfs_threads_warps}
\end{figure}

Best configuration for BFS is 2 warps, 8 threads per warp (16 threads total). 
Best configuration for TC is 4 warps, 16 threads per warp (64 threads total).


\begin{figure}
    \begin{center}
        \includegraphics[width=0.49\textwidth]{pictures/BFS_cores_clusters.pdf}
    \end{center}
    \caption{Scaling analysis of BFS for varying numbers of clusters and cores per cluster}
    \label{fig:bfs_cores_clusters}
\end{figure}


Edges per core on cycle. Compare with Spla on other GPUs.

\subsection{Scaling limitations analysis}

%sum(scoreboard stalls * lsu_percent) / sum(instr) * 100
To analyze the reasons for limited scaling as the number of threads increases, we measured the average utilization of the ALU and LSU, in terms of stall cycles, for the best BFS configuration.
The results are presented in Fig.~\ref{fig:bfs_alu_stalls} and Fig.~\ref{fig:bfs_lsu_stalls}, respectively.
The data indicate that the LSU is the performance bottleneck within the core.

The same bottleneck was observed in the scaling analysis across clusters and cores.
Whether increasing cache sizes can alleviate this problem remains a question for future research.
We anticipate that careful cache size tuning may help identify a more efficient configuration.

\begin{figure}
    \begin{center}
        \includegraphics[width=0.49\textwidth]{pictures/BFS_alu.pdf}
    \end{center}
    \caption{ALU stalls on BFS for the best configuration}
    \label{fig:bfs_alu_stalls}
\end{figure}

\begin{figure}
    \begin{center}
        \includegraphics[width=0.49\textwidth]{pictures/BFS_lsu.pdf}
    \end{center}
    \caption{LSU stalls on BFS for the best configuration}
    \label{fig:bfs_lsu_stalls}
\end{figure}

\section{Conclusion and Future Work}

Platform presented.

Education. Metaprogramming, translators development, GPGPU programming, etc.

Graph parsing.

Geterogenious porgramming generalization. Hopac is better then MBP~\footnote{\url{https://vasily-kirichenko.github.io/fsharpblog/actors}}.

Research: Automatic memory management.

Data to code translation (automata can be translated into code instead of data structures in memory)

Other technical improvements: IDE support, type provider improvements, new OpenCL standard support, runtime extension, etc.

\bibliographystyle{ACM-Reference-Format}
\bibliography{IDE_Friendly_Interprocedural_Analyser}

\end{document}
