%Перечня рецензируемых научных изданий
\subsection*{\Large Общая характеристика работы}
\fontsize{14pt}{15pt}\selectfont
\subsubsection*{\large{Актуальность темы исследования}}
Графы используются в качестве структуры данных для представления больших объемов информации в компактной и удобной для анализа форме во многих областях, например, в биоинформатике, в графовых базах данных, при статическом анализе программ. При этом оказывается необходимым вычислять запросы к большим графам с целью выявления сложных зависимостей между их вершинами. Результатом вычисления таких запросов является множество неявных отношений между вершинами графа, то есть путей в графе. Естественно помечать ребра графа символами из некоторого конечного алфавита и выделять пути с помощью формальных грамматик над тем же алфавитом (регулярные выражения, контекстно-свободные грамматики). В настоящее время активно исследуются запросы к графам в виде контекстно-свободных (КС) грамматик, так как они позволяют описывать более широкий класс запросов, чем регулярные выражения. Также интерес представляют вложенные регулярные запросы к графам, так как они расширяют выразительную мощность регулярных выражений добавлением фильтров с логическими операциями, которые в свою очередь содержат регулярные выражения.

Однако большинство существующих алгоритмов в данной области имеют низкую производительность на больших графах, что затрудняет их анализ. Хорошую производительность показал матричный подход к вычислению КС-запросов к графам. Данный подход позволяет нагрузить основную вычислительную сложность на вычисление матричных операций. Кроме того, в процессе анализа может быть применен широкий класс матричных оптимизаций, например, разреженное представление матриц, параллельное вычисление. Но существующий матричный алгоритм в данной области позволяет лишь установить факт наличия между двумя вершинами пути определенного вида, при этом сам путь не предоставляется, хотя во многих областях нахождение пути необходимо. Поэтому для вычисления КС-запросов к графам с некоторыми семантиками не существует эффективного алгоритма. Таким образом, для большинства типов запросов к графам необходима разработка алгоритмов, эффективно работающих на больших графах.



%Взаимодействие различных компонент приложений часто реализуется с помощью встроенных языков, то есть приложение, созданное на одном языке, генерирует код на другом языке и передаёт этот код на выполнение в соответствующее окружение. Примерами могут служить динамические SQL-запросы к базам данных в Java-коде или формирование HTML-страниц в PHP-приложениях. Генерируемая программа строится таким образом, чтобы в момент выполнения результирующий фрагмент кода (строка) представлял собой корректное выражение на соответствующем языке. Такой подход весьма гибок, так как позволяет использовать для формирования фрагментов кода различные строковые операции (replace, substring и т.д.) и комбинировать код из различных источников (например, учитывать текстовый ввод пользователя, что часто используется для задания фильтров при конструировании SQL-запросов). Необходимо отметить, что такой подход не имеет дополнительных накладных расходов, присущих, например, ORM-технологиям, и это позволяет достигать высокой производительности. 

%Однако динамическое формирование программ часто происходит с помощью операций конкатенации в циклах, условных операторах или рекурсивных процедурах, что приводит к множеству возможных вариантов значений для каждого выражения, что затрудняет их анализ. Поэтому фрагменты кода на встроенных языках воспринимаются компилятором исходного языка как простые строки, что приводит к высокой вероятности возникновения ошибок во время выполнения программы. В худшем случае такие ошибки не приведут к прекращению работы приложения, что явно указало бы на проблему, но целостность данных при этом может оказаться нарушенной. Более того, например, при наличии в коде приложения встроенных SQL-запросов нельзя, не проанализировав все динамически формируемые выражения, точно ответить на вопрос о том, с какими элементами базы данных не взаимодействует система, и  удалить их. При переносе такой системы на другую СУБД необходимо гарантировать, что для всех динамически формируемых выражений значение в момент выполнения будет корректным кодом на языке новой СУБД. Кроме того, при создании приложений распространённой практикой является использование интегрированных сред разработки, выполняющих подсветку синтаксиса и автодополнение, сигнализирующих о синтаксических ошибках, предоставляющих инструменты рефакторинга. Эти возможности значительно упрощают процесс разработки и отладки приложений и полезны не только для основного языка, но и для встроенных языков. Таким образом, для решения данных задач необходимы инструменты, проводящие статический анализ динамически формируемых программ.  

\subsubsection*{\large{Степень разработанности темы исследования}}
TODO
%Существуют классические исследования, посвященные разработке компиляторов --- работы А.~Ахо, А.~Брукера, С.~Джонсона, А.~П.~Ершова,  А.~Н.~Терехова, В.~О.~Сафонова, Б.~К.~Мартыненко и др.  Однако изложенные там алгоритмы синтаксического анализа, как и методы обобщённого синтаксического анализа,  исследованные такими учёными как Масару Томита (Masaru Tomita), Элизабет Скотт (Elizabeth Scott) и Адриан Джонстон (Adrian Johnstone) из университета Royal Holloway (Великобритания), Ян Рекерс (Jan Rekers, University of Amsterdam), Элко Виссер (Eelco Visser), не могут быть применены к решению задачи анализа динамически формируемых программ, поскольку предназначены для обработки входных данных, представимых в виде линейной последовательности символов, а такое представление динамически формируемых программ не возможно.

%Анализу динамически формируемых строковых выражений посвящены работы таких зарубежных учёных как Кюнг-Гу Дох (Kyung-Goo Doh), Ясухико Минамиде (Minamide Yasuhiko), Андерс Мёллер (Anders M{\o}ller) и отечественных учёных А.~А.~Бреслава, М.~Д.~Шапот. Хорошо изучены вопросы проверки корректности динамически формируемых выражений и поиска фрагментов кода, уязвимых для SQL-инъекций. Однако данные работы исследуют отдельные проблемы статического анализа динамически формируемых программ, оставляя в стороне создание готовых алгоритмов (в частности, не строят структурное представление анализируемых программ). В связи с этим возникают проблемы масштабируемости данных результатов, например, создание на их основе более сложных видов статического анализа.

%Также важным является предоставление компонентов, упрощающих создание новых инструментов для решения конкретных задач. Данный подход хорошо исследован в области разработки компиляторов, где широкое распространение получили генераторы анализаторов и пакеты стандартных библиотек (работы А.~Ахо, А.~Брукера, С.~Джонсона и др.), однако его применение в области анализа динамически формируемого кода не исследовано. 

%В работах отечественных учёных М.~Д.~Шапот и Э.~В.~Попова, а так же зарубежных учёных Антони Клеви (Anthony Cleve), Жан-Люк Эно (Jean-Luc Hainaut), Йост Виссер (Joost Visser) рассматривается реинжиниринг информационных систем, использующих встроенные SQL-запросы, однако не формулируется общего метода для решения таких задач. Этот вопрос также не затрагивается в классических работах, посвященных реинжиниригу (Х.~Миллера, А.~Н.~Терехова, Р.~С.~Арнольда и др.).

%Таким образом, актуальной является задача дальнейшего исследования статического анализа динамически формируемых строковых выражений. Кроме этого важным является решение вопросов практического применения средств анализа динамически формируемого кода: упрощение разработки инструментов анализа и создание методов их применения в реинжиниринге программного обеспечения.

\subsubsection*{\large{Объект исследования}}

 Объектом исследования являются алгоритмы вычисления запросов к графам.

\subsubsection*{\large{Цель и задачи диссертационной работы}}

\textbf{Целью} данной работы является разработка эффективных алгоритмов вычисления запросов к графам.

Достижение поставленной цели обеспечивается решением следующих \textbf{задач}.
\begin{enumerate}
	\item Разработать матричный алгоритм вычисления КС-запросов к графам, позволяющий предоставлять по одному искомому пути для каждой пары вершин, если они существуют.
	\item Разработать матричный алгоритм вычисления КС-запросов к графам, использующий произведение Кронекера.
	\item Разработать алгоритм вычисления вложенных регулярных запросов к графам, транслирующий запрос в программу на даталоге.
\end{enumerate}

%Цели и задачи диссертационной работы соответствуют области исследований паспорта специальности 05.13.11 ``Математическое и программное обеспечение вычислительных машин, комплексов и компьютерных сетей'' --- пункту 
%1 (модели, методы и алгоритмы проектирования и анализа программ и программных систем, их эквивалентных преобразований, верификации и тестирования),
%пункту 2 (языки программирования и системы программирования, семантика программ) и пункту 10 (оценка качества, стандартизация и сопровождение программных систем).

\subsubsection*{\large{Методология и методы исследования}}
TODO

%Методология исследования основана на подходе к спецификации и анализу формальных языков, который начал активно развиваться в 50-х годах 20-го века в связи с изучением естественных языков (работы Н.~Хомского). В последствии этот подход получил широкое распространение в областях, связанных с обработкой языков программирования.
%При этом основными элементами данного подхода являются алфавит и грамматика исследуемого языка, разбиение автоматической обработки языка на выполнение таких шагов, как лексический, синтаксический и семантический анализ. Решаемые в связи с этим задачи связаны с поиском эффективных алгоритмов, выполняющих эти шаги. 

%В работе применяется алгоритм обобщённого восходящего синтаксического анализа RNGLR, созданный Элизабет Скотт (Elizabeth Scott) и Адриан Джонстон (Adrian Johnstone) из университета Royal Holloway (Великобритания). Для компактного хранения леса вывода используется структура данных Shared Packed Parse Forest (SPPF), которую предложил Ян Рекерс (Jan Rekers, University of Amsterdam). Доказательство завершаемости и корректности предложенного алгоритма проводится с применением теории формальных языков, теории графов и теории сложности алгоритмов. Приближение множества значений динамически формируемого выражения строилось в виде регулярного множества, описываемого с помощью конечного автомата.

\subsubsection*{\large{Положения, выносимые на защиту}}
\begin{enumerate}
	\item Разработан матричный алгоритм вычисления КС-запросов к графам, позволяющий предоставлять по одному искомому пути для каждой пары вершин, если они существуют. Доказана завершаемость и корректность предложенного алгоритма.
	\item Разработан матричный алгоритм вычисления КС-запросов к графам, использующий такую матричную операцию, как произведение Кронекера. Доказана завершаемость и корректность предложенного алгоритма.
	\item Разработан алгоритм вычисления вложенных регулярных запросов к графам, транслирующий запрос в программу на даталоге. Доказана завершаемость и корректность предложенного алгоритма.
\end{enumerate}


\subsubsection*{\large{Научная новизна}}

Научная новизна полученных в ходе исследования результатов заключается в следующем.

\begin{enumerate}

\item Алгоритм, предложенный в диссертации, отличается от аналогов (работы Семёна Григорьева, Джелле Хеллингса, Сяованга Чжана) активным использованием матричных операций в процессе вычисления запросов и отличается от матричного алгоритма Азимова Рустама возможностью построения по одному искомому пути для каждой пары вершин, если они существуют. Это позволяет как применять широкий класс матричных оптимизаций, так и предъявлять пути в качестве доказательства отношения определенного вида между парами вершин, что является важным результатом анализа во многих областях.

\item Алгоритм, предложенный в диссертации, отличается от аналогов (работы Семёна Григорьева, Джелле Хеллингса, Сяованга Чжана) активным использованием матричных операций в процессе вычисления запросов и отличается от матричного алгоритма Азимова Рустама использованием в процессе вычисления запросов произведения Кронекера и представлением КС-грамматики запроса в виде рекурсивного автомата. Это позволяет более эффективно вычислять сложные запросы к большим графам.

\item Алгоритм, предложенный в диссертации, отличается от аналогов (работы Хорхе Переса, Георга Готтлоба, Марсело Аренаса) трансляцией запроса в программу на даталоге и её последующим вычислением с помощью существующих техник. Это позволяет контролировать процесс вычисления запросов и исследовать только необходимую часть графа, что критически важно при анализе больших графов.

\end{enumerate}

\subsubsection*{\large{Теоретическая и практическая значимость работы}}
TODO
%Теоретическая значимость диссертационного исследования заключается в разработке формального алгоритма синтаксического анализа динамически формируемого кода, решающего задачу построения конечного представления леса вывода, не решаемую ранее, а также в формальном доказательстве завершаемости и корректности разработанного алгоритма. 

%На основе полученных в работе научных результатов был разработан инструментарий (Software Development Kit, SDK), предназначенный для создания средств статического анализа динамически формируемых выражений. Данный инструментарий позволяет автоматизировать создание лексических и синтаксических анализаторов при решении задач реинжиниринга --- изучения и инвентаризации систем, автоматизации трансформации выражений на встроенных языках. Предложенный инструмент также может использоваться при реализации поддержки встроенных языков в интегрированных средах разработки.

%С использованием разработанного инструментария было реализовано расширение к инструменту ReSharper (ООО ``ИнтеллиДжей Лабс'', Россия), предоставляющее поддержку встроенного T-SQL в проектах на языке программирования C\# в среде разработки Microsoft Visual Studio. Так же было выполнено внедрение результатов работы в промышленный проект по переносу хранимого SQL-кода с MS-SQL Server 2005 на Oraclе 11gR2 (ЗАО ``Ланит-Терком'', Россия). 


\subsubsection*{\large{Степень достоверности и апробация результатов}}
Достоверность и обоснованность результатов исследования опирается на использование формальных методов исследуемой области, выполнение формальных доказательств и инженерные эксперименты.

Основные результаты работы были доложены на ряде международных научных конференций: ... Дополнительной апробацией является то, что разработка предложенных алгоритмов была поддержана ... (РНФ?, РФФИ?)

\subsubsection*{\large{Публикации по теме диссертации}}
 Все результаты диссертации изложены в 3 научных работах, из которых 3~[1,2,3] содержат основные результаты работы и индексируются Scopus. Работы~[1,2,3] написаны в соавторстве. В~[1] Р.~Азимову принадлежит разработка алгоритма, доказательство его корректности и завершаемости, работа над текстом. В~[2] Р.~Азимову принадлежит работа над доказательствами корректности и завершаемости алгоритма, работа над текстом. В~[3] автору принадлежит схема трансляции запросов к графам в программу на даталоге, доказательство корректности и завершаемости алгоритма, работа над текстом.


%\subsubsection*{\large{Объем и структура работы}}
%Диссертация состоит из~введения, шести глав, заключения и~списка литературы. Полный объем диссертации \textbf{125}~страниц текста с~\textbf{26}~рисунками и~\textbf{8}~таблицами. Список литературы содержит \textbf{106}~наименований.

%\subsection*{\Large Содержание работы}


\newcounter{firstbib}


%\vfill
%\small
%\centering
%\hrule
%\vspace{2.5pt}
%Печать\\
%Печать\\
%Печать\\
%\vspace{2.5pt}
%\hrule
%\vspace{2.5pt}
%Печать\\
%Печать\\
%Печать\\
