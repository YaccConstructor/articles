\chapter{Обзор}\label{ch:ch1}
В данной главе введены основные термины и определения, используемые в работе, а также рассмотрены основные алгоритмы вычисления КС-запросов к графам и используемые в ходе исследования инструменты. 

\section{Теория графов и линейная алгебра}\label{sec:ch1/sec1}
В данном разделе вводится ряд обозначений, а также представляется основная информация об активно развивающемся направлении в решении задач из теории графов с помощью операций линейной алгебры.

Стандарт GraphBLAS. Графы представлены в виде матриц смежности. Используются практические результаты для вычисления операций над разреженными матрицами.

Операции над матрицами и соответсвующие им трансформации графов. Транспонирование, поэлементное сложение/умножение, обычное умножение.

Определение полукольца. Матричное умножение на полукольце. Ключевая идея GraphBLAS.

Таблица стандартных полуколец в GraphBLAS.

Предстваление матриц в GraphBLAS. CSR, CSC.

Таблица классических алгоритмов на графах с классической и матричными сложностями.

Конечные автоматы.

Рекурсивные автоматы.

Произведение Кронекера.


\section{Формальные языки и грамматики}\label{sec:ch1/sec2}
В данном разделе вводится ряд обозначений, а также представляется основная информация из теории формальных языков.

Алфавит

Цепочка символов

Язык

Грамматика

Вывод цепочки

Язык пораждаемый грамматикой

Дерево вывода цепочки

Одним из распространённых способов классификации грамматик является иерархия по Хомскому. Грамматики типа 0 1 2 3. КС-грамматика.

Нормальная форма хомского для КС-грамматик.

\section{Постановка задачи вычисления КС-запросов к графам}\label{sec:ch1/sec3}
В данном разделе вводится формальная постановка задачи вычисления КС-запросов к графам.

Хеллингс, контекстно-свободные отношения $R_A$. Выделим три возможных семантики запросов и приведем формальные постановки задачи вычисления КС-запросов для них.

Постановка задачи с реляционной семантикой запросов.

Постановка задачи с семантикой запросов одного пути.

Постановка задачи с семантикой запросов всех путей.

\section{Существующие алгоритмы вычисления КС-запросов к графам}\label{sec:ch1/sec4}
В данном разделе рассмотрены основные алгоритмы вычисления КС-запросов к графам.

Алгоритм Вэлианта, как вычисление КС-запросов для линейных графов. Может быть обобщён до ацикличных. Временная сложность.

Брэдфорд. Кратчайшие пути, языки Дика, определенный вид графа. Временная сложность.

Hellings/китайцы. Для всех семантик, всех графов. Временная сложность.

GLL, все семантики, все графы. Временная сложность.

Бразильцы. Все графы. Временная сложность.

\section{Используемые инструменты}\label{sec:ch1/sec5}
В данном разделе рассмотрены инструменты, используемые в ходе исследования.

GraphBLAS/SuiteSparse, на языке Си. Разреженные матрицы, их формат, вычисление операций на цпу.

RedisGraph графовая база данных. Разреженные матрицы, их формат. Выразительность языка запросов.

CFPQ\_data созданный датасет. Описать какие графы RDF и запросы к ним. Таблица.

\section{Выводы}\label{sec:ch1/sec6}
На основе проведённого обзора можно сделать следующие выводы, обосновывающие необходимость проведения исследования в области вычисления КС-запросов к графам.
\begin{itemize}
	\item Проблема вычисления КС-запросов к графам актуальна в нескольких областях: графовые базы данных, биоинформатика, статический анализ программ.
	\item Формулирование алгоритмов на графах в терминах операций линейной алгебры перспективное направление для улучшения производительности при работе с большими графами.
	\item Не проводилось исследований о возможность формулировки алгоритмов вычисления КС-запросов к графам в терминах операций линейной алгебры.
\end{itemize}

Обзор также позволяет выявить следующие подходы, технологии и средства.
\begin{itemize}
	\item Для построения алгоритма КС-запросов к графам с использованием операций линейной алгебры целесообразно придерживаться стандарта GraphBLAS.
	\item Представление матриц должно быть разреженным.
	\item Для вычисления на цпу можно использовать реализацию SuiteSparse.
	\item В качестве хранилища данных можно использовать RedisGraph.
	\item В качестве датасета для экспериментального исследования можно использовать CFPQ\_data.
\end{itemize}


\FloatBarrier
