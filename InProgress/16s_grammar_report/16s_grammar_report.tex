\documentclass[12pt]{article}  % standard LaTeX, 12 point type

\usepackage{geometry}

\usepackage{amsmath}
\usepackage{amsfonts,latexsym}
\usepackage{amsthm}
\usepackage{amssymb}
\usepackage[utf8]{inputenc} % Кодировка
\usepackage[english,russian]{babel} % Многоязычность
\usepackage{verbatim}
\usepackage{longtable}
%\usepackage{lscape}
\usepackage{csvsimple}
\usepackage[toc,page]{appendix}
\usepackage{booktabs}

\usepackage{float}
\usepackage{array}
\usepackage{multirow}
\usepackage{caption}
\usepackage{graphicx}
\usepackage{ucs}
\usepackage{rotating}
\usepackage{pdflscape}
\usepackage{afterpage}
\usepackage{capt-of}% or use the larger `caption` package

% unnumbered environments:

\theoremstyle{remark}
\newtheorem*{remark}{Remark}
\newtheorem*{notation}{Notation}
\newtheorem*{note}{Note}

\setlength{\parskip}{5pt plus 2pt minus 1pt}
\newcolumntype{C}{>{\centering\arraybackslash}p{1.3cm}}
\renewcommand\appendixpagename{Приложение}
\graphicspath{{pics/}}

\title{Использование КС-грамматики для распознавания \\ 16s рРНК}
\author{Семён Григорьев, Дмитрий Ковалёв}
\date{\today}

\begin{document}

\newgeometry{left=0.8in,right=0.8in,top=1in,bottom=1in}

\maketitle 

\section{Введение}

Одна из задач, возникающих в биоинформатике --- задача поиска и классификации маркерных цепочек, использующихся для обнаружения и классификации организмов.
При этом, многие маркерные цепочки обладают достаточно богатой вторичной структурой, что позволяет сделать предположение, что её можно использовать для решения данной задачи.
Более того, известно, что некоторые участки обладают достаточно консервативной вторичной структурой.
Напрмиер, центральный домен 16s~\cite{!!!}.
Использование вторичной структуры при решении задач поиска и классификации цепочек не является новой идеей.
Она используется как в широко распространённых инструментах, типа Infernal~\cite{!!!}, так и во многих теоритических исследованиях~\cite{!!!,!!!,!!!}.

Вторичная структура цепочек может быть описана с помощью грамматик. 
В зависимости от требуемой точности могут использоваться разные классы граммтик: регулярные, контекстно-свободные, конъюнктивные.
В данной работе будут использоваться контекстно-свободные грамматики и соответствующий класс языков.
В результате, по аналогии с поиском по регулярному выражению, задача поиска сведётся к поиску структурного шаблона, заданного контекстно-свободной граммтикой.
Так как вторичная структура больших цепочек может быть достаточно сложной, потому соответствующая грамматика также оказывается сложной.
При этом часто необходимо искать баланс между сложностью граммтики, которая непосредственно связана с детализацией описания вторичной структуры, и производительностью и точностью поиска.

Одна из маркерных цепочек, часто используемая в настоящее время --- это 16s rRNA.
Данный отчёт описывает эксперимент по распознованию 16s с исподльзованием информации только о её вторичной структуре, описанной контекстно-свободной граммтикой.

\section{Описание вторичной структуры спомощью грамматики}

Для спецификации грамматики был исользован язык YARD~\cite{!!!}, основанный на ECFG~\cite{!!!} с различнвми расширениями.
В правых частях можно использовать конструкции регулярных выражений.

В таблице~\ref{tbl1} представлены основные конструкции языка описания граммтики и примитивы, использовавшиеся при её написании, и дано их описание.
Так как описание несовпадений в стеме в общем случае является сложной задачей (если вообще разрешимой в 
терминах контекстно-свободных граммтик), поэтому были использованы такие правила, как $stem\_e1{<}s{>}$, описывающие приближение множества стемов с несовпадениями.

\begin{table}[h]
    \centering
    \renewcommand{\arraystretch}{1.5}
    \begin{tabular}{|c|>{\centering}p{9cm}|}
        \hline
        Грамматическая конструкция & Описание 
        \tabularnewline \hline
        $ any $ & Один из нуклеотидов $A, U, C, G$. 
        \tabularnewline \hline
        $ any^*[n..m] $ & Цепочка нуклеотидов длины от $n$ до $m$. 
        \tabularnewline \hline
        $stemN{<}s{>}$  & Стем высоты $N$ со свободной частью $s$ (последовательность любых конструкций грамматики). 
        \tabularnewline \hline
        $mk\_stem{<}s{>}$ & Стем произвольной высоты (от $0$ до $N$) со свободной частью $s$.
        \tabularnewline \hline
        $stem\_e1{<}s{>}$ & Стем позволяющий одно несовпадение, при этом требующий, чтобы подрят было не менее двух парных элементов. 
        \tabularnewline \hline
    \end{tabular}    
    \caption{Базовые конструкции грамматики}
    \label{tbl1}
\end{table}

\begin{table}[h]
    \centering
    \renewcommand{\arraystretch}{2}
    \begin{tabular}{c | c}
        $stem4{<}any^*[3..5]{>}$ & $mk\_stem{<} any^*[1..2] \ stem2{<} any^*[3..4] {>} \ any^*[2..5] {>}$ \\
        \includegraphics[width=1.5cm]{stem4.pdf} & \includegraphics[width=2cm]{mk_stem.pdf} \\
    \end{tabular}
    \caption{Примеры описания структур}
\end{table}

\section{Эксперименты}

Для проведения эксперимента прежде всего необходима контекстно-свободная грамматика, описывающая вторичную структуру искомой цепочки.
Построение грамматики, задающей вторичную структуру в настоящий момент выполняется вручную, однако возможен и вывод грамматики, но это тема для отдельного исследования.

Построенная нами, используемая в эксперименте грамматика приведена в приложении \ref{grammar}.
В качестве образца была использована эталонная вторичная структура 16s E.Coli.
Терминальный алфавит состоит из четырех символов-нуклеотидов: $A, U, C, G$.
Для спецификации стемов активно используются параметризуемые правила, что позволяет сделать граммтику достаточно компактной.
Ключевое слово \verb|inline| является служебным и используется для того, чтобы подсказать генератору синтаксических анализаторов, как именно преобразовывать граммитку в ходе работы.

Всего было поставлено два эксперимента: обработка базы известных последовательностей 16s и поиск 16s в полноразмерных размеченных геномах.
Целью первого эксперимента была оценка качества составленной нами граммитики на предмет полноты детектирования цепочек: вычислялся процент нераспознанных цепочек относительно всех обработанных.
Цель второго --- оценка количесва ложных срабатываний. Вместе с этим оценивалась и полнота поиска.

Для первого эксеримента была использована база цепочек проекта Silvia~\cite{!!!}, которая содержит более 20 тысяч различных последовательностей.
Результаты данного эксперимента приведены в таблице~\ref{16sBase}, где + --- количество распознанных цепочек для соответствующей домене, - --- количество нераспознанных.
В итоге, при использовании граммтики для центрального домена распознанно 98.16\% цепочек, являющихся 16s бактерий, а при спользовании граммтики для 5'M домена --- 63.13\%. 
Кроме этого, можно заметить, что 5'M домен лучше разделяет домены.

\begin{table}[h]
    \centering
    \begin{tabular}{c >{\centering}p{3cm} *{6}{C}}
        \toprule
        \multirow{2}{*}{Домен} & \multirow{2}{*}{\parbox{3cm}{\centering Стартовый нетерминал}} & \multicolumn{2}{c}{Бактерии} & \multicolumn{2}{c}{Эукариоты} & \multicolumn{2}{c}{Археи} \\
        \cmidrule(lr){3-4}
        \cmidrule(lr){5-6}
        \cmidrule(lr){7-8}
        & & + & - & + & - & + & - \\
        \midrule
        Центральный & h19 & 17878 & 335 & 2153 & 3165 & 306 & 13 \\
        5'M & h3 & 11498 & 6715 & 64 & 5254 & 81 & 238\\
        \bottomrule
    \end{tabular}
    \caption{Результаты анализа базы организмов}
    \label{16sBase}
\end{table}

Для второго эксперимента использовались 100 размеченных геномов из базы NCBI.
Поиск по центральному домену показал очень большое количество ложных срабатываний, поэтому резуьтаты здесь приводиться не будут.
Часть результатов поиска поиска по 5'M домену приведены в таблице~\ref{!!!}. 
В таблице указан код организма в NCBI, его название, количество размеченных и правильно обнаруженных последовательностей 16s (Expected и Covered, соответственно), количество ложных срабатываний (FP-intervals), средняя длина и отклонение для цепочек, соответствующих ложным срабатваниям.
Полседние два столбца могут помочь оценить объём работы по дополнительной фильтрации ложных срабатываний.
В итоге, для 100 геномов получены следующие результаты:
\begin{itemize}
\item среднее количество ложных срабатываний на геном равно 299, отклонение равно 221,54;
\item средняя доля правильно обнаруженных цепочек равна 0,98 при отклонении 0,11.
\end{itemize}

Также, в таблице~\ref{!!!} приведены результаты поиска по совмещённой граммтике для 5'M и центрального домена.
Обращает на себя резкое снижение количества ложных срабатваний.
При этом, однако, уменьшвется и количество положительных совпадений.


\afterpage{
\begin{landscape}
\csvloop{
    file=final_report_middle.csv,
    no head,              % no special treatment of first line
    column count=7,       % since no first line is given, tell about column count
    respect all,
    before reading=
        \begin{table}[p]\centering\footnotesize
        \begin{tabular}{llccccc}\toprule,
    command=\csvlinetotablerow,
    late after line=\\,
    late after first line=\\\midrule,
    late after last line=\\\bottomrule,
    after reading=
        \end{tabular}
        \caption{Результаты анализа полноразмерных геномов (центральный домен)}
        \end{table}
}
\end{landscape}
}

\afterpage{
    \begin{landscape}
        \csvloop{
            file=final_report_head.csv,
            no head,              % no special treatment of first line
            column count=7,       % since no first line is given, tell about column count
            respect all,
            before reading=
            \begin{table}[p]\centering\footnotesize
                \begin{tabular}{llccccc}\toprule,
                    command=\csvlinetotablerow,
                    late after line=\\,
                    late after first line=\\\midrule,
                    late after last line=\\\bottomrule,
                    after reading=
                \end{tabular}
                \caption{Результаты анализа полноразмерных геномов (5'M домен)}
            \end{table}
        }
    \end{landscape}
}

\pagebreak  


\section{Заключение}

Данный эксперимент показал наличие возможности использовать вторичную структуру цепочки для предварительной фильтрации кандидатов, однако требуется существенная доработка используемых алгоритмов.
В дальнейшем планируется выполнение работ по следующим направлениям.
\begin{itemize}
\item Поиск, разработка и применение алгоритмов автоматический вывод грамматик для построения граммтики, описывающей вторичную структуру цепочки.
\item Анализ ложных срабатываний и пропущенных кандидатов, с целью выявить их особенности, и доработать соответствующим образом алгоритм поиска.
\item Разработка методов ументшения количества ложных срабатываний.
\end{itemize}

\begin{appendices}
\section{Грамматика 16S на языке YARD, использовавшаяся в эксперименте}
\label{grammar}
\verbatiminput{16S_grammar.yrd}
\end{appendices}
\end{document}