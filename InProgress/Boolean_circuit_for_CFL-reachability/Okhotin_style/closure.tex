\section{Closure properties of languages with polynomial rational indices}
\label{sec:closure}
Given a context-free language $L$ having polynomial rational index it is interesting to find which language operations preserve this property.  Boasson et al. \cite{RatBasic} give following useful relations for polynomial indices of two languages $L$ and $L'$:
\begin{itemize}
\item $\rho_{L \cup L'} \le  \max{(\rho_L, \rho_{L'})} $
\item $\rho_{LL'} \le \rho_L + \rho_{L'}$
\item $\rho_{L \cap R}(n) \le \rho_L(nm)$, where $R$ is a regular language recognised by an $m$ state automaton
\item $\rho_{h(L)}(n) \le \rho_L(n)$ and $\rho_{h^{-1}(L)}(n) < n(\rho_L(n) +1)$, where $h: \Sigma^* \rightarrow \Delta^*$ is a homomorphism.
\end{itemize}
\begin{theorem}
Context-free languages with polynomial rational indices are closed under intersection and quotient with a regular language, union, reversal, substitution, concatenation, Kleene plus, prefix and  insertion of a regular language.
\end{theorem}
\begin{proof}

At first, from the relations above it is easy observable that the family of context-free languages with polynomial rational indices is a full trio. Every full trio is closed under prefix and quotient with regular languages. Obviously, CFLs with polynomial rational indices languages are closed under reversal. 
\\
Next we show that context-free languages with polynomial rational indices are closed under Kleene plus (we consider this variation on the Kleene star operation because empty strings are omitted in the definition of CFL-reachability problem).
\\ 
\textit{Kleene star.} Let $L$ be a language with polynomial rational index. Consider language $L^{+}$ with the start nonterminal $S$. By definition of the Kleene plus operation, a rightmost derivation from $S$ generates a sequence of one or more start nonterminals $A$ from $L_{G_{A}}$, each of which generates some string in $L$. Let $D$ be a directed labelled graph with $n$ nodes. Suppose there are nodes $u$ and $v$ in $D$ such that:
\begin{enumerate}
\item $v$ is not $L_{G_{A}}$-reachable from $u$ and
\item $v$ is  $L^+$-reachable from $u$
\end{enumerate}
Then $v$ is reachable from $u$ via concatenation of words in $L_{G_{A}}$. Consider the longest shortest path $u\pi_1 v$ between $u$ and $v$. It can be obtained by joining $(A, u, i), (A, i, j), ..., (A, w, v)$ into $(S, u, v)$.  If $L$ has the polynomial rational index, then for every triple $(A, i, j)$ corresponding shortest path $i \pi j$ has at most polynomial length. There are no more than $O(n)$ such triples in concatenation because there are no repetitions of the same node in the sequence of start and end nodes of triples (otherwise $u\pi_1 v$ is not the shortest path, for example, path $u \rightarrow i \rightarrow k \rightarrow l \rightarrow i \rightarrow j \rightarrow v$ can be replaced with shorter path $u \rightarrow i  \rightarrow j \rightarrow v$), so $u\pi_1 v$ has at most polynomial length. In other words, we have $\rho_{L^+}(n) \le n(\rho_L(n))$.
\\
Family of languages with polynomial rational indices is a full trio closed under union, concatenation and Kleene star, therefore it is a full ALF. Full AFLs is known to be closed under substitution.
\\
\textit{Insertion of a regular language.} To prove closure under insertion of a regular language, the following PDA can be constructed. Let $L$ be a context-free language with polynomial rational index and let $M$ be a PDA recognising $L$. New PDA $M'$ for insertion of a regular language $R$ recognised by finite automation $F$ into $L$ can be obtained as follows: duplicate all states in $M$, initial state is placed in first set and final states reside in the second set. Every state in the first set has its own copy of outgoing arcs of the initial state of $F$. Every ingoing arc of final state of $F$ is connected directly to every state of the second set of $M$. In other words, every state from the first set is the inial state of $F$ and every state from the second set is a final state of $F$. All acs from $F$ are labelled the same way as they are labelled in $F$.
Consider intersection of $M'$ and arbitrary finite automotion $F'$ with $n$ states (to deal with $\varepsilon$-labelled arcs, loops with $\varepsilon$ can be added to every state of finite automotion). The longest non-empty shortest path $i \pi j$ in the intersection of $M'$ and $F'$ consists of three sub-paths: $i\pi_1k$, $k\pi_2m$ and $m\pi_3j$, where $i\pi_1k$ ($m\pi_3j$) is the shortest paths in the intersection of $F'$ and first (second) set of states of $M$ respectively, and $k\pi_2m$ is the shortest path in the intersection of $F'$ and $F$. $k\pi_2m$ has at most polynomial length because all regular languges have polynomial fringe property, $i\pi_1k$ and $m\pi_3j$ have polynomial length because $M$ is a PDA for language with rational index, therefore  $i \pi j$ has polynomial length.
\end{proof}


Using closure properties, it is easier to find new subclasses of context-free languages for which CFL-reachability problem is in NC.
\begin{example}[Metalinear languages.]
\\
Let $G = (\Sigma, N, P, S)$ be a context-free grammar. $G$ is \textit{meatalinear} if all productions of $P$ are of the following forms:
\begin{enumerate}
\item $S \rightarrow A_1A_2...A_k$, where $A_i \in N - \{S\}$
\item $A \rightarrow u$, where $A \in N - \{S\}$ and $u \in (\Sigma^*((N-\{S\}) \cup {\varepsilon})\Sigma^*)$
\end{enumerate}


The width of a metalinear grammar is $max\{k$ | $S \rightarrow A_1A_2...A_k \}$. Metalinear languages of width 1 are obviously linear languages. It is easy to see that every metalinear language is a concatenation of $k$ linear languages. Linear languages have polynomial rational index,  CFLs with polynomial rational index are closed under concatenation, so metalinear languages have polynomial rational index.
\end{example}

