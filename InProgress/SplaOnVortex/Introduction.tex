\section{Introduction}

Sparse linear algebra has emerged as a powerful paradigm for high-performance graph analysis.
A wide range of problems---from graph traversing to clustering---can be reduced to efficient algebraic operations over matrices and vectors.
GraphBLAS API~\cite{9460572} follows this idea and defines a standardized set of building blocks: sparse matrices and vectors, algebraic structures like monoids and semirings, and fundamental operations such as matrix-matrix multiplication.
GraphBLAS is specifically designed to serve as a foundational layer for the development of scalable, linear-algebra-based graph algorithms.


While highly tuned CPU implementations of GraphBLAS---most notably SuiteSparse:GraphBLAS\footnote{Source code of SuiteSparse:GraphBLAS on GitHub: \url{https://github.com/DrTimothyAldenDavis/GraphBLAS}}~\cite{10.1145/3577195}---deliver strong performance on multi-core systems, implementing the GraphBLAS API efficiently on general-purpose graphics processing units (GPGPUs) remains a significant challenge.
While GPGPUs is a promising platform for linear algebra based computations, they introduce well-known obstacles for sparse workloads, including irregular memory access patterns and load imbalance.
Additionally, creating generalized kernels capable of operating not only on primitive data types like floats or integers but also on user-defined custom types presents a nontrivial engineering task. 
Despite these challenges, several efforts have been made to create GPU-accelerated libraries for linear-algebra-based graph analysis, such as GraphBLAST\footnote{GraphBLAST project page: \url{https://github.com/gunrock/graphblast}}~\cite{10.1145/3466795} which uses CUDA and the portable Spla\footnote{Spla project page: \url{https://github.com/SparseLinearAlgebra/spla}} library which uses OpenCL.

RISC-V becoming popular.
Not only CPUs, but specific devices, including GPUs.
One of the actively developed RISC-V ISA based GPGPU is Vortex.
Is Vortex suitable for linear algebra based graph analysis?

We do the following contribution in this paper. 
\begin{enumerate}
    \item Port Spla on Vortex. Technical improvements of both. 
    \item Investigate scaling in simulator. For BFS and Triangle Count (TC).
    \item FPGA? Resources?
\end{enumerate}