\section{Conclusion and Future Work}

Brahma.FSharp --- a tool to create cross-platform GPGPU-enabled .NET application presented. 
We demonstrated portability of application by evaluating them on a set of platforms including RISC-V with PowerVR GPPGU and ARM with Mali GPGPU.

While work still in progress, Brahma.FSharp allows one to create linear algebra related kernels performant enough to be integrated in libraries like Math.Net Numerics to offload generic linear algebra transparently to GPGPU.
Such an integration is planned to the nearest future.

Also, within the translator improvements, it is necessary to improve performance of data transferring between managed and native memory for complex types such as discriminated unions.

While agent-based approach for communications is a native for both OpenCL and F\#, mailbox processor may not be the best choice for it especially in cases with high-frequent CPU-GPU communications.
It may be better to use more performant libraries like Hopac~\footnote{Hopac and mailbox processor performance comparison: \url{https://vasily-kirichenko.github.io/fsharpblog/actors}} or even provide light-weight wrapper for direct access to command queue for latency-critical code.

One of nontrivial problem for the future research is an automatic memory management.
For now, GPGPU-related memory should be cleaned manually, but .NET has automatic garbage collector. 
How can we offload buffers management on it with ability ti switch to manual control if required. 

%Other technical improvements: IDE support, runtime extensions, etc.

%Education. Metaprogramming, translators development, GPGPU programming, etc.