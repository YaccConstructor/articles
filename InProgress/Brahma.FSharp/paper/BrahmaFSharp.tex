\section{Brahma.FSharp}

In this section we present our platform for GPGPU programming in F\#.

Research project: SPbU and JetBrains Research.

Blah-Blah-Blah!!!!

\subsection{Architecture}

Based on F\# code quotation to OpenCL translator.

Driver is OpenCL.NET~\footnote{OpenCL.NET --- low level .NET bindings for OpenCL. Project site [visited: 20.06.2017]: \url{https://openclnet.codeplex.com/}.}

Picture.

Workflow:

Details of some blocks are described below.

\subsection{Translator}

Classical techniques: var scopes,

Structs, tuples, etc

\subsection{OpenCL specific operations}

Atomic functions.

Memory transferring.

\subsection{OpenCL type provider}

The proposed type provider generates an F# type containing static methods that are typed the same as the OpenCL C functions. The provider is parameterized with the path to the file that contains the OpenCL C code being loaded. Any number of functions may be present in the file, as well as single-line comments and C preprocessor macros. The OpenCL C source code is analyzed by the lexer and the parser. The resulting abstract syntax tree is traversed, and function signatures are extracted. The type provider uses these signatures to generate F# static methods, which are then added to the user runtime namespace.

It was taken into account that in C-like languages, including OpenCL C, arrays are commonly passed by pointers. The implemented type provider supports generating either reference types or array types for pointer types in OpenCL C sources, based on another type provider parameter.

[example of TP usage]

The Brahma.FSharp translator has been adapted to support provided functions. The translator receives the name of the OpenCL C file being loaded; the file's contents are passed as text to the OpenCL driver along with the translated F# quotation. This provides necessary functionality.
