\section{Evaluation}

In this section we provide experiments\footnote{Related sources: \url{!!!}} with Brahma.FSharp platform which are aimed to demonstrate its main features.

We evaluated Brahma.FSharp in two cases listed below and described in respective sections.
\begin{enumerate}
\item The first one is an image convolution.  to demonstrate async
\item Second one is a matrix multiplication.  to demonstrate generics, local and private memory support. To evaluate on different devices.
\end{enumerate}

On several platforms.
\begin{itemize}
  \item Intel
  \item NVIDIA
  \item ImTech, PowerVR, RISC-V
  \item Qualcomm, Mali, ARM
\end{itemize}

\subsection{Image Convolution}

Reading and writing.
F\# MailboxProcessor  used for composing of data reading, data processing on GPGPU, and data processing on CPU.

Graphics, tables.


~\cite{aleaGPUasync}

Gray scale.

Multiple GPU-s.

\subsection{Matrix Multiplication}

Classical task for GPGPU.

Several optimizations.

Generic kernels parametrized by types and operations. 

Code examples.

Sequence of optimizations inspired by !!!\footnote{\url{!!!}}.  
Not all, but memory
Square matrix. 

Flexibility. Kernels are parametrized by operations and id.
Unsafe Min-plus using max value as ID.
Matrix of 'e

More accurate min-plus using options.

\begin{listing}[h]
  \begin{minted}[linenos]{fsharp}
let mXmKernel    
   (opAdd: Quotations.Expr<'a -> 'b -> 'a>) 
   (opMult: Quotations.Expr<'e -> 'f -> 'b>) 
   (zero: Quotations.Expr<'a>) ... (* other parameters *)  =
      ... // Supplementary code
      let kernel = <@ fun 2dRange m1 m2 res ->  // Quoted code
        ... 
        let acc = %zero // Embedded identity value
        let lBuf = localArray lws // captured from context
        ... 
        acc <- (%opAdd) acc ((%opMult) x y) // Embedded operations
        ... @>
      ... // Supplementary code
  
let intArithmeticKernel = mXmKernel <@ (+) @> <@ ( * ) @> <@ 0 @>
let intMinPlusKernel = 
    mXmKernel <@ (min) @> <@ (+) @> <@ Int.MaxValue @>
  \end{minted}
  \caption{An example of masking operation definition}
  \label{lst:mXm_kernels}
\end{listing}

Graphics, tables.
