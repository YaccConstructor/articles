\section{Introduction}

Last decades utilization of GPGPUs not only in scientific or dedicated applications, but also in regular business applications becomes more popular.
In such cases not peak performance, but transparent offloading of computations to accelerator has come into focus.
As a result, respective tools for integration of GPGPUs into such platforms as JVM~\cite{rootbeer,jcuda,ScalaGPU} or .NET~\cite{FSCLPhD,aleaGPUasync} are developed.
Note that in real-world application the problem no only to offload some computations on GPGPU, but to orchestrate heterogenous asynchronous application that involves computations on possible several GPGPUs.

At the same time, utilization of existing functional languages and creation new ones for GPGPU programming, looks promising due to their safety, flexibility, ability to use advanced optimization techniques and to create high-level abstractions.
That lead to such projects as Futhark~\cite{10.1145/3140587.3062354}, Lift~\cite{10.5555/3049832.3049841}, AnyDSL~\cite{10.1145/3276489}, Accelerate~\cite{10.1145/1926354.1926358}.
 
Nowadays there are very few combination of mature business application development platform and functional programming language.
One of them is a .NET platform and F\# programming language. 
There are several tools, such as Alea.GPU~\cite{aleaGPUasync}, FCSL~\cite{FSCLPhD}, ILGPU\footnote{ILGPU project web page: \url{https://ilgpu.net/}}, that allows one to integrate GPGPUs into .NET application without using such unsafe and low-level mechanisms like string-level kernels creation. 
While FSCL and Alea.GPU use F\# to create kernels, ILGPU works on IL level that limits ability to use high-level features and nontrivial optimizations.
 
In this work we propose a \textbf{Brahma.FSharp}\footnote{
    Sources of Brahma.FSharp: \url{https://github.com/YaccConstructor/Brahma.FSharp}.
}---the tool for portable GPGPU-enabled .NET applications development that provides transparent and safe integration with accelerators---and demonstrate it's portability across variety of platforms and devices. 