\section{План лекций}
\begin{enumerate}
  \item Введение. Базовые определения. Обзор курса.
  \item Регулярные языки, конечные автоматы (детерминированные, недетерминированные), регулярные выражения. Детерминизация, $\varepsilon-$замыкание, минимизация.
  \item Взаимные преобразования способов задания.
  \item Теоретико-языковые свойства регулярных языков. Лемма о накачке, замкнутость относительно операций.
  \item Алгоритмы вычисления операций.
  \item Грамматики, переписывающие системы. КС-граммтики (обыкновенные граммтики). Вывод в граммтике, неоднозначные грамматики, существенно неоднозначные языки, дерево вывода.
  \item Лево(право)-линейные грамматики и регулярные языки.
  \item КС граммтики и КС языки. Лемма о накачке, замкнутотсть относительно операций, проверка пустоты.
  \item !!!
  \item !!!
  \item !!!
  \item !!!
  \item !!!
  \item !!!
  \item !!!
  \item !!!
\end{enumerate}
