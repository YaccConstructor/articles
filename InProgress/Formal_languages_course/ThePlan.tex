\section{План занятий}
\begin{enumerate}
  \item Введение. О чём курс: общая структура, что будет и чего не будет. Правила получения оценки за курс. Базовые определения.
  \item Иерархия Хомского. Основные классы языков. За пределами иерархии Хомского. Нестроковые языки.
  \item Задача поиска пути с ограничениями в терминах формальных языков. Варианты постановки, прикладное значение, теоретические вопросы.  
  \item Регулярные языки, конечные автоматы (детерминированные, недетерминированные), регулярные выражения. Операции над ними. Операции над автоматами как операции над их матрицами смежности. Поиск путей с регулярными ограничениями.
  \item Контекстно-свободные языки. Нормальная и ослабленная нормальная формы Хомского. Поиск путей с КС ограничениями. CYK и Hellings.
  \item Матричный алгоритм КС запросов.
  \item Тензорный алгоритм КС запросов.
  \item Дерево разбора и поиск путей. SPPF. 
  \item Синтаксический анализ языков программирования. Лексика и синтаксис. Тонкости, проблемы, инструменты.
  \item ANTLR, LL, ещё раз про неоднозначности.
  \item Семантика языков программирования. Интерпретаторы. Что делать с деревом разбора. 
  \item Атрибутные грамматики.
  \item Немного о том, что за КС тоже есть жизнь.
\end{enumerate}


Общая цель курса --- посмотреть на формальные языки с прикладной точки зрения. При этом предлагается попробовать применить их сразу в двух областях: классический синтаксический анализ языков программирования и анализ графов.

В ходе курса будет предложено разработать небольшой инструментарий для выполнения запросов к графам. Окажется, что алгоритмы для некоторых задач анализа графов непосредственно основаны на алгоритмах из теории формальных языков и синтаксического анализа. Далее, будет предложено разработать язык запросов, позволяющий использовать разработанные алгоритмы. Необходимо будет разработать сам язык, лексический и синтаксический  анализаторы для него, интерпретатор. Интерпретатор будет использовать разработанные алгоритмы выполнения запросов к графам.

Примерные темы задач с баллами.
\begin{enumerate}
  \item [5] Развернуть репозиторий. Научиться подгружать графы, запрашивать у них вершины и рёбра. Консольный клиент.
  \item [2] Регулярка в ДКА. 
  \item [2] Граф в НКА.
  \item [5] Регулярные запросы через тензорное произведение. 
  \item [11] Сравнение производительности пересечения автоматов.
  \item [2] КС граммтики. Преобразование в ОНФХ. 
  \item [5] Построение рекурсивного конечного автомата и его минимизация.
  \item [5] CTK
  \item [5] Hellings
  \item [5] Матрицы
  \item [5] Тензоры
  \item [15] Сравнение производительности КС алгоритмов.
  \item [5] Разработать конкретный синтаксис языка запросов. Документация. 
  \item [5] Реализовать его парсер.
  \item [3] Печать дерева в DOT.
  \item [20] Реализовать интерпретатор языка запросов.
\end{enumerate}