\section{Правила работы на курсе}

Коротко зафиксируем основные правила работы на курсе: из чего формируется оценка, как сдавать домашние работы и т.д.

\subsection{Оценка за курс}

Оценка за курс складывается из баллов, полученных за работу в семестре. Баллы начисляются за следующее.
\begin{itemize}
    \item За домашние работы (балл за каждую задачу указывается отдельно). При этом у каждой работы есть жёсткий дедлайн и после него балл уменьшается вдвое.
    \item За летучки (короткие, 5-10 минут, контрольные работы). Летучка оценивается от 1 до 0 баллов с шагом 0.25. Сами по себе баллы за летучки не суммируются, но служат для корректировки баллов за домашние работы.
\end{itemize}

Итоговая оценка за курс --- это взвешенная сумма баллов за задачи, где вес --- баллы за ближайшую справа летучку.
Пусть, например, в курсе было 4 задачи и 2 летучки, упорядоченны хронологически как это показано в таблице~\ref{tbl:grad_example}. 
В этой же таблице можно увидеть, какие баллы получил Эталон Примерович за задачи и летучки, а также его итоговый балл, который вычисляется следующим образом:
$$
\underbrace{(4+5)}_{\text{Первые две домашки}}*\underbrace{0.25}_{\text{Летучка 1}} + \underbrace{(2.5+3)}_{\text{Вторые две домашки}}*\underbrace{0.75}_{\text{Летучка 2}} = 6.375.
$$

\begin{table}[h]
    \caption{Пример оценок за курс}
    \label{tbl:grad_example}
\begin{center}
    \begin{tabular}{ | c | c | c | c | c | c | c | c |}
        \hline
        ФИО & ДЗ 1 & ДЗ 2 & Летучка 1 & ДЗ 3 & ДЗ 4 & Летучка 2 &  Итог \\ 
        \hline
        \hline
        Эталон Примерович & 4 & 5 & 0.25 & 2.5 & 3 & 0.75 & 6.375 \\  
        \hline
    \end{tabular}
\end{center}
\end{table}


Для трансляции баллов в оценку используется таблица~\ref{tbl:ects}.

\begin{table}[h]
    \caption{Конвертация баллов в оценки}
    \label{tbl:ects}
\begin{center}
    \begin{tabular}{ | c | c | c |}
        \hline
        Балл & ECTS & Классика \\ 
        \hline
        \hline
        91--100 & A & 5 \\  
        81--90  & B & 4 \\  
        71--80  & C & 4 \\  
        61--70  & D & 3 \\  
        51--60  & E & 3 \\  
         0--50  & F & 2 \\   
        \hline
    \end{tabular}
\end{center}
\end{table}

Решение задач можно продолжать до тех пор, пока не будет набрано достаточно баллов для получения неотрицательной оценки (с учётом понижения баллов после дедлайнов и с учётом летучек\footnote{Поэтому летучки имеет смысл писать всегда. Они могут влияют на задачи, даже если те сданы после написания летучки.}). 
Если даже при всех решённых задачах баллов не достаточно для получения неотрицательной оценки, то выдаются дополнительные задачи\footnote{Дополнительные задачи выдаются только после решения основных.}. 
Баллы за дополнительные задачи таковы, чтобы обеспечить разве что получение минимальной положительной оценки. 

Все текущие результаты оформляются в виде таблицы на Google Drive.
Таблица доступна всем студентам на чтение.

\subsection{Домашние задачи}

Домашние задачи представляют из себя практические задачи на программирование или постановку экспериментов (сравнение и анализ производительности различных алгоритмов, решений).
Они анонсируются по ходу семестра вместе с баллами и дедлайнами.
У задачи есть два дедлайна: \textbf{мягкий} и \textbf{жёсткий}.
\textbf{Мягкий} наступает через 4 дня после выдачи задачи (если не оговорено иное) и нужен для того, чтобы гарантировать своевременную проверку задачи. 
Задача, ревью на которую запрошено до мягкого дедлайна гарантированно будет проверена с возможностью исправить замечания до наступления жёсткого дедлайна. Если задача сдана после мягкого дедлайна, то своевременность проверки уже не гарантируется. То есть возможны следующие ситуации/
\begin{itemize}
    \item Задача сразу решена правильно и тогда не зависимо от времени проверки за неё даётся полный балл
    \item Задача решена не правильно но проверена и исправлена до жёсткого дедлайна. Тогда получается полный балл. Но, как было сказано выше, нет гарантии, что хватит времени на исправление.
    \item Задача решена не правильно и это выяснилось после жёсткого дедлайна: тогда за неё уже нельзя получить полный балл.
\end{itemize}
\textbf{Жёсткий} дедлайн не может наступить раньше, чем через 6 дней с момента анонса задачи, как правило равен неделе, но может быть и больше. После жёсткого дедлайна баллы разово снижаются в два раза\footnote{То есть задача, сданная в любой момент после жёсткого дедлайна стоит в два раза меньше, чем сданная вовремя. Судя по всему, Эталон Примерович из таблицы~\ref{tbl:grad_example} сдал задачу 3 после жёсткого дедлайна.}. Любая задача может быть либо зачтена полностью, либо не зачтена вообще\footnote{То есть частично решённых задач быть не может. А значит, и неполных баллов.}. Полный балл за все задачи --- 100. 
Каждый раз задаётся (и, соответственно, сдаётся) одна задача\footnote{Таким образом, идеальный расклад --- одна задача в неделю. На некоторые задачи, правда, может выделяться больше времени.}. 

Язык программирования --- Python. Набор инструментов и структура репозитория для домашних задач фиксированы и представлены здесь: \url{https://github.com/JetBrains-Research/formal-lang-course}. 
Набор библиотек следующий.
\begin{itemize}
    \item cfpq\_data для работы со входными данными (графами).
    \item sciPy для базовых операций с матрицами.
    \item pygraphblas для оптимизированных операций с матрицами на CPU.
    \item pyspbla (pycubool?) для булевых матриц на GPU.
    \item pyformlang для работы с языками (регулярные выражения, конечные автоматы, КС грамматики).
    \item ANTLR для синтаксического анализа.
    \item pyTest для организации тестов.
    \item pyDOT для визуализации графов.
\end{itemize}

Так как часть задач будет состоять в постановке экспериментов, то для них требуется оформление отчёта по поставленным экспериментам. 
Отчёт оформляется как Python notebook в Google Colab, который содержит как код для постановки экспериментов (чтобы было видно, что и как замеряли), так и весь сопроводительный текст, графики и т.д.

Качество оформления отчёта влияет на то, будет ли зачтена задача. То есть, даже если всё замерено грамотно, но отчёт оформлен небрежно, то задача может быть не зачтена.

Задачи сдаются через GitHub. Процедура выглядит следующим образом.
\begin{enumerate}
    \item Студент создаёт репозиторий, в который приглашает ассистентов, помогающих с проверкой домашних заданий (анонсируются на первой паре). Запись об этом репозитории добавляется в таблицу с результатами. 
    \item Студенты распределяются преподавателем среди ассистентов.
    \item Репозиторий снабжается readme, системой автоматической сборки, системой автоматического тестирования, проверкой качества кода. Для упрощения этого шага необходимо пользоваться предоставленным шаблоном: достаточно сделать его fork. 
    \item Новая задача решается в отдельной ветке. Если это задача на постановку экспериментов, то ссылка на notebook добавляется в readme репозитория. При решении задачи необходимо следовать структуре, заданной в репозитории-шаблоне. Задача снабжается тестами. Качество тестов --- обязанность студента. Задача может быть не принята из-за плохого тестового покрытия. После того, как все автоматические проверки пройдены, открывается реквест в основную ветку. Если и реквест прошёл все автоматические проверки, то запрашивается ревью у соответствующего ассистента\footnote{Запрос ревью --- есть попытка сдачи, которая регламентируется дедлайнами. Запрашивать ревью можно только на полностью решённую задачу. Нельзя запрашивать ревью на частичные решения.}. 
    \item Ассистент выполняет проверку, оставляет комментарии, выносит вердикт. Если задача зачтена, то реквест мёржится и закрывается студентом, а балл за задачу заносится в таблицу ассистентом. Иначе вносятся исправления, добавляются к этому же реквесту\footnote{Реквест не закрывать. Это позволяет отслеживать историю замечаний.}, заново запрашивается ревью\footnote{Запрашивать ревью обязательно, так как каждый коммит проверять никто не будет.}. и так до тех пор, пока задача не будет зачтена\footnote{Ну, или до тех пор, пока не отпадёт необходимость её решать.}.
\end{enumerate}

Необходимо иметь ввиду, что многие задачи связаны между собой (следующая использует результаты предыдущих), поэтому не всегда возможно безболезненно не делать какую-то одну задачу из середины списка.

\subsection{Летучки}

Летучка --- небольшая, на 5--10 минут, проверочная работа, проводимая в начале пары. 
Летучка проверяет базовые знания: определения, теоремы, алгоритмы, свойства алгоритмов. 
Она содержит одно задание, которое может быть как теоретическим (написать определение чего либо), так и практическим (показать шаги какого-то алгоритма для заданного входа).

Летучка оценивается от 0 до 1 балла с шагом в 0.25 и используется как вес для домашних работ. Оценки летучек не обсуждаются и не корректируются. Летучки не переписываются.

Точная дата летучки заранее не анонсируется, однако их примерное местоположение может быть известно заранее, так как обычно они привязываются к тем или иным блокам материала.

Точное количество летучек также заранее не известно, но обычно это 4--5. Так как могут быть уважительные причины пропустить летучку, то в конце семестра за две случайно выбранные летучки с нулём баллов выставляется 0.75 баллов\footnote{Просто чтобы написать самому было лучше, чем прогулять и понадеяться на случайность.}. 

При занятиях в аудитории летучка пишется на листке бумаги, на котором, кроме решения задач, указывается ФИО студента и вариант (который назначается преподавателем). Листочки сдаются строго по истечению отведённого времени.

При занятии в удалённом формате всё происходит точно также, только сдаётся фотография (скан) листочка. Способ сдачи указывается преподавателем перед началом летучки. Время необходимое на отправку фото входит во время всей летучки\footnote{Так что будьте осторожны. Если на летучку дали 10 минут, то через 10 минут фото листочка должно быть у преподавателя. А не через 10 минут срочно начнётся поиск внезапно пропавшего телефона и т.д.}.

Примеры вопросов на летучку.
\begin{enumerate}
    \item Какова сложность алгоритма Хеллингса относительно размера графа?
    \item В чём отличие НФХ от ОНФХ?
    \item Дайте определение рекурсивного конечного автомата.
    \item Нарисуйте начальное состояние таблицы CYK для данного входа и грамматики.
    \item Принимается ли данная цепочка данным автоматом? Если да, то предъявите соответствующую последовательность переходов.
    \item Какова сложность тензорного произведения относительно размера исходных матриц?
\end{enumerate}