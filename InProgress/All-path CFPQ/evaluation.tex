\section{Evaluation}
The goal of this evaluation is to investigate the applicability of the proposed matrix-based algorithm to CFPQ with all-path query semantics and to provide a comparison of the most performant linear algebra-based CFPQ algorithms. We will compare the following CFPQ implementations:
\begin{itemize}
	\item $MtxSingle$ --- the implementation from~\cite{10.1145/3398682.3399163} of the matrix-based CFPQ algorithm for the single-path query semantics,
	\item $Tns$ --- the implementation from~\cite{kron} of the Kronecker product-based CFPQ algorithm for all three query semantics including the all-path query semantics,
	\item $MtxAll$ --- the implementation of the proposed matrix-based CFPQ algorithm for all-path query semantics which utilizes SuiteSparse\footnote{SuiteSparse is a sparse matrix software that includes GraphBLAS API implementation. Project web page: \url{http://faculty.cse.tamu.edu/davis/suitesparse.html}. Access date: 20.03.2021.}~\cite{Davis2018Algorithm9S} implementation of GraphBLAS API for matrix manipulations.
\end{itemize}
{\setlength{\tabcolsep}{0.25em}
	\begin{table*}[t]
		{
			\caption{Index creation time in seconds and memory in megabytes where we use "$err$" in case of out of memory error }
			\label{tbl:index_creation}
			\small
			\rowcolors{4}{black!2}{black!10}
			\begin{tabular}{|l|l|l|l|l|l|l|l|l|l|l|l|l|l|l|l|l|l|l|}
				\hline
				\multicolumn{1}{|c|}{\multirow{3}{*}{Graph}} & \multicolumn{1}{c|}{\multirow{3}{*}{\#V}} & \multicolumn{1}{c|}{\multirow{3}{*}{\#E}} & \multicolumn{8}{c|}{G1}                                                                                               & \multicolumn{8}{c|}{G2}                                                                                               \\ \cline{4-19} 
				\multicolumn{1}{|c|}{}                       & \multicolumn{1}{c|}{}                     & \multicolumn{1}{c|}{}                     & \multicolumn{2}{c|}{MtxAll} & \multicolumn{2}{c|}{Tns} & \multicolumn{2}{c|}{MtxSingle} & \multicolumn{2}{c|}{MtxRel} & \multicolumn{2}{c|}{MtxAll} & \multicolumn{2}{c|}{Tns} & \multicolumn{2}{c|}{MtxSingle} & \multicolumn{2}{c|}{MtxRel} \\ \cline{4-19} 
				\multicolumn{1}{|c|}{}                       & \multicolumn{1}{c|}{}                     & \multicolumn{1}{c|}{}                     & Time         & Mem          & Time        & Mem        & Time           & Mem           & Time         & Mem          & Time         & Mem          & Time        & Mem        & Time           & Mem           & Time         & Mem          \\ \hline
				pathways                                     & 6 238                                     & 18 598                                    & 0.04         & 91           & 0.02        & 123        & 0.01           & 671           & 0.01         & 140          & 0.01         & 49           & 0.01        & 122        & 0.01           & 671           & 0.01         & 140          \\ \hline
				go-hierarchy                                 & 45 007                                    & 980 218                                   & 22.12        & 38797        & 0.17        & 265        & 1.41           & 660           & 0.08         & 254          & 15.66        & 28447        & 0.24        & 252        & 0.84           & 671           & 0.09         & 255          \\ \hline
				enzyme                                       & 48 815                                    & 109 695                                   & 0.4          & 307          & 0.04        & 137        & 0.01           & 216           & 0.01         & 181          & 0.02         & 61           & 0.02        & 132        & 0.01           & 217           & 0.01         & 181          \\ \hline
				eclass\_514en                                & 239 111                                   & 523 727                                   & 25.02        & 14416        & 0.24        & 205        & 0.23           & 216           & 0.07         & 180          & 0.22         & 126          & 0.27        & 193        & 0.16           & 216           & 0.06         & 181          \\ \hline
				go                                           & 272 770                                   & 534 311                                   & 11.8         & 8290         & 1.58        & 282        & 1.45           & 215           & 1.00         & 244          & 1.13         & 990          & 1.27        & 243        & 0.93           & 217           & 0.94         & 246          \\ \hline
				geospecies                                   & 450 609                                   & 2 311 461                                 & 4.45         & 2691         & 0.08        & 218        & 0.06           & 2250          & 0.02         & 3069         & 0.34         & 156          & 0.01        & 196        & 0.01           & 2251          & 0.01         & 3069         \\ \hline
				taxonomy                                     & 5 728 398                                 & 14 922 125                                & $err$        & $err$        & 4.42        & 2018       & 2.73           & 1962          & 1.13         & 965          & 19.13        & 27232        & 3.56        & 1776       & 1.15           & 2250          & 0.72         & 1175         \\ \hline
			\end{tabular}
		}
	\end{table*}
}


{\setlength{\tabcolsep}{0.25em}
	\begin{table}
		{
			\caption{Index creation time in seconds and memory in megabytes for $geo$ query}
			\label{tbl:index_creation_geo}
			\small
			\rowcolors{4}{black!2}{black!10}
				\begin{tabular}{|c|l|l|l|l|l|l|l|l|}
					\hline
					\multirow{2}{*}{Graph}           & \multicolumn{2}{c|}{MtxAll} & \multicolumn{2}{c|}{Tns} & \multicolumn{2}{c|}{MtxSingle} & \multicolumn{2}{c|}{MtxRel} \\ \cline{2-9} 
					& Time         & Mem          & Time        & Mem        & Time           & Mem           & Time         & Mem          \\ \hline
					\multicolumn{1}{|l|}{geospecies} & 32.06        & 44235        & 26.32       & 19537      & 15.54          & 22941         & 7.48         & 7645         \\ \hline
				\end{tabular}
		}
	\end{table}
}


All implementations utilize CPU and use matrices in sparse format. First, we measured the execution time and required memory of the index creation. Then we compared the practical applicability of the paths extraction for both implementations $MtxAll$ and $Tns$ of the CFPQ with all-path query semantics. The source code is available on GitHub\footnote{Sources of all CFPQ implementations: \url{https://github.com/JetBrains-Research/CFPQ\_PyAlgo}. Access date: 20.03.2021.}.

For evaluation, we used a PC with Ubuntu 18.04 installed.
It has Intel core i7-6700 CPU, 3.4GHz, and DDR4 64Gb RAM.
We only measure the execution time of the algorithms themselves, thus we assume an input graph is loaded into RAM in the form of its adjacency matrix in the sparse format.

\subsection{Dataset Description}

We use the graphs and respective queries from the CFPQ\_Data dataset\footnote{CFPQ\_Data dataset GitHub repository: \url{https://github.com/JetBrains-Research/CFPQ_Data}. Access date: 20.03.2021.} provided in~\cite{10.1145/3398682.3399163} that contains the real-world RDFs and queries $g_1, g_2, geo$ 
that are variations of the \textit{same-generation query}~\cite{FndDB} --- an important example of real-world queries that are context-free but not regular.




\subsection{Evaluation Results}
The results of the index creation for all three implementations are presented in Tables~\ref{tbl:index_creation} and \ref{tbl:index_creation_geo}. We can see that the most performant index creation is in $MtxSingle$ implementation, especially on big graphs. But $MtxSingle$ applicable only for single-path query semantics  and cannot restore all paths of interest. Thus, for single-path query semantics we can use a more simple index to restore only one path for each vertex pair. However, the Kronecker product-based implementation $Tns$ uses a more complex but compact index and consumes less memory. The implementation $MtxAll$ of the proposed matrix-based CFPQ algorithm for all-path query semantics has comparable to $Tns$ execution time on small graphs but significantly slower execution time on big graphs with complex structure. Also, the $MtxAll$ 
consumes significantly more memory than $Tns$. The reason for such behavior is that the proposed matrix-based algorithm is trying to store the information of all founded paths more explicitly. Also, the index constructed by $MtxAll$ is less compact than the one constructed by $Tns$. On the biggest $taxonomy$ graph and query $g_1$ we even have the out of memory error for $MtxAll$ implementation.

After constructing the index, we compared the execution time of the path extraction for CFPQ with all-path query semantics using both $MtxAll$ and $Tns$ implementations. The results of path extraction for graphs $go$ and $eclass\_514en$ are presented in Figures~\ref{fig:extractTimeTns} and \ref{fig:extractTimeMtx} (boxplots are standard, medians are indicated and outliers are omitted). For computation termination, we limit the maximum path length to 10. After that, we extract paths for each vertex pair and group the execution time by the number of paths returned. We can see that the path extraction running time of the implementation $MtxAll$ of the proposed matrix-based algorithm is up to 1000 times faster than for the Kronecker product-based implementation $Tns$. As was mentioned above, in the proposed matrix-based CFPQ algorithm we construct an index with more explicit information about all founded paths. Thus, paths can be restored significantly faster than using the Kronecker product-based algorithm.

We can conclude the following.

\begin{itemize}
	\item 
	Our evaluation together with the evaluation from~\cite{10.1145/3398682.3399163, kron} allow us to conclude that the most performant algorithm for the CFPQ with relational query semantics when the paths extraction is not required is Azimov's matrix-based algorithm from~\cite{Azimov:2018:CPQ:3210259.3210264}.
	\item For single-path query semantics when only one path per each vertex pair is required, the modification of Azimov's matrix-based CFPQ algorithm~\cite{10.1145/3398682.3399163} is most performant.
	\item For all-path query semantics, the proposed matrix-based and the Kronecker product-based CFPQ algorithms have the following tradeoffs. If it is necessary to frequently recalculate the index for a changing graph or a path query then the best choice is the Kronecker product-based algorithm~\cite{kron} with faster and less memory consuming index construction. If it is necessary to extract paths many times for once constructed index then the proposed matrix-based CFPQ algorithm is preferable.
\end{itemize}


\begin{figure*}
	\begin{subfigure}{0.32\textwidth}
		\includegraphics[width=\linewidth,trim=0 0 -1.5cm 0]{pictures/tensor_eclass_514en_10_small.pdf}
		\caption{$eclass\_514en$ and $g_1$} \label{fig:extractTimeEclassTns}
	\end{subfigure}
	\hspace*{\fill} % separation between the subfigures
	\begin{subfigure}{0.32\textwidth}
		\includegraphics[width=\linewidth,trim=0 0 -1.5cm 0]{pictures/tensor_go_10_small.pdf}
		\caption{$go$ and $g_1$ for small number of paths} \label{fig:extractTimeGoSmallTns}
	\end{subfigure}
	\hspace*{\fill} % separation between the subfigures
	\begin{subfigure}{0.32\textwidth}
		\includegraphics[width=\linewidth,trim=0 0 -1.5cm 0]{pictures/tensor_go_10_big.pdf}
		\caption{$go$ and $g_1$ for big number of paths} \label{fig:extractTimeGoBigTns}
	\end{subfigure}
	\caption{Execution time of the Kronecker product-based path extraction algorithm from~\cite{kron} implemented in $Tns$ depending on the number of paths returned}
	\label{fig:extractTimeTns}
\end{figure*}

\begin{figure*}
	\begin{subfigure}{0.32\textwidth}
		\includegraphics[width=\linewidth,trim=0 0 -1.5cm 0]{pictures/allMatr_eclass_514en_10_small.pdf}
		\caption{$eclass\_514en$ and $g_1$} \label{fig:extractTimeEclassMtx}
	\end{subfigure}
	\hspace*{\fill} % separation between the subfigures
	\begin{subfigure}{0.32\textwidth}
		\includegraphics[width=\linewidth,trim=0 0 -1.5cm 0]{pictures/allMatr_go_10_small.pdf}
		\caption{$go$ and $g_1$ for small number of paths} \label{fig:extractTimeGoSmallMtx}
	\end{subfigure}
	\hspace*{\fill} % separation between the subfigures
	\begin{subfigure}{0.32\textwidth}
		\includegraphics[width=\linewidth,trim=0 0 -1.5cm 0]{pictures/allMatr_go_10_big.pdf}
		\caption{$go$ and $g_1$ for big number of paths} \label{fig:extractTimeGoBigMtx}
	\end{subfigure}
	\caption{Execution time of the proposed matrix-based path extraction algorithm implemented in $MtxAll$ depending on the number of paths returned}
	\label{fig:extractTimeMtx}
\end{figure*}
