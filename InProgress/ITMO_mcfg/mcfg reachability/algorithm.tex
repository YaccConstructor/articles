\section{Matrix-based MCFL-reachability algorithm}
\label{sec:all-path-algo}
In this section, we introduce the matrix-based algorithm for the MCFL-reachability problem.

\subsection{Algorithm}
\begin{definition}
\label{def:sets-end-alter}
Let $G = (N, \Sigma, P, S, d)$ be an MCFG. Then $\forall p: A \rightarrow f(B,C) \in P$ we define:
\begin{itemize}
    \item $end\_B(p) = \{2(i - 1)|$ iff $B^i$ is the leftmost in the component of rhs of $p\} \cup \{2(i - 1) + 1 |$ iff $B^i$ is the rightmost in the component of rhs of $p\}$, each of the sets is considered ordered by the components of the $A$ and inside the component from left to right;
    \item $end\_C(p)$ defined similarly with $end\_B(p)$ but with an offset of $2d(B)$, that is, $end\_C(p)$ will be of the form $2(i - 1) + 2d(B)$ and $2(i - 1) + 1 + 2d(B)$;
    \item $end\_A$ is ordered set of pairs where the first element of the pair corresponds to the leftmost nonterminal of some component in the rule and the second element --- to the rightmost, i.e. the elements of pairs are elements of the sets $end\_B(p)$ and $end\_C(p)$ defined above;
    \item $(2(i - 1)) \in alter\_B(p)$ or $(2(i - 1) + 1) \in alter\_B(p)$ iff $(2(i - 1)) \notin end\_B(p)$ or $(2(i - 1) + 1) \notin end\_B(p)$, $\forall 1 \leq i \leq d(B)$;
    \item $alter\_C(p)$ defined similarly with $alter\_B(p)$ but with an offset of $2d(B)$.
\end{itemize}
\end{definition}

Note: $|alter\_B(p)| = |alter\_C(p)|$ for grammars in the described normal form.

Let $G = (N, \Sigma, P, S, d)$ be an MCFG, $D = (V,E, \Sigma)$ --- labeled directed graph, where $|V| = n$. The proposed algorithm is presented in Listing~\ref{lst:algo1}. The $MCFL\_rechability$ procedure takes as input a graph and a multiple context-free grammar in \emph{normal form}. 

At the first stage, the algorithm processes rules where there are only terminals on their rhs. Thus, the algorithm restores the paths that can be obtained in one application of the rule. The $update$ procedure is presented in Listing~\ref{lst:algo2}. It is used to update all necessary matrices for rules with nonterminal $B$ in rhs with a new value according to the new paths found. In the update procedure, only the index of the value is recalculated, taking into account the sets $end\_B(p)$, $end\_C(p)$, $alter\_B(p)$ and $alter\_C(p)$ and the value $True$ is added according to the calculated index.

In line 6 of Listing~\ref{lst:algo1} the $d(A)$ paths of length 1 or 0 corresponding to the components of the rule $A \rightarrow (a_1, \dots, a_{d(A)})$ are found. That is, each $(l_i, r_i)$ is a pair of vertices between which there is a path derived from the $i$-component of the rule. We note two facts about this index. First of all, there are $d(A)$ pairs in this index, that is, the number of elements in it is even. Second, such index can be encoded as a $(n+1)-ary$ number ($n$ --- number of vertices in the graph $D$). The second fact allows us to use the $FromIndex$ algorithm ($ToIndex$ inverse to it) to convert such a number to the $(n+1)-ary$ numeral system. The parity of the number of elements allows us to divide the index in half and write the first part in the row number of the matrix and the second part in the column number. Thus, we write the fact of the restored paths for the nonterminal into a square Boolean matrix by dividing the index in half and translating each part into the desired numeral system (let these numbers be $i$ and $j$), and then put $True$ value in the cell $(i,j)$.



\begin{algorithm}[H]
\floatname{algorithm}{Listing}
\footnotesize
\begin{algorithmic}[1]
\caption{MCFL-reachability algorithm}
\label{lst:algo1}
\State{$G = (N, \Sigma, P, S, d)$ --- MCFG, $D = (V,E, \Sigma)$ --- directed edge-labeled graph}
\State{$A_p, B_p, B_p\_new, C_p, C_p\_new$ $\gets$ Boolean matrices of proper size}

\Procedure{$MCFL\_reachability$}{$G$, $D$}
\For{$p \in P : p = A \rightarrow (a_1, \dots, a_{d(A)})$}
    \For{$(l_1, a_1, r_1), \dots, (l_{d(A)}, a_{d(A)}, r_{d(A)}) \in E$}
        \State{$index \gets (l_1,r_1,l_2,\dots,r_{d(A)})$}
        \State{$update(A, index)$}
        \Comment{add information about terminating rules}
    \EndFor
\EndFor

\For{$p \in P : p = A \rightarrow f(B,C)$}
    \State{$B_p\_new, C_p\_new \gets B_p, C_p$} \Comment{all values are new for the first iteration}
\EndFor

\While{matrices $B_p$, $C_p$ are changing}
    \For{$p \in P : p = A \rightarrow f(B,C)$} \Comment{consider nonterminating rules}
        \State{$A_p\_new \gets B_p\_new \times C_p + B_p \times C_p\_new$} \Comment{use only new values}
        \State{$B_p\_new, C_p\_new \gets$ empty Boolean matrices of proper size}
        
        \For{$(i,j): A_p\_new[i,j] = True \wedge A_p[i,j] = False$}
            \State{$A_p[i,j] \gets True$}
            \State{$index \gets transform\_index(ToIndex(i), ToIndex(j), p)$}
            \State{$update(A, index)$}
            \Comment{add new information for nonterminal $A$}
        \EndFor
    \EndFor
\EndWhile
\State{$Res \gets$ Boolean matrix of proper size}
\For{$p \in P : p = S \rightarrow f(B,C)$}
\Comment{collect all information for the start nonterminal $S$}
\For{$(i,j): S_p[i,j] = True$}
\State{$index \gets transform\_index(ToIndex(i), ToIndex(j), p)$}
\Comment{size of index is equal to 2 since $d(S) = 1$}
\State{$Res[index[0], index[1]] \gets True$}
\EndFor
\EndFor
\Return $Res$
\EndProcedure



\end{algorithmic}
\end{algorithm}

\begin{algorithm}[H]
\floatname{algorithm}{Listing}
\footnotesize
\begin{algorithmic}[1]
\caption{The procedure for updating matrix values}
\label{lst:algo2}

\Procedure{$update$}{$B$, $index$}
\Comment{update matrices for all rules with $B$ in the rhs}
    \For{$p \in P : p = A \rightarrow f(B,C) $}
    \State{$i_B, j_B \gets$ empty lists}
    \For{$end \in end\_B(p)$}
        \State{$i_B.append(index[end])$}
    \EndFor
            
    \For{$alter \in alter\_B(p)$}
        \State{$j_B.append(index[alter])$}
    \EndFor
    \State{$B_p[FromIndex(i_B), FromIndex(j_B)] \gets True$}
    \EndFor
    
    \For{$p: A\rightarrow f(C,B) $}
    \State{$i_B, j_B \gets$ empty lists}
    \For{$alter \in alter\_B(p)$}
        \State{$i_B.append(index[alter - 2d(C)])$}
    \EndFor

    \For{$end \in end\_B(p)$}
        \State{$j_B.append(index[end - 2d(C)])$}
    \EndFor
    \State{$B_p[FromIndex(i_B), FromIndex(j_B)] \gets True$}
    \EndFor
\EndProcedure

\end{algorithmic}
\end{algorithm}

Further, the algorithm uses five matrices for each nonterminal rule. Namely, for a rule $p$ of the form $A \rightarrow f(B, C)$, the algorithm supports the matrix $B_p$ in which the result is stored, taking into account the sets in Definition~\ref{def:sets-end-alter} for the nonterminal $B$, as well as the matrix $B_p\_new$, which stores the result, taking into account the sets, which was obtained only at the previous step. Similarly for the nonterminal $C$. Also, the information about found paths corresponding to this rule is stored in matrix $A_p$, taking into account the set $end\_A(p)$. And after processing the terminating rules, it is necessary to add new results to the supported matrices. This is exactly what the algorithm in lines 8-9 does.

Next, the algorithm proceeds to the consideration of nonterminating rules. Namely, in line 12, the algorithm calculates new paths in the graph using four matrices for nonterminals from the rhs of the rule. Further, the algorithm update the matrices $B_p\_new$ and $C_p\_new$ to store only new values that was added at this iteration. Also, algorithm writes only new values to the matrix $A_p$ for the nonterminal $A$ and propagate new results among all matrices for the nonterminal $A$ in the rhs of other rules. 

The algorithm works while at least one value has appeared on the current iteration using the loop in lines 10-17. As the last step, the algorithm collects values from all rules where there is the starting nonterminal on the left-hand side (lhs) and puts these values into the matrix $Res$ for the MCFL-reachabilty result. The indices must be recalculated using the $transform\_index$ procedure presented in Listing~\ref{lst:algo3}.

\begin{algorithm}[H]
\floatname{algorithm}{Listing}
\footnotesize
\begin{algorithmic}[1]
\caption{The procedure for index transformation}
\label{lst:algo3}

\Procedure{$transform\_index$}{$i$, $j$, $p$}
\Comment{transform the indices $i$ and $j$ taking into account the set $end\_A(p)$}
    \State{$A \gets $ the nonterminal in the lhs of $p$}
    \State{$index \gets$ empty list}
            \For{$(end_l, end_r) \in end\_A(p)$}
                \If{$end_l < 2d(B)$}
                    \State{$pos \gets$ position $end_l$ in $end\_B(p)$} \State{$index.append(i[pos])$}
                \Else
                \State{$pos \gets$ position $end_l$ in $end\_C(p)$} \State{$index.append(j[pos])$}
                \EndIf
                
                \If{$end_r < 2d(B)$}
                    \State{$pos \gets$ position $end_r$ in $end\_B(p)$} \State{$index.append(i[pos])$}
                \Else
                \State{$pos \gets$ position $end_r$ in $end\_C(p)$} \State{$index.append(j[pos])$}
                \EndIf
            \EndFor
    \Return $index$
\EndProcedure

\end{algorithmic}
\end{algorithm}


% Consider a $2m$ - dimensional matrix $Res$ for $A \in N$ of size $(n + 1)^{2m}$ ($2m$ is Cartesian product degree). For convenience of notation, we assume that $Res[l_1, r_1, l_2, r_2, \dots, l_{d(A)}, r_{d(A)}, 0, \dots, 0] = Res[l_1, r_1, l_2, r_2, \dots, l_{d(A)}, r_{d(A)}]$.

% Denote matrix $Res$ for $A \in N$ like $Res_A$.

% Boolean matrix $Res_A$ indexing can be conveniently represented as $(l_1, r_1,$  $\dots, l_{d(A)}, r_{d(A)})$. This representation denote $(n+1)-ary$ number, what can be useful in the implementation. We call \emph{FromIndex} the algorithm obtaining a number from an index for matrix $Res_A$ and \emph{ToIndex} inverse to it. Matrix $Res_A$ can be represented as a square matrix of size $FromIndex((n+1, \dots, n+1)) \times FromIndex((n+1, \dots, n+1))$ ($|(n+1, \dots, n+1)|$ = $2m$), since each index of the matrix $Res_A$ has an even number of elements. That is, we can halve the index and store the left side in the row index, and the right side in the column index. If necessary, we can get the full index by concatenating the results of \emph{ToIndex}.

% We construct the algorithm in such a way that it is true that $\forall i: 1 \leq i \leq d(A) \ \ \exists \pi_i$ --- path between $l_i$ and $r_i: A \xLongrightarrow[G]{*} (l(\pi_1), \dots, \pi_{d(A)})$ iff $Res[l_1, r_1, \dots, l_{d(A)}, r_{d(A)}] = True$. We call it \emph{main condition} of the algorithm.

% The algorithm consists of two main steps. The first step is to process the terminating rules of $G$. For this consider $p: A \rightarrow (a_1, \dots, a_{d(A)}) \in P$. To satisfy the main condition, set $Res[l_1, r_1, \dots$ $, l_{d(A)}, r_{d(A)}] = True$ iff $(l_1, a_1, r_1), \dots, (l_{d(A)}, a_{d(A)} r_{d(A)}) \in E$ (all combinations).

% The second step is to consider non-terminating rules. For this consider $p: A \rightarrow f(B,C) \in P$.

% Construct two boolean matrices $B_p$ and $C_p$ from $Res_B$, $end\_B(p)$, $alter\_B(p)$, $Res_C$, $end\_C(p)$ and $alter\_C(p)$. 

% Describe construction $B_p$. Consider $\forall i_{res} = (l_1, r_1, \dots, l_{d(B)}, r_{d(B)}): Res_B[l_1, r_1, \dots, l_{d(B)}, r_{d(B)}] = True$ and construct index by taking the elements of $i_{res}$ at the positions pointed by the the elements $end\_B(p)$ and putting them in the order of consideration of the elements $end\_B(p)$. Denote the resulting index $i_B$. Similarly, we construct $j_B$ using $alter\_B(p)$. At the end we put $B_p[FromIndex(i_B)$ $, FromIndex(j_B)] = True$.

% Similarly, we construct $C_p$ but in the process of construction consider the values of $end\_C(p)$ and $alter\_C(p)$ reduced by $2d(B)$. We cannot store the values from these sets immediately equal to the positions for the index, since it is necessary to distinguish them from the value from the sets $end\_B(p)$, $alter\_B(p)$.

% Further, we calculate $A_p = B_p \times C_p$.

% At the end, update the Boolean matrix $Res_A$ using $A_p$. To do this, consider $\forall (i,j): A_p[i,j] = True$, compute $i_A = ToIndex(i)$ and $j_A = ToIndex(j)$. We get the index for the matrix $Res_A$ by putting the elements $i_A$ and $j_A$ in the right positions using ordered pairs from $end\_A$.


\subsection{Example}

Let us consider some steps of the algorithm using an example. As input we take the graph $D_1$ and multiple context-free grammar $G'_1$ from Section~\ref{sec:preliminaries}.

At the first step, the algorithm process the terminating rules of $G_1$ (lines 4-7 in listing~\ref{lst:algo1}). Also, the algorithm updates the matrices with the $\_new$ suffix for the nonterminals in the rhs of each rule. Consider, for example, what the $B_p\_new$ and $C_p\_new$ for the rule $S^1 \rightarrow S_1^1 S_2^1 S_1^2 S_2^2$ looks like after these updates. For this rule, the algorithm uses sets from the Definition~\ref{def:sets-end-alter} that look as follows.



\begin{center}
\begin{itemize}
    \item $end\_B = \{0\}$
    \item $alter\_B = \{1,2,3\}$
    \item $end\_C = \{7\}$
    \item $alter\_C = \{4,5,6\}$
    \item $end\_A = \{(0, 7)\}$
\end{itemize}
\end{center}

\begin{center}
$B_p = B_p\_new = $ {\footnotesize
\bordermatrix{\text{ }&(2,3,2)&(2,3,4)&(4,3,2)&(4,3,4)\cr
                     (1)&True&True&.&.\cr
                     (5)&.&.&True&True}
}
\end{center}






\begin{center}
$C_p = C_p\_new = $ $C_p$ = \bordermatrix{\text{ }&(1)&(5)\cr
                     (2,3,2)&True&.\cr
                     (4,3,2)&.&True\cr
                     (2,3,4)&True&.\cr
                     (4,3,3)&.&True}
\end{center}



% At the first step, the algorithm process the terminal rules of $G_1$ (row 5-8 in listing~\ref{lst:algo1}). Finally put $\forall N \in \{A,B,C,D\}, x \in \Sigma: N \rightarrow x \in P,$ $Res_N[i,j] = True$, where $(i, x, j) \in E$ and $\forall N \in \{S_1, S_2\}: (N^1, N^2) \rightarrow (x_1, x_2) \in P,$ $Res_N[i_1,j_1,i_2,j_2] = True$, where $(i_1, x_1, j_1), (i_2, x_2, j_2) \in E$. At the end of this step, the matrix $Res$ looks like this. We present the matrix $Res$ without boolean decomposition for brevity. As noted above, the indexing of the original matrix $Res_N$ for some nonterminal N can be obtained by concatenating the row and column index. For example, if $i=(1,2)$ and $j = (3,4)$ then $index = (1,2,3,4)$.

% {\footnotesize
% \bordermatrix{\text{ }&(1)&(2)&(3)&(4)&(5)&(2,1)&(3,2)&(3,4)&(4,5)\cr
%                      (1)&.&\{A\}&.&.&.&.&.&.&.\cr
%                      (2)&\{D\}&.&\{B\}&.&.&.&.&.&.\cr
%                      (3)&.&\{C\}&.&\{C\}&.&.&.&.&.\cr
%                      (4)&.&.&\{B\}&.&\{D\}&.&.&.&.\cr
%                      (5)&.&.&.&\{A\}&.&.&.&.&.\cr
%                      (1,2)&.&.&.&.&.&.&\{S_1\}&\{S_1\}&.\cr
%                      (2,3)&.&.&.&.&.&\{S_2\}&.&.&\{S_2\}\cr
%                      (4,3)&.&.&.&.&.&\{S_2\}&.&.&\{S_2\}\cr
%                      (5,4)&.&.&.&.&.&.&\{S_1\}&\{S_1\}&.}
% }

At the second step, the algorithm processes nonterminating rules (lines 10-17 in listing~\ref{lst:algo1}). Consider the rule with nonterminal $S$ in the lhs. First, the algorithm multiplies the matrices $B_p$, $B_p\_new$, $C_p$ and $C_p\_new$ in the specified order (line 12 in listing~\ref{lst:algo1}). As a result, the matrix $A_p\_{new}$ is computed of the following form. Note that we omit rows and columns with all zero values.

\begin{center}
    $A_p\_new$ = \bordermatrix{\text{ }&(1)&(5)\cr
                            (1)&True&True\cr
                            (5)&True&True}


\end{center}

Next, the algorithm will try to propagate the information and consider other rules. However, no other paths will be obtained further. At the last stage, the algorithm builds a matrix for the starting nonterminal, which in this case will coincide with the previous matrix.

\begin{center}
$Res = A_p$ = \bordermatrix{\text{ }&(1)&(5)\cr
                            (1)&True&True\cr
                            (5)&True&True}
\end{center}                          

Thus, the algorithm received information that there are paths that satisfy the grammar $G_1$ from vertex 1 to 5, from 5 to 1, from 1 to 1, and from 5 to 5. Therefore, the multiple context-free relation $R_{D_1, L(G_1)} = \{(1, 5), (5, 1), (1, 1), (5, 5)\}$.

% First, the algorithm builds the matrix $B_p$ and $C_p$ (row 8-27 in listing~\ref{lst:algo1}, using the procedure $build\_B_pC_p$). For example, considering $i=(1,2)$ and $j=(3,4)$, where $Res_{S_1}[i,j] = True$, in this case initializes $B_p[(1), (2,3,4)] = True$, since $end\_B = \{0\}$ and $alter\_B = \{1,2,3\}$, that is, using the indexing of the original matrix $Res_{S_1}$ $index = (1,2,3,4)$, the first set dictates what needs to be chosen $(1)$ as the index row of the matrix $B_p$ (since $(1)$ is at position 0 in $index$). The second set dictates what needs to be chosen $(2,3,4)$ as the index column for the same reasons.

% $B_p$ = \bordermatrix{\text{ }&(2,3,2)&(4,3,2)&(2,3,4)&(4,3,3)\cr
%                      (1)&True&.&True&.\cr
%                      (5)&.&True&.&True}
             
% $C_p$ = \bordermatrix{\text{ }&(1)&(5)\cr
%                      (2,3,2)&True&.\cr
%                      (4,3,2)&.&True\cr
%                      (2,3,4)&True&.\cr
%                      (4,3,3)&.&True}

% Second, the algorithm multiply $B_p$ and $C_p$.

% $A_p = B_p \times C_p$ = \bordermatrix{\text{ }&(1)&(5)\cr
%                                       (1)&True&True\cr
%                                       (5)&True&True}

% At the end of the iteration, the matrix $Res_S$ is updated (row 41-59 in listing~\ref{lst:algo1}). At this step, it is known that $i$ is obtained using the set $end\_B$ and j using the set $end\_C$, where $A_p[i,j]=True$. The algorithm consider $(end_l, end_r) = (0,7)$. Since $end_l < 2d(B) = 4$ it takes the position of $end_l$ in $end\_B$ ($end\_B$ is ordered set) and puts in $index$ the value at that position (in this case $pos = 0$) in $i_A$ (which was received in the row 42). Similarly for $end_r$.

% $Res_S =$ \bordermatrix{\text{ }&(1)&(5)\cr
%                                       (1)&True&True\cr
%                                       (5)&True&True}

\subsection{Correctness and complexity}
Similarly to the proof of the correctness of the matrix-based CFL-reachability algorithm from~\cite{azimov2018context}, it can be shown that the following theorem holds by the induction on the iteration number and on the height of derivation trees.

\begin{mytheorem}\label{thm:correctness}
Let $G = (N, \Sigma, P, S, d)$ be an MCFG in \emph{normal form}, $D =(V, E, \Sigma)$ be an labeled directed graph and $Res$ matrix obtained as a result of the algorithm (listing~\ref{lst:algo1}). Then $(v_0, v_n) \in R_{D,L(G)}$ iff $Res[v_0,v_n] = True$.
\end{mytheorem}

The most critical part of the algorithm is the evaluation of new values for $B_p \times C_p$ in line 13. Next, we will estimate the time needed for computing $B_p \times C_p$. Assume the sizes of Boolean matrices $B_p$ and $C_p$ are $(n+1)^q \times (n+1)^r$ and $(n+1)^r \times (n+1)^s$, respectively. Then for every nonterminating rule $p: A \rightarrow f(B,C)$ we define the \textit{degree} of this rule $e(p) = d(A) + d(B) + d(C)$. Also, we define the multiplication unit of $p$ as follows.

\begin{definition}
Let $G = (N, \Sigma, P, S, d)$ be an MCFG in \emph{normal form}. For every nonterminating rule $p: A \rightarrow f(B,C)$, define the number $i(p)$, called the multiplication unit of p, as $d(A) + d(B) + d(C) - 2 \cdot max\{d(A), d(B), d(C)\}$.
\end{definition}

According to~\cite{nakanishi1997efficient}, the $B_p \times C_p$ operations can be computed in $O(n^{e(p) - 0.624i(p)})$. Now we must estimate the number of iterations of the algorithm in listing~\ref{lst:algo1}. In~\cite{nakanishi1997efficient} the algorithm solves the MCFL recognition problem, and only $n$ iterations are needed. Thus, the MCFL recognition algorithm in~\cite{nakanishi1997efficient} computes the result in time $O(n^{e(p) - 0.624i(p) + 1})$. In our MCFL-reachability algorithm, we may need to make more iterations. The matrices $B_p$ (or $C_p$) can only change if a new tuple of paths $(\pi_1, \ldots, \pi_{d(B)})$ was found, such that $B \xLongrightarrow[G]{*} (l(\pi_1), \dots, l(\pi_{d(B)}))$. Such tuples of paths are described in the algorithm using the indices $(l_1, r_1, l_2, \ldots, r_{d(B)})$ where $\forall 1 \leq i \leq d(B)$, $\pi_i$ --- is a path from vertex $l_i$ to vertex $r_i$. In the worst case, exactly one new index will appear on each iteration. Thus, the number of iterations does not exceed the number of distinct indices. Let $m$ be the dimension of the MCFG $G$. Then the number of distinct indices is $O(n^{2m})$. Therefore, the following theorem holds.

\begin{mytheorem}\label{thm:complexity}
Let $G = (N, \Sigma, P, S, d)$ be an MCFG in \emph{normal form} with the dimension $m$, $D =(V, E, \Sigma)$ be an labeled directed graph. Let $p'$ be a rule such that $e(p') - 0.624i(p') = max\{e(p) - 0.624i(p) \mid p\in P\}$, and let $e' = e(p')$, $i' = i(p')$. The algorithm in listing~\ref{lst:algo1} computes the result in $O(|V|^{e' - 0.624i' + 2m})$.
\end{mytheorem}

This is the first naive time complexity bound for MCFL-reachabilty problem. For example, for the grammar $G'_1$ from Section~\ref{sec:preliminaries}, the $e' = 5$, the $i' = 1$, and the $m = 2$. Thus the time complexity bound for this grammar is $O(|V|^{8,376})$. However, $O(n^{2m})$ iterations can be reached only on special artificial graphs, and usually, on real-world graphs the number of iterations is small. Also, there are known techniques for improving this complexity bound by recursive multiplication of submatrices~\cite{valiant1975general}. Finally, the complexity bound can be improved by taking into account the sparsity of matrices and the sparse matrix operation.