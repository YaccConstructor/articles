\section{Preliminaries}

We introduce !!!!

\subsection{Context-Free Path Querying}

Graph, grammar, etc.

Let $i\pi j$ denote a unique path between nodes $i$ and $j$ of the graph and $l(\pi)$ denotes a unique string which is obtained from the concatenation of edge labels along the path $\pi$.
For a context-free grammar $G = (\Sigma, N, P, S)$ and directed labelled graph $D = (Q, \Sigma, \delta)$, a triple $(A, i, j)$ is \textit{realizable} iff there is a path $i\pi j$ such that nonterminal $A \in N$ derives $l(\pi)$.

\subsection{Tensor-Based algorithm for CFPQ}

\begin{algorithm}[H]
\begin{algorithmic}[1]
\caption{Kronecker product context-free recognizer for graphs}
\label{alg:Kronecker}
\Function{contextFreePathQuerying}{D, G}
\EndFunction
\end{algorithmic}
\end{algorithm}

\subsection{Planar Graphs}

A planar graph $G = (V, E)$ is a graph that can be embedded in the plane.

Outer face - unbounded face in specific embedding.

Directed graph (\textit{digraph})

...

\subsection{Dynamic reachability algorithms}

We consider algorithms that solve the problem of reachability in planar directed graphs. In the \textit{dynamic reachability problem} we are given a graph $G$ subject to edge updates (insertions or deletions) and the goal is to design a data structure that would allow answering queries about the existence of a path.

We need to answer the queries of type: "Is there a directed path from $u$ to $v$ in $G$?". If vertex $u$ in all the queries is fixed we say that algorithm is \textit{single-source}. It is said to be \textit{all-pairs} if vertices $u, v$ can be any vertices of planar digraph $G$, in this case it can be also called \textit{dynamic transitive closure}.

We say that the algorithm is \textit{fully dynamic} if it supports both additions and deletions of edges. It is said to be \textit{semi dynamic} if it supports only one of these updates. If semi dynamic algorithm supports additions only it is called \textit{incremental}, if deletions only - \textit{decremental}.





