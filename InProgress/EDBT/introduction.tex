\section{Introduction}
Graph data models are widely used in many areas, for example, bioinformatics~\cite{Bio}, graph databases~\cite{graphDB}. In these areas, it is often required to process large graphs. The most common among graph queries are navigational queries. The result of query evaluation is a set of implicit relations between nodes of the graph, i.e. paths in the graph. A natural way to specify these relations is by specifying paths using formal grammars (regular expressions, context-free grammars) over the alphabet of edge labels. Context-free grammars are actively used in graphs queries because of the limited expressive power of regular expressions. 

The result of context-free path query evaluation is usually a set of triples $(A, m, n)$ such that there is a path from node $m$ to node $n$, whose labeling is derived from a non-terminal $A$ of the given context-free grammar. This type of query is evaluated using the \textit{relational query semantics}~\cite{hellingsRelational}. Another example of path query semantics is \textit{single-path query semantics}~\cite{hellingsPathQuerying} which requires to present a single path from node $m$ to node $n$ whose labeling is derived from a non-terminal $A$ for all triples $(A, m, n)$ evaluated using relational query semantics. There is a number of algorithms for context-free path query evaluation using these semantics~\cite{GLL, hellingsRelational, RDF}.

Existing algorithms for context-free path query evaluation w.r.t. these semantics demonstrate poor performance when applied to big data. One of the most common technique for efficient big data processing is \textit{GPGPU} (General-Purpose computing on Graphics Processing Units), but these algorithms do not allow to use this technique efficiently. The algorithms for context-free language recognition had a similar problem until Valiant~\cite{valiant} proposed a parsing algorithm which computes a recognition table by computing matrix transitive closure. Thus, the active use of matrix operations (such as matrix multiplication) in the process of a transitive closure computation makes it possible to efficiently apply GPGPU computing techniques~\cite{matricesOnGPGPU}.

We address the problem of creating an algorithm for context-free path query evaluation using the relational and the single-path query semantics which allows us to speed up computations with GPGPU by using the matrix operations.

The main contribution of this paper can be summarized as follows:
\begin{itemize}
	\item We show how the context-free path query evaluation w.r.t. relational and single-path query semantics can be reduced to the calculation of matrix transitive closure.
	\item We introduce an algorithm for context-free path query evaluation w.r.t. relational query semantics which is based on matrix operations that make it possible to speed up computations by means of GPGPU.
	\item We provide a formal proof of correctness of the proposed algorithm.
\end{itemize}
