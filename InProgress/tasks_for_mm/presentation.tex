\documentclass{beamer}
\usepackage{beamerthemesplit}
\usepackage{wrapfig}
\usetheme{SPbGU}
\usepackage{pdfpages}
\usepackage{amsmath}
\usepackage{cmap} 
\usepackage[T2A]{fontenc} 
\usepackage[utf8]{inputenc}
\usepackage[english,russian]{babel}
\usepackage{indentfirst}
\usepackage{amsmath}
\usepackage{tikz}
\usepackage{multirow}
\usepackage[noend]{algpseudocode}
\usepackage{algorithm}
\usepackage{algorithmicx}
\usetikzlibrary{shapes,arrows}
%usepackage{fancyvrb}
%\usepackage{minted}
%\usepackage{verbments}

\beamertemplatenavigationsymbolsempty
\title[]{YaccConstructor}
\subtitle[YaccConstructor]{Курсовые проекты 2017}
% То, что в квадратных скобках, отображается в левом нижнем углу. 
\institute[]{
Лаборатория языковых инструментов JetBrains \\
Санкт-Петербургский государственный университет \\
Математико-механический факультет }

% То, что в квадратных скобках, отображается в левом нижнем углу.
\author[YC Team]{}

\date{28 сентября 2017г.}

\definecolor{orange}{RGB}{179,36,31}

\begin{document}
{
\begin{frame}[fragile]
  \begin{tabular}{p{2.5cm} p{5.5cm} p{2cm}}
   \begin{center}
      \includegraphics[height=1.5cm]{pictures/JBLogo3.pdf}
    \end{center}
    &
    \begin{center}
      \includegraphics[height=1.5cm]{pictures/SPbGU_Logo.png}
    \end{center}
    &
    \begin{center}
      \includegraphics[height=1.5cm]{pictures/YC_logo.pdf}
    \end{center} 
  \end{tabular}
  \titlepage
\end{frame}
}

\begin{frame}[fragile]
  \transwipe[direction=90]
  \frametitle{YaccConstructor}
  \begin{itemize}
    \item Исследования в области формальных языков и синтаксического анализа
    \item Исследовательские задачи
    \item Открытый исходный код
    \begin{itemize}
      \item \url{https://github.com/YaccConstructor}
    \end{itemize}
    \item F\# --- основной язык разработки
    \begin{itemize}
      \item А также Coq, Haskel $\dots$
    \end{itemize}
  \end{itemize}
\end{frame}

\begin{frame}[fragile]
  \transwipe[direction=90]
  \frametitle{Области интересов}
  \begin{itemize}
    \item Граммтики, другие способы формализации языков
    \item Поиск путей в графах с ограничениями, выраженными в терминах языков
    \begin{itemize}
      \item Статический анализ кода
      \item Графовые базы данных
    \end{itemize}
    \item Вывод грамматик (grammar inference)
  \end{itemize}
\end{frame}

\begin{frame}[plain,c]
 \transwipe[direction=90]
 \begin{center}
  \Huge Задачи
 \end{center}
\end{frame}

\begin{frame}
  \transwipe[direction=90]
  \frametitle{Пересесение контекстно-свободных граммтик}
  \begin{itemize}
    \item Руководитель: Семён Григорьев (\url{rsdpisuy@gmail.com})
    \item \textbf{Задача.} Существует алгоритм пересечения произвольной и нерекурсивной КС грамматик, основанный на CYK. Можно ли построить алгоритм для решения этой задачи, основанный на матричных операциях?
    \item Подробное описание задач: \url{https://goo.gl/WKbCSo}
  \end{itemize}
\end{frame}

\begin{frame}
  \transwipe[direction=90]
  \frametitle{Конъюнктивный синтаксический анализ графов}
  \begin{itemize}
    \item Руководитель: Рустам Азимов (rustam.azimov19021995@gmail.com)
    \item Задача: Существует алгоритм синтаксического анализа графов для линейных конъюнктивных грамматик. В нашей лаборатории был разработан алгоритм синтаксического анализа графов, работающий с любой конъюнктивной грамматикой. Требуется извлечь набор данных и грамматики, использованных для первого алгоритма, реализовать второй алгоритм и провести апробацию реализации на извлеченных данных.
    \item Подробное описание задачи: 	https://github.com/YaccConstructor/YaccConstructor/issues/279
  \end{itemize}
\end{frame}

\begin{frame}
  \transwipe[direction=90]
  \frametitle{????}
  \begin{itemize}
    \item Руководитель:
    \item Задача:
  \end{itemize}
\end{frame}

            
\begin{frame}
\transwipe[direction=90]
\frametitle{Контакты}
\begin{itemize}
  \item Семён Григорьев: \url{rsdpisuy@gmail.com}
  \item Артём Горохов: \url{gorohov.art@gmail.com}
  \item Рустам Азимов: \url{rustam.azimov19021995@gmail.com}
  \item Исходный код YaccConstructor: \url{https://github.com/YaccConstructor}
\end{itemize}
\end{frame}
\end{document}
