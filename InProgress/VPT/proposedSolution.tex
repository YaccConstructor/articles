\section{Proposed Solution}

The goal of our research is to figure out whether distillation~\cite{hamilton2021700} could be a solution to the intermediate data structures problem in linear algebra-based programs.
To answer this question we developed a library of matrix operations in POT language: a simple functional language used by Geoff Hamilton in his distiller.  

We use a quad-tree representation~\cite{qtree} for matrices because it avoids indexing and is natural for functional programming since it is expressible in terms of algebraic data types.
Moreover, it makes it possible to represent both sparse and dense matrices naturally, and express basic operations over such a representation via recursive functions which traverse this tree-like structure.
Finally, this structure allows to conveniently exploit divide-and-conquer parallelism in matrices handling functions.

We selected two different target hardware platforms.
The first one is Reduceron~\cite{naylorRunciman2012} --- a general-purpose functional-language-specific processor.
The second one is program-specific hardware for arbitrary functional programs FHW~\cite{Edwards2019FHWP} which utilizes the flexibility of FPGA to create hardware for a particular program.
While the first case is more typical, the second one might provide higher performance for specific tasks.

At the current stage, we propose to use distillation as the first step of program optimization which, we hope, should reduce memory traffic, and then compile a distilled program to two different hardware platforms by using the respective compiler with platform-specific optimizations.
For evaluation, we propose to create a library of linear algebraic operations, such as matrix-matrix, matrix-vector, and matrix-scalar operations.
Programs of interest are compositions of these basic functions.