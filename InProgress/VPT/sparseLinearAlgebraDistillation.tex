\documentclass[submission,copyright,creativecommons]{eptcs}
\pagestyle{plain}
\providecommand{\event}{VPT 2022} % Name of the event you are submitting to
\usepackage{breakurl}             % Not needed if you use pdflatex only.
\usepackage{underscore}           % Only needed if you use pdflatex.
\usepackage{lineno}
\usepackage{multirow}
\usepackage{xcolor}

\usepackage{minted}
\usepackage{multicol} 


\providecommand{\keywords}[1]
{
  \small	
  \textbf{\textit{Keywords---}} #1
}


\linenumbers
\title{Distillation of Sparse Linear Algebra}

%\author[1,4]{Aleksey Tyurin}
%\author[1,4]{Ekaterina Vinnik}
%\author[1,2,4]{Daniil Berezun}
%\author[1,2,4]{Semyon Grigorev}
%\author[3,4]{Geoff Hamilton}

\author{Aleksey Tyurin
\institute{Saint Petersburg University, Russia}
\institute{JetBrains Research, Russia}
\email{alekseytyurinspb@gmail.com}
\and
Ekaterina Vinnik
\institute{Saint Petersburg University, Russia}
\institute{JetBrains Research, Russia}
\email{catherine.vinnik@gmail.com}
\and
Mikhail Nikoliukin
\institute{National Research University \\ Higher School of Economics, Russia}
\email{mnnikolyukin@edu.hse.ru}
%\email{michael.nik999@gmail.com}
\and
Daniil Berezun
\institute{Saint Petersburg University, Russia}
\institute{JetBrains Research, Russia}
\email{d.berezun@spbu.ru}
%\email{daniil.berezun@jetbrains.com}
\and
Semyon Grigorev
\institute{Saint Petersburg University, Russia}
\institute{JetBrains Research, Russia}
\email{s.v.grigoriev@spbu.ru}
%\email{semyon.grigorev@jetbrains.com}
\and
Geoff Hamilton
\institute{School of Computing, \\ Dublin City University, Ireland}
\email{geoffrey.hamilton@dcu.ie}
}

\def\titlerunning{Distillation of Sparse Linear Algebra}
\def\authorrunning{A. Tyurin, E. Vinnik, et al}

\definecolor{azure(colorwheel)}{rgb}{0.0, 0.5, 1.0}

\newcommand{\db}[1]{{\color{violet} #1}}
\newcommand{\at}[1]{{\color{azure(colorwheel)} #1}}

\begin{document}

\maketitle

\begin{abstract}
  Linear algebra (LA) is a common language for many areas, providing high-performance solutions by utilizing the highly parallelizable nature of LA operations.
  However, intermediate data structures arising during LA expressions evaluation are one of the primary performance bottlenecks, especially when sparse data structures are used.
  We show that program distillation can be efficiently used to optimize LA-based programs by minimizing occurrence and evaluation of intermediate data structures.
\end{abstract}


% Each submission must include on its first page the paper title; 
% authors and their affiliations; 
% contact author's email; 
% abstract; 
% and **three to four keywords** that will be used to assist the PC in selecting appropriate reviewers for the paper. 
\keywords{fusion, high-level synthesis, sparse data processing, linear algebra}

\section{Introduction}

Scalable high-performance graph analysis is an actual challenge.
There is a big number of ways to attack this challenge~\cite{Coimbra2021} and the first promising idea is to utilize general-purpose graphic processing units (GPGPU).
Such existing solutions, as CuSha~\cite{10.1145/2600212.2600227} and Gunrock~\cite{7967137} show that utilization of GPUs can improve the performance of graph analysis, moreover it is shown that solutions may be scaled to multi-GPU systems.
But low flexibility and high complexity of API are problems of these solutions.

The second promising thing which provides a user-friendly API for high-performance graph analysis algorithms creation is a GraphBLAS API~\cite{7761646} which provides linear algebra based building blocks to create graph analysis algorithms.
The idea of GraphBLAS is based on a well-known fact that linear algebra operations can be efficiently implemented on parallel hardware.
Along with that, a graph can be natively represented using matrices: adjacency matrix, incidence matrix, etc.
While reference CPU-based implementation of GraphBLAS, SuiteSparse:GraphBLAS~\cite{10.1145/3322125}, demonstrates good performance in real-world tasks, GPU-based implementation is challenging.

One of the challenges in this way is that real data are often sparse, thus underlying matrices and vectors are also sparse, and, as a result, classical dense data structures and respective algorithms are inefficient. 
So, it is necessary to use advanced data structures and procedures to implement sparse linear algebra, but the efficient implementation of them on GPU is hard due to the irregularity of workload and data access patterns.
Though such well-known libraries as cuSPARSE show that sparse linear algebra operations can be efficiently implemented for GPGPU, it is not so trivial to implement GraphBLAS on GPGPU. 
First of all, it requires \textit{generic} sparse linear algebra, thus it is impossible just to reuse existing libraries which are almost all specified for operations over floats.
The second problem is specific optimizations, such as masking fusion, which can not be natively implemented on top of existing kernels.
Nevertheless, there is a number of implementations of GraphBLAS on GPGPU, such as GraphBLAST~\cite{yang2019graphblast}, GBTL~\cite{7529957}, which show that GPGPUs utilization can improve the performance of GraphBLAS-based graph analysis solutions.
But these solutions are not portable because they are based on Nvidia Cuda stack.
Moreover, the scalability problem is not solved: all these solutions support only single-GPU, not multi-GPU computations.

To provide portable GPU implementation of GraphBLAS API we developed a \textit{SPLA} library\footnote{Source code available at: \url{https://github.com/JetBrains-Research/spla}}.
This library utilizes OpenCL for GPGPU computing to be portable across devices of different vendors.
Moreover, it is initially designed to utilize multiple GPGPUs to be scalable.
To sum up, the contribution of this work is the following.
\begin{itemize}
    \item Design of portable GPU GraphBLAS implementation proposed. The design involves the utilization of multiple GPUS. Additionally, the proposed design is aimed to simplify library tuning and wrappers for different high-level platforms and languages creation. 
    \item Subset of GraphBLAS API, including such operations as masking, matrix-matrix multiplication, matrix-matrix e-wise addition, is implemented. The current implementation is limited by COO and CSR matrix representation format and uses basic algorithms for some operations, but work in progress and more data formats will be supported and advanced algorithms will be implemented in the future.
    \item Preliminary evaluation on such algorithms as breadth-first search (BFS) and triangles counting (TC), and real-world graphs shows portability across different vendors and promising performance: for some problems Spla is comparable with GraphBLAST. Surprisingly, for some problems, the proposed solution on embedded Intel graphic card shows better performance than SuiteSparse:GraphBLAS on the respective CPU. At the same time, the evaluation shows that further optimization is required.
\end{itemize} 
\section{Proposed Solution}

% The goal of our research is to figure out whether distillation~\cite{hamilton2021700} could be a solution to the intermediate data structures problem in linear algebra-based programs.
The goal of our research is to find out if the linear algebra-based programs can be efficiently optimized by program distillation~\cite{hamilton2021700} by eliminating intermediate data structures and computations.
To answer this question, we developed a library of matrix operations~\db{add link to our implementation} in POT language: a simple functional language used by Hamilton in his distiller~\db{add link to the Geoff's github}.  

% We use a quad-tree representation~\cite{qtree} for matrices because it avoids indexing and is natural for functional programming since it is expressible in terms of algebraic data types.
We use a quad-tree matrix representation since it both avoids indexing and can be implemented via algebraic data types, which itself is very natural for functional programming.
% Moreover, it makes it possible to represent both sparse and dense matrices naturally, and express basic operations over such a representation via recursive functions which traverse this tree-like structure.
Besides, it provides a natural way to represent both sparse and dense matrices, as well as makes it possible to express basic operations over the representation via recursive functions traversing the tree-like structure.
% Finally, this structure allows to conveniently exploit divide-and-conquer parallelism in matrices handling functions.
Finally, the quad-tree representation allows natively exploiting divide-and-conquer parallelism in matrices handing functions.

We selected two different target hardware platforms.
The first one is Reduceron~\cite{naylorRunciman2012} --- a general-purpose functional-language-specific processor.
The second is program-specific hardware for arbitrary functional programs FHW~\cite{Edwards2019FHWP} which utilizes the flexibility of FPGA to create hardware for a particular program.
While the first case is more typical, the second might provide higher performance for specific tasks.

% At the current stage,
For now, we propose to use program distillation as the first step of program optimization which, we hope, should reduce memory % traffic,
usage, and then compile a distilled program to the two different hardware platforms by using the respective compiler with platform-specific optimizations.
For evaluation, we propose \db{\{propose or some of them are already implemented?!? =)) \}} to create a library of linear algebraic operations, such as matrix-matrix, matrix-vector, and matrix-scalar operations.
Programs of interest are compositions of these basic functions.

%\section{Implementation}

\at{Initially neither Reduceron nor FHW were stable enough to run our sparse routines. 
Reduceron supported only limited number of arguments for each function, while FHW had support only for algebraic data types with less than 64 number of fields which was not the case when translating our benchmarks. 
So both issues have been fixed.
On top of that, appropriate initial values were set for System Verilog signals in FHW to make it possible to utilize Vivado to handle further development.

FHW relies on External Core feature of GHC as a frontend, which has been removed since GHC $>$ 7.6.3. 
To mitigate this dependency, to make the code base more maintainable, and to overcome other issues, like overflows during CPS-transformation and support for partially applied functions, we have opted for GRIN~\cite{GRIN} as an intermediate representation between POT language and dataflow representation of FHW.
GRIN makes defunctionalization, which is essential for hardware generation, more convenient and provides extensive points-to analysis. Finally, FHW assumes the presence of a hardware garbage collector, but does not implement it. We also have not implemented this feature yet.

The distiller at the moment produces function duplicates during residualization, which is non-trivial to resolve. 
Such duplicates increase the consumption of logic blocks in hardware, so we remove them before translation. The usage of De Bruijn indexes makes it possible to rename only function names to determine whether two functions are duplicates.
}
\section{Preliminary Evaluation}

For now, we have implemented some basic functions for the proposed library, which are used in the current evaluation stage: matrix-to-matrix element-wise addition (\verb|mtxAdd|),
matrix-to-scalar \emph{apply-to-all} operation (\verb|map|),
masking (\verb|mask|), which takes a subset of matrix elements, and Kronecker product (\verb|Kron|).
The following examples, which are a combination of the implemented functions, are used for the evaluation. 
The examples are fairly practical, for example, one could see a sequence of element-wise additions in a Luby's maximal independent set algorithm, and masking is a one of key operation in GraphBLAS standard~\cite{buluc2017graphblas}.

\begin{itemize}
\item Sequential addition of four matrices:\\
  \verb|seqAdd m1 m2 m3 m4 = mtxAdd (mtxAdd (mtxAdd m1 m2) m3) m4|
\item Masking of two matrices addition:\\
  \verb|addMask m1 m2 m3 = mask (mtxAdd m1 m2) m3|
\item Masking of two matrices Kronecker product:\\
  \verb|kronMask m1 m2 m3 = mask (kron m1 m2) m3 |  
\item Element-wise processing of two matrices addition:\\
  \verb|addMap m1 m2 = map f (mtxAdd m1 m2)|  
\item Element-wise processing of two matrices Kronecker product:\\
  \verb|kronMap m1 m2 = map f (kron m1 m2)|  
\end{itemize}

We compare original versions of these functions and distilled ones in three ways.
First, we use the interpreter of the POT language to measure the number of reductions and memory reads inside \verb|case| expressions.
We use the simulator shipped with Reduceron to measure the number of clock ticks necessary to evaluate a program, and Vivado's simulator for FHW-compiled programs to measure the number of both clock ticks and memory writes that a program produces.
It is worth noting that Reduceron has somewhat fixed clock frequency, while frequency for FHW-generated hardware varies depending on a particular program. 
Since we do not have external memory at the moment, and all the data lives inside the generated scheme, the logic is not synthesizable for reasonably sized matrices in the case of FHW.
We get similar clock frequencies for distilled and non-distilled programs for inputs with smaller matrices and hence assume that clock frequencies are also similar below. 
Thus, we provide only the number of ticks instead of time.

A set of sparse matrices of appropriate sizes provided at~\cite{Matrices} is used. 
The matrices are converted into boolean ones since POT language lacks the needed primitives at the moment.
Average results for several hundreds of different inputs are presented in table~\ref{tbl:evaluationResults}.

\begin{table}[ht]
    \centering    
    \begin{tabular}{|c|c|c|c|c||c|c|c|c|c|}
        \hline
        \multirow{2}{*}{Function} &  \multicolumn{4}{c||}{Matrix size}  & \multicolumn{2}{c|}{Interpreter}            & Reduceron & \multicolumn{2}{c|}{FHW}\\
        \cline{2-10}
                                  &   m1 & m2 & m3 & m4                & Red-s & Reads                               & Ticks     & Ticks & Writes \\
        \hline
        \hline
        seqAdd   & $64 \times 64$ & $64 \times 64$ & $64 \times 64$ & $64 \times 64$ & 2.7          & 1.9        & 1.8 & 1.4 & 1.1 \\ 
        addMask  & $64 \times 64$ & $64 \times 64$ & $64 \times 64$ & --             & 2.1          & 1.8        & 1.4 & 1.4 & 1.1\\ 
        kronMask & $64 \times 64$ & $2 \times 2$   &$128 \times 128$& --             & 2.2          & 1.9        & 1.4 & 2.7 & 2.5\\ 
        addMap   & $64 \times 64$ & $64 \times 64$ & --             & --             & 2.5          & 1.7        & 1.7 & 1.5 & 1\\
        kronMap  & $64 \times 64$ & $2 \times 2$   & --             & --             & 2.9          & 2.2        & 1.8 & 2.0 & 1\\ 
        \hline
        
    \end{tabular}
    \caption{Evaluation results: original program to distilled one ratio of measured metrics is presented}
    \label{tbl:evaluationResults}
\end{table}

We can see that on average distillation provides up to 3 and 2 times improvement in terms of reductions and memory reads respectively for the interpreter. 
The number of reductions is also considerably reduced for hardware benchmarks. 
The lack of matches between ticks for FHW and Reduceron is justified by architecture distinction. 
Finally, from the last column one could see memory consumption reduction, which supports our approach.
All this hopefully makes the proposed solution viable, and we look forward to coming up with full-fledged experiments that would target real hardware and real life competitors like C++ implementations.

\section{Future Work}

We show that distillation is a promising way to optimize linear algebra based programs, thus it is a promising a way to optimize machine learning and graph processing procedures.

In the future, first of all, we should close a technical debt and make the distiller more stable to handle all important cases: current implementation can not handle such important functions as matrix-matrix multiplication.
Along with it, we should improve the input language to make it more user-friendly.
The main challenge here is to find the balance between language expressivity and the practicality of distillation for it.
Having basic workflow implemented we should explore how to utilize distillation in the best way for each particular platform. 
For example, which level of distillation is the best for our particular problem and set of functions?
Can we exploit more parallelism using distillation?
Can we efficiently exploit the tail-modulo-cons property of the distilled program?
What are the limitations of distillation: whether all important cases can be handled?

When the language and the distiller will be stable enough, we plan to implement a full-featured generic linear algebra library power enough to express basic graph analysis algorithms and to create and train neural networks.
After that, a number of graph analysis algorithms and neural networks will be implemented and evaluated.

Along with it we plan to improve both FHW and Reduceron and compilers for it in order to make them mature enough to handle real-world examples.
For example, it is necessary to support out-of-chip memory.

\bibliographystyle{eptcs}
\bibliography{sparseLinearAlgebraDistillation}
\end{document}
