\section{Preliminaries}
The function $nnz(A)$ denotes the number of non-zero elements in matrix $A$.
\\
Let $i\pi j$ denote a unique path between nodes $i$ and $j$ of the graph and $l(\pi)$ denotes a unique string which is obtained from the concatenation of edge labels along the path $\pi$.
For a context-free grammar $G = (\Sigma, N, P, S)$ and directed labelled graph $D = (Q, \Sigma, \delta)$, a triple $(A, i, j)$ is \textit{realizable} iff there is a path $i\pi j$ such that nonterminal $A \in N$ derives $l(\pi)$.
\\
Short description of the Rustam algorithm + pseudocode~\ref{Rustam}. 
\begin{algorithm}[H]
\begin{algorithmic}[1]
\caption{Context-free recognizer for graphs}
\label{alg:graphParse}
\Function{contextFreePathQuerying}{D, G}
    
    \State{$n \gets$ the number of nodes in $D$}
    \State{$E \gets$ the directed edge-relation from $D$}
    \State{$P \gets$ the set of production rules in $G$}
    \State{$T \gets$ the matrix $n \times n$ in which each element is $\varnothing$}
    \ForAll{$(i,x,j) \in E$}
    \Comment{Matrix initialization}
        \State{$T_{i,j} \gets T_{i,j} \cup \{A~|~(A \rightarrow x) \in P \}$}
    \EndFor    
    \While{matrix $T$ is changing}
       
        \State{$T \gets T \cup (T \times T)$}
        \Comment{Transitive closure $T^{cf}$ calculation} 
    \EndWhile
\State \Return $T$
\EndFunction
\end{algorithmic}
\end{algorithm}
