\section{Introduction}

The secondary structure of RNA is tightly related to biological functions of organisms and plays an important role in classification and recognition problems.
One of the approaches to analyze the secondary structure of RNA is based on formal language methods.
Namely, one can process RNA sequence as a string over 4-letter alphabet $\{G,A,C,U\}$ and use formal language methods to describe properties of this string and then analyze strings w.r.t. described properties by parsing them. 

Context-free languages (CFL-s) are the most prominent in this area, while probabilistic context-free grammars are widely used for secondary structure description and related tasks~\cite{dowell2004evaluation,knudsen1999rna}.
But more expressive language classes are required to describe some important features of secondary structure. For example, pseudoknots cannot be expressed in terms of a context-free grammar, but can be expressed with a conjunctive grammar~\cite{zier2013rna} which were proposed by Alexander Okhotin~\cite{10.5555/543313.543323} and are the natural extension of CFG. 

For some problems, it is necessary to find all derivable substrings of a given string~\cite{durbin1996biological}.
This case is the string-matching problem also known as a string-searching problem.
The classical example of it is to find substrings matched by the given regular expression (or regular template). 
But if one tries to find a substring with a specific secondary structure, then it is necessary to use at least context-free template (context-free grammar) and, as a result, utilize an appropriate parsing algorithm. 

Most CFG-based approaches suffer the same issue: the computational complexity is poor.
Traditionally used CYK~\cite{kasami1966efficient,Younger:1966:CLP:1441427.1442019} runs with a cubic time complexity and demonstrates poor performance on long strings or big grammars~\cite{liu2005parallel}.
We argue that more efficient algorithms are needed in fields such as bioinformatics where large amount of data is common.

Asymptotically most efficient parsing algorithm is Valiant's algorithm~\cite{Valiant:1975:GCR:1739932.1740048} which is based on matrix multiplication.
Okhotin generalized this algorithm to conjunctive and boolean grammars~\cite{Okhotin:2014:PMM:2565359.2565379}. 
Moreover, in comparison to CYK, Valiant’s algorithm simplifies the utilization of parallel techniques to improve performance by offloading critical computations onto matrices multiplication.
However, this algorithm is not suitable for the string-matching problem because it computes a huge amount of redundant information.

In this paper we present the modification of Valiant's algorithm which improves the utilization of GPGPU and parallel computations by processing some submatrices products concurrently.
Also, the proposed algorithm can be easily utilized for the string-matching problem.
We also prove the correctness of our algorithm and analyze its time complexity.
The proposed solution was implemented using fast matrix multiplication algorithms and parallel techniques. Evaluation of the implementation shows that GPGPU version is up to 20 times faster then GPGPU-based implementation of the Valiant's algorithm.
