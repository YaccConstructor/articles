% This is samplepaper.tex, a sample chapter demonstrating the
% LLNCS macro package for Springer Computer Science proceedings;
% Version 2.20 of 2017/10/04
%
\documentclass[runningheads]{llncs}
%
\usepackage{graphicx}
\usepackage{amsmath,amsfonts,latexsym,amssymb,euscript,xr}
\usepackage{tikz} 
\usepackage{pgfplots} 
\usepackage{sidecap} 
\usepackage{soul}
\usepackage{xcolor}
\usepackage{multirow}
\usepackage{alltt}
\usepackage[misc,geometry]{ifsym}


% Package to generate and customize Algorithm as per ACM style
\usepackage[ruled, linesnumbered, noend]{algorithm2e}
\renewcommand{\algorithmcfname}{Listing}
\SetAlFnt{\small}
\SetAlCapFnt{\small}
\SetAlCapNameFnt{\small}
\SetAlCapHSkip{0pt}
\IncMargin{-\parindent}

% Used for displaying a sample figure. If possible, figure files should
% be included in EPS format.
%
% If you use the hyperref package, please uncomment the following line
% to display URLs in blue roman font according to Springer's eBook style:
% \renewcommand\UrlFont{\color{blue}\rmfamily}

\begin{document}
%
\title{Modification of Valiant's Parsing Algorithm for the String-Searching Problem\thanks{The research was supported by the Russian Science Foundation, grant No. 18-11-00100.}}
%
\titlerunning{Valiant's Parsing for String-Searching}
% If the paper title is too long for the running head, you can set
% an abbreviated paper title here
%
\author{Yuliya Susanina\inst{1,2}\orcidID{0000-0003-3904-3764} \Letter \and
Anna Yaveyn\inst{1}\orcidID{0000-0001-9733-5429} \and
Semyon Grigorev\inst{1,2}\orcidID{0000-0002-7966-0698}}
%
\authorrunning{Yuliya Susanina, Anna Yaveyn, and Semyon Grigorev}
% First names are abbreviated in the running head.
% If there are more than two authors, 'et al.' is used.
%
\institute{Saint Petersburg State University, Universitetskaya nab. 7/9 \\ St. Petersburg, 199034 Russia \\
\and JetBrains Research, Primorskiy prospekt 68-70, Building 1 \\ St. Petersburg 197374, Russia\\
\email{jsusanina@gmail.com} \Letter,
\email{anya.yaveyn@yandex.ru},
\email{semyon.grigorev@jetbrains.com}
}
%
\maketitle              % typeset the header of the contribution
%
\begin{abstract}
Some string-matching problems can be reduced to parsing: verification whether some sequence can be derived in the given grammar. 
To apply parser-based solutions to such area as bioinformatics, one needs to improve parsing techniques so that the processing of large amounts of data was possible.
The most asymptotically efficient parsing algorithm that can be applied to any context-free grammar is a matrix-based algorithm proposed by Valiant.
This paper presents a modification of the Valiant’s algorithm, which facilitates efficient utilization of modern hardware in highly-parallel implementation. 
Moreover, the modified version significantly decreases the number of excessive computations, accelerating the search of substrings.

\keywords{Context-free grammar \and Parsing \and Valiant's algorithm \and String-matching \and Secondary structure.}
\end{abstract}
%
%
%

\section{Introduction}

Scalable high-performance graph analysis is an actual challenge.
There is a big number of ways to attack this challenge~\cite{Coimbra2021} and the first promising idea is to utilize general-purpose graphic processing units (GPGPU).
Such existing solutions, as CuSha~\cite{10.1145/2600212.2600227} and Gunrock~\cite{7967137} show that utilization of GPUs can improve the performance of graph analysis, moreover it is shown that solutions may be scaled to multi-GPU systems.
But low flexibility and high complexity of API are problems of these solutions.

The second promising thing which provides a user-friendly API for high-performance graph analysis algorithms creation is a GraphBLAS API~\cite{7761646} which provides linear algebra based building blocks to create graph analysis algorithms.
The idea of GraphBLAS is based on a well-known fact that linear algebra operations can be efficiently implemented on parallel hardware.
Along with that, a graph can be natively represented using matrices: adjacency matrix, incidence matrix, etc.
While reference CPU-based implementation of GraphBLAS, SuiteSparse:GraphBLAS~\cite{10.1145/3322125}, demonstrates good performance in real-world tasks, GPU-based implementation is challenging.

One of the challenges in this way is that real data are often sparse, thus underlying matrices and vectors are also sparse, and, as a result, classical dense data structures and respective algorithms are inefficient. 
So, it is necessary to use advanced data structures and procedures to implement sparse linear algebra, but the efficient implementation of them on GPU is hard due to the irregularity of workload and data access patterns.
Though such well-known libraries as cuSPARSE show that sparse linear algebra operations can be efficiently implemented for GPGPU, it is not so trivial to implement GraphBLAS on GPGPU. 
First of all, it requires \textit{generic} sparse linear algebra, thus it is impossible just to reuse existing libraries which are almost all specified for operations over floats.
The second problem is specific optimizations, such as masking fusion, which can not be natively implemented on top of existing kernels.
Nevertheless, there is a number of implementations of GraphBLAS on GPGPU, such as GraphBLAST~\cite{yang2019graphblast}, GBTL~\cite{7529957}, which show that GPGPUs utilization can improve the performance of GraphBLAS-based graph analysis solutions.
But these solutions are not portable because they are based on Nvidia Cuda stack.
Moreover, the scalability problem is not solved: all these solutions support only single-GPU, not multi-GPU computations.

To provide portable GPU implementation of GraphBLAS API we developed a \textit{SPLA} library\footnote{Source code available at: \url{https://github.com/JetBrains-Research/spla}}.
This library utilizes OpenCL for GPGPU computing to be portable across devices of different vendors.
Moreover, it is initially designed to utilize multiple GPGPUs to be scalable.
To sum up, the contribution of this work is the following.
\begin{itemize}
    \item Design of portable GPU GraphBLAS implementation proposed. The design involves the utilization of multiple GPUS. Additionally, the proposed design is aimed to simplify library tuning and wrappers for different high-level platforms and languages creation. 
    \item Subset of GraphBLAS API, including such operations as masking, matrix-matrix multiplication, matrix-matrix e-wise addition, is implemented. The current implementation is limited by COO and CSR matrix representation format and uses basic algorithms for some operations, but work in progress and more data formats will be supported and advanced algorithms will be implemented in the future.
    \item Preliminary evaluation on such algorithms as breadth-first search (BFS) and triangles counting (TC), and real-world graphs shows portability across different vendors and promising performance: for some problems Spla is comparable with GraphBLAST. Surprisingly, for some problems, the proposed solution on embedded Intel graphic card shows better performance than SuiteSparse:GraphBLAS on the respective CPU. At the same time, the evaluation shows that further optimization is required.
\end{itemize} 
\section{\bf Formal languges}

In this section we introduce basic definitions from formal language theory, and describe Valiant's parsing algorithm which we use as a base for our solution.

An alphabet $\Sigma$ is a finite nonempty set of symbols.
$\Sigma^{*}$ is a set of all finite strings over $\Sigma$.
A contex-free grammar $G_S$ is a quadruple $(\Sigma, N, R, S)$, where $\Sigma$ is a finite set of terminals, $N$ is a finite set of nonterminals, $\Sigma \cup N = \varnothing$, $R$ is a finite set of productions of the form $A \rightarrow \beta$, where $A \in N, \beta \in V^{*}$, $V = \Sigma \cup N$ and $S \in N$ is a start symbol.
Context-free grammar $G_S = (\Sigma, N, R, S)$ is said to be in Chomsky normal form if all productions in $R$ are of the form: $A \rightarrow BC$, $A \rightarrow a$, $S \rightarrow \varepsilon$, where $A, B, C \in N, a \in \Sigma, \varepsilon$ is an empty string.
$L_{G}(S) = \{ \omega | S\xrightarrow[G_S]{*} \omega\}$ is a language specified by the grammar $G_{S} = (\Sigma, N, R, S)$, where $A \xrightarrow[G_S]{*} \omega$ means that $\omega$ can be derived in a finite number of rules applications from the start symbol $S$.

\subsection{\bf \it Valiant's parsing algorithm}

Tabular parsing algorithms construct a matrix $T$ cells of which are filled with nonterminals from which the corresponding substring can be derived. 
These algorithms are usually work with the grammar in Chomsky normal form.
Namely, $T_{i, j} =  \{ A | A \in N, a_{i + 1} \dots a_{j} \in L_{G}(A)\} \quad \forall i < j$, where $G_S=(\Sigma, N, R, S)$.

The elements of $T$ are filled successively beginning with $T_{i - 1, i} = \{ A | A \rightarrow a_{i} \in R\}.$
Then, $T_{i, j} = f(P_{i, j}),$ where
$P_{i, j} = \bigcup\limits_{k = i + 1}^{j - 1} T_{i,k} \times T_{k, j}$ and
$f(P) = \{A | \exists A \rightarrow BC \in R : (B, C) \in P\}.$
Finally, the input string $a_{1}a_{2} \dots a_{n}$ belongs to $L_{G}(S)$ iff $S \in T_{0, n}$.

If all elements are filled sequentially, the time complexity of this algorithm is $O(n^3)$.
Valiant proposed to offload the most intensive computations to the Boolean matrix multiplication. 
As the most time-consuming is computing $\bigcup\limits_{k = i + 1}^{j - 1} T_{i, k} \times T_{k, j}$, Valiant anged the computation of $T_{i, j}$, in order to use multiplication of submatrices of $T$.
Multiplication of two submatrices of parsing table $T$ is defined as follows.
Let $X \in (2^N)^{m \times l}$ and $Y \in (2^N)^{l \times n}$ be two submatrices of parsing table $T$. 
Then, $X \times Y = Z$, where $Z \in (2^{N \times N})^{m \times n}$ and $Z_{i, j} = \bigcup\limits_{k = 1}^{l} X_{i, k} \times Y_{k, j}$.

Note that the computation of $X \times Y$  can be replaced by the multiplication of $|N|^2$ Boolean matrices (for each nonterminal pair).
Denote the matrix corresponding to the pair $(B, C) \in N \times N$ as $Z^{(B, C)}$, then $Z_{i, j}^{(B, C)} = 1$ iff $(B, C) \in Z_{i, j}$.
It should also be noted that $Z^{(B, C)} = X^{B} \times Y^{C}$.
Each Boolean matrix multiplication can be computed independently.
Following these changes, time complexity of this algorithm is $O(|G|BMM(n)log(n))$ for an input string of length $n$, where $BMM(n)$ is the number of operations needed to multiply two Boolean matrices of size $n \times n$.

Valiant's algorithm written as proposed by Okhotin is presented in listing~\ref{algo:valiant}.
All elements of $T$ and $P$ are initialized by empty sets.
Then, the elements of these two table are successively filled by two recursive procedures.

% Algorithm1
\begin{algorithm}[h]
\SetAlgoNoLine
\KwIn{Grammar $G = (\Sigma, N, R, S), w = a_{1} \dots a_{n}, n \geq 1, a_{i} \in \Sigma$, where  $n + 1 = 2^k$}
\underline{main()}{:}{

 \textit{compute(0, n + 1)\;}
 accept if and only if $S \in T_{0, n}$
 \linebreak
 }

\underline{compute(\textit{l, m})}{:}{

 \If {$m - l \geq 4$}{
     \textit{compute(l, $\frac{l+m}{2}$)\;
     compute($\frac{l+m}{2}$, m)}}
 \textit{complete(l, $\frac{l+m}{2}$, $\frac{l+m}{2}$, m)}
 \linebreak
 }

\underline{complete(\textit{l, m}, $l^\prime$, $m^\prime$)}{:}{

 \lIf {$m - l = 4$ and $m = l^\prime$}{$T_{l, l + 1} = \{A | A \rightarrow a_{l+ 1} \in R\}$}
 \lElseIf{$m - l = 1$ and $m < l^\prime$}{ $T_{l, l'} = f(P_{l, l'})$}
 \ElseIf{$m - l > 1$}{
    $leftgrounded = (l, \frac{l+m}{2}, \frac{l+m}{2}, m), rightgrounded = (l', \frac{l'+m'}{2}, \frac{l'+m'}{2}, m')$,

    $bottom = (\frac{l+m}{2}, m, l', \frac{l'+m'}{2}), left = (l, \frac{l+m}{2}, l', \frac{l'+m'}{2})$,

    $right = (\frac{l+m}{2}, m, \frac{l'+m'}{2}, m'), top = (l, \frac{l+m}{2}, \frac{l'+m'}{2}, m')$\;
    complete(bottom)\;
    $P_{left} = P_{left} \cup (T_{leftgrounded} \times T_{bottom})$\;
    complete(left)\;
    $P_{right} = P_{right} \cup (T_{bottom} \times T_{rightgrounded})$\;
    complete(right)\;
    $P_{top} = P_{top} \cup (T_{leftgrounded} \times T_{right})$\;
    $P_{top} = P_{top} \cup (T_{left} \times T_{rightgrounded})$\;
    complete(top)
    }
 }
\caption{Parsing by matrix multiplication: Valiant's Version}
\label{algo:valiant}
\end{algorithm}

The procedure $compute(l, m)$ computes correct values of $T_{i,j}$ for all $l \le i < j < m$.

The procedure $complete(l, m, l', m')$ constructs the submatrix $T_{i, j}$ for all $l \le i < m$, $l' \le j < m'$. This procedure assumes $T_{i, j}$ for all $l \leq i < j < m,  l' \leq i < j < m'$ are already constructed and the current value of  $P[i, j] =  \{ (B, C) |\exists k, (m \le k < l'), a_{i + 1} \dots a_{k} \in L(B), a_{k + 1} \dots a_{j} \in L(C)\}$ for all $l \leq i < m,  l' \leq j < m'$. The submatrix division during the procedure call is shown in figure~\ref{fig2}.


\begin{figure}
\vspace{3mm}
 \begin{center}
    \begin{minipage}{0.48\textwidth}
        \centering
        \includegraphics[width=6cm]{pictures/splitting_with_grounded.pdf}
        \caption{Matrix partition used in \textit{complete(l, m, l', m')} procedure.}
        \label{fig1}
    \end{minipage}\hfill
    \begin{minipage}{0.48\textwidth}
        \centering
        \includegraphics[width=6cm]{pictures/layers.pdf}
        \caption{Matrix partition on V-shaped layers used in modification.}
        \label{fig2}
    \end{minipage}
 \end{center}
\vspace{-8mm}
\end{figure}

A simple example of parsing with the Valiant's algorithm is presented in figure~\ref{fig3}.
Only several steps are shown, but it is enough to point out at this version and our approach differences.

\begin{figure}
\vspace{3mm}
 \begin{center}
 \includegraphics[width=12cm]{pictures/valbeg2.pdf}
    \caption{An example of beginning of Valiant's algorithm}
    \label{fig3}
\end{center}
\vspace{-8mm}
\end{figure}
\section{Modified Valiant's algorithm}

In this section, we propose how to rearrange the order in which submatrices are processed in the algorithm.
The different order improves the independence of submatrices handling and facilitates the implementation of parallel submatrix processing.

\subsection{Layered submatrices processing}

We propose to divide the parsing table into layers of disjoint submatrices of the same size (see Figure~\ref{fig2}).
Such division is possible because the derivation of a substring of the fixed length does not depend on either left or right contexts.
An appropriate order of substrings processing guarantees the disjointness of submatrices which form a layer.
Each layer consists of square matrices which size is a power of 2.
The layers are computed successively in the bottom-up order.
Each matrix in the layer can be handled independently, which facilitates parallelization of layer processing.

\begin{figure}[h]
\vspace{3mm}
 \begin{center}
 \includegraphics[width=12cm]{pictures/modivis2.pdf}
    \caption{An example of the modification of Valiant's algorithm}
    \label{fig4}
 \end{center}
\vspace{-8mm}
\end{figure}

Figure~\ref{fig4} demonstrates the modified algorithm.
The lowest layer (submatrices of size 1) has already been computed.
The second layer is filled in in the steps 1-2.
Although the original algorithm computes the same matrix, the modified one needs only two steps using parallel computation of submatrix products.

The modified version of Valiant's algorithm is presented in Listing~\ref{algo:modified}.
The procedure \textit{main()} computes the lowest layer $(T_{l, l+1})$, and then divides the table into layers, and computes them with the \textit{completeVLayer()} function.
Thus, \textit{main()} computes all elements of parsing table $T$.

We define \textit{left(subm)}, \textit{right(subm)}, \textit{top(subm)}, \textit{bottom(subm)}, \textit{rightgrounded(subm)} and \textit{leftgrounded(subm)} functions which return the submatrices for matrix $\textit{subm} = (l, m, l', m')$ according to the original Valiant's algorithm (Figure~\ref{fig2}).


\begin{algorithm}[!h]
\SetAlgoNoLine
\KwIn{$G = (\Sigma, N, R, S), w = a_{1} \dots a_{n}, n \geq 1, n + 1 = 2^p, a_{i} \in \Sigma$ }
\underline{main()}{:}{

 \lFor {$l \in \{1, \ldots, n \}$}{$T_{l, l + 1} = \{A | A \rightarrow a_{l + 1} \in R\}$}
 \For{$1 \le i < p - 1 $}{
 \textit{layer = constructLayer($i$)}\;
 \textit{completeVLayer(layer)}
 }
 accept if and only if $S \in T_{0, n}$
 \BlankLine
 }

\underline{constructLayer(i)}{:}{
 \BlankLine
 $\{(k2^i, (k+1)2^i, (k + 1)2^i, (k+2)2^i) \, |\, 0 \le k < 2^{p - i} - 1\}$
 \BlankLine
    }
\underline{completeLayer(M)}{:}{
\BlankLine
\If {$\forall (l, m, l', m') \in M \quad (m - l = 1)$}{\lFor{$ (l, m, l', m') \in M$}{$T_{l, l'} = f(P_{l, l'})$}}
\Else{
\textit{completeLayer($\{\textit{bottom(subm)}\, |\,\textit{subm} \in M \})$}\;
\textit{completeVLayer(M)}
}
\BlankLine
}

\underline{completeVLayer(M)}{:}{
 \BlankLine
 $\textit{multiplicationTasks}_1 = \linebreak
    \{\textit{left(subm)}, \textit{leftgrounded(subm)}, \textit{bottom(subm)}\, 
    |\,\textit{subm} \in M \} \cup \linebreak  \{\textit{right(subm)}, \textit{bottom(subm)}, \textit{rightgrounded(subm)}\, |\,\textit{subm} \in M\}$\;
 \BlankLine
 \textit{multiplicationTask$_2$} = $\{\textit{top(subm)}, \textit{leftgrounded(subm)}, \textit{right(subm)}\, |\,\textit{subm} \in M\}$\;
 \BlankLine
 \textit{multiplicationTask$_3$} = $\{\textit{top(subm)}, \textit{left(subm)}, \textit{rightgrounded(subm)}\, |\,\textit{subm} \in M\}$\;
 \BlankLine
 \textit{performMultiplications(multiplicationTask$_1$)}\;
 \textit{completeLayer($\{\textit{left(subm)}\, |\,subm \in M \} \cup \{\textit{right(subm)}\, |\,\textit{subm} \in M \}$)}\;
 \textit{performMultiplications(multiplicationTask$_2$)}\;
 \textit{performMultiplications(multiplicationTask$_3$)}\;
 \textit{completeLayer($\{top(subm)\, |\,subm \in M \}$)}

 }
 \BlankLine

 \underline{performMultiplication(tasks)}{:}{\\
 \lFor{$ (m, m1, m2) \in \textit{tasks}$}{$P_{m} = P_{m} \cup (T_{m1} \times T_{m2})$}
 }

\caption{Parsing by Matrix Multiplication: Modified Version}
\label{algo:modified}
\end{algorithm}


The procedure \textit{completeVLayer(M)} takes an array of disjoint submatrices $M$ which represents a layer.
For each \textit{subm = (l, m, l', m') $\in M$} this procedure computes \textit{left(subm), right(subm), top(subm)}.
The procedure assumes that the elements of \textit{bottom(subm)} and $T_{i, j}$ for all $i$ and $j$ such that $l \leq i < j < m$ and $  l' \leq i < j < m'$ are already constructed.
Also it is assumed that the current value of
$P_{i, j} =  \{ (B, C) | \exists k, (m \le k < l'), a_{i + 1} \dots a_{k} \in L_G(B), a_{k + 1} \dots a_{j} \in L_G(C)\} $ for all $i$ and $j$ such that $l \leq i < m$ and $l' \leq j < m'$.

The procedure \textit{completeLayer(M)} also takes an array of disjoint submatrices $M$, but unlike the previous one, it computes $T_{i, j}$ for all $(i, j) \in subm$.
This procedure requires exactly the same assumptions on $T_{i, j}$  and $P_{i, j}$  as in the previous case.

In other words, \textit{completeVLayer(M)} computes the entire layer \textit{M} \linebreak and \textit{completeLayer($M_{2}$)} is a helper function which is necessary for computation of smaller square submatrices $subm_{2} \in M_{2}$ inside of \textit{M}.

Finally, the procedure \textit{performMultiplication(tasks)}, where \textit{tasks} is an array of triples of submatrices, performs the basic step of the algorithm: matrix multiplication.
It is worth mentioning that $|tasks| \ge 1$ and each task can be computed independently, while the original algorithm handles one \textit{task} per step sequentially.
So, the practical implementation of this procedure can easily utilize different techniques of parallel array processing.

\subsection{Correctness and complexity}

We provide the proof of correctness and time complexity for the proposed modification in this section.
To do it we should prove correctness of subprocedure \textit{completeLayer}.

\begin{lemma}
Let $M$ be a layer. If for all $(l, m, l', m') \in M$:
\begin{enumerate}
  \item $T_{i, j} = \{ A |  a_{i + 1} \dots a_{j} \in L_G(A)\}$ for all $i$ and $j$ such that $l \leq i < j < m$ and $l' \leq i < j < m'$;
  \item $P_{i, j} =  \{ (B, C) |\exists k, (m \le k < l'): a_{i + 1} \dots a_{k} \in L_G(B), a_{k + 1} \dots a_{j} \in L_G(C)\}$ for all $l \leq i < m$ and $l' \leq j < m'$.
\end{enumerate}

Then the procedure \textit{completeLayer(M)}, returns correctly computed sets of $T_{i, j}$ for all $l \leq i \le m$ and $l' \leq j \le m'$ for all $(l, m, l', m') \in M$.
\end{lemma}

\begin{proof}
Proof by induction on $m - l$.
\end{proof}

\begin{theorem}
Algorithm from listing~\ref{algo:modified} correctly computes $T_{i, j}$ for all i and j, thus an input string $a = a_{1}a_{2} \dots a_{n} \in L_{G}(S)$ if and only if $S \in T_{0, n}$.
\end{theorem}

\begin{proof}
Primarily to prove the theorem, we show by induction that all layers of the parsing table T are computed correctly.

\underline{\textbf{Basis:}} layer of size $1 \times 1$.
Parsing table \textit{T} consists of one layer of size 1 and its elements are correctly computed in lines 2-3 in listing~\ref{algo:modified}.

\underline{\textbf{Inductive step:}} assume any layer of size less than or equal to $2^{p - 2} \times 2^{p - 2}$ are computed correctly. 

Define layer of size $2^{p - 1} \times 2^{p - 1}$ as M. 
Hereinafter \textit{subm = (l, m, l', m')} is a typical element of layer M.

Consider \textit{completeVLayer(M)} call. 

Firstly, \textit{performMultiplications(multiplicationTask$_1$)} adds to each P$_{i,j}$ all pairs 
$(B, C)$ such that $\exists k$, $(\frac{l+m}{2} \le k < l')$, $a_{i + 1} \dots a_{k} \in L_{G}(B)$, $a_{k + 1} \dots a_{j} \in L_{G}(C)$ for all $(i, j)$ $\in leftsublayer(M)$
and
$(B, C)$ such that $\exists k$, $(m \le k < \frac{l'+m'}{2})$, $a_{i + 1} \dots a_{k} \in L_{G}(B)$, $a_{k + 1} \dots a_{j} \in L_{G}(C)$ for all $(i, j)$ $\in rightsublayer(M)$.
Now \textit{completeLayer(leftsublayer(M) $\cup$ rightsublayer(M))} can be called and it returns correctly computed \textit{leftsublayer(M) $\cup$ rightsublayer(M)}.

Then \textit{performMultiplications} called with arguments 
\textit{multiplicationTask$_2$} and \textit{multiplicationTask$_3$} adds pairs 
$(B, C)$ such that $\exists k$, $(\frac{l+m}{2} \le k < m)$, $a_{i + 1} \dots a_{k} \in L_{G}(B)$, $a_{k + 1} \dots a_{j} \in L_{G}(C)$ 
and 
$(B, C)$ such that $\exists k$, $(l' \le k < \frac{l'+m'}{2})$, $a_{i + 1} \dots a_{k} \in L_{G}(B)$, $a_{k + 1} \dots a_{j} \in L_{G}(C)$
to each P$_{i,j}$ for all $(i, j)$ $\in topsublayer(M)$. 
So as $m = l'$ (from the construction of the layer), condition for elements of matrix $P$ are fulfilled.
Now \textit{completeLayer(topsublayer(M))} can be called and it returns correctly computed \textit{topsublayer(M)}.

All $T[i, j]$ $\forall (i, j) \in M$ are computed correctly.

Thus, \textit{completeVLayer(M)} returns correct $T_{i, j}$ for all $(i, j)$ $\in M$ for any layer M of parsing table T and lines 4-6 in listing~\ref{algo:modified} return all $T_{i, j} =  \{ A | A \in N, a_{i + 1} \dots a_{j} \in L_{G}(A)\}$.
\end{proof}


\begin{lemma}
Let \textit{calls$_{i}$} is a number of the calls of \textit{completeVLayer(M)} where for all $(l, m, l', m') \in M$ with $m - l = 2^{p - i}$.
\begin{itemize}
 \item for all $i \in \{ 1, .., p - 1\}$  $\sum_{n=1}^{calls_i}{|M|}$ is exactly $2^{2i - 1} - 2^{i - 1}$;
 \item for all $ i \in \{ 1, .., p - 1\}$ products of submatrices of size $2^{p - i} \times 2^{p - i}$ are calculated exactly $2^{2i - 1} - 2^{i}$ times.
\end{itemize}
\end{lemma}

\begin{proof}

Prove the first statement by induction on i.

\underline{\textbf{Basis:}} i = 1. \textit{calls$_{1}$} and $|M| = 1$. So, $2^{2i - 1} - 2^{i - 1} = 2^1 - 2^0 = 1$.

\underline{\textbf{Inductive step:}} assume that $\sum_{n=1}^{calls_i}{|M|}$ is exactly $2^{2i - 1} - 2^{i - 1}$ for all $i \in \{ 1, .., j\}$.

Let us consider $i = j + 1$.

Firstly, note that function $\textit{costructLayer(i)}$ returns $2^{p - i} - 1$ matrices of size $2^i$, so in the call of \textit{completeVLayer(costructLayer(k - i))}  \textit{costructLayer(k - i)} returns $2^i - 1$ matrices of size $2^{p - i}$. 
Secondly, \textit{completeVLayer(M)} is called 3 times for the left, right and top submatrices of size $2^{p - (i - 1)}$. Finally, \textit{completeVLayer(M)} is called 4 times for the bottom, left, right and top submatrices of size $2^{p - (i - 2)}$, except $2^{i - 2} - 1$ matrices which were already computed.

Then, $\sum_{n=1}^{calls_i}{|M|} = 2^{i} - 1 + 3 \times (2^{2(i - 1) - 1} - 2^{(i - 1) - 1}) + 4 \times (2^{2(i - 2) - 1} - 2^{(i - 2) - 1}) - (2^{i - 2} - 1) = 2^{2i - 1} - 2^{i - 1}$.

Now we know that $\sum_{n=1}^{calls_{i-1}}{|M|}$  is $2^{2(i - 1) - 1} - 2^{(i - 1) - 1}$ and we can calculate the number of products of submatrices of size $2^{p - i} \times 2^{p - i}$. 
During these calls \textit{performMultiplications} run 3 times, $|multiplicationTask1| = 2 \times 2^{2(i - 1) - 1} - 2^{(i - 1) - 1}$ and $|multiplicationTask2|$ = $|multiplicationTask3| = 2^{2(i - 1) - 1} - 2^{(i - 1) - 1}$. So, the number of products of submatrices of size $2^{p - i} \times 2^{p - i}$ is $ 4 \times (2^{2(i - 1) - 1} - 2^{(i - 1) - 1}) = 2^{2i - 1} - 2^{i}$.
\end{proof}

\begin{theorem}
Let $|G|$ be a length of the description of the grammar G and let n be a length of an input string. Then algorithm from listing~\ref{algo:modified} calculates matrix \textit{T} in $\mathcal{O}(|G|BMM(n)\log{n})$ where BMM(n) is the number of operations needed to multiply two Boolean matrices of size $n \times n$.
\end{theorem}

\begin{proof}
The proof is almost identical with the proof of theorem 1 given by Okhotin~\cite{Okhotin:2014:PMM:2565359.2565379}, because, as shown in the last lemma, the Algorithm 1 has the same number of products of submatrices.
\end{proof}

To summarize, the correctness of the modification was proved and it was shown that the time complexity remained the same as in Valiant's version.


\subsection{Algorithm for substrings}

Next, we show how our modification can be applied to the string-matching problem.

To find all substrings of size $s$, which can be derived from a start symbol for an input string of size $n = 2^p$, we need to compute layers with submatrices of size not greater than $2^{l'}$, where $2^{l' - 2} < s \le 2^{l' - 1}$.

Let $l' = p - (m - 2)$ and consequently $(m - 2) = p - l'$.
For any  $m \le i \le p$ products of submatrices of size $2^{p - i}$ are calculated exactly $2^{2i - 1} - 2^{i}$ times and each of them imply multiplying $\mathcal{O}(|G|)$ Boolean submatrices.
Now we estimate the number of operations needed to find all substrings:

\begin{equation*}
\begin{array}{c}
C \cdot \sum\limits_{i=m}^p 2^{2i - 1} \cdot 2^{\omega(p - i)} \cdot f(2^{p - i}) =
C \cdot 2^{\omega l'}\sum\limits_{i=2}^{l'} 2^{(2 - \omega)i} \cdot 2^{2(p - l') - 1} \cdot f(2^{l' - i}) \le \\
C \cdot 2^{\omega l'} f(2^{l'}) \cdot 2^{2(p - l') - 1} \sum\limits_{i=2}^{l'} 2^{(2 - \omega)i} =
\mathrm{BMM}(2^{l'}) \cdot 2^{2(p - l') - 1} \sum\limits_{i=2}^{l'} 2^{(2 - \omega)i}
\end{array}
\end{equation*}

Thus, time complexity for searching all substrings is  $O(|G|\mathrm{BMM}(2^{l'})(l' - 1))$, while time complexity for the full input string is $O(|G|\mathrm{BMM}(2^p)(p - 1))$. 
The Valiant's algorithm completely calculate at least 2 triangle submatrices of size $\frac{n}{2}$, as shown in Figure~\ref{fig5}, thus the minimum asymptotic complexity is $O(|G|\mathrm{BMM}(2^{p - 1})(p - 2))$.
Thus we can conclude that the modification is asymptotically faster than the original algorithm for substrings of size $s \ll n$.

\begin{figure}
\vspace{3mm}
 \begin{center}
 \includegraphics[width=12cm]{pictures/valsubstring.pdf}
    \caption{The number of elements necessary to compute in Valiant's algorithm. It is necessary to calculate at least 2 triangle submatrices of size $\frac{n}{2}$.}
    \label{fig5}
 \end{center}
\vspace{-8mm}
\end{figure}

\section{Evaluation}

This section describes the methodology and answers the following research questions.

\begin{enumerate}
    \item Does fusion via distillation give any benefits at the software and hardware levels?
    \item What are the properties of the generated hardware?
    \item Does the generated hardware outperform software implementations?
\end{enumerate}

\subsection{Methodology}

Our focus is on creating a basis for future research and experiments, thus we make our experiments as much reproducible as possible\footnote{\url{https://github.com/sedwards-lab/fhw/tree/sparse-linear-algebra-distillation/examples/QTreeBenchmarks/diploma/verilog-bool-no-nnz-inlined} (online; accessed:
2022-06-07) Here one could find all the results. For each benchmark all statistics are specified: matrix names, their sizes, collected metrics for both hardware and software benchmarks.}. We benchmarked the following list of chained functions. The choice is prompted by the current state of the distiller: at the moment, it does not successfully distill matrix multiplication. However, the functions are still practical enough, for example, chained addition could be seen in Luby's maximal independent set algorithm and clearly describe the applicability of the proposed approach.

\begin{itemize}
    \item \mintinline{Haskell}{mAdd (==False) (||) (mAdd (==False) (||) m1 m2) m3}
    \item \mintinline{Haskell}{mask (mAdd (== False) (||) m2 m3) (m1)}
    \item \mintinline{Haskell}{map (==Zero) (to_nat) (mAdd (==False) (||) m1 m2}
    \item \mintinline{Haskell}{map (==Zero) (to_nat) (kron (==False) (&&) m1 m2}
\end{itemize}

Above, \mintinline{Haskell}{Zero} and \texttt{to\_nat} are corresponding definitions for Peano arithmetics, since the \texttt{.pot} language does not have any primitives. For the same reason, we operated with boolean matrices. Such functions could be abstracted with free variables and then instantiated in the emitted Haskell code. However, to get maximum from distillation, we provided all the information about the functions. 

For these functions, we compared the execution time of distilled and not distilled hardware generated circuits, execution time of original and distilled Haskell code and reference \textit{Suite Sparse}\footnote{\url{https://github.com/DrTimothyAldenDavis/GraphBLAS} (online; accessed:
2022-06-07), Suite Sparse library sources.}\textsuperscript{,}\footnote{The library also uses different variations of coordinate formats (opaque to the user) and not a quadtree representation.} variants of these functions in C\texttt{++}. Note that SuiteSparse does not support recursive data types, thus only the first two function chains were implemented in SuiteSparse (since Peano number is essentially a linked list). We did not replace Peano numbers with integers, so our experiments could be interpreted easier. For hardware experiments we collected execution time and the number of memory writes and reads, to access how well fusion is performed. For software experiments we only measured the execution time. Also note that we measured only the time, required to execute the lines above, not including any IO, required to get and evaluate function arguments. But in hardware benchmarks we measured the time required to pass arguments into the circuit's memory, because such IO is inevitable. It is tricky to make such measures in Haskell due to laziness, thus the programs were compiled with \texttt{--fno-full-laziness} to turn off memoization. Also all the arguments were forced to normal form via \texttt{force} and \texttt{evaluate}. Haskell programs were compiled\footnote{GHC 8.10.4.} with \texttt{-O2 --fno-full-laziness} and Suite Sparse was compiled with default flags and linked as a shared library to C\texttt{++} code.

We took matrices from SuiteSparse matrix collection with sizes ranging from \texttt{64x64} to \texttt{512x512}. We limited ourselves with such sizes due to the fact that this is the maximum sizes that fit into \texttt{bram} with $2^{16}$ address space. Such number of \texttt{bram} blocks is available only on really expensive FPGA boards, thus in practice sizes would be smaller to achieve better utilization. Once again, it models the situation when data fits into the cache, since \texttt{bram} in our circuits will operate as a cache in real application.

\subsection{Experiments}

Table~\ref{tab:bench_results} shows the results of all execution time benchmarks. To evaluate execution time for hardware simulation, implementation stage was performed to assess the maximum frequency of FPGA device used for synthesis and implementation, and the number of execution cycles was multiplied by the number of nanoseconds a clock cycle takes. The frequencies were equal within the same benchamark set, i.e., frequency was not affected by distillation. We used \texttt{xcu250figd2104-2L} device\footnote{\url{https://www.xilinx.com/products/boards-and-kits/alveo/u250.html}  (online; accessed:
2022-06-07)} for synthesis and implementation stages. It is not really a casual and affordable chip, but it contains enough \texttt{bram} for our evaluation to see scalability. In the table a median across several benchmarks is shown. 

As it could be seen, distillation steadily increases performance: up to 2x speedup for hardware simulation and up to 3x for software benchmarks. The results are maintained within the borders of the corresponding confidence interval and the borders are not shown for brevity. Hardware speedup is lower due to the different execution semantics, dataflow is not reduction-based and distillation is a reduction-based transformation. Note that generated hardware appears to be less performant than both Haskell and C\texttt{++}, which a bit contradicts the results from~\cite{oldfhw}. For hardware benchmarks \texttt{time (IO)} shows the execution time including the time needed to transfer the data though the arguments, \texttt{time (no IO)} does not include it in its turn. It could be seen that not all the benchmarks are computationally extensive enough to cover memory transferring costs, but for more complex examples the ratio would be better. Since we basically transfer the matrices node by node from a file in the testbench, we have probably the lowest possible latency, and in practice it would be higher if reading from DDR, however the bandwidth could be increased. Noticeably, running times for \texttt{mMaskAdd} for C\texttt{++} and distilled Haskell are similar, which shows that fusion really provides some extra performance: SuiteSparse at the moment does not implement any fusion.

Table~\ref{tab:mem_results} summarizes the ratios between distilled and not distilled hardware circuits memory reads and writes. Since in general case distillation removes extra pattern matching, essentially it saves memory reads and writes. The eventual number of memory reads and writes is implementation dependent, thus the table shows what share of speedup is prompted by saving memory operations. Distillation successfully reduces the number of memory accesses, about 15\% on average. \texttt{mMapKron} has a bit higher ratio due to the fact that \texttt{Nat} numbers require additional memory accesses, since the type is recursive. It could also be seen that a major part of speedups is attributed to saved memory accesses. 

Finally, table~\ref{tab:resource_util} shows device resources utilization ratios between distilled and not distilled hardware circuits and frequencies. Columns are device primitives: registers, lookup tables, \texttt{bram} blocks or multiplexers. Utilization for both types of circuits is below 1\% of available resources on the device, except for the memory. Memory blocks utilization is about 30\% (since we choose larger \texttt{brams} to store larger matrices). Apart from that, distilled circuits could have both higher and lower utilization. Since the hardware generation is primarily syntax-directed it follows from the distilled program structure. For example, distillation might glue two recursive functions into one (in that case, memory utilization would be lower, because each cluster of mutually recursive functions possesses its own heap) or make more recursive functions than in the original program. The frequencies are the same, however, they possibly could be made better with more intelligent buffer allocation.

\subsection{Discussion}
Answering the research questions above.

\begin{enumerate}
    \item Fusion gives significant benefits, however at the hardware level the benefits are a bit smaller since hardware semantics is not reduction based. The benefits at the hardware level are mostly determined by the reduced number of memory accesses (each access takes 2 clock cycles). Notably, distilled Haskell implementation of \texttt{mMaskAdd} has similar performance with C\texttt{++}. 
    \item Device utilization is low, but such circuits could be copied on the same device to provide better utilization and higher parallelism. Resource utilization might be both better and worse after distillation, depending on the transformed program itself since translation is syntax-directed. Frequency could be increased by more intelligent buffering strategy.
    \item Although we use low-latency design with \texttt{bram}s that take 2 clock cycles per request and transfer matrices from files, which does not have any latency in simulation, we get slower execution time than Haskell and C\texttt{++} counterparts. It could be partly due to excessive buffering performed by FHW at the moment. Also there is no pipelining for recursive calls, i.e. only one set
of function argument tokens are allowed to enter a tail-recursive function call until a result is finally generated. Further CPS transformation hinders parallelization, which could be made more explicit with SSA. Some other optimizations exist that may significantly influence the performance. Also, since device utilization is about 1\%, such circuits could be copied on one device and provide more parallelism. A more detailed discussion could be found at~\cite{Edwards2019FHWP}.
\end{enumerate}

Distillation clearly showed its applicability to optimization of sparse linear algebra routines and notably it still could be combined with other techniques, like rewrite rules to achieve better results. High-level synthesis has a room for improvements by increasing pipelining, parallelism and frequency and the generated hardware could be improved from usability perspective: a support for arbitrary sized matrices is desirable. Thus we will focus on these directions. Probably a better solution would be to embed \texttt{.pot} language into e.g. Haskell to leverage its type system (to be able to use some rewrite rules as well), and add support for primitive types and parallel primitives to be able to conduct a more scalable comparison with SuiteSparse (since SuiteSparse is multithreaded). For such embedding different execution models could be implemented, including hardware synthesis, for which SSA form of GRIN looks promising, as well as extra optimizations shipped with GRIN. For hardware synthesis, an interesting direction is achieving predictable results in hardware from certain modifications in software. This property partly holds for the current approach, since the translation is syntax- directed. More information on this could be found at~\cite{predict}.

\pagebreak

\begin{table}[t]
\scriptsize
\centering
\caption*{mAddAdd}
\begin{tabular}{|c|c|c|c|c|c|c|c|c|c|} 
\hline
\rowcolor{LightBlue}
\multicolumn{3}{|c|}{Matrices dimensions} & Haskell & Haskell (distilled) & \multicolumn{2}{c|}{FHW} & \multicolumn{2}{c|}{FHW (distilled)} & {C\texttt{++}}\\
% \rowcolor{LightBlue}
\hline
m1 & m2 & m3 & time & time & time (no IO) & time (IO) & time (no IO) & time (IO) & time \\ 
\hline
64 & 64 & 64 & 29 us & 20 us & 76 us & 170 us & 64 us & 158 us & 14 us\\ 
128 & 128 & 128 & 94 & 79 & 146 & 476 & 134 & 469 & 30 \\
256 & 256 & 256 & 123 & 103 & 202 &  681 & 168 & 662 & 44\\
512 & 512 & 512 & 219 & 143 & 474 & 1192 & 375 & 1093 & 49\\
\hline
\end{tabular}

\caption*{mMaskAdd}
\begin{tabular}{|c|c|c|c|c|c|c|c|c|c|} 
\hline
\rowcolor{LightBlue}
\multicolumn{3}{|c|}{Matrices dimensions} & Haskell & Haskell (distilled) & \multicolumn{2}{c|}{FHW} & \multicolumn{2}{c|}{FHW (distilled)} & {C\texttt{++}}\\
% \rowcolor{LightBlue}
\hline
m1 & m2 & m3 & time & time & time (no IO) & time (IO) & time (no IO) & time (IO) & time \\ 
\hline
64 & 64 & 64 & 10 us & 7 us & 64 us & 133 us & 46 us & 111 us & 18 us\\ 
128 & 128 & 128 & 38 & 30 & 118 & 322 & 75 & 292 & 33 \\
256 & 256 & 256 & 48 & 42 & 168 &  498 & 104 & 456 & 46\\
512 & 512 & 512 & 126 & 76 & 400 & 762 & 300 & 729 & 65\\
\hline
\end{tabular}

\caption*{mMapAdd}
\begin{tabular}{|c|c|c|c|c|c|c|c|c|c|} 
\hline
\rowcolor{LightBlue}
\multicolumn{3}{|c|}{Matrices dimensions} & Haskell & Haskell (distilled) & \multicolumn{2}{c|}{FHW} & \multicolumn{2}{c|}{FHW (distilled)} & {C\texttt{++}}\\
% \rowcolor{LightBlue}
\hline
m1 & m2 & m3 & time & time & time (no IO) & time (IO) & time (no IO) & time (IO) & time \\ 
\hline
64 & 64 & --- & 45 us & 37 us & 189 us & 253 us & 137 us & 202 us & ---\\ 
128 & 128 & --- & 162 & 105 & 524 & 685 & 397 & 579 & --- \\
256 & 256 & --- & 312 & 216 & 1047 &  1360 & 680 & 986 & ---\\
512 & 512 & --- & 436 & 273 & 1346 & 1776 & 900 & 1330 & ---\\
\hline
\end{tabular}

\caption*{mMapKron}
\begin{tabular}{|c|c|c|c|c|c|c|c|c|c|} 
\hline
\rowcolor{LightBlue}
\multicolumn{3}{|c|}{Matrices dimensions} & Haskell & Haskell (distilled) & \multicolumn{2}{c|}{FHW} & \multicolumn{2}{c|}{FHW (distilled)} & {C\texttt{++}}\\
% \rowcolor{LightBlue}
\hline
m1 & m2 & m3 & time & time & time (no IO) & time (IO) & time (no IO) & time (IO) & time \\ 
\hline
2 & 64 & --- & 64 us & 36 us & 212 us & 242 us & 94 us & 125 us & ---\\ 
2 & 128 & --- & 137 & 68 & 434 & 502 & 199 & 266 & --- \\
2 & 256 & --- & 364 & 126 & 1004 &  1188 & 449 & 636 & ---\\
4 & 128 & --- & 302 & 94 & 694 & 763 & 330 & 401 & ---\\
\hline
\end{tabular}



\caption{Execution time}
\label{tab:bench_results}

\end{table}
\begin{table}[h]
\scriptsize
\begin{minipage}{0.45\linewidth}
\centering
\caption*{mAddAdd}
\begin{tabular}{|c|c|c|c|c|c|c|} 
\hline
\rowcolor{LightBlue}
\multicolumn{3}{|c|}{Matrices dimensions} & \multicolumn{2}{c|}{Ratio ($\frac{FHW}{FHW_{distilled}}$)}\\
% \rowcolor{LightBlue}
\hline
m1 & m2 & m3 & writes & reads\\ 
\hline
64 & 64 & 64 & 1.10 & 1.15\\ 
128 & 128 & 128 & 1.02 & 1.05\\
256 & 256 & 256 & 1.03 & 1.06\\
512 & 512 & 512 & 1.10 & 1.16\\
\hline
\end{tabular}
\end{minipage}
\begin{minipage}{0.45\linewidth}
\centering
\caption*{mMaskAdd}
\begin{tabular}{|c|c|c|c|c|c|c|} 
\hline
\rowcolor{LightBlue}
\multicolumn{3}{|c|}{Matrices dimensions} & \multicolumn{2}{c|}{Ratio ($\frac{FHW}{FHW_{distilled}}$)}\\
% \rowcolor{LightBlue}
\hline
m1 & m2 & m3 & writes & reads\\ 
\hline
64 & 64 & 64 & 1.13 & 1.26\\ 
128 & 128 & 128 & 1.06 & 1.11\\
256 & 256 & 256 & 1.08 & 1.09\\
512 & 512 & 512 & 1.10 & 1.16\\
\hline
\end{tabular}
\end{minipage}
\begin{minipage}{0.45\linewidth}
\centering
\caption*{mMapAdd}
\begin{tabular}{|c|c|c|c|c|c|c|} 
\hline
\rowcolor{LightBlue}
\multicolumn{3}{|c|}{Matrices dimensions} & \multicolumn{2}{c|}{Ratio ($\frac{FHW}{FHW_{distilled}}$)}\\
% \rowcolor{LightBlue}
\hline
m1 & m2 & m3 & writes & reads\\ 
\hline
64 & 64 & --- & 1.10 & 1.21\\ 
128 & 128 & --- & 1.07 & 1.14\\
256 & 256 & --- & 1.07 & 1.19\\
512 & 512 & --- & 1.10 & 1.21\\
\hline
\end{tabular}
\end{minipage}
\hfill
\begin{minipage}{0.45\linewidth}
\centering
\caption*{mMapKron}
\begin{tabular}{|c|c|c|c|c|c|c|} 
\hline
\rowcolor{LightBlue}
\multicolumn{3}{|c|}{Matrices dimensions} & \multicolumn{2}{c|}{Ratio ($\frac{FHW}{FHW_{distilled}}$)}\\
% \rowcolor{LightBlue}
\hline
m1 & m2 & m3 & writes & reads\\ 
\hline
2 & 64 & --- & 1.71 & 1.88\\ 
2 & 128 & --- & 1.72 & 1.87\\
2 & 256 & --- & 1.65 & 1.83\\
4 & 128 & --- & 1.81 & 1.91\\
\hline
\end{tabular}
\end{minipage}

\caption{Memory accesses}
\label{tab:mem_results}
\end{table}

\begin{table}[h]
\scriptsize
\centering
\begin{tabular}{|l|c|c|c|c|c|c|c|c|c|} 
\hline
\rowcolor{LightBlue}

{Benchmark} & \multicolumn{8}{c|}{Ratio (${\frac{FHW}{FHW_{distilled}}}$)} & {Frequency}\\
\hline
{} & FDRE & LUT3 & LUT6 & LUT5 & LUT4 & LUT2 & RAMB36E2 & MUXF7 & {} \\
% \rowcolor{LightBlue}
\hline
mAddAdd & 0.3 & 0.3 & 0.3 & 0.5 & 0.3 & 0.3 & 0.5 & 0.5 & 200 MHz\\ 
mMaskAdd & 0.5 & 0.5 & 0.7 & 0.4 & 0.7 & 0.5 & 0.7 & 0.6 & 200 MHz\\
mMapAdd & 1 & 0.9 & 0.9 & 1.2 & 1 & 1.1 & 1.1 & 1.2 & 200 MHz\\
mMapKron & 1.5 & 1.5 & 1.3 & 2 & 2 & 1.8 & 1.4 & 1.7 & 200 MHz\\
\hline
\end{tabular}
\caption{Resource utilization}
\label{tab:resource_util}
\end{table}
\pagebreak

\section{Conclusion and Future Work}

We present !!!

Our evaluation shows that !!!

First direction for future research is a more detailed CFPQ algorithms investigation.
We should do More evaluation on sparse matrices on GPGPUs.

Also it is nesessary to implement and evaluate solutions for graphs which is not fit in RAM.
There is a set of technics for huge matrices multiplication.
Is it possible to dopt it for CFPQ

Another direcion is a dataset improvement.
More data.
More grammars/queries.


%
\bibliographystyle{splncs04}
\bibliography{main}

\end{document}
