\section{Conclusion and Future Work}
\label{section_conclusion}

In this paper, we have shown, how the context-free path query evaluation w.r.t. relational query semantics can be reduced to the calculation of matrix transitive closure. Also, we introduced the matrix-based algorithm for computing this transitive closure, which allows us to efficiently apply GPGPU computing techniques. In addition, we provided a formal proof of the correctness of the proposed algorithm. Finally, we have shown the practical applicability of the proposed algorithm by running different implementations of our algorithm on some conventional benchmarks.

The active use of matrix operations (such as matrix multiplication) in the proposed algorithm makes it possible to efficiently apply a wide class of matrix optimizations and computing techniques (GPGPU, parallel processing, sparse matrix representation, distributed-memory computation, etc.)

We can identify several open problems for future research. In this paper, we have considered only one semantics of context-free path querying, but there are other important semantics, such as single-path and all-path query semantics~\cite{hellingsPathQuerying}. Context-free path querying, implemented with the algorithm~\cite{GLL}, can process the queries in all-path query semantics by constructing a parse forest. It is possible to construct a parse forest for a linear input by matrix multiplication~\cite{okhotin_cyk}. Whether it is possible to generalize this approach for a graph input is an open question.

In our algorithm, we calculate the matrix transitive closure naively, but there are algorithms for the transitive closure calculation, which are asymptotically better. Therefore, the question is if it is possible to apply these algorithms for matrix transitive closure calculation for context-free path querying.

Also, there are conjunctive~\cite{okhotinConjAndBool} and Boolean grammars~\cite{okhotinBoolean}, which have more expressive power, than context-free grammars. Path querying with conjunctive and Boolean grammars is known to be undecidable~\cite{hellingsRelational}, but our algorithm can be trivially generalized to work for these grammars because parsing with conjunctive and Boolean grammars can be expressed by matrix multiplication~\cite{okhotin_cyk}. It is not clear, how the results of our algorithm can be interpreted in this case. Our conjecture is that it produces an upper approximation of the solution. Also, path querying problem w.r.t. conjunctive grammars can be applied to static code analysis~\cite{zhang2017context}.

\FloatBarrier
