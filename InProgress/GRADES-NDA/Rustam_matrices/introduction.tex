\section{Introduction}
Graph data models are widely used in many areas, for example, graph databases~\cite{graphDB}, bioinformatics~\cite{Bio}. In these areas, it is often required to process queries for large graphs. The most common among graph queries are navigational queries. The result of query evaluation is a set of implicit relations between nodes of the graph, i.e. paths in the graph. A natural way to specify these relations is by specifying paths using formal grammars (regular expressions, context-free grammars) over the alphabet of edge labels. Context-free grammars are actively used in graphs queries because of the limited expressive power of regular expressions. For example, classical \textit{same-generation queries}~\cite{FndDB} cannot be expressed by regular expressions.

The result of context-free path query evaluation is usually a set of triples $(A, m, n)$ such that there is a path from the node $m$ to the node $n$, whose labeling is derived from a non-terminal $A$ of the given context-free grammar. This type of query is evaluated using the \textit{relational query semantics}~\cite{hellingsRelational}. There is a number of algorithms for context-free path query evaluation using this semantics~\cite{GLL, hellingsRelational, RDF, GraphQueryWithEarley}.

The existing algorithms for context-free path query evaluation w.r.t. this semantics demonstrate poor performance when applied to big data. One of the most common technique for efficient big data processing is \textit{GPGPU} (General-Purpose computing on Graphics Processing Units), but these algorithms do not allow to use this technique efficiently. The algorithms for context-free language recognition had a similar problem until Valiant~\cite{valiant} proposed a parsing algorithm which computes a recognition table by computing matrix transitive closure. Thus, the active use of matrix operations (such as matrix multiplication) in the process of a transitive closure computation makes it possible to efficiently apply GPGPU computing techniques~\cite{matricesOnGPGPU}.

One of the open problems in this area is to generalize Valiant's matrix-based approach to the context-free path query evaluation. Valiant's algorithm computes the transitive closure of an upper triangular matrix by increasing the length of the paths considered. We use another definition of the matrix transitive closure which does not depend explicitly on the path length, since the cyclic graphs contain paths of infinite length. Additionally, we compute the transitive closure of an arbitrary matrix, since the context-free path query evaluation requires to process arbitrary graphs. 

We address the problem of creating a matrix-based algorithm for context-free path query evaluation using the relational query semantics which allows us to speed up computations with GPGPU.

The main contribution of this paper can be summarized as follows:
\begin{itemize}
	\item We show how the context-free path query evaluation w.r.t. the relational query semantics can be reduced to the calculation of matrix transitive closure.
	\item We introduce a matrix-based algorithm for context-free path query evaluation w.r.t. the relational query semantics which is based on matrix operations that make it possible to speed up computations by means of GPGPU.
	\item We provide a formal proof of correctness of the proposed algorithm.
	\item We show the practical applicability of the proposed algorithm by running different implementations of our algorithm on some popular ontologies.
\end{itemize}
