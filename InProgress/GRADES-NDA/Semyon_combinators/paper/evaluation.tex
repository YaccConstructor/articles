\section{Evaluation}

Some experiments on real data and comarison with existing solutions

Comparison with GLL

Comparison with ~\cite{ScalaGraphParsing}

\subsection {Classical RDFs}

\textbf{Query 1} is based on the grammar for retrieving concepts on the same layer (presented in figure~\ref{grammarQ1}).
For this query our algorithm demonstrates up to 1000 times better performance and provides identical results as compared to the presented in~\cite{CFGonRDF} for $Q_1$. 


\begin{figure}[ht]
%   \begin{center}
   \centering
%   \begin{subfigure}[b]{0.4\textwidth}

   \[
\begin{array}{rl}
   0: & S \rightarrow \text{\textit{subClassOf}}^{-1} \ S \ \text{\textit{subClassOf}} \\ 
   1: & S \rightarrow \text{\textit{type}}^{-1} \ S \ \text{\textit{type}} \\ 
   2: & S \rightarrow \text{\textit{subClassOf}}^{-1} \ \text{\textit{subClassOf}} \\ 
   3: & S \rightarrow \text{\textit{type}}^{-1} \ \text{\textit{type}} \\ 
\end{array}
\]
   \caption{Grammar for query 1}
   \label{grammarQ1}
   \end{figure}
%   \hspace{2em}


\textbf{Query 2} is based on the grammar for retrieving concepts on the adjacent layers (presented in figure~\ref{grammarQ2}). 
Note that this query differs from the original query $Q_2$ from the paper~\cite{CFGonRDF} in the following points.
First of all, we count only triples for the nonterminal $S$ because only paths derived from it correspond to the paths between concepts on adjacent layers.
Algorithm presented in~\cite{CFGonRDF} returns triples for all nonterminals.
Moreover, the grammar $\mathcal{G}_2$ presented in~\cite{CFGonRDF}, describes paths not only between concepts on adjacent layers.
For example, path ``$\text{\textit{subClassOf} \textit{subClassOf}}^{-1}$'' can be derived in $\mathcal{G}_2$, but it is a path between concepts on the same layer, not adjacent.
We changed the grammar to fit the query to the description provided in the paper~\cite{CFGonRDF}. 
Thus results of our query differs from results for $Q_2$ which can be found in~\cite{CFGonRDF}.


\begin{figure}[h]%{0.4\textwidth}
   \centering
   \[
\begin{array}{rl}
   0: & S \rightarrow B \ \text{\textit{subClassOf}} \\ 
   0: & S \rightarrow \text{\textit{subClassOf}} \\ 
   1: & B \rightarrow \text{\textit{subClassOf}}^{-1} \ B \ \text{\textit{subClassOf}} \\
   2: & B \rightarrow \text{\textit{subClassOf}}^{-1} \ \text{\textit{subClassOf}} \\ 
\end{array}
\]
   \caption{Grammar for query 2}
   \label{grammarQ2}        
   \end{figure}



Integration with Neo4J

\subsection{Static code analysis}

Contex-free language reachabiliy framework.

Graph represenataion of program may be stored in graph DB (refs to!!!)

Alias analysis.


