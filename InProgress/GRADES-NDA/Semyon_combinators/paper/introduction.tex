\section{Introduction}

Creation of data-centric applications which use graph structured data (graph data bases, social graph, code analysis applications) requires 
developed in general-purpose languages 
special languages for graph traversing/querying (SPARQL, cypher, gremlin, etc)
, data access, languages integration for graph-structured data (or graph DB) access.

One of type of avigation queries is a language-constarined path querying~\cite{!!!}.
Many of languages allow users to specify regular constraints, but some tasca requerd more powerfull -- context-free constarints.
For example, classical same generation query is not a regular, also CFL reacability can be used for static code analysis.
Not only reachability information, but strucure of results: for debugging and futher processing.

Integration with general purpose programming languages is a classical problem.
Similar to problem with SQL : correctnes, type safety, etc. Special DSL vs Combinators (LINQ~\cite{LINQ1, LINQ2}, etc)
cfSparql --- separated language
String-embedded DSLs.
Language may be specified wit grammar.
In this area parer combinator technique is a classiacal alternative for specialized DSLs for grammar specification. 
An idea to use combinators for graph traversing proposed in~\cite{ScalaGraphParsing}, but has some problems: cycles, left-recursive grammars

Combinators for arbitrary grammars on GLL and GLL for graphs. 

We show how to compose this ideas and get general solution for arbitrary CF grammars combinators, structural representation.
We make the following contributions in this paper.
\begin{enumerate}
\item Combinators for CF path querying with structural representation of result.
 Transparent integration of query language into general-purpose programming language. Compositionality (subquerying mechanism)
\item Implementation in Scala. Generalization of linear parsing. Integration with Neo4J graph data base. Available on gitHub:\url{https://github.com/YaccConstructor/Meerkat}
\item Evaluation on realistic data, which shows that it is applicable. Comparison  with other tools 
for CF path querying.
\end{enumerate}