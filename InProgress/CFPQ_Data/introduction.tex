\section{Introduction}
\textbf{Здесь надо написать про то, что такое CFPQ и зачем оно надо.}
\textbf{А также про то, почему надо собрать коллекцию графов в одном месте}
Представление данных с помощью помеченных графов находит своё применение в биоинформатике [12], в статическом анализе кода и многих других областях. 
Всё более популярными становятся графовые базы данных [9]. 
При работе с такими данными зачастую возникают запросы навигации и поиска путей, удовлетворяющих заданным ограничениям. 
Результат обработки такого рода запросов, как правило, представляет собой набор отношений между вершинами графа. 
Один из естественных способов определить подобные отношения над помеченным графом — указать соответствующие пути, используя формальные грамматики над алфавитом меток рёбер. 
Такие отношения могут быть заданы с помощью регулярных или контекстно-свободных языков [17].
Поскольку путь в помеченном графе можно рассматривать как слово в формальном языке, то соответствующие запросы навигации естественным образом могут быть выражены с помощью контекстно-свободных грамматик. 
Таким образом встает вопрос о необходимости разработки и реализации алгоритмов поиска путей с контекстно~-свободными ограничениям (далее CFPQ алгоритмы).
Ввиду широкой применимости контекстно-свободных запросов в перечисленных выше практических областях, критически важной становится потребность в измерении производительности алгоритмов реализующих эти запросы. 
Для того, чтобы показать применимость алгоритма на практике, возникает необходимость проведения экспериментального анализа на данных, моделирующих реальные сценарии. 
Также это позволяет исследователям сравнивать производительность предлагаемого ими решения с уже существующими.
Однако поиск и подготовка необходимых для проведения экспериментального анализа данных могут занять весьма продолжительное время. 
Одним из решений таких проблем во многих областях исследований является использование единого стандартизированного набора данных. 
Например, в биоинформатике очень важно иметь набор данных для проверки производительности алгоритмов кластеризации [3] и проекции данных [14]. 
А в области машинного обучения необходимо иметь стандартный набор данных, позволяющий исследователям выбирать какой метод лучше подходит для решения конкретной задачи[10]. 
В области алгоритмов, реализующих контекстно-свободные запросы к помеченным графам, на данный момент в большинстве работ, даже недавних, эксперименты проводятся на фиксированном наборе мелкомасштабных, неразнообразных графов, с использованием нестандартизованных экспериментальных протоколов и базовых показателей, что затрудняет сравнение результатов из разных публикаций.

\subsection{Present work}
\textbf{Здесь мы даем обзор CFPQ\_Data.}
Коллекция состоит из более чем 40 графов из широкого диапозона областей.
Все графы представлены в стандартном формате RDF на сайте коллекции. \href{https://github.com/JetBrains-Research/CFPQ_Data}{https://github.com/JetBrains-Research/CFPQ\_Data}.
Для облегчения работы с данными мы предоставляем загрузчики данных и реализации наиболее популярных алгоритмов на основе Python, а также стандарт проведения экспериментов и базовых показателей работы алгоритма.
Кроме того, мы сообщаем результаты экспериментального исследования, сравнивающего наиболее популярные алгоритмы в этой области.

\subsection{Related work}
\textbf{Здесь пишем про уже имеющиеся работы в области.}
\textbf{А также про представленные в работах графы.}
Так например про RDF надо написать, что-то вроде Small graphs is a set of popular semantic web ontologies. This set is introduced by Xiaowang Zhang in "Context-Free Path Queries on RDF Graphs".
Потом еще про MemoryAliases что-то вроде MemoryAliases — real-world data for points-to analysis of C code.
First part is a dataset form Graspan tool. The original data is placed here. This part is placed in Graspan folder.
Second part is a part of dataset form "Demand-driven alias analysis for C". This part is placed in small folder.
И не забыть про синтетические графы.
WorstCase — graphs with two cylces; the query Brackets is a grammar for the language of correct bracket sequences.
SparseGraph — graphs generated with NetworkX to emulate sparse data. The grammar provided is a variant of the same-generation query.
ScaleFree — graphs generated by using the Barab'asi-Albert model of scale-free networks. Use with grammar ** an\_bm\_cm\_dn**, which is a query for AnBmCmDn language.
FullGraph — cycle graphs, all edges are labeled with the same token. Use with A\_star queries, which produce full graph on that dataset.
LUBM - the Lehigh University Benchmark graphs.