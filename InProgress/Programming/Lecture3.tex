\section{Лекция 3}

	Ещё раз произменения: в реквесте должно быть только то, что непосредственно относится к сдаваемой домашке.

	Ещё раз про функции, про то, как выделять и разделять функциональность, не надо запихивать всё в одну функцию. Про то, где должны быть проверки.

	Про консоль.

	Про обработку крайних случаев. Про исключения.

	Про тесты и ошибки: нашёл ошибку --- создал тест.

	Про стиль кодирования: про пробелы вокруг скобок и операций, про отступы и переводы строк. Про соглашения о наименовании. camlCase CamlCase


	Про единицы измерения.

    Базовые структуры данных, алгоритмы и их выражение в F\#. Функция. Рекурсия и итерация.  Базовые типы и основы работы с ними: матрицы, массивы, списки, структуры. 


    Числа Фибоначчи.