\section{Лекция3. Контекстно-свободные грамматики}

    4. Отладка кода. Некоторые методы отладки: отладочная печать, логгирование, использование пошаговых отладчиков. Некоторые шаги отладки: формулировка гипотезы и её проверка, локализация ошибки, работа с тестами. Практика по использованию отладчика.
    5. Структура проекта вообще и на языке F# в частности. Разбиение кода на модули, файлы, библиотеки. Переиспользование кода. Зависимости между модулями, библиотеками. Пространства имён.
    6. Базовые структуры данных, алгоритмы и их выражение в F#. Функция. Ветвления, рекурсия и итерация. Изменяемость/неизменяемость, понятие переменной, понятие связывания. Базовые типы и основы работы с ними: численные типы, логические, строки, матрицы, массивы, списки, структуры. 
Домашняя работа 3
    7. Сортировки: пузырьком, вставкой, Хоара. Различные сценарии использования: поддержание отсортированного набора, сортировка всего набора целиком. Некоторые особенности реализации: наивная функциональная реализация Хоара, реализация на массиве.
    8. Основы машинного представления данных. Представления чисел. Представление чисел с плавающей точкой. Проблемы переполнения. Битовые операции. Строки, кодировки.
Домашняя работа 4
    9. Основы анализа алгоритмов. Модель вычислителя. Понятие элементарной операции. Асимптотика, «О»-символика.
    10. Постановка эксперимента и оформление результатов. Эксперимент по сравнению и анализу производительности.  Точность проведения замеров. Особенности работы с управляемыми средами (JIT, сборка мусора). «Масштабы времени», цель эксперимента и точность измерений, инструменты измерений. Базовая статистическая обработка данных. Способы визуализации результатов.  
Домашняя работа 5
    11. Контрольная работа.
    12. Разбор домашних работ, разбор контрольной.
Раздел 2: Структуры данных и алгоритмы.
    1. Понятие типа данных. Системы типов: статические, динамические, строгие, нестрогие. Примеры языков с разными системами типов. Понятие о разной выразительности («мощности») систем. Приведение типов: автоматическое, ручное. Вывод типов по Хиндли-Милнеру.
    2. Алгебраические типы данных: кортежи, DU. Примеры на F#. Единицы измерения.
Домашняя работа 6
    3. Обобщённые типы данных. Понятие о полиморфизме. Ad-hoc полиморфизм и бинарные операции. Типовые параметры и ограничения на них в F#. Структурный полиморфизм в Ocaml.
    4. Списки, деревья: как формальные объекты, структуры данных и как примеры алгебраических обобщённых типов. Реализация списка и дерева. Обходы списков и деревьев. 
Домашняя работа 7
    5. Длинная арифметика. Практика работы со списками. Ещё раз о проблеме переполнения. Целочисленная арифметика на списках. 
    6. Граф как формальный объект и как структура данных. Понятие о бинарном отношении и его свойствах: транзитивность, рефлексивность. (Не)Ориентированные, (не)помеченные графы. Способы представления графов: список смежности, матрица смежности.
Домашняя работа 8
    7. Базовые алгоритмы на графах. Обходы в глубину и ширину, построение транзитивного замыкания, поиск кратчайшего пути.
    8. Регулярные выражения и конечные автоматы. Определения. Построение автомата по регулярному выражению. Применения регулярных выражений (поиск, анализ текста, моделирование систем, анализ программного кода).
    9. Линейная алгебра. Основы: матрица, вектор, полукольцо, кольцо, поле. Сведение некоторых задач к операциям линейной алгебры (транзитивное замыкание, кратчайшие пути, пересечение автоматов). Особенности практического использования такого подхода: разреженные структуры данных, абстрактность, композициональность.
Домашняя работа 9
    10. Контрольная работа.
    11. Разбор домашних работ, разбор контрольной.
Раздел 3: Парадигмы программирования
    1. Устройство языков программирования: лексика, синтаксис, семантика. Определения, примеры. Шаги обработки кода: лексический и синтаксический анализы, «семантический» анализ.
    2. Лексический и синтаксический анализ. Введение в формальные языки как способ описания синтаксиса. Способы реализации лексических и синтаксических анализаторов. ANTLR, fslex+fsyacc, FParsec, YaccConstructor.
Домашняя работа 10
    3. Устройство сред разработки и компиляторов, интерпретаторов: общие шаги, классические возможности, JIT/AOT. Примеры из .NET, F#, JVM, LLVM.
    4. Интерпретация и компиляция. Особенности, разновидности интерпретаторов, основные шаги.  Особенности, разновидности компиляторов, основные шаги. Примеры, пример реализации простого интерпретатора.
Домашняя работа 11 
    5. Парадигмы программирования-1. Структурное программирование: машины Тьюринга, архитектура фон Неймана, языки-представители. Объектно-ориентированное программирование, основные понятия, инкапсуляция, наследование, полиморфизм. Языки-представители. Пример объектно-ориентированного кода на F#. Функциональное программирование. Понятие лямбда-исчисления, основные принципы и особенности функционального программирования. Языки представители, Haskell, F#, Ocaml. Программирование в зависимых типах.
    6. Парадигмы программирования-2. Логическое программирование, Пролог. Рекурсивное программирование, Рефал. Стековое программирование, Форт. Визуальное программирование, визуальное моделирование, UML, предметно-ориентированное моделирование.

Период обучения (модуль): семестр 2.
№ п/п
Наименование темы (раздела, части)
Вид учебных занятий
Количество часов
I.
Программный продукт, проект
практические занятия
4


лабораторные работы
4


контрольные работы
0


самостоятельная работа
8
II.
Объектно-ориентированное программирование
практические занятия
4


лабораторные работы
4


контрольные работы
0


самостоятельная работа
8
III.
Функциональное программирование
практические занятия
2


лабораторные работы
2


контрольные работы
2


самостоятельная работа
6
IV.
Параллельное программирование
практические занятия
1


лабораторные работы
7


контрольные работы
2


самостоятельная работа
10
V.
Промежуточная аттестация
самостоятельная работа
8


зачёт
2

Раздел 1: Программный продукт, проект.
    1. Программа, проект, продукт – что есть что, различия. Жизненный цикл продукта.
    2. Открытый исходный код: окружение, инструменты, лицензии. Экосистема проектов с открытым исходным кодом. Непрерывная интеграция: задачи, облачный сервис AppVeyor, настройка сборки, матрица сборки. Облачный сервис Travis. Инструменты анализа качества, линтеры, покрытие тестами. Инструменты планирования и управления проектом: Trello, Pivotal Tracker. Средства коммуникации: Slack, Gitter. Багтрекер GitHub Issues. Другие средства управления проектом GitHub. Авторское право и лицензии.
    3. Документация, комментирование, автоматическая генерация документации по комментариям. Публикация документации на gh-pages.
    4. Визуальное моделирование, UML. Метафора моделирования, цель моделирования. Диаграммы UML. Диаграмма классов: синтаксис, синтаксис свойств, агрегация и композиция. Диаграмма компонентов. Диаграмма случаев использования. Диаграммы активностей, последовательностей, конечных автоматов. Генерация кода по диаграммам конечных автоматов. Диаграммы развёртывания. Примеры CASE-инструментов. Предметно-ориентированные визуальные языки.
Домашняя работа 1
Раздел 2: Объектно-ориентированное программирование
    1. Основы. Инкапсуляция и наследование. Интерфейс. Множественное наследование и множественная реализация интерфейса. Абстрактный класс. Примеры ООП на F#. 
    2. Некоторые базовые паттерны проектирования.
    3. Исключения и обработка ошибок. Бросание и обработка исключений. Перебрасывание исключений. Объявление своих классов-исключений. Некоторые особенности использования исключений (легковесность в OCaml vs тяжеловесность в .NET).
    4. GUI (winforms, GTK и т.д). Основы событийно-ориентированного программирования. Основы разработки GUI (вёрстки).
Домашняя работа 2
Раздел 3: Функциональное программирование
    1. Основы проектирования с использованием ФП. Сравнение ФП и ООП дизайна и их совмещение.  Railway programming как способ жить без исключений. 
    2. Особенности чистого функционального программирования. Неизменяемые структуры данных, проблемы с эффективностью и их возможные решения. Отсутствие побочных эффектов.
Домашняя работа 3 
    3. Контрольная работа.
Раздел 4: Параллельное программирование
    1. Архитектуры, подходы, парадигмы. SIMD, MIMD, SPMD. Асинхронное программирование, параллельное программирование. Процессы и потоки: многопроцессорность и многопоточность. Гонки по данным, блокировки.  
    2. Базовые примитивы работы с потоками и разделяемыми ресурсами в F#. Функция lock. Запуск функции в отдельном потоке. Особенности работы с исключениями. Общее состояние. Плюсы и минусы неизменяемости.
    3. Array.Parallel, ParallelSeq и другие высокоуровневые средства параллельного программирования на F#. Линейная алгебра и параллелизм: бонусы, проблемы, возможные решения.
Домашняя работа 4
    4. Actor-ориентированное программирование как реализация асинхронного (concurrent) программирования. Коммуникация на сообщениях. Mailbox processor и Hopac как реализации. Примеры использования Mailbox processor и Hopac.
Домашняя работа 5
    5. Контрольная работа.

Период обучения (модуль): семестр 3.
№ п/п
Наименование темы (раздела, части)
Вид учебных занятий
Количество часов
I.
Особенности исследовательских проектов
практические занятия
4


лабораторные работы
4


самостоятельная работа
10
II.
Продвинутые техники программирования
практические занятия
7


лабораторные работы
7


самостоятельная работа
15
III.
Программирование на GPGPU
практические занятия
4


лабораторные работы
4


самостоятельная работа
12
IV.
Промежуточная аттестация
самостоятельная работа
3


зачёт
2

Раздел 1: Особенности исследовательских проектов.
    1. Жизненный цикл исследовательских проектов и возможные пути развития. Отличие от прикладных/промышленных проектов и сходство с ними. Цели и задачи исследовательских проектов. 
    2. Эксперименты: воспроизводимость, анализ результатов, оформление результатов. Примеры соответствующих инструментальных средств. Цели и задачи экспериментов. Экспериментальное исследование, сравнение, проверка гипотезы.
Домашняя работа 1
    3. Оформление результатов в виде текста: статья, технический отчёт и другие типы текстов. Особенности процесса их написания. Типичная структура и особенности. Курсовая/диплом как научная работа и отчёт по ним как научный текст.
    4. Представление результатов в виде презентации, доклада. Различные виды докладов, презентаций. Типичные структуры. Пример презентации для курсовой.
Домашняя работа 2
Раздел 2: Продвинутые техники программирования 
    1. Метапрограммирование вообще и в F# в частности. Понятие метапрограммирования. Подходы к реализации техник метапрограммирования.  Программирование времени выполнения и времени компиляции. Примеры: шаблоны в С++ (тьюринг-полнота), системы макросов, вычислимые выражения. Встраивание языков.
    2. Рефлексия, вообще и в .NET. Сборки в .NET, сильные и слабые имена сборок. Получение информации о сборках, типах, полях, методах и т.д., создание экземпляров объектов, вызов методов. Компиляция F#-кода во время выполнения.
Домашняя работа 3
    3. Разбор домашней работы 2   
    4. Поставщики типов: их типы, решаемые с их помощью задачи. Особенности использования. Особенности создания. Примеры готовых поставщиков и их использования.
    5. F# quotations. Трансформация кода во время выполнения. Возможности и ограничения. Типизированный и нетипизированный варианты. Примеры использования.
    6. Событийно-ориентированное программирование, реактивное программирование RxExtension.
Домашняя работа 4
    7. Проверочная работа
Раздел 3: Программирование на GPGPU
    1. Основы. Архитектура, логическая модель вычислителя, плюсы/минусы, сферы применения.
    2. OpenCL, логическая и физическая модели, переносимость, ядра, атомарные операции, сравнение с CUDA. Основы языка OpenCL C. Примеры простых ядер.
    3. Программирование на GPGPU с использованием высокоуровневых средств. Программирование GPGPU на F# как пример метапрограммирования. Особенности и проблемы использования высокоуровневых средств.
Домашняя работа 5
    4. Проверочная работа.
