\section{Домашняя работа 9}

В задачах ниже необходимо максмально переиспольховать результаты предыдущих домашних работ. Для реализации синтаксического нализатора можно использовать не только FsYacc, рассматриваемый на паре, но и другие инструменты (например FParsec). Нельзя писать анализатор руками.

\begin{enumerate}
  \item \textbf{(4 балла)} Разработать библиотеку конечных автоматов, использующую разреженные матрицы из пердыдущей работы для пердставления переходов автомата, и предоставляющую следующие возможности.
  \begin{itemize}
    \item Построение автомата по регулярному выражению. Можно ограничится заданием регулярного выражения через конструкторы типа.
    \item Построение пересечения двух автоматов.
    \item Возможность проверять, принимается ли строка автоматом.
  \end{itemize}
  \item \textbf{(7 баллов)} Релизовать синтаксический анализатор регулярных выражений, позволяющий в предыдущей задаче задавать регулярное выражение как строку. поддерживаемые операции регулярных выражений: конкатенация, альтернатива, звезда Клини. Алфавит регулярных выражений --- строчные и прописные латинские символы, цифры, арифметические знаки, знаки припенания. Обеспечить построение автомата по регулярному выражению. Предусмотреть возможность задавать регулярное выражение с консоли, а на выход получать представление автомата в DOT.
  \item \textbf{(7 баллов)} Реализовать синтаксический анализатор для арифметических выражений над целыми числами. Поддерживаемые операции: сложение, умножение, вычитание, деление. Также бывают группирующие скобки. Числа могут быть очень большими (но целыми). Реализовать вычисление значения выражения на основе операций длинной арифметики. Предусмотреть возможность задавать выражение с консоли, а на выход получать результат его вычисления (в консоль) и дерево разбора в формате DOT.
  \item \textbf{(8 баллов)} Расширить язык регулярных выражения следующими конструкциями.
  \begin{itemize}
    \item Операцией пересечения, повторения один или более раз, повторения 0 или 1 раз. Все эти операции могут встречаться в произвольном месте выражения.
    \item Функцией проверки, что строка принадлежит языку, задаваемому выражением. 
    \item Функцией поиска всех подстрок, удовлетворяющих заданному регулярному выражению.
    \item Функцией печати атомата, задаваемого выражением, в файл в формате DOT. 
    \item Функцией печати результата в консоль.
    \item Переменными. Переменные могут использоваться в правых частях всех выражений.
  \end{itemize} 
  Реализовать интерпретатор получившегося языка. Предусмотреть возможность его консольного запуска: на входе файл с кодом на нашем языке, на выходе --- результат интерпретации.
  \item \textbf{(8 баллов)} Расширить язык арифметики следующими конструкциями.
  \begin{itemize}
    \item Операцией возведения в степень, взятия остатка (от целочисленного деления), модуля, добавить унарный минус.
    \item Функцией перевода числа в двоичную систему исчисления. 
    \item Функцией печати результата в консоль.
    \item Переменными. Переменные могут использоваться в правых частях всех выражений.
  \end{itemize} 
Реализовать интерпретатор получившегося языка. Предусмотреть возможность его консольного запуска: на входе файл с кодом на нашем языке, на выходе --- результат интерпретации.
 
\end{enumerate}
