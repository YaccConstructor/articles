\section{Домашняя работа 7 }

%По идее, длинную арифметику можно приклеить сюда же. И на лекциях испеваю и домашка логично продолжается.

\begin{enumerate}
    \item \textbf{(1 балл)} Реализовать самостоятельно полиморфный непустой список (далее будем называть этот тип \texttt{MyList}). Реализовать для него функции сортировки, вычисления длины, конкатенации, map, iter. Реализовать преобразование из стандартного списка в MyList.
    \item \textbf{(1 балл)} На основе \texttt{MyList} реализовать тип \texttt{MyString}, представляющий строку как список символов. Реализовать преобразование стандартной строки в \texttt{MyString} и конкатенацию строк для  \texttt{MyString}.
    \item \textbf{(1 балл)} Реализовать тип дерева с произвольным количеством потомков в каждом узле (использовать \texttt{MyList}) \texttt{MyTree}. Каждый узел должен хранить данные произвольного типа.

    \item \textbf{(1 балл)} Пусть есть \texttt{MyTree}, хранящий в узлах целые числа. Реализовать функции, которые находят максимальный хранимый элемент, среднее значение всех хранимых элементов.
\end{enumerate}