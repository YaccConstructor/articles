\section{Лекция 10}

    	Рассказать про иерархию типов и про коллекции (Seq)

	Дальше будет много про формальные языки. Неплохой конспект на русском: \url{https://neerc.ifmo.ru/wiki/index.php?title=%D0%9A%D0%B0%D1%82%D0%B5%D0%B3%D0%BE%D1%80%D0%B8%D1%8F:%D0%A2%D0%B5%D0%BE%D1%80%D0%B8%D1%8F_%D1%84%D0%BE%D1%80%D0%BC%D0%B0%D0%BB%D1%8C%D0%BD%D1%8B%D1%85_%D1%8F%D0%B7%D1%8B%D0%BA%D0%BE%D0%B2}. Там, правда, без линейной алгебры.

    Регулярные выражения и конечные автоматы. Определения. Построение автомата по регулярному выражению. Применения регулярных выражений (поиск, анализ текста, моделирование систем, анализ программного кода).

    Устройство языков программирования: лексика, синтаксис, семантика. Определения, примеры. Шаги обработки кода: лексический и синтаксический анализы, ``семантический'' анализ.
    
    Лексический и синтаксический анализ. Введение в формальные языки как способ описания синтаксиса. Контекстно-свободные граммтики как система переписываний. 
    Способы реализации лексических и синтаксических анализаторов. \href{https://www.antlr.org/}{ANTLR}, \href{https://fsprojects.github.io/FsLexYacc/}{fslex+fsyacc}, \href{https://www.quanttec.com/fparsec/about/}{FParsec}.
