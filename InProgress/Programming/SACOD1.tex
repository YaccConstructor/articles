\section{Основы обработки изображений}

\subsection{Форматы изображений}

Векторный, растровый.

Нас интересует растровый.

И для простоты сразу битмапа (bmp): двумерный массив пикселей, где для каждого хранится цвет.
Естественно, ещё и метаданные вокруг, но они, в основном, про то, как ситать файл, а не про само изображение.
Цвет --- либо RGB, либо градации серого (Grayscale). На пиксель от 1 до 64 бит. В grayscale 16 или 32.

\href{Bitmap in .NET}{https://docs.microsoft.com/en-us/dotnet/api/system.drawing.bitmap?view=dotnet-plat-ext-5.0}

\subsection{Цифровые фильтры изображений}

Собель для поиска границ, размытие по Гауссу, машинное обучение.

Примеры: 
\begin{itemize}
\item \url{https://www.codingame.com/playgrounds/2524/basic-image-manipulation/filtering}
\item \url{https://lodev.org/cgtutor/filtering.html}
\end{itemize}

\subsection{Домашняя работа 1}

\begin{enumerate}
    \item \textbf{4 балла.} Реализовать приложение с графическим интерфейсом пользователя, позволяющее открыть папку с изображениями, выбрать изображение, просмотреть его, просмотреть информацию о нём (размер в пикселях, размер в мегабайтах).

    \item \textbf{3 балла.} Расширить приложение графической компонентой задания матричного фильтра. Необходимо предусмотреть возможность выбора типа фильтра, дефолтных значений, размера фильтра, корректировку весов.

    \item \textbf{3 балла.} Расширить приложение возможностью отображать одновременно два изображения: до и после применения фильтра. Предусмотреть возможность сохранять результат применения фильтра.

    \item \textbf{8 баллов.} Реализовать применение матричных фильтров с использованием GPGPU. Интегрировать с разработанным графическим интерфейсом. Предусмотреть возможность применения нескольких фильтров последовательно.

    \item \textbf{8 баллов.} Расширить разрабатываемое приложение возможностью потоковой обработки изображений: выбираем папку с изображениями и ко всем применяем заданные фильтры. Результаты применения фильтров сохраняются в отдельную выбранную папку.

    \item \textbf{10 баллов.} Подготовить отчёт с анализом производительности и масштабируемости полученного решения.
\end{enumerate}