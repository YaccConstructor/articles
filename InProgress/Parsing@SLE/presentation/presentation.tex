\documentclass{beamer}
\usepackage{beamerthemesplit}
\usepackage{wrapfig}
\usetheme{SPbGU}
\usepackage{pdfpages}
\usepackage{amsmath}
\usepackage{cmap} 
\usepackage[T2A]{fontenc} 
\usepackage[utf8]{inputenc}
\usepackage[english,russian]{babel}
\usepackage{indentfirst}
\usepackage{amsmath}
\usepackage{tikz}
\usepackage{multirow}
\usepackage[noend]{algpseudocode}
\usepackage{algorithm}
\usepackage{algorithmicx}
\usetikzlibrary{shapes,arrows}
%usepackage{fancyvrb}
%\usepackage{minted}
%\usepackage{verbments}


\newtheorem{rutheorem}{Теорема}
\newtheorem{ruproof}{Доказательство}
\newtheorem{rudefinition}{Определение}
\newtheorem{rulemma}{Лемма}
\beamertemplatenavigationsymbolsempty

\title[]{Parsing techniques for graph analysis}
%\subtitle[YaccConstructor]{Parsing techniques for graph analysis}
% То, что в квадратных скобках, отображается в левом нижнем углу. 
\institute[SPbU]{
JetBrains Research, Programming Languages and Tools Lab  \\
Saint Petersburg University
}

% То, что в квадратных скобках, отображается в левом нижнем углу.
\author[Ekaterina Verbitskaia]{Ekaterina Verbitskaia}

\date{Oktober 22, 2017}

\definecolor{orange}{RGB}{179,36,31}

\begin{document}
{
\begin{frame}[fragile]
  \begin{tabular}{p{2.5cm} p{5.5cm} p{2cm}}
   \begin{center}
      \includegraphics[height=1.5cm]{pictures/JBLogo3.pdf}
    \end{center}
    &
    \begin{center}
      \includegraphics[height=1.5cm]{pictures/SPbGU_Logo.png}
    \end{center}
    &
    \begin{center}
      \includegraphics[height=1.5cm]{pictures/YC_logo.pdf}
    \end{center} 
  \end{tabular}
  \titlepage
\end{frame}
}

\begin{frame}[fragile]
  \transwipe[direction=90]
  \frametitle{YaccConstructor}
  \begin{itemize}
    \item Исследования в области лексического и синтаксического анализа
    \item Открытый исходный код
    \begin{itemize}
      \item \url{https://github.com/YaccConstructor}
    \end{itemize}
    \item Основной язык разработки --- F\#
  \end{itemize}
\end{frame}

\begin{frame}
  \transwipe[direction=90]
  \frametitle{Требования к знаниям и навыкам}
  \begin{itemize}
    \item Знакомство с функциональным программированием (F\#, OCaml, Haskell и т.д.)
    \item Умение читать и понимать научные статьи
    \item Умение читать и понимать чужой код
    \item Базовые знания в области формальных языков и синтаксичского анализа
    \item Навыки работы с Git/GitHub, Microsoft VisualStudio
  \end{itemize}
\end{frame}

\begin{frame}[plain,c]
 \transwipe[direction=90]
 \begin{center}
  \Huge Синтаксический анализ в биоинформатике
 \end{center}
\end{frame}

\begin{frame}[fragile]
\transwipe[direction=90]
\frametitle{Задачи}
\begin{itemize}
\item Цель: создание инструмента для решения различных задач поиска в метагеномных сборках на 
основе синтаксического анализа
\item Задачи
\begin{itemize}
\item Распределённый синтаксический анализ
\item Выравнивание строки на граф через синтаксический анализ
\item Извлечение КС-грамматики (CF grammar inference)
\item Контекстно-свободное сжатие метагеномной сборки
\end{itemize}
\item Подробное описание задач: \url{https://goo.gl/J57Cin}
\item ``Направления развития'': курсовая, диплом, стажировка
\end{itemize}
\end{frame}
            
\begin{frame}
\transwipe[direction=90]
\frametitle{Контакты}
\begin{itemize}
  \item Почта: \url{rsdpisuy@gmail.com}
  \item Исходный код YaccConstructor: \url{https://github.com/YaccConstructor}
  \item Google+ сообщество: \url{https://goo.gl/DuPWkM}
\end{itemize}
\end{frame}
\end{document}
