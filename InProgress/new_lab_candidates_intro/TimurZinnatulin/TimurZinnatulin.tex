\documentclass[xcolor=table,aspectratio=169]{beamer}
\usepackage{beamerthemesplit}
\usepackage{wrapfig}
\usetheme{SPbGU}
\usepackage{pdfpages}
\usepackage{amsmath}
\usepackage{cmap}
\usepackage[T2A]{fontenc}
\usepackage[utf8]{inputenc}
\usepackage[english]{babel}
\usepackage{indentfirst}
\usepackage{mathtools}
\usepackage{tikz}
\usepackage{multirow}
\usepackage[noend]{algpseudocode}
\usepackage{algorithm}
\usepackage{algorithmicx}
\usepackage{fancyvrb}

\usepackage{minted}

\usetikzlibrary{calc}
\usetikzlibrary{shapes,arrows}
\usetikzlibrary{arrows,automata}
\usetikzlibrary{positioning}

\usepackage{fontawesome}

\usetikzlibrary{shapes.callouts}

\usepackage{xparse}

%for [[ ]]
\usepackage{stmaryrd}


\tikzset{
    invisible/.style={opacity=0,text opacity=0},
    visible on/.style={alt=#1{}{invisible}},
    alt/.code args={<#1>#2#3}{%
      \alt<#1>{\pgfkeysalso{#2}}{\pgfkeysalso{#3}} % \pgfkeysalso doesn't change the path
    },
}

\NewDocumentCommand{\mycallout}{r<> O{opacity=0.8,text opacity=1} m m m}{%
\tikz[remember picture, overlay]\node[align=center, fill=cyan!20, text width=#5cm,
#2,visible on=<#1>, rounded corners,
draw,rectangle callout,anchor=pointer,callout relative pointer={(230:1cm)}]
at (#3) {#4};
}

%\newcommand{\tikzmark}[1]{\tikz[overlay,remember picture,baseline=-0.5ex] \node (#1) {};}



\usepackage{tabularx}
\newcolumntype{Y}{>{\raggedleft\arraybackslash}X}

\renewcommand{\thealgorithm}{}

\newtheorem{mytheorem}{Theorem}
\renewcommand{\thealgorithm}{}

\newcommand{\tikzmark}[1]{\tikz[overlay,remember picture] \node (#1) {};}
\def\Put(#1,#2)#3{\leavevmode\makebox(0,0){\put(#1,#2){#3}}}

\newcommand{\ltz}{$< 1$}


\tikzset{
    state/.style={
           rectangle,
           rounded corners,
           draw=black, very thick,
           minimum height=2em,
           inner sep=2pt,
           text centered,
           },
}

\makeatletter
\AtBeginEnvironment{minted}{\dontdofcolorbox}
\def\dontdofcolorbox{\renewcommand\fcolorbox[4][]{##4}}
\makeatother

\beamertemplatenavigationsymbolsempty

\title[Зиннатулин Тимур Раифович]{Зиннатулин Тимур Раифович}
\institute[JB Research, SPbSU]{
JetBrains Research, Programming Languages and Tools Lab  \\
Saint Petersburg State University
}


\author[Семён Григорьев]{Arseniy Terekhov, Vlada Pogozhelskaya, Vadim Abzalov, \\ Timur Zinnatulin, \textbf{Semyon Grigorev}}

\date{24 мая 2021г.}

\begin{document}

\begin{frame}[fragile] \frametitle{Зиннатулин Тимур Раифович}
      \begin{minipage}[m]{0.45\linewidth}
  \raisebox{-0.5\totalheight}{\includegraphics[width=\textwidth]{pictures/ProfilePhoto.jpg}}
  \end{minipage}\hfill
  \begin{minipage}[m]{0.5\linewidth}
  Кандидат на бакалаврский грант

  \vfill

  \begin{itemize}
        \item Работает с нами с сентября 2019 года (2 курс, семестровый проект)
        \item Закончит третий курс Мат-Мех факультета в июне 2021 года
        \item Тема курсовой работы: ``Реализация поиска путей с регулярными и контекстно-свободными ограничениями в графовой базе данных RedisGraph''
        \item Получатель стипендии
        \item Сфера интересов: !!!!
  \end{itemize}
  \end{minipage}

\end{frame}


            


\begin{frame}[fragile] \frametitle{Исследовательская работа}
  
    \begin{itemize}
        \item Соавтор одной опубликованной работы: ``Multiple-Source Context-Free Path Querying in Terms of Linear Algebra''. Arseniy Terekhov, Vlada Pogozhelskaya, Rustam AzimovVadim Abzalov, \textbf{Timur Zinnatulin}, Semyon Grigorev; 2021; International Conference on Extending Database Technology (EDBT)

        \item И одного препринта: ``One Algorithm to Evaluate Them All: Unified Linear Algebra Based Approach to Evaluate Both Regular and Context-Free Path Queries''. Ekaterina Shemetova, Rustam Azimov, \textbf{Egor Orachev}, Ilya Epelbaum, Semyon Grigorev; 2021
        \item Реализовал поддержку запросов с КС ограничениями в синтаксисе Cypher
    \end{itemize}
  \pause
  \vfill
  Планы: !!!
  \begin{itemize}
        \item !!!
        \item !!!
        \item !!!
  \end{itemize}

\end{frame}



\end{document}
