\section{Производные для КС запросов}

\subsection{Производные}

Общая теория.

Определения.

\subsection{Парсинг на производных}

Статьи~\cite{DBLP:journals/corr/abs-1010-5023,Adams:2016:CPP:2908080.2908128,Might:2011:PDF:2034574.2034801,andersenparsing}
Реализации.
На Scala~\footnote{\url{https://github.com/djspiewak/parseback}}, на Racket~\footnote{\url{https://bitbucket.org/ucombinator/derp-3/src/86bca8a720231e010a3ad6aefd1aa1c0f35cbf6b/src/derp.rkt?at=master&fileviewer=file-view-default}}.

\subsection{Адоптация для КС запросов}

Для регулярных запросов над графами~\cite{Nole:2016:RPQ:2949689.2949711}.
Хорошо работают в распределённых системах, в которых реализовван параллелизм уровня вершин. 
Например Google Pregel.



\subsection{Вопросы и задачи}
\begin{enumerate}
  \item Предъявить несколько выводов для одной цепочки.
  \item Построить выводы
  \item Построить деревья вывода !!! Перенести из раздела про SPPF
\end{enumerate}

