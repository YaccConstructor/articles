\section{Через тензорное произведение}

\subsection{Рекурсивные автоматы}

Определение, примеры.

\begin{figure}[h]
\begin{center}
\begin{tikzpicture}[shorten >=1pt,node distance=2cm,on grid,auto] 
   \node[state, initial] (q_0)   {$0 \{S\}$}; 
   \node[state] (q_1) [right=of q_0] {$1$}; 
   \node[state] (q_2) [right=of q_1] {$2$}; 
   \node[state, accepting] (q_3) [right=of q_2] {$3$};
    \path[->] 
    (q_0) edge  node {a} (q_1)          
    (q_1) edge  node {S} (q_2)
    (q_2) edge  node {b} (q_3)
    (q_1) edge[bend left, above]  node {b} (q_3);
\end{tikzpicture}
\end{center}

\caption{Рекурсивный автомат для грамматики!!!}
\label{input}
\end{figure}


\subsection{Тензорное произведение}

Матриц и графов

Сперва дадим классическое определение тензорного произведения двух неориентированных графов.
\begin{definition}
Пусть даны два графа: $\mathcal{G}_1 = \langle V_1, E_1\rangle$ и $\mathcal{G}_2 = \langle V_2, E_2\rangle$. 
Тензорным произведением этих графов будем называть граф $\mathcal{G}_3 = \langle V_3, E_3\rangle$, где $V_3 = V_1 \times V_2$, $E_3 = \{ ((v_1,v_2),(u_1,u_2)) \mid (v_1,u_1) \in E_1 \text{ и } (v_2,u_2) \in E_2 \}$.
\end{definition}

Иными словами, тензорным произведением двух графов является граф, такой что:
\begin{enumerate}
 \item множество вершин --- это прямое произведение множемтв вершин исходных графов;
 \item ребро между вершинами $v=(v_1,v_2)$ и $u=(u_1,u_2)$ существует тогда и только тогда, когда существуют рёбра между парами вершин $v_1$, $u_1$ и $v_2$, $u_2$ в соответсвующих графах. 
\end{enumerate}

Для того, чтобы построить тензорное произведение ориентированных графов, необходимо в предыдущем определении, в условии существования реба в результирующем графе, дополнительно потребовать, чтобы направления рёбер совпадали.
Данное требование получается естесвенным образом, если считать, что пары вершин, задающие ребро упорядочены, поэтому формальное определение отличаться не будет.

Нетрудно заметить, что матрица смежности графа $\mathcal{G}_3$ равна тенорному произведению матриц смежностей исходных графов.

Осталось добавить метки к рёбрам.
Это приведёт к логичному усилению требованя к существованию ребра: метки рёбер в исходных графах должны совпадать.
Таким образом, мы получаем следующее определение тензорного произведения ориентированных графов с метками на рёбрах.
\begin{definition}

Пусть даны два ориентированных графа с метками на рёбрах: $\mathcal{G}_1 = \langle V_1, E_1, L_1 \rangle$ и $\mathcal{G}_2 = \langle V_2, E_2, L_2 \rangle$.
Тензорным произведением этих графов будем называть граф $\mathcal{G}_3 = \langle V_3, E_3, L_3\rangle$, где $V_3 = V_1 \times V_2$, $E_3 = \{ ((v_1,v_2),l,(u_1,u_2)) \mid (v_1,l,u_1) \in E_1 \text{ и } (v_2,l,u_2) \in E_2 \}$, $L_3=L_1 \cap L_2$.

\end{definition}


Рассмотрим пример.
В качестве одного из графов возьмём рекурсивный автомат, построенный ранее!!!.
Его матрица смежности выглядит следующим образом.
$$ M_1 =
\begin{pmatrix} 
. & [a] & . & . \\
. & . & [S] & [b] \\
. & . & . & [b] \\
. & . & . & . 
\end{pmatrix}
$$


\begin{figure}[h]
\begin{center}
\begin{tikzpicture}[shorten >=1pt,node distance=2cm,on grid,auto] 
   \node[state] (q_0)   {$0$}; 
   \node[state] (q_1) [above right=of q_0] {$1$}; 
   \node[state] (q_2) [right=of q_1] {$2$}; 
   \node[state] (q_3) [right=of q_2] {$3$};
    \path[->] 
    (q_0) edge  node {a} (q_1)          
    (q_1) edge  node {a} (q_2)
    (q_2) edge  node {a} (q_0)
    (q_2) edge[bend left, above]  node {b} (q_3)
    (q_3) edge[bend left, below]  node {b} (q_2);
\end{tikzpicture}
\end{center}

\caption{The input graph}
\label{input}
\end{figure}


Второй граф представлен на изображении~\ref{input}.
Его матрица смежности имеет следующий вид.
$$ M_2 =
\begin{pmatrix} 
. & [a] & . & . \\
. & . & [a] & . \\
[a] & . & . & [b] \\
. & . & [b] & . 
\end{pmatrix}
$$



Вычислим $M_1 \otimes M_2$.

$$
M_1 \otimes M_2 = 
\begin{pmatrix} 
. & [a] & . & . \\
. & . & [S] & [b] \\
. & . & . & [b] \\
. & . & . & . 
\end{pmatrix}
\otimes 
\begin{pmatrix} 
. & [a] & . & . \\
. & . & [a] & . \\
[a] & . & . & [b] \\
. & . & [b] & . 
\end{pmatrix}
=
\begin{pmatrix} 
. & . & . & .  &  . & [a] & . & .  &  . & . & . & .  &  . & . & . & .   \\
. & . & . & .  &  . & . & [a] & .  &  . & . & . & .  &  . & . & . & .   \\
. & . & . & .  &  . & [a] & . & .  &  . & . & . & .  &  . & . & . & .   \\
. & . & . & .  &  . & . & . & .    &  . & . & . & .  &  . & . & . & .   \\

. & . & . & .  &  . & . & . & .    &  . & . & . & .  &  . & . & . & .   \\
. & . & . & .  &  . & . & . & .    &  . & . & . & .  &  . & . & . & .   \\
. & . & . & .  &  . & . & . & .    &  . & . & . & .  &  . & . & . & [b] \\
. & . & . & .  &  . & . & . & .    &  . & . & . & .  &  . & . & [b] & . \\

. & . & . & .  &  . & . & . & .    &  . & . & . & .  &  . & . & . & .   \\
. & . & . & .  &  . & . & . & .    &  . & . & . & .  &  . & . & . & .   \\
. & . & . & .  &  . & . & . & .    &  . & . & . & .  &  . & . & . & [b] \\
. & . & . & .  &  . & . & . & .    &  . & . & . & .  &  . & . & [b] & . \\

. & . & . & .  &  . & . & . & .    &  . & . & . & .  &  . & . & . & .   \\
. & . & . & .  &  . & . & . & .    &  . & . & . & .  &  . & . & . & .   \\
. & . & . & .  &  . & . & . & .    &  . & . & . & .  &  . & . & . & .   \\
. & . & . & .  &  . & . & . & .    &  . & . & . & .  &  . & . & . & . 

\end{pmatrix}
$$

asdasd

\subsection{Алгоритм}

По грамматике строим автомат.

В цикле: пересекли через тензорное произведениеб замкнули, чтобы нацти пути из начальной в конечную в граммтике, поставили туда нетерминалы.

Можно вычислять только разницу.
Для этого, правда, потребуется держать ещё одну матрицу.
И надо проверять, что вычислительно дешевле: поддерживать разницу и потом каждый раз поэлементно складывать две матрицы или каждый раз вычислять полностью произведение.

Всего несколько матриц.
Разреженные.
Необходимо отметить, что для реальных графов и запросов результат тензорного произведения будет очень разрежен.
На готовых либах должно быть быстро.

\subsection{Вопросы и задачи}
\begin{enumerate}
\item !!!
\item !!!
\end{enumerate}