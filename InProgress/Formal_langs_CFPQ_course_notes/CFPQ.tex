\section{Задача о поиске путей с ограничениями в терминах формальных языков}


Что, откуда и зачем.

История вопроса.


\subsection{Постановка задачи }



Функция $\omega(\pi) = \omega((v_0, l_0, v_1),(v_1,l_1,v_2),\dots,(v_{n-1},l_n,v_n)) = l_0 \cdot l_1 \cdot \ldots \cdot l_n $ строит слово по пути посредством конкатенации меток рёбер вдоль этого пути.
Очевидно, для пустого пути данная функция будет возвращать пустое слово, а для пути длины $n  > 0$ --- непустое слово длины $n$.

Путь $G = \langle \Sigma, N, P \rangle$ --- контекстно-свободная граммтика.
Будем считать, что $L \subseteq \Sigma$.
Мы не фиксируем стартовый нетерминал в определении граммтики, поэтому, чтобы описать язык, задаваемый ей, нам необходимо отдельно зафиксировать стартовый нетерминал.
Таким образом, будем говорить, что $L(G,N_i) = \{ w | N_i \xRightarrow[G]{*} w  \}$ --- это язык задаваемый граммтикой $G$ со стартовым нетерминалом $N_i$.

Задача достижимости:

Задача поиска путей:

В задаче поиска путей мы можем накладывать дополнительные ограничения на путь (например, чтобы он был простым или кратчайшим), но это уже другая история.


\subsection{О разрешимости задачи}

Сведение к задаче о пересечениии с регулярным.

Замкнутость регулярных.

Проверка пустоты.

Замкнутость контекстно-свободных.
Проверка пустоты.

Про другие классы языков: конъюнктивные, булевы, многокомпонентные.

\subsection{Области применения}
Где применятеся

Статанализ.

Графовые БД.

Куча ссылок. Примеры?

\subsection{Вопросы и задачи}
\begin{enumerate}
  \item Пучть есть граф. Задайте грамматику дл поиска всех путей, таких, что....
  \item Существует ли в графе !!! путь из А в Б, такой что!!!
  \item Для графа !!! постройте все пути, удовлетворяющие !!!!

  \item Задача 1
  \item Задача 2
\end{enumerate}
