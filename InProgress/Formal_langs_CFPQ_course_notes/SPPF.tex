\section{Сжатое представление леса разбора}

Матричный алгоритм даёт нам ответ на вопрос о достижимости, но не предоставляет самих путей.
Что делать, если мы хотим построить все пути, удовлетворяющие ограичениям?

Проблема в том, что множество путей может быть бесконечным. 
Можем ли мы предложить конечную структуру, однозначно описывающую такое множество?
Вспомним, что пересечение контекстно-свободного языка с регулярным --- это контекстно-свободный язык.
Мы знаем, что конекстно-свободный язык можно описать коньекстно-своюодной граммтикой, которая конечна.
Это и есть решение нашего вопроса. 
Осталось толко научиться строить такую грамматику.

\subsection{Дерево вывода}

Дерево вывода цепочки в граммтике

\begin{itemize}
\item Корневое. Корень помечен стартовым нетерминалом.
\item Упорядоченное
\item Соотношение узлов и правил
\item Крона --- цепочка
\end{itemize}

\subsection{Неоднозначные граммтики}

Левосторонний и правосторонний выводы.

Существенно неоднозначные языки

\subsection{Лес разбора как представление контекстно-свободной грамматики}

Добавление информации в классическое дерево разбора. 
Координаты и промежуточные узлы.

\subsection{Вопросы и задачи}
\begin{enumerate}
  \item Постройте дерево вывода цепочки $w=aababb$ в грамматике $G=\langle\{a,b\},\{S\},\{S\rightarrow \varepsilon \ | \ a \ S \ b \ S \}, S \rangle$.
  \item Постройте все левосторонние выводы цепочки $w=ababab$ в грамматике $G=\langle\{a,b\},\{S\},\{S\rightarrow \varepsilon \ | \ a \ S \ b \ | S \ S\}, S \rangle$.
  \item Постройте все правосторонние выводы цепочки $w=ababab$ в грамматике $G=\langle\{a,b\},\{S\},\{S\rightarrow \varepsilon \ | \ a \ S \ b \ | S \ S\}, S \rangle$.
  \item \label{t1}Постройте все деревья вывода цепочки $w=ababab$ в грамматике $G=\langle\{a,b\},\{S\},\{S\rightarrow \varepsilon \ | \ a \ S \ b \ | S \ S\}, S \rangle$, соответствующие левосторонним выводам.
  \item \label{t2}Постройте все деревья вывода цепочки $w=ababab$ в грамматике $G=\langle\{a,b\},\{S\},\{S\rightarrow \varepsilon \ | \ a \ S \ b \ | S \ S\}, S \rangle$, соответствующие правосторонним выводам.
  \item Как связаны между собой леса, полученные в предыдущих двух задачах (\ref{t1} и \ref{t2})? Какие выводы можно сделать из такой связи?
  \item Постройте сжатое представление леса разбора, полученного в задаче~\ref{t1}.
  \item Постройте сжатое представление леса разбора, полученного в задаче~\ref{t2}.
  \item \label{t3}Предъявите контекстно-свободную граммтику существенно неоднозначного языка. 
        Возьмите цепочку длины болше пяти, при надлежащую этому языку, и постройте все деревья вывода этой цепочки в предъявленной граммтике. 
  \item Постройте сжатое представление леса, полученного в задаче~\ref{t3}.
\end{enumerate}
