\documentclass[12pt]{article}  % standard LaTeX, 12 point type

\usepackage{geometry}

\usepackage{amsmath}
\usepackage{amsfonts,latexsym}
\usepackage{amsthm}
\usepackage{amssymb}
\usepackage[utf8]{inputenc} % Кодировка
\usepackage[english,russian]{babel} % Многоязычность
\usepackage{verbatim}
\usepackage{longtable}
\usepackage{csvsimple}
\usepackage[toc,page]{appendix}
\usepackage{booktabs}

\usepackage{float}
\usepackage{array}
\usepackage{multirow}
\usepackage{caption}
\usepackage{graphicx}
\usepackage{ucs}
\usepackage{rotating}
\usepackage{pdflscape}
\usepackage{afterpage}
\usepackage{xcolor}
\usepackage{capt-of}% or use the larger `caption` package
\usepackage{url}

% unnumbered environments:

\theoremstyle{remark}
\newtheorem*{remark}{Remark}
\newtheorem*{notation}{Notation}
\newtheorem*{note}{Note}

\setlength{\parskip}{5pt plus 2pt minus 1pt}
\newcolumntype{C}{>{\centering\arraybackslash}p{1.3cm}}
\graphicspath{{pics/}}

\newcommand{\checkme}[1]{\textcolor{red}{#1}}
\newcommand{\reviewme}[1]{\textcolor{blue}{#1}}
\newcommand{\cho}[1]{\textcolor{violet}{#1}}

%Теория формальных языков и алгоритмы синтаксического анализа для анализа граф-струкурированных данных
%
\title{Разработка методов, средств и алгоритмов для высокопроизводительного анализа данных, основанного на линейной алгебре}
\author{Семён Григорьев}
\date{\today}

\begin{document}

\newgeometry{left=0.8in,right=0.8in,top=1in,bottom=1in}

\maketitle

\section{Сведения о проекте}

\subsection{Название проекта}

\textbf{ru}\\
%
Теория и практика высокопроизводительных вычислений с использованием методов и алгоритмов линейной алгебры
\\
или
\\
Разработка методов и средств высокопроизводительных вычислений, основанных на линейной алгебре.
\\
или
\\
Разработка методов и средств высокопроизводительного анализа данных, основанных на линейной алгебре.
\\
или
\\
Теория и практика высокопроизводительных вычислений с использованием методов и алгоритмов линейной алгебры
\\
или
\\
Разработка методов, средств и алгоритмов высокопроизводительного анализа данных, основанных на линейной алгебре.
\\
или
\\
Разработка методов, средств и алгоритмов высокопроизводительных вычислений, основанных на линейной алгебре.
\\
\\
\textbf{en}\\
Development of methods, tools, and algorithms for linear algebra-based high-performance data analysis

\subsection{Приоритетное направление развития науки, технологий и техники в Российской Федерации, критическая технология}
%


\subsection{Направление из Стратегии научно-технологического развития Российской Федерации (утверждена Указом Президента Российской Федерации от 1 декабря 2016 г. \textnumero 642 "О Стратегии научно-технологического развития Российской Федерации") (при наличии)}
%

\subsection{Ключевые слова (приводится не более 15 терминов)}

\textbf{ru}\\
%
Высокопроизводительные вычисления, линейная алгебра, массово-параллельные архитектуры, разреженные структуры данных, графовые базы данных, алгоритмы анализа графов, статический анализ кода, параллельные алгоритмы, суперкомпиляция, смешанные вычисления, специализация, языки запросов к графам.
\\
\\
\textbf{en}\\

High-performance computing, linear algebra, massively-parallel hardware, sparse data structures, graph databases, graph analysis algorithms, static code analysis, parallel algorithms, supercompilation, mixed computations, partial evaluation, specialization, graph query languages.



\subsection{+Аннотация проекта}
%(объемом не более 2 стр.; в том числе кратко – актуальность решения указанной выше научной проблемы и научная новизна)
\textbf{ru}\\

Высокопроизводительная обработка больших объёмов данных --- актуальная прикладная область, требующая качественных теоретических результатов для решения возникающих прикладных задач. Одним из активно изучаемых и разрабатываемых в последнее время подходов к решению данной проблемы, является подход, основанный на использовании примитивов и операций линейной алгебры для описания прикладных алгоритмов. Здесь можно вспомнить такие задачи анализа графов, как поиск кратчайших путей или построение транзитивного замыкания, которые выражаются через произведение матриц в соответствующих полукольцах. Обобщая эту идею, был предложен стандарт GraphBLAS, который описывает необходимый для разработки прикладных алгоритмов набор примитивов и операций линейной алгебры. Несмотря на то, что изначально данный стандарт задумывался для применения в области разработки алгоритмов анализа графов, он оказался применим и для разработки решений в области машинного обучения и биоинформатики, показав таким образом большой потенциал предлагаемой в нём идеи.

Использование данного стандарта, и в целом линейной алгебры, для решения прикладных задач позволяет существенно повысить скорость разработки новых прикладных решений и производительность обработки данных даже при использовании существующих программных и аппаратных средств. Однако в последнее время было показано, что для дальнейшего роста производительности и более широкого распространения подхода (и, как следствие, стандарта) необходимо решить ряд крупных исследовательских задач. Основные из них связаны с такими особенностями, как разреженность реальных данных и необходимость предоставлять абстрактные процедуры, оперирующие такими понятиями, как моноид или полукольцо.

Первая из них связана с тем, что современные процессорные архитектуры, в том числе архитектуры графических ускорителей, плохо справляются с обработкой разреженных данных и, в частности, с операциями разреженной линейной алгебры (матрицы и вектора содержат мало значимых элементов). Это связано, с одной стороны, с нерегулярным шаблоном доступа в память, что приводит к падению эффективности кэшей процессора, а с другой, с тем, что ряд важных в данной области оптимизаций программного кода не реализуем в принятых в данной области языках программирования. Кроме этого, работа с разреженными структурами данных требует разработки специализированных алгоритмов и даже обычное поэлементное сложение двух матриц становится нетривиальной задачей в случае, когда они разрежены. Решение этой проблемы необходимо для развития подхода, основанного на разреженной линейной алгебре, и невозможно без создания специализированных архитектур и языков программирования. Работы в этом направлении активно ведутся во всём мире. В данном проекте будет также решаться данная проблема, причём совместным дизайном (co-design) специализированного языка, компилятора и аппаратной архитектуры, что должно позволить использовать возможности каждого из компонентов. 

Вторая проблема связана с малой выразительностью распространённых языков программирования, применяемых при разработке для параллельных и массово-параллельных архитектур. Как правило, это языки семейства Си (CUDA C, OpenCL C), которые хотя и являются языками общего назначения и полны по Тьюрингу, не позволяют естественным образом выражать, например, факт параметризации функции некоторым полукольцом. Это затрудняет разработку соответствующего программного обеспечения, делает его менее надёжным, так как компиляторы данных языков дают относительно мало статических гарантий. Более того, для упрощения разработки в гетерогенных системах вида многоядерный центральный процессор и несколько специализированных ускорителей, нужны соответствующие способы абстракции. Всё это приводит к необходимости поиска новых методов и средств разработки. Более того, так как обработка большого объёма данных уже стала необходимостью и в прикладных решениях, возникают требования интеграции соответствующих средств в языки высокого уровня и соответствующие платформы. В рамках данного проекта будет вестись разработка такого средства, позволяющего прозрачным для разработчика образом создавать прикладные решения на языке высокого уровня с использованием многоядерных систем и графических ускорителей общего назначения (GPGPU).

Кроме решения системных проблем, необходимо развивать и прикладной уровень. Здесь в настоящее время выделяются два направления: графовые базы данных и статический анализ кода. Хотя эти две области до сих пор во многом развиваются независимо, всё больше решений в области статического анализа кода начинает использовать графовые базы данных для хранения структурных представлений программ, соответствующие языки запросов и алгоритмы для их анализа. Однако многие решения, предложенные в графовых базах данных, в частности алгоритмы поиска путей, необходимые для межпроцедурного анализа кода, и основанные на операциях линейной алгебры, еще не применяются в статическом анализе кода, хотя и ведётся активная работа по поиску эффективных параллельных алгоритмов для решения этих задач. В данном проекте будет проведено исследование применимости параллельных алгоритмов, предложенных в графовых базах данных, к задачам статического анализа кода. Отдельного внимания заслуживает интеллектуальная обработка результатов статического анализа кода методами машинного обучения, так как разработчику, при работе с большим объёмом кода, сложно анализировать "сырые" данные и ему требуется помощь. Работы в этом направлении ведутся достаточно активно, в частности, применительно к упомянутым выше алгоритмам межпроцедурного анализа кода. 

Отдельное внимание необходимо уделить вопросам использования операций линейной алгебры при выполнении запросов к графовым базам данных. Хотя на практике такой подход уже зарекомендовал себя, например, в графовой базе данных RedisGraph, до сих пор не существует формального описания таких процедур трансляции в операции линейной алгебры для конкретных языков, в частности Cypher --- наиболее распространенного языка запросов к графам. Это, с одной стороны, не позволяет утверждать, что потенциал рассматриваемого подхода использован полностью, а с другой, не позволяет давать гарантии корректности выполнения запросов, что отрицательно сказывается на надёжности разрабатываемых систем.

Таким образом, данный проект посвящен развитию подхода к высокопроизводительной обработке больших данных, основанного на использовании примитивов и операций разреженной линейной алгебры. Будут вестись как разработка методов и средств разработки решений, основанных на линейной алгебре, для современных аппаратных платформ, так и разработка нового программно-аппаратного решения для ускорения алгоритмов, описанных в терминах линейной алгебры. Кроме этого, будет вестись исследование прикладных областей, в которых обсуждаемый подход может быть применим. А именно, будут изучаться вопросы применимости алгоритмов, основанных на операциях линейной алгебры, к анализу программного кода, а также вопросы трансляции языков запросов к графовым базам данных в операции линейной алгебры.

Коллектив исполнителей включает специалистов по теории языков программирования, теории графов, построению компиляторов, методам оптимизации программ,  разработке языков программирования и алгоритмов анализа графов, разработке и реализации параллельных алгоритмов с использованием современных аппаратных платформ. Это позволит организовать плодотворное сотрудничество и обеспечить комплексный подход к решению задач, а также привлечь талантливых студентов к изучению соответствующих областей науки и работе над проектом.
\\
\\
\textbf{en}\\

High-performance big data processing is a topical applied area that requires qualitative theoretical results for solving problems arising in practice. One of the recently and actively studied and developed approaches towards the problem is based on the usage of linear algebra operations and primitives to describe the applied algorithms, such as transitive closure and shortest path routines from graph analysis which are expressible via matrix multiplication over a corresponding semiring. Generalizing this idea, GraphBLAS specification has been proposed which describes linear algebra primitives and operations necessary to implement applied algorithms. Though the initial design was primarily aimed at the implementation of graph analysis algorithms, it appeared to be useful in expressing algorithms in other areas including but not limited to machine learning and bioinformatics, having shown the prominent potential of the proposed idea.

Despite that GraphBLAS specification and linear algebra in general powered by modern software and hardware make it possible to ease the development as well as increase the performance of data processing, it has been shown that in order to further enhance the performance and make the approach (as well as the specification) more widespread, a number of research challenges should be solved. This challenges are induced by such peculiarities as the sparsity of data in practice, and the necessity to provide abstract procedures that operate with such structures as monoid or semiring.
    
The former challenge is related to the fact that modern processor architectures, including graphical processors, are inefficient when dealing with sparse data and, in particular, sparse linear algebra operations (when matrices and vectors are filled mostly with zero elements). On the one hand, it happens due to the irregularity of memory accesses that incurs a drop in efficiency of processor caches. On the other hand, it happens because some important for the area software optimizations cannot be expressed in the adopted programming languages. Moreover, working with sparse data structures requires the implementation of specialized algorithms which are not trivial, even in the case of element-wise addition of two sparse matrices. The solution for this challenge is required for the evolution of the sparse linear algebra-based approach and is not possible without the design of specialized hardware architectures and programming languages. The challenge is widely addressed over the world, while the distinctive feature of this work is that it aims to solve the challenge with the co-design of specialized hardware, software, and an optimizing compiler, which makes it possible to fully utilize the advantages of each.

The latter challenge is induced by the low expressive power of programming languages that are widely used in parallel and massively-parallel architectures. As a rule, these are C-family languages (CUDA C, OpenCL C) which, despite being general-purpose and Turing-complete, do not allow to conveniently express, e.g., the fact of a function being parameterized by a semiring. This hampers the development of related software and reduces its safety since the compilers of such languages give relatively few static guarantees. Further, to ease the development of heterogeneous systems, such as multicore CPU bundled with several hardware accelerators, certain abstractions are needed. This leads to the need for developing new methods and tools for software development. Finally, since big data processing has become a necessity in applied solutions, there is a requirement for the integration of the corresponding tools into high-level languages and related platforms. The development of such a tool that allows a developer to transparently create applied solutions in high-level language using multicore systems and general-purpose graphics acceleration processors (GPGPU) will be performed within this project.

Apart from tackling system problems, the applied software also needs to be evolved. There are presently two notable directions in this area: graph databases and static code analysis. Though these areas have been evolving separately in many ways, more static code analysis solutions begin to utilize graph databases to store structural representations of programs, and related query languages and algorithms for the analysis itself. However, many solutions proposed by graph databases, in particular, path querying algorithms essential for interprocedural code analysis, based on linear algebra remain unused in static code analysis, even though there is active research towards efficient parallel algorithms for static code analysis. This project will carry out research on whether parallel algorithms offered for graph databases could be applied for static code analysis. Of particular note is intelligent processing of the static code analysis results with machine learning methods since a developer could find it difficult to conceptualize "raw" data while working with a huge codebase and may need additional help. This area, including interprocedural code analysis, is being actively researched.

Special attention should be paid to the usage of linear algebra operations in the case of graph database query execution. Although this approach established itself in practice, for example, in RedisGraph graph database, there is no formal specification for the translation of such procedures into linear algebra operations, in particular for Cypher, which is the most popular graph query language. On the one hand, this fact signifies that the potential of the considered approach is not completely used up. On the other hand, there is no way to guarantee query execution correctness which affects the safety of the systems developed.

Thus the project is devoted to the development of the approach for high-performance big data processing based on the usage of sparse linear algebra operations and primitives. Both the linear algebra-based methods and tools for software solutions development for modern hardware platforms, and a new software-hardware solution for the acceleration of the algorithms expressed in terms of linear algebra, will be developed. Besides, the areas where the approach could be potentially applied will be studied. Namely, the applicability of linear algebra-based algorithms to code analysis and graph databases query languages translation into linear algebra operations will be considered.

The research group constitutes of the specialists in programming languages theory, graph theory, compiler development, program optimizations, development of programming languages and graph analysis algorithms, design and implementation of parallel algorithms using modern hardware platforms. This will facilitate productive cooperation and provide an integrated approach to solving the problem as well as attracting talented students to study relevant fields of science while working on a project.


\subsection{+Ожидаемые результаты и их значимость}
%(указываются результаты, их научная и общественная значимость (соответствие предполагаемых результатов мировому уровню исследований, возможность практического использования предполагаемых результатов проекта в экономике и социальной сфере))

\textbf{ru}\\

Проект направлен на разработку методов, средств и алгоритмов, необходимых для создания высокопроизводительных решений по обработке больших объёмов данных. Ожидаются как теоретические результаты в области языков программирования и оптимизации программ, алгоритмов анализа графов, языков запросов к графам, так и прикладные результаты, такие как средства разработки программ для графических ускорителей общего назначения и библиотеки алгоритмов для высокопроизводительной обработки графов.

В современном мире высокопроизводительные вычисления, в частности анализ графов, стали неотъемлемой частью пользовательских приложений, например, социальных сетей или навигаторов. Такие приложения разрабатываются с использованием языков высокого уровня на таких платформах, как JVM или .NET. Для того, чтобы предоставить разработчикам возможность использовать прозрачным образом возможности гетерогенных сред при программировании на языках высокого уровня, в проекте поставлена задача создания специализированного средства разработки. 
Ожидается создание функционального прототипа указанного средства и нескольких прикладных решений на его основе, а также анализ принципов построения подобных средств и способов их использования.  В частности, будет реализовано подмножество стандарта GraphBLAS для платформы .NET, что даст возможность использовать высокопроизводительный анализ графов с использованием графических ускорителей в приложениях на данной платформе.

В области разработки программно-аппаратного комплекса для высокопроизводительных вычислений на основе разреженной линейной алгебры ожидаются, во-первых, теоретические результаты в области методов оптимизации программ, специфичных для данной области, в частности, ожидаются результаты в области суперкомпиляции. Во-вторых, ожидается создание прототипа такого комплекса и реализация подмножества стандарта GraphBLAS на его основе, с последующей реализацией нескольких прикладных алгоритмов. Ввиду сложности и комплексности данной задачи, сложно гарантировать достижение превосходства над существующими решениями, однако полученные результаты помогут оценить перспективность выбранного направления и внесут вклад в развитие актуальной области. 

Ожидается, что работа над формальным описанием семантики языка запросов Cypher приведёт к созданию верифицированного в системе Coq описания семантики языка, а также процедур трансляции его конструкций в операции линейной алгебры. В дальнейшем это позволит получить корректные по построению процедуры трансляции запросов к графам в термины разреженной линейной алгебры. Также ожидается, что результаты данной работы станут частью проекта по разработке стандарта языка запросов к графам GQL Standard. 

В результате работы над параллельными алгоритмами статического анализа кода ожидается создание конкретных параллельных алгоритмов, эффективно использующих возможности современных многоядерных процессоров и графических ускорителей общего назначений. Получение таких алгоритмов позволит создать эффективные средства статического анализа больших объёмов кода. Кроме этого, планируется создание средств интеллектуального анализа кода на основе результатов работы указанных выше алгоритмов и методов машинного обучения.

\\
\\
\textbf{en}\\

The aim of the project is to develop methods and algorithms for high-performance big data processing. 
The plan is to obtain theoretical results in programming languages theory, program optimizations, graph analysis algorithms, and graph query languages, as well as to create applications such as developer tools for GPGPUs and libraries for high-performance graph processing. 

High-performance data processing is an essential part of many user applications such as social networks and navigation systems. These applications are developed in high-level languages on such platforms as .NET and JVM. To facilitate efficient use of the features of heterogeneous systems in the applications, one of the tasks is to create specialized development tools. It is planned to develop a prototype solution (as well as several applications based on it) for high-level programming of heterogeneous systems, which include a multiple core processor along with several GPGPU. It is also planned to analyse the principles of creation and use of such tools. One particular task is to implement a substantial subset of the GraphBLAS standard for the .NET platform aimed to facilitate the use of high-performance graph analysis on GPGPU in the applications developed for the .NET platform.   

The next task is to create a hardware and software system to use sparse linear algebra for high-performance data processing. Besides developing a prototype of the system, it is planned to implement a subset of GraphBLAS API based on it and to implement several applied algorithms. It is also intended to obtain theoretical results in program optimisation methods, in particular, supercompilation. This task is complex and complicated, thus it is hard to guarantee that the solution will be better than the existing. Nevertheless, the results will be useful to evaluate whether this direction is promising.  

Formal specification of the semantics of the Cypher query language is planned to be done in the Coq proof management system. As a result, the semantics of the language, its properties, as well as the translation of it into the linear algebra operations will be formally verified. This will allow for correct-by-construction translation of graph queries into the sparse linear algebra. The results of this project is intended to be a part of GQL Standard --- the standard for the graph query language. 

The creation of parallel static code analysis algorithms for handling big software systems is planned. These algorithms should utilize modern parallel hardware, such as multicore CPUs and GPGPUs, in order to achieve high performance. Moreover, the development of the solutions for intelligent static code analysis based on machine learning and utilization of results of the mentioned static code analysis algorithms is planned.

\subsection{+В состав научного коллектива будут входить}
%
\begin{itemize}
\item 8 исполнителей проекта (включая руководителя)
\item в том числе 8  исполнителей в возрасте до 39 лет включительно,
\item из них: 4 очных аспирантов, адъюнктов, интернов, ординаторов, студентов.
\end{itemize}

\subsection{+Планируемый состав научного коллектива с указанием фамилий, имен, отчеств (при наличии) членов коллектива, их возраста на момент подачи заявки, ученых степеней, должностей и основных мест работы, формы отношений с организацией (трудовой договор, гражданско-правовой договор) в период реализации проекта.}
4
\begin{itemize}
  \item Семён Вячеславович Григорьев, 31 год, к.ф.-м.н., доцент СПбГУ, трудовой договор.
  \item Даниил Андреевич Березун, 28 лет, к.ф.-м.н., старший преподаватель СПбГУ, трудовой договор.
  \item Антон Викторович Подкопаев, 28 лет, к.ф.-м.н., доцент НИУ ВШЭ (СПб), гражданско-правовой договор.
  \item Тимофей Александрович Брыксин, 34 года, к.т.н., доцент СПбГУ, трудовой договор
  \item Рустам Шухратуллович Азимов, 26 лет, магистр (СПбГУ, Математическое обеспечение и администрирование информационных систем), аспирант третьего года обучения (СПбГУ, Математико-механический факультет), ассистент СПбГУ, трудовой договор
  \item Алексей Валерьевич Тюрин, 22 года, студент первого курса магистратуры СПбГУ, гражданско-правовой договор
  \item Арсений Константинович Терехов, 22 года, студент первого курса магистратуры ИТМО (Санкт-Петербургский государственный университет информационных технологий, механики и оптики), гражданско-правовой договор
  \item Егор Станистлавович Орачев, 22 года, студент четвертого курса бакалавриата СПбГУ, гражданско-правовой договор
\end{itemize}
\\
\\
\textbf{en}
\begin{itemize}
  \item Semyon Viacheslavovich Grigorev, PhD, associate professor SPbSU, 31
  \item Daniil Andreevich Berezun, PhD, senior lecturer SPbSU, 28
  \item Anton Victorovicg Podkopaev, PhD, associate professor NRU HSE in St. Petersburg, 28
  \item Timofey Alexandrovich Bryksin, PhD, associate professor SPbSU, 34
  \item Rustam Shuhratullovich Azimov, master of Information Technologies, PhD student SPbSU, assistant SPbSU, 26 
  \item Aleksey Valerievich Tyurin , master student SPbSU, 22
  \item Arseniy Konstantinovich Terekhov, master student ITMO, 22
  \item Egor S. Orachev, bachelor student SPbSU, 22
\end{itemize}


\textbf{+Соответствие профессионального уровня членов научного коллектива задачам проекта}

\textbf{ru}\\
%
Руководитель, Семён Вячеславович Григорьев, является доцентом кафедры информатики СПбГУ и кандидатом физико-математических наук. Опыт руководства исследовательскими работами и преподавания составляет 7 лет. За это время под его руководством защищено 8 магистерских диссертаций, 15 выпускных квалификационных работ бакалавра, 2 дипломных работы специалиста, больше 15 курсовых работ. В настоящее время под его руководством работают два аспиранта. За время преподавательской деятельности занимался подготовкой и чтением курсов по теории графов, алгоритмам анализа графов, теории формальных языков, алгоритмам и структурам данных. Имеет опыт руководства грантами (РФФИ 19-37-90101; программа УМНИК, 162ГУ1/2013 и 5609ГУ1/2014) исследовательскими группами и отдельными исследовательскими работами. Также имеет опыт исполнения грантов (РФФИ 15-01-05431, РФФИ 18-01-00380, РНФ 18-11-00100). Область научных интересов включает теорию формальных языков, теорию графов, алгоритмы синтаксического анализа, разработку параллельных алгоритмов, аппаратные ускорители параллельных вычислений.

Даниил Андреевич Березун является старшим преподавателем кафедры современного программирования факультета математики и компьютерных наук СПбГУ и кандидатом физико-математических наук. Опыт руководства исследовательскими работами и преподавательской деятельности составляет более 6 лет. За это время под его руководством были защищены 6 выпускных квалификационных работы бакалавра, более 8 курсовых работ. За время преподавательской деятельности занимался подготовкой и чтением курсов по компиляции, разработке языковых процессоров, метавычислениям, семантикам языков программирования, операционным системам и архитектуре компьютера, а также практиками по различным языкам программирования. В настоящее время под его руководством работают 3 магистранта. Имеет опыт исполнения грантов (РФФИ 18-01-00380). Область научных интересов включает анализ, разработку и реализацию языков программирования,  метапрограммирование и метавычисления, математическую логику, семантику языков программирования, автоматическую генерацию программ, основанную на семантике, блокчейн и распределённые технологии.

Антон Викторович Подкопаев является кандидатом физико-математических наук и доцентом
департамента информатики НИУ ВШЭ в Санкт-Петербурге. Опыт руководства исследовательскими работами и
преподавательской деятельности составляет более 5 лет. Имеет опыт по механизации семантик языков программирования и доказательств их свойств в системе Coq (A. Podkopaev, O. Lahav, V. Vafeiadis. Bridging the Gap Between Programming Languages and Hardware Weak Memory Models. 2019; E. Moiseenko, A. Podkopaev, O. Lahav, O. Melkonian, V. Vafeiadis. Reconciling Event Structures with Modern Multiprocessors. 2020; S.-H. Lee, M. Cho, A. Podkopaev, S. Chakraborty, C.-K. Hur, O. Lahav, V. Vafeiadis. Promising 2.0: Global Optimizations in Relaxed Memory Concurrency. 2020; C. Watt, C. Pulte, A. Podkopaev, G. Barbier, S. Dolan, S. Flur, J. Pichon-Pharabod, S. Guo. Repairing and Mechanising the JavaScript Relaxed Memory Model. 2020), которая планируется к использованию в проекте. За время преподавательской деятельности занимался подготовкой и чтением курсов по компиляции, разработке языковых процессоров и семантикам языков программирования. Имеет опыт исполнения грантов (РФФИ 18-01-00380). Область научных интересов интересов включает семантику языков программирования, многопоточное программирование, слабые модели памяти и функциональное программирование.

Тимофей Александрович Брыксин является кандидатом технических наук, работает доцентом кафедры системного программирования СПбГУ. Руководит исследовательскими проектами с 2009 года, за это время под его руководством было защищено четыре выпускных квалификационных работы магистра, 32 выпускных квалификационных работ бакалавра, больше сотни курсовых работ. В настоящее время под его руководством работают два аспиранта. За время преподавательской деятельности занимался разработкой и чтением курсов по программной инженерии, проектированию программного обеспечения, курсов по программированию на C++, Java, Kotlin, Python, Haskell, курса по применению методов машинного обучения в программной инженерии. Область научных интересов включает в себя статический анализ программного кода, процессы и инструментальные среды разработки программного обеспечения, применение статистических методов (в частности, методов машинного обучения и анализа данных) в программной инженерии.

Рустам Шухратуллович Азимов является аспирантом математико-механического факультета СПбГУ по направлению информатика и ассистентом кафедры информационно-аналитических систем СПбГУ. Защитил магистерскую диссертацию на тему "Синтаксический анализ графов через умножение матриц". Имеет публикации по теме проекта ("Context-Free Path Querying by Kronecker Product", "Context-Free Path Querying with Single-Path Semantics by Matrix Multiplication", "Path Querying with Conjunctive Grammars by Matrix Multiplication", "Context-Free Path Querying by Matrix Multiplication", "Синтаксический анализ графов с использованием конъюнктивных грамматик", "Синтаксический анализ графов и задача генерации строк с ограничениями"). Имеет опыт исполнения грантов (РНФ 18-11-00100 и РФФИ 19-37-90101). Область научных интересов: теория формальных языков, запросы к графам, языки запросов, поиск путей в графах, матричные операции, параллельные алгоритмы.

Алексей Валерьевич Тюрин является студентом первого курса магистратуры математико-механического факультета СПбГУ по направлению Программная инженерия. Имеет публикации по теме проекта ("Optimizing GPU Programs By Partial Evaluation"). Прошёл летние стажировки в компании JetBrains. Область научных интересов включает метапрограммирование, математическую логику, аппаратное ускорение программ.

Терехов Арсений является студентом 1го курса корпоративной магистерской программы компании JetBrains по направлению "Прикладная математика и информатика" на базе университета ИТМО. Закончил бакалавриат математико-механического факультета СПбГУ по направлению "математическое обеспечение и администрирование информационных систем". Также окончил Computer Science Center по направлению "Software Engineering". Имеет индексируемую в Scopus публикацию ("Context-Free Path Querying with Single-Path Semantics by Matrix Multiplication"). Прошёл две летние стажировки в компаниях Яндекс и JetBrains, также учавствовал в двух проектах под руководством работников компании JetBrains. На текущий момент является стажёром ООО "ИнтеллиДжей Лабс". Облать научных интересов: теория формальных языков, линейная алгебра, графовые базы данных, вычисления на графических процессорах.

Егор Станиславович Орачев является студентом четвертого курса бакалавриата математико-механического факультета СПбГУ по направлению "Программная инженерия". Имеет индексируемую в Scopus публикацию по теме проекта ("Context-Free Path Querying by Kronecker Product"). Принимал участие в летней школе Ланит-Терком. Прошел летнюю стажировку в лаборатории JetBrains Research. Область научных интересов включает теорию формальных языков, графовые базы данных, высокопроизводительные вычисления, программирование на графических процессорах.
\\
\\
\textbf{en}

The lead of the group, Semyon V. Grigorev, is an associate professor of the faculty of Mathematics and Mechanics of Saint Petersburg State University and has a Ph.D. in mathematics and physics. He has 7 years of experience in teaching and being a leader and a manager of research projects. He has supervised 8 master dissertations, 15 graduation theses of bachelors, 2 graduation theses of specialists, and more than 15 course works. Two Ph.D. students are being supervised by him now. The following courses were prepared and taught: graph theory, formal language theory, algorithms and data structures. Semyon has experience in being a leader of both grants  (RFBR 19-37-90101; FASIE, 162ГУ1/2013 and 5609ГУ1/2014), and research groups and projects. Also, he has participated in grants (RFBR 15-01-05431, RFBR 18-01-00380, RSF 18-11-00100). Research interests include formal language theory, graph theory, parsing algorithms, parallel algorithms, high-performance computig, hardware accelerators.

Daniil A. Berezun is a senior lecturer of the department of Mathematics and Computer Science of Saint Petersburg State University, and has a Ph.D. in mathematics and physics. He has supervised 6 graduation theses of bachelor and more than 8 course works. Three master students are being supervised by him now. The following courses were prepared and taught: compiler techniques, language processors development, programming languages semantics, metacomputations, operating systems, computer architecture, and several practices in programming languages. Daniil has participated in the grant RFBR 18-01-00380. Research interests include analysis, design, and implementation of programming languages, programming languages semantics, metaprogramming and metacomputations, semantic-based automated program generation, blockchain, and distributed systems.

Anton V. Podkopaev has a Ph.D. in mathematics and physics, and is an associate professor at the Applied Mathematics and Informatics chair of NRU HSE in St. Petersburg, Russia. He has been supervising research projects and lecturing for more than 5 years. Anton has broad experience in mechanization of programming language semantics and their properties in the Coq proof assistant (A. Podkopaev, O. Lahav, V. Vafeiadis. Bridging the Gap Between Programming Languages and Hardware Weak Memory Models. 2019; E. Moiseenko, A. Podkopaev, O. Lahav, O. Melkonian, V. Vafeiadis. Reconciling Event Structures with Modern Multiprocessors. 2020; S.-H. Lee, M. Cho, A. Podkopaev, S. Chakraborty, C.-K. Hur, O. Lahav, V. Vafeiadis. Promising 2.0: Global Optimizations in Relaxed Memory Concurrency. 2020; C. Watt, C. Pulte, A. Podkopaev, G. Barbier, S. Dolan, S. Flur, J. Pichon-Pharabod, S. Guo. Repairing and Mechanising the JavaScript Relaxed Memory Model. 2020),
which is planned to be used in the project. The following courses were prepared and taught: compiler techniques, language processors development, programming languages semantics. Anton has participated in the grant RFBR 18-01-00380. His research interests include analysis, design, and implementation of programming languages, programming languages semantics, concurrency, weak memory models, and functional programming.

Timofey A. Bryksin holds a Ph.D. in Software Engineering. He is an associate professor at the Software Engineering chair of Saint Petersburg State University. He has been leading research projects since 2009, during this time he has supervised four master dissertations, 32 theses of bachelors, and more than a hundred course works. Currently, he supervises two Ph.D. students. The following courses were prepared and taught: Software Engineering, Software Design, various Programming 101 courses (for C++, Java, Python, Kotlin, Haskell), Machine Learning and Software Engineering. Research interests include static analysis of code, processes and tools in software engineering, application of data science methods (machine learning in particular) in software engineering.

Rustam Sh. Azimov is a Ph.D. student at the faculty of Mathematics and Mechanics at Saint Petersburg State University and an assistant at the Information and Analytical Systems chair of Saint Petersburg State University. He has a masters degree, his master's thesis is "Graph parsing by matrix multiplication". Rustam has publications which are related to this project ("Context-Free Path Querying by Kronecker Product", "Context-Free Path Querying with Single-Path Semantics by Matrix Multiplication", "Path Querying with Conjunctive Grammars by Matrix Multiplication", "Context-Free Path Querying by Matrix Multiplication", "Graph parsing by using conjunctive grammars", "Graph parsing and constrained string generation problem"). He has participated in grants (RSF 18-11-00100 и RFBR 19-37-90101). Research interests include formal language theory, graph querying, query languages, linear algebra, parallel algorithms.

Aleksey V. Tyurin is a 1-st year MSc student of Software Engineering at the faculty of Mathematics and Mechanics at Saint Petersburg State University. Aleksey has  a publication on the related topic "Optimizing GPU Programs By Partial Evaluation" and was an intern at JetBrains. Research interests include metaprogramming, mathematical logic, and hardware acceleration.

Arseniy K. Terekhov is a 1st-year master's student at the ITMO University, the master program "Applied mathematics and computer science". Arseniy successfully finished the bachelor's education program at St Petersburg University, specialization "Software Engineering and information systems", and an education program of Computer Science Center, specialization "Software Engineering". He has a publication "Context-Free Path Querying with Single-Path Semantics by Matrix Multiplication" indexed in Scopus. Arseniy was an intern at Yandex and JetBrains software development company. He worked on two research projects led by employers of JetBrains. Currently, he is an intern at JetBrains Research laboratory. Research interests include formal languages, linear algebra, GPGPU computing, and graph databases.

Egor S. Orachev is a 4th-year student at the faculty of Mathematics and Mechanics at Saint Petersburg State University, specialization is "Software engineering". He has a Scopus indexed publication which is related to this project ("Context-Free Path Querying by Kronecker Product"). He participated in the summer school organized by Lanit-Tercom. Egor completed an internship at the JetBrains Research laboratory. Research interests include formal languages, graph databases, high performance computing, and GPU programming.


\subsection{+Планируемый объем финансирования проекта Фондом по годам (указывается в тыс. рублей)}
2021 г. 6000 тыс. рублей,
2022 г. 6000 введите планируемый объем финансирования в 2022 г. тыс. рублей,
2023 г. 6000 введите планируемый объем финансирования в 2023 г. тыс. рублей.

\subsection{Научный коллектив по результатам проекта в ходе его реализации предполагает опубликовать в рецензируемых российских и зарубежных научных изданиях не менее}
%Приводятся данные за весь период выполнения проекта. Уменьшение количества публикаций (в том числе отсутствие информации в соответствующих полях формы) по сравнению с порогом, установленным в пункте 16.2 конкурсной документации является основанием недопуска заявки к конкурсу.

10 публикаций

из них 8 в изданиях, индексируемых в базах данных «Сеть науки» (Web of Science Core Collection) или «Скопус» (Scopus).

\textbf{Информация о научных изданиях, в которых планируется опубликовать результаты проекта, в том числе следует указать в каких базах индексируются данные издания - «Сеть науки» (Web of Science Core Collection), «Скопус» (Scopus), РИНЦ, иные базы, а также указать тип публикации - статья, обзор, тезисы, монография, иной тип}
\begin{itemize}
  \item Proceedings of Joint International Workshop on Graph Data Management Experiences \& Systems (Grades) and Network Data Analytics (Nda), издатель  ACM, Scopus, статья
  \item Proceedings of International Conference on Extending Database Technology (EDBT), издатель OpenProceedings.org, Scopus, статья
  \item Proceedings of the ACM SIGPLAN Workshop on Partial Evaluation and Program Manipulation, издатель  ACM, Scopus, статья
  \item IEEE International Symposium on Parallel and Distributed Processing Workshops and Phd Forum, издатель  IEEE, Scopus, статья
  \item Lecture Notes in Computer Science, издатель Springer US, Scopus, Web of Science, статья
  \item ACM SIGPLAN International Conference on Software Language Engineering, издатель  ACM, Scopus, статья
  \item Proceedings of the ACM SIGPLAN International Conference on Certified Programs and Proofs, издатель ACM, Scopus, статья
  \item IEEE/ACM International Conference on Automated Software Engineering, издатель IEEE/ACM, Scopus, статья
  \item IEEE/ACM International Conference on Software Engineering, издатель IEEE/ACM, Scopus, статья
  \item The ACM Joint European Software Engineering Conference and Symposium on the Foundations of Software Engineering, издатель ACM, Scopus, статья
  
  
  
  
\end{itemize}

\textbf{Иные способы обнародования результатов выполнения проекта}
\begin{itemize}
\item Участие в постерных сессиях при конференциях SIGMOD, SPLASH, ICFP, VLDB
\item Проведение открытых лекций
\item Доклады на научных семинарах
\end{itemize}

\subsection{Число публикаций членов научного коллектива, опубликованных в период с 1 января 2016 года до даты подачи заявки}

\checkme{46, из них 34 – опубликованы в изданиях, индексируемых в Web of Science Core Collection или в Scopus.}

\subsection{Планируемое участие научного коллектива в международных коллаборациях (проектах) (при наличии)}

\vline
Руководитель проекта подтверждает, что
\begin{itemize}
\item все члены научного коллектива (в том числе руководитель проекта) удовлетворяют пунктам 6, 7, 13 конкурсной документации;
\item на весь период реализации проекта он будет состоять в трудовых отношениях с организацией;
\item при обнародовании результатов любой научной работы, выполненной в рамках поддержанного Фондом проекта, он и его научный коллектив будут указывать на получение финансовой поддержки от Фонда и организацию, а также согласны с опубликованием Фондом аннотации и ожидаемых результатов поддержанного проекта, соответствующих отчетов о выполнении проекта, в том числе в информационно-телекоммуникационной сети «Интернет»;
\item помимо гранта Фонда проект не будет иметь других источников финансирования в течение всего периода практической реализации проекта с использованием гранта Фонда;
\item проект не является аналогичным по содержанию проекту, одновременно поданному на конкурсы научных фондов и иных организаций;
\item проект не содержит сведений, составляющих государственную тайну или относимых к охраняемой в соответствии с законодательством Российской Федерации иной информации ограниченного доступа;
\item доля членов научного коллектива в возрасте до 39 лет включительно в общей численности членов научного коллектива будет составлять не менее 50 процентов в течение всего периода практической реализации проекта;
\item в установленные сроки будут представляться в Фонд ежегодные отчеты о выполнении проекта и о целевом использовании средств гранта.
\end{itemize}

\section{Содержание проекта}

\subsection{+Научная проблема, на решение которой направлен проект}

\textbf{ru}\\

Проект направлен на изучение и разработку методов и средств, позволяющих использовать операции и примитивы линейной алгебры для получения высокопроизводительных решений в широком спектре прикладных областей, включающем (но не ограниченном) статический анализ кода, графовые базы данных, анализ социальных сетей.

Хотя прикладное значение линейной алгебры хорошо известно, наиболее широкое распространение на практике она получила в виде математических библиотек линейной алгебры (например, различные реализации BLAS, Basic Linear Algebra Subprograms). При этом, применение алгоритмов, основанных на линейной алгебре, в различных прикладных областях хотя и является традиционным (например, операций над матрицами смежности при анализе графов), до недавнего времени не существовало единого взгляда на такой подход к решению прикладных задач. Появление же стандарта GraphBLAS (Aydın Bulu\c{c}, Timothy Mattson, Scott McMillan, Jos\'{e} Moreira, Carl Yang, "The GraphBLAS C API Specification", version 1.3.0, 2019), с одной стороны, привело к общему базису использование примитивов и операций линейной алгебры в широком спектре прикладных задач, а также поставило ряд вопросов, ответы на которые активно ищутся мировым научным сообществом в настоящее время. Вопросы затрагивают все уровни, начиная от аппаратного через системный до прикладного. Без работы над аппаратным уровнем невозможно обеспечить максимальную эффективность решений. На системном уровне разрабатываются библиотеки и средства разработки, позволяющие, с одной стороны, минимизировать затраты на разработку прикладных решений основанных на линейной алгебре, с другой, эффективно использовать имеющиеся аппаратные средства. Прикладной уровень включает изучение областей, в которых перспективно применение решений на основе операций и примитивов линейной алгебры, а также разработку инструментов и приложений для конечных пользователей, основанных на линейной алгебре.

На аппаратном и системном уровне существенная часть вопросов связана со следующими ключевыми особенностями GraphBLAS. Во-первых, данный стандарт рассматривает прежде всего разреженные структуры данных (разреженные матрицы, вектора), что связано с разреженностью большинства прикладных объектов. Это существенно снижает эффективность традиционных аппаратных и программных средств параллельных вычислений (такие как использование векторизации, SIMD, GPGPU) из-за нерегулярного доступа к памяти, вызванного особенностью форматов представления данных. Это, в свою очередь, стимулирует поиск новых аппаратных архитектур и моделей и средств программирования для них и для уже существующих платформ. Во-вторых, постулируется необходимость реализации абстрактных примитивов и операций, параметризованных такими объектами, как моноид или полукольцо. Это существенно отличает его от принятого подхода в современных библиотеках линейной алгебры, поддерживающих, в основном, числа с плавающей точкой (с одинарной или двойной точностью) и соответствующие операции. В-третьих, набор примитивов и операций разрабатывается исходя из того, что они должны стать строительными блоками для прикладных алгоритмов, что выдвигает более нетривиальные требования как к составу самого набора, так и к возможности легко строить композиции операций, входящих в него. Последние два пункта вынуждают искать новые подходы к разработке библиотек реализаций операций линейной алгебры и алгоритмов на их основе, так как необходимо совместить разработку высокопроизводительных решений, традиционно ведущуюся на языках низкого уровня, и создание удобного уровня абстракций, возможное, в основном, с использованием соответствующих языков высокого уровня. Это, в свою очередь, приводит к необходимости разрабатывать новые языки программирования и техники оптимизаций программ, написанных на них.

Сложность при работе в данном направлении заключается в том, что традиционно для разработки высокопроизводительных решений использовались императивные (зачастую С-подобные) языки. Однако многие востребованные оптимизации, такие как устранение промежуточных структур данных (в частности дефорестация), частичное применение к известным данным (специализация), реализуемы в языках, тяготеющих к функциональной парадигме. Стоит отметить, что указанные выше оптимизации могут рассматриваться как частные случаи суперкомпиляции, предложенной и активно изучавшейся В.Ф.Турчиным и А.П.Ершовым ещё в 70-е годы. И хотя работы по созданию языка для высокопроизводительных вычислений, поддерживающих подобные оптимизации активно ведутся в настоящее время (например, язык Futhark, поддерживающий kernel fusion, частный случай дефорестации: Троэлс Хенриксен, "Futhark: Purely Functional GPU-Programming with Nested Parallelism and In-Place Array Updates", 2017) и обсуждаются в сообществе GraphBLAS (Карл Янг "GraphBLAST: A High-Performance Linear Algebra-based Graph Framework on the GPU", 2020), вопрос о границах применимости таких оптимизаций и о применимости суперкомпиляции для их выполнения всё ещё остаётся открытым.

Вместе с этим, прикладной уровень, где ведётся поиск новых областей применения и способов использования линейной алгебры, нуждается в усиленном изучении и структуризации. Использование операций линейной алгебры для выполнения запросов к графовым базам данных уже подтвердило свою практическую ценность, например, в графовой базе данных RedisGraph (Pieter Cailliau, Tim Davis et al., "RedisGraph GraphBLAS Enabled Graph Database", 2019), практически полностью основанной на представлении графов в виде матриц и использующей операции над ними для выполнения пользовательских запросов. Несмотря на это, всё ещё не существует формального описания процедуры трансляции прикладных языков запросов в операции линейной алгебры, хотя и предпринимаются попытки формально описать семантику некоторых языков запросов к графовым базам данных, например, в работах Надима Франсиса "Formal Semantics of the Language Cypher" и Джозефа Мартона "Formalising openCypher Graph Queries in Relational Algebra". Однако в данных работах не используются системы автоматической проверки доказательств, использование которых стало стандартом "де факто" при описании формальных свойств языков и связанных с ними инфраструктур (трансляторов, оптимизаторов, компиляторов), ввиду их сложности и невозможности вручную проверить корректность описания и доказательства свойств.

Другой показательный пример --- статический анализ кода, где большой объём обрабатываемых данных делает востребованными параллельные (например, работы Хайбо Ю "Parallelizing Flow-Sensitive Demand-Driven Points-to Analysis", 2020 и Ронга Гу "Towards Efficient Large-Scale Interprocedural Program Static Analysis on Distributed Data-Parallel Computation", 2021) и инкрементальные (например, работа Йенса Ван дер Пласа "Incremental Flow Analysis through Computational Dependency Reification", 2020) алгоритмы (дабы избегать повторных вычислений при незначительных изменениях кода), а многие ключевые задачи могут быть сформулированы в терминах достижимости с особого вида ограничениями в графах. При этом, применимость алгоритмов для решения таких задач, построенных на основе линейной алгебры и позволяющих использовать современные программные и аппаратные средства для параллельной и инкрементальной обработки данных, в контексте статического анализа всё ещё остаётся мало изученной ввиду того, что они были предложены сравнительно недавно (работы Рустама Азимова "Context-free path querying by matrix multiplication", 2018, и Егора Орачева "Context-Free Path Querying by Kronecker Product", 2020).
\\
\\
\textbf{en}\\
The project is aimed at studying and developing methods and tools that allow for using operations and primitives of linear algebra to obtain high-performance solutions in a wide range of applied areas, including (but not limited to) static code analysis, graph databases, and social network analysis.

Although the applied value of linear algebra is well known, in practice, it is most widely used in the form of mathematical libraries of linear algebra (for example, various implementations of BLAS, Basic Linear Algebra Subprograms). At the same time, even though the use of algorithms based on linear algebra in various applied areas is common (for example, operations on adjacency matrices in the graph analysis), there was no common view on such an approach to solve the applied problems until recently. One the one hand, the creation of the GraphBLAS standard (Aydın Bulu\c{c}, Timothy Mattson, Scott McMillan, Jos\'{e} Moreira, Carl Yang, "The GraphBLAS C API Specification", version 1.3.0, 2019) led to a common basis for the use of primitives and operations of linear algebra in a wide range of applied problems. On the other hand, it raised many questions which are currently actively researched by the international scientific community. The questions concern every  level, starting from the hardware through the system level to the applied level. Without improving the hardware, it is impossible to ensure the maximum efficiency of solutions. The system level includes the development of libraries and tools, on the one hand, allowing to minimize the cost of developing applied solutions based on linear algebra, on the other hand, fascilitating efficient use of the available hardware. The applied level includes studying of areas in which the application of solutions based on operations and primitives of linear algebra is promising, as well as the development of linear algebra-based tools and applications for end-users.

At the hardware and system level, a significant part of the questions is related to the following key features of GraphBLAS. First, this standard primarily considers sparse data structures (sparse matrices, vectors) due to the sparsity of the most applied objects. This significantly reduces the efficiency of traditional hardware and software parallel computations (such as the use of vectorization, SIMD, GPGPU) due to irregular memory access caused by the peculiarity of data presentation formats. This, in turn, stimulates the search for the new hardware architectures and models and programming tools for them and for existing platforms. Second, it postulates the need to implement abstract primitives and operations parameterized by objects such as a monoid or semiring, which significantly distinguishes it from the existing approaches in modern linear algebra libraries that support mainly floating-point numbers (single or double precision) and related operations. Third, a set of primitives and operations is developed on the basis that they should become building blocks for applied algorithms, which puts forward more nontrivial requirements both for the composition of the set itself and for the ability to easily build compositions of operations included in it. The last two points force us to look for new approaches to the development of libraries of linear algebra operations and algorithms based on them, since it is necessary to combine the development of high-performance solutions, traditionally carried out in low-level languages, and the creation of a convenient abstraction level, which is possible mainly using the corresponding high-level languages. This, in turn, leads to the need to develop new programming languages and optimization techniques for programs written in them.

The difficulty in working in this direction lies in the fact that traditionally imperative (often C-like) languages were used to develop high-performance solutions, but many popular optimizations, such as the elimination of intermediate data structures (in particular, deforestation), partial application to known data (specialization), are implemented in languages tending to the functional paradigm. It is worth noting that the above optimizations can be considered as special cases of supercompilation, proposed and actively studied by V.F. Turchin and A.P. Ershov back in the 70s. And although work on the creation of a language for high-performance computing that supports such optimizations is actively underway (for example, the Futhark language supporting kernel fusion, a special case of the deforestation: Troels Henriksen, "Futhark: Purely Functional GPU-Programming with Nested Parallelism and In-Place Array Updates", 2017) and is discussed in the GraphBLAS community (Carl Yang "GraphBLAST: A High-Performance Linear Algebra-based Graph Framework on the GPU", 2020), the question of the applicability of such optimizations and the applicability of supercompilation to perform them still remains open.

At the same time, the applied level, where the search for new areas of application and ways of using linear algebra is being carried out, needs careful research and structuring. For example, the use of linear algebra operations for querying graph databases has already proven its practical value, for example, in the RedisGraph graph database (Pieter Cailliau, Tim Davis et al., "RedisGraph GraphBLAS Enabled Graph Database", 2019), which is almost completely based on the representation of graphs in the form of matrices and using operations on them to perform user queries. Nevertheless, there is still no formal description of the procedure for translating applied query languages into linear algebra operations, although attempts to formally describe the semantics of some query languages for graph databases are being made, for example, in the works of Nadime Francis "Formal Semantics of the Language Cypher" and J\'{o}zsef Marton "Formalizing openCypher Graph Queries in Relational Algebra". However, these works do not use systems of automatic proof checking, the use of which has become a de facto standard when describing the formal properties of languages and related infrastructures (translators, optimizers, compilers), due to their complexity and impossibility to manually check the correctness of the description and proof of properties.

Another illustrative example is static code analysis, where a large amount of processed data makes parallel (for example, works by Haibo Yu "Parallelizing Flow-Sensitive Demand-Driven Points-to Analysis", 2020, and Rong Gu "Towards Efficient Large-Scale Interprocedural Program Static Analysis on Distributed Data-Parallel Computation", 2021) and incremental (for example, the work by Jens Van der Plas "Incremental Flow Analysis through Computational Dependency Reification", 2020) algorithms in demand (to avoid repeated computations with minor code changes), and many key problems can be formulated in terms of reachability with special constraints in graphs. At the same time, the applicability of algorithms for solving such problems based on linear algebra which allow one to use of modern software and hardware for parallel and incremental data processing, in the context of static code analysis is still poorly understood due to the fact that they were proposed relatively recently (works by Rustam Azimov "Context-free path querying by matrix multiplication", 2018, and Egor Orachev "Context-Free Path Querying by Kronecker Product", 2020).


\subsection{+Научная значимость и актуальность решения обозначенной проблемы}

\textbf{ru}\\

Проблема высокопроизводительных вычислений, в частности в области обработки больших объёмов данных, остро стоит во многих как фундаментальных, так и прикладных областях. Подход, основанный на использовании примитивов и операций разреженной линейной алгебры, показал себя перспективным в решении данной проблемы. Поэтому всестороннее его изучение и развитие является актуальной задачей.

Так, разработка методов и средств программирования для современных массово-параллельных платформ, включающих графические ускорители общего назначения, позволит перенести решения, основанные на стандарте GraphBLAS, на эти платформы, тем самым увеличив производительность решений, как показано, например, Карлом Янгом и его коллегами в работе "GraphBLAST: A High-Performance Linear Algebra-based Graph Framework on the GPU". Вместе с этим, найденные удачные методы и разработанные удобные средства программирования массово-параллельных платформ могут найти своё применение при создании решений в других областях, связанных с разработкой высокопроизводительных решений. 

Разработка новых программно-аппаратных средств преследует ту же цель и актуальна в связи с тем, что, как показано в ряде недавних работ (например, W. S. Song, V. Gleyzer, A. Lomakin, and J. Kepner, "Novel graph
processor architecture, prototype system, and results"; Z. Zhang, H. Wang, S. Han, and W. J. Dally, "Sparch: Efficient architecture for sparse matrix multiplication"), современные архитектуры плохо подходят для высокопроизводительной обработки разреженных данных. Что показывает необходимость разработки новых архитектур, специфичных для решаемой задачи. Вместе с этим, актуальной является и проблема оптимизации исходного кода, требующая специфичных подходов, плохо реализуемых в современных языках, используемых в области высокопроизводительных вычислений, что, например, активно обсуждается в работе Карла Янга "GraphBLAST: A High-Performance Linear Algebra-based Graph Framework on the GPU" и в серии работ Троэлса Хенриксена (Troels Henriksen) и его команды, посвященных разработке языка программирования Futhark (функциональный язык программирования для высокопроизводительных систем). 


Описание формальных свойств языков и программных систем с использованием средств автоматической проверки корректности является общепризнанной, так как позволяет, с одной стороны, убедиться в корректности проводимых рассуждений, что невозможно без привлечения автоматических средств, ввиду большой сложности рассматриваемых систем, а с другой --- предоставлять корректные по построению инструменты и системы. Активная разработка таких проектов, как SQLCert (http://datacert.lri.fr/sqlcert/), являющего частью проекта DataCert (http://datacert.lri.fr/), поддерживаемого Agence Nationale de la Recherche (Франция), направленного на создание сертифицировано инфраструктуры исполнения SQL-запросов с использованием Coq, включающей и формальное описание языка запросов SQL, показывает востребованность подобных решений. Создание аналогичного проекта для анализа граф-структурированных данных (в частности, графовых баз данных), особенно с применением операций линейной алгебры, необходимо, так как это один из перспективных путей к высокопроизводительной обработки больших данных. Результаты в данной области необходимы при разработке систем, дающих гарантии целостности данных, безопасности, корректности обработки и т.д.

Понимание границ применимости алгоритмов, основанных на линейной алгебре, в статическом анализе поможет в поиске решений для высокопроизводительной обработки больших объёмов кода, что является актуальной задачей, связанной с такими важными аспектами, как поиск и устранение ошибок в программных системах, что связано с вопросами надёжности и безопасности программного обеспечения. Кроме этого, исследования в данной области необходимы для создания унифицированных компонент для высокопроизводительных вычислений. Так, например, одно из активно развивающихся направлений --- использование графовых баз данных и соответствующих алгоритмов для статического анализа кода (Oscar Rodrigez-Prieto, Alan Mycroft, Francisco Ortin, "An Efficient and Scalable Platform for Java Source Code Analysis using Overlaid Graph Representations", 2020). Это, в том числе, показывает необходимость изучения языков запросов и возможностей их трансляции в операции линейной алгебры.
Вместе с этим, оценка возможности использовать результаты работы алгоритмов статического анализа кода для улучшения решений, основанных на машинном обучении, необходима для создания интеллектуальных инструментов анализа большого объёма кода.
\\
\\
\textbf{en}\\

High-performance computing for big data analysis is a critical technology for both theoretical and applied areas. It is shown that the linear algebra-based approach is a promising way to achieve high performance in many areas. Thus it is necessary to investigate this approach in details and improve it.

The creation of new methods and tools for modern parallel hardware (multicore CPUs, GPGPUs) programming allows one to port GraphBLAS-based solutions on such platforms and improve performance of related applied solutions, which was shown by Carl Yang in "GraphBLAST: A High-Performance Linear Algebra-based Graph Framework on the GPU". The created methods and tools can also be applied for the development of new high-performance solutions in other areas.

The development of new software-hardware toolchains is also aimed to improve the performance of linear algebra-based solutions. It is necessary to create new toolchains, because it was shown that the available hardware is not suitable for high-performance processing of sparse data:  W. S. Song, V. Gleyzer, A. Lomakin, and J. Kepner, "Novel graph processor architecture, prototype system, and results"; Z. Zhang, H. Wang, S. Han, and W. J. Dally, "Sparch: Efficient architecture for sparse matrix multiplication". Thus, the creation of new hardware and related software tools is necessary. Moreover, specific software optimizations are required for linear algebra-based algorithms. These optimizations are actively discussed in the community (Carl Yang et al, "GraphBLAST: A High-Performance Linear Algebra-based Graph Framework on the GPU"; Troels Henriksen and Futhark research team), but many of such optimizations are too hard to implement for programming languages currently used for high-performance computing.

The formal description of programming languages and software systems which use respective tools is the best practice in the community. Such an approach allows one to avoid errors in descriptions and algorithms using theorem provers and other related tools. Moreover, it allows one to create correct-by-construction systems and tools. Such project as SQLCert (http://datacert.lri.fr/sqlcert/), the part of DataCert (http://datacert.lri.fr/, Agence Nationale de la Recherche, France), is aimed to create certified infrastructure for SQL queries evaluation using Coq proof assistant. The creation of a similar project aimed at the certification of graph-structured data analysis algorithms and related query languages is required because linear algebra-based graph analysis algorithms is a base for high-performance data analysis in many areas. Certified algorithms are necessary to create systems with provable correctness, safety, and other important properties. 

It is necessary to investigate the applicability of linear algebra-based algorithms for static code analysis in order to find high-performance solutions in this area. Such solutions are necessary because they allow one to analyze real-world software systems and detect problems related to their correctness, safety, and other important properties. Moreover, it is necessary to create unified components for graph analysis in different areas. It was shown that utilization of graph databases is a promising way to create modern tools for static code analysis (Oscar Rodrigez-Prieto, Alan Mycroft, Francisco Ortin, "An Efficient and Scalable Platform for Java Source Code Analysis using Overlaid Graph Representations", 2020). Thus, it is necessary to improve the expressivity of graph query languages and to improve the performance of graph query algorithms. Linear algebra-based algorithms is a way to solve these problems. Also, it is necessary to find ways to utilize the results of such static code analysis to improve machine learning based solutions for code analysis because it should allow one to create efficient tools for intelligent analysis of huge software systems.  

\subsection{+Конкретная задача (задачи) в рамках проблемы, на решение которой направлен проект, ее масштаб и комплексность}

\textbf{ru}\\
%
В рамках улучшения программно-аппаратной поддержки примитивов и операций разреженной линейной алгебры, предусматриваемых стандартом GraphBLAS, ставится две задачи. Первая --- разработка новых подходов и инструментальных средств для создания высокопроизводительных библиотек разреженной линейной алгебры для современного аппаратного обеспечения с использованием существующих языков программирования. Здесь предполагается, с одной стороны, поиск, разработка, реализация и экспериментальное исследование алгоритмов работы с разреженными матрицами, работающих на таких аппаратных ускорителях, как графические процессоры общего назначения (GPGPU) и ПЛИС (FPGA). С другой стороны, планируется разработать подход и соответствующие инструментальные средства разработки программного обеспечения, позволяющие прозрачным для дальнейшего использования образом объединить преимущества языков высокого уровня, такие как удобные абстракции для распределённого, параллельного и асинхронного программирования или богатую систему типов, дающую существенные статические гарантии корректности кода, с высокой производительностью специализированного аппаратного обеспечения.  

Вторая задача --- разработка нового программно-аппаратного стека, предназначенного для разработки высокопроизводительных решений на основе операций и примитивов разреженной линейной алгебры. В рамках данной задачи планируется совместная разработка (co-design) специализированного языка программирования высокого уровня (domain specific language, DSL), позволяющего описывать необходимые операции и примитивы линейной алгебры, методов и алгоритмов оптимизации программ на разработанном DSL, его компилятора, и аппаратной архитектуры, специализированной для разработанного языка и направленной на повышение производительности. Вместе с этим, планируется разработка библиотеки, максимально удовлетворяющей стандарту GraphBLAS и экспериментальное исследование полученного решения на прикладных алгоритмах.

В качестве подзадачи планируется исследование таких методов оптимизации, как суперкомпиляция, смешанные вычисления, в двух направлениях. Первое: применение суперкомпиляции для оптимизации программ, написанных на разработанном DSL. Здесь планируется изучить применимость существующих техник суперкомпиляции и их эффект на производительность целевого кода. Возможно, будут разрабатываться новые методы суперкомпиляции, специализированные для разрабатываемого языка и предметной области. Второе: применение частичных вычислений для оптимизации решений во время выполнения. Здесь необходимо найти сценарии, при которых специализация на данные, становящиеся известными во время выполнения, позволяет повысить производительность решения.

В рамках исследования перспективных областей применения алгоритмов, основанных на линейной алгебре, планируется работа в двух прикладных направлениях, являющихся на сегодняшний момент одними из самых активно использующих линейную алгебру. Первое направление --- это графовые базы данных. И здесь планируется изучить границы применимости линейной алгебры для выполнения запросов к графовым базам данных. Для этого планируется разработать схему трансляции языка запросов Cypher в операции линейной алгебры и доказать её корректность. Для этого планируется использовать систему автоматического доказательства теорем Coq. Отдельной задачей ставится интеграция полученных результатов с международным стандартом языка запросов к графовым базам данных (GQL Standard, https://www.gqlstandards.org/), так как кроме схемы трансляции планируется получить формальное описание семантики языка Cypher (на основе которого разрабатывается стандарт), а данные артефакты могут быть полезны при формальном описании языка, необходимом для стандарта.

Второе направление --- статический анализ программного кода. Здесь планируется изучение применимости различных алгоритмов поиска путей с ограничениями в терминах формальных языков к статическому анализу кода и адекватность получаемых результатов для создания прикладных решений, в частности на основе статистических подходов и методов машинного обучения. Необходимо оценить эффективность алгоритмов, основанных на линейной алгебре, для анализа больших объёмов кода, в том числе, эффект от применения GPGPU для решения подобных задач. Вместе с этим, планируется разработать методы использования получаемых результатов анализов для построения прикладных решений, основанных на методах машинного обучения, что позволит улучшить интеллектуальный анализ больших объёмов кода.
\\
\\
\textbf{en}\\
 
 We formulate two tasks on the improvement of hardware and software support of sparse linear algebra primitives and operations introduced by GraphBLAS. The first one is to develop new approaches and tools for development of high-performance linear algebra-based libraries for modern hardware (multicore CPUs, GPGPUs) using high-level programming languages. We plan to develop and evaluate new algorithms of sparse linear algebra for such accelerators as GPGPUS and FPGA. Also, we plan to propose an approach to utilize high-level languages and respective abstractions for parallel and asynchronous programming for transparent development of software for hardware accelerators. Respective tools and libraries should be developed.
 
 The second task is to create a new software-hardware toolchain for high-performance linear algebra-based solutions. It includes co-design of high-level linear algebra oriented domain-specific language (DSL), optimizations methods and algorithms for this DSL, respective compiler, and specialized hardware, optimized for the DSL. For evaluation, we plan to develop a subset of GraphBLAS in the DSL.
 
 We plan to investigate such methods of optimization, as supercompilation and partial evaluation. Supercompilation will be applied for the optimization of programs written in the DSL. The efficiency of such an approach for linear algebra-based algorithms will be investigated. New problem-specific methods of supercompilation will be developed if required. The partial evaluation will be applied to optimize programs in runtime. We plan to find scenarios when specialization on data being known in running time can improve performance.
 
 We plan to investigate two promising areas of linear algebra-based algorithms application. The first area is of graph databases. We plan to investigate the applicability of linear algebra-based algorithms for parallel processing of graph queries. Translation of Cypher graph query language to the operation of linear algebra will be formally described and verified using Coq proof assistant. Also, we plan to share our results with the graph query language standardization community (GQLStandard, https://www.gqlstandards.org/). Such results as a formal description of Cypher semantics can be useful for standardization.
 
 The second is static code analysis. We plan to investigate the applicability of parallel algorithms for context-free path querying in static code analysis. It is important to evaluate GPGPU based algorithms on a large codebase. Also, we plan to find a way to use the results of such algorithms to improve machine learning based solutions for code analysis.

\subsection{+Научная новизна исследований, обоснование достижимости решения поставленной задачи (задач) и возможности получения запланированных результатов}

\textbf{ru}\\

Рассматриваемая в проекте область активно развивается. Все поставленные задачи интересуют специалистов в соответствующих областях, что подтверждается наличием работ, опубликованных в недавнее время в рецензируемых профильных журналах и представленных на ведущих профильных конференциях, в том числе участниками проекта. Это позволяет гарантировать новизну ожидаемых результатов и их соответствие мировому уровню.

Поскольку некоторые задачи очень трудны, гарантировать их полное решение невозможно. Таковой, например, является задача разработки нового программно-аппаратного стека для высокопроизводительных решений, основанных на разреженной линейной алгебре. Однако получение даже частичных результатов или улучшение существующих (например, изучение и описание границ применимости суперкомпиляции для оптимизации подпрограмм линейной алгебры) будет существенным вкладом. Вместе с этим, в проекте предусмотрено решение ряда интересных и ожидаемо разрешимых задач.

Так, например, опыт участников в разработке и исследовании методов суперкомпиляции и смешанных вычислений (Д.А. Березун, А.В. Тюрин), а также в разработке, как алгоритмов на основе линейной алгебры и GraphBLAS, так и самих алгоритмов линейной алгебры, применимых для анализа графов (С.В. Григорьев, Р.Ш. Азимов, А.К. Терехов, Е.С. Орачев), должен позволить всесторонне подойти к изучению вопроса создания специализированного языка для описания примитивов и операций разреженной линейной алгебры и применимости методов суперкомпиляции и смешанных вычислений для оптимизации программ написанных на нём. Так как применимость суперкомпиляции в данном контексте мало изучена, получение как положительных (создание языка и эффективного суперкомпилятора для него), так и отрицательных (выводы о неприменимости тех или иных техник  в рамках изучаемой задачи) результатов будет существенным вкладом.

Опыт участников в разработке алгоритмов выполнения запросов к графам, основанных на линейной алгебре, в использовании этих алгоритмов при исполнении запросов к реальным графовым базам данных (С.В. Григорьев, Р.Ш. Азимов, А.К. Терехов, Е.С. Орачев), в формализации различных аспектов языков программирования (в том числе семантик) и доказательстве их свойств (А.В. Подкопаев, Д.А. Березун), позволят всесторонне подойти к вопросу применимости операций линейной алгебры для выполнения запросов к графовым базам данных. Отметим, что некоторые конструкции языка Cypher будут формально изучаться с этой точки зрения впервые, однако у участников есть опыт практической реализации данных конструкций и их трансляции в операции линейной алгебры, что поможет получить здесь новые результаты. Вместе с этим, некоторые аспекты формального описания семантики языка Cypher изучены достаточно хорошо, но не полностью, и здесь планируется опираться на результаты таких исследователей, как Джозеф Мартон (József Marton) и Надим Фрэнсис (Nadime Francis).

Разработка и экспериментальное исследование параллельных инкрементальных алгоритмов статического анализа кода, основанных на достижимости с ограничениями в терминах формальных языков, активно ведётся в настоящее время, многие вопросы и прикладные задачи всё ещё не решены окончательно, поэтому анализ применимости существующих, ещё не изученных до конца, алгоритмов и разработка новых в данной области будет являться новыми результатами. При этом, участниками проекта разработан ряд алгоритмов достижимости и поиска путей с ограничениями в терминах формальных языков, позволяющих использовать современные параллельные архитектуры (многоядерные процессоры и GPGPU) и показавших свою высокую производительность и потенциал к инкрементальной обработке данных. Это позволит провести всестороннее исследование применимости данных алгоритмов в контексте статического анализа кода, и при необходимости внести в них модификации, продиктованные спецификой предметной области. 

Вместе с этим, опыт участников в разработке прикладных решений анализа программного кода, в том числе, основанных на статистических методах и методах машинного обучения (Т.А. Брыксин), должен позволить детально изучить возможность использования результатов рассматриваемых алгоритмов для улучшения прикладных решений. Данное направление (использование результатов межпроцедурного анализа кода, представленных в матричном виде, для улучшения задач обработки кода методами машинного обучения) начало развиваться сравнительно недавно, однако результаты Юлей Суй (Yulei Sui), представленные в 2020 году, показывают его перспективность. Имеющийся у участников проекта опыт должен позволить существенно продвинуться в данном направлении и получить новые результаты.
\\
\\
\textbf{en}\\

The problems listed here are of interest to researchers: a number of papers is published in the last few years in peer-reviewed journals and presented at the high-rated conferences. Some of these papers are written by the members of this project. This guarantees the attention of the scientific community to the results and their good quality.

Since some of the problems are nontrivial, we cannot guarantee that a complete solution will be found during this project. The example of a nontrivial problem is the development of hardware-software system for high-performance computations based on sparse linear algebra. But even finding a partial solution or improving the existing results (for example, an investigation of limitations of supercompilation applicability for optimization of linear algebra-based algorithms)
will be a significant contribution. Note, that we also plan to work on problems which are likely to be solved completely.

The experience of project members in such areas as compilers design, supercompilation (D. Berezun, A. Tyurin), linear algebra-based algorithms development, graph analysis (S. Grigorev, R. Azimov, A. Terekhov, E. Orachev) allows us to develop a specialized language for linear algebra operation description and to investigate the applicability of supercompilation to optimize algorithms written in the developed language comprehensively. Both negative and positive results in this area are important: the creation of language and efficient supercompiler for it as well as proving that some supercompilation techniques are not applicable in particular cases.

The experience of project members in such areas as linear algebra-based graph querying algorithms development, application of these algorithms in the context of real-world graph databases (S. Grigorev, R. Azimov, A. Terekhov, E. Orachev), formalization and mechanization of different properties of programming languages (including formal semantics description) (A. Podkopaev, D. Berezun) allows us to investigate the problem of the applicability of linear algebra to graph querying comprehensively. Note that we plan to provide the first formal description for some parts of Cypher language. These parts were implemented by project members in real-world applications, which is instrumental in getting new results. On the other hand, a partial formal description of Cypher is provided by Jozsef Marton and Nadime Francis, and we plan to use these results as a base for our research and to improve upon them.

The development of parallel and incremental algorithms for static code analysis based on formal language constrained reachability is a relevant research topic. As far as there are a big number of open questions in this area, it is important to investigate the applicability of existing algorithms and to develop new algorithms. A number of parallel context-free path querying algorithms for multicore CPU and GPGPU are developed by project members. These algorithms demonstrate high performance in the context of graph databases. We plan to investigate the applicability of these algorithms for static code analysis and modify them to handle area-specific cases.

The experience of project members in static code analysis and related tools creation, and in machine learning-based static code analysis (T. Bryksin) allows us to investigate the applicability of the described analysis results to improve the quality of respective tools. As it was recently shown by Yulei Sui, the utilization of results of interprocedural code analysis is a promising way to improve machine learning-based code analysis solutions. Our experience helps us to improve existing results and investigate new areas to use this approach.

\subsection{+Современное состояние исследований по данной проблеме, основные направления исследований в мировой науке и научные конкуренты}

\textbf{ru}\\

В последние годы роль линейной алгебры в различных прикладных областях существенно возросла. В частности, был разработан стандарт GraphBLAS API (Aydın Buluç, Timothy Mattson, Scott McMillan, Jos\'{e} Moreira, Carl Yang, The GraphBLAS C API Specification, version 1.3.0, 2019), описывающий примитивы линейной алгебры и операции над ними, необходимые для описания алгоритмов анализа графов. Необходимо отметить, что данный стандарт показал свою применимость далеко за пределами задач, связанных с анализом графов. Например, в таких областях, как машинное обучение (J. Kepner, M. Kumar, J. Moreira, P. Pattnaik, M. Serrano, and H. Tufo, Enabling massive deep neural networks with the GraphBLAS, 2017) и биоинформатика (O. Selvitopi, S. Ekanayake, G. Guidi, G. Pavlopoulos, A. Azad, and A. Bulu\c{c}, "Distributed many-to-many protein sequence alignment using sparse matrices", 2020). В настоящее время исследователи всего мира работают над развитием данного подхода. В частности, ищутся возможности сведения классических фундаментальных и прикладных задач к операциям линейной алгебры (например, частичное решение сведения поиска в глубину получено только в 2019 году Daniele G. Spampinato в работе "Linear Algebraic Depth-First Search", а полного решения всё ещё не существует), разрабатываются новые алгоритмы для выполнения базовых операций, исследуются форматы представления разреженных структур данных, характерных для прикладных областей, разрабатываются архитектуры аппаратных ускорителей операций линейной алгебры и принципы их построения.

В частности, предпринималось несколько попыток реализации стандарта GraphBLAS на графических ускорителях общего назначения. Наиболее успешный проект возглавляет Айдын Булуч (Aydın Buluç, "GraphBLAST: A High-Performance Linear Algebra-based Graph Framework on the GPU"). Кроме того, Тимоти Дэвис (Timothy A. Davis, "Algorithm 1000: SuiteSparse:GraphBLAS: Graph Algorithms in the Language of Sparse Linear Algebra", 2019) также планирует использование графических ускорителей в своей реализации стандарта. Однако в этих проектах ученые  столкнулись с несколькими серьёзными проблемами, обсуждаемыми, в том числе, в указанной статье. Первая --- необходимость специфических оптимизаций, устраняющих промежуточные структуры данных и объединяющих код, работающий с одними и теми же данными (kernel fusion), которые трудно реализуемы для Си-подобных языков, на которых в настоящее время ведутся разработки указанных проектов. Вместе с этим, подобные оптимизации хорошо изучены для других, более высокоуровневых и дающих больше статических гарантий языков (T. Henriksen, N. G. W. Serup, M. Elsman, F. Henglein, and C. E. Oancea, "Futhark: Purely functional gpu-programming with nested parallelism and in-place array updates", 2017) и структур данных (O. Kiselyov, A. Biboudis, N. Palladinos, and Y. Smaragdakis, "Stream fusion, to completeness", 2017). Но результаты данных исследований показывают, что выбранные в них способы решения проблемы весьма ограничены и, в частности, не позволяют реализовывать операции разреженной линейной алгебры. С другой стороны, для функциональных языков существует более общая техника, называемая дефорестацией, позволяющая выполнять требуемые оптимизации (P. Wadler, "Deforestation: transforming programs to eliminate trees", 1990). Более общий подход к оптимизациям, включающий данную технику, называется суперкомпиляция и хорошо изучен. Так, группы под руководством В.Ф.Турчина занимались изучением автоматического преобразования программ посредством суперкомпиляции. Создание применимых на практике решений, основанных на данных методах оптимизации программ является активно исследуемой областью. Так, например, И.Г.Ключников и С.А.Романенко в 2009 году сумели применить идеи суперкомпиляции для функций высших порядков (И.Г.Ключников, "Суперкомпиляция функций высших порядков", 2010), а затем предложили многорезультатную и многоуровневую суперкомпиляцию (I.Klyuchnikov, S.A.Romanenko, "Multi-result supercompilation as branching growth of the penultimate level in metasystem transitions", 2011; I.Klyuchnikov, S.A.Romanenko, "Higher-level supercompilation as a metasystem transition", 2012) позволяющие ещё лучше оптимизировать программы. Однако существенной проблемой на пути применения суперкомпиляции является то, что создание эффективного суперкомпилятора для реального языка программирования --- это трудоёмкая задача и даже если она решена, всё равно эффект применения суперкомпилятора в общем случае остаётся непредсказуемым. 

Параллельно ведутся разработки специализированных архитектур процессоров для операций линейной алгебры. Стоит разделить два направления. Первое --- создание процессоров для решения задач, связанных с анализом графов. Подробный анализ текущего состояния области приводится в работе Y. Horawalavithana "On the design of an efficient hardware accelerator for large scale graph analytics", 2016. Среди работ в данном направлении выделяется работа W. S. Song, V. Gleyzer, A. Lomakin, and J. Kepner, "Novel graph processor architecture, prototype system, and results", 2016, тем, что в ней предлагается создать процессор, набор инструкций которого ориентирован на операции разреженной линейной алгебры. Однако данная работа ограничивается операцией перемножения матриц, чего недостаточно для реализации всех необходимых функций. Второе направление связано с разработкой специализированных архитектур для определённых операций линейной алгебры (чаще всего это умножение разреженной матрицы на разреженную матрицу). В этой области активно работают несколько групп и ими опубликованы следующие работы: Z. Zhang, H. Wang, S. Han, and W. J. Dally, "Sparch: Efficient architecture for sparse matrix multiplication", 2020; M. Soltaniyeh, R. P. Martin, and S. Nagarakatte, "Synergistic cpu-fpga acceleration of sparse linear algebra", 2020; S. Pal, J. Beaumont, D. Park, A. Amarnath, S. Feng, C. Chakrabarti, H. Kim, D. Blaauw, T. Mudge, and R. Dreslinski, "Outerspace: An outer product based sparse matrix multiplication accelerator", 2018. Недостатком этих работ является то, что они нацелены на ускорение одной конкретной операции. Общим недостатком всех указанных работ является то, что они не включают в рассмотрение оптимизации на уровне языка и его компилятора (оптимизатора), а ведь многие оптимизации выполнимы только на этом уровне. Например, выполнение упомянутого выше объединения операций с целью устранить промежуточные структуры данных (скажем, при последовательном сложении нескольких матриц) на аппаратном уровне вряд ли возможно, в то время как на уровне компилятора для этого существуют указанные ранее техники.

Учитывая то, что перспективные оптимизации лучше всего разработаны для функциональных языков программирования, необходимо обратить внимание на работы, связанные с компиляцией программ на функциональных языках в специализированные аппаратные архитектуры. С одной стороны, активно разрабатываются специализированные процессоры для функциональных языков программирования: M. Naylor and C. Runciman, "The Reduceron: Widening the Von Neumann bottleneck for graph reduction using an fpga", 2008; A. Boeijink, P. K. F. Hölzenspies, and J. Kuper, "Introducing the Pilgrim: A processor for executing lazy functional languages", 2011; R. Coelho, F. Tanus, A. Moreira, and G. Nazar, "Acqua: A parallel accelerator architecture for pure functional programs", 2020. Данные проекты нацелены на создание процессора общего назначения, специализированного для функциональных языков программирования, и на текущий момент недостаточно производительны, чтобы послужить платформой для ускорения операций линейной алгебры. С другой же стороны, активно развивается направление, позволяющее компилировать код на функциональных языках в архитектуры, специализированные для данного кода. Таким образом, этот подход позволяет получать проблемно-специфичные ускорители для конкретных задач. Данный подход хорошо исследован в работах Ричарда Тоунсвенда (Richard Townsend, "Compiling Irregular Software to Specialized Hardware", 2019; "From functional programs to pipelined dataflow circuits", 2017) под руководством Стефена Эдвардса (Stephen A. Edwards). Данный подход перспективен для решения поставленных в рамках данного проекта задач, однако в указанных исследованиях не уделяется особого внимания высокоуровневым оптимизациям исходных программ и изучению применимости данного подхода для создания ускорителей для алгоритмов, основанных на операциях линейной алгебры.


Формализацией семантик языков запросов к данным и построением сертифицированных в Coq средств анализа и исполнения запросов активно занимается группа исследователей в рамках проекта DataCert (http://datacert.lri.fr/), цель которого сертифицировать и проверифицировать с использованием системы Coq системы обработки данных. В рамках данного проекта описана в Coq семантика языка SQL: Véronique Benzaken, Evelyne Contejean, "SQLCert: Coq mechanisation of SQL’s compilation: Formally reconciling SQL and (relational) algebra", 2017. Для языков запросов к графам подобные результаты ещё не представлены, однако, для языка Cypher описана его формальная семантика "на бумаге": J\'{o}zsef Marton, Gábor Szárnyas, and Dániel Varró, "Formalising openCypher Graph Queries in Relational Algebra", 2017; Nadime Francis, et al, "Formal Semantics of the Language Cypher", 2018. Кроме этого, активно развивается проект, описывающий семантику Cypher в терминах языка Datalog: Filip Murlak, Jan Posiadała, and Paweł Susicki, "On the semantics of Cypher's implicit group-by", 2019. Однако данные работы не используют средств автоматической проверки доказательств (типа Coq), не рассматривают вопрос трансляции конструкций исходного языка в операции линейной алгебры, и не описывают некоторые важные с прикладной и содержательные с исследовательской точек зрения особенности языка, такие как недавно предложенное Тобиасом Линдакером расширение, позволяющее описывать регулярные и контекстно-свободные ограничения на пути ((Tobias Lindaaker, Path Pattern, https://github.com/thobe/openCypher/blob/rpq/cip/1.accepted/CIP2017-02-06-Path-Patterns.adoc). Отметим, что данное расширение было недавно реализовано на практике и показало свою применимость для решения прикладных задач (Arseniy Terekhov, et al, "Multiple-Source Context-Free Path Querying in Terms of Linear Algebra", (2021)). Это делает задачу формального описания данного расширения более актуальной.

Разработка и исследование параллельных инкрементальных алгоритмов межпроцедурного статического анализа кода активно ведётся несколькими группами исследователей. За последнее время получен ряд результатов, включающий, в том числе, результаты Ю Су и коллег (Y. Su, D. Ye, J. Xue and X. Liao, "An Efficient GPU Implementation of Inclusion-Based Pointer Analysis", 2016; "Parallel Pointer Analysis for Large-Scale Software", 2015), показавших перспективность использования графических ускорителей в данной области. Также, в данной области известны результаты Ронг Гу (R. Gu et al., "Towards Efficient Large-Scale Interprocedural Program Static Analysis on Distributed Data-Parallel Computation", 2021), рассматривающего, прежде всего, распределённые алгоритмы, а также результаты Хайбо Ю (H. Yu, Q. Sun, K. Xiao, Y. Chen, T. Mine and J. Zhao, "Parallelizing Flow-Sensitive Demand-Driven Points-to Analysis", 2020) и  Цзишэн Чжао (Jisheng Zhao, Michael G. Burke, and Vivek Sarkar, "Parallel sparse flow-sensitive points-to analysis", 2018). Данные работы основаны на наблюдении, что многие виды межпроцедурного анализа, как было показано ещё Томасом Репсом, сводимы к поиску путей с контекстно-свободными ограничениями в графе, и в них предлагаются различные варианты параллельных алгоритмов для решения данной задачи. Однако в современных работах не рассматривается подход, основанный на линейной алгебре. Вместе с этим, для поиска путей с контекстно-свободными ограничениями за последнее время был предложен ряд алгоритмов, основанных на операциях линейной алгебры, и было показано, что такие алгоритмы позволяют использовать современные аппаратные средства для параллельных вычислений, такие как многоядерные центральные процессоры и графические ускорители общего назначения. Сделано это было в работах Рустама Азимова (Rustam Azimov, et al, "Context-free path querying by matrix multiplication", 2018), Терехова Арсения (Arseniy Terekhov, et al, "Context-Free Path Querying with Single-Path Semantics by Matrix Multiplication", 2020), Егора Орачева (Egor Orachev, et al, "Context-Free Path Querying by Kronecker Product", 2020). Однако данные работы рассматривали применение указанных алгоритмов а графовых базах данных, где аналогичная задача, поставленная Михалисом Яннакакисом, изучалась во многом независимо. Что привело к тому, что применимость эффективных параллельных алгоритмов, предложенных в сообществе, занимающемся графовыми базами данных, до сих пор не изучена.

Важно, что для практического применения статического анализа необходимо не только обеспечить эффективное решение исходной задачи, но и предоставить пользователю (разработчику) механизмы дальнейшего анализа полученных результатов и их применения для решения прикладных задач. Одно из перспективных направлений --- это создание интеллектуальных помощников на основе методов машинного обучения. Так, в работе Юлей Суй (Yulei Sui, Xiao Cheng, Guanqin Zhang, and Haoyu Wang, "Flow2Vec: value-flow-based precise code embedding", 2020) предлагается подход, позволяющий использовать результаты поиска путей с контекстно-свободными ограничениями для улучшения качества работы прикладных решений, основанных на методах машинного обучения. Хотя в работе и рассматривается подход, основанный на линейной алгебре, в ней не обсуждаются вопросы эффективной реализации соответствующих алгоритмов и не изучаются границы применимости данного подхода.
\\
\\
\textbf{en}\\

The importance of linear algebra-based approaches in applications grows rapidly. GraphBLAS API standard (Aydın Buluc, TimothyMattson, Scott McMillan, Jos ́e Moreira, Carl Yang," The GraphBLAS C API Specification", version1.3.0, 2019) which describes primitives and operations of linear algebra for graph analysis, was proposed recently. It was shown that this standard is applicable not only for graph analysis algorithms development, but also for algorithms development in such areas as machine learning (J. Kepner, M. Kumar, J. Moreira, P. Pattnaik, M.Serrano, and H. Tufo, "Enabling massive deep neural networks with the GraphBLAS", 2017) and bioinformatics (O. Selvitopi, S. Ekanayake, G. Guidi, G. Pavlopoulos, A. Azad, and A. Buluc, "Distributed many-to-many protein sequence alignment using sparse matrices", 2020). Thus many researchers take an effort to improve this approach. Ways to reduce tasks to linear algebra operations are investigated. For example, partial reduction of the depth-first search was proposed in 2019 by Daniele G. Spampinato in "Linear Algebraic Depth-First Search", while the complete reduction is still not developed. On the other hand, new algorithms for basic operations (such as matrix-matrix multiplication), new data structures for the representation of sparse matrices and vectors, and new hardware accelerators for linear algebra operations are developed.

One of the most successful attempts to implement GraphBLAS API on GPGPU is a GraophBLAST project, led by Aydin Bulu\c{c} (Aydın Bulu\c{c}, "GraphBLAST: A High-Performance Linear Algebra-based GraphFramework on the GPU"). Moreover, Timothy A. Davis also plans o support GPGPU in his implementation of GraphBLAS (Timothy A. Davis, "Algorithm 1000: SuiteSparse:GraphBLAS: Graph Algorithms in the Language of Sparse Linear Algebra", 2019). These projects faced some problems. One of them is that to achieve high performance, specific optimizations for intermediate data structures removing and code-mixing (kernel fusion) are required. But it is very hard to implement such optimizations for C-like languages which are used in these and similar projects. Although such optimizations exist for more high-level and strong languages  (T. Henriksen, N. G. W. Serup, M. Elsman, F. Henglein, and C. E. Oancea, "Futhark: Purely functional gpu-programming with nested parallelism and inplace array updates", 2017) and data structures  (O. Kiselyov, A. Biboudis, N. Palladinos, and Y. Smaragdakis, "Stream fusion, to completeness", 2017), these research shows that proposed in them techniques are limited and cannot be applied for sparse linear algebra optimization. On the other hand, required optimizations are well studied in the context of functional language optimization. Namely, deforestation (P. Wadler, "Deforestation: transforming programs to eliminate trees", 1990) --- the more general technique which is a partial case of supercompilation. A group led by V.F.Turchin worked on automatic program transformation by means of supercompilation. The creation
of tools that are based on these techniques and can be used for real-world problems is an actively developing area. For example, in 2009 I.G. Klyuchnikov and S.A. Romanenko showed that supercompilation can be applied for higher-order functions (I. Klyuchnikov, "Higher-Order Supercompilation", 2010), and after that, they propose multi-result and multilevel supercompilation (I.Klyuchnikov, S.A.Romanenko, "Multi-result supercompilation as branching growth of the penultimate level in metasystem transitions", 2011; I.Klyuchnikov, S.A.Romanenko, "Higher-level supercompilation as a metasystem transition", 2012). But it is too hard to implement a supercompiler for general-purpose programming language. Moreover, a general supercompiler can demonstrate unpredictable behavior in some cases. 

Another actively developed direction is the creation of specialized hardware accelerators for linear algebra operations. First of all, special processors for graph analysis are created. Overview of the current state of this area is presented in "On the design of an efficient hardware accelerator for large scale graph analytics" (Y. Horawalavithana, 2016). Thau, W. S. Song, V. Gleyzer, A. Lomakin, and J. Kepner in "Novel graph processor architecture, prototype system, and results", 2016, propose to create a processor with a linear algebra-based instruction set. But the resulting processor is limited by performing matrix-matrix multiplication, which is not enough to express a wide range of algorithms. Also, specialized accelerators for specific operations are developed. The most promising results are published by  Z. Zhang, H. Wang, S. Han, and W. J.Dally, "Sparch: Efficient architecture for sparse matrix multiplication", 2020; M. Soltaniyeh, R. P.Martin, and S. Nagarakatte, "Synergistic cpu-fpga acceleration of sparse linear algebra", 2020; S.Pal, J. Beaumont, D. Park, A. Amarnath, S. Feng, C. Chakrabarti, H. Kim, D. Blaauw, T. Mudge, and R. Dreslinski, "Outerspace: An outer product based sparse matrix multiplication accelerator", 2018. But all these works are limited to accelerate one specific operation (matrix-matrix or matrix-vector multiplication). Also, in these works software-level optimizations are not involved: operations shuffling cannot be done on the hardware level, but can be done in the compilation step using techniques mentioned above.

Functional programming languages allow one to apply promising software optimizations, thus it is necessary to investigate techniques for hardware acceleration of functional programs. One direction is a specialized general-purpose processors creation for functional languages: M. Naylor and C. Runciman, "The Reduceron: Widening the Von Neumann bottleneck for graph reduction using an fpga", 2008; A. Boeijink, P. K. F. Ḧolzenspies, and J. Kuper, "Introducing the Pilgrim: A processor for executing lazy functional languages", 2011; R. Coelho, F. Tanus, A. Moreira, and G. Nazar, "Acqua: A parallel accelerator architecture for pure functional programs", 2020. These projects are aimed to develop general-purpose processors and are not performant enough currently. Another direction is a compilation of functional programs to a program-specific hardware accelerator. This approach is well studied and is actively developed by Richard Townsen and Stephen A. Edwards: "Compiling Irregular Software to Specialized Hardware", 2019; "From functional programs to pipelined dataflow circuits", 2017. This approach can help to solve problems stated in our research, but there are no results on the composition of this approach with such software optimization techniques as supercompilation.

DataCert project (http://datacert.lri.fr/) is aimed to create a certified stack for relational database querying. As a part of this project, such data querying languages as SQL are formally described in Coq. For example, the semantics of SQL is described: Veronique Benzaken, Evelyne Contejean, "SQLCert: Coq mechanization of SQL’s compilation: Formally reconciling SQL and (relational) algebra", 2017. For graph query languages such results are not presented, but formal semantics of Cypher query language is a relevant research topic: Jozsef Marton, Gabor Szarnyas, and Daniel Varro, "Formalising openCypher Graph Queries in Relational Algebra", 2017; Nadime Francis, et al, "Formal Semantics of the Language Cypher", 2018.  Another work in this direction is a Cypher.pl project which describes Syper in terms of Datalog: Filip Murlak, Jan Posiadala, and Pawel Susicki, "On the semantics of Cypher’s implicit group-by", 2019. Note, that these results do not contain any parts of Cypher description mechanized in Coq or other similar systems. Moreover, there is no formal description of such important extensions of Cypher, as the one that introduces regular and context-free path constraints (Tobias Lindaaker, Path Pattern, https://github.com/thobe/openCypher/blob/rpq/cip/1.accepted/CIP2017-02-06-Path-Patterns.adoc). Note that the importance of this extension was shown by Arseniy Terekhov in "Multiple-Source Context-Free Path Querying in Terms of Linear Algebra", 2021, where this extension was implemented as a part of the RedisGraph database.

Several groups are working on the parallel and incremental algorithms for interprocedural static code analysis. For example, Y. Su, D. Ye, J. Xue and X. Liao, "An Efficient GPU Implementation of Inclusion-Based Pointer Analysis", 2016; "Parallel Pointer Analysis for Large-Scale Software", 2015; R. Gu et al., "Towards Efficient Large-Scale Interprocedural Program Static Analysis on Distributed Data-Parallel Computation", 2021; H. Yu, Q. Sun, K. Xiao, Y. Chen, T. Mine, and J. Zhao, "Parallelizing Flow-Sensitive Demand-Driven Points-to Analysis", 2020; Jisheng Zhao, Michael G. Burke, and Vivek Sarkar, "Parallel sparse flow-sensitive points-to analysis", 2018. These works are based on the idea that a big number of interprocedural code analyses can be reduced to context-free path querying, as was shown by Thomas Reps. The utilization of modern linear algebra-based algorithms for context-free path querying is not investigated. Such algorithms were proposed recently by Rustam Azimov ("Context-free path querying by matrix multiplication", 2018), Arseniy Terekhov ("Context-Free Path Querying with Single-Path Semantics by Matrix Multiplication", 2020), Egor Orachev ("Context-Free Path Querying by Kronecker Product", 2020). But these novel algorithms were not evaluated in the context of static code analysis. 

It is important not only to improve the performance of static code analysis but also to provide mechanisms for interpretation of results of such analysis. One promising way is to involve machine learning-based solutions. For example, Yulei Sui, Xiao Cheng, Guanqin Zhang, and Haoyu Wang, in "Flow2Vec: value-flow-based precise code embedding", 2020, show how to utilize results of context-free path querying-based algorithms to improve the quality of machine learning based applications for static code analysis. This work does not take into account such important questions as the performance of the proposed solution and the utilization of modern algorithms for context-free path querying. 


\subsection{+Предлагаемые методы и подходы, общий план работы на весь срок выполнения проекта и ожидаемые результаты }
%(объемом не менее 2 стр.; в том числе указываются ожидаемые конкретные результаты по годам; общий план дается с разбивкой по годам)

\textbf{ru}\\

При разработке новых подходов и инструментальных средств для создания высокопроизводительных библиотек разреженной линейной алгебры для современного аппаратного обеспечения с использованием существующих языков программирования планируется использовать средства и методы метапрограммирования, в частности техники квазицитирования (code quotation) и генерации кода во время выполнения, что позволит бесшовно для конечного пользователя интегрировать высокоуровневый и низкоуровневый языки. В качестве низкоуровневого языка предполагается использовать OpenCL C, так как на сегодняшний момент это самый зрелый стандарт для разработки переносимого высокопроизводительного кода, выполняемого на различных устройствах, включая многоядерные процессоры, графические ускорители общего назначения, программируемые логические интегральные схемы (ПЛИС, FPGA). В качестве языка высокого уровня предлагается использовать язык F#, так как этот язык является функциональным языком программирования со строгой статической типизацией и развитыми средствами метапрограммирования. Кроме этого, этот язык интегрирован с платформой .NET, одной из самых популярных платформ для разработки прикладных решений. Вместе с этим, предполагается использование использование таких высокоуровневых средств организации асинхронных вычислений, как модель авторов, основанная на передаче сообщений между независимыми долгоживущими асинхронно выполняющимися процессами. Эти и другие абстракции представлены в языке F# и их использование должно позволить построить средство разработки, упрощающее программирование асинхронных гетерогенных систем с несколькими графическими ускорителями.


При разработке нового программно-аппаратного стека, предназначенного для разработки высокопроизводительных решений на основе операций и примитивов разреженной линейной алгебры, планируется использовать методы суперкомпиляции для оптимизации программ. В частности, планируется использование результатов Ильи Ключникова и разработанного им суперкомпилятора HOSC. Ожидается, что так как в нашем случае планируется создание библиотеки с ограниченным набором функций и типов, на ограниченном языке, проблему непредсказуемости результатов суперкомпиляции удастся решить, сузив общие решения до рассматриваемого частного случая. Вместе с этим, для обеспечения аппаратной поддержки планируется использовать подход, основанный на создании специализированной архитектуры по программе на функциональном языке программирования. В частности, планируется применять результаты, полученные Ричардом Таусендом. А именно, планируется на функциональном языке программирования реализовать базовый набор типов данных и операций, предусматриваемых стандартом GraphBLAS, далее реализовать некоторые алгоритмы, используя реализованные операции, после чего оптимизировать их с применением методов суперкомпиляции, что, в частности, должно устранить создание промежуточных структур данных. Полученная программа будет транслироваться в специализированную аппаратную архитектуру, в результате чего будет получаться проблемно-специфичный процессор. Использование такого подхода вместе с ПЛИС, позволит получать специализированную аппаратную поддержку для конкретных задач и алгоритмов анализа данных.

При формальном описании свойств языка Cypher и доказательстве его свойств будет использоваться система автоматической проверки доказательств Coq, что позволит не только строго записать необходимые утверждения, но и гарантировать, что в них нет ошибок. На начальных этапах формализации планируется использовать результаты Надима Франсиса и Джозефа Мартона, содержащие описание семантики языка Cypher, однако, лишь на бумаге. В качестве первого шага планируется проанализировать эти описания и описать семантику в системе Coq. Далее планируется заняться описанием недавно предложенного расширения языка Cypher, позволяющего выражать регулярные и контекстно-свободные ограничения (Tobias Lindaaker, Path Pattern, https://github.com/thobe/openCypher/blob/rpq/cip/1.accepted/CIP2017-02-06-Path-Patterns.adoc). После чего будет изучаться возможность трансляции конструкция языка Cypher (вместе с указанным расширением) в операции линейной алгебры. Задача здесь: описать максимальное подмножество языка, выразимое в терминах линейной алгебры. На этом шаге планируется опираться на результаты Флориса Гертса, касающиеся различных аспектов семантики языка MATLANG. Результаты будут зафиксированы в Coq и снабжены доказательствами корректности. 

При изучении применимости алгоритмов, основанных на операциях линейной алгебры, к задачам статического анализа кода планируется использование результатов, полученных Рустамом Азимовым, Арсением Тереховым и Егором Орачевым. А именно, планируется исследовать предложенные ими алгоритмы и, при необходимости, модифицировать с учётом особенностей решаемых задач. Кроме этого, планируется изучить применимость результатов работы таких алгоритмов для улучшения методов интеллектуального анализа кода с использованием методов машинного обучения. Здесь предполагается отталкиваться от результатов Юлей Суй и улучшить их.


***июль 2021 -- июнь 2022***

Будут проведены разработка и экспериментальное исследование системы программирования графических процессоров на языке высокого уровня (F#). Будут выявлены основные возможности и ограничения прозрачной интеграции вычислений на графическом ускорителе общего назначения в язык высокого уровня со статической типизацией F# на основе методов метапрограммирования времени исполнения. Будет реализован прототип соответствующего инструмента, реализована библиотека базовых операций разреженной линейной алгебры, проведено её экспериментальное исследование, проанализированы результаты использования полученного средства разработки и самой библиотеки. На основе этого анализа будут сформулированы задачи на следующий год. 

Будет выполнена формализация семантики ядра языка запросов Cypher с использованием системы автоматической проверки доказательств Coq. Начнутся подготовительные работы по формализации расширения, позволяющего выражать регулярные и контекстно-свободные ограничения. Начнётся обсуждение полученных результатов с сообществом GQL.

Будет проведено исследование применимости суперкомпиляции для оптимизации программ, написанных с использованием операций и примитивов линейной алгебры. А именно, будет проведён анализ существующих суперкомпиляторов для функциональных языков, выбран поддерживающий достаточно богатый для описания необходимых типов и функций язык. Далее, на выбранном языке будет реализована библиотека основных типов и операций разреженной линейной алгебры, после чего на основе этой библиотеки будут реализованы элементарные программы, эффект от специализации которых и будет изучаться. Планируется добиться устранения создания промежуточных матриц при выполнении последовательности арифметических операций над ними (сложение и умножение матриц), а также при взятии маски. Для этого, вероятно, будут выполнены улучшения выбранного суперкомпилятора.

Будет проведено экспериментальное исследование алгоритмов поиска путей с контекстно-свободными ограничениями в контексте решения задач статического анализа кода на языке программирования Java. Для этого будет создана инфраструктура, включающая компоненты по построению необходимых графов и запросов (грамматик) по исходному коду (на основе инструментов типа Soot или WALA), выполнению запросов на построенных графах с использованием алгоритмов, основанных на линейной алгебре, выполняемых на многоядерных ЦПУ и на графических ускорителях общего назначения, замеру требуемого для выполнения запроса времени и памяти. Результатам исследования будут подвергнуты анализу и сравнению с аналогичными результатами, полученными с использованием других алгоритмов. По итогам анализа будут сформулированы задачи по улучшению алгоритмов, которые будут решаться в следующем году.


***июль 2022 -- июнь 2023***

По результатам предыдущего года будет вестись улучшение суперкомпилятора. Вместе с этим будет расширяться библиотеки примитивов линейной алгебры функциями, предусмотренными стандартом GraphBLAS, что может потребовать дополнительных улучшений в суперкомпиляторе. Будет разработан набор базовых алгоритмов анализа графов на основе разрабатываемой библиотеки, таких как обход в ширину, построение транзитивного замыкания. Будет проведено исследование эффективности суперкомпиляции при оптимизации таких алгоритмов. Будут выявлены и проанализированы случаи, когда полное устранение промежуточных матриц или векторов невозможно.

Начнутся эксперименты по трансляции выбранного функционального языка в специализированное аппаратное обеспечение. Будет проведён анализ влияния суперкомпиляции на производительность целевого решения. На данном этапе планируется сравнивать производительность с использованием симуляторов (типа modelsim). По итогам анализа будут сформулированы задачи по улучшению транслятора и суперкомпилятора, которые будут решаться в следующем году. 

По результатам предыдущего года будет вестись улучшение средства программирования графических ускорителей общего назначения с использованием языка высокого уровня F#. Вместе с этим, будет вестись исследование применимости высокоуровневых абстракций асинхронного и параллельного программирования, предоставляемых языком программирования F#, для создания средств программирования гетерогенных систем и систем с несколькими графическими сопроцессорами общего назначения. Будет вестись разработка и реализация такого средства. Ожидается, что будет получен прототип, демонстрирующий возможности использования статических гарантий и создания обобщённых типов и операций при написании библиотек абстрактной (абстрагируются понятия моноида, полукольца) разреженной линейной алгебры.

Будет вестись формализация семантики расширения языка Cypher, позволяющего выражать регулярные и контекстно-свободные ограничения. Будет получено полное (с учётом указанного расширения) формальное описание не изменяющего граф подмножества языка запросов Cypher. Будут начаты работы по формальному описанию трансляции конструкций данного подмножества в операции линейной алгебры.

По результатам предыдущего года будет вестись улучшение алгоритмов поиска путей с контекстно-свободным ограничениями. Также начнётся изучение применимости результатов соответствующих анализов для улучшения результатов обработки программного кода методами машинного обучения.

***июль 2023 -- июнь 2024***

Буду выполнены запланированные в предыдущем году улучшения суперкомпилятора и транслятора. Будет завершена адаптация инструмента компиляции функционального языка в специализированную аппаратную архитектуру к разработанному языку. Это позволит провести экспериментальное исследование полученного решения с использованием ПЛИС (FPGA). По результатам исследования и сравнения с аналогичными решениями, а так же с современными решениям, использующими многоядерные ЦПУ и графические ускорители, будут сделаны выводы о применимости выбранного подхода для создания программно-аппаратного стека для высокопроизводительных вычислений на основе операций линейной алгебры над разреженными структурами данных.

Будет завершена работа над прототипом средства программирования гетерогенных систем, содержащих графические ускорители общего назначения, с использование языка высокого уровня F#. Будет проведено его экспериментальное исследование. Для этого с его помощью будет реализовано базовое подмножество стандарта GraphBLAS, реализованы некоторые алгоритмы анализа графов, использующие данное подмножество. Далее будет проведен анализ как быстродействия полученного решения по сравнению с решениями, основанными на других реализациях стандарта, так и особенности разработки с использованием разработанного средства. После чего будут сделаны выводы о применимости выбранного подхода к разработке высокопроизводительных абстрактных решений, использующих примитивы и операции линейной алгебры над разреженными структурами, с использованием языков высокого уровня.

Будет завершена работа по формальному описанию трансляции конструкций языка Cypher в операции линейную алгебру. Будет получена классификация конструкций на транслирующиеся в операции линейной алгебры и невыразимые в ней. На основе этого будут сформулированы рекомендации по разработке систем, использующих Cypher в качестве языка запросов, и матрицы и операции над ними для представления и обработки данных.

Будут созданы и проанализированы прикладные решения по статическому анализу кода, основанные на методах машинного обучения, использующих данные статического анализа, основанного на поиске путей с контекстно-свободными ограничениями. Будут сделаны выводы о применимости такого подхода для уточнения интеллектуальных средств анализ программного кода.
\\
\\
\textbf{en}\\

Metaprogramming techniques, including code quotations, and runtime code generation will be used for the development of new approaches to high-performance sparse linear algebra-based libraries and applications development by using high-level programming languages and modern hardware. This way we plan to provide transparent integration of high-level and low-level languages. We plan to use an OpenCL C language as a low-level language for high-performance computing because it is a portable language for such platforms as CPU, GPGPU, FPGA. As a high-level language, we plan to use the F# programming language. This language is functional-first and statically strongly typed, and provides powerful techniques for metaprogramming. Moreover, this language is based on the .NET platform which is one of the most popular platforms for application development. Also, we plan to use an actors-based model (message passing-based model) for asynchronous computations. Respective abstractions are native for F#, thus it allows us to develop a tool that simplifies development for heterogeneous systems with multiple GPGPUs.

In the part, related to new hardware-software platform development, supercompilation will be used for the optimization of programs based on linear algebra primitives.  Namely, we plan to use the results of Ilya Kluchnikov, and the HOSC supercompiler developed by him. We expect, that we can solve the problem of the unpredictable behavior of supercompiler by tuning it for or specific case: library of basic linear algebra operations written in the restricted language. We plan to use functional language to hardware compiler for hardware accelerator creation. Namely, we plan to use the results of Richard Townsend who shows that it is possible to create a program-specific accelerator for programs written in a functional language. The set of basic types and operations described in GraphBLAS will be implemented in Haskell-like functional language. After that, a number of linear algebra-based algorithms will be implemented, and supercompilation will be used to optimize them. We hope, that intermediate data structures (matrices and vectors) will be removed as a result of supercompilation. The resulting optimized program will be translated to hardware. This way we plan to provide FPGA-based accelerators for data analysis algorithms.

Coq proof assistant will be used for formal description of the Cypher query language. As a first step, we plan to analyze the papers of Nadime Francis and József Marton which contain a description of the Cypher semantics, and mechanize the Cypher semantics in Coq. The second step is an investigation and formalization in Coq of Cypher language extension, which allows for specifying regular and context-free path constraints ((Tobias Lindaaker, Path Pattern, https://github.com/thobe/openCypher/blob/rpq/cip/1.accepted/CIP2017-02-06-Path-Patterns.adoc). After that, we plan to detect which instructions of the extended version of Cypher can be translated to linear algebra operations. The goal is to find the maximal subset of Cypher language which can be described in terms of linear algebra. We plan to use the results of Floris Geerts on MATLANG language at this step. All results will be mechanized in Coq. 
Results of Rustam Azimov, Arseniy Terekhov, and Egor Orachev will be used for the analysis of the applicability of linear algebra-based algorithms to static code analysis. Respective algorithms will be evaluated and modified. Also, we plan to find a way to use the results of these algorithms for improving machine learning based program analysis solutions. We plan to improve the results of Yulei Sui.  

***June 2021 -- July 2022***

A tool for GPGPU programming using F# programming language will be developed. Strengths and limitations of the proposed solution will be investigated. The library of basic sparse linear algebra operation will be implemented using the proposed solution. This library will be evaluated. Results of this evaluation and the experience of tool utilization will be analyzed in order to formulate tasks for the next year.  

The Cypher semantics will be described using the Coq proof assistant. Preliminary investigation of Cypher extension which allows one to specify the context-free and regular constraints will be started. Discussion of the results with GQL community will be started.

The applicability of supercompilation for linear algebra-based algorithms will be investigated. The supercompiler for such functional language that it is expressive enough to describe necessary data structures and operations will be chosen. The library of basic structures and operations will be implemented using the selected language. The effect of specialization of programs written using the created library will be investigated. We expect, that intermediate matrix creation in chained matrix operations, such as element-wise addition or multiplication, will be removed using supercompilation.

The linear algebra-based context-free path querying algorithms will be evaluated in the context of static code analysis of Java programs. Tools for graphs and grammars extraction will be developed using Soot or WALA tools. An integrated environment for algorithms evaluation, time and memory consumption measurement, will be developed. Parallel linear algebra-based algorithms will be compared with other parallel solutions. Results analysis will be done in order to find a way to improve algorithms. These improvements will be done in the following year.

***June 2022 -- July 2023***

The supercompiler will be improved. Also, the respective library will be extended with new functions from GraphBLAS. The supercompiler will be improved to support new functions. A number of graph analysis algorithms will be implemented using the library. The applicability of supercompilation for such algorithms optimization will be investigated. Moreover, cases, when it is not possible to fully remove intermediate data structures will be described.

Works on the compilation of the functional language to FPGA will be started. The optimization effect of supercompilation in terms of generated hardware performance will be investigated. At this step, we plan to use emulation (modelsim or similar tools) instead of FPGA. Tasks on the improvement of the translator and the supercompiler will be formulated.

Improvements of GPGPU programming tool will be finished. The applicability of the native for F# high-level abstractions for asynchronous and parallel programming for heterogeneous multi-GPU systems programming will be investigated. The development of the respective tools will be started. We believe that a prototype that demonstrates the strengths of static code checks and generic types for the development of the library for abstract (in terms of monoids and semirings) sparse linear algebra will be developed.

The formal semantic of the extended Cypher language will be described in Coq. Formalization of the translation of the subset of Cypher to linear algebra will be started.

Context-free path querying algorithms will be improved. The investigation of the applicability of these algorithms' results for improvement of machine learning-based static code analysis will be started. 

***June 2023 -- July 2024***

The supercompiler and the translator will be improved and finally integrated. As a result, the evaluation of the proposed solution for hardware accelerators creation for linear algebra-based algorithms will be done using FPGA. The proposed solution will be compared with other solutions including CPU and GPU-based solutions for linear algebra-based data analysis. The results of evaluation and comparison will be analyzed. 

The development of the prototype of a tool for heterogeneous multi-GPGPU systems programming using F# will be finished. The prototype will be evaluated by implementing a subset of GraphBLAS standard and a number of GraphBLAS inspired algorithms. The performance of the developed algorithms will be analyzed. Finally, we plan to estimate the applicability of the proposed approach to develop high-performance linear algebra-based solutions using high-level languages.

The mechanization of Cypher queries to linear algebra operation translation will be finished. The instructions of the Cypher language will be classified with respect to its mapping to linear algebra operations. As a result, recommendations for the development of high-performance query processing systems will be formulated.

The applied solutions for machine learning-based static code analysis which utilizes context-free path querying algorithms will be developed and evaluated. 

\subsection{+Имеющийся у научного коллектива научный задел по проекту, наличие опыта совместной реализации проектов}

\textbf{ru}\\

Руководитель проекта и многие его участники обладают опытом в разработке и исследовании алгоритмов, основанных на линейной алгебре, в том числе, алгоритмов поиска путей с контекстно-свободными ограничениями, являющегося одним из ключевых при межпроцедурном анализе кода, что подтверждается перечисленными ниже и некоторыми другими работами.
\begin{itemize}
  \item Azimov, Grigorev, "Context-free path querying by matrix multiplication", GRADES-NDA 2018.
  \item Orachev, Epelbaum, Azimov, Grigorev, "Context-Free Path Querying by Kronecker Product", ADBIS 2020;
  \item Terekhov, Khoroshev, Azimov, Grigorev, "Context-Free Path Querying with Single-Path Semantics by Matrix Multiplication", GRADES-NDA 2020.
\end{itemize}

Руководитель принимал успешное участие в совместной с А.В. Подкопаевым и Д.А. Березуном работе над проектам в рамках гранта РФФИ 18-01-00380. Также, С.В.Григорьев являлся исполнителем грантов РФФИ 15-01-05431 и Фонда содействия развитию малых форм предприятий в технической сфере (программа УМНИК, проекты N 162ГУ1/2013 и N 5609ГУ1/2014), руководителем гранта РФФИ 19-37-90101, а также является руководителем научной группы, в соавторстве с участниками которой опубликованы указанные выше и некоторые другие работы. Кроме этого, С.В. Григорьев являлся основным исполнителем гранта РНФ 18-11-00100.  

Р.Ш.Азимовым и С.В.Григорьевым предложен алгоритм поиска путей с контекстно-свободными ограничениями на основе матричных операций, доказана его корректность, получена оценка временной сложности (Rustam Azimov and Semyon Grigorev. 2018. Context-free path querying by matrix multiplication). Кроме того, предложено обобщение данного алгоритма, в котором в качестве ограничений над путями используются конъюнктивные грамматики, позволяющие выражать более сложные запросы к графам. Для обобщенного алгоритма также доказана корректность и получена оценка временной сложности. Также, совместно с другими участниками проекта, А.К. Тереховым и Е.С. Орачевым, предложены и другие обобщения этого алгоритма, в частности позволяющие найти один и все пути, удовлетворяющие ограничениям, или выполняющие поиск из заранее заданного множества вершин. Применимость именно этих алгоритмов к статическому анализу кода и предполагается исследовать, а в дальнейшем и модифицировать с учетом специфики возникающих в данной области задач.

Е.С.Орачевым разработана библиотека операций разреженной булевой линейной алгебры на CUDA C. Данная библиотека, с одной стороны, станет основой для экспериментального исследования применимости алгоритмов поиска путей для задач статического анализа кода. С другой, она станет "мерилом" для других разрабатываемых решений.

А.К.Терехов, совместно с С.В.Григорьевым реализовал полнофункциональную поддержку запросов с регулярными и контекстно-свободными ограничениями для графовой базы данных (Terekhov, Grigorev, et al, "Multiple-Source Context-Free Path Querying in Terms of Linear Algebra", принята на EDBT-2021). Для этого потребовалось реализовать расширение языка запросов Cypher, позволяющее выражать соответствующий тип ограничений и транслировать его в операции линейной алгебры. в терминах которых выражается используемый алгоритм поиска путей. Анализ выявленных в рамках данного проекта проблем послужил мотивацией для формализации семантики языка Cypher и построения его формального транслятора в операции линейной алгебры, а полученный опыт будет использован для решения поставленных уже в этом проекте задач. 

Руководитель проекта принимал разработку системы Brahma.FSharp на основе языка программирования F#, позволяющую прозрачным для разработчика образом использовать в программах на F# код, написанный на языке OpenCL C, с сохранением гарантий типовой безопасности (Smirenko, Grigorev, "F# OpenCL Type Provider", TyDe-2018). Этот проект будет взят за основу при разработки средств программирования графических процессоров с использованием языков высокого уровня. 

Д.А. Березун имеет опыт исследований в области семантики языков программирования, метавычислений и программной специализации. В частности, им предложен алгоритм компиляции нетипизированного лямбда исчисления в низкоуровневое представление посредством игровой семантики программ и частичных вычислений (D.Berezun, N.D.Jones, "Compiling untyped lambda calculus to lower-level code by game semantics and partial evaluation", 2017; D.Berezun, N.D.Jones, "Working Notes: Compiling ULC to Lower-level Code by Game Semantics and Partial Evaluation", 2016). Кроме того, им была показана корректность предложенного алгоритма и его обобщения, а также предложено обобщение понятия головной линейной редукции термов (Д.Березун, "Полная головная линейная редукция", 2017).

Кроме этого, Д.А.Березуном, совместно с А.В.Тюриным и С.В.Григоревым получены результаты, показывающие применимость смешанных вычислений для оптимизации программ, выполняемых на графических процессорах общего назначения (Aleksey Tyurin, Daniil Berezun, and Semyon Grigorev, "Optimizing GPU programs by partial evaluation" PPoPP-2020). Результаты данного исследования послужат отправной точкой при исследовании применимости суперкомпиляции для оптимизации программ, построенных на основе операций линейной алгебры, так как они позволяют оценить эффект специализации, как частного случая суперкомпиляции, на выполнение программ на параллельных архитектурах.

Т.А.Брыксин имеет опыт исследований в области применения методов машинного обучения в программной инженерии. Работа его исследовательской группы затрагивает самые разные аспекты этой области: построение векторных пространств фрагментов программного кода (T.Bryksin, V.Petukhov, I.Alexin, S.Prikhodko, A.Shpilman, V.Kovalenko, N.Povarov, "Using Large-Scale Anomaly Detection on Code to Improve Kotlin Compiler", MSR, 2020), целых проектов (E.Bogomolov, Y.Golubev, A.Lobanov, V.Kovalenko, T.Bryksin, "Sosed: a tool for finding similar software projects", ASE, 2020) или даже стиля написания кода программистами (V.Kovalenko, E.Bogomolov, T.Bryksin, A.Bacchelli, "Building Implicit Vector Representations of Individual Coding Style", CHASE, 2020). В настоящее время в группе Т.А.Брыксина ведутся исследования по созданию эффективных инструментов применения моделей машинного обучения, а также по применению полученных результатов для решения практических задач в программной инженерии: рекомендации рефакторингов, автодополнения кода, вывода типов в динамически типизированных языках, обнаружения устаревших комментариев в коде, автоматической генерации комментариев и сообщений к коммитам в системах контроля версий и т.п.

А.В.Подкопаев имеет опыт исследований в области слабых моделей памяти. Так, им были проведены несколько
доказательств корректности компиляции между языковыми и процессорными моделями памяти
(A. Podkopaev, O. Lahav, V. Vafeiadis, "Promising Compilation to ARMv8 POP", 2017;
A. Podkopaev, O. Lahav, V. Vafeiadis, "Bridging the Gap Between Programming Languages and Hardware Weak Memory Models", 2019;
E. Moiseenko, A. Podkopaev, O. Lahav, O. Melkonian, V. Vafeiadis, "Reconciling Event Structures with Modern Multiprocessors", 2020)
и принято участие как в разработке новых моделей памяти для языков программирования
(S.-H. Lee, M. Cho, A. Podkopaev, S. Chakraborty, C.-K. Hur, O. Lahav, V. Vafeiadis, "Promising 2.0: Global Optimizations in Relaxed Memory Concurrency", 2020),
так и в исправлении существующих моделей широко распространенных языков
(C. Watt, C. Pulte, A. Podkopaev, G. Barbier, S. Dolan, S. Flur, J. Pichon-Pharabod, S. Guo,
"Repairing and Mechanising the JavaScript Relaxed Memory Model", 2020).
В рамках большинства упомянутых публикаций А.В. Подкопаев участвовал в механизации доказательств свойств моделей памяти
в системе интерактивного доказательства теорем Coq.

Кроме этого, участники проекта создали набор данных, необходимый для экспериментального исследования разрабатываемых решений. Он представлен и используется в работе "Evaluation of the Context-Free Path Querying Algorithm Based on Matrix Multiplication". В ходе исследований планируется его использование и расширение при экспериментальных исследованиях различных алгоритмов.


\subsection{+Перечень оборудования, материалов, информационных и других ресурсов, имеющихся у научного коллектива для выполнения проекта}
\textbf{ru}\\
%
У коллектива имеется необходимое аппаратное и программное обеспечения для разработки и проведения экспериментального исследования алгоритмов, использующих многоядерные процессоры, графические ускорители общего назначения, а также для проведения иных запланированных задач.

\subsection{+План работы на первый год выполнения проекта}

\textbf{ru}\\

Планируется работа над заранее намеченными на этот год исследовательскими задачами, предоставление результатов на конференциях и подготовка результатов к печати. Также будет проведено осмысление полученных результатов с возможной формулировкой новых задач. Распределение задач между основными исполнителями проекта приведено в следующем разделе.

Также на первый год планируется 3 поездки с докладами на международные конференции (в среднем по 100000 рублей).
\\
\\
\textbf{en}\\
During the first year, it is planned to work on research questions listed in this plan, to present results at conferences, and prepare results for publication. Also it is planned to collaborate for understanding the new results. As a result, some new problems will be formulated. Detailed plan for each team member is presented below.

Also, 3 trips to international conferences are planned (on average, 100000 rub. each) in order to give talks.



\subsection{+Планируемое на первый год содержание работы каждого основного исполнителя проекта (включая руководителя проекта)}

\textbf{ru}\\

С.В.Григорьев, совместно с А.К.Тереховым займётся разработкой технологии высокоуровневого программирования массово-параллельных систем, позволяющей программировать графические процессоры общего назначения с использованием языка высокого уровня (F#) прозрачным для разработчика образом. Будут изучены возможности и ограничения трансляции подмножества языка F# в язык OpenCL C. Также будут изучаться основные принципы использования подобного инструмента с целью поиска баланса между высокоуровневыми абстракциями и необходимостью детального контроля для достижения высокой производительности. Также будет проведено экспериментальное исследования полученной реализации на примере разработки библиотеки базовых операций линейной алгебры: умножения разреженных матриц, поэлементного сложения разреженных матриц.

Т.А.Брыксин, Р.Ш. Азимов и Е.С. Орачев займутся экспериментальным исследованием алгоритмов поиска путей с контекстно-свободными ограничениями, основанных на операциях линейной алгебры, применительно к статическому анализу кода, в том числе статистическими методами, в частности методами машинного обучения. Будет проведено экспериментальное исследование алгоритмов поиска путей с контекстно-свободными ограничениями для решения задач межпроцедурного потоко- и контекстно-чувствительного анализа указателей для языка программирования Java. Также будет проведена оценка пригодности результатов данных анализов для дальнейшей статистической обработки и применимость для решения прикладных задач анализа кода.

Формализацией семантики языка запросов Cypher с использованием системы автоматической проверки доказательств Coq займётся А.В.Подкопаев совместно с С.В.Григорьевым. На первом этап планируется формализации ядра языка Cypher,а также описание его списание семантики в терминах реляционной алгебры в системе Coq.

А.В.Тюрин и Д.А.Березун будут заниматься вопросами применимости суперкомпиляции для оптимизации алгоритмов, реализованных на основе операций линейной алгебры. В частности, будет проведён анализ существующих решений в области суперкомпиляции с целью выбрать наиболее подходящий для исследуемых задач язык и суперкомпилятор для него. Ожидается, что будет найден модельный функциональный язык, поддерживающий необходимые конструкции и суперкомпилятор для него. Далее будет выполнена реализация базовых типов данных, таких как разреженная матрица в виде дерева квадрантов или списка координат ненулевых элементов, и основных операций (поэлементное сложение матриц, умножение матриц, применение маски) на выбранном подмножестве функционального языка.
После чего будет проведено экспериментальное исследование суперкомпилятора на простых программах, использующих реализованные типы данных и операции. Например, последовательное сложение нескольких матриц или применение маски к результату некоторой операции. Будет проведён анализ того, какие структуры данных и принципы написания операций (кода) позволяют достичь наилучшего результата в смысле производительности итогового решения. 

К обсуждению всех задач, работе над ними, и написанию статей будут привлекаться включенные в состав научного коллектива студенты, магистры и аспиранты.

\subsection{+Ожидаемые в конце первого года конкретные научные результаты}
%(форма изложения должна дать возможность провести экспертизу результатов и оценить степень выполнения заявленного в проекте плана работы)

\textbf{ru}\\

Будет разработан подход на основе метапрограммирования для программирования графических процессоров с использованием языков высокого уровня. Будет реализован прототип соответствующего инструментального средства и проведено его экспериментальное исследование. По итогам, одна работа будет представлена на конференции. Результаты будут опубликованы в сборнике докладов, индексируемом в Scopus.

Будет проведено экспериментальное исследование применимости суперкомпиляции для оптимизации алгоритмов, выраженных в терминах операций линейной алгебры. Полученные результаты будут проанализированы, на основе результатов анализа будут сформулированы конкретные задачи, решение которых планируется на второй год работы.

Будет формально описана семантика ядра языка запросов Cypher (не изменяющее граф подмножество) с использованием инструмента Coq. Результаты будут представлены сообществу GQL, представлены на профильной конференции и опубликованы в сборнике материалов индексируемом в Scopus. 

Будет проведено экспериментальное исследование применимости алгоритмов поиска путей с контекстно-свободными ограничениями, основанными на операциях линейной алгебры, в рамках задач статического анализа кода. Будут проанализированы полученные результаты и сформулированы направления и конкретные задачи по улучшению рассмотренных алгоритмов. По итогам, одна работа будет представлена на конференции. Результаты будут опубликованы в сборнике докладов, индексируемом в Scopus.

Над прочими заявленными темами будет вестись работа, однако результаты будут опубликованы во второй год проекта.
\\
\\
\textbf{en}\\

Metaprogramming-based approach to utilize GPGPUs using high-level programming languages will be developed. The prototype of the respective tool will be created and evaluated. Results will be presented at a conference and published in proceedings indexed in Scopus.

Applicability of supercompilation for linear algebra-based algorithms optimization will be investigated. Results will be analyzed in order to formulate detailed plan for the next year. 

Semantic of the reading-only subset of the Cypher graph query language will be mechanized in the Coq proof assistant. The results will be presented to the GQL community and discussed. Also, the results will be presented at a conference and published in proceedings indexed in Scopus.

Linear algebra-based context-free path querying algorithms will be evaluated on the static code analysis tasks. The results will be analyzed in order to formulate detailed plan for the next year. Also, the results will be presented at a conference and published in proceedings indexed in Scopus.

The work on other topics will be in progress, but the results will be published during the second year of the project.

\subsection{+Перечень планируемых к приобретению руководителем проекта за счет гранта Фонда оборудования, материалов, информационных и других ресурсов для выполнения проекта}
%(в том числе – описывается необходимость их использования для реализации проекта)

\textbf{ru}\\
%
Не более 400 тыс. рублей ежегодно будет тратиться на поездки с докладами на конференции. Расходов на оборудование не предполагается.


\end{document}