\documentclass[12pt,a4paper]{cibb}

\usepackage{subfigure,graphicx}
\usepackage{amsmath,amsfonts,latexsym,amssymb,euscript,xr}
\usepackage{enumitem}


% Package to generate and customize Algorithm as per ACM style
\usepackage[ruled, linesnumbered, noend]{algorithm2e}
\renewcommand{\algorithmcfname}{Listing}
\SetAlFnt{\small}
\SetAlCapFnt{\small}
\SetAlCapNameFnt{\small}
\SetAlCapHSkip{0pt}
\IncMargin{-\parindent}


\title{\large $\ $\\ \bf MODIFICATION OF VALIANT'S PARSING ALGORITHM FOR STRING-SEARCHING PROBLEM}

\author{Yuliya Susanina$^{(1),(2)}$, Anna Yaveyn$^{(1)}$, Semyon Grigorev$^{(1),(2)}$}
\address{$\ $\\(1) Saint Petersburg State University, 7/9 Universitetskaya nab., St. Petersburg, 199034 Russia\\
jsusanina@gmail.com,
anya.yaveyn@yandex.ru
\\
%
\bigskip
(2) JetBrains Research, Universitetskaya emb., 7-9-11/5A, St.Petersburg, Russia\\
semen.grigorev@jetbrains.com
}


\abstract{Context-free grammar, string-matching, Valiant's algorithm, secondary structure, parsing.
\\[17pt]
{\bf Abstract.} 
Some string-matching problems can be reduced to parsing: verification whether some subsequence can be derived in the given grammar. 
To apply parser-based solutions to such area as bioinformatics, one needs to improve parsing techniques so that the processing of a large amount of data was possible.
The most asymptotically efficient parsing algorithm that can be applied to any context-free grammar is a matrix-based algorithm proposed by Valiant.
This paper presents a modification of the Valiant’s algorithm, which allows one to simplify highly parallel implementation, and efficiently utilize modern massively-parallel hardware. 
Moreover, the modified version decreases a large amount of excessive computations and accelerates the substrings searching. }

\begin{document}
\thispagestyle{myheadings}
\pagestyle{myheadings}
\markright{\tt Proceedings of CIBB 2019}%check year

\section{Introduction}

Scalable high-performance graph analysis is an actual challenge.
There is a big number of ways to attack this challenge~\cite{Coimbra2021} and the first promising idea is to utilize general-purpose graphic processing units (GPGPU).
Such existing solutions, as CuSha~\cite{10.1145/2600212.2600227} and Gunrock~\cite{7967137} show that utilization of GPUs can improve the performance of graph analysis, moreover it is shown that solutions may be scaled to multi-GPU systems.
But low flexibility and high complexity of API are problems of these solutions.

The second promising thing which provides a user-friendly API for high-performance graph analysis algorithms creation is a GraphBLAS API~\cite{7761646} which provides linear algebra based building blocks to create graph analysis algorithms.
The idea of GraphBLAS is based on a well-known fact that linear algebra operations can be efficiently implemented on parallel hardware.
Along with that, a graph can be natively represented using matrices: adjacency matrix, incidence matrix, etc.
While reference CPU-based implementation of GraphBLAS, SuiteSparse:GraphBLAS~\cite{10.1145/3322125}, demonstrates good performance in real-world tasks, GPU-based implementation is challenging.

One of the challenges in this way is that real data are often sparse, thus underlying matrices and vectors are also sparse, and, as a result, classical dense data structures and respective algorithms are inefficient. 
So, it is necessary to use advanced data structures and procedures to implement sparse linear algebra, but the efficient implementation of them on GPU is hard due to the irregularity of workload and data access patterns.
Though such well-known libraries as cuSPARSE show that sparse linear algebra operations can be efficiently implemented for GPGPU, it is not so trivial to implement GraphBLAS on GPGPU. 
First of all, it requires \textit{generic} sparse linear algebra, thus it is impossible just to reuse existing libraries which are almost all specified for operations over floats.
The second problem is specific optimizations, such as masking fusion, which can not be natively implemented on top of existing kernels.
Nevertheless, there is a number of implementations of GraphBLAS on GPGPU, such as GraphBLAST~\cite{yang2019graphblast}, GBTL~\cite{7529957}, which show that GPGPUs utilization can improve the performance of GraphBLAS-based graph analysis solutions.
But these solutions are not portable because they are based on Nvidia Cuda stack.
Moreover, the scalability problem is not solved: all these solutions support only single-GPU, not multi-GPU computations.

To provide portable GPU implementation of GraphBLAS API we developed a \textit{SPLA} library\footnote{Source code available at: \url{https://github.com/JetBrains-Research/spla}}.
This library utilizes OpenCL for GPGPU computing to be portable across devices of different vendors.
Moreover, it is initially designed to utilize multiple GPGPUs to be scalable.
To sum up, the contribution of this work is the following.
\begin{itemize}
    \item Design of portable GPU GraphBLAS implementation proposed. The design involves the utilization of multiple GPUS. Additionally, the proposed design is aimed to simplify library tuning and wrappers for different high-level platforms and languages creation. 
    \item Subset of GraphBLAS API, including such operations as masking, matrix-matrix multiplication, matrix-matrix e-wise addition, is implemented. The current implementation is limited by COO and CSR matrix representation format and uses basic algorithms for some operations, but work in progress and more data formats will be supported and advanced algorithms will be implemented in the future.
    \item Preliminary evaluation on such algorithms as breadth-first search (BFS) and triangles counting (TC), and real-world graphs shows portability across different vendors and promising performance: for some problems Spla is comparable with GraphBLAST. Surprisingly, for some problems, the proposed solution on embedded Intel graphic card shows better performance than SuiteSparse:GraphBLAS on the respective CPU. At the same time, the evaluation shows that further optimization is required.
\end{itemize} 
\section{\bf Formal languges}

In this section we introduce basic definitions from formal language theory, and describe Valiant's parsing algorithm which we use as a base for our solution.

An alphabet $\Sigma$ is a finite nonempty set of symbols.
$\Sigma^{*}$ is a set of all finite strings over $\Sigma$.
A contex-free grammar $G_S$ is a quadruple $(\Sigma, N, R, S)$, where $\Sigma$ is a finite set of terminals, $N$ is a finite set of nonterminals, $\Sigma \cup N = \varnothing$, $R$ is a finite set of productions of the form $A \rightarrow \beta$, where $A \in N, \beta \in V^{*}$, $V = \Sigma \cup N$ and $S \in N$ is a start symbol.
Context-free grammar $G_S = (\Sigma, N, R, S)$ is said to be in Chomsky normal form if all productions in $R$ are of the form: $A \rightarrow BC$, $A \rightarrow a$, $S \rightarrow \varepsilon$, where $A, B, C \in N, a \in \Sigma, \varepsilon$ is an empty string.
$L_{G}(S) = \{ \omega | S\xrightarrow[G_S]{*} \omega\}$ is a language specified by the grammar $G_{S} = (\Sigma, N, R, S)$, where $A \xrightarrow[G_S]{*} \omega$ means that $\omega$ can be derived in a finite number of rules applications from the start symbol $S$.

\subsection{\bf \it Valiant's parsing algorithm}

Tabular parsing algorithms construct a matrix $T$ cells of which are filled with nonterminals from which the corresponding substring can be derived. 
These algorithms are usually work with the grammar in Chomsky normal form.
Namely, $T_{i, j} =  \{ A | A \in N, a_{i + 1} \dots a_{j} \in L_{G}(A)\} \quad \forall i < j$, where $G_S=(\Sigma, N, R, S)$.

The elements of $T$ are filled successively beginning with $T_{i - 1, i} = \{ A | A \rightarrow a_{i} \in R\}.$
Then, $T_{i, j} = f(P_{i, j}),$ where
$P_{i, j} = \bigcup\limits_{k = i + 1}^{j - 1} T_{i,k} \times T_{k, j}$ and
$f(P) = \{A | \exists A \rightarrow BC \in R : (B, C) \in P\}.$
Finally, the input string $a_{1}a_{2} \dots a_{n}$ belongs to $L_{G}(S)$ iff $S \in T_{0, n}$.

If all elements are filled sequentially, the time complexity of this algorithm is $O(n^3)$.
Valiant proposed to offload the most intensive computations to the Boolean matrix multiplication. 
As the most time-consuming is computing $\bigcup\limits_{k = i + 1}^{j - 1} T_{i, k} \times T_{k, j}$, Valiant anged the computation of $T_{i, j}$, in order to use multiplication of submatrices of $T$.
Multiplication of two submatrices of parsing table $T$ is defined as follows.
Let $X \in (2^N)^{m \times l}$ and $Y \in (2^N)^{l \times n}$ be two submatrices of parsing table $T$. 
Then, $X \times Y = Z$, where $Z \in (2^{N \times N})^{m \times n}$ and $Z_{i, j} = \bigcup\limits_{k = 1}^{l} X_{i, k} \times Y_{k, j}$.

Note that the computation of $X \times Y$  can be replaced by the multiplication of $|N|^2$ Boolean matrices (for each nonterminal pair).
Denote the matrix corresponding to the pair $(B, C) \in N \times N$ as $Z^{(B, C)}$, then $Z_{i, j}^{(B, C)} = 1$ iff $(B, C) \in Z_{i, j}$.
It should also be noted that $Z^{(B, C)} = X^{B} \times Y^{C}$.
Each Boolean matrix multiplication can be computed independently.
Following these changes, time complexity of this algorithm is $O(|G|BMM(n)log(n))$ for an input string of length $n$, where $BMM(n)$ is the number of operations needed to multiply two Boolean matrices of size $n \times n$.

Valiant's algorithm written as proposed by Okhotin is presented in listing~\ref{algo:valiant}.
All elements of $T$ and $P$ are initialized by empty sets.
Then, the elements of these two table are successively filled by two recursive procedures.

% Algorithm1
\begin{algorithm}[h]
\SetAlgoNoLine
\KwIn{Grammar $G = (\Sigma, N, R, S), w = a_{1} \dots a_{n}, n \geq 1, a_{i} \in \Sigma$, where  $n + 1 = 2^k$}
\underline{main()}{:}{

 \textit{compute(0, n + 1)\;}
 accept if and only if $S \in T_{0, n}$
 \linebreak
 }

\underline{compute(\textit{l, m})}{:}{

 \If {$m - l \geq 4$}{
     \textit{compute(l, $\frac{l+m}{2}$)\;
     compute($\frac{l+m}{2}$, m)}}
 \textit{complete(l, $\frac{l+m}{2}$, $\frac{l+m}{2}$, m)}
 \linebreak
 }

\underline{complete(\textit{l, m}, $l^\prime$, $m^\prime$)}{:}{

 \lIf {$m - l = 4$ and $m = l^\prime$}{$T_{l, l + 1} = \{A | A \rightarrow a_{l+ 1} \in R\}$}
 \lElseIf{$m - l = 1$ and $m < l^\prime$}{ $T_{l, l'} = f(P_{l, l'})$}
 \ElseIf{$m - l > 1$}{
    $leftgrounded = (l, \frac{l+m}{2}, \frac{l+m}{2}, m), rightgrounded = (l', \frac{l'+m'}{2}, \frac{l'+m'}{2}, m')$,

    $bottom = (\frac{l+m}{2}, m, l', \frac{l'+m'}{2}), left = (l, \frac{l+m}{2}, l', \frac{l'+m'}{2})$,

    $right = (\frac{l+m}{2}, m, \frac{l'+m'}{2}, m'), top = (l, \frac{l+m}{2}, \frac{l'+m'}{2}, m')$\;
    complete(bottom)\;
    $P_{left} = P_{left} \cup (T_{leftgrounded} \times T_{bottom})$\;
    complete(left)\;
    $P_{right} = P_{right} \cup (T_{bottom} \times T_{rightgrounded})$\;
    complete(right)\;
    $P_{top} = P_{top} \cup (T_{leftgrounded} \times T_{right})$\;
    $P_{top} = P_{top} \cup (T_{left} \times T_{rightgrounded})$\;
    complete(top)
    }
 }
\caption{Parsing by matrix multiplication: Valiant's Version}
\label{algo:valiant}
\end{algorithm}

The procedure $compute(l, m)$ computes correct values of $T_{i,j}$ for all $l \le i < j < m$.

The procedure $complete(l, m, l', m')$ constructs the submatrix $T_{i, j}$ for all $l \le i < m$, $l' \le j < m'$. This procedure assumes $T_{i, j}$ for all $l \leq i < j < m,  l' \leq i < j < m'$ are already constructed and the current value of  $P[i, j] =  \{ (B, C) |\exists k, (m \le k < l'), a_{i + 1} \dots a_{k} \in L(B), a_{k + 1} \dots a_{j} \in L(C)\}$ for all $l \leq i < m,  l' \leq j < m'$. The submatrix division during the procedure call is shown in figure~\ref{fig2}.


\begin{figure}
\vspace{3mm}
 \begin{center}
    \begin{minipage}{0.48\textwidth}
        \centering
        \includegraphics[width=6cm]{pictures/splitting_with_grounded.pdf}
        \caption{Matrix partition used in \textit{complete(l, m, l', m')} procedure.}
        \label{fig1}
    \end{minipage}\hfill
    \begin{minipage}{0.48\textwidth}
        \centering
        \includegraphics[width=6cm]{pictures/layers.pdf}
        \caption{Matrix partition on V-shaped layers used in modification.}
        \label{fig2}
    \end{minipage}
 \end{center}
\vspace{-8mm}
\end{figure}

A simple example of parsing with the Valiant's algorithm is presented in figure~\ref{fig3}.
Only several steps are shown, but it is enough to point out at this version and our approach differences.

\begin{figure}
\vspace{3mm}
 \begin{center}
 \includegraphics[width=12cm]{pictures/valbeg2.pdf}
    \caption{An example of beginning of Valiant's algorithm}
    \label{fig3}
\end{center}
\vspace{-8mm}
\end{figure}
\section{\bf Modified Valiant's algorithm}

In this section we describe the reorganization of submatrices processing order in the Valiant's algorithm which simplify independent handling of submatrices. As a result, proposed modification can facilitate implementation of parallel submatrix processing.

\subsection{\bf \it Layered submatrices processing}

The main change of this modification is the possibility to divide the parsing table into layers of disjoint submatrices of the same size.
The idea of division we have made from the reorganization of the matrix multiplication order is presented in figure~\ref{fig2}.
Each layer consists of square matrices which size is power of 2.
The layers are computed successively in the bottom-up order.
Each matrix in the layer can be handled independently, which can help to implement parallel version of layer processing function.

\begin{figure}[h]
\vspace{3mm}
 \begin{center}
 \includegraphics[width=12cm]{pictures/modivis2.pdf}
    \caption{An example of the modification of Valiant's algorithm}
    \label{fig4}
 \end{center}
\vspace{-8mm}
\end{figure}

A simple example of the modification is shown in figure~\ref{fig4}.
The lowest layer (submatrices which size is 1) is already computed and filling of the matrix starts with the second layer (subfigures 1-2).
Note that the same process is presented in figure~\ref{fig3}, but here it can be done only in two steps using parallel computation of submatrix products.

The modified version of Valiant's algorithm is presented in listing~\ref{algo:modified}.
The procedure \textit{main()} computes the lowest layer $(T_{l, l+1})$, and then divide the table into layers, described earlier, and computes them through the \textit{completeVLayer()} call.
Thus, \textit{main()} computes all elements of parsing table $T$.
(Hereinafter, we use layer to mean set of submatrices.)

For brevity, we define \textit{left(subm), right(subm), top(subm), bottom(subm), \linebreak rightgrounded(subm)} and \textit{leftgrounded(subm)} functions which returns the submatrices for matrix $subm = (l, m, l', m')$ according to the original Valiant's algorithm (figure~\ref{fig2}).

Also denote some subsidiary functions for matrix layer $M$:
\begin{itemize}[noitemsep, nolistsep]
    \item[$-$] \textit{bottomsublayer(M)} $ = \{bottom(subm)\, |\,subm \in M \}$,
    \item[$-$] \textit{leftsublayer(M)} $ = \{\textit{left(subm)}\, |\,subm \in M \}$,
    \item[$-$] \textit{rightsublayer(M)} $ =\{\textit{right(subm)}\, |\,subm \in M \}$,
    \item[$-$] \textit{topsublayer(M)} $ = \{top(subm)\, |\,subm \in M \}$.
\end{itemize}

\begin{algorithm}[!h]
\SetAlgoNoLine
\KwIn{$G = (\Sigma, N, R, S), w = a_{1} \dots a_{n}, n \geq 1, n + 1 = 2^p, a_{i} \in \Sigma$ }
\underline{main()}{:}{

 \For {$l \in \{1, \ldots, n \}$}{$T_{l, l + 1} = \{A | A \rightarrow a_{l + 1} \in R\}$}
 \For{$1 \le i < p - 1 $}{
 layer = \textit{constructLayer(i)}\;
 \textit{completeVLayer(layer)}
 }
 accept if and only if $S \in T_{0, n}$
 \BlankLine
 }

\underline{constructLayer(i)}{:}{
 \BlankLine
 $\{(k2^i, (k+1)2^i, (k + 1)2^i, (k+2)2^i) \, |\, 0 \le k < 2^{p - i} - 1\}$
 \BlankLine
    }
\underline{completeLayer(M)}{:}{
\BlankLine
\If {$\forall (l, m, l', m') \in M \quad (m - l = 1)$}{\For{$ (l, m, l', m') \in M$}{$T_{l, l'} = f(P_{l, l'})$\;}}
\Else{
\textit{completeLayer(bottomsublayer(M))}\;
\textit{completeVLayer(M)}
}
\BlankLine
}

\underline{comleteVLayer(M)}{:}{
 \BlankLine
 \textit{multiplicationTasks$_1$ = \linebreak
    \{$left(subm)$, $leftgrounded(subm)$, $bottom(subm)\, |\,subm \in M \} \cup \linebreak  \{right(subm), bottom(subm), rightgrounded(subm)\, |\,subm \in M\}$\;}
 \BlankLine
 multiplicationTask$_2$ = $\{top(subm), leftgrounded(subm), right(subm)\, |\,subm \in M\}$\;
 \BlankLine
 multiplicationTask$_3$ = $\{top(subm), left(subm), rightgrounded\, |\,subm \in M\}$\;
 \BlankLine
 \textit{performMultiplications(multiplicationTask$_1$)}\;
 \textit{completeLayer(leftsublayer(M) $\cup$ rightsublayer(M))}\;
 \textit{performMultiplications(multiplicationTask$_2$)}\;
 \textit{performMultiplications(multiplicationTask$_3$)}\;
 \textit{completeLayer(topsublayer(M))}

 }
 \BlankLine

 \underline{performMultiplication(tasks)}{:}{\\
 \For{$ (m, m1, m2) \in \textit{tasks}$}{$P_{m} = P_{m} \cup (T_{m1} \times T_{m2})$\;}
 }

\caption{Parsing by matrix multiplication: Modified Version}
\label{algo:modified}
\end{algorithm}


The procedure \textit{completeVLayer(M)} takes an array of disjoint submatrices $M$ which represents a layer.
For each \textit{subm = (l, m, l', m') $\in M$} this procedure computes \textit{left(subm), right(subm), top(subm)}.
The procedure assumes that the elements of \textit{bottom(subm)} and $T_{i, j}$ for all $i$ and $j$ such that $l \leq i < j < m$ and $  l' \leq i < j < m'$ are already constructed.
Also it is assumed that the current value of
$P_{i, j} =  \{ (B, C) | \exists k, (m \le k < l'), a_{i + 1} \dots a_{k} \in L_G(B), a_{k + 1} \dots a_{j} \in L_G(C)\} $ for all $i$ and $j$ such that $l \leq i < m$ and $l' \leq j < m'$.

The procedure \textit{completeLayer(M)} also takes an array of disjoint submatrices $M$, but unlike the previous one, it computes $T_{i, j}$ for all $(i, j) \in subm$.
This procedure requires exactly same assumptions on $T_{i, j}$  and $P_{i, j}$  as in the previous case.

In the other words, \textit{completeVLayer(M)} computes the entire layer \textit{M} \linebreak and \textit{completeLayer($M_{2}$)} is a support function which is necessary for computation of smaller square submatrices $subm_{2} \in M_{2}$ inside of \textit{M}.

Finally, the procedure \textit{performMultiplication(tasks)}, where \textit{tasks} is an array of a triple of submatrices, perform basic step of algorithm: matrix multiplication. It is worth mentioning that, as distinct from the original algorithm, here $|tasks| \ge 1$ and each task can be computed independently.
So, practical implementation of this procedure can easily involve different techniques of parallel array processing, such as OpenMP.

\subsection{\bf \it Algorithm for substrings}

Next we show how our modification can be applied to the string-matching problem.

So if we want to find all substrings of size $s$ which can be derived from a start symbol for an input string of size $n = 2^p$, we need to compute layers with submatrices of size not greater than $2^{l'}$, where $2^{l' - 2} < s \le 2^{l' - 1}$.

Let $l' = p - (m - 2)$ and consequently $(m - 2) = p - l'$.

For any  $m \le i \le p$ products of submatrices of size $2^{p - i}$ are calculated exactly $2^{2i - 1} - 2^{i}$ times and each of them imply multiplying $\mathcal{O}(|G|)$ Boolean submatrices.

\begin{equation}
\begin{array}{c}
C \sum\limits_{i=m}^p 2^{2i - 1} \cdot 2^{\omega(p - i)} \cdot f(2^{p - i}) =
C \cdot 2^{\omega l'}\sum\limits_{i=2}^{l'} 2^{(2 - \omega)i} \cdot 2^{2(p - l') - 1} \cdot f(2^{l' - i}) \le \\
C \cdot 2^{\omega l'} f(2^{l'}) \cdot 2^{2(p - l') - 1} \sum\limits_{i=2}^{l'} 2^{(2 - \omega)i} =
BMM(2^{l'}) \cdot 2^{2(p - l') - 1} \sum\limits_{i=2}^{l'} 2^{(2 - \omega)i}
\end{array}
\end{equation}

Thus, time complexity for searching all substrings is  $O(|G|BMM(2^{l'})(l' - 1))$, while time complexity for the full input string is $O(|G|BMM(2^p)(p - 1))$. In contract to the modification, Valiant's algorithm completely calculate at least 2 triangle submatrices of size $\frac{n}{2}$ (as shown in figure~\ref{fig5}) which mean minimum asymptotic complexity  $O(|G|BMM(2^{p - 1})(p - 2))$. Thus we can conclude that the modification is asymptotically faster for substrings of size $s \ll n$  than the original algorithm.

\begin{figure}[h]
\vspace{3mm}
 \begin{center}
 \includegraphics[width=12cm]{pictures/valsubstring.pdf}
    \caption{The number of elements necessary to compute in Valiant's algorithm. That means it is nessesary to calculate at least 2 triangle submatrices of size $\frac{n}{2}$.}
    \label{fig5}
 \end{center}
\vspace{-8mm}
\end{figure}

\section{Conclusion and Future Work}

We present !!!

Our evaluation shows that !!!

First direction for future research is a more detailed CFPQ algorithms investigation.
We should do More evaluation on sparse matrices on GPGPUs.

Also it is nesessary to implement and evaluate solutions for graphs which is not fit in RAM.
There is a set of technics for huge matrices multiplication.
Is it possible to dopt it for CFPQ

Another direcion is a dataset improvement.
More data.
More grammars/queries.



\section*{\bf Acknowledgments}

The research was supported by the Russian Science Foundation grant 18-11-00100 and a grant from JetBrains Research.


%\bibliographystyle{apalike}
\bibliographystyle{ieeetr}

{\fontsize{10}{10}\selectfont
%\begin{thebibliography}{99}
\setlength{\parskip}{0pt}

\bibliography{main}

%\end{thebibliography}
}

\end{document}
