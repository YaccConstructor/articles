\section{\bf Introduction}

Recent  research has  shown  that the theory of formal languages and, in particular, context-free languages can be used in bioinformatics~\cite{rivas,knudsen,yuan,dowell}. ....

A good example of this usage is the recognition and classification problems in bioinformatics, some of them are based on the research claiming that the secondary structure of the DNA and RNA nucleotide sequence contains important information about the organism species. The specific features of the secondary structure can be described by some context-free grammar, and therefore the recognition problem can be reduced to parsing---verification if some nucleotide sequence can be derived in this grammar.
%That means we try to find the substrings in DNA or RNA sequences possessing these specific features and further we can draw conclusions about the organism's origin based on the availability and location of the found substrings.

Such field of application as bioinformatics requires working with a large amount of data, so it is necessary to find highly efficient parsing algorithm. 

The majority of parsing algorithms either has the cubic-time complexity (Kasami~\cite{Kas}, Younger~\cite{Younger:1966:CLP:1441427.1442019}, Earley~\cite{Earley:1970:ECP:362007.362035}) or could work only with sub-classes of context-free grammars (Bernardy, Claussen~\cite{Bernardy:2013:EDP:2544174.2500576}), but still asymptotically more efficient parsing algorithm that can be applied to any context-free grammar is algorithm based on matrix multiplication proposed by Leslie Valiant~\cite{Valiant:1975:GCR:1739932.1740048}. Moreover, Okhotin generalized this algorithm to conjunctive and Boolean grammars which are the natural extensions of context-free grammars with more expressive power~\cite{Okhotin:2014:PMM:2565359.2565379}. 

In this paper, we propose the modification of Valiant's algorithm which allows to compute some matrix products concurrently and show the applicability of our approach in bioinformatics research, especially in addressing the string-matching problem.