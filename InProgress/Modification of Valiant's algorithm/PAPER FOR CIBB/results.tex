\section{\bf Modified Valiant's algorithm}

In this section, we propose how to rearrange the order in which submatricess are precessed in the algorithm.
The different order impreves independdence of submatrices handling and facilitates the implementation of parallel submatrix processing.

\subsection{\bf \it Layered submatrices processing}

We propose to divide the parsing table into layers of disjoint submatrices of the same size (see figure~\ref{fig2}).
Such division is possible because derivation of substring of fixed length is not depended on left an right contexts.
Appropriate order of substrings processing allows us to guarantyy disjointness of submatrices which form layer.
Each layer consists of square matrices which size is a power of 2.
The layers are computed successively in the bottom-up order.
Each matrix in the layer can be handled independently, which facilitates parallelization of layer processing.

\begin{figure}
\vspace{3mm}
 \begin{center}
 \includegraphics[width=12cm]{pictures/modivis2.pdf}
    \caption{An example of the modification of Valiant's algorithm}
    \label{fig4}
 \end{center}
\vspace{-8mm}
\end{figure}

Figure~\ref{fig4} demonstartes the modified algorithm.
The lowest layer (submatrices of size 1) has already been computed.
The second layer is filled in in steps 1-2.
Although the original algorithm computes the same matrix, the modified one needs two steps using parallel computation of submatrix products.

The modified version of Valiant's algorithm is presented in listing~\ref{algo:modified}.
The procedure \textit{main()} computes the lowest layer $(T_{l, l+1})$, and then divides the table into layers, and computes them with the \textit{completeVLayer()} function.
Thus, \textit{main()} computes all elements of parsing table $T$.

We define \textit{left(subm)}, \textit{right(subm)}, \textit{top(subm)}, \textit{bottom(subm)}, \textit{rightgrounded(subm)} and \textit{leftgrounded(subm)} functions which returns the submatrices for matrix $subm = (l, m, l', m')$ according to the original Valiant's algorithm (figure~\ref{fig2}).


\begin{algorithm}[!h]
\SetAlgoNoLine
\KwIn{$G = (\Sigma, N, R, S), w = a_{1} \dots a_{n}, n \geq 1, n + 1 = 2^p, a_{i} \in \Sigma$ }
\underline{main()}{:}{

 \lFor {$l \in \{1, \ldots, n \}$}{$T_{l, l + 1} = \{A | A \rightarrow a_{l + 1} \in R\}$}
 \For{$1 \le i < p - 1 $}{
 layer = \textit{constructLayer(i)}\;
 \textit{completeVLayer(layer)}
 }
 accept if and only if $S \in T_{0, n}$
 \BlankLine
 }

\underline{constructLayer(i)}{:}{
 \BlankLine
 $\{(k2^i, (k+1)2^i, (k + 1)2^i, (k+2)2^i) \, |\, 0 \le k < 2^{p - i} - 1\}$
 \BlankLine
    }
\underline{completeLayer(M)}{:}{
\BlankLine
\If {$\forall (l, m, l', m') \in M \quad (m - l = 1)$}{\lFor{$ (l, m, l', m') \in M$}{$T_{l, l'} = f(P_{l, l'})$}}
\Else{
\textit{completeLayer($\{\textit{bottom(subm)}\, |\,subm \in M \})$}\;
\textit{completeVLayer(M)}
}
\BlankLine
}

\underline{completeVLayer(M)}{:}{
 \BlankLine
 \textit{multiplicationTasks$_1$ = \linebreak
    \{$left(subm)$, $leftgrounded(subm)$, $bottom(subm)\, |\,subm \in M \} \cup \linebreak  \{right(subm), bottom(subm), rightgrounded(subm)\, |\,subm \in M\}$\;}
 \BlankLine
 multiplicationTask$_2$ = $\{top(subm), leftgrounded(subm), right(subm)\, |\,subm \in M\}$\;
 \BlankLine
 multiplicationTask$_3$ = $\{top(subm), left(subm), rightgrounded\, |\,subm \in M\}$\;
 \BlankLine
 \textit{performMultiplications(multiplicationTask$_1$)}\;
 \textit{completeLayer($\{\textit{left(subm)}\, |\,subm \in M \} \cup \{\textit{right(subm)}\, |\,subm \in M \}$)}\;
 \textit{performMultiplications(multiplicationTask$_2$)}\;
 \textit{performMultiplications(multiplicationTask$_3$)}\;
 \textit{completeLayer($\{top(subm)\, |\,subm \in M \}$)}

 }
 \BlankLine

 \underline{performMultiplication(tasks)}{:}{\\
 \lFor{$ (m, m1, m2) \in \textit{tasks}$}{$P_{m} = P_{m} \cup (T_{m1} \times T_{m2})$}
 }

\caption{Parsing by matrix multiplication: Modified Version}
\label{algo:modified}
\end{algorithm}


The procedure \textit{completeVLayer(M)} takes an array of disjoint submatrices $M$ which represents a layer.
For each \textit{subm = (l, m, l', m') $\in M$} this procedure computes \textit{left(subm), right(subm), top(subm)}.
The procedure assumes that the elements of \textit{bottom(subm)} and $T_{i, j}$ for all $i$ and $j$ such that $l \leq i < j < m$ and $  l' \leq i < j < m'$ are already constructed.
Also it is assumed that the current value of
$P_{i, j} =  \{ (B, C) | \exists k, (m \le k < l'), a_{i + 1} \dots a_{k} \in L_G(B), a_{k + 1} \dots a_{j} \in L_G(C)\} $ for all $i$ and $j$ such that $l \leq i < m$ and $l' \leq j < m'$.

The procedure \textit{completeLayer(M)} also takes an array of disjoint submatrices $M$, but unlike the previous one, it computes $T_{i, j}$ for all $(i, j) \in subm$.
This procedure requires exactly the same assumptions on $T_{i, j}$  and $P_{i, j}$  as in the previous case.

In other words, \textit{completeVLayer(M)} computes the entire layer \textit{M} \linebreak and \textit{completeLayer($M_{2}$)} is a support function which is necessary for computation of smaller square submatrices $subm_{2} \in M_{2}$ inside of \textit{M}.

Finally, the procedure \textit{performMultiplication(tasks)}, where \textit{tasks} is an array of triples of submatrices, performs the basic step of algorithm: matrix multiplication.
It is worth mentioning that $|tasks| \ge 1$ and each task can be computed independently, while the original algorithm handles one \textit{task} per step sequentially.
So, the practical implementation of this procedure can easily utilizes different techniques of parallel array processing, such as OpenMP.

\subsection{\bf \it Algorithm for substrings}

Next, we show how our modification can be applied to the string-matching problem.

To find all substrings of size $s$, which can be derived from a start symbol for an input string of size $n = 2^p$, we need to compute layers with submatrices of size not greater than $2^{l'}$, where $2^{l' - 2} < s \le 2^{l' - 1}$.

Let $l' = p - (m - 2)$ and consequently $(m - 2) = p - l'$.
For any  $m \le i \le p$ products of submatrices of size $2^{p - i}$ are calculated exactly $2^{2i - 1} - 2^{i}$ times and each of them imply multiplying $\mathcal{O}(|G|)$ Boolean submatrices.

\begin{equation*}
\begin{array}{c}
C \sum\limits_{i=m}^p 2^{2i - 1} \cdot 2^{\omega(p - i)} \cdot f(2^{p - i}) =
C \cdot 2^{\omega l'}\sum\limits_{i=2}^{l'} 2^{(2 - \omega)i} \cdot 2^{2(p - l') - 1} \cdot f(2^{l' - i}) \le \\
C \cdot 2^{\omega l'} f(2^{l'}) \cdot 2^{2(p - l') - 1} \sum\limits_{i=2}^{l'} 2^{(2 - \omega)i} =
BMM(2^{l'}) \cdot 2^{2(p - l') - 1} \sum\limits_{i=2}^{l'} 2^{(2 - \omega)i}
\end{array}
\end{equation*}

Thus, time complexity for searching all substrings is  $O(|G|BMM(2^{l'})(l' - 1))$, while time complexity for the full input string is $O(|G|BMM(2^p)(p - 1))$. 
In contract to the modification, Valiant's algorithm completely calculate at least 2 triangle submatrices of size $\frac{n}{2}$ (as shown in figure~\ref{fig5}) which means the minimum asymptotic complexity is $O(|G|BMM(2^{p - 1})(p - 2))$. Thus we can conclude that the modification is asymptotically faster for substrings of size $s \ll n$  than the original algorithm.

\begin{figure}
\vspace{3mm}
 \begin{center}
 \includegraphics[width=12cm]{pictures/valsubstring.pdf}
    \caption{The number of elements necessary to compute in Valiant's algorithm. That means it is nessesary to calculate at least 2 triangle submatrices of size $\frac{n}{2}$.}
    \label{fig5}
 \end{center}
\vspace{-8mm}
\end{figure}
