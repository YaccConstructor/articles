% This is samplepaper.tex, a sample chapter demonstrating the
% LLNCS macro package for Springer Computer Science proceedings;
% Version 2.20 of 2017/10/04
%
\documentclass[runningheads]{llncs}
%
\usepackage{graphicx}
\usepackage{multirow} 
\usepackage{fancyvrb}

% Used for displaying a sample figure. If possible, figure files should
% be included in EPS format.
%
% If you use the hyperref package, please uncomment the following line
% to display URLs in blue roman font according to Springer's eBook style:
% \renewcommand\UrlFont{\color{blue}\rmfamily}

\begin{document}
%
\title{ON SECONDARY STRUCTURE ANALYSIS BY USING FORMAL GRAMMARS AND ARTIFICIAL NEURAL NETWORKS\thanks{Supported by the Russian Science Foundation grant 18-11-00100}}
%
\titlerunning{Formal Grammars \& Neural Networks}
% If the paper title is too long for the running head, you can set
% an abbreviated paper title here
%
\author{Polina Lunina\inst{1,2}\orcidID{0000-0002-7172-2647} \and
Semyon Grigorev\inst{1,2}\orcidID{0000-0002-7966-0698}}
%
\authorrunning{Polina Lunina, Semyon Grigorev}
% First names are abbreviated in the running head.
% If there are more than two authors, 'et al.' is used.
%
\institute{Saint Petersburg State University, 7/9 Universitetskaya nab., St. Petersburg, 199034, Russia \and
JetBrains Research, Primorskiy prospekt 68-70, Building 1, St. Petersburg 197374, Russia\\
\email{lunina\_polina@mail.ru, semyon.grigorev@jetbrains.com, s.v.grigoriev@spbu.ru}\\
}
%
\maketitle              % typeset the header of the contribution
%
\begin{abstract}
A way to combine formal grammars and artificial neural networks for biological sequences processing was recently proposed.
In this approach, an ordinary grammar encodes primitive features of the RNA secondary structure, parsing is utilized for features extraction and artificial neural network~--- for processing of the extracted features.
Parsing is a bottleneck of the solution: input sequences should first be parsed before processing with a trained model which is a time-consuming operation when working with huge biological databases.
In this work, we solve this problem by employing staged learning and limiting parsing to be used only during network training.
We also compare networks which represent the parsing result in two different ways: by a vector and a bitmap image.
Finally, we evaluate our solution on tRNA classification tasks.

\keywords{DNN \and CNN \and Machine Learning \and Secondary Structure \and Genomic Sequences \and Formal Grammars \and Parsing.}
\end{abstract}
%
%
%

\section{Introduction}

Context-Free Path Querying (CFPQ) is an actively developed area in graph database analysis.
CFPQ is also used for static code analysis~\cite{Reps,10.1145/193173.195287,Zheng}, RDF querying~\cite{10.1007/978-3-319-46523-4_38,MEDEIROS201975}, biological data analysis~\cite{cfpqBio}.

Most of research is focused on developping algorithms for CFPQ evaluation~\cite{hellingsRelational,ward2008distributed,cfpqBio,MEDEIROS201975,Azimov:2018:CPQ:3210259.3210264,Grigorev:2017:CPQ:3166094.3166104}, whereas specification languages for context-free queries are not investigated enough.
Best to our knowledge, only one extension for Sparql supports context-free constarints: cfSPARQL~\cite{10.1007/978-3-319-46523-4_38}.
There is also a proposal for CFPQ as a part of Cypher\footnote{Proposal with path pattern syntax for openCypher: \url{https://github.com/thobe/openCypher/blob/rpq/cip/1.accepted/CIP2017-02-06-Path-Patterns.adoc}.
It is shown that context-free constraints can be expressed with the proposed syntax. Access date: 30.03.2020} language, but there is no implementation for it yet.
We believe that more research should be conducted on the specification languages fo context-free constraints in graph querying.

It is worth noting that graph analysis is often only a part of a more complex system, usually implemented in a general-purpose language.
Since a graph query language is unsuitable to implement a whole system, there should be means of integration of them into general-purpose programming languages.
There are many ways to integrate them ranging from creating graph queries from string values of a general-purpose language to implementing a special embedded domain specific language, and even more sophisticated.

Although simple, the string manipulating approach does not provide a developper with any safety guarantees.
There is no way to ensure that a string generated by an application is a valid query or, in case it is not, to provide any feedback.
This makes string manipulating technique error prone, the code --- unclear and hard to maintain.

Safety of an embedded DSL entirely depends on its implementation.
Some general-purpose languages with powerfull type systems (such as \haskell{}, \ocaml{} or \scala{}) or the ones supporting hygienic macros (such as \scheme{} or \rust{}) facilitate creating safe and reliable DSLs.
Still, they typically lack full support of a development environment: it may be harder to debug queries or issues with composability may arise.

There is a general trend towards imposing more restricting type systems on programming languages.
Among many others are typing annotations for \python{} and \typescript{} code and nullability checks in \kotlin{}.
Typing graphs and query languages improves  readability and simplifies maintainance~\cite{10.1145/2076623.2076653}.

Parser combinators are the answer to the integration of parsing into a general-purpose programming language.
Recursive descend parsers are encoded as functions of the host language, while grammar constructions such as sequencing and choice are implemented as higher-order functions.
This idea was first introduced in~\cite{burge} and further developped in numerous works.
Notable development is monadic parser combinators~\cite{hutton1996monadic}.
In this approach, one can not only parse the input, but simultaneously run semantics calculation if parsing succeeds.
Paper~\cite{izmaylova2016practical} proposed the first monadic parser combinator library which solves the long-standing problem of inability to handle ambiguous and left-recursive grammars.
A library for graph querying was developped~\cite{10.1145/3241653.3241655} based on this work.
The core idea is to use generalized parser combinators as both a way to formulate a query and to execute it.
This approach inherits benefits of combinatory parsing: ease of code reuse, type safety guaranteed by the host language and, since the parser is simply a function, the integrated development support.

Besides integration, it can compute both the single-source and all pairs semantics, as well as execute user actions.
The~single-source semantics is relevant to many real-world applications, including manual data analysis.
It also may be less time-intensive, since on average it needs to expore only a subgraph of the input graph.
Many querying algorithms are only capable to compute all pairs reachability which makes them unsuitable for some applications.

In this paper we make the following contributions.
\begin{itemize}
  \item We demonstrate how to use combinatory-based graph querying on example.
  \item We illustrate such features of the approach as type-safety, flexibility (composability and generics), IDE support and computing user-defined actions.
  \item We evaluate single-source context-free path querying on some real-world RDFs.
  \begin{itemize}
    \item Based on our evaluation, the most common case in RDF context-free querying is when the number of paths in the answer set is big, but they are small.
    \item We demonstrate that the single-source CFPQ can feasibly be used to evaluate such queries.
    \item We conclude that there is a need for a further detailed analysis of both theoretical time and space complexity of single-source CFPQ.
  \end{itemize}
\end{itemize}

\section{Ordinary Context-Free Grammars and Artificial Neural Networks for Secondary Structure Analysis}

In this section we provide a brief description of the approach for the genomic sequences secondary structure analysis which is proposed in~\cite{grigorevcomposition}.

The secondary structure of RNA sequences can be viewed as a composition of stems~\cite{MQbioinformatics19}. 
So one can create ordinary context-free grammar which describes a set of such compositions and use it to extract actual features of the given sequence. 
Grammar $G_0$ is an example of such grammar.
This grammar is used in~\cite{grigorevcomposition} as well as in the present work is presented in figure~\ref{gram}.
This grammar considers only the conventional base pairs (line \textbf{5}) and describes the recursive composition of stems which are at least three base pairs in height (lines \textbf{7-12}).
Stems may be connected by an arbitrary sequence of length from 2 up to 10, and loops also have length from 2 up to 10 (line \textbf{2}).
These parameters were tuned manually as a result of the number of experiments and can be changed to provide a better grammar for some specific goal.

\begin{figure}
\begin{Verbatim}[numbers=left,xleftmargin=5mm]
s1: stem<s0>
any_str : any_smb*[2..10]
s0: any_str | any_str stem<s0> s0
any_smb: A | U | C | G
stem1<s>: A s U | G s C | U s A | C s G 
stem2<s>: stem1< stem1<s> >
stem<s>:  
      A stem<s> U 
    | U stem<s> A 
    | C stem<s> G 
    | G stem<s> C 
    | stem1< stem2<s> >  
\end{Verbatim}
\caption{Context-free grammar $G_0$ for RNA secondary structure features description}
\label{gram}
\end{figure}

The result of a parsing algorithm for an input string $w$ and a fixed grammar non-terminal $N$ (start nonterminal) is an upper-triangular boolean matrix $M_N$, where $M_N [i,j] = 1$, iff the substring $w[i,j-1]$ is derivable from $N$.
This means that, for the grammar $G_0$, a matrix contains $1$ in a cell iff a correspondent substring folds to a  stem of height at least 3.
Such stem results in a diagonal chain of one-s in the matrix.
Figure~\ref{fig:example} presents the parsing result for sequence

\[
w_1 = CCCCATTGCCAAGGACCCCACCTTGGCAATCCC
\]

w.r.t the grammar $G_0$.
Colored boxes map a substring which folds to a stem to correspondent cells in the matrix. 
Besides, this matrix contains other non-zero cells, because parser detects all possible foldings for all possible substrings. 
It can be either noise or some important information about the secondary structure. 
One of the tasks that neural network should perform is to process such matrices and filter all the insignificant contacts between the nucleotides.

\begin{figure}[h]
\begin{center}
\centering
\includegraphics[width=0.8\textwidth]{figures/4.pdf}
\caption{Parsing result for sequence which should fold to
stem}
\label{fig:example}
\end{center}
\end{figure}

The parsing result in a form of a matrix can be linearized, compressed into a byte or integer vector, and be further handled by a dense neural network, as described in~\cite{grigorevcomposition}.
Unfortunately, linearization breaks data locality: a diagonal chain of one-s, which signifies a high stem, is local in a matrix, but is broken apart during its linearization.
We see it to be an argument to investigate the applicability of convolutional networks for parsing result handling, as a boolean matrix can be converted to a black-white bitmap image.
In this paper, we provide an empirical comparison of networks which handle vectors and images.

Another problem is a bad performance of the earlier solution.
Since the trained network handles parsing result, each input sequence should first be parsed.
Parsing is a very time-consuming step: context-free parsing has cubic complexity in terms of the input length.
Even if we use matrix-based parsing algorithm~\cite{Azimov:2018:CPQ:3210259.3210264} which utilizes GPGPU, performance is insufficient.
We believe it would be better to avoid the parsing step at the final stage of the solution.
In this work, we propose a way to solve this problem by building a network which handles raw sequences, not parsing results.

\section{Convolutional Neural Network Utilization}

First, we describe how to use a convolutional network for parsing result processing. Parsing result is a boolean matrix that contains the information about sequence secondary structure features and we utilize the artificial neural network to detect sufficient features and find patterns in their appearance.
Therefore, we need to transform these boolean matrices to some data structure acceptable by the neural network.
Currently, we came up with two possible ways: vectorization and conversion to a black and white image.

The first way is to drop out the nullary bottom left triangle, vectorize the top right triangle row by row and transform it into a byte vector.
This approach reduces the input size, but it requires all the input sequences to have equal length.
Thus we propose to either cut sequences to be of some predefined length or to pad them up with some blank symbols.
Vectorization breaks data locality which makes learning harder: the network should restore back the relations broken during linearization.
This also means that learning takes more time.

The second way is to represent the matrix as an image: the false bits of the matrix as white pixels and the true bits as black ones.
This approach makes it possible to process sequences of different lengths since the images are easily transformed to a specified size.
Data locality is also preserved: the information about relative positions of extracted basic features does not get lost which should improve learning.

The architecture of the neural network that takes vectorized data as an input is described in~\cite{grigorevcomposition} and it consists of the long sequence of interchangeable dense and dropout layers with aggressive batch normalization. 
To handle images, we propose to use a network which consists of a small number of convolutional layers, linearization, and dense network which has a similar architecture as for vectorized data.
An example of the proposed architecture is provided in figure~\ref{nn} (network \textbf{\texttt{N1}}).

\section{Parsing Step Elimination}

Another improvement that we came up with concerns parsing elimination in the context of our solution. 
The idea is to create a model which can handle original sequences instead of the parsing matrices. For that, we propose to use two-staged learning: first, a network which solves a subtask is trained and then it is used as pretrained layers in the training of the resulting network.
In our solution, we first train a neural network to handle parsing results which performs classification according to a problem at hands. We create two networks in order to compare different architectures: one of them handles vectorized parsing result, the other handles parsing result represented as a bitmap image. After that, we extend these neural networks by several input layers that take the initial nucleotide sequence as an input and convert it to the parsing result which is handled appropriately by the pretrained layers.

Figure~\ref{nn} represents the detailed description of these three neural networks architectures.
Here \textbf{\texttt{N1}} is a network which handles images, \textbf{\texttt{N2}} is a network which handles vectorized parsing results, and \textbf{\texttt{N0}} is an additional block which converts the input sequence into a set of features which can be handled by using \textbf{\texttt{N1}} or \textbf{\texttt{N2}}. So, firstly we train \textbf{\texttt{N1}} and \textbf{\texttt{N2}} on parsed data. After that, for vector-based network we combine the extension \textbf{\texttt{N0}} and the whole original sequence of layers and for image-based network we use the similar architecture, except we remove the convolutional layer from the extended model, thus, the first layer at the junction of the blocks corresponds to the linearized image.

\begin{figure}[h]
\begin{center}
\centering
\includegraphics[width=12cm]{figures/nn_arch.pdf}
\caption{Neural networks architectures}
\label{nn}
\end{center}
\end{figure}

To sum up, we developed a technique to process parsing matrices as images by convolutional neural networks. 
Also, we built a model that handles sequences and requires parsing only for training the network it is based on. 
This removes the parsing step from the usage of the trained model.

\section{Evaluation}

The goal of this evaluation is to assess the performance scaling of Spla on Vortex.
Due to limitations in atomic operation support within the RTL implementation, all experiments were performed using the SimX functional simulator.

\subsection{Environment}

Initial testing revealed issues with floating-point operations, which produced incorrect results for some hardware configurations.
Consequently, we limited subsequent experiments to Breadth-First Search (BFS) and Triangle Counting (TC), excluding Single-Source Shortest Path (SSSP) and PageRank.
To keep simulation times manageable, we used a single graph from the SuiteSparse matrix collection\footnote{A diverse collection of sparse matrices from various domains: \url{http://sparse.tamu.edu/}}: soc-Epinions1, with 75~888 vertices and 508~837 edges.


We conducted two series of experiments.
The first varies the number of warps and threads per warp while keeping the number of clusters and cores fixed (at 2 and 4, respectively), with the goal of selecting the best core configuration while preserving multi-core execution to account for cache effects.
The second series, using the best configuration identified in the first step, varies the number of clusters and cores per cluster to assess scaling at the core and cluster levels.
Cache sizes were set to their default values: 16 KB for $L_1$, 1 MB for $L_2$, and 2 MB for $L_3$.

We use the number of cycles reported by SimX as a performance metric.
For multi-core configurations, we report the maximum cycle count across all cores.
During the experiments, we encountered unexpected behavior in SimX that led to out-of-memory exceptions. 
Therefore, some data points are missing from the graphs below.

\subsection{Results}

In figures~\ref{fig:tc_threads_warps} and~\ref{fig:bfs_threads_warps}

\begin{figure}
    \begin{center}
        \includegraphics[width=0.49\textwidth]{pictures/TC_threads_warps.pdf}
    \end{center}
    \caption{Scaling analysis of triangle counting for varying numbers of warps and threads per warp}
    \label{fig:tc_threads_warps}
\end{figure}

\begin{figure}
    \begin{center}
        \includegraphics[width=0.49\textwidth]{pictures/BFS_threads_warps.pdf}
    \end{center}
    \caption{Scaling analysis of BFS for varying numbers of warps and threads per warp}
    \label{fig:bfs_threads_warps}
\end{figure}

Best configuration for BFS is 2 warps, 8 threads per warp (16 threads total). 
Best configuration for TC is 4 warps, 16 threads per warp (64 threads total).


\begin{figure}
    \begin{center}
        \includegraphics[width=0.49\textwidth]{pictures/BFS_cores_clusters.pdf}
    \end{center}
    \caption{Scaling analysis of BFS for varying numbers of clusters and cores per cluster}
    \label{fig:bfs_cores_clusters}
\end{figure}


Edges per core on cycle. Compare with Spla on other GPUs.

\subsection{Scaling limitations analysis}

%sum(scoreboard stalls * lsu_percent) / sum(instr) * 100
To analyze the reasons for limited scaling as the number of threads increases, we measured the average utilization of the ALU and LSU, in terms of stall cycles, for the best BFS configuration.
The results are presented in Fig.~\ref{fig:bfs_alu_stalls} and Fig.~\ref{fig:bfs_lsu_stalls}, respectively.
The data indicate that the LSU is the performance bottleneck within the core.

The same bottleneck was observed in the scaling analysis across clusters and cores.
Whether increasing cache sizes can alleviate this problem remains a question for future research.
We anticipate that careful cache size tuning may help identify a more efficient configuration.

\begin{figure}
    \begin{center}
        \includegraphics[width=0.49\textwidth]{pictures/BFS_alu.pdf}
    \end{center}
    \caption{ALU stalls on BFS for the best configuration}
    \label{fig:bfs_alu_stalls}
\end{figure}

\begin{figure}
    \begin{center}
        \includegraphics[width=0.49\textwidth]{pictures/BFS_lsu.pdf}
    \end{center}
    \caption{LSU stalls on BFS for the best configuration}
    \label{fig:bfs_lsu_stalls}
\end{figure}
\section{Conclusion and Future Work}

Platform presented.

Education. Metaprogramming, translators development, GPGPU programming, etc.

Graph parsing.

Geterogenious porgramming generalization. Hopac is better then MBP~\footnote{\url{https://vasily-kirichenko.github.io/fsharpblog/actors}}.

Research: Automatic memory management.

Data to code translation (automata can be translated into code instead of data structures in memory)

Other technical improvements: IDE support, type provider improvements, new OpenCL standard support, runtime extension, etc.

%
% ---- Bibliography ----
%
% BibTeX users should specify bibliography style 'splncs04'.
% References will then be sorted and formatted in the correct style.
%
\bibliographystyle{splncs04}
\bibliography{main}

\end{document}
