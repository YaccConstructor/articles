\documentclass{beamer}
\usepackage{beamerthemesplit}
\usepackage{wrapfig}
\usetheme{SPbGU}
\usepackage{pdfpages}
\usepackage{amsmath}
\usepackage{cmap} 
\usepackage[T2A]{fontenc} 
\usepackage[utf8]{inputenc}
\usepackage[english,russian]{babel}
\usepackage{indentfirst}
\usepackage{amsmath}
\usepackage{tikz}
\usepackage{multirow}
\usepackage[noend]{algpseudocode}
\usepackage{algorithm}
\usepackage{algorithmicx}
\usetikzlibrary{shapes,arrows}
%usepackage{fancyvrb}
%\usepackage{minted}
%\usepackage{verbments}


\title[]{YaccConstructor}
\subtitle[YaccConstructor]{Задачи на весенний семестр 2019}
% То, что в квадратных скобках, отображается в левом нижнем углу. 
\institute[]{
Лаборатория языковых инструментов JetBrains \\
Санкт-Петербургский государственный университет \\
Математико-механический факультет }

% То, что в квадратных скобках, отображается в левом нижнем углу.
\author[Семён Григорьев]{Семён Григорьев}

\date{8 февраля 2019г.}

\definecolor{orange}{RGB}{179,36,31}

\begin{document}
{
\begin{frame}[fragile]
  \begin{tabular}{p{2.5cm} p{5.5cm} p{2cm}}
   \begin{center}
      \includegraphics[height=1.5cm]{pictures/JBLogo3.pdf}
    \end{center}
    &
    \begin{center}
      \includegraphics[height=1.5cm]{pictures/SPbGU_Logo.png}
    \end{center}
    &
    \begin{center}
      \includegraphics[height=1.5cm]{pictures/YC_logo.pdf}
    \end{center} 
  \end{tabular}
  \titlepage
\end{frame}
}

\begin{frame}[fragile]
  \transwipe[direction=90]
  \frametitle{YaccConstructor}
  \begin{itemize}
    \item Исследования в области теории формальных языков, алгоритмов синтаксического 
    анализа, графовых баз данных
    \item Исследовательская группа лаборатории языковых инстументов JetBrains Research
    \begin{itemize}
      \item \url{https://research.jetbrains.org/groups/plt_lab}
    \end{itemize}
    \item Открытый исходный код
    \begin{itemize}
      \item \url{https://github.com/YaccConstructor}
    \end{itemize}
  \end{itemize}
\end{frame}


\begin{frame}[fragile]
\transwipe[direction=90]
\frametitle{Прикладные задачи}
  Общее требование: хорошие знания в теории формальных языков и алгоритмах синтаксического анализа
  \begin{itemize}
    \item Производные (Brzozowski’s derivatives) для поиска путей с контекстно-свободными 
    ограничениям
    \begin{itemize}
       \item \footnotesize{\url{https://github.com/YaccConstructor/YaccConstructor/issues/306}}
       \item Google Pregel, Scala
       \item 1 человек
       \item Бакалаврский диплом
    \end{itemize}
    \item Распределённый синтаксический нализ графов на основе GLL-алгоритма
    \begin{itemize}
       \item \footnotesize{\url{https://github.com/YaccConstructor/YaccConstructor/issues/212}}
       \item F\#, распределённые вычисления
       \item 1 человек
       \item Бакалаврский диплом
    \end{itemize}
    \item Восстановление после ошибок на основе синтаксического анализа графа
    \begin{itemize}
       \item \footnotesize{\url{https://github.com/YaccConstructor/YaccConstructor/issues/206}}
       \item F\#
       \item 1 человек
       \item Бакалаврский диплом
    \end{itemize}
    \item Распознование шафла контекстно-свободных языков
    \begin{itemize}
       \item \footnotesize{\url{https://github.com/YaccConstructor/YaccConstructor/issues/260}}
       \item F\#, smt-решатели
       \item 1 человек
       \item Бакалаврский диплом
    \end{itemize}
  \end{itemize}  
\end{frame}

\begin{frame}[fragile]
\transwipe[direction=90]
\frametitle{Веб-приложение для классификации цепочек на основе вторичной структуры}
  \begin{itemize}
    \item \footnotesize{\url{https://github.com/YaccConstructor/YC.Bio/issues/10}}
    \item Разработать инфраструктуру, автоматизирующую основные шаги: сборку, тестирование, развёртывание
    \begin{itemize}
       \item Хорошее знание соответствующих технических решений (ci-сервера, системы контроля версий)
       \item 1 человек
       \item Курсовая, семестровый проект
    \end{itemize}
    \item Разработать пользовательский интерфейс (веб)
    \begin{itemize}
       \item Навыки веб-разработки (фронтенд), знание соответствующих языков, библиотек, подходов, технологий
       \item 1 человек
       \item Курсовая, семестровый проект
    \end{itemize}
    \item Разработать серверную компоненту для запуска в облаке
    \begin{itemize}
       \item Знание соответствующих языков и технологий (облачных платформ), Python, навыки работы с GPGPU
       \item 1 человек
       \item Курсовая, семестровый проект
    \end{itemize}
  \end{itemize}  
\end{frame}


\begin{frame}[fragile]
\transwipe[direction=90]
\frametitle{Теоретические исследования}
  \begin{itemize}
    \item Решение теоретических задач
    \begin{itemize}
       \item \footnotesize{\url{https://github.com/YaccConstructor/YaccConstructor/issues/328}}
       \item Теория графов, теория алгоритмов, теория формальных языков 
       \item 2 человека 
       \item Курсовая, семестровый проект
    \end{itemize}
    \item Реализация алгоритма построения группы\footnote{На самом деле, представления группы} по формальной грамматике
    \begin{itemize}
       \item \footnotesize{\url{https://github.com/YaccConstructor/LangToGroup}}
       \item Теория формальных языков, алгебра (теория групп), теория алгоритмов, Haskell 
       \item 1 человек
       \item Семестровый проект
    \end{itemize}
  \end{itemize}  
\end{frame}



\begin{frame}
\transwipe[direction=90]
\frametitle{Контакты}
\begin{itemize}
  \item Почта (она же google hangout для оперативной связи): \url{rsdpisuy@gmail.com}
  \item GitHub: \url{https://github.com/YaccConstructor}
  \item Профиль на JetBrains Research: \url{https://research.jetbrains.org/researchers/gsv}
\end{itemize}
\end{frame}
\end{document}
