%% For double-blind review submission, w/o CCS and ACM Reference (max submission space)
%\documentclass[sigplan,review,anonymous]{acmart}\settopmatter{printfolios=true,printccs=false,printacmref=false}
%% For double-blind review submission, w/ CCS and ACM Reference
%\documentclass[sigplan,review,anonymous]{acmart}\settopmatter{printfolios=true}
%% For single-blind review submission, w/o CCS and ACM Reference (max submission space)
%\documentclass[sigplan,review]{acmart}\settopmatter{printfolios=true,printccs=false,printacmref=false}
%% For single-blind review submission, w/ CCS and ACM Reference
\documentclass[sigplan,review]{acmart}\settopmatter{printfolios=true}
%% For final camera-ready submission, w/ required CCS and ACM Reference
%\documentclass[sigplan]{acmart}\settopmatter{}


%% Conference information
%% Supplied to authors by publisher for camera-ready submission;
%% use defaults for review submission.
\acmConference[PL'18]{ACM SIGPLAN Conference on Programming Languages}{January 01--03, 2018}{New York, NY, USA}
\acmYear{2018}
\acmISBN{} % \acmISBN{978-x-xxxx-xxxx-x/YY/MM}
\acmDOI{} % \acmDOI{10.1145/nnnnnnn.nnnnnnn}
\startPage{1}

%% Copyright information
%% Supplied to authors (based on authors' rights management selection;
%% see authors.acm.org) by publisher for camera-ready submission;
%% use 'none' for review submission.
\setcopyright{none}
%\setcopyright{acmcopyright}
%\setcopyright{acmlicensed}
%\setcopyright{rightsretained}
%\copyrightyear{2018}           %% If different from \acmYear

%% Bibliography style
\bibliographystyle{ACM-Reference-Format}
%% Citation style
%\citestyle{acmauthoryear}  %% For author/year citations
%\citestyle{acmnumeric}     %% For numeric citations
%\setcitestyle{nosort}      %% With 'acmnumeric', to disable automatic
                            %% sorting of references within a single citation;
                            %% e.g., \cite{Smith99,Carpenter05,Baker12}
                            %% rendered as [14,5,2] rather than [2,5,14].
%\setcitesyle{nocompress}   %% With 'acmnumeric', to disable automatic
                            %% compression of sequential references within a
                            %% single citation;
                            %% e.g., \cite{Baker12,Baker14,Baker16}
                            %% rendered as [2,3,4] rather than [2-4].


%%%%%%%%%%%%%%%%%%%%%%%%%%%%%%%%%%%%%%%%%%%%%%%%%%%%%%%%%%%%%%%%%%%%%%
%% Note: Authors migrating a paper from traditional SIGPLAN
%% proceedings format to PACMPL format must update the
%% '\documentclass' and topmatter commands above; see
%% 'acmart-pacmpl-template.tex'.
%%%%%%%%%%%%%%%%%%%%%%%%%%%%%%%%%%%%%%%%%%%%%%%%%%%%%%%%%%%%%%%%%%%%%%


%% Some recommended packages.
\usepackage{booktabs}   %% For formal tables:
                        %% http://ctan.org/pkg/booktabs
\usepackage{subcaption} %% For complex figures with subfigures/subcaptions
                        %% http://ctan.org/pkg/subcaption


\begin{document}

%% Title information
\title[]{TITLE (Extended Abstract)?}         %% [Short Title] is optional;
                                        %% when present, will be used in
                                        %% header instead of Full Title.
% \titlenote{with title note}             %% \titlenote is optional;
                                        %% can be repeated if necessary;
                                        %% contents suppressed with 'anonymous'
% \subtitle{Subtitle}                     %% \subtitle is optional
% \subtitlenote{with subtitle note}       %% \subtitlenote is optional;
                                        %% can be repeated if necessary;
                                        %% contents suppressed with 'anonymous'


%% Author information
%% Contents and number of authors suppressed with 'anonymous'.
%% Each author should be introduced by \author, followed by
%% \authornote (optional), \orcid (optional), \affiliation, and
%% \email.
%% An author may have multiple affiliations and/or emails; repeat the
%% appropriate command.
%% Many elements are not rendered, but should be provided for metadata
%% extraction tools.

\author{Kirill Smirenko}

\affiliation{
%  \position{Position2a}
%  \department{Department2a}             %% \department is recommended
  \institution{St.Petersburg State University}           %% \institution is required
  \streetaddress{Universitetski pr., 28}
  \city{St.Petersburg}
  \postcode{198504}
  \country{Russia}                   %% \country is recommended
}
\email{k.smirenko@gmail.com}         %% \email is recommended

%% Author with single affiliation.
\author{Semyon Grigorev}
%\authornote{Saint Petersburg State University}          %% \authornote is optional;
                                        %% can be repeated if necessary
\orcid{0000-0002-7966-0698}             %% \orcid is optional
\affiliation{
  \position{Associate Professor}
  \institution{St.Petersburg State University}           %% \institution is required
  \streetaddress{Universitetski pr., 28}
  \city{St.Petersburg}
  \postcode{198504}
  \country{Russia}                   %% \country is recommended
}
\email{semen.grigorev@jetbrains.com}          %% \email is recommended


%% Abstract
%% Note: \begin{abstract}...\end{abstract} environment must come
%% before \maketitle command
 \begin{abstract}
 Text of abstract \ldots.
 Text of abstract \ldots.
 Text of abstract \ldots.
 Text of abstract \ldots.
 Text of abstract \ldots.
 \end{abstract}


%% 2012 ACM Computing Classification System (CSS) concepts
%% Generate at 'http://dl.acm.org/ccs/ccs.cfm'.
\begin{CCSXML}
<ccs2012>
<concept>
<concept_id>10011007.10011006.10011008.10011009.10011012</concept_id>
<concept_desc>Software and its engineering~Functional languages</concept_desc>
<concept_significance>500</concept_significance>
</concept>
<concept>
<concept_id>10011007.10011006.10011008.10011009.10010175</concept_id>
<concept_desc>Software and its engineering~Parallel programming languages</concept_desc>
<concept_significance>100</concept_significance>
</concept>
</ccs2012>
\end{CCSXML}
\ccsdesc[500]{Software and its engineering~Functional languages}
\ccsdesc[100]{Software and its engineering~Parallel programming languages}%% End of generated code


%% Keywords
%% comma separated list
\keywords{Type Providers, Metaprogramming, Generic Programming, GPGPU, OpenCL}  %% \keywords are mandatory in final camera-ready submission


%% \maketitle
%% Note: \maketitle command must come after title commands, author
%% commands, abstract environment, Computing Classification System
%% environment and commands, and keywords command.
\maketitle


\section{Introduction}

Graphical processor units (GPUs) are commonly used for fast computations. Their multi-core architecture benefits high-load computations in applied science, computer vision, bioinformatics and other areas. This approach is called General Purpose Graphical Processor Unit, or GPGPU.~\cite{CUDA_to_OpenCL, GPGPU_1}

Several frameworks for GPGPU programming are known. The most popular one is CUDA[, a platform for parallel GPGPU computations developed by Nvidia in 2007]~\cite{CUDA}. Another important project is Open Computing Language (OpenCL), an open standard for cross-platform parallel computing on different platforms, including GPGPU~\cite{OpenCL}.

The technologies mentioned provide special programming languages: CUDA C/C++, OpenCL C/C++. However using higher-level languages, such as C\# or F\#, can be more convenient for GPGPU development, for the following reasons: they are used more commonly for general software development (REWRITE? REFERENCE?); also, they are both strongly and statically typed, which, with the help of integrated development environments (IDEs), facilitates development and improves the reliability of software [REFERENCE?]. In this article, in context of GPGPU development, we are going to call programming in special CUDA/OpenCL languages \i{lower-level development}, and coding in C\#/F\# \i{higher-level development}.

There are several instruments for higher-level development~\cite{Brahma_FSharp, FSCL, AleaGPU}. TODO

TODO --- Alea GPU, CUSP, ManagedCuda, their disadvantages

% Brahma.FSharp

\section{F\# type providers}

A type provider~\cite{syme2012strongly} is a component of F\# programming language that provides types, properties and methods during runtime ~\cite{TypeProviders}. One major application of F\# type providers is integration with dynamic data sources, such as SQL, CSV, JSON, in a strongly typed way~\cite{FSharpData}. TODO --- type parameters

A common alternative is to use code generation. However, type providers ensure better integration with user context because they work at runtime, while the generated code must be replaced each time the data source is modified.

\section{OpenCL type provider}

Figure~\ref{architecture} summarizes the proposed solution. TODO --- describe

\begin{figure}[h]
\centering
\includegraphics[width=8.5cm]{graphics/architecture.pdf}
\caption{Overview of the solution}
\label{architecture}
\end{figure}

TODO: Lexer, Parser --- FsLex, YaccConstructor

TODO: parameters

TODO: IDE screenshot (autocompletion)

% Restrictions
There are several restrictions to our solution:
\begin{itemize}
    \item TODO --- move restrictions from coursework slides/source code
    \item the F\# functions generated by the type provider cannot be called, and are intended only for usage in quotations that are translated to OpenCL code via Brahma.FSharp;
    \item OpenCL grid parameters and other constants can be passed only in OpenCL C header files.
\end{itemize}

\section{Evaluation}

Brahma.FSharp example? Something else?


%% Acknowledgments
% \begin{acks}                            %% acks environment is optional
%                                         %% contents suppressed with 'anonymous'
%   %% Commands \grantsponsor{<sponsorID>}{<name>}{<url>} and
%   %% \grantnum[<url>]{<sponsorID>}{<number>} should be used to
%   %% acknowledge financial support and will be used by metadata
%   %% extraction tools.
%   This material is based upon work supported by the
%   \grantsponsor{GS100000001}{National Science
%     Foundation}{http://dx.doi.org/10.13039/100000001} under Grant
%   No.~\grantnum{GS100000001}{nnnnnnn} and Grant
%   No.~\grantnum{GS100000001}{mmmmmmm}.  Any opinions, findings, and
%   conclusions or recommendations expressed in this material are those
%   of the author and do not necessarily reflect the views of the
%   National Science Foundation.
% \end{acks}


%% Bibliography
\bibliography{ext-abstract}




%% Appendix
% \appendix
% \section{Appendix}

% Text of appendix \ldots

\end{document}
