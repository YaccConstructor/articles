\documentclass[12pt]{article}  % standard LaTeX, 12 point type

\usepackage{geometry}

\usepackage{amsmath}
\usepackage{amsfonts,latexsym}
\usepackage{amsthm}
\usepackage{amssymb}
\usepackage[utf8]{inputenc} % Кодировка
\usepackage[english,russian]{babel} % Многоязычность
\usepackage{verbatim}
\usepackage{longtable}
\usepackage{csvsimple}
\usepackage[toc,page]{appendix}
\usepackage{booktabs}

\usepackage{float}
\usepackage{array}
\usepackage{multirow}
\usepackage{caption}
\usepackage{graphicx}
\usepackage{ucs}
\usepackage{rotating}
\usepackage{pdflscape}
\usepackage{afterpage}
\usepackage{capt-of}% or use the larger `caption` package
\usepackage{url}

% unnumbered environments:

\theoremstyle{remark}
\newtheorem*{remark}{Remark}
\newtheorem*{notation}{Notation}
\newtheorem*{note}{Note}

\setlength{\parskip}{5pt plus 2pt minus 1pt}
\newcolumntype{C}{>{\centering\arraybackslash}p{1.3cm}}
\graphicspath{{pics/}}

\title{Разработка методов анализа граф-структурированных данных, основанных на теории формальных языков}
\author{Семён Григорьев}
\date{\today}

\begin{document}

\newgeometry{left=0.8in,right=0.8in,top=1in,bottom=1in}

\maketitle

\section{Сведения о проекте}

\subsection{Название проекта}

\textbf{ru}\\
%
Использование формальных грамматик и искусственных нейронных сетей для анализа вторичной структуры геномных и протеомных  последовательностей
\\
\\
\textbf{en}\\
Utilization of formal grammars and artificial neural networks for secondary structure analysis of the genomic and proteomic sequences

\subsection{Приоритетное направление развития науки, технологий и техники в Российской Федерации, критическая технология}
%


\subsection{Направление из Стратегии научно-технологического развития Российской Федерации (утверждена Указом Президента Российской Федерации от 1 декабря 2016 г. \textnumero 642 ``О Стратегии научно-технологического развития Российской Федерации'') (при наличии)}
%

\subsection{Ключевые слова (приводится не более 15 терминов)}

\textbf{ru}\\
%
Формальные грамматики, синтаксический анализ, параллельные алгоритмы, вторичная структура, РНК, геномные последовательности, белки, протеомные последовательности, метагеномная сборка, искусственные нейронные сети.
\\
\\
\textbf{en}\\
Formal grammars, syntax analysis, parsing, parallel algorithms, secondary structure, RNA, genomic sequences, proteins, proteomic sequences, metagenomic assembly, artificial neural network


\subsection{Аннотация проекта}
%(объемом не более 2 стр.; в том числе кратко – актуальность решения указанной выше научной проблемы и научная новизна)
\textbf{ru}\\
%
\\
\\
\textbf{en}\\

\subsection{Ожидаемые результаты и их значимость}
%(указываются результаты, их научная и общественная значимость (соответствие предполагаемых результатов мировому уровню исследований, возможность практического использования предполагаемых результатов проекта в экономике и социальной сфере))

\textbf{ru}\\
%
\\
\\
\textbf{en}\\

\subsection{В состав научного коллектива будут входить}
%
\beign{itemize}
\item исполнителей проекта (включая руководителя)
\item в том числе !!!  исполнителей в возрасте до 39 лет включительно,
\item из них: !!! очных аспирантов, адъюнктов, интернов, ординаторов, студентов.
\end{itemize}

\subsection{Планируемый состав научного коллектива с указанием фамилий, имен, отчеств (при наличии) членов коллектива, их возраста на момент подачи заявки, ученых степеней, должностей и основных мест работы, формы отношений с организацией (трудовой договор, гражданско-правовой договор) в период реализации проекта.}


\textbf{Соответствие профессионального уровня членов научного коллектива задачам проекта}
\textbf{ru}
\\
\\
\textbf{en}

\subsectopn{Планируемый объем финансирования проекта Фондом по годам (указывается в тыс. рублей)}
2020 г. - тыс. рублей,
2021 г. - введите планируемый объем финансирования в 2021 г. тыс. рублей,
2022 г. - введите планируемый объем финансирования в 2022 г. тыс. рублей.

\subsection{Научный коллектив по результатам проекта в ходе его реализации предполагает опубликовать в рецензируемых российских и зарубежных научных изданиях не менее}
%Приводятся данные за весь период выполнения проекта. Уменьшение количества публикаций (в том числе отсутствие информации в соответствующих полях формы) по сравнению с порогом, установленным в пункте 16.2 конкурсной документации является основанием недопуска заявки к конкурсу.

!!! публикаций

из них !!!введите число:!!! в изданиях, индексируемых в базах данных «Сеть науки» (Web of Science Core Collection) или «Скопус» (Scopus).

\textbf{Информация о научных изданиях, в которых планируется опубликовать результаты проекта, в том числе следует указать в каких базах индексируются данные издания - «Сеть науки» (Web of Science Core Collection), «Скопус» (Scopus), РИНЦ, иные базы, а также указать тип публикации - статья, обзор, тезисы, монография, иной тип}


\textbf{Иные способы обнародования результатов выполнения проекта}

\subsection{Число публикаций членов научного коллектива, опубликованных в период с 1 января 2015 года до даты подачи заявки}

!!!введите число:!!!, из них !!!введите число:!!! – опубликованы в изданиях, индексируемых в Web of Science Core Collection или в Scopus.

\subsection{Планируемое участие научного коллектива в международных коллаборациях (проектах) (при наличии)}


\vline
Руководитель проекта подтверждает, что
\begin{itemize}
\item все члены научного коллектива (в том числе руководитель проекта) удовлетворяют пунктам 6, 7, 13 конкурсной документации;
\item на весь период реализации проекта он будет состоять в трудовых отношениях с организацией;
\item при обнародовании результатов любой научной работы, выполненной в рамках поддержанного Фондом проекта, он и его научный коллектив будут указывать на получение финансовой поддержки от Фонда и организацию, а также согласны с опубликованием Фондом аннотации и ожидаемых результатов поддержанного проекта, соответствующих отчетов о выполнении проекта, в том числе в информационно-телекоммуникационной сети «Интернет»;
\item помимо гранта Фонда проект не будет иметь других источников финансирования в течение всего периода практической реализации проекта с использованием гранта Фонда;
\item проект не является аналогичным по содержанию проекту, одновременно поданному на конкурсы научных фондов и иных организаций;
\item проект не содержит сведений, составляющих государственную тайну или относимых к охраняемой в соответствии с законодательством Российской Федерации иной информации ограниченного доступа;
\item доля членов научного коллектива в возрасте до 39 лет включительно в общей численности членов научного коллектива будет составлять не менее 50 процентов в течение всего периода практической реализации проекта;
\item в установленные сроки будут представляться в Фонд ежегодные отчеты о выполнении проекта и о целевом использовании средств гранта.
\end{itemize}

\section{Содержание проекта}

\subsection{Научная проблема, на решение которой направлен проект}

\textbf{ru}\\
%
\\
\\
\textbf{en}\\

\subsection{Научная значимость и актуальность решения обозначенной проблемы}

\textbf{ru}\\
%
\\
\\
\textbf{en}\\


\subsection{Конкретная задача (задачи) в рамках проблемы, на решение которой направлен проект, ее масштаб и комплексность}

\textbf{ru}\\
%
\\
\\
\textbf{en}\\

\subsection{Научная новизна исследований, обоснование достижимости решения поставленной задачи (задач) и возможности получения запланированных результатов}

\textbf{ru}\\
%
\\
\\
\textbf{en}\\

\subsection{Современное состояние исследований по данной проблеме, основные направления исследований в мировой науке и научные конкуренты}

\textbf{ru}\\
%
\\
\\
\textbf{en}\\

\subsection{Предлагаемые методы и подходы, общий план работы на весь срок выполнения проекта и ожидаемые результаты }
%(объемом не менее 2 стр.; в том числе указываются ожидаемые конкретные результаты по годам; общий план дается с разбивкой по годам)

\textbf{ru}\\
\\
\\
\textbf{en}\\

\subsection{Имеющийся у научного коллектива научный задел по проекту, наличие опыта совместной реализации проектов}

\textbf{ru}\\
%
Руководитель проекта обладает опытом в разработке и исследовании алгоритмов синтаксического анализа, и их применении в различных областях, что подтверждается соответствующими статьями (Grigorev, Ragozina, "Context-free path querying with structural representation of result", SECR-2017; Azimov, Grigorev, "Context-free path querying by matrix multiplication", GRADES-NDA-2018; Verbitskaia, Kirillov, Nozkin, Grigorev, "Parser combinators for context-free path querying", Scala-2018)

В том числе, у руководителя имеется опыт применения формальных грамматик и алгоритмов синтаксического анализа для решения задач в области биологии (биоинформатики), что подтверждается выступлениями на тематических конференциях Biata-2017/2018, BIOINFORMATICS-2019.

Кроме того, руководителем был предложен метод совмещения формальных грамматик и ИНС для анализа вторичной структуры, который предполагается развивать в рамках данного исследования. Метод был изложен в статье "The Composition of Dense Neural Networks and Formal Grammars for Secondary Structure Analysis" и представлен на конференции BIOINFORMATICS-2019.

Руководитель принимал успешное участие в совместной работе над проектами в рамках грантов РФФИ (15-01-05431 и 18-01-00380), Фонда содействия развитию малых форм предприятий в технической сфере
(программа УМНИК, проекты N 162ГУ1/2013 и N 5609ГУ1/2014), а также является руководителем научной группы, в соавторстве с участниками которой опубликованы указанные выше и некоторые другие работы.

\subsection{Перечень оборудования, материалов, информационных и других ресурсов, имеющихся у научного коллектива для выполнения проекта}
\textbf{ru}\\
%


\subsection{План работы на первый год выполнения проекта}

\textbf{ru}\\
%
\\
\\
\textbf{en}\\

\subsection{Ожидаемые в конце первого года конкретные научные результаты}
%(форма изложения должна дать возможность провести экспертизу результатов и оценить степень выполнения заявленного в проекте плана работы)

\textbf{ru}\\
%
\\
\\
\textbf{en}\\

\subsection{Перечень планируемых к приобретению руководителем проекта за счет гранта Фонда оборудования, материалов, информационных и других ресурсов для выполнения проекта}
%(в том числе – описывается необходимость их использования для реализации проекта)

\textbf{ru}\\
%
Не более 200 тыс. рублей ежегодно будет тратиться на поездки с докладами на конференции. Расходов на оборудование не предполагается.


\end{document}
