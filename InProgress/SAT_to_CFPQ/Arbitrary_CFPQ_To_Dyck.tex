\section{From Arbitrary CFPQ to Dyck Query}\label{sec:cfpq_to_dyck}

This reduction is inspired by the construction described in~\cite{OptimalDLR}.

Consider a context-free grammar $\mathcal{G}=(\Sigma, N, P, S)$ in BNF where $\Sigma$ is a terminal alphabet, $N$ is 
a nonterminal alphabet, $P$ is a set of productions, $S \in N$ is a start nonterminal.
Also we denote a directed labeled graph by $G=(V,E,L)$ where $E \subseteq V \times L \times V$ and $L \subseteq \Sigma$. 

We should construct new input graph $G'$ and new grammar $\mathcal{G'}$ such that $\mathcal{G'}$ specifies a Dyck language and there is a simple mapping from $\text{CFPQ}(\mathcal{G'}, G')$ to $\text{CFPQ}(\mathcal{G}, G)$.
Step-by-step example with description is provided below.
 
Let the input grammar is 
\begin{align*}
S & \rightarrow a \ S \ b \ | \ a \ C \ b 
\\
C & \rightarrow c \ | \ C \ c
\end{align*}

The input graph is presented in fig.~\ref{input}.

\begin{figure}
\resizebox{.5\textwidth}{!}
{
\begin{tikzpicture}[shorten >=1pt,node distance=2cm,on grid,auto] 
   \node[state] (q_0)   {$0$}; 
   \node[state] (q_1) [above right=of q_0] {$1$}; 
   \node[state] (q_2) [right=of q_1] {$2$}; 
   \node[state] (q_3) [right=of q_2] {$3$};
   \node[state] (q_4) [right=of q_3] {$4$};
   \node[state] (q_5) [right=of q_4] {$5$};
    \path[->] 
    (q_0) edge  node {a} (q_1)          
    (q_1) edge  node {a} (q_2)
    (q_2) edge  node {a} (q_0)
    (q_2) edge  node {c} (q_3)
    (q_3) edge  node {c} (q_4)
    (q_4) edge[bend left, above]  node {b} (q_5)
    (q_5) edge[bend left, below]  node {b} (q_4);
\end{tikzpicture}
}

\caption{The input graph}
\label{input}
\end{figure}

\begin{enumerate}
\item Let $\Sigma_{()} =\{ t_( , t_)  | t \in \Sigma \}$.
\item Let $N_{()} = \{ N_( , N_) | N \in N  \}$.
\item Let $M_{\mathcal{G}} = (V_{\mathcal{G}}, E_{\mathcal{G}}, L_{\mathcal{G}})$ is a directed 
labeled graph, where $L_{\mathcal{G}} \subseteq (\Sigma_{()} \cup N_{()})$.
This graph is created the same manner as described in~\cite{OptimalDLR} but we do not require the grammar be in CNF.
Let $x \in V_{\mathcal{G}}$ and $y \in V_{\mathcal{G}}$ is ``start'' and ``final'' vertices respectively. 
This graph may be treated as a finite automaton, so it can be minimized and we can compute an $\varepsilon$-closure if the input grammar contains $\varepsilon$ productions.
The graph $M_{\mathcal{G}}$ for our example is presented in fig.~\ref{mod}.

\begin{figure}
\includegraphics[width=.5\textwidth]{dot/grammar_1.pdf}

\caption{The $M_{\mathcal{G}}$ graph}
\label{mod}

\end{figure}


The minimized graph is presented in fig.~\ref{minimized}.
\begin{figure}
\includegraphics[width=.5\textwidth]{dot/grammar_min.pdf}


\caption{The minimized $M_{\mathcal{G}}$}
\label{minimized}

\end{figure}


\item For each $v \in V$ create $M_{\mathcal{G}}^v$: unique instance of $M_{\mathcal{G}}$.
\item New graph $G^{'}$ is a graph $G$ where each label $t$ is replaced with $t_{)}^i$ and some additional edges are created:
\begin{itemize}
\item Add an edge $(v', S_(, v)$ for each $v \in V$. 
\item And the respective $M_{\mathcal{G}}^v$ for each $v \in V$:
  \begin{itemize}
    \item reattach all edges outgoing from $x^v$ (``start'' vertex of $M_{\mathcal{G}}^v$) to $v$;
    \item reattach all edges incoming to $y^v$ (``final'' vertex of $M_{\mathcal{G}}^v$) to $v$.    
  \end{itemize}
  New input graph is ready. It is presented in fig.~\ref{newInput}.

\begin{figure*}  
  \includegraphics[width=.9\textwidth]{dot/input_new_min.pdf}
 
  \caption{New input graph}
  \label{newInput}

\end{figure*}

\end{itemize}

\item New grammar $\mathcal{G'}=(\Sigma^{'}, N', P', S')$ where $\Sigma^{'} = \Sigma_{()} \cup N_{()}$, $N' = \{ S' \}$, $P' = \{ S' \rightarrow b_( \ S' \ b_); S' \rightarrow b_( \ b_) \ | \ b_(, b_{)} \in \Sigma^{'} \} \cup \{ S' \rightarrow S' \ S' \}$ is a set of productions, $S' \in N'$ is a start nonterminal.
\end{enumerate}

Now, if $\text{CFPQ}(\mathcal{G'}, G')$ contains a pair $(u'_0, v')$ such that $e=(u'_0, S_( , u'_1) \in E'$ is an extension edge (step 5, first subitem), then  $(u'_1, v') \in \text{CFPQ}(\mathcal{G}, G)$.
In our example, we can find the following path: $7 \xrightarrow{S_(} 1 \xrightarrow{S_)} 22 \xrightarrow{b_(} 25 \xrightarrow{C_(} 26 \xrightarrow{a_(} 1   
\xrightarrow{a_)} 2 \xrightarrow{C_)} 33 \xrightarrow{C_(} 34 \xrightarrow{c_(} 2  \xrightarrow{c_)} 3 \xrightarrow{C_)} 43 \xrightarrow{c_(} 3 \xrightarrow{c_)} 4 \xrightarrow{b_)} 5$. 
Edge $7 \xrightarrow{S_(} 1$ is the extension, so (1,5) should be in $\text{CFPQ}(\mathcal{G}, G)$ and it is true.


