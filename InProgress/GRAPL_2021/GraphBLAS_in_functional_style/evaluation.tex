\section{Evaluation}

While our implementation of GraphBLAS API is on very early stage, we cannot evaluate it on well-known linear algebra based algorithms. 
But in order to !!! Elementwise addition. 

We perform our experiments on the PC with Ubuntu 18.04 installed and with the following hardware configuration: !!! CPU, !!! RAM, !!!GPGPU with !!!!.

our solution on CPU and GPGPU.
For comparison we choose the following libraries.
\begin{itemize}
\item SuiteSparse as a ...
\item Math.NET Numerics\footnote{Library which provides numerical computations primitives for .NET: \url{https://numerics.mathdotnet.com/}. Access date: 12.01.2021.}
\item GraphBLAST
\end{itemize}



Dataset description. Matrices form SuiteSparse collection\footnote{!!!}

\begin{table}[h]
    \centering
    \caption{Matrices for evaluation}
    \label{matrices}  
    \begin{tabular}{ | c || c | c | c | }
    \hline
    Name & Size & NNZ & NNZ in square \\ \hline
    \hline
    wing & 62 032 & 243 088 & 714,200 \\ \hline
    luxembourg\_osm & 114 599 & 119 666 & 4 582 \\ \hline
    amazon0312 & 400 727 & 3 200 440 & 14 390 544 \\ \hline
    amazon-2008 & 735 323 & 5 158 388 & 25 366 745 \\ \hline
    web-Google & 916 428 & 5 105 039 & 30 811 855 \\ \hline
    webbase-1M & 1 000 005 & 3 105 536 & 51 111 996 \\ \hline
    cit-Patents & 3 774 768 & 16 518 948 & 469 \\ \hline
    \end{tabular}
\end{table}

For each matrix !!!!.
For .NET-based implementations \textit{BenchmarkDotNet}\footnote{\textit{BenchmarkDotNet} allows one to automate benchmarking process for .NET platform. Project web page: \url{https://benchmarkdotnet.org/}. Access date: 12.01.2021.} is used.
Results of performance evaluation are presented in table~\ref{tbl_results}.
Time is measured in !!!

\begin{table}[h]
    \centering
    \caption{evaluation results for CSR, GTX 2070, time in ms}
    \label{platform-comparison}

    \begin{tabular}{|c||c|c|c|}
    \hline
    Название            & GraphBLAS-sharp & SuiteSparse & CUSP        \\
    \hline
    \hline
    wing            & $1,8 \pm 0,1$      & $1,9\pm 0,1$   & $0,5\pm 0,2$   \\
    \hline
    luxembourg\_osm & $2,9 \pm 0,3$      & $1.9\pm 0,5$   & $0,5\pm 0,1$   \\
    \hline
    amazon0312      & $17,0 \pm 0,8$      & $28,9\pm 0,2$  & $2,8\pm 0,1$   \\
    \hline
    amazon-2008     & $12,2 \pm 0,8$     & $50,1\pm 2,4$  & $3,5\pm 0,1$   \\
    \hline
    web-Google      & $18,4 \pm 0,6$     & $58.8\pm 0,7$  & $3,6\pm 0,1$   \\
    \hline
    webbase-1M      & $70,7 \pm 1,0$      & $72,9\pm 0,4$  & $24,6\pm 2,1$  \\
    \hline
    cit-Patents     & $54,6 \pm 1,2$      & $157,4\pm 1,2$ & $8,5\pm 1,2$   \\     
    \hline
    \end{tabular}
\end{table}

\begin{table}[h]

    \centering
    \caption{Evaluation results for element-wise multiplication, GTX 2070, time in ms}
    \label{mult-comparison}
    
    \begin{tabular}{|c||c|c|}
    \hline
    Название            & GraphBLAS-sharp & SuiteSparse    \\
    \hline
    \hline
    wing            & $2,5 \pm 0,4$      & $1,0 \pm 0,1$ \\
    \hline
    luxembourg\_osm & $2,6 \pm 0,3$       & $1,4 \pm 0,3$ \\
    \hline
    amazon0312      & $13,0 \pm 1,0$     & $23,0 \pm 0,9$ \\
    \hline
    amazon-2008     & $9,1 \pm 0,8$    & $35,2 \pm 4,0$ \\
    \hline
    web-Google      & $14,7 \pm 0,8$      & $43,9 \pm 0,2$  \\
    \hline
    webbase-1M      & $55,4 \pm 1,2$      & $31,0 \pm 1,6$ \\
    \hline
    cit-Patents     & $47,9 \pm 0,9$      & $107,9 \pm 0,4$  \\     
    \hline
    \end{tabular}
\end{table}
We can see, that !!!! results analysis and conclusion.
