\documentclass[conference]{IEEEtran}
\IEEEoverridecommandlockouts
% The preceding line is only needed to identify funding in the first footnote. If that is unneeded, please comment it out.
\usepackage{cite}
\usepackage{amsmath,amssymb,amsfonts}
\usepackage{algorithmic}
\usepackage{graphicx}
\usepackage{textcomp}
\usepackage{hyperref}
\usepackage{booktabs}
\usepackage[table]{xcolor}

\def\BibTeX{{\rm B\kern-.05em{\sc i\kern-.025em b}\kern-.08em
    T\kern-.1667em\lower.7ex\hbox{E}\kern-.125emX}}
\begin{document}

\title{The Library of GPGPU-Powered Sparse Boolean Linear Algebra Operations
\thanks{Identify applicable funding agency here. If none, delete this.}
}

\author{
\IEEEauthorblockN{Egor Orachev}
\IEEEauthorblockA{\textit{Saint Petersburg State University} \\
%\textit{name of organization (of Aff.)}\\
St. Petersburg, Russia \\
egor.orachev@gmail.com}
\and

\IEEEauthorblockN{Maria Karpenko}
\IEEEauthorblockA{\textit{ITMO University} \\
%\textit{name of organization (of Aff.)}\\
St. Petersburg, Russia \\
mkarpenko.spb@gmail.com}
\and

% \IEEEauthorblockN{Vasily Kuporosov}
% \IEEEauthorblockA{\textit{HSE University} \\
% %\textit{name of organization (of Aff.)}\\
% St. Petersburg, Russia \\
% vvkuporosov@edu.hse.ru}
% \and

% Не бейте меня. Просто хотел места больше свободного сделать

\IEEEauthorblockN{Artem Khoroshev}
\IEEEauthorblockA{\textit{Computation Biology} \\
\textit{Department} \\
\textit{BIOCAD}\\
St. Petersburg, Russia \\
arthoroshev@gmail.com}
\and

\IEEEauthorblockN{Semyon Grigorev}
\IEEEauthorblockA{\textit{Saint Petersburg State University, }\\
                          %{7/9 Universitetskaya nab.}\\
                  \textit{JetBrains Research,} \\
                          %{Primorskiy prospekt 68-70, Building 1,} 
                  {St. Petersburg, Russia}\\
s.v.grigoriev@spbu.ru, \\ semyon.grigorev@jetbrains.com}
}

\maketitle

\begin{abstract}
Sparse matrices are widely applicable in data analysis, and the theory of matrix processing is well-established and introduces a wide range of different algorithms for basic operations such as matrix-matrix and matrix-vector multiplication, factorization, etc. 
To make this observation practical, GraphBLAS API provides a set of respective building blocks, allows one to reduce algorithms to sparse linear algebra operations. 
While GPGPU utilization for high-performance linear algebra is a common practice, the high complexity of GPGPU programming makes the implementation of GraphBLAS API on GPGPU challenging. 
In this work, we present a GPGPU library of sparse operations for an important case --- Boolean algebra ---, which is based on modern algorithms for sparse matrix processing.
We provide Python !!!
Our evaluation shows that !!!
We hope that our results help to move the development of the GPGPU version of GraphBLAS API forward.
\end{abstract}

\begin{IEEEkeywords}
sparse linear algebra, GPGPU, boolean semiring, sparse boolean matrix
\end{IEEEkeywords}


\section{Introduction}

Scalable high-performance graph analysis is an actual challenge.
There is a big number of ways to attack this challenge~\cite{Coimbra2021} and the first promising idea is to utilize general-purpose graphic processing units (GPGPU).
Such existing solutions, as CuSha~\cite{10.1145/2600212.2600227} and Gunrock~\cite{7967137} show that utilization of GPUs can improve the performance of graph analysis, moreover it is shown that solutions may be scaled to multi-GPU systems.
But low flexibility and high complexity of API are problems of these solutions.

The second promising thing which provides a user-friendly API for high-performance graph analysis algorithms creation is a GraphBLAS API~\cite{7761646} which provides linear algebra based building blocks to create graph analysis algorithms.
The idea of GraphBLAS is based on a well-known fact that linear algebra operations can be efficiently implemented on parallel hardware.
Along with that, a graph can be natively represented using matrices: adjacency matrix, incidence matrix, etc.
While reference CPU-based implementation of GraphBLAS, SuiteSparse:GraphBLAS~\cite{10.1145/3322125}, demonstrates good performance in real-world tasks, GPU-based implementation is challenging.

One of the challenges in this way is that real data are often sparse, thus underlying matrices and vectors are also sparse, and, as a result, classical dense data structures and respective algorithms are inefficient. 
So, it is necessary to use advanced data structures and procedures to implement sparse linear algebra, but the efficient implementation of them on GPU is hard due to the irregularity of workload and data access patterns.
Though such well-known libraries as cuSPARSE show that sparse linear algebra operations can be efficiently implemented for GPGPU, it is not so trivial to implement GraphBLAS on GPGPU. 
First of all, it requires \textit{generic} sparse linear algebra, thus it is impossible just to reuse existing libraries which are almost all specified for operations over floats.
The second problem is specific optimizations, such as masking fusion, which can not be natively implemented on top of existing kernels.
Nevertheless, there is a number of implementations of GraphBLAS on GPGPU, such as GraphBLAST~\cite{yang2019graphblast}, GBTL~\cite{7529957}, which show that GPGPUs utilization can improve the performance of GraphBLAS-based graph analysis solutions.
But these solutions are not portable because they are based on Nvidia Cuda stack.
Moreover, the scalability problem is not solved: all these solutions support only single-GPU, not multi-GPU computations.

To provide portable GPU implementation of GraphBLAS API we developed a \textit{SPLA} library\footnote{Source code available at: \url{https://github.com/JetBrains-Research/spla}}.
This library utilizes OpenCL for GPGPU computing to be portable across devices of different vendors.
Moreover, it is initially designed to utilize multiple GPGPUs to be scalable.
To sum up, the contribution of this work is the following.
\begin{itemize}
    \item Design of portable GPU GraphBLAS implementation proposed. The design involves the utilization of multiple GPUS. Additionally, the proposed design is aimed to simplify library tuning and wrappers for different high-level platforms and languages creation. 
    \item Subset of GraphBLAS API, including such operations as masking, matrix-matrix multiplication, matrix-matrix e-wise addition, is implemented. The current implementation is limited by COO and CSR matrix representation format and uses basic algorithms for some operations, but work in progress and more data formats will be supported and advanced algorithms will be implemented in the future.
    \item Preliminary evaluation on such algorithms as breadth-first search (BFS) and triangles counting (TC), and real-world graphs shows portability across different vendors and promising performance: for some problems Spla is comparable with GraphBLAST. Surprisingly, for some problems, the proposed solution on embedded Intel graphic card shows better performance than SuiteSparse:GraphBLAS on the respective CPU. At the same time, the evaluation shows that further optimization is required.
\end{itemize} 
% \section{Related work}

Existing libraries, algorithns, frameworks.
\section{About Libraries}

% Details on implementation. 
% Architecture.

% TODO:
% - stunning architecture diagram
% - exciting example of usage
% - unbelievable python API showcase 

Implemented sparse boolean linear algebra libraries for OpenCL and NVIDIA Cuda platforms are called
\textit{clBool}\footnote{clBool project: \url{https://replace/me/with/actual/url}} and \textit{cuBool}\footnote{cuBool project: \url{https://github.com/JetBrains-Research/cuBool}} respectively.
Projects are hosted at GitHub platform. The source code is licensed under MIT license.
The build process is configured with CMake tool. 
This process is straightforward and requires setup of only basic components and instruments, such as compiler, build configuration tools and platform-specific development kits.

Conceptual libraries architecture is depicted at figure~\ref{fig:generic_architecture}. 
The \textbf{Core} module itself is written in the C++ programming language, which is well-suited for performance and resource critical computational tasks. 
This module provides basic functionality, manages global state and answers user requests. 
The \textbf{Specifics} module provides actual operations and primitives implementation for concrete target execution platform.
Library exposes C compatible API, what gives expressiveness and allows to embed that API into other execution environments via interoperability mechanisms.
\textbf{Wrapper} modules generally encapsulates such functionality and provide it for target high-level runtimes, such as Python or .NET. 

Since library interface, core functionality and high-level wrappers stay the same for different platform specific implementations, it is convenient to create several Specifics modules for Cuda or OpenCL backends.
At this time clBool and cuBool have distinct infrastructures, but it can be integrated into single library with common interface and structure. 
This integration is something to be done in near future.
This process requires careful selection of the interface to allow the end user to properly configure the library for specific tasks, as well as the option to automatically select a specific implementation depending on the capabilities of the target device.

\begin{figure}[t]
    \centering
    % Todo: maybe draw this diagram more compact (vertically)
    \includegraphics[width=0.4\textwidth]{generic_architecture.png}
    \caption{Conceptual sparse boolean linear algebra library architecture}
    \label{fig:generic_architecture}
\end{figure}

Libraries operate on boolean semiring with values set \{\textit{true}, \textit{false}\} with \textit{false} as 
a neutral element, '+' operation defined as logical \textit{or} and '*' defined as logical \textit{and}. Values are also denoted as $\{1,~0\}$ respectively, and the abbreviation $\textit{NNZ(M)}$ gives the number of non-zero cells of the matrix $M$.

Main primitive is sparse matrix of boolean values, stored in one of the sparse formats. Sparse vector 
primitive is not presented, since its utilization is relatively rare presented in practical computational 
tasks. But its support is something to be added in far future. Primary available operations and functions are following.

\begin{itemize}
    \item Create sparse matrix $M$ of size $m \times n$.
    \item Delete sparse matrix $M$ and release all its internal resources.
    \item Fill the matrix $M$ with values $L = \{(i,j)_k\}_k$. The result of this operation is $M_{i,j} = 
    \textit{true}$ for each $(i, j) \in L$, and $M_{i,j} = \textit{false}$ for the rest of matrix values.
    \item Read matrix $M$ values $L = \{(i, j)~|~M_{i,j} = \textit{true}\}$.
    \item Matrix-matrix multiply-add operation $C \mathrel{+}= M \times N$.
    \item Matrix-matrix add operation $M \mathrel{+}= N$.
    \item Matrix-matrix Kronecker product $K = M \otimes N$.
\end{itemize}
\section{Implementation Details}

Currenly, our goal is to evaluate the applicability of the proposed algorithm, thus we implemented its naive version.
We compute the transitive closure from scratch on each iteration and do not use any incremental techniques.
In our implementation we use PyGraphBLAS\footnote{GitHub repository of PyGraphBLAS, a Python wrapper for GraphBLAS API: \url{https://github.com/michelp/pygraphblas}. Access date: 07.07.2020.} --- a Python wrapper for SuiteSparse library~\cite{10.1145/3322125}\footnote{Web page of SuiteSparse:GraphBLAS library: \url{http://faculty.cse.tamu.edu/davis/GraphBLAS.html}. Access date: 07.07.2020.}.
SuiteSparse is a C implementation of GraphBLAS~\cite{7761646} standard which introduces linear algebra building blocks for implementation of graph analysis algorithms.
Thus we provide a highly-optimized parallel CPU implementation of the naive version of the algorithm\footnote{Implementation of the described algorithm is published here: \url{anonimized/url/to/our/repo/with/sources}. Access date: 07.07.2020.}.%{https://github.com/JetBrains-Research/CFPQ_PyAlgo}. Access date: 07.07.2020.}.

At present, we do not integrate with a graph database and a graph query language.
We suppose that the input graph is stored in a file, while the query is expressed in terms of a context-free grammar and is also stored in file.
As it was shown in~\cite{10.1145/3398682.3399163}, it is possible to integrate SuiteSparse based implementation in the RedisGraph database.
Providing integration with a query language requires a lot of technical effort to extend the language.
There are existing proposals, for example to extend the Cypher language\footnote{Cypher language extension proposal which introduces a syntax to express context-free queries: \url{https://github.com/thobe/openCypher/blob/rpq/cip/1.accepted/CIP2017-02-06-Path-Patterns.adoc}. Access date: 07.07.2020.}.

Paths extraction is implemented in Python by using PyGraphBLAS.
Since lazy evaluation is not natural for Python, we cap the maximal number of paths to extract in the implementation.

\section{Evaluation}

This section describes the methodology and answers the following research questions.

\begin{enumerate}
    \item Does fusion via distillation give any benefits at the software and hardware levels?
    \item What are the properties of the generated hardware?
    \item Does the generated hardware outperform software implementations?
\end{enumerate}

\subsection{Methodology}

Our focus is on creating a basis for future research and experiments, thus we make our experiments as much reproducible as possible\footnote{\url{https://github.com/sedwards-lab/fhw/tree/sparse-linear-algebra-distillation/examples/QTreeBenchmarks/diploma/verilog-bool-no-nnz-inlined} (online; accessed:
2022-06-07) Here one could find all the results. For each benchmark all statistics are specified: matrix names, their sizes, collected metrics for both hardware and software benchmarks.}. We benchmarked the following list of chained functions. The choice is prompted by the current state of the distiller: at the moment, it does not successfully distill matrix multiplication. However, the functions are still practical enough, for example, chained addition could be seen in Luby's maximal independent set algorithm and clearly describe the applicability of the proposed approach.

\begin{itemize}
    \item \mintinline{Haskell}{mAdd (==False) (||) (mAdd (==False) (||) m1 m2) m3}
    \item \mintinline{Haskell}{mask (mAdd (== False) (||) m2 m3) (m1)}
    \item \mintinline{Haskell}{map (==Zero) (to_nat) (mAdd (==False) (||) m1 m2}
    \item \mintinline{Haskell}{map (==Zero) (to_nat) (kron (==False) (&&) m1 m2}
\end{itemize}

Above, \mintinline{Haskell}{Zero} and \texttt{to\_nat} are corresponding definitions for Peano arithmetics, since the \texttt{.pot} language does not have any primitives. For the same reason, we operated with boolean matrices. Such functions could be abstracted with free variables and then instantiated in the emitted Haskell code. However, to get maximum from distillation, we provided all the information about the functions. 

For these functions, we compared the execution time of distilled and not distilled hardware generated circuits, execution time of original and distilled Haskell code and reference \textit{Suite Sparse}\footnote{\url{https://github.com/DrTimothyAldenDavis/GraphBLAS} (online; accessed:
2022-06-07), Suite Sparse library sources.}\textsuperscript{,}\footnote{The library also uses different variations of coordinate formats (opaque to the user) and not a quadtree representation.} variants of these functions in C\texttt{++}. Note that SuiteSparse does not support recursive data types, thus only the first two function chains were implemented in SuiteSparse (since Peano number is essentially a linked list). We did not replace Peano numbers with integers, so our experiments could be interpreted easier. For hardware experiments we collected execution time and the number of memory writes and reads, to access how well fusion is performed. For software experiments we only measured the execution time. Also note that we measured only the time, required to execute the lines above, not including any IO, required to get and evaluate function arguments. But in hardware benchmarks we measured the time required to pass arguments into the circuit's memory, because such IO is inevitable. It is tricky to make such measures in Haskell due to laziness, thus the programs were compiled with \texttt{--fno-full-laziness} to turn off memoization. Also all the arguments were forced to normal form via \texttt{force} and \texttt{evaluate}. Haskell programs were compiled\footnote{GHC 8.10.4.} with \texttt{-O2 --fno-full-laziness} and Suite Sparse was compiled with default flags and linked as a shared library to C\texttt{++} code.

We took matrices from SuiteSparse matrix collection with sizes ranging from \texttt{64x64} to \texttt{512x512}. We limited ourselves with such sizes due to the fact that this is the maximum sizes that fit into \texttt{bram} with $2^{16}$ address space. Such number of \texttt{bram} blocks is available only on really expensive FPGA boards, thus in practice sizes would be smaller to achieve better utilization. Once again, it models the situation when data fits into the cache, since \texttt{bram} in our circuits will operate as a cache in real application.

\subsection{Experiments}

Table~\ref{tab:bench_results} shows the results of all execution time benchmarks. To evaluate execution time for hardware simulation, implementation stage was performed to assess the maximum frequency of FPGA device used for synthesis and implementation, and the number of execution cycles was multiplied by the number of nanoseconds a clock cycle takes. The frequencies were equal within the same benchamark set, i.e., frequency was not affected by distillation. We used \texttt{xcu250figd2104-2L} device\footnote{\url{https://www.xilinx.com/products/boards-and-kits/alveo/u250.html}  (online; accessed:
2022-06-07)} for synthesis and implementation stages. It is not really a casual and affordable chip, but it contains enough \texttt{bram} for our evaluation to see scalability. In the table a median across several benchmarks is shown. 

As it could be seen, distillation steadily increases performance: up to 2x speedup for hardware simulation and up to 3x for software benchmarks. The results are maintained within the borders of the corresponding confidence interval and the borders are not shown for brevity. Hardware speedup is lower due to the different execution semantics, dataflow is not reduction-based and distillation is a reduction-based transformation. Note that generated hardware appears to be less performant than both Haskell and C\texttt{++}, which a bit contradicts the results from~\cite{oldfhw}. For hardware benchmarks \texttt{time (IO)} shows the execution time including the time needed to transfer the data though the arguments, \texttt{time (no IO)} does not include it in its turn. It could be seen that not all the benchmarks are computationally extensive enough to cover memory transferring costs, but for more complex examples the ratio would be better. Since we basically transfer the matrices node by node from a file in the testbench, we have probably the lowest possible latency, and in practice it would be higher if reading from DDR, however the bandwidth could be increased. Noticeably, running times for \texttt{mMaskAdd} for C\texttt{++} and distilled Haskell are similar, which shows that fusion really provides some extra performance: SuiteSparse at the moment does not implement any fusion.

Table~\ref{tab:mem_results} summarizes the ratios between distilled and not distilled hardware circuits memory reads and writes. Since in general case distillation removes extra pattern matching, essentially it saves memory reads and writes. The eventual number of memory reads and writes is implementation dependent, thus the table shows what share of speedup is prompted by saving memory operations. Distillation successfully reduces the number of memory accesses, about 15\% on average. \texttt{mMapKron} has a bit higher ratio due to the fact that \texttt{Nat} numbers require additional memory accesses, since the type is recursive. It could also be seen that a major part of speedups is attributed to saved memory accesses. 

Finally, table~\ref{tab:resource_util} shows device resources utilization ratios between distilled and not distilled hardware circuits and frequencies. Columns are device primitives: registers, lookup tables, \texttt{bram} blocks or multiplexers. Utilization for both types of circuits is below 1\% of available resources on the device, except for the memory. Memory blocks utilization is about 30\% (since we choose larger \texttt{brams} to store larger matrices). Apart from that, distilled circuits could have both higher and lower utilization. Since the hardware generation is primarily syntax-directed it follows from the distilled program structure. For example, distillation might glue two recursive functions into one (in that case, memory utilization would be lower, because each cluster of mutually recursive functions possesses its own heap) or make more recursive functions than in the original program. The frequencies are the same, however, they possibly could be made better with more intelligent buffer allocation.

\subsection{Discussion}
Answering the research questions above.

\begin{enumerate}
    \item Fusion gives significant benefits, however at the hardware level the benefits are a bit smaller since hardware semantics is not reduction based. The benefits at the hardware level are mostly determined by the reduced number of memory accesses (each access takes 2 clock cycles). Notably, distilled Haskell implementation of \texttt{mMaskAdd} has similar performance with C\texttt{++}. 
    \item Device utilization is low, but such circuits could be copied on the same device to provide better utilization and higher parallelism. Resource utilization might be both better and worse after distillation, depending on the transformed program itself since translation is syntax-directed. Frequency could be increased by more intelligent buffering strategy.
    \item Although we use low-latency design with \texttt{bram}s that take 2 clock cycles per request and transfer matrices from files, which does not have any latency in simulation, we get slower execution time than Haskell and C\texttt{++} counterparts. It could be partly due to excessive buffering performed by FHW at the moment. Also there is no pipelining for recursive calls, i.e. only one set
of function argument tokens are allowed to enter a tail-recursive function call until a result is finally generated. Further CPS transformation hinders parallelization, which could be made more explicit with SSA. Some other optimizations exist that may significantly influence the performance. Also, since device utilization is about 1\%, such circuits could be copied on one device and provide more parallelism. A more detailed discussion could be found at~\cite{Edwards2019FHWP}.
\end{enumerate}

Distillation clearly showed its applicability to optimization of sparse linear algebra routines and notably it still could be combined with other techniques, like rewrite rules to achieve better results. High-level synthesis has a room for improvements by increasing pipelining, parallelism and frequency and the generated hardware could be improved from usability perspective: a support for arbitrary sized matrices is desirable. Thus we will focus on these directions. Probably a better solution would be to embed \texttt{.pot} language into e.g. Haskell to leverage its type system (to be able to use some rewrite rules as well), and add support for primitive types and parallel primitives to be able to conduct a more scalable comparison with SuiteSparse (since SuiteSparse is multithreaded). For such embedding different execution models could be implemented, including hardware synthesis, for which SSA form of GRIN looks promising, as well as extra optimizations shipped with GRIN. For hardware synthesis, an interesting direction is achieving predictable results in hardware from certain modifications in software. This property partly holds for the current approach, since the translation is syntax- directed. More information on this could be found at~\cite{predict}.

\pagebreak

\begin{table}[t]
\scriptsize
\centering
\caption*{mAddAdd}
\begin{tabular}{|c|c|c|c|c|c|c|c|c|c|} 
\hline
\rowcolor{LightBlue}
\multicolumn{3}{|c|}{Matrices dimensions} & Haskell & Haskell (distilled) & \multicolumn{2}{c|}{FHW} & \multicolumn{2}{c|}{FHW (distilled)} & {C\texttt{++}}\\
% \rowcolor{LightBlue}
\hline
m1 & m2 & m3 & time & time & time (no IO) & time (IO) & time (no IO) & time (IO) & time \\ 
\hline
64 & 64 & 64 & 29 us & 20 us & 76 us & 170 us & 64 us & 158 us & 14 us\\ 
128 & 128 & 128 & 94 & 79 & 146 & 476 & 134 & 469 & 30 \\
256 & 256 & 256 & 123 & 103 & 202 &  681 & 168 & 662 & 44\\
512 & 512 & 512 & 219 & 143 & 474 & 1192 & 375 & 1093 & 49\\
\hline
\end{tabular}

\caption*{mMaskAdd}
\begin{tabular}{|c|c|c|c|c|c|c|c|c|c|} 
\hline
\rowcolor{LightBlue}
\multicolumn{3}{|c|}{Matrices dimensions} & Haskell & Haskell (distilled) & \multicolumn{2}{c|}{FHW} & \multicolumn{2}{c|}{FHW (distilled)} & {C\texttt{++}}\\
% \rowcolor{LightBlue}
\hline
m1 & m2 & m3 & time & time & time (no IO) & time (IO) & time (no IO) & time (IO) & time \\ 
\hline
64 & 64 & 64 & 10 us & 7 us & 64 us & 133 us & 46 us & 111 us & 18 us\\ 
128 & 128 & 128 & 38 & 30 & 118 & 322 & 75 & 292 & 33 \\
256 & 256 & 256 & 48 & 42 & 168 &  498 & 104 & 456 & 46\\
512 & 512 & 512 & 126 & 76 & 400 & 762 & 300 & 729 & 65\\
\hline
\end{tabular}

\caption*{mMapAdd}
\begin{tabular}{|c|c|c|c|c|c|c|c|c|c|} 
\hline
\rowcolor{LightBlue}
\multicolumn{3}{|c|}{Matrices dimensions} & Haskell & Haskell (distilled) & \multicolumn{2}{c|}{FHW} & \multicolumn{2}{c|}{FHW (distilled)} & {C\texttt{++}}\\
% \rowcolor{LightBlue}
\hline
m1 & m2 & m3 & time & time & time (no IO) & time (IO) & time (no IO) & time (IO) & time \\ 
\hline
64 & 64 & --- & 45 us & 37 us & 189 us & 253 us & 137 us & 202 us & ---\\ 
128 & 128 & --- & 162 & 105 & 524 & 685 & 397 & 579 & --- \\
256 & 256 & --- & 312 & 216 & 1047 &  1360 & 680 & 986 & ---\\
512 & 512 & --- & 436 & 273 & 1346 & 1776 & 900 & 1330 & ---\\
\hline
\end{tabular}

\caption*{mMapKron}
\begin{tabular}{|c|c|c|c|c|c|c|c|c|c|} 
\hline
\rowcolor{LightBlue}
\multicolumn{3}{|c|}{Matrices dimensions} & Haskell & Haskell (distilled) & \multicolumn{2}{c|}{FHW} & \multicolumn{2}{c|}{FHW (distilled)} & {C\texttt{++}}\\
% \rowcolor{LightBlue}
\hline
m1 & m2 & m3 & time & time & time (no IO) & time (IO) & time (no IO) & time (IO) & time \\ 
\hline
2 & 64 & --- & 64 us & 36 us & 212 us & 242 us & 94 us & 125 us & ---\\ 
2 & 128 & --- & 137 & 68 & 434 & 502 & 199 & 266 & --- \\
2 & 256 & --- & 364 & 126 & 1004 &  1188 & 449 & 636 & ---\\
4 & 128 & --- & 302 & 94 & 694 & 763 & 330 & 401 & ---\\
\hline
\end{tabular}



\caption{Execution time}
\label{tab:bench_results}

\end{table}
\begin{table}[h]
\scriptsize
\begin{minipage}{0.45\linewidth}
\centering
\caption*{mAddAdd}
\begin{tabular}{|c|c|c|c|c|c|c|} 
\hline
\rowcolor{LightBlue}
\multicolumn{3}{|c|}{Matrices dimensions} & \multicolumn{2}{c|}{Ratio ($\frac{FHW}{FHW_{distilled}}$)}\\
% \rowcolor{LightBlue}
\hline
m1 & m2 & m3 & writes & reads\\ 
\hline
64 & 64 & 64 & 1.10 & 1.15\\ 
128 & 128 & 128 & 1.02 & 1.05\\
256 & 256 & 256 & 1.03 & 1.06\\
512 & 512 & 512 & 1.10 & 1.16\\
\hline
\end{tabular}
\end{minipage}
\begin{minipage}{0.45\linewidth}
\centering
\caption*{mMaskAdd}
\begin{tabular}{|c|c|c|c|c|c|c|} 
\hline
\rowcolor{LightBlue}
\multicolumn{3}{|c|}{Matrices dimensions} & \multicolumn{2}{c|}{Ratio ($\frac{FHW}{FHW_{distilled}}$)}\\
% \rowcolor{LightBlue}
\hline
m1 & m2 & m3 & writes & reads\\ 
\hline
64 & 64 & 64 & 1.13 & 1.26\\ 
128 & 128 & 128 & 1.06 & 1.11\\
256 & 256 & 256 & 1.08 & 1.09\\
512 & 512 & 512 & 1.10 & 1.16\\
\hline
\end{tabular}
\end{minipage}
\begin{minipage}{0.45\linewidth}
\centering
\caption*{mMapAdd}
\begin{tabular}{|c|c|c|c|c|c|c|} 
\hline
\rowcolor{LightBlue}
\multicolumn{3}{|c|}{Matrices dimensions} & \multicolumn{2}{c|}{Ratio ($\frac{FHW}{FHW_{distilled}}$)}\\
% \rowcolor{LightBlue}
\hline
m1 & m2 & m3 & writes & reads\\ 
\hline
64 & 64 & --- & 1.10 & 1.21\\ 
128 & 128 & --- & 1.07 & 1.14\\
256 & 256 & --- & 1.07 & 1.19\\
512 & 512 & --- & 1.10 & 1.21\\
\hline
\end{tabular}
\end{minipage}
\hfill
\begin{minipage}{0.45\linewidth}
\centering
\caption*{mMapKron}
\begin{tabular}{|c|c|c|c|c|c|c|} 
\hline
\rowcolor{LightBlue}
\multicolumn{3}{|c|}{Matrices dimensions} & \multicolumn{2}{c|}{Ratio ($\frac{FHW}{FHW_{distilled}}$)}\\
% \rowcolor{LightBlue}
\hline
m1 & m2 & m3 & writes & reads\\ 
\hline
2 & 64 & --- & 1.71 & 1.88\\ 
2 & 128 & --- & 1.72 & 1.87\\
2 & 256 & --- & 1.65 & 1.83\\
4 & 128 & --- & 1.81 & 1.91\\
\hline
\end{tabular}
\end{minipage}

\caption{Memory accesses}
\label{tab:mem_results}
\end{table}

\begin{table}[h]
\scriptsize
\centering
\begin{tabular}{|l|c|c|c|c|c|c|c|c|c|} 
\hline
\rowcolor{LightBlue}

{Benchmark} & \multicolumn{8}{c|}{Ratio (${\frac{FHW}{FHW_{distilled}}}$)} & {Frequency}\\
\hline
{} & FDRE & LUT3 & LUT6 & LUT5 & LUT4 & LUT2 & RAMB36E2 & MUXF7 & {} \\
% \rowcolor{LightBlue}
\hline
mAddAdd & 0.3 & 0.3 & 0.3 & 0.5 & 0.3 & 0.3 & 0.5 & 0.5 & 200 MHz\\ 
mMaskAdd & 0.5 & 0.5 & 0.7 & 0.4 & 0.7 & 0.5 & 0.7 & 0.6 & 200 MHz\\
mMapAdd & 1 & 0.9 & 0.9 & 1.2 & 1 & 1.1 & 1.1 & 1.2 & 200 MHz\\
mMapKron & 1.5 & 1.5 & 1.3 & 2 & 2 & 1.8 & 1.4 & 1.7 & 200 MHz\\
\hline
\end{tabular}
\caption{Resource utilization}
\label{tab:resource_util}
\end{table}
\pagebreak

\section{Conclusion and Future Work}

We present !!!

Our evaluation shows that !!!

First direction for future research is a more detailed CFPQ algorithms investigation.
We should do More evaluation on sparse matrices on GPGPUs.

Also it is nesessary to implement and evaluate solutions for graphs which is not fit in RAM.
There is a set of technics for huge matrices multiplication.
Is it possible to dopt it for CFPQ

Another direcion is a dataset improvement.
More data.
More grammars/queries.


%\section*{Acknowledgment}
%
%The preferred spelling of the word ``acknowledgment'' in America is without 
%an ``e'' after the ``g''. Avoid the stilted expression ``one of us (R. B. 
%G.) thanks $\ldots$''. Instead, try ``R. B. G. thanks$\ldots$''. Put sponsor 
%acknowledgments in the unnumbered footnote on the first page.


\bibliographystyle{./IEEEtran}
\bibliography{./Sparse_Boolean_Algebra_on_GPGPU}

\end{document}
