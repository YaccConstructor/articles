\section{Evaluation}

% Evaluation of the proposed implemenation(s).

% What we need:
% - idea of the evaluation
% - what we want to show and measure
% - time performance, memory consumption, ops/per sec
% - what data we need?
% - very sparse, different distributions

We evaluate the applicability of the proposed libraries for analysis of some real-world graph data.
The experiments are designed as a computational tasks, that arise as stand-alone or intermediate steps
in the solving of practical problems.  

\textbf{!!! Target machine description !!!}

For performance evaluations, we selected N various square matrices which are widely used for sparse matrices benchmarks
from the Sparse Matrix Collection at University of Florida\footnote{T. Davis. The SuiteSparse Matrix Collection (the University of Florida Sparse Matrix Collection). Home page: \url{https://sparse.tamu.edu/}. Access date: 23.01.2021.}.
The name and size of the matrix data are summarized in the table~\ref{table:sparse_matrices}. 



% https://sparse.tamu.edu/LAW/hollywood-2009 
% https://sparse.tamu.edu/SNAP/soc-LiveJournal1
% https://sparse.tamu.edu/LAW/indochina-2004
% https://sparse.tamu.edu/SNAP/roadNet-CA
% https://sparse.tamu.edu/DIMACS10/belgium_osm

{\setlength{\tabcolsep}{0.3em}
\begin{table}
\centering
{
\caption{Matrix data}
\label{table:sparse_matrices}
\scriptsize
\rowcolors{2}{black!2}{black!10}
\begin{tabular}{|l|c|c|c|c|c|}
\hline
Matrix          & Size       & Non-zero   & Nnz/row   & Max nnz/row & Nnz of $M^2$ \\
\hline
\hline
first           & a          & b          & c         & d            & e              \\
second          & a          & b          & c         & d            & e              \\
\hline
\end{tabular}
}
\end{table}
}