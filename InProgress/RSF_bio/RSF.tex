\documentclass[12pt]{article}  % standard LaTeX, 12 point type

\usepackage{geometry}

\usepackage{amsmath}
\usepackage{amsfonts,latexsym}
\usepackage{amsthm}
\usepackage{amssymb}
\usepackage[utf8]{inputenc} % Кодировка
\usepackage[english,russian]{babel} % Многоязычность
\usepackage{verbatim}
\usepackage{longtable}
\usepackage{csvsimple}
\usepackage[toc,page]{appendix}
\usepackage{booktabs}

\usepackage{float}
\usepackage{array}
\usepackage{multirow}
\usepackage{caption}
\usepackage{graphicx}
\usepackage{ucs}
\usepackage{rotating}
\usepackage{pdflscape}
\usepackage{afterpage}
\usepackage{capt-of}% or use the larger `caption` package
\usepackage{url}

% unnumbered environments:

\theoremstyle{remark}
\newtheorem*{remark}{Remark}
\newtheorem*{notation}{Notation}
\newtheorem*{note}{Note}

\setlength{\parskip}{5pt plus 2pt minus 1pt}
\newcolumntype{C}{>{\centering\arraybackslash}p{1.3cm}}
\graphicspath{{pics/}}

\title{Использование формальных грамматик для анализа вторичной структуры генмных и протеомных  последовательностей}
\author{Семён Григорьев}
\date{\today}

\begin{document}

\newgeometry{left=0.8in,right=0.8in,top=1in,bottom=1in}

\maketitle

\section{Сведения о проекте}

\subsection{Название проекта}

\textbf{ru}\\
Использование формальных грамматик для анализа вторичной структуры генмных и протеомных  последовательностей
\\
\\
\textbf{en}\\

\subsection{Направление из Стратегии НТР РФ}

Н3 Переход к персонализированной медицине, высокотехнологичному здравоохранению и технологиям здоровьесбережения, в том числе за счет рационального применения лекарственных препаратов (прежде всего антибактериальных)

\subsection{Обоснование соответствия тематики проекта направлению из Стратегии НТР РФ: необходимо кратко сформулировать научную проблему (проблемы) и конкретные задачи в рамках выбранного направления, решению которых будет посвящен проект, обосновать соответствие проекта направлению}
\textbf{ru}\\
Анализ (детектирование микроорганизмов)

Анализ генетической информации

Поиск новых лекарственных препаратов (антибактериальных в том числе)
\\
\\
\textbf{en}\\


\subsection{Ключевые слова (приводится не более 15 терминов)}

\textbf{ru}\\
Формальные граммтики, синтаксический анализ, параллельные алгоритмы, вторичная структура, РНК, генмноые последовательности, белки, протеомные последовательности, метагеномная сборка.
\\
\\
\textbf{en}\\


\subsection{Аннотация проекта }
%(объемом не более 2 стр.; в том числе кратко – актуальность решения указанной выше научной проблемы и научная новизна)
\textbf{ru}\\
Проект посвящён исследованию применимости формальных граммтик для анализа вторичной структуры различных биологических последовательностей, например, геномных или протеомных.

Применение результатов теории формальных языков для анализа биологических последовательностей исследуется давно, однако появились новые результаты и требуется анализ.

Недостаточность современных методов: неточность, ресурсоёмкость.

Требуется применение новых классов граммтик, разработка новых алгоритмов.
И даже подходов.

Поиск новых организмов, улучшение предсказания функций белков и т.д.
\\
\\
\textbf{en}\\


\subsection{Ожидаемые результаты и их значимость}
%(указываются результаты, их научная и общественная значимость (соответствие предполагаемых результатов мировому уровню исследований, возможность практического использования предполагаемых результатов проекта в экономике и социальной сфере))

\textbf{ru}\\
Теоретические результаты --- формальные методы описания и анализа вторичной структуры.
Классы граммтик и конкртеные граммтики для конкретных задач

Применение на практике: классификация, поиск
\\
\\
\textbf{en}\\


\section{Содержание проекта}

\subsection{Научная проблема, на решение которой направлен проект}

\textbf{ru}\\
Формальные методы описания и анализа вторичной структуры

Способы и методы анализа вторичной структуры биологических поледовательностей.
Особенности задачи --- особенности алгоритмов.
Большой объём данных --- дополнительные требования к алгоритмам: параллельные вычисления.
\\
\\
\textbf{en}\\



\subsection{Научная значимость и актуальность решения обозначенной проблемы}

\textbf{ru}\\
Формальные методы описания

Алгоритмы стнтаксического анализа.

Классификация, обнаружение и т.д.
\\
\\
\textbf{en}\\



\subsection{Конкретная задача (задачи) в рамках проблемы, на решение которой направлен проект, ее масштаб и комплексность}

\textbf{ru}\\
Изучение применимости новых классов грмматик.

Предложение конкретных граммтик для конкретных важных задач (поиск маркерных последовательностей)

Разработка параллельных алгоритмов синтаксического анализа

Разработка алгоритмов и решений для анализа вторичной структуры с использованием методов теории формальных языков и машинного обучения.
\\
\\
\textbf{en}\\


\subsection{Научная новизна исследований, обоснование достижимости решения поставленной задачи (задач) и возможности получения запланированных результатов}

\textbf{ru}\\
Текстовый анализ

Грамматики, описывающе первичную струкруту.

Вторичная структура --- через энергии и т.д.

Сложные элементы вторичной структуры.

Существование наработок, решающих демонстрационные задачи.
\\
\\
\textbf{en}\\


\subsection{Современное состояние исследований по данной проблеме, основные направления исследований в мировой науке и научные конкуренты}

\textbf{ru}\\
Применение конъюнктивных граммтик исследовано крайне слабо, но активно развивается (2? работы).

ФОрмальные граммтики для белков прям сейчас (Витольд).

Формальные граммтики для вторичной структуры (Девушка с конфы)
\\
\\
\textbf{en}\\



\subsection{Предлагаемые методы и подходы, общий план работы на весь срок выполнения проекта и ожидаемые результаты }
%(объемом не менее 2 стр.; в том числе указываются ожидаемые конкретные результаты по годам; общий план дается с разбивкой по годам)

\textbf{ru}\\
Предлагается построить алгоритмы для синтаксического анлиза,

Сильно неоднозначные гаммтики, что не характерно для языков програмирования, для которых разрабатывались многие алгоритмы.


Предполагается применить для анализа РНК и белков.
Подбор граммтик


2019-2020


2020-2021
\\
\\
\textbf{en}\\


\subsection{Имеющийся у руководителя проекта научный задел по проекту, наличие опыта совместной реализации проектов}

\textbf{ru}\\
Синтаксический анализ, статья по биологам. Выступления на биата.
\\
\\
\textbf{en}\\


\subsection{Перечень оборудования, материалов, информационных и других ресурсов, имеющихся у руководителя проекта для выполнения проекта }
\textbf{ru}\\
\\
\\
\textbf{en}\\


\subsection{План работы на первый год выполнения проекта}

\textbf{ru}\\
Эксперименты с 16s и химерами. Эксперименты с белками. Конъюнктивные граммтики. Работа над агоритмами синтаксического анализа
\\
\\
\textbf{en}\\

\subsection{Ожидаемые в конце первого года конкретные научные результаты}
%(форма изложения должна дать возможность провести экспертизу результатов и оценить степень выполнения заявленного в проекте плана работы)

\textbf{ru}\\
Граммтики

Алгоритм.

Парсер.
\\
\\
\textbf{en}\\

\subsection{Перечень планируемых к приобретению руководителем проекта за счет гранта Фонда оборудования, материалов, информационных и других ресурсов для выполнения проекта}
%(в том числе – описывается необходимость их использования для реализации проекта)

\textbf{ru}\\
ГПУ!
\\
\\
\textbf{en}\\

\end{document}
