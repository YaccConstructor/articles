%15 min preso!
\documentclass[xcolor=table,aspectratio=169]{beamer}
\usepackage{beamerthemesplit}
\usepackage{wrapfig}
\usetheme{SPbGU}
\usepackage{pdfpages}
\usepackage{amsmath}
\usepackage{cmap}
\usepackage[T2A]{fontenc}
\usepackage[utf8]{inputenc}
\usepackage[english]{babel}
\usepackage{indentfirst}
\usepackage{amsmath}
\usepackage{tikz}
\usepackage{multirow}
\usepackage[noend]{algpseudocode}
\usepackage{algorithm}
\usepackage{algorithmicx}
\usepackage{fancyvrb}
\usepackage{hyperref} 
\usetikzlibrary{calc}
\usetikzlibrary{shapes,arrows}
\usetikzlibrary{arrows,automata}
\usetikzlibrary{positioning}

\usepackage{tabularx}
\newcolumntype{Y}{>{\raggedleft\arraybackslash}X}

\renewcommand{\thealgorithm}{}

\newtheorem{mytheorem}{Theorem}
\renewcommand{\thealgorithm}{}

\newcommand{\tikzmark}[1]{\tikz[overlay,remember picture] \node (#1) {};}
\def\Put(#1,#2)#3{\leavevmode\makebox(0,0){\put(#1,#2){#3}}}

\newcommand{\ltz}{$< 1$}


\tikzset{
    state/.style={
           rectangle,
           rounded corners,
           draw=black, very thick,
           minimum height=2em,
           inner sep=2pt,
           text centered,
           },
}

\beamertemplatenavigationsymbolsempty

\title[Incremental parsing]{Incremental Parsing with Error Recovery}
\institute[RRI SPb]{Saint Petersburg Research Center}

% То, что в квадратных скобках, отображается в левом нижнем углу.
\author[Semyon Grigorev]{Semyon Grigorev}

\date{March 16, 2023}

\begin{document}
{
\begin{frame}[fragile]
  \begin{table}
  \centering
  %\includegraphics[height=1.5cm]{pictures/SPbGU_Logo.png}
  \end{table}
  \titlepage
\end{frame}
}

\begin{frame}[fragile]
  \frametitle{Mentor: Semyon Grigorev}
  \begin{itemize}
    \item PhD, static code analysis, 2016
    \item 10 years in JetBrains
    \item 9 years in St Petersburg University (Associate professor) 
    \item Research area: syntax guided data analysis 
    \item s.v.grigorev@spbu.ru
  \end{itemize}
\end{frame}

\begin{frame}[fragile]
  \frametitle{Incremental Parsing With Error Recovery}
  
  \begin{itemize}
    \item Code editor for IDEs
    \begin{itemize}
      \item Incremental Parsing
        \begin{itemize}
          \item Fast reaction on code changes
        \end{itemize}
      \pause  
      \item Error recovery
      \begin{itemize}
        \item Small number of false syntax error reports        
        \item Handling of partially correct code (during editing)
        \item Fixes proposals
      \end{itemize}
    \end{itemize}
    \pause
    \item Example: Tree-Sitter
    
  \end{itemize}
\end{frame}


\begin{frame}[fragile]
  \frametitle{Prject Details}

  \begin{itemize}
    \item Goal: design and development of GLL-based incremental parsing algorithm with error recovery
    \begin{itemize}
      \item Based on existing prototype in F\#
    \end{itemize}
    \pause
    \item Tasks
    \begin{itemize}
      \item Support incremental paring in existing prototype 
      \item Optimize performance of existing prototype
      \item Development of evaluation evaluation environment
      \begin{itemize}
        \item Grammar of real-world language
        \item Cases for incremental parsing
        \item Cases for error recovery
      \end{itemize} 
    \end{itemize}
    
  \end{itemize}
\end{frame}

\begin{frame}[fragile]
  \frametitle{Requrements}
  \begin{itemize}
    \item Strong background in formal language theory and parsing algorithms ((G)LR, (G)LL)
    \item Strong background in F\# or similar programming languages (e.g. Ocaml)
    \item Experience in nontrivial algorithms design, implementation, optimization.
    \item Deep knowledge in data structures internals (with respect to .NET and F\#) 
  \end{itemize}
 
\end{frame}

\end{document}
