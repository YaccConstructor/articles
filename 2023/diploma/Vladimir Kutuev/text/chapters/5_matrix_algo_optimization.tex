\section{Оптимизация реализации матричного алгоритма}

В pygraphblas при умножении двух матриц необходимо передать методу умножения полукольцо, в котором происходит сложение и умножение элементов матрицы. В реализации алгоритмов поиска путей с контекстно-свободными ограничениями используются разреженные булевы матрицы, поэтому для умножения использовалось полукольцо BOOL.LOR\_LAND: домен --- булевы значения $\{true$, $false\}$, сложение --- LOR (дизъюнкция), умножение --- LAND (конъюнкция). Однако разреженность матриц меняет множество значений, которые может принимать элемент матрицы: $true$, $false$ и <<нет значения>>. При этом в реализации матричного алгоритма элементы матрицы никогда не принимают значения $false$. Изначально во всех матрицах нет значений, затем некоторые элементы матрицы в соответствии с простыми правилами грамматики инициализируются значением $true$. Затем в матрицу могут добавляться только значения $true$, если была найдена новая пара вершин, между которыми существует путь. Таким образом, полукольцо BOOL.LOR\_LAND является избыточным для реализации данного алгоритма. Для избавления от этой избыточности можно воспользоваться полукольцом с операциями ANY (выбирает любой из переданных аргументов) в качестве сложения и PAIR (возвращает 1 в полукольце, если оба операнда --- присутствующие в матрице элементы) в качестве умножения. В силу того, как заполняется матрица в данном алгоритме, использование полукольца BOOL.ANY\_PAIR при умножении матриц не изменит результат умножения, но операции ANY и PAIR проще и выполняются быстрее, чем LOR и LAND.

Данная оптимизация была применена в реализации матричного алгоритма в CFPQ\_PyAlgo\footnote{\url{https://github.com/FormalLanguageConstrainedPathQuerying/CFPQ_PyAlgo}, пользователь vkutuev. Дата посещения 22.05.2023}.