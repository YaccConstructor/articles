\section*{Введение}

Статический анализ играет важную роль в задаче поиска ошибок в коде. Однако многие из методов статического анализа основаны на сопоставлении участков кода с некоторым шаблоном: если участок кода соответствует шаблону, считается, что он содержит ошибку. Такой подход может приводить как к пропуску существующих ошибок, так и выдаче ложных предупреждений. Алгоритмы статического анализа, которые выделяют ошибки на основе различных свойств программы, например, контекстов вызова и потока управления, позволяют обнаруживать больше истинных ошибок и сообщать меньше ложных предупреждений.

Многие виды статического анализа могут быть сформулированы как задача контекстно-свободной (КС) достижимости в графе~\cite{context_sensitive_points_to_analysis_for_java, dataflow_analysis_via_graph_reachability, precise_interprocedural_dataflow_analysis, program_analysis_via_graph_reachability}. Один из примеров --- анализ псевдонимов (анализ указателей). Он позволяет обнаруживать использование освобождённой памяти, взаимные блокировки и обращение к выделенной памяти через тип с несоответствующим размером~\cite{alias_analysis}. Ещё один пример --- Points-to анализ, учитывающий поля, особенностью которого являются большие грамматики, используемые для анализа.

Существует множество алгоритмов, решающих задачу КС-достижимости~\cite{cfpq_algo_1, cfpq_algo_2, cfpq_algo_3, rustam_algorithm}. Среди них можно выделить алгоритмы, основанные на операциях линейной алгебры, так как такие операции хорошо поддаются распараллеливанию. В исследовании Никиты Мишина и др.~\cite{eval_cfpq} был произведён сравнительный анализ времени работы нескольких реализаций матричного алгоритма Рустама Азимова~\cite{rustam_algorithm}, основанных на различных специализированных матричных библиотеках. Однако графы, на которых проводилось исследование, значительно меньше получаемых по исходному коду для статического анализа. 

В данной работе предлагается изучить различные алгоритмы КС-достижимости в графах и сравнить их производительность на больших графах, полученных по реальным программам.
