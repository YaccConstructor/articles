\section*{Заключение}
В ходе работы были достигнуты следующие результаты.
\begin{itemize}
    \item Предложена модификация тензорного алгоритма для Points-to анализа, учитывающего поля, показавшая наилучшее время работы на небольших графах среди алгоритмов, основанных на операциях линейной алгебры. Однако эксперименты показали, что из-за большого потребления памяти алгоритм практически неприменим для данного вида анализа.
    \item Оптимизирована реализация матричного алгоритма из библиотеки CFPQ\_PyAlgo, эффективность оптимизации экспериментально проверена.
    \item Проведены замеры производительности реализаций алгоритмов КС-достижимости, которые показали, что алгоритмы, основанные на операциях линейной алгебры, достаточно эффективны для анализа псевдонимов, но из-за больших размеров грамматики малоэффективны для Points-to анализа, учитывающего поля.
\end{itemize}

Так как реализации алгоритма, основанного на произведении Кронекера, и его вариации плохо применимы для Points-to анализа, учитывающего поля, в силу очень большого объёма потребляемой памяти под пересечение графа и рекурсивного автомата, в дальнейшем можно рассмотреть следующие шаги по оптимизации матричного алгоритма для применения его к данному типу анализа.
\begin{itemize}
    \item Разработка инкрементальной версии матричного алгоритма, в которой полученные на предыдущих итерациях данные не будут пересчитываться в дальнейшем, что позволит сократить затрачиваемое на умножение матриц время.
    \item Специализация матричного алгоритма для данного анализа, не требующего перевода грамматики в ОНФХ и вычисления достижимости для нетерминала $FlowsTo$ непосредственно, что позволит значительно сократить количество перемножений матриц.
\end{itemize}
