% !TeX spellcheck = ru_RU
% !TEX root = vkr.tex

При проведении данной работы были достигнуты следующие результаты. 

\begin{itemize}
    \item Проведен обзор и выбрано два доступных аналога для проведения сравнительного анализа, а именно, тензорный алгоритм и алгоритм на основе языка Datalog. Последний является принятым базовым инструментом.
    \item Собран набор данных, состоящий из графов и регулярных запросов, для проведения экспериментального исследования. В него вошли графы собранные по RDF-данным и данным социальных сетей, а также множество популярных запросов.
    \item Спроектирован инструмент, который позволяет автоматизировать проведение экспериментов, поддерживает генерацию данных для запуска запросов на графах, получение множества начальных вершин и запуск различных типов запросов\footnote{Benchmark suite for RPQ evaluation:~\href{https://github.com/bahbyega/paths-benchmark}{https://github.com/bahbyega/paths-benchmark} (accessed: 20.05.2023)}.
    \item Проведено экспериментальное исследование алгоритма и сравнение с аналогами. В ходе анализа было выявлено, что подход к реализации алгоритма поиска путей с ограничениями в виде регулярных языков на основе поиска в ширину является жизнеспособным и показывает приемлемые результаты на реальных данных. 
    Более того, на RDF-данных алгоритм показывает кратный рост производительности относительно аналога, реализованного на Datalog, а также работает в среднем быстрее алгоритма, основанного на тензорном произведении. Тем не менее, анализ зависимости времени исполнения запроса к числу стартовых вершин выявил проблемы реализации алгоритма MSBFS для решения задачи достижимости от каждой стартовой вершины, что может быть исследовано в дальнейшей работе. 
    \end{itemize}

В дальнейшем работа может быть развита в следующих направлениях.
\begin{itemize}
    \item Расширение набора данных, на котором исследуется алгоритм. В него можно включить искусственно сгенерированные графы с определенными особенностями структуры, а также реальные графы из других прикладных областей.
    \item Интегрирование алгоритма с графовой базой данных, а также с языком запросов к графовой базе данных. Это позволит провести еще более полноценное сравнение алгоритма с аналогами, в которые можно будет включить системы графовых баз данных. 
\end{itemize}
\noindent 
