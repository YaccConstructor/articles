% !TEX TS-program = xelatex
% !BIB program = bibtex
% !TeX spellcheck = ru_RU

% About magic macroses see also
% https://tex.stackexchange.com/questions/78101/

\input{header.tex}
\input{header2.tex}

% Dependencies
\usepackage{totcount}
\usepackage{algorithm}% http://ctan.org/pkg/algorithms
\usepackage[noend]{algpseudocode}% http://ctan.org/pkg/algorithmicx
\usepackage{multirow}
\usepackage{nicematrix}
\usepackage{tikz}
\usetikzlibrary{fit}
\usetikzlibrary{arrows,automata}

% Custom
\theoremstyle{definition}
\newtheorem{rudefinition}{Определение}[section]
\newtheorem{example}{Example}[section]
\newtheorem{theorem}{Theorem}[section]
\newtheorem{proposition}[theorem]{Proposition}
\newtheorem{lemma}[theorem]{Lemma}
\newtheorem{corollary}[theorem]{Corollary}
\newtheorem{conjecture}[theorem]{Conjecture}
\newtheorem{note}[theorem]{Утверждение}
\newfontfamily{\cyrillicfonttt}{Liberation Mono}


\begin{document}

%% Если что-то забыли, при компиляции будут ошибки Undefined control sequence \my@title@<что забыли>@ru
%% Если англоязычная титульная страница не нужна, то ее можно просто удалить.
\filltitle{ru}{
    %% Актуально только для курсовых/практик. ВКР защищаются не на кафедре а в ГЭК по направлению, 
    %%   и к моменту защиты вы будете уже не в группе.
    chair              = {Кафедра системного программирования},
    group              = {17Б.11-мм},
    %% Макрос filltitle ненавидит пустые строки, поэтому обязателен хотя бы символ комментария на строке
    %% Актуально всем.
    title              = {Реализация алгоритма поиска путей в графовых базах данных через тензорное произведение на GPGPU},
    % 
    %% Здесь указывается тип работы. Возможные значения:
    %%   coursework - отчёт по курсовой работе;
    %%   practice - отчёт по учебной практике;
    %%   prediploma - отчёт по преддипломной практике;
    %%   master - ВКР магистра;
    %%   bachelor - ВКР бакалавра.
    type               = {bachelor},
    author             = {Орачев Егор Станиславович},
    % 
    %% Актуально только для ВКР. Указывается код и название направления подготовки. Типичные примеры:
    %%   02.03.03 <<Математическое обеспечение и администрирование информационных систем>>
    %%   02.04.03 <<Математическое обеспечение и администрирование информационных систем>>
    %%   09.03.04 <<Программная инженерия>>
    %%   09.04.04 <<Программная инженерия>>
    %% Те, что с 03 в середине --- бакалавриат, с 04 --- магистратура.
    specialty          = {09.03.04 <<Программная инженерия>>},
    % 
    %% Актуально только для ВКР. Указывается шифр и название образовательной программы. Типичные примеры:
    %%   СВ.5006.2017 <<Математическое обеспечение и администрирование информационных систем>>
    %%   СВ.5162.2020 <<Технологии программирования>>
    %%   СВ.5080.2017 <<Программная инженерия>>
    %%   ВМ.5665.2019 <<Математическое обеспечение и администрирование информационных систем>>
    %%   ВМ.5666.2019 <<Программная инженерия>>
    %% Шифр и название программы можно посмотреть в учебном плане, по которому вы учитесь. 
    %% СВ.* --- бакалавриат, ВМ.* --- магистратура. В конце --- год поступления (не обязательно ваш, если вы были в академе/вылетали).
    programme          = {СВ.5080.2017 <<Программная инженерия>>},
    % 
    %% Актуально только для ВКР, только для матобеса и только 2017-2018 годов поступления. Указывается профиль подготовки, на котором вы учитесь.
    %% Названия профилей можно найти в учебном плане в списке дисциплин по выбору. На каком именно вы, вам должны были сказать после второго курса (можно уточнить в студотделе).
    %% Вот возможные вариканты:
    %%   Математические основы информатики
    %%   Информационные системы и базы данных
    %%   Параллельное программирование
    %%   Системное программирование
    %%   Технология программирования
    %%   Администрирование информационных систем
    %%   Реинжиниринг программного обеспечения
    % profile            = {Системное программирование},
    % 
    %% Актуально всем.
    supervisorPosition = {Доцент кафедры информатики, к.\,ф.-м.\,н.},
    supervisor         = {С. В. Григорьев},
    % 
    %% Актуально только для практик и курсовых. Если консультанта нет, закомментировать или удалить вовсе.
    % consultantPosition = {должность ООО <<Место работы>> степень},
    % consultant         = {К.К. Консультант},
    %
    %% Актуально только для ВКР.
    reviewerPosition   = {Разработчик биоинформатического ПО, ЗАО “БИОКАД”},
    reviewer           = {А.С. Хорошев},
}

\filltitle{en}{
    chair              = {Software Engineering},
    group              = {17B.11-mm},
    title              = {Context-Free path querying by tensor product for graph databases on GPGPU},
    type               = {bachelor},
    author             = {Egor Orachev},
    % 
    %% Possible choices:
    %%   02.03.03 <<Software and Administration of Information Systems>>
    %%   02.04.03 <<Software and Administration of Information Systems>>
    %%   09.03.04 <<Software Engineering>>
    %%   09.04.04 <<Software Engineering>>
    %% Те, что с 03 в середине --- бакалавриат, с 04 --- магистратура.
    specialty          = {09.03.04 <<Software Engineering>>},
    % 
    %% Possible choices:
    %%   СВ.5006.2017 <<Software and Administration of Information Systems>>
    %%   СВ.5162.2020 <<Programming Technologies>>
    %%   СВ.5080.2017 <<Software Engineering>>
    %%   ВМ.5665.2019 <<Software and Administration of Information Systems>>
    %%   ВМ.5666.2019 <<Software Engineering>>
    programme          = {СВ.5080.2017 <<Software Engineering>>},
    % 
    %% Possible choices:
    %%   Mathematical Foundations of Informatics
    %%   Information Systems and Databases
    %%   Parallel Programming
    %%   System Programming
    %%   Programming Technology
    %%   Information Systems Administration
    %%   Software Reengineering
    % profile            = {Software Engineering},
    % 
    %% Note that common title translations are:
    %%   кандидат наук --- C.Sc. (NOT Ph.D.)
    %%   доктор ... наук --- Sc.D.
    %%   доцент --- docent (NOT assistant/associate prof.)
    %%   профессор --- prof.
    supervisorPosition = {C.Sc., docent},
    supervisor         = {S.V. Grigorev},
    % 
    % consultantPosition = {position at ``Company'', degree if present},
    % consultant         = {C.C. Consultant},
    % %
    reviewerPosition   = {Bioinformatics Software Engineer, BIOCAD},
    reviewer           = {A.S. Khoroshev},
}
\maketitle
\setcounter{tocdepth}{2}
\tableofcontents

\section*{Введение}
% !TeX spellcheck = ru_RU
% !TEX root = vkr.tex

Данные, представимые в виде графов, встречаются повсеместно: при анализе компьютерных и социальных сетей, в биоинформатике и статическом анализе кода~\cite{gb_math}. Графы, возникающие в этих областях, могут содержать миллионы узлов и ребер, поэтому существует необходимость в высокопроизводительных инструментах анализа больших графов. В связи с этим многообещающей становится идея использования графических ускорителей общего назначения --- GPGPU. Существующие решения уже сейчас доказывают, что использование GPGPU может повысить производительность алгоритмов анализа графов~\cite{cusha}~\cite{mGPU}, однако ценой служит более сложная модель программирования~\cite{blast}. \\ Иным является подход к организации вычислений над графами, основанный на стандарте GraphBLAS~\cite{gb_math}, который определяет базовые примитивы для построения графовых алгоритмов в терминах линейной алгебры. Свойством такого подхода является способность оперировать богатым набором графов различных типов с помощью небольшого набора матричных операций над полукольцами. Например, умножение матрицы на вектор, как показано на рисунке~\ref{fig:bfs_step}, является шагом в алгоритме поиска в ширину. Решения, основанные на стандарте GraphBLAS, производительны, масштабируемы и имеют более дружественное API~\cite{sevengr}. Тем не менее на данный момент нет полноценных инструментов, реализующих стандарт GraphBLAS на графических процессорах общего назначения. Текущие реализации GraphBLAS на GPGPU (например, GraphBLAST~\cite{blast}) показывают, что использование GPGPU действительно может улучшить производительность инструментов такого рода, однако разработчики сталкиваются не только с проблемами, связанными с реализацией обобщенных операций на графических процессорах с помощью стандартных инструментов языка C++, но и с переносимостью решений, основанных на программно-аппаратной платформе CUDA.

\begin{figure}[h!]
    \centering
    \includegraphics[width=0.6\linewidth]{pictures/MatrixBFS.png}
    \caption{Вычисление одного шага в алгоритме поиска в ширину\footnotemark}
    \label{fig:bfs_step}
\end{figure}

Одним из возможных подходов к реализации GraphBLAS на GPU является использование языка высокого уровня, а также библиотек, динамически транслирующих конструкции и объекты данного языка в низкоуровневый код, способный исполнятся на графическом процессоре видеокарты. Данный подход был опробован на прототипе GraphBLAS-sharp\footnote{Репозиторий библиотеки GraphBLAS-sharp: \url{https://github.com/YaccConstructor/GraphBLAS-sharp}. Дата посещения: 09.03.2022}, разработанном на кафедре системного программирования СПбГУ, который показал  жизнеспособность этой идеи. В качестве библиотеки для взаимодействия с OpenCL прототип использует библиотеку Brahma.FSharp, которая также разрабатывается на кафедре системного программирования СПбГУ. Библиотека позволяет использовать подмножество языка F\# для написания OpenCL ядер и предоставляет интерфейс для работы с ними. В ходе работы над прототипом GraphBLAS-sharp был выявлен ряд недостатков библиотеки \\ Brahma.FSharp, перечисленных ниже.
 
\begin{enumerate}
% \item Наличие больших накладных расходов на копирование данных между управляемой, неуправляемой и видеопамятью. (?)
\item Отсутствие атомарных операций для произвольных типов данных, что не позволяет реализовывать некоторые алгоритмы, которые могут оказаться лучше аналогов.
\item Отсутствие поддержки трансфера пользовательских типов данных
из управляемой памяти в видеопамять.
\item Отсутствие возможности вручную управлять выделением памяти на OpenCL устройстве.
\item Отсутствие возможности исполнения нескольких OpenCL ядер параллельно.
\end{enumerate}

\footnotetext{GraphBLAS [Электронный ресурс] // Википедия. Свободная энциклопедия. – URL: \url{https://en.wikipedia.org/wiki/GraphBLAS} (дата обращения: 01.01.2022).}


\section{Постановка задачи}
% Task

Целью данной работы является разработка алгоритма для задачи достижимости с регулярными ограничениями, который будет основан на операциях линейной алгебры и будет эффективно решать задачу для небольшого количества стартовых вершин.

Для выполнения этой цели в рамках данной работы были поставлены следующие задачи.

 \begin{enumerate}
 \item Провести обзор существующих алгоритмов, решающих задачу достижимости в графе с ограничениями в виде формальных языков.
 \item Разработать алгоритм, решающий задачу достижимости в случае нескольких стартовых вершин для регулярных ограничений, в терминах линейной алгебры.
 \item Реализовать разработанный алгоритм с помощью операций линейной алгебры.
 \item Провести экспериментальное исследование эффективности реализованного алгоритма.
 \end{enumerate}


\section{Обзор предметной области}
% !TeX spellcheck = ru_RU
% !TEX root = vkr.tex

\label{sec:relatedworks}
В данном разделе приводится терминология используемая в работе, представляются основные подходы для решения задачи поиска путей с регулярными ограничениями.

\subsection{Основы теории формальных языков}

В этой секции вводятся ключевые понятия из теории формальных языков, которые нужны для введения ограничений на пути в графе с помощью регулярных языков.

\begin{rudefinition} \emph{Формальной грамматикой} называется четверка \\ $\langle V_N, V_T, P, S \rangle$, где
    \begin{itemize}
        \item $V_N, V_T$ --- конечные и непересекающиеся алфавиты нетерминалов и терминалов соответственно;
        \item $P$ --- конечное множество правил;
        \item $S$ --- стартовый нетерминал.
    \end{itemize}
\end{rudefinition}

\begin{rudefinition} \emph{Праволинейной грамматикой} называется формальная грамматика, правила которой могут быть заданы как $A \rightarrow aB$, $A \rightarrow a$, $A \rightarrow \epsilon$.
\end{rudefinition}

\begin{rudefinition}
    \emph{Детерминированным конечным автоматом без эпсилон-переходов} называется пятерка $\langle Q, \Sigma, P, Q_{s}, F \rangle$, где
    \begin{itemize}
        \item $Q$ --- конечное непустое множество состояний;
        \item $\Sigma$ --- конечный входной алфавит;
        \item $P$ --- отображение $Q \times \Sigma \rightarrow Q$;
        \item $Q_{s} \subseteq Q$ --- множество начальных состояний;
        \item $F \subseteq Q$ --- множество конечных состояний.
    \end{itemize}
\end{rudefinition}

Далее, будем называть детерминированный конечный автомат без эпсилон-переходов --- конечным автоматом.

\begin{rudefinition} \emph{Регулярным выражением} над алфавитом $\Sigma$ называется
    \begin{itemize}
        \item $a$, если $a$ --- пустая строка, или $a \in \Sigma$;
        \item $e$ $\cdot$ $f$ (конкатенация), если $e$ и $f$ --- регулярные выражения;
        \item $e$ | $f$ (перечисление), если $e$ и $f$ --- регулярные выражения;
        \item $e^*$ (звезда Клини), если $e$ --- регулярное выражение.
    \end{itemize}
\end{rudefinition}

Для любого регулярного выражения существует представление в виде конечного автомата. Этот факт используется при построении алгоритмов, основанных на выражении регулярных ограничений с помощью конечных автоматов.

\subsection{Основные термины из теории графов}

\begin{rudefinition}\emph{Матрицей смежности} графа $\mathcal{G}$ называется матрица $M^{n \times n}$, где $n$ --- число вершин в графе, ячейка $M[i, j]$ имеет значение $l$, если существует ребро $e$ между вершинами $i$, $j$ с меткой $l$.
\end{rudefinition}

Заметим, что по каждой метке $l$ графа $\mathcal{G}$ можно построить матрицу смежности $\mathcal{M}^l$. Тогда ячейки матрицы $\mathcal{M}^l$ будут отражать факт наличия ребра $e$ с конкретной меткой $l$. Такое представление называется булевой декомпозицией матриц смежности графа. Оно позволяет хранить граф в виде набора булевых матриц, которыми удобно оперировать при построении алгоритмов на основе операций линейной алгебры.

Теперь можно сформулировать задачу поиска путей в графе с ограничениями, которые заданы с помощью регулярных языков. В частности, в данной работе рассматривается один из типов задачи поиска путей --- задача достижимости. Сформулируем эту задачу двумя способами. Они
отличаются в постановке результата, который ожидается в ходе решения задачи достижимости.

\begin{rudefinition} \emph{Задача достижимости с ограничениями в виде регулярных языков}. Пусть имеются:
    \begin{itemize}
        \item граф $\mathcal{G} = \langle V, E, L\rangle$;
        \item регулярный язык $\mathcal{L}$;
        \item множество стартовых $V_s \subseteq V$ и финальных $V_f \subseteq V$ вершин.
    \end{itemize}
    Рассмотрим пути $\pi = (v_0, e_0, \dots, e_n, v_n), e_k = (v_{k-1}, l_k, v_{k})$. Сопоставим каждому пути слово $W(\pi)=(l_0l_1 \dots l_{n-1}) \in \mathcal{L}$. Необходимо:

    \begin{itemize}
        \item Найти \textit{множество}, состоящее из вершин $v_n \in V$, для которых существует хотя бы один путь $\pi$ с началом в $v_0 \in V$ такой, что $W(\pi) \in \mathcal{L}$, $v_0 \in V_s$, $v_n \in V_f$.
        \item Найти все \textit{пары} вершин $v_0, v_n \in V$, для которых существует хотя бы один путь $\pi$ такой, что $W(\pi) \in \mathcal{L}$, $v_0 \in V_s$, $v_n \in V_f$.
    \end{itemize}
\end{rudefinition}

\subsection{Существующие решения}

В литературе выделяют несколько основных подходов к решению задачи поиска путей с ограничениями в виде регулярных языков. Они включают в себя подходы на основе конечных автоматов, использование программ на языке Datalog, построение индекса путей в графе для последующего исполнения регулярного запроса на основе индекса. Каждый из этих подходов будет подробнее рассмотрен в этой части обзора.

Алгоритмы для решения задачи PRQ на основе конечных автоматов строятся следующим образом. Входные данные в виде графа и регулярного языка представляются с помощью конечных автоматов. Для начала строится конечный автомат, который будет задавать регулярный язык. Далее, входной граф представляется в виде конечного автомата, у которого все состояния являются стартовыми и финальными. В итоге задачу поиска путей в графе с ограничениями, заданными с помощью регулярных языков, можно свести к задаче нахождения пересечения двух конечных автоматов.

Также вычислить пересечение конечных автоматов можно с использованием произведения Кронекера, которое применяется к матрицам из булевых декомпозиций матриц смежности для входного графа и представления конечного автомата в виде другого графа. Подробнее, использование матричных операций и произведения Кронекера рассматривается в разделе~\ref{2.4matr}.

\subsubsection{Подход с использованием языка Datalog}

Программа на Datalog представляет собой конечное множество правил. Правило --- это выражение следующей формы.

\begin{center}
    $h :$-- $p_1, p_2, \dots, p_n.$
\end{center}

Выражения $h, p_1, \dots, p_n$ представляют собой формулы вида \\$R(t_1, \dots, t_k)$, где $t_i$ может быть либо константой, либо переменной. Выражение $p_i$ --- факт, если все его $t_i$ являются константами.

    Для того, чтобы исполнять регулярные запросы с помощью программ на языке логического программирования Datalog, нужно представить граф в виде списка фактов о его ребрах.
    \begin{align*}
        edge(u_1 & ,l_1, v_1) \\
        edge(u_2 & ,l_2, v_2) \\
                 & \dots      \\
        edge(u_n & ,l_n, v_n)
    \end{align*}
    Где каждое ребро c меткой $l_i$, соединяет вершины $u_i$ $v_i$.

    Пример запроса $l^*$, который вычисляет транзитивное замыкание графа, содержащее рёбра с меткой $l$ можно описать следующим образом.

    \begin{center}
        $path(x, y) :$-- $edge(x, l, z), path(z, y).$\\
        $path(x, y) :$-- $edge(x, l, y).$
    \end{center}

    Имея представление запроса в виде регулярного выражения, можно получить программу на языке Datalog, которая будет исполнять этот запрос. Для этого нужно представить регулярное выражение с помощью правил грамматики, после чего транслировать грамматику во множество правил Datalog.

    Так, регулярное выражение $l^*$ может быть представлено правилами $S \rightarrow l$ $S$ и $S \rightarrow \epsilon$, которые легко транслируются в представленные выше правила Datalog.

Таким образом, для выполнения сравнения с этим подходом необходимо реализовать трансляцию регулярных выражений в программы на Datalog.

\subsubsection{Подход, основанный на построении индекса}

Существует множество работ на тему построения индекса путей в графе. Одни из них предлагают запоминать ключевые структуры в графе~\cite{related_frequent_fragments}, как, например, часто встречающиеся подграфы или другие фрагменты, описывающие основные свойства графа. Другой подход основан на индексировании путей и используется в работе~\cite{related_fletcher}, где авторы хранят пути размера не больше k, k --- целочисленная константа небольшого значения. Такой индекс позволяет исполнить любой регулярный запрос, состоящий не более чем из k меток, за один просмотр. Для запросов большего размера пути разделяются на части, длина которых кратна k, после чего запрос исполняется за несколько просмотров индекса.

Основным преимуществом построения индекса является то, что он позволяет быстро исполнить запрос по уже построенному пути. Однако использование индекса сказывается на размере потребляемой памяти во время работы алгоритма. Так, в исследовании~\cite{related_fletcher} самый большой граф имеет 131828 вершин, и для него авторы не смогли построить индекс, длина пути которого больше 2, в виду ограничений по памяти.

\subsection{Поиск путей с ограничениями с помощью операций над матрицами}\label{2.4matr}

Алгоритмы на основе матричных операций представляют еще один класс решений описанной задачи. Существует ряд алгоритмов, которые решают более общую задачу --- задачу КС-достижимости, однако они могут быть применены к регулярным ограничения на пути в том числе.

\subsubsection{Тензорный алгоритм}

Один из таких алгоритмов основан на произведении Кронекера~\cite{related_kron}. Он использует операции матричного умножения и не требует модификации изначальной контекстно-свободной грамматики.

На вход алгоритму подается конечный автомат, описывающий граф. Вторым аргументом алгоритм получает рекурсивный автомат, описывающий ограничения на пути в графе. Аналогично тому, как регулярное выражение может быть записано в виде конечного автомата, КС-грамматика может быть выражена с помощью рекурсивного автомата. Рекурсивный автомат может быть представлен в виде графа. Тогда, благодаря тензорному произведению, можно вычислить пересечение конечного автомата, соответствующего входному графу и рекурсивного автомата, соответствующего КС-ограничениям. После чего полученный автомат пересечения транзитивно замыкается. Эти операции применяются в цикле для булевой декомпозиции матриц смежности автомата, соответствующего КС-ограничениям, пока любая из матриц булевой декомпозиции меняется.

Операции вычисления транзитивного замыкания и тензорного произведения находят пути в графе для всех пар вершин. В случае, когда имеется множество стартовых вершин, производительность этого алгоритма может упасть, так как все вершины графа буду считаться начальными, и для всех них посчитаются пути до других вершин.


\section{Алгоритм, основанный на операциях линейной алгебры и поиске в ширину}
% !TeX spellcheck = ru_RU
% !TEX root = vkr.tex

\subsection{Поддержка обобщенных атомарных операций}
Когда речь идёт о библиотеке обобщенных вычислений на GPGPU, проблема использования произвольных атомарных операций встает весьма остро. Их отсутствие существенно ограничивает спектр доступных алгоритмов, в том числе необходимых в задачах линейной алгебры. Стандарт OpenCL позволяет использовать некоторый набор атомарных функций над небольшим множество примитивных типов, однако для реализации обобщенных алгоритмов этого оказывается недостаточно. К сожалению, общепринятого решения этой проблеме найти не удалось. Среди аналогичных инструментов только ILGPU имеет возможность конструировать и использовать произвольные атомарные операции на основе неатомарных. Для этого в библиотеке есть статический метод \verb|MakeAtomic|, контракт которого приведен в листинге~\ref{lst:makeAtomic}. В качестве реализации используется цикл активного ожидания.

\begin{lstlisting}[caption= Контракт метода MakeAtomic, language=C, frame=single, label={lst:makeAtomic}]
public T MakeAtomic<T, TOperation, TCompareExchangeOperation>(
    ref T target,
    T value,
    TOperation operation,
    TCompareExchangeOperation compareExchangeOperation)
\end{lstlisting}

% \footnote{Один из способов расширить множество доступных атомарных операций: \url{https://stackoverflow.com/a/31865819/17106499}. Дата посещения 09.03.2022}
Существует и альтернативный способ расширить множество доступных для использования атомарных операций. Этот способ использует знание о том, что устройства, реализующие стандарт OpenCL, обеспечивают поддержку атомарных операций над 64-битными целочисленным типом \text{long}, в то время как поддержка других 64-битных типов (например, \text{double}) не гарантируется. Это ограничение можно обойти за счет использования union структуры, объединяющей любой 64-битный тип с типом \text{long}\footnote{Ссылка на предлагаемое решение: \url{https://stackoverflow.com/questions/31863587/atomic-operations-with-double-opencl/31865819#31865819}. (Дата посещение: 27.05.2022)}. Очевидным недостатком такого решения является то, что оно подходит только для типов размером 64 бита.

В данной работе поддержка произвольных атомарных операций выполнена за счет генерации функций, реализующих спинлок. Для этого язык ядра был расширен функцией \verb|atomic| с сигнатурой, приведенной в листинге~\ref{lst:atomic}. Несмотря на то, что это обыкновенная функция языка F\#, её применение внутри кода ядра ограниченно. Так, существуют ограничения на используемые в качестве первого параметра выражения --- это должна быть либо переменная, либо обращение к массиву по индексу, а также ограничены возможности ее частичного применения. Обработкой этой функции занимается соответсвующий шаг на этапе преобразования синтаксического дерева (как изображено на рисунке~\ref{fig:transl2}), который генерирует спинлок и заменяет вызов функции \verb|atomic| на вызов функции, реализующей этот спинлок. Кроме того, он преобразует код ядра таким образом, чтобы обеспечить возможность выделения памяти под мьютексы --- для каждого массива, доступ к которому необходимо ограничить взаимным исключением, выделяется дополнительный массив 32-битных целых чисел соответствующей длины, заполненный нулями. Таким образом, например, код ядра на языке F\#, приведенный в листинге~\ref{lst:atomicKernel}, преобразуется в код на языке OpenCL C, приведенный в листинге~\ref{lst:generatedAtomic}. Стоит также отметить, что в случае, когда вызов функции \text{atomic} можно преобразовать в вызов нативной атомарной функции OpenCL, спинлок не генерируется.

\begin{figure}[h!]
\centering
\includegraphics[scale=0.25]{pictures/Modified.png}
\caption{Этап преобразования функции atomic в модели трансляции}
\label{fig:transl2}
\end{figure}

\newpage

\begin{lstlisting}[caption=Сигнатура функции atomic, language=Caml, frame=single, label={lst:atomic}]
val atomic : ('a -> 'b) -> ('a -> 'b)
\end{lstlisting}

\begin{lstlisting}[caption=Код OpenCL ядра на языке F\#, language=Caml, frame=single, label={lst:atomicKernel}]
<@
    fun (range: Range1D) (buf: ClArray<int>) ->
        atomic (fun x -> x + 1) buf.[0] |> ignore
@>
\end{lstlisting}

\begin{lstlisting}[caption=Код OpenCL ядра после трансляции в C, language=C, frame=single, label={lst:generatedAtomic}]
#pragma OPENCL EXTENSION cl_khr_global_int32_base_atomics: enable
#pragma OPENCL EXTENSION cl_khr_local_int32_base_atomics: enable

int baseFunc(private int x)
{
    return (x + 1);
}

int atomicFunc(__local int* localAccMutex, __local int* x) {
    int oldValue;
    bool flag = 1;
    while (flag) {
        int old = atom_xchg(&localAccMutex[0], 1);
        if (old == 0) {
            oldValue = *x;
            *x = baseFunc(*x);
            atom_xchg(&localAccMutex[0], 0);
            flag = 0;
        };
        barrier(CLK_LOCAL_MEM_FENCE);
    };
    return oldValue;
}
\end{lstlisting}

% Первые тесты этого решения показали и его основной недостаток: спинлок плохо подходит для использования в программах для GPGPU. Работоспособность текущей реализации сильно зависит от платформы, на которой исполняется код, поэтому имеет смысл сделать транслятор машинозависимым. Однако, как показывает практика, даже это не дает гарантий, что генерируемый код будет вести себя единообразно на всех возможных устройствах.

Предложенное решение имеет и свои недостатки. Так, например, при использовании сгенерированной атомарной функции выделяется дополнительная память под массив мьютексов, что может быть неочевидно для пользователя. Кроме того, решение обладает низкой переносимостью из-за особенностей реализации OpenCL на каждом конкретном устройстве --- как показывает практика, нет гарантий того, что генерируемый код будет вести себя единообразно на всех возможных устройствах.

\subsection{Поддержка трансфера пользовательских типов данных}
В задачах, связанных с обработкой графов, метки ребер могут иметь произвольный тип. В связи с этим стандарт GraphBLAS не налагает ограничений на тип элементов матрицы. Эталонная реализация \\ GraphBLAS на CPU --- SuiteSparse --- позволяет использовать любою структуру фиксированного размера. Поэтому возникает необходимость обеспечить поддержку использования пользовательских структур данных. Кроме структур, интерес вызывает возможность использования размеченных объединений в ядрах OpenCL. Так, с помощью типа данных \verb|Option| естественно выражается наличие в графе ребра с некоторым весом или его отсутствие. Использование типа \verb|Option| для реализации GraphBLAS может решить проблему <<явных нулей>>\footnote{Обсуждение необходимости фильтровать явные нули: \url{https://github.com/GraphBLAS/LAGraph/issues/28}. Дата обращения: 09.03.2022}, которая вызывает дискуссии в сообществе до сих пор.

Несмотря на то, что транслятор уже умел транслировать некоторые пользовательские типы данных (кортежи, структуры, размеченные объединения), трансфер данных таких типов из управляемой памяти в видеопамять и наоборот реализован не был. Среди аналогичных инструментов лишь ILGPU и FSCL поддерживают использование пользовательских типов данных. Рантайм библиотеки ILGPU способен оперировать любыми типами, которые удовлетворяют ограничению \text{unmanaged}\footnote{Документация, описывающая unmanaged типы: \url{https://docs.microsoft.com/en-us/dotnet/csharp/language-reference/builtin-types/unmanaged-types} (Дата обращение: 27.05.2022)}. На самом деле, множество доступных типов еще более ограничено непреобразуемым (blittable) типами --- такими типами, которые представлены одинаково в управляемой и неуправляемой памяти. Благодаря этому ограничению становится возможным не копировать данные из управляемой памяти в неуправляемую, а использовать их сразу, защитив предварительно от сборщика мусора с помощью метода \verb|GCHandle.Alloc| с параметром \verb|GCHandleType.Pinned|. Несмотря на то, что ограничение \text{unmanaged} позволяет использовать широкий набор примитивных типов, а также обобщенные структуры, некоторые преобразуемые типы, например \text{bool}, а также размеченные объединения недоступны для использования в ILGPU. Рантайм FSCL обеспечивает поддержку преобразуемых типов, однако данные таких типов явно копируются из управляемой памяти в неуправляемую, что негативно сказывается на производительности.

Система, реализованная в данной работе, способна обрабатывать как преобразуемые типы данных, так и непреобразуемые. В случае преобразуемых типов данные при трансфере явно копируются из управляемой памяти в неуправляемую. В противном случае копирования не происходит, и данные доступны сразу. Для обеспечения трансфера данных был реализован класс \verb|CustomMarshaler|, который содержит методы, описанные ниже.
\begin{itemize}
    \item \verb|IsBlittable(Type): bool| --- определяет, является ли тип данных непреобразуемым.
    \item \verb|WriteToUnmanaged('a[]): int| --- копирует данные из управляемой памяти в неуправляемую. 
    \item \verb|ReadFromUnmanaged(IntPtr, length: int): unit| --- копирует данные из неуправляемой памяти в управляемую. 
\end{itemize}

На данный момент реализована поддержка трансфера следующих типов данных:
\begin{itemize}
    \item кортежей;
    \item записей;
    \item размеченных объединений;
    \item пользовательских структур.
\end{itemize}

\subsection{Улучшение модели управления памятью}
В прежней версии библиотеки выделение памяти на OpenCL устройстве происходило неявно. Такой подход имел ряд существенных недостатков:
\begin{itemize}
    \item было невозможно выделять память напрямую на OpenCL устройстве;
    \item было невозможно выставлять нужные флаги буфера при создании;
    \item было невозможно автоматически освобождать выделенную память.
\end{itemize}
Все эти факторы негативно сказывались на производительности. Большинство же аналогичных библиотек предоставляют возможности для более тонкого управления памятью и временем жизни выделенных объектов. Так, например, библиотека ILGPU имеет абстракцию верхнего уровня над массивом данных на OpenCL устройстве --- \verb|ArrayView|.

Для решения описанных выше проблем были введены абстракции, которые помогли обеспечить необходимый уровень контроля за управлением памятью. Они перечислены ниже.
\begin{itemize}
    \item \verb|ClBuffer| --- класс, который абстрагирует участок памяти на OpenCL устройстве. Согласно идиоме RAII, он инкапсулирует владение ресурсом (памятью в данном случае). Захват ресурса происходит в конструкторе, а освобождение --- в методе Dispose интерфейса IDisposable. 
    \item \verb|ClArray| --- высокоуровневая обертка над \verb|ClBuffer|, предоставляющая стандартный интерфейс взаимодействия с одномерным массивом данных.
    \item \verb|ClCell| --- высокоуровневая обертка над \verb|ClBuffer|, предоставляющая интерфейс взаимодействия с одноэлементным множеством.
\end{itemize}

Более подробная иерархия классов изображена на рисунке~\ref{fig:mem}.


\subsection{Общая архитектура решения}
В настоящей работе была также переработана общая архитектура решения. Необходимость этих изменений продиктована не только связью с обновлением механизмов управления памятью, но также с требованием реализовать нативную поддержку параллельного исполнения OpenCL команд. Ранее такая возможность почти полностью отсутствовала.

Иерархия классов предлагаемого решения изображена на рисунке~\ref{fig:ar}.
В предлагаемой архитектуре можно выделить 3 слоя. Самый нижний --- слой абстракций внешней библиотеки OpenCL.NET. Эта библиотека предоставляет интерфейс для вызова нативных функций OpenCL. 

Над ним располагается слой, целью которого является следующее:
\begin{enumerate}
    \item обеспечить возможность трансляции F\# кода в OpenCL, а также трансфера данных из управляемой памяти в видеопамять;
    \item предоставить более удобный интерфейс для взаимодействия с \\ OpenCL;
    \item сохранить гибкость используемых абстракций.
\end{enumerate}
В этом слое расположены классы, необходимые для непосредственного взаимодействия F\# и OpenCL --- транслятор из F\# в OpenCL \verb|FSQuotationToOpenCLTranslator| и маршаллер \verb|CustomMarshaler|. Абстракции этого слоя почти полностью повторяют набор абстракций слоя OpenCL.NET, обеспечивая таким образом необходимую гибкость, однако добавляют также ряд особенностей, призванных упростить взаимодействие с пользователем. Так, например, асинхронная очередь команд \verb|MailboxProcessor<Msg>|, которая является абстракцией более высокого уровня над \verb|OpenCL.Net.CommandQueue|, позволяет удобнее организовать контроль над временем жизни объекта в памяти OpenCL устройства, а также позволяет синхронизировать выполнение операций в нескольких параллельных очередях. За счет этого и достигается требуемая параллельность на уровне исполнения команд.

На самом верхнем уровне располагается слой, который призван упростить модель параллельного программирования, скрыв абстракции \\ OpenCL за функциональным интерфейсом. Для этого было реализовано вычислительное выражение \verb|ClTask|, которое хранит контекст вычислений, а также функции \verb|runSync| и \verb|inParallel|, запускающие вычисления синхронно или параллельно.

\begin{figure}
\centering
\includegraphics[scale=0.3]{pictures/Mem (1).png}
\caption{Иерархия абстракций в модели управления памятью}
\label{fig:mem}
\end{figure}

\begin{figure}
\centering
\includegraphics[scale=0.25]{pictures/class.png}
\caption{Общая архитектура решения}
\label{fig:ar}
\end{figure}

\section{Эксперимент}
% !TeX spellcheck = ru_RU
% !TEX root = vkr.tex

Данный раздел посвящен описанию проводимого экспериментального исследования. В предыдущих главах были представлены основные подходы к исполнению запросов с регулярными ограничениями, а также описан новый алгоритм, основанный на операциях линейной алгебры и поиске в ширину. Сравнение производительности этого алгоритма с реализациями других подходов является основной целью данного исследования.

\subsection{Исследовательские вопросы и метрики}

В качестве основной метрики производительности рассматривается время работы алгоритмов на регулярных запросах.
Цель экспериментального исследования --- ответить на два исследовательских вопроса:

\begin{itemize}
    \item \textbf{RQ1}: Какова производительность разработанного алгоритма по сравнению с существующими аналогами?
    \item \textbf{RQ2}: Как влияет размер множества стартовых вершин на производительность реализации разработанного алгоритма?
\end{itemize}

\subsection{Окружение и конфигурации}

Экспериментальное исследование проводится на сервере под управлением ОС Ubuntu 20.04 со следующими характеристиками: процессор Intel Core i7-6700 CPU, 4 ядра с частотой 3.40 ГГц; объем оперативной памяти 64 Гб.

Для проведения сравнения с аналогом на основе языка Datalog было принято решение использовать язык Souffle~\cite{souffle}, который является одним из диалектов языка Datalog и преодолевает некоторые ограничения классического варианта. Souffle часто используется в задачах языкового анализа и имеет сейчас большую популярность. Установлена версия Souffle 2.4, которая была сконфигурирована с размером слова для примитивного типа 64 бита и запускалась с флагами \verb|-j {число ядер процессора}|, который используется для параллельного исполнения программы, и \verb|--magic-transform=*| для преобразования программы в эквивалентную с потенциально меньшим числом промежуточных вычислений.

Для подхода, основанного на построении индекса, была найдена реализация\footnote{Repository for the prototype code for index RPQ paper: \href{https://github.com/darroyue/Ring-RPQ}{https://github.com/darroyue/Ring-RPQ} (accessed: 20.05.2023)}, находящаяся в свободном доступе, которую автору удалось запустить, однако этот инструмент выдает неверный результат при запуске на сгенерированных по шаблонам запросах, по этой причине он был не включен в сравнительный анализ.

Реализация тензорного алгоритма для нескольких стартовых вершин была взята из репозитория CFPQ\_PyAlgo\footnote{Repository for developing, testing and benchmarking CFPQ algorithms implemented in PyGraphBLAS: \href{https://github.com/FormalLanguageConstrainedPathQuerying/CFPQ_PyAlgo}{https://github.com/FormalLanguageConstrainedPathQuerying/CFPQ\_PyAlgo} (accessed: 20.05.2023)}.

\subsection{Условия эксперимента}

В качестве исходных данных были выбраны графы, собранные по реальным RDF данным и данным социальных сетей. Их характеристики представлены в таблице. В таблице выделено количество вершин и рёбер графа, а также, число уникальных меток, содержащихся на рёбрах. Графы класса RDF расположены в первой части таблицы, графы социальных сетей --- во второй.

\definecolor{lightgray}{gray}{0.9}
\begin{table}[!ht]
    \centering
    \rowcolors{1}{}{lightgray}
    \begin{tabular}{|c|c|c|c|}
        \hline
        Graph      & \#V       & \#E        & \#L \\ \hline \hline
        enzyme     & 48 815    & 86 543     & 14  \\
        eclass     & 239 111   & 360 248    & 10  \\
        go         & 582 929   & 1 437 437  & 47  \\
        geospecies & 450 609   & 2 201 532  & 158 \\
        taxonomy   & 5 728 398 & 14 922 125 & 21  \\
        \hline
        advogato   & 6 541     & 51 127     & 3   \\
        youtube    & 15 088    & 27 257 790 & 5   \\
        \hline
    \end{tabular}
    \caption{Графы.}
\end{table}

Стоит отметить, что графы, собранные по данным социальных сетей являются достаточно плотными в сравнении с графами, построенными по RDF-данным, так как имеют существенно большее число ребер, чем вершин. Это может привести к тому, что во время обхода этих графов будут пройдены более длинные пути, что может сказаться на времени исполнения алгоритмов.

Регулярные запросы к графам были сгенерированы на основе самых популярных шаблонов запросов, собранных в  работе~\cite{related_oneforall}. В запросы входят классические операторы регулярных выражений такие, как конкатенация и перечисление, а также звезда Клини и множество комбинаций этих операторов. Всего отобрано 16 шаблонов, они представлены в таблице~\ref{table:queries}.

\begin{table}[!ht]
    \centering
    \rowcolors{1}{}{lightgray}
    \begin{tabular}{|c|c||c|c|}
        \hline
        Name  & Query                           & Name     & Query                                           \\ \hline \hline
        $q_0$ & $a^*$                           & $q_{8}$  & $a \cdot b$                                     \\
        $q_1$ & $a \cdot b^*$                   & $q_{9}$  & $a \cdot b \cdot c$                             \\
        $q_2$ & $a \cdot b^* \cdot c^*$         & $q_{10}$ & $a \cdot b \cdot c \cdot d$                     \\
        $q_3$ & $a \cdot b^* \cdot c$           & $q_{11}$ & $(a \cdot b)^+~|~(c \cdot d)^+$                 \\
        $q_4$ & $a^* \cdot b^*$                 & $q_{12}$ & $(a \cdot (b \cdot c)^*)^+~|~(d \cdot e)^+$     \\
        $q_5$ & $a \cdot b \cdot c^*$           & $q_{13}$ & $(a \cdot b \cdot (c \cdot d)^*)^+~|~(e~|~f)^*$ \\
        $q_6$ & $(a~|~b~|~c~|~d~|~e)^+$         & $q_{14}$ & $(a~|~b)^+~(c~|~d)^+$                           \\
        $q_7$ & $(a~|~b~|~c~|~d~|~e) \cdot f^*$ & $q_{15}$ & $a \cdot b \cdot (c~|~d~|~e)$                   \\
        \hline
    \end{tabular}
    \caption{Шаблоны запросов.}\label{table:queries}
\end{table}

Для каждого графа отбиралось 5 самых популярных меток, после чего по ним на основе шаблонов генерировалось случайным образом 5 запросов. Важно отметить, что в графе $advogato$ меньше 5 различных меток, по этой причине в сгенерированных для этого графа запросах допускается повторение меток.

В зависимости от числа вершин, на которых запускаются регулярные запросы, их можно разделить на несколько видов:
\begin{itemize}
    \item Single-source запросы, для которых алгоритм запускается от одной исходной вершины.
    \item Multiple-source запросы запускаются от определенного изначально множества вершин в графе.
    \item All-pairs запросы запускаются от множества всех вершин графа.
\end{itemize}

Для выбора множества начальных вершин было решено создавать случайные выборки из нужного числа вершин. Так, для single-source запросов создавалось множество из 100 начальных вершин, для каждой из которых запускался запрос. Для multiple-source запросов создавалась выборка из 2, 5, 10, 100, 1000, 10000 случайных вершин для каждого графа, на которой запускались запросы. При этом финальными считались все вершины графа. Скрипты для создания запросов и генерации вершин доступны в репозитории\footnote{Benchmark suite for RPQ evaluation:~\href{https://github.com/bahbyega/paths-benchmark}{https://github.com/bahbyega/paths-benchmark}~(accessed: 20.05.2023)}.

\subsection{Результаты сравнения алгоритмов}
\textbf{RQ1}: Какова производительность разработанного алгоритма по сравнению с существующими аналогами?

На рисунках 3--4 представлены результаты эксперимента по сравнению производительности алгоритмов на графах, построенных по RDF данным. Введены следующие обозначения: алгоритм, основанный на поиске в ширину, обозначен MSBFS, алгоритм, основанный на тензорном произведении --- Tensor, решение, основанное на Datalog и реализованное с помощью диалекта Souffle --- Datalog. Для всех шаблонов в виде гистограмм представлено среднее арифметическое значение времени исполнения алгоритмов с отклонением от среднего на соответствующем запросе.

Прежде всего эксперимент был поставлен на графах наименьшего размера --- \textit{enzyme} и \textit{eclass} для случая single-source запросов, который представлен в первой колонке. Результаты показывают, что реализации алгоритмов в терминах матричных операций в среднем на порядок быстрее реализации на Datalog. В частности, алгоритм на основе тензорного произведения работает более чем в 10 раз быстрее аналога на Datalog, алгоритм MSBFS --- в среднем в 100 раз быстрее. Важно отметить, что программы для аналога на Datalog генерировались по соответствующим представлениям регулярных ограничений в виде грамматик. Реализовать каждый из запросов для конкретного графа можно эффективнее, используя дополнительные средства языка Souffle, что могло повлиять на показатели времени, представленные на рисунках.

Далее, три реализаций запускались на multiple-source запросах. Результаты для наибольшего из множеств начальных вершин (10000 вершин) представлены на рисунках 3--4 во второй колонке. На множестве из 10000 стартовых вершин Tensor и Datalog показывают близкую производительность, проигрывая MSBFS. При этом показатели времени исполнения тензорным алгоритмом имеют более высокое отклонение от среднего значения. Это может объясняться тем, что конкретный шаблон представлен несколькими запросами с метками, расположенными в разном порядке. Из-за чего результат исполнения запроса может сильно отличаться по количеству достижимых вершин. Так, для шаблонов $q_6$, $q_8$, $q_9$, $q_{10}$ большое значение имеет порядок, в котором расставлены метки в шаблоне, потому что в них отсутствует оператор звезды Клини.

После чего проводился эксперимент на графе \textit{go}. Из него видно, что производительность MSBFS оказалась незначительно хуже алгоритма, основанного на тензорном произведении. Это может быть объяснено структурой графа \textit{go}. Он имеет большое число ребер с одной и той же меткой, которая чаще всего встречалась в запросах, что повлияло на количество шагов алгоритма MSBFS.

По экспериментам на графах \textit{geospecies} и \textit{taxonomy} также можно выделить лучшую производительность MSBFS и Tensor относительно Datalog. Однако тензорный алгоритм демонстрирует выбросы на некоторых запросах, когда запрос запускается на множестве из 10000 вершин. Кроме того, важно отметить, что производительность алгоритмов зависит от сложности запроса. Так, запросы $q_7$ и $q_{13}$, содержащие в себе большее количество меток и операторов, чем остальные запросы, исполняются в среднем дольше остальных.

% !TeX spellcheck = ru_RU
% !TEX root = vkr.tex

\newcolumntype{C}{ >{\centering\arraybackslash} m{4cm} }
\newcommand\myvert[1]{\rotatebox[origin=c]{90}{#1}}
\newcommand\myvertcell[1]{\multirowcell{5}{\myvert{#1}}}
\newcommand\myvertcelll[1]{\multirowcell{4}{\myvert{#1}}}
\newcommand\myvertcellN[2]{\multirowcell{#1}{\myvert{#2}}}

\afterpage{
    \clearpage
    \thispagestyle{empty}
    \begin{landscape}
        \centering
        \begin{figure}
            \begin{tabular}{cc}
                \includegraphics[width=120mm]{pictures/enzyme_ss.pdf} & \includegraphics[width=120mm]{pictures/enzyme_ms10000.pdf} \\
                \includegraphics[width=120mm]{pictures/eclass_ss.pdf} & \includegraphics[width=120mm]{pictures/eclass_ms10000.pdf} \\
                \includegraphics[width=120mm]{pictures/go_ss.pdf}     & \includegraphics[width=120mm]{pictures/go_ms10000.pdf}     \\
            \end{tabular}
            \caption{Результаты эксперимента на наборе RDF-данных для single-source и multiple-source (10000 вершин) запросов.}
        \end{figure}
    \end{landscape}
    \clearpage
}

% !TeX spellcheck = ru_RU
% !TEX root = vkr.tex

\afterpage{
    \clearpage
    \thispagestyle{empty}
    \begin{landscape}
        \centering
        \begin{figure}
            \begin{tabular}{cc}
                \includegraphics[width=120mm]{pictures/geospecies_ss.pdf} & \includegraphics[width=120mm]{pictures/geospecies_ms10000.pdf} \\
                \includegraphics[width=120mm]{pictures/taxonomy_ss.pdf}   & \includegraphics[width=120mm]{pictures/taxonomy_ms10000.pdf}   \\
            \end{tabular}
            \caption{Результаты эксперимента на наборе RDF-данных для single-source и multiple-source (10000 вершин) запросов.}
        \end{figure}
    \end{landscape}
    \clearpage
}


\textbf{RQ2}: Как влияет размер множества стартовых вершин на производительность реализации разработанного алгоритма?

\begin{figure}
    \begin{tabular}{cc}
        \includegraphics[width=85mm]{pictures/chunks-enzyme.pdf}     & \includegraphics[width=85mm]{pictures/chunks-eclass.pdf}   \\
        \includegraphics[width=85mm]{pictures/chunks-geospecies.pdf} & \includegraphics[width=85mm]{pictures/chunks-taxonomy.pdf} \\
        \includegraphics[width=85mm]{pictures/chunks-advogato.pdf}   & \includegraphics[width=85mm]{pictures/chunks-youtube.pdf}  \\
    \end{tabular}
    \caption{Результаты эксперимента с multiple-source запросами для разных выборок стартовых вершин.}
\end{figure}\label{chunks_bfs_vs_tensor}

Результаты эксперимента по анализу производительности алгоритма для различного числа стартовых вершин представлены на рисунке~\ref{chunks_bfs_vs_tensor}. На нем изображена зависимость времени исполнения запроса от числа стартовых вершин в графе для алгоритмов MSBFS и Tensor. Для выявления зависимости числа стартовых вершин ко времени исполнения была взята еще одна реализация MSBFS, обозначенная MSBFSPairs, которая, в отличие от MSBFS, находит не множество достижимых вершин, а пары начальная--конечная вершина. Основной особенностью этой реализации является то, что инициализация матриц стартовыми вершинами увеличивает размер перемножаемых матриц в $n$ раз, где $n$ --- число стартовых вершин. Это позволяет запомнить от какой стартовой вершины была достигнута каждая из вершин в конечном множестве достижимых. Отмечено время исполнения каждого из запросов, а также выделено среднее время их исполнения. Результаты показывают, что с ростом числа стартовых вершин время работы алгоритма MSBFS меняется несущественно. Это может объясняться тем, что инициализация матриц стартовыми вершинами в его реализации не влияет на размер перемножаемых матриц, что не создает дополнительных расходов во время работы алгоритма, в то время как алгоритмы Tensor и MSBFSPairs страдают от существенного снижения производительности при росте числа стартовых вершин.

Также, на основе этого рисунка можно сделать вывод о том, что время исполнения запроса зависит не только от размера графа, но и от его структуры и разреженности. Так, по анализу данных социальных сетей можно сделать вывод о том, что алгоритм на основе тензорного произведения демонстрирует лучшую производительность для более плотных графов, таких как \textit{advogato} и \textit{youtube}. Именно эти графы имеют наименьшее число различных меток, граф \textit{youtube} также имеет наибольшее число ребер среди всех графов датасета при существенно меньшем числе вершин. Это приводит к тому, что алгоритму, основанному на поиске в ширину, приходится выполнять больше шагов во время обхода графа, так как на этих графах с каждым шагом алгоритма более вероятно увеличение фронта обхода.

Результаты анализа экспериментального исследования позволяют заключить, что реализация алгоритма, основанного на поиске в ширину, показывает приемлемую производительность и во многих случаях работает быстрее, чем реализации аналогов. Более того, реализация алгоритма MSBFS для нахождения всего множества достижимых вершин показывает несущественное ухудшение производительности с ростом числа стартовых вершин. В то же время, реализация алгоритма MSBFSPairs, результатом которого является множество пар начальная--конечная вершина, показала худшие временные характеристики. Более того, заметным является падение производительности обоих реализаций MSBFS на графах социальных сетей, что может быть исследовано в дальнейшем на более широком наборе данных.


\pagebreak

\section*{Заключение}
% !TeX spellcheck = ru_RU
% !TEX root = vkr.tex

При проведении данной работы были достигнуты следующие результаты. 

\begin{itemize}
    \item Проведен обзор и выбрано два доступных аналога для проведения сравнительного анализа, а именно, тензорный алгоритм и алгоритм на основе языка Datalog. Последний является принятым базовым инструментом.
    \item Собран набор данных, состоящий из графов и регулярных запросов, для проведения экспериментального исследования. В него вошли графы собранные по RDF-данным и данным социальных сетей, а также множество популярных запросов.
    \item Спроектирован инструмент, который позволяет автоматизировать проведение экспериментов, поддерживает генерацию данных для запуска запросов на графах, получение множества начальных вершин и запуск различных типов запросов\footnote{Benchmark suite for RPQ evaluation:~\href{https://github.com/bahbyega/paths-benchmark}{https://github.com/bahbyega/paths-benchmark} (accessed: 20.05.2023)}.
    \item Проведено экспериментальное исследование алгоритма и сравнение с аналогами. В ходе анализа было выявлено, что подход к реализации алгоритма поиска путей с ограничениями в виде регулярных языков на основе поиска в ширину является жизнеспособным и показывает приемлемые результаты на реальных данных. 
    Более того, на RDF-данных алгоритм показывает кратный рост производительности относительно аналога, реализованного на Datalog, а также работает в среднем быстрее алгоритма, основанного на тензорном произведении. Тем не менее, анализ зависимости времени исполнения запроса к числу стартовых вершин выявил проблемы реализации алгоритма MSBFS для решения задачи достижимости от каждой стартовой вершины, что может быть исследовано в дальнейшей работе. 
    \end{itemize}

В дальнейшем работа может быть развита в следующих направлениях.
\begin{itemize}
    \item Расширение набора данных, на котором исследуется алгоритм. В него можно включить искусственно сгенерированные графы с определенными особенностями структуры, а также реальные графы из других прикладных областей.
    \item Интегрирование алгоритма с графовой базой данных, а также с языком запросов к графовой базе данных. Это позволит провести еще более полноценное сравнение алгоритма с аналогами, в которые можно будет включить системы графовых баз данных. 
\end{itemize}
\noindent 


\setmonofont[Mapping=tex-text]{CMU Typewriter Text}
\bibliographystyle{ugost2008ls}
\bibliography{vkr}

\end{document}
