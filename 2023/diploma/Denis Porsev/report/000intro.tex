% !TeX spellcheck = ru_RU
% !TEX root = vkr.tex

Современные компьютерные архитектуры позволяют легко обрабатывать линейные и иерархические структуры данных, такие как списки, стеки и деревья. Однако часто модель хранения данных представлена в виде графа. Такое представление применяется во множестве сфер: анализе социальных сетей, семантических сетей, анализе потока управления программ, биоинформатике и многих других.

Задача обработки данных, которые представлены в виде графа, имеет неструктурированный характер. Несмотря на это, некоторые алгоритмы анализа графов можно преобразовать к последовательности матрично-векторных операций, представив граф в виде матрицы смежности. Это позволяет выразить алгоритмы в терминах линейной алгебры, благодаря чему на практике за счет лучшего распараллеливания задачи удается получить эффективные реализации этих алгоритмов.

Более того, матрица смежности может быть представлена в разреженном виде, что позволяет сократить объем используемой памяти, так как в матрице хранятся только ненулевые элементы. Библиотеки разреженной линейной алгебры позволяют удобно работать с таким представлением. Они предоставляют набор примитивных операций таких, как матричное умножение, сложение, транспонирование, умножение вектора на матрицу и других, для работы с данными в разреженном виде.

Большую популярность в этом направлении набирает \verb|GraphBLAS|~\cite{intro_graphblas} --- стандарт, описывающий набор примитивных операций над матрицами и векторами, которые необходимы для выражения алгоритмов анализа графов в терминах линейной алгебры~\cite{intro_graphlastd}. Существует несколько реализаций этого стандарта, как, например, \verb|SuiteSparse:GraphBLAS|~\cite{intro_suitesparse}, которые удобно использовать для создания высокоэффективных алгоритмов анализа графов.

Одни из таких алгоритмов --- алгоритмы поиска путей в графе. В частности, в терминах линейной алгебры могут быть выражены алгоритмы поиска кратчайшего пути в графе, поиска путей между множеством начальных и конечных вершин. В случае, когда важно знать только о существовании пути между вершинами, говорят, что решают задачу достижимости в графе. При этом на множество искомых путей в графе могут накладываться дополнительные ограничения с помощью формальных языков. Для этого требуют, чтобы метки на ребрах искомого пути при конкатенации образовывали слова, принадлежащие заданному над некоторым алфавитом формальному языку.

Для создания ограничений используют регулярные и контекстно--свободные языки. Регулярные ограничения на пути в графе были впервые введены в работах~\cite{intro_rpq_term_appear1, intro_rpq_term_appear2} и нашли широкое применение в языках запросов к графовым базам данных~\cite{intro_rpq_massive_graphs}, которые позволяют эффективно хранить и обрабатывать связанные данные, такие как социальные сети, сети транспорта и многое другое. Контекстно--свободные языки позволяют выразить более широкий класс ограничений и используются также в областях статического анализа кода~\cite{intro_cfpq_code_analysis}, биоинформатике~\cite{intro_cfpq_bioinformatics} и других.

Запросы с регулярными ограничениями (RPQ) являются хорошо изучеными. Тем не менее в работе~\cite{intro_rpq_ineffective} было исследовано два подхода к решению задачи RPQ в языке запросов к графовым базам данных \verb|SPARQL|~\cite{intro_sparql}. Авторы работы выявили, что каждый из этих подходов эффективен только для конкретных видов графов.

В рамках учебной практики третьего курса автором данной работы был предложен новый алгоритм для решения задачи RPQ от множества стартовых вершин. Он был разработан на основе алгоритма поиска в ширину и реализован через примитивные операции разреженной линейной алгебры с помощью GraphBLAS. Для полученного алгоритма требуется провести экспериментальное исследование и сравнить его с аналогичными решениями.
