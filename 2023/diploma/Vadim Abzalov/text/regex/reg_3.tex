\begin{ruexample}
    Регулярное выражение $R_{RDF}^3$ для класса графов RDF, построенное по шаблону $Q_{4}^{2}$ на терминальных символах $type$ и $subClassOf$ соответственно.
\begin{align}
\begin{split}
\label{reg:rdf_reg_3}
(type ~|~ subClassOf)^*
\end{split}
\end{align}
\end{ruexample}

\begin{ruexample}
    Контекстно-свободная грамматика, соответствующая регулярному выражению~(\ref{reg:rdf_reg_3}).
\begin{align}
\begin{split}
\label{cfg:rdf_reg_3}
S & \to \varepsilon \\
S & \to type \ S \\
S & \to subClassOf \ S \\
\end{split}
\end{align}
\end{ruexample}

\begin{ruexample}
    Рекурсивный автомат, соответствующий регулярному выражению~(\ref{reg:rdf_reg_3}).
\end{ruexample}

    \begin{align}
    \label{rsm:rdf_reg_3}
        \begin{tikzpicture}[node distance=2.7cm,shorten >=1pt,on grid,auto, x=20mm, y=20mm]
           \node[state, initial, accepting] (q_0)   {$0 \{S\}$};
           \path[->]
            (q_0) edge[out=30, in=150, looseness=8, above] node {$type$} (q_0)
            (q_0) edge[out=330, in=210, looseness=8, below] node {$subClassOf$} (q_0)
            ;
        \end{tikzpicture}
    \end{align}

\begin{ruexample}
    Регулярное выражение $R_{MA}^3$ для класса графов MemoryAliases, построенное по шаблону $Q_{4}^{2}$ на терминальных символах $a$ и $d$ соответственно.
\begin{align}
\begin{split}
\label{reg:memory_aliases_reg_3}
(a ~|~ d)^*
\end{split}
\end{align}
\end{ruexample}

\begin{ruexample}
    Контекстно-свободная грамматика, соответствующая регулярному выражению~(\ref{reg:memory_aliases_reg_3}).
\begin{align}
\begin{split}
\label{cfg:memory_aliases_reg_3}
S & \to \varepsilon \\
S & \to a \ S \\
S & \to d \ S \\
\end{split}
\end{align}
\end{ruexample}

\begin{ruexample}
    Рекурсивный автомат, соответствующий регулярному выражению~(\ref{reg:memory_aliases_reg_3}).
\end{ruexample}

    \begin{align}
    \label{rsm:memory_aliases_reg_3}
        \begin{tikzpicture}[node distance=2.7cm,shorten >=1pt,on grid,auto, x=20mm, y=20mm]
           \node[state, initial, accepting] (q_0)   {$0 \{S\}$};
           \path[->]
            (q_0) edge[out=30, in=150, looseness=8, above] node {$a$} (q_0)
            (q_0) edge[out=330, in=210, looseness=8, below] node {$d$} (q_0)
            ;
        \end{tikzpicture}
    \end{align}
