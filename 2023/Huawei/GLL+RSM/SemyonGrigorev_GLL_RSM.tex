%15 min preso!
\documentclass[xcolor=table,aspectratio=169]{beamer}
\usepackage{beamerthemesplit}
\usepackage{wrapfig}
\usetheme{SPbGU}
\usepackage{pdfpages}
\usepackage{amsmath}
\usepackage{cmap}
\usepackage[T2A]{fontenc}
\usepackage[utf8]{inputenc}
\usepackage[english]{babel}
\usepackage{indentfirst}
\usepackage{amsmath}
\usepackage{tikz}
\usepackage{multirow}
\usepackage[noend]{algpseudocode}
\usepackage{algorithm}
\usepackage{algorithmicx}
\usepackage{fancyvrb}
\usepackage{hyperref} 
\usetikzlibrary{calc}
\usetikzlibrary{shapes}
\usetikzlibrary{arrows,automata}
\usetikzlibrary{positioning}
\usetikzlibrary{fit}
\usetikzlibrary{shapes.callouts}
\usetikzlibrary{shapes.misc}
\usepackage{xparse}

\usepackage{etoolbox,refcount}
\usepackage{multicol}

\usepackage{fontawesome5}
\usepackage{fontawesome}
\usepackage{xcolor}
\usepackage{nicematrix}
\usetikzlibrary{arrows,automata,fit,calc, positioning}
\usepackage{soul}
\usepackage[makeroom]{cancel}

\usepackage{tabularx}
\newcolumntype{Y}{>{\raggedleft\arraybackslash}X}

\renewcommand{\thealgorithm}{}

\newtheorem{mytheorem}{Theorem}
\renewcommand{\thealgorithm}{}

\newcommand{\tikzmark}[1]{\tikz[overlay,remember picture] \node (#1) {};}
\def\Put(#1,#2)#3{\leavevmode\makebox(0,0){\put(#1,#2){#3}}}

\newcommand{\ltz}{$< 1$}

\tikzset{
    state/.style={
           rectangle,
           rounded corners,
           draw=black, very thick,
           minimum height=2em,
           inner sep=2pt,
           text centered,
           },
}

\tikzset{
    invisible/.style={opacity=0,text opacity=0},
    visible on/.style={alt=#1{}{invisible}},
    alt/.code args={<#1>#2#3}{%
      \alt<#1>{\pgfkeysalso{#2}}{\pgfkeysalso{#3}} % \pgfkeysalso doesn't change the path
    },
}

\tikzset{cross/.style={cross out, draw=black, minimum size=2*(#1-\pgflinewidth), inner sep=0pt, outer sep=0pt, ultra thick},cross/.default={1pt}}



%callout relative pointer={(230:0.5cm)}]

\newcounter{countitems}
\newcounter{nextitemizecount}
\newcommand{\setupcountitems}{%
  \stepcounter{nextitemizecount}%
  \setcounter{countitems}{0}%
  \preto\item{\stepcounter{countitems}}%
}
\makeatletter
\newcommand{\computecountitems}{%
  \edef\@currentlabel{\number\c@countitems}%
  \label{countitems@\number\numexpr\value{nextitemizecount}-1\relax}%
}
\newcommand{\nextitemizecount}{%
  \getrefnumber{countitems@\number\c@nextitemizecount}%
}
\newcommand{\previtemizecount}{%
  \getrefnumber{countitems@\number\numexpr\value{nextitemizecount}-1\relax}%
}
\makeatother    
\newenvironment{AutoMultiColItemize}{%
\ifnumcomp{\nextitemizecount}{>}{3}{\begin{multicols}{2}}{}%
\setupcountitems\begin{itemize}}%
{\end{itemize}%
\unskip\computecountitems\ifnumcomp{\previtemizecount}{>}{3}{\end{multicols}}{}}


\beamertemplatenavigationsymbolsempty

\title[GLL+RSM]{RSM-based Generalized LL Parsing Algorithm}
\subtitle{All You Need is Generalized GLL}
%\institute[PL\&T@SPbSU]{
%Saint Petersburg State University
%}

% То, что в квадратных скобках, отображается в левом нижнем углу.
\author[Semyon Grigorev]{Semyon Grigorev}

%\date{September 26, 2022}

\begin{document}
{
\begin{frame}[fragile]
  %\begin{table}
  %\centering
  %\includegraphics[height=1.5cm]{pictures/SPbGU_Logo.png}  
  %\end{table}
  \titlepage
\end{frame}
}


% LL
% Combinators, left recursive productions
% ANTLR
% TreeSitter
% Iguana
% GLL
% Error recovery
% Incremental parsing
% Code generation
% Scanerless
% PEG -> GLL
% String-embedded languages processing


\begin{frame}[fragile]
  \frametitle{Recursive State Machine}
  \begin{definition}[Recursive State Machine]
    \textit{Recursive state machine} (RSM) is a tuple $\mathcal{R} = \langle \mathcal{N},\Sigma,B,B_S,Q,Q_S\rangle$
    \begin{itemize}
       \item $N$ is a set of nonterminal symbols 
       \item $\Sigma$ is a set of terminal symbols 
       \item $Q$ is a set of states of the RSM 
       \item $Q_S$ is a set of the start states of all \textit{boxes} 
       \item $B=\{B_{N_i} \mid N_i \in \mathcal{N}\}$ is a set of \textbf{boxes}, where each \\ $B_{N_i} = \langle Q_{N_i}, q_S, Q_F^{N_i}, \delta \rangle$ is a deterministic finite automaton without $\varepsilon$-transitions  
        \begin{itemize}
            \item $Q_{N_i} \subseteq Q$ is a set of states of the box 
            \item $q_S \in Q_{N_i} \cap Q_S$ is the start state of the box
            \item $Q_F^{N_i} \subseteq Q_{N_i}$ is a set of final states of the box 
            \item $\delta \subseteq Q_{N_i} \times  (\Sigma\cup Q_S) \times Q_{N_i}$ is a transition function of the box
        \end{itemize}
    \item $B_S \in B$ is the start box of the RSM
    \end{itemize}
\end{definition}

\end{frame}

\begin{frame}[fragile]
  \frametitle{Configuration of RSM}  
  \begin{definition}[Configuration of RSM]
    A \textit{Configuration} $C_{\mathcal{R}}$ of the computation of the RSM $\mathcal{R}=\langle \mathcal{N},\Sigma,B,B_S,Q,Q_S \rangle$ over the graph $D=\langle V,E,L \rangle$ is a tuple $(q,v,\mathcal{S})$
    \begin{itemize}
        \item $q \in Q$ is a current state of RSM
        \item $\mathcal{S}$ is the current stack, which frames have one of two types
        \begin{itemize} 
            \item Return addresses frame (elements of $Q$) to specify states to continue computation after the call is finished
            \item Parsing tree node to store fragments of a parsing tree
        \end{itemize}
        \item $v \in V$ is the current vertex (current position in the input)
    \end{itemize}
\end{definition}
\end{frame}

\begin{frame}[fragile]
  \frametitle{Transition Step}
  \begin{definition}[Step of RSM Computation]\label{def:rsm_transition}
    A \textit{transition step} of the RSM specifies how to get new configurations of RSM, given the current configuration. $C_{\mathcal{R}} \vdash W$ denotes that $\mathcal{R}$ can go to each configuration in $W$ from the configuration $C_{\mathcal{R}}$
    \begin{align*}
    (q,v,w_0::s::\mathcal{S})  \vdash & \{ (q',v',t::w_0::s::\mathcal{S}) \mid (q,t,q') \in \delta, (v,t,v') \in E\} \\
                       & \cup \{(s', v, q'::w_0::s::\mathcal{S}) \mid (q,s',q') \in \delta, s' \in Q_S \} \\
                       & \cup \{(s,v,\textit{Node}(N_i, w_0)::\mathcal{S}) \mid q \in Q_F^{N_i}\}
    \end{align*}
    , where $w_0$ is a possibly empty sequence of terminals and nonterminal nodes. 
\end{definition}
\end{frame}

\begin{frame}[fragile]
  \frametitle{Extended RSM for grammar $S \to a \ b \mid a \ S \ b$}
  \begin{figure}
    \centering
    \begin{tikzpicture}[->,>=stealth',shorten >=1pt,auto,node distance=2.5cm, semithick]  

  \node[initial,state]   (F)              {$q_4$};
  \node[state]           (G) [right of=F] {$q_5$};
  \node[accepting,state] (H) [right of=G] {$q_6$};
  \node[draw=black, fit= (F) (G) (H), inner sep=0.35cm] (J) {};
  \node[below right] at (J.north west) {S'};

  \path (F) edge              node {$q_0$} (G)
        (G) edge              node {$\$$} (H); 

  \node[initial,state] (A) [below = 2.7cm of F] {$q_0$};
  \node[state]         (B) [right of=A] {$q_1$};
  \node[state]         (D) [above right of=B] {$q_2$};
  \node[accepting,state]         (C) [right of=B] {$q_3$};
  \node [draw=black, fit= (A) (C) (D), inner sep=0.25cm] (E) {};
  \node[below right] at (E.north west) {S};

  \path (A) edge              node {a} (B)
        (B) edge              node {$q_0$} (D)
            edge              node {b} (C)
        (D) edge              node {b} (C);
\end{tikzpicture}    
\end{figure}
\end{frame}


\begin{frame}[fragile]
  \frametitle{Example}
  \begin{minipage}{0.3\textwidth}
  \begin{figure}
    \centering  
  
    \begin{tikzpicture}[->,>=stealth',shorten >=1pt,auto,node distance=1.8cm, semithick]  

  \node[initial,state]   (F)              {$q_4$};
  \node[state]           (G) [right of=F] {$q_5$};
  \node[accepting,state] (H) [right of=G] {$q_6$};
  \node[draw=black, fit= (F) (G) (H), inner sep=0.35cm] (J) {};
  \node[below right] at (J.north west) {S'};

  \path (F) edge              node {$q_0$} (G)
        (G) edge              node {$\$$} (H); 

  \node[initial,state] (A) [below = 2.7cm of F] {$q_0$};
  \node[state]         (B) [right of=A] {$q_1$};
  \node[state]         (D) [above right of=B] {$q_2$};
  \node[accepting,state]         (C) [right of=B] {$q_3$};
  \node [draw=black, fit= (A) (C) (D), inner sep=0.25cm] (E) {};
  \node[below right] at (E.north west) {S};

  \path (A) edge              node {a} (B)
        (B) edge              node {$q_0$} (D)
            edge              node {b} (C)
        (D) edge              node {b} (C);
\end{tikzpicture}  

\begin{tikzpicture}[->,>=stealth',shorten >=1pt,auto,node distance=2.8cm, semithick]
      
  \node[state] (A)                    {$v_0$};
  \node[state]         (B) [right of=A] {$v_1$};
  
  \path (A) edge  [loop left] node {a} (A)
            edge  [bend left] node {b} (B)
        (B) edge  [bend left] node {b} (A);
\end{tikzpicture}

\end{figure}
\end{minipage}
\begin{minipage}{0.69\textwidth}
  \begin{align}
    \setcounter{equation}{0}
    \action<+->{(q_4,v_0,[])     \vdash & \{(q_0,v_0,[q_5])\} \\}
    \action<+->{(q_0,v_0,[q_5])  \vdash & \{(q_1,v_0,[a,q_5])\} \\}
    \action<+->{(q_1,v_0,[a,q_5])\vdash & \{(q_0,v_0,[q_2,a,q_5]) \nonumber\\
                            & , (q_3,v_1,[b,a,q_5])\} \\}
    \action<+->{(q_0,v_0,[q_2,a,q_5]) \vdash & \{(q_1,v_0,[a,q_2,a,q_5])\} \\}
    \action<+->{(q_3,v_1,[b,a,q_5])   \vdash & \boxed{\{(q_5,v_1,[S(a,b)])\}}\\}
    \action<+->{(q_1,v_0,[a,q_2,a,q_5]) \vdash & \{(q_3,v_1,[b,a,q_2,a,q_5]) \nonumber \\
                                   & , (q_0,v_0,[q_2,a,q_2,a,q_5])\}\\}
    \action<+->{(q_3,v_1,[b,a,q_2,a,q_5]) \vdash & \{(q_2,v_1,[S(a,b),a,q_5])\} \\}
    \action<+->{(q_2,v_1,[S(a,b),a,q_5]) \vdash & \{(q_3,v_0,[b,S(a,b),a,q_5])\} \\}
    \action<+->{(q_3,v_0,[b,S(a,b),a,q_5]) \vdash & \boxed{\{(q_5,v_0,[S(a,S(a,b),b)])\}}}
    \end{align}  
  \end{minipage}    
\end{frame}

\begin{frame}[fragile]{Paths Representation}
  \begin{definition}[Matched Range]
    \textit{Range} or a \textit{matched range} $R^{q_i,v_i}_{q_j,v_j}$ corresponds to the fact that there is a chain of transitions from the configuration $(q_i,v_i,\mathcal{S}_i)$ to the configuration $(q_j,v_j,\mathcal{S}_j)$, or $(q_i,v_i,\mathcal{S}_i) \vdash^* (q_j,v_j,\mathcal{S}_j)$. The symbol $\epsilon$ denotes the empty range: an empty sequence of configurations.
  \end{definition}
  \pause
  \begin{definition}[Path Index]
    \textit{Path index} for the graph $D=\langle V,E,L \rangle$ and the query $\mathcal{R}=\langle \mathcal{N}, \Sigma, B, B_S, Q, Q_S \rangle$ is a $\mathcal{K} \times \mathcal{K}$ square matrix $\mathcal{I}$, $\mathcal{K} = |Q|\cdot|V|$.
    Columns and rows are indexed by tuples $(q_i, v_j), q_i \in Q, v_j \in V$. $\mathcal{I}[(q_i,v_j),(q_l,v_k)]$ is a set which represents information about a $R^{q_i,v_j}_{q_l,v_k}$:
\begin{itemize}
    \item $t \in \Sigma$ means that $(q_i,v_j,\mathcal{S}') \vdash (q_l,v_k,t::\mathcal{S}') $; 
    \item $N_m \in \mathcal{N}$ means that $(q_i,v_j,\mathcal{S}') \vdash^* (q_l,v_k,N_m::\mathcal{S}')$; 
    \item $I_{q_a,v_b}$ is an \textit{intermediate point}: $R^{q_i,v_j}_{q_l,v_k}$ was combined from $R^{q_i,v_j}_{q_a,v_b}$ and $R^{q_a,v_b}_{q_l,v_k}$ $$(q_i,v_j,\mathcal{S}') \vdash^* (q_a,v_b,\mathcal{S}'') \vdash^* (q_l,v_k,\mathcal{S}''')$$
\end{itemize}
  \end{definition}
\end{frame}

\begin{frame}[fragile]
  \begin{center}
      
  \scalebox{.65}{
    \begin{tabular}{l c c c}
      \hline \\
      & 
      Configuration
      &
      Path
      &
      Stack
      \\
      \hline \action<+->{ \\  
      1 
      &
      $\vdash (q_4,v_0,(q_4,v_0),\epsilon)$ 
      & 
      
      &
      \vspace{-0.2cm}
      \begin{tikzpicture}[->,>=stealth',shorten >=1pt,auto,node distance=2.8cm, semithick]
        \node[state]         (A)              {$q_4,v_0$};
      \end{tikzpicture} 
      \\
      \\
      \hline} \action<+->{ \\ 
      2 &
      $(q_4,v_0,(q_4,v_0),\epsilon) \vdash \{(q_0,v_0,(q_0,v_0),\epsilon)\}$ 
      &
      
      &
      \multirow{2}{*}[0.15cm]{
      \begin{minipage}[t]{0.2\textwidth}
      \begin{tikzpicture}[->,>=stealth',shorten >=1pt,auto,node distance=2.8cm, semithick]
        \node[state]         (A)              {$q_4,v_0$};
        \node[state]         (B) [right of=A] {$q_0,v_0$};
        \path (B) edge         node {$q_5, \epsilon $} (A);
      \end{tikzpicture}
      \end{minipage}
      } 
      \\}\action<+->{
      3 & 
      $(q_0,v_0,(q_0,v_0),\epsilon) \vdash \{(q_1,v_0,(q_0,v_0),R^{q_0,v_0}_{q_1,v_0})\}$ 
      &
      $[(q_0,v_0),(q_1,v_0)] \xleftarrow{} a$
      & 
      \\
      \\
      \hline }\action<+->{
      \\
      4 
      &
      $(q_1,v_0,(q_0,v_0),R^{q_0,v_0}_{q_1,v_0}) \vdash \{\xcancel{(q_0,v_0,(q_0,v_0),\epsilon)}, (q_3,v_1,(q_0,v_0),R^{q_0,v_0}_{q_3,v_1})\}$ 
      &
      \begin{minipage}[t]{0.2\textwidth}
      $\begin{array}{l}
      \left[(q_1,v_0),(q_3,v_1)\right] \xleftarrow{} b \\ 
      \left[(q_0,v_0),(q_3,v_1)\right] \xleftarrow{} (q_1,v_0) \\
      \end{array}
      $
      \end{minipage}
      &
      \multirow{2}{*}[0.2cm]{
      \begin{minipage}[t]{0.2\textwidth}
        \scalebox{.95}{
      \begin{tikzpicture}[->,>=stealth',shorten >=1pt,auto,node distance=2.8cm, semithick]
        \node[state]         (A)              {$q_4,v_0$};
        \node[state]         (B) [above of=A] {$q_0,v_0$};
        \path (B) edge         node {$q_5, \epsilon$} (A)
              (B) edge  [loop] node[above]{$q_2, R^{q_0,v_0}_{q_1,v_0}$} (B) 
               ;
      \end{tikzpicture}
        }
      \end{minipage}
      }
      \\
      
      &
      \\}\action<+->{
      5 
      & 
      $(q_3,v_1,(q_0,v_0),R^{q_0,v_0}_{q_3,v_1}) \vdash \{(q_2,v_1,(q_0,v_0),R^{q_0,v_0}_{q_2,v_1}), \boxed{(q_5,v_1,(q_4,v_0),R^{q_4,v_0}_{q_5,v_1})}\}$ 
      &
      \begin{minipage}[t]{0.2\textwidth}
      $\begin{array}{l}
      \left[(q_4,v_0),(q_5,v_1)\right] \xleftarrow{} S \\
      \left[(q_1,v_0),(q_2,v_1)\right] \xleftarrow{} S \\
      \left[(q_0,v_0),(q_2,v_1)\right] \xleftarrow{} (q_1,v_0) 
      \end{array}
      $
      \end{minipage}
      
      &
      \\
      \\}\action<+->{
      6 
      & 
      $(q_2,v_1,(q_0,v_0),R^{q_0,v_0}_{q_2,v_1}) \vdash \{(q_3,v_0,(q_0,v_0),R^{q_0,v_0}_{q_3,v_0})\}$ 
      &
      \begin{minipage}[t]{0.2\textwidth}
      $\begin{array}{l}
      \left[(q_2,v_1),(q_3,v_0)\right] \xleftarrow{} b \\
      \left[(q_0,v_0),(q_3,v_0)\right] \xleftarrow{} (q_2,v_1)
      \end{array}
      $
      \end{minipage}
      &
      \\
      \\}\action<+->{
      7 
      & 
      $(q_3,v_0,(q_0,v_0),R^{q_0,v_0}_{q_3,v_0}) \vdash \{(q_2,v_0,(q_0,v_0),R^{q_0,v_0}_{q_2,v_0}), \boxed{(q_5,v_0,(q_4,v_0),R^{q_4,v_0}_{q_5,v_0})}\}$ 
      &
      \begin{minipage}[t]{0.2\textwidth}
      $\begin{array}{l}
      \left[(q_4,v_0),(q_5,v_0)\right] \xleftarrow{} S \\
      \left[(q_1,v_0),(q_2,v_0)\right] \xleftarrow{} S \\
      \left[(q_0,v_0),(q_2,v_0)\right] \xleftarrow{} (q_1,v_0)
      \end{array}
      $
      \end{minipage}
      &
      \\
      \\}\action<+->{
      8 
      & 
      $(q_2,v_0,(q_0,v_0),R^{q_0,v_0}_{q_2,v_0}) \vdash \{\xcancel{(q_3,v_1,(q_0,v_0),R^{q_0,v_0}_{q_3,v_1})}\}$ 
      &
      \begin{minipage}[t]{0.2\textwidth}
      $\begin{array}{l}
      \left[(q_0,v_0),(q_3,v_1)\right] \xleftarrow{} (q_2,v_0) \\
      \left[(q_2,v_0),(q_3,v_1)\right] \xleftarrow{} b
      \end{array}
      $
      \end{minipage}
      &
      \\
      \\
      
      \hline}
      
      \end {tabular}
      
  }
  \end{center}
\end{frame}

\begin{frame}[fragile]{Path Index}
  \setcounter{MaxMatrixCols}{30}
  \begin{center}
    \scalebox{.65}{
      $\begin{pNiceMatrix}[first-row,last-row,first-col,last-col,nullify-dots]
        & q_0,v_0 & q_1,v_0 & q_2,v_0       & q_3,v_0       & q_4,v_0 & q_5,v_0 & q_6,v_0 & q_0,v_1 & q_1,v_1 & q_2,v_1         & q_3,v_1       & q_4,v_1    & q_5,v_1 & q_6,v_1 &          \\
q_0,v_0 &         & \{a\}   &\{I_{q_1,v_0}\}&\{I_{q_2,v_1}\}&         &         &         &         &         & \{I_{q_1,v_0}\} &\{I_{q_1,v_0}, I_{q_2,v_0}\}&         &         &         & q_0,v_0\\
q_1,v_0 &         &         &  \{S\}        &               &         &         &         &         &         &   \{S\}         & \{b\}         &         &         &         & q_1,v_0\\
q_2,v_0 &         &         &               &               &         &         &         &         &         &                 & \{b\}        &         &         &         & q_2,v_0\\
q_3,v_0 &         &         &         &               &         &         &         &         &         &                 &         &         &         &         & q_3,v_0\\
q_4,v_0 &         &         &         &               &         & \{S\}   &         &         &         &                 &         &         &\{S\}    &         & q_4,v_0\\
q_5,v_0 &         &         &         &               &         &         &         &         &         &                 &         &         &         &         & q_5,v_0\\
q_6,v_0 &         &         &         &               &         &         &         &         &         &                 &         &         &         &         & q_6,v_0\\
q_0,v_1 &         &         &         &               &         &         &         &         &         &                 &         &         &         &         & q_0,v_1\\
q_1,v_1 &         &         &         &               &         &         &         &         &         &                 &         &         &         &         & q_1,v_1\\
q_2,v_1 &         &         &         &   \{b\}       &         &         &         &         &         &                 &         &         &         &         & q_2,v_1\\
q_3,v_1 &         &         &         &         &         &         &         &         &         &         &         &         &         &         & q_3,v_1\\
q_4,v_1 &         &         &         &         &         &         &         &         &         &         &         &         &         &         & q_4,v_1\\
q_5,v_1 &         &         &         &         &         &         &         &         &         &         &         &         &         &         & q_5,v_1\\
q_6,v_1 &         &         &         &         &         &         &         &         &         &         &         &         &         &         & q_6,v_1\\
        & q_0,v_0 & q_1,v_0 & q_2,v_0 & q_3,v_0 & q_4,v_0 & q_5,v_0 & q_6,v_0 & q_0,v_1 & q_1,v_1 & q_2,v_1 & q_3,v_1 & q_4,v_1 & q_5,v_1 & q_6,v_1 &         \\
\end{pNiceMatrix}$
    }
  \end{center}

\end{frame}

\begin{frame}[fragile]
  %\frametitle{Summary}  
  \begin{figure}
    \centering
    \scalebox{.95}{
    \begin{tikzpicture}[->,>=stealth',shorten >=1pt,auto,node distance=1.5cm, semithick,scale=0.8, every node/.style={scale=0.8}]
      %\tikzstyle{every state}=[fill=red,draw=none,text=white]
    
      \node[state]         (A)              {$R^{q_4,v_0}_{q_5,v_0}$};
      \node[state]         (B) [right of=A] {$S$};
      \node[state]         (C) [right of=B] {$R^{q_0,v_0}_{q_3,v_0}$};
      \node[state]         (D) [below of=C] {$I_{q_2,v_1}$};
      \node[state]         (E) [below of=D] {$R^{q_0,v_0}_{q_2,v_1}$};
      \node[state]         (F) [right of=D] {$R^{q_2,v_1}_{q_3,v_0}$};
      \node[state]         (G) [below of=E] {$I_{q_1,v_0}$};
      \node[state]         (H) [below of=F] {$b$};
      \node[state]         (J) [below of=G] {$R^{q_0,v_0}_{q_1,v_0}$};
      \node[state]         (I) [right of=G] {$R^{q_1,v_0}_{q_2,v_1}$};
      \node[state]         (K) [below left of=J] {$a$};
      \node[state]         (L) [right of=I] {$I_{q_1,v_0}$};
      \node[state]         (M) [right of=L] {$S$};
      \node[state]         (N) [right of=M] {$R^{q_0,v_0}_{q_3,v_1}$};
      \node[state]         (O) [below of=N] {$I_{q_1,v_0}$};
      \node[state]         (P) [right of=O] {$I_{q_2,v_0}$};
      \node[state]         (R) [below of=P] {$R^{q_0,v_0}_{q_2,v_0}$};
      \node[state]         (S) [right of=R] {$R^{q_2,v_0}_{q_3,v_1}$};
      \node[state]         (T) [below of=O] {$R^{q_1,v_0}_{q_3,v_1}$};
      \node[state]         (U) [below of=T] {$b$};
      \node[state]         (V) [below of=S] {$b$};
      \node[state]         (W) [below of=R] {$I_{q_1,v_0}$};      
      \node[state]         (X) [below right of=W] {$R^{q_1,v_0}_{q_2,v_0}$};
      \node[state]         (Y) [right of=X] {$S$};
      
      \path (A) edge          (B)
            (B) edge   (C)
            (C) edge   (D)
            (D) edge   (E)
            (D) edge   (F)
            (E) edge   (G)
            (F) edge   (H)
            (G) edge   (J)
            (G) edge   (I)
            (J) edge   (K)
            (I) edge   (L)
            (L) edge   (M)
            (M) edge   (N)
            (N) edge   (O)
            (N) edge   (P)
            (O) edge   (J)
            (O) edge   (T)
            (P) edge   (R)
            (R) edge   (W)
            (P) edge   (S)
            (T) edge   (U)
            (S) edge   (V)
            (W) edge[in=270, out=220]   (J)
            (W) edge   (X)
            (X) edge   (Y)
            (Y) edge[in=0, out=89]   (C)
            
             ;
    \end{tikzpicture}
      }
    \end{figure}
\end{frame}


\end{document}
