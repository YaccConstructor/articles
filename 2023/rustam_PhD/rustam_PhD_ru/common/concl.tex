%% Согласно ГОСТ Р 7.0.11-2011:
%% 5.3.3 В заключении диссертации излагают итоги выполненного исследования, рекомендации, перспективы дальнейшей разработки темы.
%% 9.2.3 В заключении автореферата диссертации излагают итоги данного исследования, рекомендации и перспективы дальнейшей разработки темы.

\begin{enumerate}[beginpenalty=10000] % https://tex.stackexchange.com/a/476052/104425
	\item Разработан подход к поиску путей в графе с заданными КС-ограничениями на основе методов линейной алгебры, который позволяет использовать теоретические и практические достижения линейной алгебры для решения данной задачи.
	\item Разработан алгоритм, использующий предложенный подход и решающий задачи поиска путей в графе с заданными КС-ограничениями. Доказана завершаемость и корректность предложенного алгоритма. Получена теоретическая оценка сверху временной сложности алгоритма. Предложенный алгоритм использует операции над матрицами, которые позволяют применять широкий класс оптимизаций и дают возможность автоматически распараллеливать вычисления за счёт существующих библиотек линейной алгебры.
	\item Разработан алгоритм поиска путей в графе с заданными КС-ограничениями, использующий предложенный подход и не требующий преобразования входной КС-грамматики. Доказана завершаемость и корректность предложенного алгоритма. Получена теоретическая оценка сверху временной сложности алгоритма. Предложенный алгоритм позволяет работать с произвольными входными КС-грамматиками без необходимости их преобразования, что позволяет избежать значительного увеличения размеров входной грамматики и увлечения времени работы алгоритма.
	\item Предложенные алгоритмы реализованы с использованием параллельных вычислений. Проведено экспериментальное исследование разработанных алгоритмов на реальных RDF данных и графах, построенных для статического анализа программ. Было проведено сравнение полученных реализаций между собой, с существующими решениями из области статического анализа и с решениями, основанными на различных алгоритмах синтаксического анализа. Результаты сравнения показывают, что предложенные реализации для задачи достижимости позволяют ускорить время анализа до 2 порядков и потребляют до 2 раз меньше памяти по сравнению с существующими решениями, а для задач поиска одного и поиска всех путей в графе позволяют ускорить время анализа до 3 порядков и до 2 порядков снизить потребление памяти.
\end{enumerate}
