\chapter{Сравнение и соотнесение}\label{ch:ch6}
В данной главе представлено сравнение полученных результатов с основными существующими решениями задачи поиска путей в графе с заданными КС-ограничениями. Описание существующих решений представлено в разделе~\ref{sec:ch1/sec5} данной работы.

В качестве инструментов, с которыми производилось сравнение, выбраны следующие инструменты, позволяющие анализировать графы с произвольными КС-грамматиками в качестве ограничений на пути: $\textit{LL}$~\cite{medeiros2018efficient}, основанный на алгоритме LL(1); $\textit{GLL}_{\textit{R}}$ и $\textit{GLL}_{\textit{A}}$~\cite{grigorev2017context}, основанные на алгоритме синтаксического анализа GLL; $\textit{Graspan}$~\cite{graspan}, который специализируется на статическом анализе программ. Сравнение производилось с предложенными реализациями $\textit{MtxReach}_{\textit{CPU}}$, $\textit{MtxReach}_{\textit{GPU}}$, $\textit{MtxSingle}_{\textit{CPU}}$ и $\textit{MtxAll}_{\textit{CPU}}$, основанными на умножении матриц, и реализациями $\textit{KronAll}_{\textit{CPU}}$ и $\textit{KronAll}_{\textit{GPU}}$, основанными на произведении Кронекера. Для сравнения были выбраны критерии, представленные в~\cref{tab:compCriteria}. 

\begin{table} [h]
  \centering
   \begin{threeparttable}% выравнивание подписи по границам таблицы
  \caption{Критерии сравнения инструментов для поиска путей в графе с заданными КС-ограничениями}\label{tab:compCriteria}
  \begin{tabular}{| p{4.5cm} | p{3cm} | p{8cm} |}
  \hline                               
  \hline
  Критерии & Название колонки в таблице с результатами сравнения~\ref{tab:comparison} & Описание \\
  \hline
  Отсутствие необходимости преобразования входной КС-грамматики & Без преобразований & Отсутствует ли необходимость преобразования входной КС-грамматики, например в нормальную форму, для применения алгоритма и соответствующего инструмента?\\
  Восстановление путей  & Восст. путей  & Вычисляет ли алгоритм и соответствующий инструмент информацию, достаточную для восстановления одного или всех найденных путей в графе, соответствующих входным КС-ограничениям?\\
  Использование GPU & GPU & Производит ли инструмент вычисления на GPU?\\
  Использование методов линейной алгебры     & Лин. алгебра & Используются ли в алгоритме и реализованы ли в соответствующем инструменте методы линейной алгебры?\\
  \hline
  \hline
  \end{tabular}
  \end{threeparttable}
\end{table}

В таблице~\ref{tab:comparison} приведены основные результаты сравнения, которые позволяют сделать следующие выводы.

\begin{itemize}
    \item На текущий момент не существует алгоритмов поиска путей в графе с заданными произвольными КС-ограничениями, использующих методы линейной алгебры, кроме предложенных в данной работе.
    \item Существующие инструменты в основном реализованы на CPU.
    \item В рамках данной работы была предложена первая реализация на GPU алгоритма, решающего задачи поиска одного и всех путей в графе с заданными КС-ограничениями.
\end{itemize}

\begin{table} [h]
  \centering

\begin{threeparttable}
\caption{Сравнение инструментов для поиска путей в графе с заданными КС-ограничениями}\label{tab:comparison}
  
  \begin{tabular}{| p{3.5cm} || p{2.4cm} | p{2.4cm} | p{2.4cm} | p{2.4cm}l |}
  \hline                               
  \hline
  {Инструмент}              &\centering {Без преобразований}        &\centering {Восст. путей}    &\centering {GPU} &\centering {Лин. алгебра}  & \\
  \hline
  $\textit{LL}$                         &\centering  $+$                  &\centering  $-$             &\centering  $-$   &\centering  $-$  &\\
  $\textit{GLL}_{\textit{R}}$                &\centering  $+$                  &\centering  $-$              &\centering  $-$   &\centering  $-$    & \\
  $\textit{GLL}_{\textit{A}}$                &\centering  $+$                  &\centering  $+$\tnote{*}              &\centering  $-$   &\centering  $-$    & \\
  $\textit{Graspan}$                       &\centering  $-$\tnote{**}                   &\centering  $-$             &\centering  $+$\tnote{***} &\centering  $-$       &\\
  $\textit{MtxReach}_{\textit{CPU}}$                           &\centering  $-$                  &\centering  $-$             &\centering  $-$  &\centering  $+$   &\\
  $\textit{MtxReach}_{\textit{GPU}}$                          &\centering  $-$                  &\centering  $-$             &\centering  $+$  &\centering  $+$   &\\
  $\textit{MtxSingle}_{\textit{CPU}}$                           &\centering  $-$                  &\centering  $+$             &\centering  $-$\tnote{****}  &\centering  $+$   &\\
  $\textit{MtxAll}_{\textit{CPU}}$                           &\centering  $-$                  &\centering  $+$             &\centering  $-$\tnote{****}  &\centering  $+$   &\\
  $\textit{KronAll}_{\textit{CPU}}$                      &\centering  $+$                  &\centering  $+$             &\centering  $-$  &\centering  $+$    &\\
  $\textit{KronAll}_{\textit{GPU}}$                      &\centering  $+$                  &\centering  $+$             &\centering  $+$  &\centering  $+$    &\\
  \hline
  \hline
  \end{tabular}\small{
  \begin{tablenotes}
            \item[*] Рассматриваемая реализация алгоритма~\cite{grigorev2017context}, основанного на алгоритме синтаксического анализа GLL, строит лес разбора SPPF, который хранит в себе информацию, достаточную для восстановления всех найденных путей в графе.
            \item[**] Для использования инструмента $\textit{Graspan}$ требуется преобразовать входную КС-грамматику в ослабленную нормальную форму Хомского~\cite{graspan}.
            \item[***] Для инструмента $\textit{Graspan}$ также существует реализация на GPU, которую автору не удалось запустить. В работе~\cite{graspan} утверждается, что такая реализация позволяет ускорить анализ указателей в некоторых программах на языках \texttt{C/C++} от 3.5 до 650 раз.
            \item[****] На данный момент алгоритм, основанный на умножении матриц, реализован на GPU только для задачи достижимости. Однако, например, для задач поиска одного или всех путей в графе такой алгоритм в будущем возможно удастся получить высокопроизводительную реализацию на GPU с использованием библиотеки CUSP или библиотеки GraphBLAST.
  \end{tablenotes}    }
  \end{threeparttable}
\end{table}


\clearpage