\chapter*{Список сокращений и условных обозначений} % Заголовок
\addcontentsline{toc}{chapter}{Список сокращений и условных обозначений}  % Добавляем его в оглавление
\noindent
%\begin{longtabu} to \dimexpr \textwidth-5\tabcolsep {r X}
\begin{longtabu} to \textwidth {r X}
% Жирное начертание для математических символов может иметь
% дополнительный смысл, поэтому они приводятся как в тексте
% диссертации
%\(\begin{rcases}
%a_n\\
%b_n
%\end{rcases}\)  &
%\begin{minipage}{\linewidth}
%коэффициенты разложения Ми в дальнем поле соответствующие
%электрическим и магнитным мультиполям
%\end{minipage}
%\\
%\({\boldsymbol{\hat{\mathrm e}}}\) & единичный вектор \\
%\(E_0\) & амплитуда падающего поля\\

%\textbf{BEM} & boundary element method, метод граничных элементов\\
%\textbf{CST MWS} & Computer Simulation Technology Microwave Studio
%программа для компьютерного моделирования уравнений Максвелла\\

\end{longtabu}
\addtocounter{table}{-1}% Нужно откатить на единицу счетчик номеров таблиц, так как предыдующая таблица сделана для удобства представления информации по ГОСТ
