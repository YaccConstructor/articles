\chapter*{Заключение}                       % Заголовок
\addcontentsline{toc}{chapter}{Заключение}  % Добавляем его в оглавление

%% Согласно ГОСТ Р 7.0.11-2011:
%% 5.3.3 В заключении диссертации излагают итоги выполненного исследования, рекомендации, перспективы дальнейшей разработки темы.
%% 9.2.3 В заключении автореферата диссертации излагают итоги данного исследования, рекомендации и перспективы дальнейшей разработки темы.
%% Поэтому имеет смысл сделать эту часть общей и загрузить из одного файла в автореферат и в диссертацию:

\textbf{Основные результаты работы заключаются в следующем.}
%% Согласно ГОСТ Р 7.0.11-2011:
%% 5.3.3 В заключении диссертации излагают итоги выполненного исследования, рекомендации, перспективы дальнейшей разработки темы.
%% 9.2.3 В заключении автореферата диссертации излагают итоги данного исследования, рекомендации и перспективы дальнейшей разработки темы.

\begin{enumerate}[beginpenalty=10000] % https://tex.stackexchange.com/a/476052/104425
	\item Разработан подход к поиску путей в графе с заданными КС-ограничениями на основе методов линейной алгебры, который позволяет использовать теоретические и практические достижения линейной алгебры для решения данной задачи.
	\item Разработан алгоритм, использующий предложенный подход и решающий задачи поиска путей в графе с заданными КС-ограничениями. Доказана завершаемость и корректность предложенного алгоритма. Получена теоретическая оценка сверху временной сложности алгоритма. Предложенный алгоритм использует операции над матрицами, которые позволяют применять широкий класс оптимизаций и дают возможность автоматически распараллеливать вычисления за счёт существующих библиотек линейной алгебры.
	\item Разработан алгоритм поиска путей в графе с заданными КС-ограничениями, использующий предложенный подход и не требующий преобразования входной КС-грамматики. Доказана завершаемость и корректность предложенного алгоритма. Получена теоретическая оценка сверху временной сложности алгоритма. Предложенный алгоритм позволяет работать с произвольными входными КС-грамматиками без необходимости их преобразования, что позволяет избежать значительного увеличения размеров входной грамматики и увлечения времени работы алгоритма.
	\item Предложенные алгоритмы реализованы с использованием параллельных вычислений. Проведено экспериментальное исследование разработанных алгоритмов на реальных RDF данных и графах, построенных для статического анализа программ. Было проведено сравнение полученных реализаций между собой, с существующими решениями из области статического анализа и с решениями, основанными на различных алгоритмах синтаксического анализа. Результаты сравнения показывают, что предложенные реализации для задачи достижимости позволяют ускорить время анализа до 2 порядков и потребляют до 2 раз меньше памяти по сравнению с существующими решениями, а для задач поиска одного и поиска всех путей в графе позволяют ускорить время анализа до 3 порядков и до 2 порядков снизить потребление памяти.
\end{enumerate}


На основе полученных реализаций была создана платформа $\textit{CFPQ\_PyAlgo}$ для разработки и тестирования алгоритмов поиска путей в графе с заданными КС-ограничениями.

В рамках \textbf{рекомендации по применению результатов работы} в индустрии и научных исследованиях укажем, что разработанные подход и алгоритмы, полученные на его основе, применимы для поиска путей в графах с заданными КС-ограничениями с использованием методов линейной алгебры, а также что могут быть получены компактные, переносимые и высокопроизводительные реализации предложенных алгоритмов на различных параллельных вычислительных системах с использованием существующих библиотек линейной алгебры. Платформа полученная на основе предлагаемых реализаций, может быть использована для статического анализа программ~\cite{rehof2001type,zheng2008demand}, сетевого анализа~\cite{zhang2016context}, в биоинформатике~\cite{sevon2008subgraph} и т.д. Кроме того, полученные реализации могут быть интегрированы с такими графовыми базами данных, как RedisGraph. В работе~\cite{azimov2} были предложены прототипы таких реализаций. Однако для полноценной интеграции с графовой базой данных RedisGraph необходимо расширить язык запросов Cypher, используемый этой базой данных. И расширение синтаксиса этого языка запросов, позволяющее выразить КС-ограничения на пути в графе, было предложено$\footnote{Предложение для расширения синтаксиса языка запросов openCypher:\\ https://github.com/thobe/openCypher/blob/rpq/cip/1.accepted/CIP2017-02-06-Path-Patterns.adoc (date of access: 14.01.2022).}$.

Также определим \textbf{перспективы дальнейшей разработки тематики}, основной из которых является применение предложенного подхода и разработанных алгоритмов для создания специализированных инструментов для решения перечисленных прикладных задач. На основе предложенного подхода могут быть созданы алгоритмы, которые учитывают специфику графов и ограничений на пути в тех или иных прикладных областях. Например, для статического анализа программ может быть учтена структура графов, получаемых из программ на конкретном языке программирования, а также структура КС-грамматик, порождающих конкретный КС-язык для выбранного анализа (анализа указателей~\cite{zheng2008demand}, поиска уязвимостей в программе~\cite{taint} и т.д.).

На практике задачи поиска путей в больших графах редко решаются без фиксации сравнительно небольшого множества возможных начальных и/или конечных вершин в искомых путях. Таким образом, часто информация о путях между любыми вершинами является избыточной. Поэтому ещё одним направлением, представляющим интерес, является модификация всех предложенных и создание новых алгоритмов для решения задач поиска путей в графе с дополнительными ограничениями на множества начальных и конечных вершин.

В данной работе предложенный алгоритм, основанный на умножении матриц, для задач поиска одного и всех путей в графе был реализован только на CPU. Для решения этих задач в алгоритме были использованы более сложные типы для элементов матриц, что приводит к существенному увеличению затрат по памяти и не позволяет получить высокопроизводительную реализацию на GPU. Поэтому необходимо проведение дальнейших исследований по оптимизации этого алгоритма и использованных в нём алгебраических структур. В дальнейшем, возможно, удастся получить высокопроизводительные реализации на GPU этого алгоритма с использованием библиотеки CUSP или библиотеки GraphBLAST.

Кроме того, известны некоторые задачи анализа графов, ограничения на пути в которых не могут быть выражены с помощью КС-языка. Например, такая задача возникает в области статического анализа программ, когда кроме анализа указателей производится анализ путей исполнений программы на корректную последовательность вызовов её подпрограмм и возвратов из них. Например, согласно работе~\cite{linearconjunctive} такая задача является неразрешимой, однако для аппроксимации результата такого анализа может быть использованы ограничения на пути в виде линейных конъюнктивных языков~\cite{okhotin2001conjunctive}, которые принадлежат более широкому классу, чем контекстно-свободные. Таким образом, актуальным направлением является расширение предложенного подхода для решения задач поиска путей в графе с ограничениями, выраженными языками из более широких классов, чем класс КС-языков.



%И какая-нибудь заключающая фраза.

%Последний параграф может включать благодарности.  В заключение автор
%выражает благодарность и большую признательность научному руководителю
%Иванову~И.\,И. за поддержку, помощь, обсуждение результатов и~научное
%руководство. Также автор благодарит Сидорова~А.\,А. и~Петрова~Б.\,Б.
%за помощь в~работе с~образцами, Рабиновича~В.\,В. за предоставленные
%образцы и~обсуждение результатов, Занудятину~Г.\,Г. и авторов шаблона
%*Russian-Phd-LaTeX-Dissertation-Template* за~помощь в оформлении
%диссертации. Автор также благодарит много разных людей
%и~всех, кто сделал настоящую работу автора возможной.
