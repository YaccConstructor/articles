\chapter{Экспериментальное исследование}\label{ch:ch5}
Завершаемость и корректность предложенных алгоритмов формально доказаны выше, однако их производительность требует экспериментальной оценки. При этом основной интерес представляет оценка и сравнение с существующими решениями на входных данных, близких к реальным.

\section{Постановка экспериментов}\label{sec:ch5/sect1}

В данной работе было показано, как предложенный подход может быть использован для получения алгоритмов поиска путей в графе с заданными КС-ограничениями и реализации этих алгоритмов. \textit{Цель} этого экспериментального исследования~--- ответить на следующие \textit{вопросы}.

\begin{enumerate}
    \item [\textbf{[В1]}] Как показывают себя на практике полученные реализации для задачи достижимости в сравнении с существующими решениями на реальных данных?
    \item [\textbf{[В2]}] Как показывают себя на практике полученные реализации для задач поиска одного и поиска всех путей в графе по сравнению с существующими решениями на реальных данных?
    \item [\textbf{[В3]}] Как сказывается на производительности хранение в полученных реализациях информации о найденных путях в сравнении с полученными реализациями для задачи достижимости?
    \item [\textbf{[В4]}] Как показывают себя на практике полученные реализации, не требующие преобразований входной КС-грамматики, по сравнению с другими предложенными реализациями?
\end{enumerate}

Чтобы ответить на поставленные вопросы были выбраны следующие \textit{метрики}.

\begin{enumerate}
    \item [\textbf{[М1]}] Время работы реализаций.
    \item [\textbf{[М2]}] Затрачиваемая реализациями память.
\end{enumerate}

Для данного экспериментального исследования были выбраны графы, полученные из реальных RDF данных и программ, написанных на языках \texttt{C/C++}. Характеристики этих графов представлены в~\cref{tab:RDFgraphs} и~\cref{tab:Cgraphs}. В этих таблицах приведено: номер графов для более компактного их упоминания; количество вершин и дуг графов; а также количество дуг с метками, которые в дальнейшем будут фигурировать в КС-ограничениях для этих графов.

\begin{table} [htbp]
    \centering
    \begin{threeparttable}% выравнивание подписи по границам таблицы
        \caption{Характеристики графов для анализа RDF данных~\cite{zhang2016context}\tnote{*}}\label{tab:RDFgraphs}%
        \begin{tabular}{| p{1cm} || p{3cm} | p{2.2cm} | p{2.2cm} | p{3cm} | p{3cm}l |}
            \hline
            \hline
            \centering \textnumero & \centering Граф   &  \centering $|V|$ & \centering $|E|$ & \centering  $\#\textit{subClassOf}$ & \centering  $\#\textit{type}$ &\\
            \hline
            %\centering 1 & atom &  \centering	291 & \centering	425 & \centering	122 & \centering	138 & \\
            %\centering 2 & biomedical &  \centering	341 & \centering	459 & \centering	122 & \centering	130& \\
            %\centering 3 & core  & \centering	1,323 & \centering	2,752 & \centering	178 & \centering	706& \\
            \centering 1 & eclass &  \centering	239,111 & \centering	360,248 & \centering	90,962 & \centering	72,517& \\
            %\centering 5 & enzyme & \centering	48,815 & \centering	86,543 & \centering	8,163 & \centering	14,989& \\
            %\centering 6 & foaf & \centering	256 & \centering	631 & \centering	10 & \centering	174& \\
            %\centering 7 & funding & \centering	778 & \centering	1,086 & \centering	90 & \centering	304& \\
            %\centering 8 & generations & \centering	129 & \centering	273 & \centering	0 & \centering	78& \\
            \centering 2 & go & \centering	582,929 & \centering	1,437,437 & \centering	94,514 & \centering	226,481& \\
            \centering 3 & go\_h & \centering	45,007 & \centering	490,109 & \centering	490,109 & \centering	0& \\
            %\centering 11 & pathways & \centering	6,238 & \centering	12,363 & \centering	3,117 & \centering	3,118& \\
            %\centering 12 & people & \centering	337 & \centering	640 & \centering	33 & \centering	161& \\
            %\centering 13 & pizza & \centering	671 & \centering	1,980 & \centering	259 & \centering	365& \\
            %\centering 14 & skos & \centering	144 & \centering	252 & \centering	1 & \centering	70& \\
            \centering 4 & taxonomy & \centering	5,728,398 & \centering	14,922,125 & \centering	2,112,637 & \centering	2,508,635& \\
            \centering 5 & taxonomy\_h & \centering	2,112,625 & \centering	32,876,289 & \centering	32,876,289 & \centering	0& \\
            %\centering 17 & travel & \centering	131 & \centering	277 & \centering	30 & \centering	90& \\
            %\centering 18 & univ & \centering	179 & \centering	293 & \centering	36 & \centering	84& \\
            %\centering 19 & wine & \centering	733 & \centering	1,839 & \centering	126 & \centering	485& \\
            \hline
            \hline
        \end{tabular}
        \small{
        \begin{tablenotes}
            \item[*] $|V|$~--- количество вершин графа, $|E|$~--- количество дуг, а $\#\textit{subClassOf}$ и $\#\textit{type}$~--- количества дуг с метками $\textit{subClassOf}$ и $\textit{type}$ соответственно.
        \end{tablenotes}    }
    \end{threeparttable}
\end{table}

\begin{table} [htbp]
    \centering
    \begin{threeparttable}% выравнивание подписи по границам таблицы
        \caption{Характеристики графов для статического анализа программ~\cite{graspan}\tnote{*}}\label{tab:Cgraphs}%
        \begin{tabular}{| p{1cm} || p{4.5cm} | p{2.2cm} | p{2.2cm} | p{2.2cm} | p{2.2cm}l |}
            \hline
            \hline
            \centering \textnumero & \centering Граф & \centering $|V|$ & \centering $|E|$ & \centering  $\#a$ & \centering  $\#d$ &\\
            \hline
            \centering 6 & apache\_httpd\_ptg & \centering	1,721,418 & \centering	1,510,411 & \centering	362,799 & \centering	1,147,612 & \\
            \centering 7 & arch\_after\_inline & \centering	3,448,422 & \centering	2,970,242 & \centering	671,295 & \centering	2,298,947 & \\
            \centering 8 & block\_after\_inline & \centering	3,423,234 & \centering	2,951,393 & \centering	669,238 & \centering	2,282,155 & \\
            \centering 9 & crypto\_after\_inline & \centering	3,464,970 & \centering	2,988,387 & \centering	678,408 & \centering	2,309,979 & \\
            \centering 10 & drivers\_after\_inline & \centering	4,273,803 & \centering	3,707,769 & \centering	858,568 & \centering	2,849,201 & \\
            \centering 11 & fs\_after\_inline & \centering	4,177,416 & \centering	3,609,373 & \centering	824,430 & \centering	2,784,943 & \\
            \centering 12 & init\_after\_inline & \centering	2,446,224 & \centering	2,112,809 & \centering	481,994 & \centering	1,630,815 & \\
            \centering 13 & ipc\_after\_inline & \centering	3,401,022 & \centering	2,931,498 & \centering	664,151 & \centering	2,267,347 & \\
            \centering 14 & kernel\_after\_inline & \centering	11,254,434 & \centering	9,484,213 & \centering	1,981,258 & \centering	7,502,955 & \\
            \centering 15 & lib\_after\_inline & \centering	3,401,355 & \centering	2,931,880 & \centering	664,311 & \centering	2,267,569 & \\
            \centering 16 & mm\_after\_inline & \centering	2,538,243 & \centering	2,191,079 & \centering	498,918 & \centering	1,692,161 & \\
            \centering 17 & net\_after\_inline & \centering	4,039,470 & \centering	3,500,141 & \centering	807,162 & \centering	2,692,979 & \\
            \centering 18 & postgre\_sql\_ptg & \centering	5,203,419 & \centering	4,678,543 & \centering	1,209,597 & \centering	3,468,946 & \\
            \centering 19 & security\_after\_inline & \centering	3,479,982 & \centering	3,003,326 & \centering	683,339 & \centering	2,319,987 & \\
            \centering 20 & sound\_after\_inline & \centering	3,528,861 & \centering	3,049,732 & \centering	697,159 & \centering	2,352,573 & \\
            \hline
            \hline
        \end{tabular}
        \small{
        \begin{tablenotes}
            \item[*] $|V|$~--- количество вершин графа, $|E|$~--- количество дуг, а $\#a$ и $\#d$~--- количества дуг с метками $a$ и $d$ соответственно.
        \end{tablenotes}    }
    \end{threeparttable}
\end{table}

В данном экспериментальном исследовании будут выполнены два анализа: анализ RDF данных~\cite{zhang2016context} и анализ указателей в программах на языках \texttt{C/C++}~\cite{graspan}. Оба этих анализа используют КС-ограничения в виде языков, описывающих различные правильные скобочные последовательности с двумя типами скобок. Правила вывода КС-грамматики $G_{\textit{RDF}}$ для анализа RDF данных имеют следующий вид.
\[
	\begin{array}{rcclcrccl}
	0: & S & \rightarrow & \overline{\text{\emph{subClassOf}}} \ S \ \text{\emph{subClassOf}}  & \quad & 2: & S & \rightarrow & \overline{\text{\emph{type}}} \ S \ \text{\emph{type}}    \\
	1: & S & \rightarrow & \overline{\text{\emph{subClassOf}}} \ \text{\emph{subClassOf}}      & \quad & 3: & S & \rightarrow & \overline{\text{\emph{type}}} \ \text{\emph{type}}  \\
	
	\end{array}
\]

В то же время правила вывода КС-грамматики $G_{C}$ для анализа указателей в программах на языках \texttt{C/C++} выглядят следующим образом.
	\[
	\begin{array}{rcclcrccl}
	0: & S & \rightarrow & \overline{\text{\emph{d}}} \ V \ \text{\emph{d}}  & \quad & 4: & V_3 & \rightarrow & \text{\emph{a}} \ V_2 \ V_3   \\
	1: & V & \rightarrow & V_1 \ V_2 \ V_3    & \quad & 5: & V_1 & \rightarrow & \varepsilon  \\
	2: & V_1 & \rightarrow & V_2 \ \overline{\text{\emph{a}}} \ V_1      & \quad & 6: &  V_2 & \rightarrow & \varepsilon \\
	3: & V_2 & \rightarrow & S     & \quad & 7: & V_3 & \rightarrow & \varepsilon  \\
	
	\end{array}
	\]

КС-язык, порождаемый грамматикой $G_{\textit{RDF}}$, описывает КС-ограничения, которые позволяют находить вершины в графе, находящиеся на одном уровне некоторой иерархии~\cite{zhang2016context}, а язык, порождаемый грамматикой $G_{C}$, позволяет находить в программах на языках \texttt{C/C++} указатели, которые указывают на одну область памяти~\cite{zheng2008demand}.

Стоит отметить, что в приведённых грамматиках используются не только метки $\text{\emph{subClassOf}}$, $\text{\emph{type}}$, $\text{\emph{a}}$ и $\text{\emph{d}}$, но также метки $\overline{\text{\emph{subClassOf}}}$, $\overline{\text{\emph{type}}}$, $\overline{\text{\emph{a}}}$ и $\overline{\text{\emph{d}}}$. Дело в том, что для проведения указанных анализов необходимо рассматривать пути в графах, также содержащие обратные дуги. Поэтому для всех дуг $(i, x, j)$ указанных графов, где $x \in \{\text{\emph{subClassOf}}, \text{\emph{type}}, \text{\emph{a}}, \text{\emph{d}}\}$ в явном виде добавлялись обратные дуги $(j, \overline{x}, i)$. Все графы и соответствующие грамматики доступны в наборе данных $\textit{CFPQ\_Data}\footnote{\text{CFPQ\_Data}~--- набор данных для задач поиска путей в графе с заданными КС-ограничениями: https://jetbrains-research.github.io/CFPQ\_Data/ (дата обращения: 14.01.2022).}$.

Для проведения описанных анализов были использованы реализации, предложенные в данной диссертации и описанные в~\cref{sec:ch3/sect5} и в~\cref{sec:ch4/sect5}, а также существующие решения, находящиеся в открытом доступе и которые автору удалось запустить. Полный список обозначений для используемых в сравнении реализаций представлен ниже.

\begin{itemize}
    \item $\textbf{MRC}$~--- реализация $\textit{MtxReach}_{\textit{CPU}}$ предложенного алгоритма, использующего умножение матриц, для задачи достижимости.
    \item $\textbf{MRG}$~--- реализация $\textit{MtxReach}_{\textit{GPU}}$ на GPU предложенного алгоритма, использующего умножение матриц, для задачи достижимости.
    \item $\textbf{MSC}$~--- реализация $\textit{MtxSingle}_{\textit{CPU}}$ предложенного алгоритма, использующего умножение матриц, для задачи поиска одного пути.
    \item $\textbf{MAC}$~--- реализация $\textit{MtxAll}_{\textit{CPU}}$ предложенного алгоритма, использующего умножение матриц, для задачи поиска всех путей.
    \item $\textbf{KAC}$~--- реализация $\textit{KronAll}_{\textit{CPU}}$ предложенного алгоритма, использующего произведение Кронекера, для задачи поиска всех путей.
    \item $\textbf{KAG}$~--- реализация $\textit{KronAll}_{\textit{GPU}}$ на GPU предложенного алгоритма, использующего произведение Кронекера, для задачи поиска всех путей.
    \item $\textbf{Graspan}\footnote{Graspan~--- инструмент для статического анализа программ: https://github.com/Graspan/Graspan-C (дата обращения: 14.01.2022).}$~--- инструмент для высокопроизводительного параллельного статического анализа программ~\cite{graspan}, написанный на языке \texttt{C++} и использующий CPU. Стоит отметить, что для инструмента \textit{Graspan} также существует реализация на GPU, однако её запустить не удалось.
    \item $\textbf{LL}\footnote{LL~--- реализация алгоритма достижимости в графе с заданными КС-ограничениями, основанного на алгоритме LL: https://gitlab.com/ciromoraismedeiros/rdf-ccfpq (дата обращения: 14.01.2022).}$~--- реализация алгоритма достижимости в графе с заданными КС-ограничениями~\cite{medeiros2018efficient}, основанного на алгоритме синтаксического анализа LL, написанная на языке \texttt{Go} и использующая CPU.
    \item $\textbf{GLL}_{\textbf{R}}$ и $\textbf{GLL}_{\textbf{A}}\footnote{GLL~--- реализации алгоритмов достижимости и поиска всех путей в графе с заданными КС-ограничениями, основанные на алгоритме GLL: https://github.com/JetBrains-Research/GLL4Graph (дата обращения: 14.01.2022).}$~--- реализации алгоритмов достижимости и поиска всех путей в графе с заданными КС-ограничениями~\cite{grigorev2017context}, основанных на алгоритме синтаксического анализа GLL, написанные на языке \texttt{Java} и использующие CPU. В то время как реализация $\textit{GLL}_{\textit{A}}$ строит SPPF для решения задачи поиска всех путей, реализация $\textit{GLL}_{\textit{R}}$ предназначена для решения задачи достижимости и опускает это построение.
\end{itemize}

Кроме того, рассматриваемые реализации используют различные способы представления КС-ограничений, поэтому для некоторых реализаций КС-грамматики $G_{\textit{RDF}}$ и $G_C$ приводились в ослабленную нормальную форму Хомского или для них строился соответствующий рекурсивный автомат. Стоит отметить, что приведение КС-грамматики $G_{\textit{RDF}}$ в ослабленную нормальную форму Хомского увеличило количество нетерминалов грамматики с 1 до 7 и количество правил вывода с 4 до 10, а приведение в такую форму КС-грамматики $G_C$ увеличило количество нетерминалов с 5 до 13 и количество правил вывода с 8 до 34.

Для сравнения производительности рассматриваемых реализаций вычислялось среднее время работы за 5 запусков, в соответствии с \textbf{[M1]}, а также потребляемая память, в соответствии с \textbf{[M2]}. Для задач поиска одного и всех путей в графе сравнивались только время работы и затраченная память на вычисление информации, достаточной для восстановления одного или всех путей в графе. Время восстановления найденных путей не измерялось, так как вид информации о восстановленных путях, а также выбор из многообразия способов её получения сильно зависят от области применения этих алгоритмов, что требует проведения отдельного исследования. Эксперименты проводились на компьютере под управлением ОС Ubuntu 18.04 с процессором Intel(R) Core(TM) i7-6700 CPU @ 3.40GHz, DDR4 64GB оперативной памяти, файлом подкачки размера 64GB и видеокартой GeForce GTX 1070 с 8GB DDR5 памяти.

\section{Результаты}\label{sec:ch5/sect2}

\paragraph{[В1].} Чтобы ответить на первый вопрос было проведено сравнение всех реализаций, позволяющих получить информацию о достижимости в графе с заданными КС-ограничениями для анализа RDF данных и статического анализа программ. Так как все рассматриваемые реализации позволяют получить такую информацию, то все они участвовали в этом сравнении. Однако для алгоритма, основанного на произведении Кронекера, на текущий момент не существует специализированной для решения задачи достижимости реализации и для данного сравнения были выбраны реализации $\textit{KronAll}_{\textit{CPU}}$ и $\textit{KronAll}_{\textit{GPU}}$ для решения задачи поиска всех путей в графе. А из 2 реализаций, основанных на алгоритме GLL и из 4 предложенных реализаций, основанных на умножении матриц, были выбраны только реализации, специализированные для решения задачи достижимости. В соответствии с \textbf{[M1]} среднее время работы в секундах рассматриваемых реализаций для анализа RDF данных представлено в~\cref{tab:RDFresults}, а для анализа указателей в программах на языках \texttt{C/C++}~--- в~\cref{tab:Cresults}. Также в таблицах указаны номера рассматриваемых графов и размер множества, являющегося ответом на поставленные задачи достижимости. То есть $\#\textit{result}$~--- это количество пар вершин $(i, j)$ таких, что существует хотя бы один путь из вершины $i$ в вершину $j$, образующий слово из языка, порождаемого КС-грамматикой $G_{\textit{RDF}}$ или $G_C$. Кроме того, в таблицах выделено наименьшее время анализа для каждого графа.

\begin{table} [htbp]
    \centering
    \begin{threeparttable}% выравнивание подписи по границам таблицы
        \caption{Время работы в секундах алгоритмов достижимости в графах с заданными КС-ограничениями для анализа RDF данных~\cite{zhang2016context}\tnote{*}}\label{tab:RDFresults}%
        \begin{tabular}{| p{0.6cm} || p{2cm} | p{1.7cm} | p{1.7cm} | p{1.4cm} | p{1.4cm} | p{1.4cm} | p{1.4cm} | p{0.9cm} l |}
            \hline
            \hline
            \centering \textnumero   & \centering $\#\textit{result}$ & \centering $\textit{Graspan}$ & \centering  $\textit{LL}$ & \centering  $\textit{GLL}_{\textit{R}}$ & \centering  $\textit{MRC}$ & \centering  $\textit{MRG}$ & \centering  $\textit{KAC}$ & \centering  $\textit{KAG}$ &\\
            \hline
            %\centering 1 & \centering	6 & \centering	\textbf{0.0}  & \centering	0.008 & \centering	0.008 & \centering	0.001 & \centering	0.005 & \centering	0.001 & \\
            %\centering 2 & \centering	47 & \centering	0.025 & \centering	0.008 & \centering	0.009 & \centering	\textbf{0.001} & \centering	0.006 & \centering	\textbf{0.001} & \\
            %\centering 3 & \centering	204 & \centering	0.025  & \centering	0.039 & \centering	0.019 & \centering	\textbf{0.001} & \centering	0.01 & \centering	0.004 & \\
            \centering 1 & \centering	90,994 & \centering	2.5  & \centering	9.3 & \centering	1.5 & \centering	\textbf{0.1} & \centering	\textbf{0.1} & \centering	0.3 & \centering 0.2 &\\
            %\centering 5 & \centering	396 & \centering	0.125  & \centering	1.157 & \centering	0.221 & \centering	\textbf{0.006} & \centering	0.016 & \centering	0.03 & \\
            %\centering 6 & \centering	36 & \centering	\textbf{0.0} & \centering	0.005 & \centering	0.006 & \centering	0.001 & \centering	0.005 & \centering	0.001 & \\
            %\centering 7 & \centering	58 & \centering	\textbf{0.0} & \centering	0.014 & \centering	0.012 & \centering	0.001 & \centering	0.006 & \centering	0.002 & \\
            %\centering 8 & \centering	12 & \centering	0.025  & \centering	0.003 & \centering	0.005 & \centering	\textbf{0.0} & \centering	0.003 & \centering	0.001 & \\
            \centering 2 & \centering	640,316 & \centering	12.8  & \centering	46.0 & \centering	5.6 & \centering	1.2 & \centering	\textbf{0.8} & \centering	3.2 & \centering 3.1 &\\
            \centering 3 & \centering	588,976 & \centering	0.9  & \centering	51.4 & \centering	3.7 & \centering	\textbf{0.1} & \centering	0.2 & \centering	0.2 & \centering 0.2 &\\
            %\centering 11 & \centering	884 & \centering	0.075  & \centering	0.292 & \centering	0.068 & \centering	\textbf{0.004} & \centering	0.009 & \centering	0.016 & \\
            %\centering 12 & \centering	51 & \centering	\textbf{0.0}	 &\centering	0.008 & \centering	0.007 & \centering	0.001 & \centering	0.009 & \centering	0.002 & \\
            %\centering 13 & \centering	1,356 & \centering	\textbf{0.0}  & \centering	0.037 & \centering	0.015 & \centering	0.001 & \centering	0.01 & \centering	0.004 & \\
            %\centering 14 & \centering	30 & \centering	\textbf{0.0}  & \centering	0.002 & \centering	0.005 & \centering	0.001 & \centering	0.005 & \centering	0.001 & \\
            \centering 4 & \centering	151,706 & \centering	3,938.0 & \centering	253.1 & \centering	45.5 & \centering	\textbf{1.0} & \centering	\textbf{1.0} & \centering	6.0 & \centering 3.9 &\\
            \centering 5 & \centering	5,351,657 & \centering	3,817.1  & \centering	3,023.8 & \centering	OOT & \centering	\textbf{10.9} & \centering	OOM & \centering	11.7 & \centering OOM &\\
            %\centering 17 & \centering	52 & \centering	0.033  & \centering	0.005 & \centering	0.006 & \centering	\textbf{0.001} & \centering	0.009 & \centering	\textbf{0.001} & \\
            %\centering 18 & \centering	25 & \centering	\textbf{0.0} & \centering	0.004 & \centering	0.006 & \centering	0.001 & \centering	0.009 & \centering	0.001 & \\
            %\centering 19 & \centering	565 & \centering	0.033  & \centering	0.037 & \centering	0.013 & \centering	\textbf{0.001} & \centering	0.009 & \centering	0.004 & \\
            \hline
            \hline
        \end{tabular}
        \small{
        \begin{tablenotes}
            \item[*] $\#\textit{result}$~--- размер результирующего множества; $\textit{Graspan}$~--- инструмент~\cite{graspan} для статического анализа программ; $\textit{LL}$~--- реализация алгоритма~\cite{medeiros2018efficient}, основанного на алгоритме синтаксического анализа LL, $\textit{GLL}_{\textit{R}}$~--- реализация алгоритма~\cite{grigorev2017context} достижимости, основанного на алгоритме GLL; $\textit{MRC}$ и $\textit{MRG}$~--- реализации на CPU и GPU предложенного алгоритма, использующего умножение матриц, для задачи достижимости; $\textit{KAC}$ и $\textit{KAG}$~--- реализации на CPU и GPU предложенного алгоритма, использующего произведение Кронекера, для задачи поиска всех путей.
        \end{tablenotes}    }
    \end{threeparttable}
\end{table}


\begin{table} [htbp]
    \centering
    \begin{threeparttable}% выравнивание подписи по границам таблицы
        \caption{Время работы в секундах алгоритмов достижимости в графах с заданными КС-ограничениями для статического анализа программ~\cite{graspan}\tnote{*}}\label{tab:Cresults}%
        \begin{tabular}{| p{0.6cm} || p{2.2cm} | p{1.7cm} | p{1.6cm} | p{1.4cm} | p{1.4cm} | p{1.4cm} | p{1.4cm} | p{0.9cm}l |}
            \hline
            \hline
            \centering \textnumero   & \centering $\#\textit{result}$ & \centering $\textit{Graspan}$ & \centering  $\textit{LL}$ & \centering  $\textit{GLL}_{\textit{R}}$ & \centering  $\textit{MRC}$ & \centering  $\textit{MRG}$ & \centering  $\textit{KAC}$ & \centering  $\textit{KAG}$ &\\
            \hline
            \centering 6 & \centering	92,806,768 & \centering	2,619.1	 & \centering 8,390.4 & \centering	OOT & \centering	536.7 & \centering	\textbf{135.0} & \centering	6,165.0 & \centering OOM &\\
            \centering 7 & \centering	5,339,563 & \centering	49.8 & \centering	928.2 & \centering	130.8 & \centering	119.9 & \centering	\textbf{34.5} & \centering	307.1 & \centering 96.7 &\\
            \centering 8 & \centering	5,351,409	 & \centering 51.3 & \centering	924.9	 & \centering 113.0 & \centering	123.9 & \centering	\textbf{34.4} & \centering	311.7 & \centering 96.8 &\\
            \centering 9 & \centering	5,428,237 & \centering	52.4 & \centering	935.4 & \centering	128.8 & \centering	122.1 & \centering	\textbf{34.7} & \centering	314.2 & \centering 98.0 &\\
            \centering 10 & \centering	18,825,025 & \centering	330.2 & \centering	3,660.7 & \centering	371.2 & \centering	279.4 & \centering	\textbf{69.8} & \centering	1,381.5 & \centering OOM &\\
            \centering 11 & \centering	9,646,475 & \centering	95.4 & \centering	2,000.8 & \centering	167.7 & \centering	105.7 & \centering	\textbf{49.6} & \centering	533.1 & \centering 148.4 &\\
            \centering 12 & \centering	3,783,769	 & \centering 39.2 & \centering	644.4	 & \centering 87.2 & \centering	45.8 & \centering	\textbf{24.6} & \centering	215.9 & \centering 68.7 &\\
            \centering 13 & \centering	5,249,389	 & \centering 55.3  & \centering 898.5 & \centering	109.4	 & \centering 79.5 & \centering	\textbf{34.0} & \centering	301.3 & \centering 95.6 &\\
            \centering 14 & \centering	16,747,731 & \centering	161.7  & \centering OOM & \centering	614.0	 & \centering 378.1	 & \centering \textbf{104.8}	 & \centering 978.8 & \centering 292.9 &\\
            \centering 15 & \centering 	5,276,303 & \centering 	52.9  & \centering 	900.2 & \centering 	111.1 & \centering 	121.8	 & \centering \textbf{34.1} & \centering 	300.7 & \centering 96.0 &\\
            \centering 16	 & \centering 3,990,3	 & \centering 39.1 & \centering	671.3 & \centering	77.9 & \centering	84.1 & \centering	\textbf{25.5} & \centering	226.6 & \centering 71.8 &\\
            \centering 17 & \centering	8,833,403 & \centering	95.2 & \centering	1,851.0 & \centering	160.6 & \centering	206.3	 & \centering \textbf{55.2} & \centering	684.7 & \centering 176.1 &\\
            \centering 18 & \centering	90,661,446 & \centering	1,711.9 & \centering	OOM & \centering	OOT & \centering	969.9 & \centering	\textbf{170.4} & \centering	5,072.0 & \centering OOM &\\
            \centering 19 & \centering	5,593,387 & \centering	56.4 & \centering	942.7 & \centering	115.8 & \centering	181.7 & \centering	\textbf{35.1} & \centering	320.7 & \centering 99.2 &\\
            \centering 20 & \centering	6,085,269 & \centering	58.9 & \centering	968.8	 & \centering 120.1 & \centering	133.6	 & \centering \textbf{36.1} & \centering	339.5& \centering 103.9 &\\
            \hline
            \hline
        \end{tabular}
        \small{
        \begin{tablenotes}
            \item[*] $\#\textit{result}$~--- размер результирующего множества; $\textit{Graspan}$~--- инструмент~\cite{graspan} для статического анализа программ; $\textit{LL}$~--- реализация алгоритма~\cite{medeiros2018efficient}, основанного на алгоритме синтаксического анализа LL, $\textit{GLL}_{\textit{R}}$~--- реализация алгоритма~\cite{grigorev2017context} достижимости, основанного на алгоритме GLL; $\textit{MRC}$ и $\textit{MRG}$~--- реализации на CPU и GPU предложенного алгоритма, использующего умножение матриц, для задачи достижимости;  $\textit{KAC}$ и $\textit{KAG}$~--- реализации на CPU и GPU предложенного алгоритма, использующего произведение Кронекера, для задачи поиска всех путей.
        \end{tablenotes}    }
    \end{threeparttable}
\end{table}

Результаты анализа RDF данных, представленные в~\cref{tab:RDFresults}, показывают, что лучшими для этого анализа и выбранных графов являются реализации $\textit{MtxReach}_{\textit{CPU}}$ ($\textit{MRC}$) и $\textit{MtxReach}_{\textit{GPU}}$ ($\textit{MRG}$) для предложенного алгоритма, основанного на умножении матриц. Стоит отметить, что все четыре предложенные для данного анализа реализации $\textit{MtxReach}_{\textit{CPU}}$, $\textit{MtxReach}_{\textit{GPU}}$, $\textit{KronAll}_{\textit{CPU}}$ ($\textit{KAC}$) и $\textit{KronAll}_{\textit{GPU}}$ ($\textit{KAG}$) показывают лучшее время работы по сравнению с другими решениями. Реализации $\textit{LL}$ и $\textit{GLL}_{\textit{R}}$ демонстрируют сравнимое с инструментом $\textit{Graspan}$ время работы, которому для проведения анализа на этих графах необходимо от примерно одной секунды до более чем часа. В то время как предложенные реализации позволяют проводить анализ на этих графах до 277 раз быстрее, чем лучшие из выбранных для сравнения существующих решений. Стоит отметить, что для данного анализа наиболее трудоёмким был анализ графа $\textit{taxonomy\_h}$ с номером 5 и с более чем 5 миллионами пар вершин в результирующем множестве. Реализация $\textit{GLL}_{\textit{R}}$ не завершила анализ за 5 часов, что отмечено в таблице с помощью сокращения OOT (Out Of Time), а реализации $\textit{MtxReach}_{\textit{GPU}}$ и $\textit{KronAll}_{\textit{GPU}}$ завершили свою работу из-за нехватки памяти, которую необходимо выделить на GPU, что отмечено как OOM (Out Of Memory).

Для сравнения реализаций на более больших графах с миллионами вершин и дуг, а также на более трудоёмком анализе с десятками миллионов пар вершин в результирующих множествах, более показательны результаты, представленные в~\cref{tab:Cresults}. Для анализа указателей в программах на языках \texttt{C/C++} наилучший результат на всех данных показала GPU реализация $\textit{MtxReach}_{\textit{GPU}}$ предложенного алгоритма, основанного на умножении матриц. Такие результаты объясняются тем, что для трудоёмких вычислений на больших графах абсолютно оправданы затраты на обмен данными между CPU и GPU для реализации высокопроизводительного анализа на GPU. Поэтому реализация $\textit{MtxReach}_{\textit{GPU}}$ позволяет производить данный анализ на выбранных графах до 19 раз быстрее, чем показывает лучшее из рассмотренных существующих решений~--- инструмент $\textit{Graspan}$. Другая предложенная GPU реализация $\textit{KronAll}_{\textit{GPU}}$ демонстрирует большее время работы, так как эта реализация не специализирована для решения задачи достижимости и вычисляет избыточную информацию, которой достаточно для восстановления всех найденных путей. Если же рассматривать только реализации на CPU, то предложенная реализация $\textit{MtxReach}_{\textit{CPU}}$ показывает сравнимые с инструментом $\textit{Graspan}$ результаты, а на графах с самым трудоёмким анализом (с номерами 6, 10 и 18) позволяет добиться почти пятикратного увеличения скорости работы. Это является впечатляющим результатом, с учётом того, что инструмент $\textit{Graspan}$ специализируется на проведении статического анализа программ. Время работы реализации $\textit{GLL}_{\textit{R}}$ сравнимо в данном анализе со временем работы реализации $\textit{MtxReach}_{\textit{CPU}}$, однако первой не хватило 5 часов для проведения двух самых трудоёмких анализов для графов с номерами 6 и 18. Наибольшее время работы показали реализации $\textit{LL}$ и $\textit{KronAll}_{\textit{CPU}}$. Однако стоит ещё раз отметить, что реализация $\textit{KronAll}_{\textit{CPU}}$ не специализирована на решении задачи достижимости. В данном анализе вычисление избыточной информации существенно сказалось на времени работы реализации $\textit{KronAll}_{\textit{CPU}}$, так как были найдены десятки миллионов пар достижимых вершин, а значит была вычислена информация для восстановления огромного количества путей, соответствующих заданным ограничениям.

В соответствии с \textbf{[M2]} для сравнения производительности рассмотренных реализаций были также измерены их затраты по памяти. Результаты этих измерений представлены в~\cref{tab:RDFmemory} и в~\cref{tab:Cmemory}, где выделены наименьшие затраты по памяти для каждого графа. Для решения задачи достижимости в процессе анализа RDF данных наименьшие затраты по памяти демонстрирует инструмент $\textit{Graspan}$, что можно увидеть в~\cref{tab:RDFmemory}. Предложенные реализации потребляют до 6 раз больше памяти на данном анализе, чем инструмент $\textit{Graspan}$, но существенно меньше, чем другие существующие решения $\textit{LL}$ и $\textit{GLL}_{\textit{R}}$. Для анализа указателей в программах на языках \texttt{C/C++} наименьшие затраты по памяти имеет реализация $\textit{MtxReach}_{\textit{GPU}}$ предложенного алгоритма, основанного на умножении матриц, что показано в~\cref{tab:Cmemory}. А реализация $\textit{MtxReach}_{\textit{CPU}}$ потребляет сравнимый объём памяти с инструментом $\textit{Graspan}$. Таким образом, для этого анализа предложенные реализации позволяют снизить потребление памяти до 2 раз по сравнению с лучшим из существующих решений. Ожидаемо предложенные реализации $\textit{KronAll}_{\textit{CPU}}$ и $\textit{KronAll}_{\textit{GPU}}$ потребляют большее количество памяти, чем инструмент $\textit{Graspan}$, из-за вычисления избыточной для решения задачи достижимости информации. Однако даже эти реализации потребляют существенно меньше памяти, чем другие существующие решения $\textit{LL}$ и $\textit{GLL}_{\textit{R}}$.

\begin{table} [htbp]
    \centering
    \begin{threeparttable}% выравнивание подписи по границам таблицы
        \caption{Затраты по памяти в мегабайтах алгоритмов достижимости в графах с заданными КС-ограничениями для анализа RDF данных~\cite{zhang2016context}\tnote{*}}\label{tab:RDFmemory}%
        \begin{tabular}{| p{0.6cm} || p{2cm} | p{1.7cm} | p{1.7cm} | p{1.4cm} | p{1.4cm} | p{1.4cm} | p{1.4cm} | p{1.0cm} l |}
            \hline
            \hline
            \centering \textnumero   & \centering $\#\textit{result}$ & \centering $\textit{Graspan}$ & \centering  $\textit{LL}$ & \centering  $\textit{GLL}_{\textit{R}}$ & \centering  $\textit{MRC}$ & \centering  $\textit{MRG}$ & \centering  $\textit{KAC}$ & \centering  $\textit{KAG}$ &\\
            \hline
            %\centering 1 & \centering	6 & \centering	  & \centering	& \centering	& \centering	& \centering	 & \centering	 & \\
            %\centering 2 & \centering	47 & \centering	 & \centering	 & \centering	 & \centering	 & \centering	 & \centering	 & \\
            %\centering 3 & \centering	204 & \centering	  & \centering	 & \centering	& \centering	 & \centering	& \centering	 & \\
            \centering 1 & \centering	90,994 & \centering	\textbf{101}  & \centering	470 & \centering 1,263	& \centering 240	 & \centering 307	 & \centering 279	 & \centering 357 &\\
            %\centering 5 & \centering	396 & \centering	  & \centering	 & \centering	& \centering	 & \centering	 & \centering	& \\
            %\centering 6 & \centering	36 & \centering	 & \centering	 & \centering	& \centering	 & \centering & \centering	 & \\
            %\centering 7 & \centering	58 & \centering	 & \centering	 & \centering	& \centering	 & \centering	 & \centering & \\
            %\centering 8 & \centering	12 & \centering	  & \centering	& \centering	& \centering	 & \centering	 & \centering	 & \\
            \centering 2 & \centering	640,316 & \centering \textbf{193}	  & \centering 1,406	 & \centering 1,192	& \centering	468 & \centering 	727 & \centering 468 &	\centering 829 & \\
            \centering 3 & \centering	588,976 & \centering \textbf{62}	 & \centering 431	 & \centering 3,177	& \centering 263	 & \centering 387	 & \centering 266 &	\centering 573 & \\
            %\centering 11 & \centering	884 & \centering	  & \centering	 & \centering	& \centering	 & \centering	 & \centering	 & \\
            %\centering 12 & \centering	51 & \centering		 &\centering	 & \centering	& \centering	 & \centering	 & \centering	 & \\
            %\centering 13 & \centering	1,356 & \centering	  & \centering	& \centering	& \centering	 & \centering	 & \centering	 & \\
            %\centering 14 & \centering	30 & \centering	  & \centering	 & \centering	& \centering	 & \centering	 & \centering	 & \\
            \centering 4 & \centering	151,706 & \centering \textbf{1,498}	 & \centering 15,200	 & \centering 9,877	 & \centering 3,229	 & \centering 1,651	 & \centering 3,229 &	\centering 2,463 & \\
            \centering 5 & \centering	5,351,657 & \centering	\textbf{1,098}  & \centering 20,395	 & \centering	OOT & \centering 6,804	 & \centering	OOM & \centering 6,804 & \centering OOM & \\
            %\centering 17 & \centering	52 & \centering	  & \centering	 & \centering	 & \centering	 & \centering	& \centering	 & \\
            %\centering 18 & \centering	25 & \centering	 & \centering	 & \centering	 & \centering	 & \centering	 & \centering	 & \\
            %\centering 19 & \centering	565 & \centering	 & \centering	 & \centering	 & \centering	 & \centering	 & \centering & \\
            \hline
            \hline
        \end{tabular}
        \small{
        \begin{tablenotes}
            \item[*] $\#\textit{result}$~--- размер результирующего множества; $\textit{Graspan}$~--- инструмент~\cite{graspan} для статического анализа программ; $\textit{LL}$~--- реализация алгоритма~\cite{medeiros2018efficient}, основанного на алгоритме синтаксического анализа LL, $\textit{GLL}_{\textit{R}}$~--- реализация алгоритма~\cite{grigorev2017context} достижимости, основанного на алгоритме GLL; $\textit{MRC}$ и $\textit{MRG}$~--- реализации на CPU и GPU предложенного алгоритма, использующего умножение матриц, для задачи достижимости; $\textit{KAC}$ и $\textit{KAG}$~--- реализации на CPU и GPU предложенного алгоритма, использующего произведение Кронекера, для задачи поиска всех путей.
        \end{tablenotes}    }
    \end{threeparttable}
\end{table}

\begin{table} [htbp]
    \centering
    \begin{threeparttable}% выравнивание подписи по границам таблицы
        \caption{Затраты по памяти в мегабайтах алгоритмов достижимости в графах с заданными КС-ограничениями для статического анализа программ~\cite{graspan}\tnote{*}}\label{tab:Cmemory}%
        \begin{tabular}{| p{0.4cm} || p{2.1cm} | p{1.7cm} | p{1.6cm} | p{1.55cm} | p{1.4cm} | p{1.4cm} | p{1.55cm} | p{1.0cm}l |}
            \hline
            \hline
            \centering \textnumero   & \centering $\#\textit{result}$ & \centering $\textit{Graspan}$ & \centering  $\textit{LL}$ & \centering  $\textit{GLL}_{\textit{R}}$ & \centering  $\textit{MRC}$ & \centering  $\textit{MRG}$ & \centering  $\textit{KAC}$ & \centering  $\textit{KAG}$ &\\
            \hline
            \centering 6 & \centering	92,806,768 & \centering	11,094	 & \centering 53,652 & \centering	OOT & \centering 11,619	 & \centering \textbf{5,585}	 & \centering 40,110 & \centering OOM &\\
            \centering 7 & \centering	5,339,563 & \centering 1,643	 & \centering 33,275	 & \centering 30,573	 & \centering 1,342	 & \centering \textbf{863}	 & \centering 2,954	 & \centering 2,209  &\\
            \centering 8 & \centering	5,351,409	 & \centering 1,638  & \centering 33,082		 & \centering 29,866  & \centering 1,331	 & \centering \textbf{849}	 & \centering 2,988	 & \centering 2,219  &\\
            \centering 9 & \centering	5,428,237 & \centering 1,660	 & \centering 33,472	 & \centering 29,317	 & \centering 1,305	 & \centering \textbf{849}	 & \centering 3,052	 & \centering 2,235 &\\
            \centering 10 & \centering	18,825,025 & \centering 3,474	 & \centering 46,060	 & \centering 49,262	 & \centering	3,181 & \centering	 \textbf{2,861} & \centering 9,012	 & \centering OOM  &\\
            \centering 11 & \centering	9,646,475 & \centering 2,330	 & \centering 41,429	 & \centering 31,128	 & \centering 1,958	 & \centering \textbf{1,099}	 & \centering 4,779	& \centering 3,723  &\\
            \centering 12 & \centering	3,783,769	 & \centering 1,170  & \centering 23,478	 & \centering 21,537 & \centering 1,015	 & \centering \textbf{687}	 & \centering 2,205 & \centering 1,649 &\\
            \centering 13 & \centering	5,249,389	 & \centering  1,620  & \centering 32,405 & \centering	29,205	 & \centering 1,268 & \centering \textbf{845}	 & \centering 2,922	 & \centering 2,173 &\\
            \centering 14 & \centering	16,747,731 & \centering	 5,083 & \centering OOM & \centering 41,449	 & \centering  3,466	 & \centering \textbf{1,959}	 & \centering 8,261  & \centering 5,763 &\\
            \centering 15 & \centering 	5,276,303 & \centering 1,624	  & \centering 32,414	 & \centering 29,432	 & \centering 	1,299	 & \centering \textbf{845}  & \centering  2,980	 & \centering 2,271  &\\
            \centering 16	 & \centering 3,990,305	 & \centering 1,219 & \centering	24,338 & \centering 86,190	 & \centering	1,036 & \centering	\textbf{691} & \centering 2,328	 & \centering 1,697  &\\
            \centering 17 & \centering	8,833,403 & \centering 2,271	 & \centering 40,185	 & \centering 31,358	 & \centering	1,888	 & \centering  \textbf{1,111} & \centering 4,680	 & \centering 4,069  &\\
            \centering 18 & \centering	90,661,446 & \centering	11,871 & \centering	OOM & \centering	OOT & \centering	11,018 & \centering	\textbf{5,297} & \centering 36,812	 & \centering OOM &\\
            \centering 19 & \centering	5,593,387 & \centering 1,688	 & \centering 33,689	 & \centering 30,228	 & \centering 1,336	 & \centering \textbf{857}	 & \centering 3,067	 & \centering 2,309  &\\
            \centering 20 & \centering	6,085,269 & \centering 1,763	 & \centering 34,324		 & \centering  31,699 & \centering	1,454	 & \centering  \textbf{887} & \centering	3,308 & \centering 2,415 &\\
            \hline
            \hline
        \end{tabular}
        \small{
        \begin{tablenotes}
            \item[*] $\#\textit{result}$~--- размер результирующего множества; $\textit{Graspan}$~--- инструмент~\cite{graspan} для статического анализа программ; $\textit{LL}$~--- реализация алгоритма~\cite{medeiros2018efficient}, основанного на алгоритме синтаксического анализа LL, $\textit{GLL}_{\textit{R}}$~--- реализация алгоритма~\cite{grigorev2017context} достижимости, основанного на алгоритме GLL; $\textit{MRC}$ и $\textit{MRG}$~--- реализации на CPU и GPU предложенного алгоритма, использующего умножение матриц, для задачи достижимости; $\textit{KAC}$ и $\textit{KAG}$~--- реализации на CPU и GPU предложенного алгоритма, использующего произведение Кронекера, для задачи поиска всех путей.
        \end{tablenotes}    }
    \end{threeparttable}
\end{table}

Таким образом, ответом на \textbf{[В1]} является следующее. Предложенные реализации для задачи достижимости по сравнению с лучшими из рассмотренных существующих решений позволяют:
\begin{itemize}
    \item ускорить время анализа RDF данных до 277 раз, увеличив при этом до 6 раз объем потребляемой памяти;
    \item ускорить время анализа указателей в программах на языках \texttt{C/C++} до 19 раз, снизив при этом потребление памяти до 2 раз.
\end{itemize}

\paragraph{[В2].} Для ответа на второй вопрос аналогичные анализы RDF данных и программ на языках \texttt{C/C++} были проведены с помощью реализаций, вычисляющих информацию, достаточную для дальнейшего восстановления одного или всех найденных путей. Среди рассматриваемых существующих решений единственной реализацией, которая вычисляет такую информацию, является реализация $\textit{GLL}_{\textit{A}}$ алгоритма поиска всех путей в графе с заданными КС-ограничениями, который основан на алгоритме GLL. В соответствии с \textbf{[M1]} среднее время работы в секундах рассматриваемых реализаций для анализа RDF данных представлено в~\cref{tab:RDFpathResults}, а для анализа указателей в программах на языках \texttt{C/C++}~--- в~\cref{tab:CpathResults}. Кроме того, в таблицах выделено наименьшее время анализа для каждого графа.

\begin{table} [htbp]
    \centering
    \begin{threeparttable}% выравнивание подписи по границам таблицы
        \caption{Время работы в секундах алгоритмов поиска одного и всех путей в графах с заданными КС-ограничениями для анализа RDF данных~\cite{zhang2016context}\tnote{*}}\label{tab:RDFpathResults}%
        \begin{tabular}{| p{0.6cm} || p{2cm} | p{2cm} | p{2cm} | p{2cm} | p{2cm} | p{2cm}l |}
            \hline
            \hline
            \centering \textnumero   & \centering $\#\textit{result}$ & \centering  $\textit{GLL}_{\textit{A}}$ & \centering  $\textit{MSC}$ & \centering  $\textit{MAC}$ & \centering  $\textit{KAC}$ & \centering  $\textit{KAG}$ &\\
            \hline
            %\centering 1 & \centering	6 & \centering	  & \centering	& \centering		 & \centering	 & \\
            %\centering 2 & \centering	47 & \centering	 & \centering	 & \centering		 & \centering	 & \\
            %\centering 3 & \centering	204 & \centering	  & \centering	 & \centering		& \centering	 & \\
            \centering 1 & \centering	90,994 & \centering	3.0  & \centering	0.2 & \centering	\textbf{0.1}	 & \centering 0.3	 & \centering 0.2 &\\
            %\centering 5 & \centering	396 & \centering	  & \centering	 & \centering		 & \centering	& \\
            %\centering 6 & \centering	36 & \centering	 & \centering	 & \centering	 & \centering	 & \\
            %\centering 7 & \centering	58 & \centering	 & \centering	 & \centering		 & \centering & \\
            %\centering 8 & \centering	12 & \centering	  & \centering	& \centering		 & \centering	 & \\
            \centering 2 & \centering	640,316 & \centering	20.1  & \centering	 2.1	 & \centering \textbf{0.5}	 & \centering 3.2	 & \centering 3.1  &\\
            \centering 3 & \centering	588,976 & \centering 140.1	 & \centering	 0.4	 & \centering \textbf{0.2}	 & \centering \textbf{0.2}	 & \centering \textbf{0.2}  &\\
            %\centering 11 & \centering	884 & \centering	  & \centering	 	 & \centering	 & \centering	 & \\
            %\centering 12 & \centering	51 & \centering		 &\centering	 	 & \centering	 & \centering	 & \\
            %\centering 13 & \centering	1,356 & \centering	  & \centering		 & \centering	 & \centering	 & \\
            %\centering 14 & \centering	30 & \centering	  & \centering	 	 & \centering	 & \centering	 & \\
            \centering 4 & \centering	151,706 & \centering	 1,878.4	 & \centering	\textbf{3.0} & \centering 5.0	 & \centering 6.0	 & \centering 3.9 &\\
            \centering 5 & \centering	5,351,657 & \centering	OOT  & \centering	 25.7	 & \centering \textbf{11.0}	 & \centering	11.7	& \centering OOM &\\
            %\centering 17 & \centering	52 & \centering	  & \centering	 	 & \centering	& \centering	 & \\
            %\centering 18 & \centering	25 & \centering	 & \centering	 	 & \centering	 & \centering	 & \\
            %\centering 19 & \centering	565 & \centering	 & \centering	 	 & \centering	 & \centering & \\
            \hline
            \hline
        \end{tabular}
        \small{
        \begin{tablenotes}
            \item[*] $\#\textit{result}$~--- размер результирующего множества; $\textit{GLL}_{\textit{A}}$~--- реализация алгоритма~\cite{grigorev2017context} поиска всех путей, основанного на алгоритме GLL; $\textit{MSC}$ и $\textit{MAC}$~--- реализации предложенного алгоритма, использующего умножение матриц, для задачи поиска одного пути и для задачи поиска всех путей соответственно; $\textit{KAC}$ и $\textit{KAG}$~--- реализации на CPU и GPU предложенного алгоритма, использующего произведение Кронекера, для задачи поиска всех путей.
        \end{tablenotes}    }
    \end{threeparttable}
\end{table}

\begin{table} [htbp]
    \centering
    \begin{threeparttable}% выравнивание подписи по границам таблицы
        \caption{Время работы в секундах алгоритмов поиска одного и всех путей в графах с заданными КС-ограничениями для статического анализа программ~\cite{graspan}\tnote{*}}\label{tab:CpathResults}%
        \begin{tabular}{| p{0.6cm} || p{2.2cm} | p{2cm} | p{2cm} | p{2cm} | p{2cm} | p{2cm}l |}
            \hline
            \hline
            \centering \textnumero   & \centering $\#\textit{result}$ & \centering  $\textit{GLL}_{\textit{A}}$ & \centering  $\textit{MSC}$ & \centering  $\textit{MAC}$ & \centering  $\textit{KAC}$& \centering  $\textit{KAG}$ &\\
            \hline
            \centering	6 & \centering	92,806,768 & \centering	OOT	  & \centering \textbf{1,611.5}	 & \centering OOM	 & \centering  6,165.0 & \centering OOM&\\
            \centering	7 & \centering	5,339,563 & \centering	728.5	 & \centering	132.8 & \centering 432.5	 & \centering 307.1	 & \centering \textbf{96.7} &\\
            \centering	8 & \centering	5,351,409	 & \centering 771.3 & \centering	111.6	 & \centering OOM & \centering	311.7 	 & \centering \textbf{96.8} &\\
            \centering	9 & \centering	5,428,237 & \centering	 750.2	 & \centering	139.1 & \centering OOM	 & \centering 314.2	 & \centering \textbf{98.0} &\\
            \centering	10 & \centering	18,825,025 & \centering	 1,222.3	 & \centering	\textbf{699.1} & \centering OOM	 & \centering 1,381.5	 & \centering OOM &\\
            \centering	11 & \centering	9,646,475 & \centering	 1,150.9	 & \centering 	\textbf{135.6} & \centering OOM	 & \centering 533.1	& \centering 148.4 &\\
            \centering	12 & \centering	3,783,769	 & \centering  368.9  & \centering	\textbf{53.4} & \centering	261.8 & \centering 215.9 & \centering 68.7 &\\
            \centering	13 & \centering	5,249,389	 & \centering   692.7	 & \centering 166.7 & \centering	405.5 & \centering 301.3	 & \centering \textbf{95.6} &\\
            \centering	14 & \centering	16,747,731 & \centering	 7,923.0  & \centering 474.9	 & \centering OOM	 & \centering 978.8  & \centering \textbf{292.9} &\\
            \centering	15 & \centering 	5,276,303 & \centering 	  712.6  & \centering 	166.0	 & \centering 437.8 & \centering  300.7	 & \centering \textbf{96.0} &\\
            \centering	16	 & \centering 3,990,305	 & \centering 396.8 & \centering	95.6 	 & \centering	301.8 & \centering 226.6	 & \centering \textbf{71.8} &\\
            \centering	17 & \centering	8,833,403 & \centering	1,010.1 & \centering	 \textbf{145.8}		 & \centering OOM & \centering 684.7	 & \centering 176.1 &\\
            \centering	18 & \centering	90,661,446 & \centering	 OOT & \centering	\textbf{2,024.0} & \centering OOM	 & \centering 5,072.0	 & \centering OOM &\\
            \centering	19 & \centering	5,593,387 & \centering	741.1	 & \centering	142.3 & \centering OOM	 & \centering 320.7	 & \centering \textbf{99.2} &\\
            \centering	20 & \centering	6,085,269 & \centering	759.0  & \centering	153.2	 & \centering OOM & \centering 339.5	& \centering \textbf{103.9} &\\
            \hline
            \hline
        \end{tabular}
        \small{
        \begin{tablenotes}
            \item[*] $\#\textit{result}$~--- размер результирующего множества; $\textit{GLL}_{\textit{A}}$~--- реализация алгоритма~\cite{grigorev2017context} поиска всех путей, основанного на алгоритме GLL; $\textit{MSC}$ и $\textit{MAC}$~--- реализации предложенного алгоритма, использующего умножение матриц, для задачи поиска одного пути и для задачи поиска всех путей соответственно; $\textit{KAC}$ и $\textit{KAC}$ ~--- реализации на CPU и GPU предложенного алгоритма, использующего произведение Кронекера, для задачи поиска всех путей.
        \end{tablenotes}    }
    \end{threeparttable}
\end{table}

Результаты анализа RDF данных, представленные в~\cref{tab:RDFpathResults}, показывают, что наименьшее время анализа выбранных графов имеют реализации $\textit{MtxSingle}_{\textit{CPU}}$ ($\textit{MSC}$) и $\textit{MtxAll}_{\textit{CPU}}$ ($\textit{MAC}$) для предложенного алгоритма, основанного на умножении матриц. Все рассматриваемые в данной таблице реализации позволяют решить задачу поиска одного пути в графе с заданными КС-ограничениями. А так как выбранной для сравнения существующей реализации $\textit{GLL}_{\textit{A}}$ понадобилось более 5 часов для самого трудоёмкого анализа графа с номером 5, то предложенные реализации позволяют решить задачу поиска одного пути в рамках анализа RDF данных до 1636 раз быстрее в сравнении с этой реализацией. Аналогичный вывод верен и для задачи поиска всех путей в графе, так как наилучшее время анализа для графа с номером 5 показала реализация $\textit{MtxAll}_{\textit{CPU}}$, которая позволяет решить задачу поиска всех путей в графе.

Для задачи поиска одного пути в графе в рамках анализа указателей в программах на языках \texttt{C/C++} наилучшее время анализа показали предложенные реализации $\textit{MtxSingle}_{\textit{CPU}}$ и $\textit{KronAll}_{\textit{GPU}}$, что показано в~\cref{tab:CpathResults}. Такие результаты объясняются тем, что реализация $\textit{MtxSingle}_{\textit{CPU}}$ является здесь единственной реализацией специализированной на задаче поиска одного пути в графе, а реализация $\textit{KronAll}_{\textit{GPU}}$ является единственной GPU реализацией в этом сравнении. Для задач поиска одного и поиска всех путей в графе в рамках анализа указателей в программах на языках \texttt{C/C++} предложенные реализации позволяют ускорить время анализа до 27 раз по сравнению с существующим решением $\textit{GLL}_{\textit{A}}$. Такой результат достигается с использованием GPU реализации $\textit{KronAll}_{\textit{GPU}}$ для анализа графа с номером 14.

В соответствии с \textbf{[M2]} для сравнения производительности рассмотренных реализаций для поиска одного и всех путей в графе были также измерены их затраты по памяти. Результаты этих измерений представлены в~\cref{tab:RDFpathMemory} и в~\cref{tab:CpathMemory}, где выделены наименьшие затраты по памяти для каждого графа. Для решения задач поиска одного и поиска всех путей в графе в рамках анализа RDF данных наименьшие затраты по памяти демонстрирует предложенная реализация $\textit{MtxAll}_{\textit{CPU}}$, что можно увидеть в~\cref{tab:RDFpathMemory}. Исключением является граф с номером 1, для которого проводится наименее трудоёмкий анализ. Таким образом, для задач поиска одного и поиска всех путей в графе в рамках анализа RDF данных предложенные реализации потребляют до 152 раз меньше памяти, чем существующее решение. При этом стоит отметить, что предложенная реализация $\textit{MtxSingle}_{\textit{CPU}}$, специализированная на решении задач поиска одного пути в графе, потребляет большее количество памяти, чем реализация $\textit{MtxAll}_{\textit{CPU}}$ для решения задачи поиска всех путей в графе. Это можно объяснить особенностями реализации $\textit{MtxAll}_{\textit{CPU}}$, написанной на языке \texttt{C++} и использующей собственную обёртку на языке \texttt{Python}, а также малым количеством различных промежуточных вершин в элементах из множества $\textit{AllPathIndex}$ результирующих матриц, которые могут занимать меньше памяти, чем элементы из множества $\textit{PathIndex}$. Также по результатам сравнения для анализа указателей в программах на языках \texttt{C/C++}, представленным в~\cref{tab:CpathMemory}, можно сделать вывод, что предложенные реализации потребляют сравнимое количество памяти с существующей реализацией $\textit{GLL}_{\textit{A}}$ для задач поиска одного и поиска всех путей в графе. Однако стоит отметить, что для большинства рассматриваемых графов наименьшее потребление памяти продемонстрировала предложенная GPU реализация $\textit{KronAll}_{\textit{GPU}}$, а в реализации $\textit{MtxAll}_{\textit{CPU}}$ необходимая информация для восстановления всех найденных путей хранилась не столь компактно и анализ прервался из-за нехватки памяти.

\begin{table} [htbp]
    \centering
    \begin{threeparttable}% выравнивание подписи по границам таблицы
        \caption{Затраты по памяти в мегабайтах алгоритмов поиска одного и всех путей в графах с заданными КС-ограничениями для анализа RDF данных~\cite{zhang2016context}\tnote{*}}\label{tab:RDFpathMemory}%
        \begin{tabular}{| p{0.6cm} || p{2cm} | p{2cm} | p{2cm} | p{2cm} | p{2cm} | p{2cm}l |}
            \hline
            \hline
            \centering \textnumero   & \centering $\#\textit{result}$ & \centering  $\textit{GLL}_{\textit{A}}$ & \centering  $\textit{MSC}$ & \centering  $\textit{MAC}$ & \centering  $\textit{KAC}$ & \centering  $\textit{KAG}$ &\\
            \hline
            %\centering	1 & \centering	6 & \centering	  & \centering	& \centering		 & \centering	 & \\
            %\centering	2 & \centering	47 & \centering	 & \centering	 & \centering		 & \centering	 & \\
            %\centering	3 & \centering	204 & \centering	  & \centering	 & \centering		& \centering	 & \\
            \centering	1 & \centering	90,994 & \centering	\textbf{49}  & \centering	257 & \centering 200		 & \centering 279	 & \centering 357 &\\
            %\centering	5 & \centering	396 & \centering	  & \centering	 & \centering		 & \centering	& \\
            %\centering	6 & \centering	36 & \centering	 & \centering	 & \centering	 & \centering	 & \\
            %\centering	7 & \centering	58 & \centering	 & \centering	 & \centering		 & \centering & \\
            %\centering	8 & \centering	12 & \centering	  & \centering	& \centering		 & \centering	 & \\
            \centering	2 & \centering	640,316 & \centering	649  & \centering	 545	 & \centering \textbf{337}	 & \centering 468	 & \centering 829  &\\
            \centering	3 & \centering	588,976 & \centering 30,444	 & \centering	 290	 & \centering \textbf{200}	 & \centering 266	 & \centering 573 &\\
            %\centering	11 & \centering	884 & \centering	  & \centering	 	 & \centering	 & \centering	 & \\
            %\centering	12 & \centering	51 & \centering		 &\centering	 	 & \centering	 & \centering	 & \\
            %\centering	13 & \centering	1,356 & \centering	  & \centering		 & \centering	 & \centering	 & \\
            %\centering	14 & \centering	30 & \centering	  & \centering	 	 & \centering	 & \centering	 & \\
            \centering	4 & \centering	151,706 & \centering	 9,108	 & \centering 3,805	 & \centering \textbf{1,595}	 & \centering 3,229	 & \centering 2,463  &\\
            \centering	5 & \centering	5,351,657 & \centering	 OOT & \centering	8,058 	 & \centering \textbf{2,720}	 & \centering 6,804		& \centering OOM &\\
            %\centering	17 & \centering	52 & \centering	  & \centering	 	 & \centering	& \centering	 & \\
            %\centering	18 & \centering	25 & \centering	 & \centering	 	 & \centering	 & \centering	 & \\
            %\centering	19 & \centering	565 & \centering	 & \centering	 	 & \centering	 & \centering & \\
            \hline
            \hline
        \end{tabular}
        \small{
        \begin{tablenotes}
            \item[*] $\#\textit{result}$~--- размер результирующего множества; $\textit{GLL}_{\textit{A}}$~--- реализация алгоритма~\cite{grigorev2017context} поиска всех путей, основанного на алгоритме GLL; $\textit{MSC}$ и $\textit{MAC}$~--- реализации предложенного алгоритма, использующего умножение матриц, для задачи поиска одного пути и для задачи поиска всех путей соответственно; $\textit{KAC}$ и $\textit{KAG}$~--- реализации на CPU и GPU предложенного алгоритма, использующего произведение Кронекера, для задачи поиска всех путей.
        \end{tablenotes}    }
    \end{threeparttable}
\end{table}

\begin{table} [htbp]
    \centering
    \begin{threeparttable}% выравнивание подписи по границам таблицы
        \caption{Затраты по памяти в мегабайтах алгоритмов поиска одного и всех путей в графах с заданными КС-ограничениями для статического анализа программ~\cite{graspan}\tnote{*}}\label{tab:CpathMemory}%
        \begin{tabular}{| p{0.6cm} || p{2.2cm} | p{2cm} | p{2cm} | p{2cm} | p{2cm} | p{2cm}l |}
            \hline
            \hline
            \centering \textnumero   & \centering $\#\textit{result}$ & \centering  $\textit{GLL}_{\textit{A}}$ & \centering  $\textit{MSC}$ & \centering  $\textit{MAC}$ & \centering  $\textit{KAC}$ & \centering  $\textit{KAG}$ &\\
            \hline
            \centering	6 & \centering	92,806,768 & \centering	OOT	  & \centering	\textbf{35,666} & \centering OOM	 & \centering 40,110 & \centering OOM &\\
            \centering	7 & \centering	5,339,563 & \centering	6,043	 & \centering	2,651 & \centering	62,423 & \centering 2,954	 & \centering  \textbf{2,209} &\\
            \centering	8 & \centering	5,351,409	 & \centering 5,874 & \centering	2,651	 & \centering OOM & \centering	2,988 	 & \centering  \textbf{2,219}  &\\
            \centering	9 & \centering	5,428,237 & \centering	 6,250	 & \centering 2,676	 & \centering OOM	 & \centering 3,052	 & \centering  \textbf{2,235}  &\\
            \centering	10 & \centering	18,825,025 & \centering	 \textbf{4,608}	 & \centering 8,332	 & \centering OOM	 & \centering 9,012	 & \centering  OOM  &\\
            \centering	11 & \centering	9,646,475 & \centering	 8,629	 & \centering 4,214	 & \centering OOM	 & \centering 4,779	& \centering  \textbf{3,723}  &\\
            \centering	12 & \centering	3,783,769	 & \centering  4,242  & \centering	1,964 & \centering	62,404 & \centering 2,205 & \centering  \textbf{1,649}  &\\
            \centering	13 & \centering	5,249,389	 & \centering   5,992	 & \centering 2,589 & \centering 62,384	 & \centering 2,922	 & \centering  \textbf{2,173} &\\
            \centering	14 & \centering	16,747,731 & \centering	 \textbf{5,136}  & \centering 	7,156 & \centering OOM	 & \centering 8,261 & \centering 5,763  &\\
            \centering	15 & \centering 	5,276,303 & \centering 	 6,094   & \centering 	2,632	 & \centering 62,421 & \centering 2,980	 & \centering \textbf{2,271}  &\\
            \centering  16	 & \centering 3,990,305	 & \centering 4,545 & \centering	 2,073	 & \centering 62,453	 & \centering 2,328	 & \centering  \textbf{1,697} &\\
            \centering	17 & \centering	8,833,403 & \centering	9,346 & \centering	 	4,201	 & \centering OOM & \centering 4,680 & \centering  \textbf{4,069} &\\
            \centering	18 & \centering	90,661,446 & \centering	 OOT & \centering \textbf{32,635}	 & \centering OOM	 & \centering 36,812	 & \centering  OOM  &\\
            \centering	19 & \centering	5,593,387 & \centering	6,157	 & \centering 2,759	 & \centering OOM	 & \centering 3,067	 & \centering  \textbf{2,309}  &\\
            \centering	20 & \centering	6,085,269 & \centering	6,842  & \centering	2,991	 & \centering OOM & \centering 3,308	& \centering  \textbf{2,415} &\\
            \hline
            \hline
        \end{tabular}
        \small{
        \begin{tablenotes}
            \item[*] $\#\textit{result}$~--- размер результирующего множества; $\textit{GLL}_{\textit{A}}$~--- реализация алгоритма~\cite{grigorev2017context} поиска всех путей, основанного на алгоритме GLL; $\textit{MSC}$ и $\textit{MAC}$~--- реализации предложенного алгоритма, использующего умножение матриц, для задачи поиска одного пути и для задачи поиска всех путей соответственно; $\textit{KAC}$ и $\textit{KAG}$~--- реализации на CPU и GPU предложенного алгоритма, использующего произведение Кронекера, для задачи поиска всех путей.
        \end{tablenotes}    }
    \end{threeparttable}
\end{table}

По полученным результатам на \textbf{[В2]} можно ответить следующим образом. Предложенные реализации для задач поиска одного и поиска всех путей в графе по сравнению с рассмотренным существующим решением позволяют:
\begin{itemize}
    \item ускорить время анализа RDF данных до 1636 раз, снизив при этом потребление памяти до 152 раз;
    \item ускорить время анализа указателей в программах на языках \texttt{C/C++} до 27 раз, потребляя при этом сравнимый объём памяти.
\end{itemize}

\paragraph{[В3].} Чтобы ответить на третий вопрос необходимо сравнить полученные результаты предложенных реализаций для задачи достижимости, поиска одного и поиска всех путей в графе. Для наглядности результаты, полученные для предложенных реализаций, были вынесены в отдельные таблицы. В соответствии с \textbf{[M1]} среднее время работы в секундах предложенных реализаций для анализа RDF данных представлено в~\cref{tab:RDFlaResults}, а для анализа указателей в программах на языках \texttt{C/C++}~--- в~\cref{tab:ClaResults}.

\begin{table} [htbp]
    \centering
    \begin{threeparttable}% выравнивание подписи по границам таблицы
        \caption{Время работы в секундах предложенных алгоритмов поиска путей в графах с заданными КС-ограничениями для анализа RDF данных~\cite{zhang2016context}\tnote{*}}\label{tab:RDFlaResults}%
        \begin{tabular}{| p{0.6cm} || p{2cm} | p{1.4cm} | p{1.4cm} | p{1.4cm} | p{1.4cm} | p{1.4cm} | p{0.9cm}l |}
            \hline
            \hline
            \centering \textnumero   & \centering $\#\textit{result}$ & \centering  $\textit{MRC}$ & \centering  $\textit{MRG}$ & \centering  $\textit{MSC}$ & \centering  $\textit{MAC}$ & \centering  $\textit{KAC}$ & \centering  $\textit{KAG}$ &\\
            \hline
            %\centering 1 & \centering	6 & \centering	0.005  & \centering 0.001  & \centering 	& \centering		 & \centering 0.001	 & \\
            %\centering 2 & \centering	47 & \centering	 0.006 & \centering	0.001 & \centering & \centering		 & \centering 0.001	 & \\
            %\centering 3 & \centering	204 & \centering 0.01	  & \centering 0.001	 & \centering & \centering		& \centering 0.004	 & \\
            \centering 1 & \centering	90,994 & \centering	0.1  & \centering	0.1 & \centering 0.2 & \centering	0.1	 & \centering 0.3	 & \centering  0.2 &\\
            %\centering 5 & \centering	396 & \centering	0.016  & \centering	0.006 & \centering & \centering		 & \centering 0.03	& \\
            %\centering 6 & \centering	36 & \centering	 0.005 & \centering	0.001 & \centering & \centering	 & \centering	0.001 & \\
            %\centering 7 & \centering	58 & \centering	 0.006 & \centering	 0.001 & \centering & \centering		 & \centering 0.002 & \\
            %\centering 8 & \centering	12 & \centering	 0.003 & \centering	0.0 & \centering & \centering		 & \centering 0.001	 & \\
            \centering 2 & \centering	640,316 & \centering 1.2	  & \centering 0.8 & \centering	 2.1	 & \centering 0.5 & \centering	 3.2	 & \centering  3.1  & \\
            \centering 3 & \centering	588,976 & \centering 0.1	 & \centering 0.2 & \centering	 0.4	 & \centering 0.2	 & \centering 0.2	 & \centering 0.2  &\\
            %\centering 11 & \centering	884 & \centering 0.009	  & \centering 0.004	 	 & \centering & \centering	 & \centering	0.016 & \\
            %\centering 12 & \centering	51 & \centering	 0.009	 &\centering 0.001	 	 & \centering	 & \centering & \centering	0.002 & \\
            %\centering 13 & \centering	1,356 & \centering	0.01  & \centering 0.001		 & \centering	 & \centering & \centering 0.004	 & \\
            %\centering 14 & \centering	30 & \centering	 0.005  & \centering 0.001	 	 & \centering	 & \centering & \centering	0.001 & \\
            \centering 4 & \centering	151,706 & \centering 1.0	 	 & \centering 1.0	 & \centering 3.0	 & \centering 5.0 & \centering	6.0 & \centering 3.9  &\\
            \centering 5 & \centering	5,351,657 & \centering	10.9   & \centering	OOM	 & \centering 25.7	 & \centering 11.0 & \centering	11.7	& \centering  OOM &\\
            %\centering 17 & \centering	52 & \centering	 0.009 & \centering	 0.001	 & \centering	& \centering & \centering	0.001 & \\
            %\centering 18 & \centering	25 & \centering	0.009 & \centering	 0.001	 & \centering	 & \centering & \centering	0.001 & \\
            %\centering 19 & \centering	565 & \centering 0.009	 & \centering 0.001	 	 & \centering	 & \centering & \centering 0.004 & \\
            \hline
            \hline
        \end{tabular}
        \small{
        \begin{tablenotes}
            \item[*] $\#\textit{result}$~--- размер результирующего множества; $\textit{MRC}$ и $\textit{MRG}$~--- реализации на CPU и GPU алгоритма, использующего умножение матриц, для задачи достижимости; $\textit{MSC}$ и $\textit{MAC}$~--- реализации алгоритма, использующего умножение матриц, для задачи поиска одного пути и для задачи поиска всех путей; $\textit{KAC}$ и $\textit{KAG}$~--- реализации на CPU и GPU предложенного алгоритма, использующего произведение Кронекера, для задачи поиска всех путей.
        \end{tablenotes}    }
    \end{threeparttable}
\end{table}

\begin{table} [htbp]
    \centering
    \begin{threeparttable}% выравнивание подписи по границам таблицы
        \caption{Время работы в секундах предложенных алгоритмов поиска путей в графах с заданными КС-ограничениями для статического анализа программ~\cite{graspan}\tnote{*}}\label{tab:ClaResults}%
        \begin{tabular}{| p{0.6cm} || p{2.2cm} | p{1.4cm} | p{1.4cm} | p{1.4cm} | p{1.4cm} | p{1.4cm} | p{0.9cm}l |}
            \hline
            \hline
            \centering \textnumero   & \centering $\#\textit{result}$ & \centering  $\textit{MRC}$ & \centering  $\textit{MRG}$ & \centering  $\textit{MSC}$ & \centering  $\textit{MAC}$ & \centering  $\textit{KAC}$ & \centering  $\textit{KAG}$ &\\
            \hline
            \centering 6 & \centering	92,806,768 & \centering	536.7	  & \centering 135.0 & \centering 1,611.5	 & \centering OOM & \centering 6,165.0 & \centering  OOM &\\
            \centering 7 & \centering	5,339,563 & \centering	119.9	 & \centering 34.5	 & \centering 132.8	 & \centering 432.5 & \centering	307.1 & \centering  96.7 &\\
            \centering 8 & \centering	5,351,409	 & \centering 123.9 & \centering 34.4		 & \centering 111.6  & \centering OOM	 & \centering 311.7	 & \centering 96.8 &\\
            \centering 9 & \centering	5,428,237 & \centering	 122.1	 & \centering 34.7	 & \centering 139.1	 & \centering OOM & \centering	314.2 & \centering 98.0 &\\
            \centering 10 & \centering	18,825,025 & \centering	 279.4	 & \centering 69.8	 & \centering 699.1	 & \centering OOM & \centering 1,381.5	 & \centering  OOM &\\
            \centering 11 & \centering	9,646,475 & \centering	 105.7	 & \centering 49.6	 & \centering 135.6	 & \centering OOM & \centering 533.1	& \centering 148.4 &\\
            \centering 12 & \centering	3,783,769	 & \centering  45.8  & \centering 24.6	 & \centering 53.4	 & \centering  261.8 & \centering 215.9 & \centering 68.7 &\\
            \centering 13 & \centering	5,249,389	 & \centering   79.5		 & \centering 34.0  & \centering 166.7	 & \centering 405.5 & \centering	301.3 & \centering 95.6 &\\
            \centering 14 & \centering	16,747,731 & \centering	 378.1  & \centering 104.8	 & \centering 474.9	 & \centering OOM & \centering  978.8 & \centering 292.9 &\\
            \centering 15 & \centering 	5,276,303 & \centering 	 121.8  	 & \centering 34.1		 & \centering 166.0 & \centering 437.8 & \centering 300.7	 & \centering 96.0 &\\
            \centering 16	 & \centering 3,990,305	 & \centering  84.1 & \centering 25.5	 	 & \centering 95.6	 & \centering 301.8 & \centering 226.6	 & \centering  71.8 &\\
            \centering 17 & \centering	8,833,403 & \centering	206.3 & \centering	  55.1		 & \centering 145.8 & \centering OOM	  & \centering 684.7 & \centering 176.1 &\\
            \centering 18 & \centering	90,661,446 & \centering	 969.9 & \centering 170.4	 & \centering 2,024.0	 & \centering OOM  & \centering	5,072.0 & \centering OOM &\\
            \centering 19 & \centering	5,593,387 & \centering	181.7	 & \centering 35.1	 & \centering 142.3	 & \centering OOM & \centering	320.7 & \centering  99.2 &\\
            \centering 20 & \centering	6,085,269 & \centering	133.6  & \centering 36.1		 & \centering 153.2  & \centering OOM & \centering 339.5	& \centering 103.9 &\\
            \hline
            \hline
        \end{tabular}
        \small{
        \begin{tablenotes}
            \item[*] $\#\textit{result}$~--- размер результирующего множества; $\textit{MRC}$ и $\textit{MRG}$~--- реализации на CPU и GPU алгоритма, использующего умножение матриц, для задачи достижимости; $\textit{MSC}$ и $\textit{MAC}$~--- реализации алгоритма, использующего умножение матриц, для задачи поиска одного пути и для задачи поиска всех путей; $\textit{KAC}$ и $\textit{KAG}$~--- реализации на CPU и GPU предложенного алгоритма, использующего произведение Кронекера, для задачи поиска всех путей.
        \end{tablenotes}    }
    \end{threeparttable}
\end{table}

Результаты анализа RDF данных, представленные в~\cref{tab:RDFlaResults}, показывают, что предложенные реализации демонстрируют сравнимое время анализа выбранных графов. Можно отметить, что хранение в предложенных реализациях информации о найденных путях для решения задачи поиска одного пути замедляет время анализа до 3 раз, а для решения задачи поиска всех путей~--- до 4 раз. В свою очередь результаты анализа указателей в программах на языках \texttt{C/C++}, представленные в~\cref{tab:ClaResults}, показывают, что наименьшее время работы среди предложенных реализаций имеет реализация $\textit{MtxReach}_{\textit{GPU}}$. Для решения задач поиска одного и поиска всех путей в графе в большинстве случаев наименьшее время работы демонстрирует реализация $\textit{KronAll}_{\textit{GPU}}$, которая замедляет анализ до 3 раз по сравнению с лучшей реализацией для задачи достижимости. Исключениями являются графы с номерами 6, 10 и 18, для которых проводился наиболее трудоёмкий анализ и реализация $\textit{KronAll}_{\textit{GPU}}$ завершила свою работу из-за нехватки памяти. В процессе анализа этих графов предложенные реализации решали задачу поиска одного пути до 12 раз дольше, чем задачу достижимости, а задачу поиска всех путей~--- до 46 раз дольше.

В соответствии с \textbf{[M2]} предложенные реализации также требуют сравнения по количеству затраченной памяти. Результаты такого сравнения представлены в~\cref{tab:RDFlaMemory} и в~\cref{tab:ClaMemory}. В процессе анализа RDF данных наименьшое потребление памяти продемонстрировала реализация $\textit{MtxAll}_{\textit{CPU}}$, которая позволяет решить задачу поиска всех путей в графе. Поэтому для данного анализа на выбранных графах с использованием предложенных реализаций не происходит увеличение объёмов потребляемой памяти при переходе от решения задачи достижимости к решению задач поиска одного и поиска всех путей в графе. В то же время в процессе анализа указателей в программах на языках \texttt{C/C++} наименьшее потребление памяти продемонстрировала реализация $\textit{MtxReach}_{\textit{GPU}}$, решающая задачу достижимости. Для решения задач поиска одного и поиска всех путей в графе в большинстве случаев наименьшие затраты по памяти демонстрирует реализация $\textit{KronAll}_{\textit{GPU}}$, которая увеличивает эти затраты до 4 раз по сравнению с лучшей реализацией для задачи достижимости. Исключениями опять же являются три графа с наиболее трудоёмким анализом. В процессе анализа этих графов предложенные реализации использовали для решения задачи поиска одного пути до 6 раз больше памяти, чем для решения задачи достижимости, а для решения задачи поиска всех путей~--- до 7 раз больше памяти.

\begin{table} [htbp]
    \centering
    \begin{threeparttable}% выравнивание подписи по границам таблицы
        \caption{Затраты по памяти в мегабайтах предложенных алгоритмов поиска путей в графах с заданными КС-ограничениями для анализа RDF данных~\cite{zhang2016context}\tnote{*}}\label{tab:RDFlaMemory}%
        \begin{tabular}{| p{0.6cm} || p{2cm} | p{1.4cm} | p{1.4cm} | p{1.4cm} | p{1.4cm} | p{1.4cm} | p{0.9cm}l |}
            \hline
            \hline
            \centering \textnumero   & \centering $\#\textit{result}$ & \centering  $\textit{MRC}$ & \centering  $\textit{MRG}$ & \centering  $\textit{MSC}$ & \centering  $\textit{MAC}$ & \centering  $\textit{KAC}$ & \centering  $\textit{KAG}$ &\\
            \hline
            %\centering 1 & \centering	6 & \centering	  & \centering & \centering	& \centering		 & \centering	 & \\
            %\centering 2 & \centering	47 & \centering	 & \centering	 & \centering & \centering		 & \centering	 & \\
            %\centering 3 & \centering	204 & \centering	  & \centering	 & \centering & \centering		& \centering	 & \\
            \centering 1 & \centering	90,994 & \centering	 240 & \centering 307	 & \centering 257 & \centering	200	 & \centering 279	 & \centering 357	&\\
            %\centering 5 & \centering	396 & \centering	  & \centering	 & \centering & \centering		 & \centering	& \\
            %\centering 6 & \centering	36 & \centering	 & \centering	 & \centering & \centering	 & \centering	 & \\
            %\centering 7 & \centering	58 & \centering	 & \centering	 & \centering & \centering		 & \centering & \\
            %\centering 8 & \centering	12 & \centering	  & \centering	& \centering & \centering		 & \centering	 & \\
            \centering 2 & \centering	640,316 & \centering 468 	  & \centering 727 & \centering	 545	 & \centering 337	 & \centering 468 	 & \centering 829	&\\
            \centering 3 & \centering	588,976 & \centering 263 	 & \centering 387  & \centering	 290	 & \centering 200	 & \centering 266	 & \centering 573	&\\
            %\centering 11 & \centering	884 & \centering	  & \centering	 	 & \centering & \centering	 & \centering	 & \\
            %\centering 12 & \centering	51 & \centering		 &\centering	 	 & \centering	 & \centering & \centering	 & \\
            %\centering 13 & \centering	1,356 & \centering	  & \centering		 & \centering	 & \centering & \centering	 & \\
            %\centering 14 & \centering	30 & \centering	  & \centering	 	 & \centering	 & \centering & \centering	 & \\
            \centering 4 & \centering	151,706 & \centering	 3,229 	 & \centering 1,651 	 & \centering 3,805	 & \centering 1,595 & \centering 3,229	 & \centering 2,463	&\\
            \centering 5 & \centering	5,351,657 & \centering 6,804	  & \centering OOM	 	 & \centering 8,058	 & \centering 2,720 & \centering	 6,804	& \centering OOM	&\\
            %\centering 17 & \centering	52 & \centering	  & \centering	 	 & \centering	& \centering & \centering	 & \\
            %\centering 18 & \centering	25 & \centering	 & \centering	 	 & \centering	 & \centering & \centering	 & \\
            %\centering 19 & \centering	565 & \centering	 & \centering	 	 & \centering	 & \centering & \centering & \\
            \hline
            \hline
        \end{tabular}
        \small{
        \begin{tablenotes}
            \item[*] $\#\textit{result}$~--- размер результирующего множества; $\textit{MRC}$ и $\textit{MRG}$~--- реализации на CPU и GPU алгоритма, использующего умножение матриц, для задачи достижимости; $\textit{MSC}$ и $\textit{MAC}$~--- реализации алгоритма, использующего умножение матриц, для задачи поиска одного пути и для задачи поиска всех путей; $\textit{KAC}$ и $\textit{KAG}$~--- реализации на CPU и GPU предложенного алгоритма, использующего произведение Кронекера, для задачи поиска всех путей.
        \end{tablenotes}    }
    \end{threeparttable}
\end{table}

\begin{table} [htbp]
    \centering
    \begin{threeparttable}% выравнивание подписи по границам таблицы
        \caption{Затраты по памяти в мегабайтах предложенных алгоритмов поиска путей в графах с заданными КС-ограничениями для статического анализа программ~\cite{graspan}\tnote{*}}\label{tab:ClaMemory}%
        \begin{tabular}{| p{0.6cm} || p{2.2cm} | p{1.4cm} | p{1.4cm} | p{1.4cm} | p{1.4cm} | p{1.4cm} | p{0.9cm}l |}
            \hline
            \hline
            \centering \textnumero   & \centering $\#\textit{result}$ & \centering  $\textit{MRC}$ & \centering  $\textit{MRG}$ & \centering  $\textit{MSC}$ & \centering  $\textit{MAC}$ & \centering  $\textit{KAC}$ & \centering  $\textit{KAG}$ &\\
            \hline
            \centering 6 & \centering	92,806,768 & \centering	11,619	  & \centering	5,585 & \centering 35,666	 & \centering OOM & \centering 40,110 & \centering OOM &\\
            \centering 7 & \centering	5,339,563 & \centering	1,342	 & \centering 863	 & \centering 2,651	 & \centering 62,423 & \centering	2,954 & \centering 2,209 &\\
            \centering 8 & \centering	5,351,409	 & \centering 1,331 & \centering 849		 & \centering 2,651 & \centering OOM	 & \centering 2,988	 & \centering 2,219 &\\
            \centering 9 & \centering	5,428,237 & \centering	 1,305	 & \centering 849	 & \centering 2,676	 & \centering OOM & \centering 3,052	 & \centering  2,235 &\\
            \centering 10 & \centering	18,825,025 & \centering	 3,181	 & \centering 2,861	 & \centering 8,332	 & \centering OOM & \centering	9,012 & \centering OOM &\\
            \centering 11 & \centering	9,646,475 & \centering	 1,958	 & \centering 1,099	 & \centering 4,214	 & \centering OOM & \centering 4,779	& \centering 3,723  &\\
            \centering 12 & \centering	3,783,769	 & \centering  1,015  & \centering 687	 & \centering 1,964	 & \centering 62,404  & \centering 2,205 & \centering  1,649 &\\
            \centering 13 & \centering	5,249,389	 & \centering   1,268		 & \centering 845  & \centering	2,589 & \centering 62,384 & \centering 2,922	 & \centering 2,173 &\\
            \centering 14 & \centering	16,747,731 & \centering 3,466   & \centering  1,959	 & \centering 7,156	 & \centering OOM & \centering 8,261 & \centering 5,763  &\\
            \centering 15 & \centering 	5,276,303 & \centering 	1,299   	 & \centering 845		 & \centering 2,632  & \centering 62,421 & \centering 2,980	 & \centering 2,271 &\\
            \centering 16 & \centering  3,990,305 & \centering 1,036 & \centering 691	 	 & \centering 2,073	 & \centering 62,453 & \centering 2,328	 & \centering 1,697 &\\
            \centering 17 & \centering	8,833,403 & \centering 1,888	 & \centering	 1,111		 & \centering 4,201  & \centering OOM	  & \centering 4,680 & \centering  4,069 &\\
            \centering 18 & \centering	90,661,446 & \centering	 11,018 & \centering 5,297	 & \centering 32,635	 & \centering OOM  & \centering	36,812 & \centering OOM &\\
            \centering 19 & \centering	5,593,387 & \centering	1,336	 & \centering 857	 & \centering 2,759	 & \centering OOM & \centering	3,067 & \centering 2,309 &\\
            \centering 20 & \centering	6,085,269 & \centering 1,454	  & \centering	887	 & \centering  2,991 & \centering OOM & \centering 3,308	& \centering  2,415  &\\
            \hline
            \hline
        \end{tabular}
        \small{
        \begin{tablenotes}
            \item[*] $\#\textit{result}$~--- размер результирующего множества; $\textit{MRC}$ и $\textit{MRG}$~--- реализации на CPU и GPU алгоритма, использующего умножение матриц, для задачи достижимости; $\textit{MSC}$ и $\textit{MAC}$~--- реализации алгоритма, использующего умножение матриц, для задачи поиска одного пути и для задачи поиска всех путей; $\textit{KAC}$ и $\textit{KAG}$~--- реализации на CPU и GPU предложенного алгоритма, использующего произведение Кронекера, для задачи поиска всех путей.
        \end{tablenotes}    }
    \end{threeparttable}
\end{table}

Таким образом, на \textbf{[В3]} можно ответить следующее. Хранение в полученных реализациях информации о найденных путях для решения задач поиска одного и поиска всех путей в графе по сравнению с предложенными реализациями для решения задачи достижимости влечёт за собой:
\begin{itemize}
    \item замедление анализа RDF данных до 4 раз;
    \item замедление анализа указателей в программах на языках \texttt{C/C++} до 46 раз с потреблением до 7 раз большего объёма памяти.
\end{itemize}

\paragraph{[В4].} Чтобы ответить на последний вопрос необходимо сравнить полученные результаты предложенных реализаций для алгоритма, основанного на произведении Кронекера и не требующего преобразования входной КС-грамматики, с предложенными реализациями алгоритма, основанного на умножении матриц. Для данного сравнения также можно использовать результаты, представленные в~\cref{tab:RDFlaResults}, \cref{tab:ClaResults}, \cref{tab:RDFlaMemory} и~\cref{tab:ClaMemory}. Так как для алгоритма, основанного на произведении Кронекера, не существует специализированных реализаций для задач достижимости и поиска одного пути в графе, то сравнивать будем реализации, решающие задачу поиска всех путей в графе. Кроме того, для задачи поиска всех путей в графе алгоритм, основанный на умножении матриц, реализован только на CPU, поэтому сравнивать будем только CPU реализации. Таким образом, для данного сравнения были выбраны реализации $\textit{MtxAll}_{\textit{CPU}}$ и $\textit{KronAll}_{\textit{CPU}}$.

В~\cref{tab:RDFlaResults} и~\cref{tab:RDFlaMemory} можно увидеть, что несмотря на увеличение размеров грамматики после её преобразования в ослабленную нормальную форму Хомского, реализация $\textit{MtxAll}_{\textit{CPU}}$ не уступает в скорости анализа RDF данных и потребляет до 2 раз меньший объём памяти, чем реализация $\textit{KronAll}_{\textit{CPU}}$, не требующая преобразований входной КС-грамматики. Причиной этого является сравнительно небольшие размеры грамматики для данного анализа, а также сравнительно небольшая трудоёмкость самого анализа. Однако другие выводы можно сделать для более трудоёмкого анализа указателей в программах на языках \texttt{C/C++}. В~\cref{tab:ClaResults} и~\cref{tab:ClaMemory} можно увидеть, что для большинства графов реализации $\textit{MtxAll}_{\textit{CPU}}$ не хватило памяти, а на остальных графах реализация $\textit{KronAll}_{\textit{CPU}}$ завершает анализ до полутора раз быстрее и потребляет до 28 раз меньше памяти. Существенное увеличение количества нетерминалов и правил вывода грамматики для такого трудоёмкого анализа с использованием алгоритма, основанного на умножении матриц, привело к необходимости хранить значительно большее количество булевых матриц с большим количеством ненулевых элементов и вычислять больше операций над этими матрицами.

Таким образом, на \textbf{[В4]} можно ответить следующее. Полученная CPU реализация, не требующая преобразований входной КС-грамматики, по
сравнению с предложенной CPU реализацией алгоритма, основанного на умножении матриц, для задачи поиска всех путей в графе:
\begin{itemize}
    \item не ускоряет анализ RDF данных и потребляет до 2 раз больше памяти из-за сравнительно небольших размеров грамматики и небольшой трудоёмкости самого анализа;
    \item ускоряет анализ указателей в программах на языках \texttt{C/C++} до полутора раз и потребляет до 28 раз меньший объём памяти.
\end{itemize}

\paragraph{Выводы.} Таким образом, полученные результаты экспериментального исследования позволяют сделать следующие выводы. Предложенный подход к задачам поиска путей в графе с заданными КС-ограничениями позволяет получать высокопроизводительные параллельные реализации, ускоряющие время анализа и потребляющие меньший объём памяти, в сравнении с существующими решениями на входных данных, близких к реальным.

\section{Ограничения}\label{sec:ch5/sect3}
Предлагаемые подход и алгоритмы поиска путей в графе с заданными КС-ограничениями накладывают ограничения на реализации этих алгоритмов. Данный раздел посвящён обсуждению этих ограничений.

Во-первых, для применения полученных реализаций требуется сформулировать необходимый анализ графов в виде задачи поиска путей с КС-ограничения. Для этого в классе КС-языков должен существовать язык, описывающий свойства путей необходимые для решения поставленной задачи анализа графов. Некоторые ограничения не могут быть описаны с использованием КС-языка. Для тех ограничений, которые удалось выразить в виде КС-языка, необходимо задать этот язык в виде, требуемом самим алгоритмом. Предложенные в данной работе алгоритмы используют КС-грамматики и рекурсивные автоматы в качестве представления КС-языков. А для алгоритмов, основанных на умножении матриц, такая грамматика должна быть преобразована в ослабленную нормальную форму Хомского. В процессе такого преобразования размеры КС-грамматики могут сильно увеличиться, что существенно скажется на производительности полученных реализаций.

Во-вторых, предложенные реализации сильно полагаются на разреженность входных графов. В случае анализа плотных графов больших размеров предложенные реализации либо будут показывать неудовлетворительное время работы, либо для этого анализа на используемой вычислительной системе может просто не хватить памяти. Также неэффективными себя показывают предложенные реализации на графах малого размера или для слишком простого анализа, когда время, затраченное на организацию параллельных вычислений, превышает или сравнимо со временем самого анализа.

Кроме того, после построения алгоритма поиска путей в графе с помощью предложенного подхода необходимо найти библиотеку линейной алгебры с реализованными необходимыми операциями или реализовать их самостоятельно. В случае использования в алгоритме стандартных операций (сложение, умножение, транспонирование и т.д.) над матрицами стандартных типов (булевого, целочисленного, с числами с плавающей точкой и т.д.) такую библиотеку найти не составляет труда. Однако в случае использования для более сложного анализа объектов линейной алгебры с некоторым пользовательским типом данных, с поиском подходящей библиотеки могут возникнуть проблемы. Если использовать одну из реализаций стандарта GraphBLAS, то необходимо убедиться, что использованный пользовательский тип данных в матрицах и векторах, а также операции над ними могут быть реализованы с использованием алгебраической структуры, схожей с полукольцом, которую позволяет построить стандарт GraphBLAS.

\clearpage