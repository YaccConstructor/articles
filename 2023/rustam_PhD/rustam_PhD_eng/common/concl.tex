%% Согласно ГОСТ Р 7.0.11-2011:
%% 5.3.3 В заключении диссертации излагают итоги выполненного исследования, рекомендации, перспективы дальнейшей разработки темы.
%% 9.2.3 В заключении автореферата диссертации излагают итоги данного исследования, рекомендации и перспективы дальнейшей разработки темы.

{\defpositions}
\begin{enumerate}[beginpenalty=10000] % https://tex.stackexchange.com/a/476052/104425
    \item An approach to the context-free path querying based on linear algebra methods that allows one to use theoretical and practical linear algebra results was developed.
    \item A CFPQ algorithm that uses the proposed approach was devised. Termination and correctness of the devised algorithm were proved, as well as its time complexity. The proposed algorithm uses matrix operations, which make it possible to apply a wide class of optimizations and allows one to automatically parallelize computations using existing linear algebra libraries.
    \item A CFPQ algorithm that uses the proposed approach and does not require a transformation of the input context-free grammar was devised. Termination and correctness of the devised algorithm were proved, as well as its time complexity. The proposed algorithm makes it possible to work with arbitrary input context-free grammars without any transformations. This allows one to avoid a significant increase in the grammar size that affects analysis performance.
    \item The devised algorithms are implemented using parallel computing techniques. An experimental study of the devised algorithms was provided using real RDF data and graphs built for static program analysis. The obtained implementations were compared with each other, with existing solutions from the field of static program analysis, and with solutions based on various parsing techniques. The comparison results show that the proposed implementations for the reachability problem allow one to obtain up to 2 orders of magnitude faster graph analysis time and consume up to 2 times less memory in comparison with existing solutions, and for the problems of finding one and all paths in a graph allow one to speed up the analysis time up to 3 orders of magnitude and consume up to 2 orders of magnitude less memory.
\end{enumerate}
