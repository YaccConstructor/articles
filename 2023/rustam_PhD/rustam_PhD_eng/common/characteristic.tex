{\actuality}
%В современном мире становится всё больше данных, которые требуют обработки и анализа. При этом графы являются одной из самых распространённых и удобных структур, позволяя компактно представлять большие объёмы информации и реализовывать эффективные алгоритмы для её анализа. Графы используются в статическом анализе программ~\cite{rehof2001type,zheng2008demand} и биоинформатике~\cite{sevon2008subgraph}, в социальных сетях~\cite{socialgraph}, сетевом анализе~\cite{zhang2016context} и т.д. Также в настоящее время активно развиваются графовые базы данных, используемые для хранения данных в виде графов и реализации запросов к ним. Следует упомянуть такие графовые базы данных, как RedisGraph$\footnote{Графовая база данных RedisGraph: https://oss.redislabs.com/redisgraph/ (дата обращения: 14.01.2022).}$ и Neo4j$\footnote{Графовая база данных Neo4j: https://neo4j.com/ (дата обращения: 14.01.2022).}$. %HyperGraphDB и OrientDB.
In the modern world, there is more and more data that requires processing and analysis. At the same time, graphs are one of the most common and convenient structures, allowing us to compactly represent large amounts of information and implement efficient algorithms for its analysis. Graphs are used in static program analysis~\cite{rehof2001type,zheng2008demand}, bioinformatics~\cite{sevon2008subgraph}, social networks~\cite{socialgraph}, network analysis~\cite{zhang2016context}, etc. Also, graph databases, used to store data in the form of graphs and implement queries to them, are currently being actively developed. For example, there are such popular graph databases as RedisGraph$\footnote{Graph database RedisGraph: https://oss.redislabs.com/redisgraph/ (date of access: 14.01.2022).}$ and Neo4j$\footnote{Graph database Neo4j: https://neo4j.com/ (date of access: 14.01.2022).}$.

%Одной из важнейших задач анализа графов является поиск путей. Имеются разные вариации этой задачи: непосредственный поиск определённых путей, задача достижимости (сами пути не ищутся, но доказывается их существование) и т.д.
One of the most important graph analysis problems is path querying. There are different variations of this problem: the direct search for certain paths, the reachability problem (the paths themselves are not requested, but it is necessary to prove their existence), etc.


%В рамках этих задач могут  задаваться специальные свойства искомых  путей. Одним из способов описания таких свойств является использование формального языка над некоторым алфавитом. Таким способом можно ограничить множество слов, получаемых конкатенацией меток на дугах рассматриваемых путей. Таким образом, в задачах поиска путей в помеченном графе с заданным формальным языком рассматриваются только те пути, которые образуют слова, принадлежащие этому языку. В настоящее время активно исследуются ограничения, представленные в виде контекстно-свободных (КС) языков~\cite{rehof2001type,reps1998program,bradford2017efficient,hellings2014conjunctive,hellings2020explaining}. Этот подход позволяет описывать более широкий набор ограничений, чем активно используемые на практике регулярные выражения~\cite{koschmieder2012regular,calvanese2000answering}.
In the path querying problems special properties of the desired paths can be specified. Such properties can be described, for example, using a formal language over some alphabet~\cite{barrett2000formal}. In this way, one can constrain the set of words obtained by concatenation of labels on the edges of some path. Thus, in path querying problems for a given labeled graph and a formal language, only paths that form words from this language are requested. Such problems are also called formal language-constrained path querying problems. Constraints represented as context-free languages (CFLs) are used in context-free path querying (CFPQ) problem and are being actively studied~\cite{rehof2001type,reps1998program,bradford2017efficient,hellings2014conjunctive,hellings2020explaining}. CFLs allow one to describe a wider set of constraints than regular expressions that are actively used in practice~\cite{koschmieder2012regular,calvanese2000answering}.

%С практической точки зрения, одним из распространённых способов получения высокопроизводительных реализаций алгоритмов анализа графов является использование методов линейной алгебры~\cite{kepner2011graph}. При этом существующие алгоритмы, фактически, переводятся на язык линейной алгебры, т.е. для представления графа используются разреженные матрицы (такие матрицы, которые имеют малое количество ненулевых элементов), а для анализа и преобразований графа используются операции над матрицами (умножение матриц, сложение, транспонирование матриц и т.д.). Например, давно известны такие представления графа, как матрица смежности или матрица инцидентности, а такому преобразованию ориентированного графа, как инвертирование направлений рёбер соответствует операция транспонирования матрицы смежности. Для тех алгоритмов анализа графов, которые позволяют такой <<перевод>>, становится возможным  использовать параллельные вычисления, в частности, на основе GPU-технологий, что позволяет существенно улучшить их производительность. Кроме того, такого рода алгоритмы зачастую просты в реализации, так как позволяют использовать существующие библиотеки линейной алгебры (SuiteSparse:GraphBLAS$\footnote{SuiteSparse:GraphBLAS~--- реализация стандарта GraphBLAS на языке Си: https://github.com/DrTimothyAldenDavis/GraphBLAS (дата обращения: 14.01.2022).}$, cuSPARSE$\footnote{Библиотека линейной алгебры cuSPARSE, используемая для работы с разреженными матрицами на GPU: https://docs.nvidia.com/cuda/cusparse/index.html (дата обращения: 14.01.2022).}$, cuBLAS$\footnote{Библиотека линейной алгебры cuBLAS на основе GPU-технологий: https://docs.nvidia.com/cuda/cublas/index.html (дата обращения: 14.01.2022).}$, cuBool$\footnote{Библиотека булевой линейной алгебры cuBool, использумая для работы с булевыми разреженными матрицами и векторами на GPU: https://github.com/JetBrains-Research/cuBool (дата обращения: 14.01.2022).}$, m4ri$\footnote{m4ri~--- библиотека для быстрой арифметики с плотными булевыми матрицами: https://github.com/malb/m4ri (дата обращения: 14.01.2022).}$, Scipy$\footnote{Scipy~--- библиотека для языка программирования Python с открытым исходным кодом, предназначенная для выполнения научных и инженерных расчётов: https://scipy.org/ (дата обращения: 14.01.2022).}$ и др.).
From a practical point of view, one of the common ways to obtain high-performance implementations of graph analysis algorithms is to use linear algebra methods~\cite{kepner2011graph}. Hence, the existing algorithms are translated into the language of sparse linear algebra. As a result, sparse matrices (matrices with a small number of nonzero elements) are used to represent a graph, and matrix operations (matrix multiplication, addition, matrix transposition, etc.) are used to perform graph analysis. For example, a graph can be represented using the adjacency matrix, and such a transformation of a directed graph as inverting the edge directions can be done by the adjacency matrix transposition. For those graph analysis algorithms that allow such translation, it becomes possible to use parallel computing, in particular, based on GPU technologies, which can significantly improve their performance. In addition, such algorithms are often easy to implement, as they allow one to use the existing linear algebra libraries (SuiteSparse:GraphBLAS$\footnote{SuiteSparse:GraphBLAS is a \texttt{C} implementation of the GraphBLAS standard: https://github.com/DrTimothyAldenDavis/GraphBLAS (date of access: 14.01.2022).}$, cuSPARSE$\footnote{GPU-based linear algebra library cuSPARSE for computing sparse matrix operations: https://docs.nvidia.com/cuda/cusparse/index.html (date of access: 14.01.2022).}$, cuBLAS$\footnote{GPU-based linear algebra library cuBLAS: https://docs.nvidia.com/cuda/cublas/index.html (date of access: 14.01.2022).}$, cuBool$\footnote{Linear algebra library cuBool for computing sparse Boolean matrix operations on GPU: https://github.com/JetBrains-Research/cuBool (date of access: 14.01.2022).}$, m4ri$\footnote{m4ri is a library for fast arithmetic with dense Boolean matrices: https://github.com/malb/m4ri (date of access: 14.01.2022).}$, Scipy$\footnote{Scipy is a free and open-source \texttt{Python} library used for scientific and technical computing: https://scipy.org/ (date of access: 14.01.2022).}$, etc.).

%Однако возможность использования методов линейной алгебры в задачах поиска путей в графе с заданными КС-ограничениями в настоящее время не исследована. Как показывает практика, существующие решения данной задачи страдают от недостаточной производительности и не справляются с постоянно растущими размерами реальных графов~\cite{kuijpers2019experimental}. В то же время создание новых решений, использующих методы линейной алгебры, позволит решить данную проблему с помощью теоретических и практических достижений линейной алгебры.
However, there have been no studies so far on the possibility of applying the linear algebra methods to CFPQ problem. The existing solutions to this problem suffer from insufficient performance and cannot cope with the constantly growing sizes of real graphs~\cite{kuijpers2019experimental}. At the same time, the creation of new linear algebra solutions allows one to solve this problem using the theoretical and practical linear algebra results.

{\progress}
%В последнее время появилось значительное число работ, посвященных классическим алгоритмам анализа графов, переведённых на язык линейной алгебры. Например, Айдын Булук (Aydin Bulu\c{c}), Упасана Шридхар (Upasana Sridhar), Питер Чжан (Peter Zhang), Арифул Азад (Ariful Azad) и Лейуань Ван (Leyuan Wang) в своих работах~\cite{bulucc2011parallel,sridhar2019delta,zhang2016gbtl,azad2015parallel,wang2016comparative} показывают применимость на практике алгебраических версий таких алгоритмов, как поиск в ширину, алгоритм Дейкстры, алгоритм Беллмана-Форда, поиск наибольшего паросочетания в двудольном графе и подсчёт количества треугольников в графе.
Recently, a significant number of works have appeared devoted to classical graph analysis algorithms translated into the linear algebra language. For example, Aydin Bulu\c{c}, Upasana Sridhar, Peter Zhang, Ariful Azad, and Leyuan Wang in their works~\cite{bulucc2011parallel,sridhar2019delta,zhang2016gbtl,azad2015parallel,wang2016comparative} have shown the practical applicability of algebraic versions of such algorithms as breadth-first search, Dijkstra's algorithm, Bellman-Ford's algorithm, algorithm for finding a matching in a bipartite graph, and algorithm for triangle counting.

%На фоне роста популярности идеи решения задач анализа графов с помощью методов линейной алгебры относительно недавно был создан стандарт GraphBLAS~\cite{graphblas}, который определяет базовые <<строительные блоки>> алгоритмов анализа графов в терминах линейной алгебры. Такими <<блоками>>, например, являются умножение и другие операции над матрицами, так как стандарт GraphBLAS использует представление графов в виде матриц смежности. Также, ввиду того, что данные на практике разрежены, целесообразно использовать разреженный формат для этих матриц. Стоит отметить, что не каждый алгоритм анализа графов можно переформулировать на языке линейной алгебры. Так, например, до сих пор это не сделано для алгоритма поиска в глубину. Также в настоящее время это не сделано и для алгоритмов поиска путей в графе с заданными КС-ограничениями.
Recently, the idea of using linear algebra methods to solve CFPQ problem has become very popular. Hence, the GraphBLAS~\cite{graphblas} standard that defines the basic building blocks of graph analysis algorithms in terms of linear algebra was recently created. This standard uses the adjacency matrices for graph representation and matrix operations to perform graph analysis. Also, since the real data are often sparse, it makes sense to use a sparse format for these matrices. Note that not every graph analysis algorithm can be formulated in terms of linear algebra. For example, it is still not done for the depth-first search algorithm~\cite{spampinato2019linear}. Also, at present, it is still not done for CFPQ algorithms.

%Задача поиска путей в графе с заданными КС-ограничениями является одной из важных задач анализа графов. Её частным случаем является задача синтаксического анализа КС-языков, в которой анализируются строки, что эквивалентно анализу только линейно-помеченных графов. Лесли Вэлиант (Leslie Valiant) провёл исследование~\cite{valiant1975general}, посвященное синтаксическому анализу КС-языков с использованием операций над матрицами. Предложенный им алгоритм для заданных строки и КС-грамматики определяет порождается ли эта строка заданной грамматикой с помощью операций умножения булевых матриц. Впервые вопрос о возможности нахождения матричного алгоритма поиска путей в графе с заданными КС-ограничениями исследовал Михалис Яннакакис (Mihalis Yannakakis)~\cite{yannakakis1990graph}. Он указывал, что алгоритм Вэлианта может быть расширен для анализа графов без циклов, но сомневался в возможности расширения алгоритма Вэлианта для работы с произвольными графами.
The CFPQ problem is one of the important graph analysis problems. Its partial case is the problem of context-free recognition, where strings are analyzed. Leslie Valiant has done research~\cite{valiant1975general} on context-free recognition using matrix operations. He proposed a subcubic algorithm that for a given string and a context-free grammar determines whether this string is generated by a given grammar using Boolean matrix multiplication. For the first time, the question of the possibility of finding a matrix-based CFPQ algorithm was raised by Mihalis Yannakakis~\cite{yannakakis1990graph}. He pointed out that Valiant's algorithm could be extended to analyze graphs without cycles (DAGs), but doubted the possibility of creating a subcubic algorithm to analyze arbitrary graphs.

%Однако для частного случая КС-ограничений существует алгоритм поиска путей в графе, сформулированный на языке линейной алгебры. Такой алгоритм был предложен Филипом Брэдфордом (Philip Bradford)~\cite{bradford2017efficient}, исследовавшим задачу достижимости в графе с заданными КС-ограничениями.
However, for a partial case of CFPQ problem, there is an algorithm formulated in the language of linear algebra. Such an algorithm was proposed by Philip Bradford~\cite{bradford2017efficient}, who studied the reachability problem in a graph with given particular context-free path constraints.

%Кроме того, существует ряд работ~\cite{hellings2014conjunctive,medeiros2018efficient,santos2018bottom,grigorev2017context}, посвященных задаче поиска путей в произвольном графе с заданными произвольными КС-ограничениями и основанных на различных алгоритмах синтаксического анализа (LR, LL, GLL, CYK). Среди них работы Семёна Григорьева, Джелле Хеллингса (Jelle Hellings), Чиро Медейроса (Ciro Medeiros) и Мартина Мюзиканте (Martin Musicante). Стоит отметить, что большинство из представленных алгоритмов требуют представить КС-ограничения на пути в графе в виде КС-грамматики в некоторой нормальной форме. Отдельный интерес представляют собой алгоритмы, не требующие дополнительных преобразований структур, описывающих входные КС-ограничения, так как почти любое такое преобразование обычно приводит к увеличению размеров этих структур, что может негативно сказаться на производительности. Кроме того, после таких преобразований могут возникнуть сложности с интерпретацией результатов анализа графа в терминах изначальной структуры, заданной пользователем. Примерами алгоритмов поиска путей в графе с заданными КС-ограничениями, не требующих преобразований входной КС-грамматики, являются алгоритмы~\cite{medeiros2018efficient,grigorev2017context}, основанные на алгоритмах синтаксического анализа LL и GLL.
In addition, there are a number of papers~\cite{hellings2014conjunctive,medeiros2018efficient,santos2018bottom,grigorev2017context} devoted to CFPQ problem with an arbitrary graph and arbitrary context-free path constraints that are based on various parsing techniques (LR, LL, GLL, CYK). In these works, Semyon Grigorev, Jelle Hellings, Ciro Medeiros, and Martin Musicante proposed algorithms that require the context-free path constraints to be represented as a context-free grammar (CFG) in some normal form. Algorithms that do not require additional transformations of structures that describe input path constraints are of particular interest, since almost any such transformation usually leads to an increase in the size of these structures, and can adversely affect performance. In addition, after such transformations, it may be difficult to interpret graph analysis results in terms of the original structure specified by the user. Examples of CFPQ algorithms that do not require transformations of the input CFG are the algorithms~\cite{medeiros2018efficient,grigorev2017context} based on the LL and GLL parsing algorithms.

%Таким образом, на текущий момент не существует алгоритма поиска путей в произвольном графе с заданными произвольными КС-ограничениями, выраженного на языке линейной алгебры. Поэтому необходимо исследовать возможность разработки таких алгоритмов.
Thus, at the moment, there is no linear algebra-based CFPQ algorithm for an arbitrary graph and arbitrary context-free path constraints. Therefore, it is necessary to explore the possibility of developing such algorithms.
%Этот раздел должен быть отдельным структурным элементом по
% ГОСТ, но он, как правило, включается в описание актуальности
% темы. Нужен он отдельным структурынм элемементом или нет ---
% смотрите другие диссертации вашего совета, скорее всего не нужен.

%{\aim} данной работы является исследование применимости методов линейной алгебры к задаче поиска путей в графе с заданными КС-ограничениями для получения высокопроизводительных реализаций на основе параллельных вычислений.
{\aim} of this work is to study the applicability of linear algebra methods to the context-free path querying problem in order to obtain high-performance implementations based on parallel computing.

%Достижение поставленной цели обеспечивается решением следующих {\tasks}.
Achieving this goal is ensured by solving the following {\tasks}.
\begin{enumerate}[beginpenalty=10000] % https://tex.stackexchange.com/a/476052/104425
  %\item Разработать подход к поиску путей в графе с КС-ограничениями на основе методов линейной алгебры.
  \item To develop an approach to the context-free path querying based on linear algebra methods.
  %\item Разработать алгоритм, использующий предложенный подход и решающий задачи поиска путей в графе с заданными КС-ограничениями.
  \item To devise a CFPQ algorithm that uses the proposed approach.
  %\item Разработать алгоритм поиска путей в графе с заданными КС-ограничениями, использующий предложенный подход и не требующий преобразования входной КС-грамматики.
  \item To devise a CFPQ algorithm that uses the proposed approach and does not require a transformation of the input context-free grammar.
  %\item Реализовать предложенные алгоритмы с использованием параллельных вычислений, провести их экспериментальное исследование на реальных данных, сравнить их с существующими реализациями, а также между собой.
  \item To implement the devised algorithms using parallel computing, conduct their experimental study on real data, and compare them with existing implementations.
\end{enumerate}


{\influence} 
%Теоретическая значимость диссертационного исследования заключается в разработке подхода к поиску путей в графе с заданными КС-ограничениями, использующего методы линейной алгебры, в разработке формальных алгоритмов, использующих полученный подход, а также в формальном доказательстве завершаемости, корректности и оценок временной сложности разработанных алгоритмов.
The theoretical influence of this thesis research lies in the development of a linear algebra based approach to the CFPQ, in the formal algorithms development using the obtained approach, as well as in the formal proof of the termination, correctness, and the time complexity of the developed algorithms.

%В ходе исследования предложенные алгоритмы реализованы с использованием параллельных вычислений, что \textbf{позволило увеличить производительность в Х раз} по сравнению с существующими реализациями. Кроме того, выполненные реализации могут быть интегрированы с такими графовыми базами данных, как RedisGraph. Это позволит расширить языки запросов к этим базам данных.
In this thesis, the proposed algorithms were implemented using parallel computing that allowed us to obtain up to 3 orders of magnitude faster graph analysis time and consume up to 2 orders of magnitude less memory compared to existing solutions. In addition, the implementations made can be integrated with graph databases such as RedisGraph. This will extend the query languages for these databases.

%{\methods} Методология исследования основана на линейной алгебре и теории графов. В работе использован стандарт GraphBLAS, объединяющий теорию графов и линейную алгебру. Кроме того, в исследовании использовалась теория формальных языков и грамматик, а также теория сложности. Наконец, для реализации алгоритмов использовались CPU и GPU-технологии.
{\methods} The research methodology is based on the linear algebra and the graph theory. The work uses the GraphBLAS standard that combines these areas. In addition, the formal language theory was used in this work, as well as the complexity theory. Finally, CPU and GPU technologies were used to implement the algorithms.%Одной из задач, для которых до сих пор не найдена формулировка в терминах линейной алгебры, является поиск путей в графе с ограничениями в виде КС-грамматик. Данная задача использует подход к анализу строк, который начал активно развиваться в 50-х годах 20-го века в связи с изучением естественных языков (работы Н.~Хомского). В последствии этот подход получил широкое распространение в различных областях, в том числе и связанных с анализом графов.
%При этом основными элементами данного подхода являются алфавит и грамматика исследуемого языка, выступающего в качестве ограничения на искомые пути в графе. Решаемые в связи с этим задачи связаны с поиском эффективных алгоритмов нахождения путей, удовлетворяющих заданным ограничениям.

{\defpositions}
\begin{enumerate}[beginpenalty=10000] % https://tex.stackexchange.com/a/476052/104425
    \item An approach to the context-free path querying based on linear algebra methods that allows one to use theoretical and practical linear algebra results was developed.
    \item A CFPQ algorithm that uses the proposed approach was devised. Termination and correctness of the devised algorithm were proved, as well as its time complexity. The proposed algorithm uses matrix operations, which make it possible to apply a wide class of optimizations and allows one to automatically parallelize computations using existing linear algebra libraries.
    \item A CFPQ algorithm that uses the proposed approach and does not require a transformation of the input context-free grammar was devised. Termination and correctness of the devised algorithm were proved, as well as its time complexity. The proposed algorithm makes it possible to work with arbitrary input context-free grammars without any transformations. This allows one to avoid a significant increase in the grammar size that affects analysis performance.
    \item The devised algorithms are implemented using parallel computing techniques. An experimental study of the devised algorithms was provided using real RDF data and graphs built for static program analysis. The obtained implementations were compared with each other, with existing solutions from the field of static program analysis, and with solutions based on various parsing techniques. The comparison results show that the proposed implementations for the reachability problem allow one to obtain up to 2 orders of magnitude faster graph analysis time and consume up to 2 times less memory in comparison with existing solutions, and for the problems of finding one and all paths in a graph allow one to speed up the analysis time up to 3 orders of magnitude and consume up to 2 orders of magnitude less memory.
\end{enumerate}

{\novelty.}
\begin{enumerate}[beginpenalty=10000] % https://tex.stackexchange.com/a/476052/104425
	
	%\item Предложен новый подход к поиску путей в графе с заданными КС-ограничениями, который позволяет использовать теоретические и практические аспекты линейной алгебры для решения задачи поиска путей в графе с КС-ограничениями.
	\item A new approach to the CFPQ that allows one to use theoretical and practical linear algebra results is proposed.
	
	%\item Впервые получен алгоритм поиска путей в произвольном графе с заданными произвольными КС-ограничениями, сформулированный в терминах линейной алгебры, что позволяет применять в этом алгоритме широкий класс оптимизаций для вычисления операций над матрицами, распараллеливать вычисления и существенно улучшить производительность.
	\item For the first time, a linear algebra based CFPQ algorithm for an arbitrary graph and arbitrary context-free path constraints was obtained. This algorithm makes it possible to apply a wide class of matrix optimizations, to use parallel computing techniques, and to significantly improve the performance.
	
	%\item В диссертации также предложен алгоритм поиска путей в графе с заданными КС-ограничениями, сформулированный в терминах линейной алгебры и не требующий преобразования входной КС-грамматики, в отличие от алгоритмов, предложенных в работах Семёна Григорьева, Джелле Хеллингса и Филиппа Брэдфорда. Таким образом, предложенный в диссертации алгоритм позволяет избежать значительного увеличения размера входной грамматики, от которого напрямую зависит его временная сложность.
	\item In this thesis, a linear algebra based CFPQ algorithm that does not require a transformation of the input context-free grammar was also proposed, in contrast to the algorithms proposed in the works of Semyon Grigorev, Jelle Hellings, and Philip Bradford. Thus, the algorithm proposed in this thesis allow one to avoid a significant increase in the grammar size that affects analysis performance.
	
	%\item Экспериментальное исследование алгоритмов поиска путей в произвольном графе с заданными произвольными КС-ограничениями, использующих операции линейной алгебры проводится впервые и позволяет судить о применимости на практике разработанных алгоритмов.
	
\end{enumerate}

{\reliability} 
%Достоверность и обоснованность результатов исследования опирается на использовании формальных методов доказательств и инженерные эксперименты.
The reliability and approbation of the research results are based on the use of formal proof methods and engineering experiments.

%Основные результаты работы были представлены на ряде международных научных конференций: Joint Workshop on Graph Data Management Experiences \& Systems (GRADES) and Network Data Analytics (NDA) (совместно с конференцией SIGMOD) 2018 (Хьюстон, Техас, США), 2020 (Портленд, Орегон, США) и 2021 (Сиань, Шэньси, Китай); 24th European Conference on Advances in Databases and Information Systems (ADBIS) 2020 (Лион, Франция); VLDB PhD Workshop совместно с 47th International Conference on Very Large Data Bases 2021 (Копенгаген, Дания). Также исследование было поддержано грантом РНФ \textnumero 18-11-00100 и грантом РФФИ \textnumero 19-37-90101.
The main results of the work were presented at a number of international scientific conferences: Joint Workshop on Graph Data Management Experiences \& Systems (GRADES) and Network Data Analytics (NDA) (co-located with the SIGMOD conference) 2018 (Houston, Texas, USA), 2020 (Portland, Oregon, USA), and 2021 (Xi'an, Shaanxi, China); 24th European Conference on Advances in Databases and Information Systems (ADBIS) 2020 (Lyon, France); VLDB PhD Workshop in cooperation with the 47th International Conference on Very Large Data Bases 2021 (Copenhagen, Denmark). The study was also supported by the RSF grant \textnumero 18-11-00100 and the RFBR grant \textnumero 19-37-90101.

%{\contribution} Автор принимал активное участие \ldots

{\publications} %Все результаты диссертации изложены в 8 научных работах~\cite{azimov1,azimov2,azimov3,azimov4,azimov5,azimov6,azimov7,azimov8}. Из них 3 работы~\cite{azimov6,azimov7,azimov8} опубликованы в журналах из <<Перечня российских рецензируемых научных журналов, в которых должны быть опубликованы основные научные результаты диссертаций на соискание ученых степеней доктора и кандидата наук>>, рекомендовано ВАК. 7 работ~\cite{azimov1,azimov2,azimov3,azimov4,azimov5,azimov6,azimov8} индексируются в базе данных Scopus. Работы~\cite{azimov1,azimov2,azimov3,azimov4,azimov6,azimov7,azimov8} написаны в соавторстве. В работе~\cite{azimov1} автору принадлежит разработка и реализация алгоритма, решающего задачу достижимости в графе с заданными КС-ограничениями с использованием методов линейной алгебры, доказательство корректности разработанного алгоритма, постановка экспериментов; соавторы участвовали в обсуждении основных идей статьи, выполняли обзор предметной области. В работе~\cite{azimov2} автору принадлежит разработка и реализация алгоритма, решающего задачу поиска одного пути в графе с заданными КС-ограничениями с использованием методов линейной алгебры, доказательство корректности разработанного алгоритма; соавторы проводили экспериментальное исследование, участвовали в формализации и улучшении изложения идей статьи. В работе~\cite{azimov3} вклад автора заключается в доказательстве корректности алгоритма поиска путей в графе с заданными КС-ограничениями, не требующего преобразований входной КС-грамматики, а также в работе над текстом; соавторам принадлежит идея алгоритма и постановка экспериментов. В работе~\cite{azimov4} автору принадлежит разработка и реализация алгоритма, решающего задачу поиска всех путей в графе с заданными КС-ограничениями с использованием методов линейной алгебры, доказательство корректности разработанного алгоритма, работа над текстом; соавторы проводили экспериментальное исследование. В работах~\cite{azimov6,azimov7} вклад автора заключается в разработке и реализации алгоритма, решающего задачу достижимости в графе с заданными ограничениями в виде конъюнктивных языков, доказательстве корректности разработанного алгоритма и постановке экспериментов; соавторы участвовали в обсуждении основных идей статьи и выполняли обзор предметной области. В работе~\cite{azimov8} автору принадлежит разработка и реализация алгоритма, решающего задачу поиска всех путей в графе с заданными КС-ограничениями с использованием матриц с множествами промежуточных вершин, а также работа над текстом; соавторы проводили экспериментальное исследование.
All results of this dissertation are presented in 8 scientific papers~\cite{azimov1,azimov2,azimov3,azimov4,azimov5,azimov6,azimov7,azimov8}. %Of these, 3 papers~\cite{azimov6,azimov7,azimov8} were published in journals from the <<List of Russian peer-reviewed scientific journals in which the main scientific results of dissertations for the degree of doctor and candidate of science should be published>>, recommended by the Higher Attestation Commission.
5 works~\cite{azimov1,azimov2,azimov3,azimov4,azimov5} are indexed in the Scopus database. Works~\cite{azimov1,azimov2,azimov3,azimov4,azimov6,azimov7,azimov8} are made with co-authors. In the paper~\cite{azimov1}, the author owns the development and implementation of an algorithm that solves the CFPQ problem with the reachability query semantics using linear algebra methods, proof of the algorithm correctness, setting up experiments; co-authors participated in the discussion of the main ideas of the paper, carried out a review of the research area. In the paper~\cite{azimov2}, the author owns the development and implementation of an algorithm that solves the single-path CFPQ problem using linear algebra methods, proof of the algorithm correctness; co-authors conducted an experimental study, participated in the formalization and improvement of the paper. In the work~\cite{azimov3}, the author's contribution consists in proving the correctness of the CFPQ algorithm that does not requires a transformation of the input grammar, as well as in working on the text; the co-authors own the idea of the algorithm and setting up the experiments. In the work~\cite{azimov4}, the author owns the development and implementation of the all-path CFPQ algorithm using linear algebra methods, proof of the algorithm correctness, work on the text; co-authors conducted an experimental study. In the papers~\cite{azimov6,azimov7}, the author's contribution is developing and implementing a path querying algorithm that solves the reachability problem in a graph with given path constraints in the form of conjunctive languages, proving the algorithm correctness, and setting up experiments; co-authors participated in the discussion of the main ideas of the paper and performed a review of the research area. In the work~\cite{azimov8}, the author is responsible for the development and implementation of the all-path CFPQ algorithm using matrices with sets of intermediate vertices, as well as work on the text; co-authors conducted an experimental study.

\paragraph{Acknowledgments.} First of all, I would like to thank my supervisor, Semyon Grigorev, for his guidance at all stages of this research, for his willingness to support and share his experience, and for his invaluable contribution to my work. I would also like to thank Dmitry Koznov for his wisdom, active participation, and numerous conversations about my dissertation, which had a great impact on my work and on me in general.

I express my gratitude to Andrey Terekhov and the Department of System Programming of St. Petersburg State University, to Dmitry Bulychev, Andrey Ivanov, as well as to JetBrains and Huawei for the unique opportunity to engage in science as the main activity.

I am grateful to Vladimir Kutuev and Vlada Pogozhelskaya for their assistance with the experiments.

I would like to express special gratitude to my wife, Svetlana Azimova, and my son, Artyom Azimov, for inspiration, love, and support, because they occupy all my thoughts and all my heart, and I do all things in my life exactly for the sake of them. I also want to thank my brother, Timur Azimov, for sharing my interests and supporting. Finally, I am grateful to my dear parents, Shukhratullo Azimov and Elena Azimova, who have been my support throughout my life and made me who I am today.
