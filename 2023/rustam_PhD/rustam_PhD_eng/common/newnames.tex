% Новые переменные, которые могут использоваться во всём проекте
% ГОСТ 7.0.11-2011
% 9.2 Оформление текста автореферата диссертации
% 9.2.1 Общая характеристика работы включает в себя следующие основные структурные
% элементы:
% актуальность темы исследования;
\newcommand{\actualityTXT}{Actuality.}
% степень ее разработанности;
\newcommand{\progressTXT}{Background.}
% цели и задачи;
\newcommand{\aimTXT}{The goal}
\newcommand{\tasksTXT}{tasks}
% научную новизну;
\newcommand{\noveltyTXT}{Scientific novelty}
% теоретическую и практическую значимость работы;
%\newcommand{\influenceTXT}{Теоретическая и практическая значимость}
% или чаще используют просто
\newcommand{\influenceTXT}{Theoretical and practical influence.}
% методологию и методы исследования;
\newcommand{\methodsTXT}{Methodology and research methods.}
% положения, выносимые на защиту;
\newcommand{\defpositionsTXT}{The main results submitted for defense.}
% степень достоверности и апробацию результатов.
\newcommand{\reliabilityTXT}{The reliability and approbation of the results.}

\newcommand{\contributionTXT}{Personal contribution.}
\newcommand{\publicationsTXT}{Publications.}


%%% Заголовки библиографии:

% для автореферата:
\newcommand{\bibtitleauthor}{Публикации автора по теме диссертации}

% для стиля библиографии `\insertbiblioauthorgrouped`
\newcommand{\bibtitleauthorvak}{В изданиях из списка ВАК РФ}
\newcommand{\bibtitleauthorscopus}{В изданиях, входящих в международную базу цитирования Scopus}
\newcommand{\bibtitleauthorwos}{В изданиях, входящих в международную базу цитирования Web of Science}
\newcommand{\bibtitleauthorother}{В прочих изданиях}
\newcommand{\bibtitleauthorconf}{В сборниках трудов конференций}

% для стиля библиографии `\insertbiblioauthorimportant`:
\newcommand{\bibtitleauthorimportant}{Наиболее значимые \protect\MakeLowercase\bibtitleauthor}

% для списка литературы в диссертации и списка чужих работ в автореферате:
\newcommand{\bibtitlefull}{References} % (ГОСТ Р 7.0.11-2011, 4)

