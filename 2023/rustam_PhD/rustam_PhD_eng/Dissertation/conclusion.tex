\chapter*{Conclusion}                       % Заголовок
\addcontentsline{toc}{chapter}{Conclusion}  % Добавляем его в оглавление

%% Согласно ГОСТ Р 7.0.11-2011:
%% 5.3.3 В заключении диссертации излагают итоги выполненного исследования, рекомендации, перспективы дальнейшей разработки темы.
%% 9.2.3 В заключении автореферата диссертации излагают итоги данного исследования, рекомендации и перспективы дальнейшей разработки темы.

\begin{enumerate}[beginpenalty=10000] % https://tex.stackexchange.com/a/476052/104425
	\item Разработан подход к поиску путей в графе с заданными КС-ограничениями на основе методов линейной алгебры, который позволяет использовать теоретические и практические достижения линейной алгебры для решения данной задачи.
	\item Разработан алгоритм, использующий предложенный подход и решающий задачи поиска путей в графе с заданными КС-ограничениями. Доказана завершаемость и корректность предложенного алгоритма. Получена теоретическая оценка сверху временной сложности алгоритма. Предложенный алгоритм использует операции над матрицами, которые позволяют применять широкий класс оптимизаций и дают возможность автоматически распараллеливать вычисления за счёт существующих библиотек линейной алгебры.
	\item Разработан алгоритм поиска путей в графе с заданными КС-ограничениями, использующий предложенный подход и не требующий преобразования входной КС-грамматики. Доказана завершаемость и корректность предложенного алгоритма. Получена теоретическая оценка сверху временной сложности алгоритма. Предложенный алгоритм позволяет работать с произвольными входными КС-грамматиками без необходимости их преобразования, что позволяет избежать значительного увеличения размеров входной грамматики и увлечения времени работы алгоритма.
	\item Предложенные алгоритмы реализованы с использованием параллельных вычислений. Проведено экспериментальное исследование разработанных алгоритмов на реальных RDF данных и графах, построенных для статического анализа программ. Было проведено сравнение полученных реализаций между собой, с существующими решениями из области статического анализа и с решениями, основанными на различных алгоритмах синтаксического анализа. Результаты сравнения показывают, что предложенные реализации для задачи достижимости позволяют ускорить время анализа до 2 порядков и потребляют до 2 раз меньше памяти по сравнению с существующими решениями, а для задач поиска одного и поиска всех путей в графе позволяют ускорить время анализа до 3 порядков и до 2 порядков снизить потребление памяти.
\end{enumerate}


$\textit{CFPQ\_PyAlgo}$ platform for developing, testing, and benchmarking CFPQ algorithm was created using the obtained implementations.

We give the following \textbf{application recommendations for the work results}. The developed approach and the obtained algorithms are applicable for the CFPQ using linear algebra. Also, provided approach allows one to obtain high-performance parallel CFPQ implementations that are compact and portable. These implementations can be created using existing linear algebra libraries. The $\textit{CFPQ\_PyAlgo}$ platform can be used in static program analysis~\cite{rehof2001type,zheng2008demand}, RDF analysis~\cite{zhang2016context}, bioinformatics~\cite{sevon2008subgraph}, etc. In addition, CFPQ implementations can be integrated with graph databases such as RedisGraph. In work~\cite{azimov2}, we provided such prototype implementations. However, to provide full integration it is necessary to extend Cypher graph query language used in RedisGraph and to support syntax for specification of context-free path constraints. Moreover, there is a proposal$\footnote{A proposal with path pattern syntax for openCypher:\\ https://github.com/thobe/openCypher/blob/rpq/cip/1.accepted/CIP2017-02-06-Path-Patterns.adoc (date of access: 14.01.2022).}$ that describes such syntax extension.

Also, we identify \textbf{prospects for further development of the topic}. First of all, the proposed approach and developed algorithms can be applied to creation of a specialized tool for a specific graph analysis problem. The CFPQ algorithms for graphs of a certain type and for specific path constraints can be created using the proposed approach. For example, for a static program analysis the structure of graphs derived from programs in a particular programming language can be taken into account, as well as the properties of a particular CFL for the chosen analysis (alias analysis~\cite{zheng2008demand}, taint analysis~\cite{taint}, etc.).

In practice, CFPQ problem is rarely solved without fixing a relatively small set of possible source and destination vertices. Thus, often the information about paths between any vertices is redundant. Therefore, another direction of further research is modification of all proposed CFPQ algorithms with additional restrictions on sets of source and destination vertices.

In this work, the proposed matrix-based CFPQ algorithm for the single-path and all-path query semantics was implemented only on CPU. For these query semantics, we use more complex data types for matrix elements. This leads to a significant increase in memory consumption and does not allow us to obtain a high-performance GPU implementation. Therefore, further research is needed to optimize this algorithm and the algebraic structures used. In the future, it may be possible to obtain high-performance GPU implementations of this algorithm using the CUSP library and the GraphBLAST library.

In addition, there are some graph analysis problems that cannot be expressed using the context-free path constraints. For example, the context-sensitive data-dependence program analysis~\cite{linearconjunctive} uses an interleaved matched-parenthesis language that is not context-free. This problem is well-known to be undecidable~\cite{linearconjunctive}. However, path constraints in the form of linear conjunctive languages~\cite{okhotin2001conjunctive} that belong to a wider class of languages than context-free ones, can be used to approximate the result of such analysis. Thus, the relevant research direction is to extend the proposed approach to solve path querying problems with path constraints expressed by languages from a broader class of languages than the CFLs.
