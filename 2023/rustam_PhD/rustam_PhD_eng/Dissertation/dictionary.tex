\chapter*{Glossary}             % Заголовок
\addcontentsline{toc}{chapter}{Glossary}  % Добавляем его в оглавление

\textbf{алфавит (англ. alphabet)} : в теории формальных языков --- конечное множество атомарных (неделимых) символов какого-либо формального языка

\textbf{вес дуги (англ. edge weight)} : значение, поставленное в соответствие данной дуге взвешенного графа, обычно вещественное число

\textbf{взвешенный граф (англ. weighted graph)} : граф, каждой дуге которого поставлено в соответствие некое значение (разновидность помеченного графа)

\textbf{графовая база данных (англ. graph database)} : разновидность баз данных с реализацией сетевой модели в виде графа и его обобщений

\textbf{достижимость (англ. reachability)} : отношение между парами вершин в графе, описывающее существование пути из одной в другую

\textbf{дуга (англ. edge, directed edge, arc)} : ориентированное ребро

\textbf{конкатенация (англ. concatenation)} : операция склеивания объектов линейной структуры, обычно строк

\textbf{кратчайший путь (англ. shortest path)} : путь во взвешенном графе, в котором минимизируется сумма весов дуг, составляющих путь

\textbf{матрица (англ. matrix)} :  математический объект, записываемый в виде прямоугольной таблицы элементов, который представляет собой совокупность строк и столбцов, на пересечении которых находятся его элементы

\textbf{метка (англ. label)} : информация ассоциированная с вершиной или дугой графа, например, натуральные числа или символы некоторого алфавита

\textbf{нетерминальный символ (англ. nonterminal)} : объект, обозначающий какую-либо сущность языка (например: формула, арифметическое выражение, команда) и не имеющий конкретного символьного значения

\textbf{плотная матрица (англ. dense matrix)} : матрица над некоторым полем, кольцом или полукольцом, состоящая в основном не из нулевых элементов этих алгебраических структур

\textbf{помеченный граф (англ. labeled graph)} : граф, вершинам или дугам которого присвоены какие-либо метки

\textbf{путь (англ. path)} : последовательность вершин и рёбер (в неориентированном графе) и/или дуг (в ориентированном графе), в которой каждый элемент инцидентен предыдущему и последующему

\textbf{разреженная матрица (англ. sparse matrix)} : матрица над некоторым полем, кольцом или полукольцом, состоящая в основном из нулевых элементов этих алгебраических структур, так, например, для матриц над числовыми полями таким элементом часто является ноль

\textbf{ребро (англ. undirected edge)} : соединяет две вершины графа

\textbf{синтаксический анализ (англ. parsing)} : в лингвистике и информатике --- процесс сопоставления линейной последовательности лексем (распознанные группы символов) естественного или формального языка с его формальной грамматикой

\textbf{слово/строка (англ. word/string)} : в теории формальных языков --- произвольная последовательность символов из некоторого алфавита

\textbf{смежность (англ. adjacency)} : понятие, используемое в отношении только двух рёбер/дуг, имеющих общую вершину, либо только двух вершин, соединённых некоторым ребром/дугой

\textbf{терминальный символ (англ. terminal)} : объект, непосредственно присутствующий в словах языка, соответствующего формальной грамматике, и имеющий конкретное, неизменяемое значение

\textbf{формальная грамматика (англ. formal grammar)} : в теории формальных языков --- способ описания формального языка

\textbf{формальный язык (англ. formal language)} : множество конечных слов (строк, цепочек) над конечным алфавитом



































%\textbf{TeX} : Cистема компьютерной вёрстки, разработанная американским профессором информатики Дональдом Кнутом

%\textbf{панграмма} : Короткий текст, использующий все или почти все буквы алфавита
