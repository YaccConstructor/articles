\chapter{Comparison and Relation}\label{ch:ch6}
In this chapter, we present a comparison and relation of the obtained results to the main existing CFPQ solutions. A description of the existing solutions is presented in section~\ref{sec:ch1/sec5}.

We compare the following existing CFPQ tools: the $\textit{LL}$ implementation of the CFPQ algorithm~\cite{medeiros2018efficient} based on the LL(1) parsing algorithm; the implementations $\textit{GLL}_{\textit{R}}$ and $\textit{GLL}_{\textit{A}}$~\cite{grigorev2017context} of the CFPQ algorithm based on the GLL parsing algorithm; the $\textit{Graspan}$~\cite{graspan} static program analysis tool. The comparison was made with the proposed matrix-based implementations $\textit{MtxReach}_{\textit{CPU}}$, $\textit{MtxReach}_{\textit{GPU}}$, $\textit{MtxSingle}_{\textit{CPU}}$, and $\textit{MtxAll}_{\textit{CPU}}$, as well as with the proposed Kronecker product-based implementations $\textit{KronAll}_{\textit{CPU}}$ and $\textit{KronAll}_{\textit{GPU}}$. The comparison criteria presented in Table~\ref{tab:compCriteria}. 

\begin{table} [h]
  \centering
  \begin{threeparttable}% выравнивание подписи по границам таблицы
  \caption{Criteria for comparing CFPQ tools}\label{tab:compCriteria}
  \begin{tabular}{| p{4.5cm} | p{3.2cm} | p{8cm} |}
  \hline                               
  \hline
  Criteria & The column name in Table~\ref{tab:comparison} & Description \\
  \hline
  Does not require a transformation of the input CFG & Without transformations & Is there no need to transform the input CFG, for example into WCNF, in order to apply the algorithm and the corresponding tool?\\
  Path extraction  & Path extraction  & Does the algorithm and the corresponding tool compute information sufficient to construct found paths corresponding to the input context-free constraints?\\
  GPU usage & GPU & Does the tool make computations on GPU?\\
  The use of linear algebra   & Linear algebra & Are the algorithm and the corresponding tool linear algebra based?\\
  \hline
  \hline
  \end{tabular}
  \end{threeparttable}
\end{table}

The main results of the comparison are presented in Table~\ref{tab:comparison}. Thus, we can conclude the following.

\begin{itemize}
    \item In this work, we proposed the first linear algebra based CFPQ algorithms for the arbitrary context-free path constraints.
    \item Existing tools are mainly implemented on CPU.
    \item In this work, we proposed the first GPU implementation of a CFPQ algorithm for the single-path and all-path query semantics.
\end{itemize}

\begin{table} [h]
  \centering
\begin{threeparttable}
  \caption{The CFPQ tool comparison}\label{tab:comparison}
  \begin{tabular}{| p{3.5cm} || p{3.2cm} | p{2.6cm} | p{2.6cm} | p{2.4cm}l |}
  \hline                               
  \hline
  {A tool}              &\centering {Without transformations}        &\centering {Path extraction}    &\centering {GPU} &\centering {Linear algebra}  & \\
  \hline
  $\textit{LL}$                         &\centering  $+$                  &\centering  $-$             &\centering  $-$   &\centering  $-$  &\\
  $\textit{GLL}_{\textit{R}}$                &\centering  $+$                  &\centering  $-$              &\centering  $-$   &\centering  $-$    & \\
  $\textit{GLL}_{\textit{A}}$                &\centering  $+$                  &\centering  $+$\tnote{*}              &\centering  $-$   &\centering  $-$    & \\
  $\textit{Graspan}$                       &\centering  $-$\tnote{**}                   &\centering  $-$             &\centering  $+$\tnote{***} &\centering  $-$       &\\
  $\textit{MtxReach}_{\textit{CPU}}$                           &\centering  $-$                  &\centering  $-$             &\centering  $-$  &\centering  $+$   &\\
  $\textit{MtxReach}_{\textit{GPU}}$                          &\centering  $-$                  &\centering  $-$             &\centering  $+$  &\centering  $+$   &\\
  $\textit{MtxSingle}_{\textit{CPU}}$                           &\centering  $-$                  &\centering  $+$             &\centering  $-$\tnote{****}  &\centering  $+$   &\\
  $\textit{MtxAll}_{\textit{CPU}}$                           &\centering  $-$                  &\centering  $+$             &\centering  $-$\tnote{****}  &\centering  $+$   &\\
  $\textit{KronAll}_{\textit{CPU}}$                      &\centering  $+$                  &\centering  $+$             &\centering  $-$  &\centering  $+$    &\\
  $\textit{KronAll}_{\textit{GPU}}$                      &\centering  $+$                  &\centering  $+$             &\centering  $+$  &\centering  $+$    &\\
  \hline
  \hline
  \end{tabular}\small{
  \begin{tablenotes}
            \item[*] The implementation of the algorithm~\cite{grigorev2017context} based on the GLL parsing algorithm, builds an SPPF that contains information sufficient to construct all found paths.
            \item[**] The $\textit{Graspan}$ tool~\cite{graspan} requires to transform the input CFG into the WCNF.
            \item[***] There is also a GPU implementation of the $\textit{Graspan}$ tool~\cite{graspan} that the author failed to run. This implementation allows to speed up the pointer analysis of \texttt{C/C++} programs from 3.5 to 650 times.
            \item[****] The proposed matrix-based CFPQ algorithm was implemented on GPU only for the reachability query semantics. However, for the matrix-based CFPQ algoritm with the single-path and all-path query semantics it may be possible to obtain a high-performance GPU implementation using the CUSP library and the GraphBLAST library.
  \end{tablenotes}    }
  \end{threeparttable}
\end{table}


\clearpage