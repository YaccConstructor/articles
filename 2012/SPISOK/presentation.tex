        \documentclass{beamer}
\usepackage{beamerthemesplit}
\usetheme{SPbGU}
%{CambridgeUS}
% Выпишем часть возможных стилей, некоторые из них могут содержать
% дополнительные опции
% Darmstadt, Ilmenau, CambridgeUS, default, Bergen, Madrid, AnnArbor,Pittsburg, Rochester,
% Antiles, Montpellier, Berkley, Berlin
\usepackage{pdfpages}
\usepackage{amsmath}
\usepackage{cmap} % for serchable pdf's
\usepackage[T2A]{fontenc} 
\usepackage[utf8]{inputenc}
\usepackage[english,russian]{babel}
\usepackage{indentfirst}
\usepackage{amsmath}
\usepackage{dot2texi}
\usepackage{tikz}
\usepackage{graphicx}
\usetikzlibrary{shapes,arrows}
% Если у вас есть логотип вашей кафедры, факультета или университета, то
% его можно включить в презентацию.

%\usefoottemplate{\vbox{}}%  \tinycolouredline{structure!25}% {\color{white}\textbf{\insertshortauthor\hfill% \insertshortinstitute}}% \tinycolouredline{structure}% {\color{white}\textbf{\insertshorttitle}\hfill}% }}

%\logo{\includegraphics[width=1cm]{SPbGU_Logo.png}}

%[GLR-анализатор]
\title[]{Применение абстракного синтаксического анализа для трансляции динамически формируемых строк}
%\subtitle[студроект]{Студенческий проект}
\institute[СПбГУ]{
Санкт-Петербургский государственный университет \\
Математико-Механический факультет \\
Кафедра системного программирования }

\author[Григорьев Семён]{Григорьев Семён Вячеславович}

\date{25 апреля 2012г.}

\begin{document}
{

\begin{frame}
    \begin{center}
        {\includegraphics[width=1cm]{SPbGU_Logo.png}}
    \end{center}
    \titlepage
\end{frame}
}

\begin{frame}
	\transwipe[direction=90]
	\frametitle{Описание проблемы}
    Встроенные языки широко распространены.
	\begin{itemize}
		\item Встроенный SQL
		\item Генерация HTML
		\item DSL
		\item ...
	\end{itemize}
\end{frame}

\begin{frame}[fragile]
	\transwipe[direction=90]
	\frametitle{Описание проблемы}
	\begin{itemize}
	    \item Динамически генерируемые строки -- тоже код на некотором формальном языке программирования, который необходимо соответствующим образом обрабатывать.
	    \pause
	    \item После преобразований необходимо гарантировать:
	    \begin{itemize}
		    \item синтаксическую корректность
		    \item корректность по отношению к изменениям "внешнего" \ кода
		        	\begin{itemize}
                		\item Переименование объектов
	                	\item Удаление объектов
	                \end{itemize}
	    \end{itemize}
	    \pause
	    \item  Необходимо помнть что:
        \begin{itemize}
	        \item возможности "внешних" \ языков могут различаться
	        \item семантика языков может отличаться
        \end{itemize}
    \end{itemize}
\end{frame}

\begin{frame}[fragile]
	\transwipe[direction=90]
	\frametitle{Пример. Динамический SQL.}
	T-SQL:
	\begin{verbatim}
        IF @X = @Y
            SET @TABLE = ‘#table1’
        ELSE SET @TABLE = ‘table2’
        SET @S = ‘SELECT x FROM ‘ + @TABLE 
                    + ‘  WHERE ISNULL(n,0) > 1’
        EXECUTE (@S)
	\end{verbatim}
	PL-SQL:
	\begin{verbatim}
        IF lv_X = lv_Y
        THEN lv_TABLE := ‘tt_table1’;
        ELSE lv_TABLE := ‘new_table2’;
        END IF;
        lv_S := ‘SELECT new_x FROM ‘ || lv_TABLE 
                || ‘  WHERE NVL(n,0) > 1’;
        OPEN new_cursor FOR lv_S;
	\end{verbatim}
\end{frame}

\begin{frame}
	\transwipe[direction=90]
	\frametitle{Область применения}
	\begin{itemize}
		\item Реинжиниринг информационных систем
		    	\begin{itemize}
		            \item Обработка встроенного SQL
   		            \item Обработка динамического SQL в БД
	            \end{itemize}
		\item Разработка IDE с поддержкой рефакторинга встроенных языков.
	\end{itemize}
\end{frame}


\begin{frame}
	\transwipe[direction=90]
	\frametitle{Задача}
	\begin{itemize}
		\item Разаботать и реализовать библиотеку для трансляции динамически формируемых строк.
		\item Реализовать проверку результатов трансляции, минимизировав затраты на разработку.
	\end{itemize}
\end{frame}


\begin{frame}
	\transwipe[direction=90]
	\frametitle{Абстрактный анализ}
	\begin{itemize}
		\item Лексический.
		\item Синксический.
		\item Основан на обрабатке компактного представления множества возможных значений формируемой строки:
	    	\begin{itemize}
	            \item регулярное выражение
	            \item граф
	            \item data-flow уравнение
            \end{itemize}
		\item Используется для проверки синтаксической корректности динамически формируемых строк.
		\item Не подходит для трансформаций динамически формируемых строк, так как не сохраняет привязку к исходному коду.
	\end{itemize}
\end{frame}

\begin{frame}
	\transwipe[direction=90]
	\frametitle{Трансляция. Сохранение привязки.}
	\begin{center}
    \begin{dot2tex}[dot]
digraph G
{
1[label="h1"]
2[label="h2"]
3[label="h3"]
5[label="h5"]
4[label="h4"]
subgraph cluster_g1{
11[label="d1"]
12[label="d2"]
13[label="d3"]
14[label="d4"]
15[label="d5"]
11->12
11->13
12->14
12->15 
}
{rank=same;  1 11}
1->2
1->3
3->4
3->5
13->2[style="dotted"]
14->4[style="dotted"]
15->4[style="dotted"]
12->5[style="dotted"]
}    
\end{dot2tex}
	\end{center}

\end{frame}

\begin{frame}
	\transwipe[direction=90]
	\frametitle{Валидация результатов трансляции}
	\begin{itemize}
		\item Необходимо проверять корректность трансляции статически.
		\item Необходимо минимизировать затраты на реализацию.
    \end{itemize}
\end{frame}

\begin{frame}
	\transwipe[direction=90]
	\frametitle{Валидация результатов трансляции}
	\begin{itemize}
	    \item Основная идея: если трансляция прошла успешно, то абстрактный парсер целевого языка должен получить столько деревьев, сколько получил абстрактный парсер исходного языка до трансляции.
        \item Пактически всё можно переиспользовать.
        \begin{itemize}
	        \item Протягивание констант.
            \item Абстрактный анализ.
	    \end{itemize}
	\end{itemize}
\end{frame}


\begin{frame}%[t]
	\transwipe[direction=90]
	\frametitle{Результаты}
	\begin{itemize}
		\item Реализована библиотека абстрактного анализа, расширенного для нуждд трансляции.
        \item Проведена апробация на примере трансляции динамического SQL из T-SQL в PL-SQL.
        \item Реализована проверка коректности результатов трансляции.
        \item Продемонстрирована возможность переиспользования всех основных копонент для реализации проверки коректности результатов трансляции.
	\end{itemize}
\end{frame}

\begin{frame}
	\transwipe[direction=90]
	\frametitle{Используемые технологии}
	\begin{itemize}
		\item F\#
        \item FSharp PowerPack
        \item FsLex
        \item FsYacc
        \item YaccConstructor
	\end{itemize}
\end{frame}

\end{document}
