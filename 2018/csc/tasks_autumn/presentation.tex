\documentclass{beamer}
\usepackage{beamerthemesplit}
\usepackage{wrapfig}
\usetheme{SPbGU}
\usepackage{pdfpages}
\usepackage{amsmath}
\usepackage{cmap} 
\usepackage[T2A]{fontenc} 
\usepackage[utf8]{inputenc}
\usepackage[english,russian]{babel}
\usepackage{indentfirst}
\usepackage{amsmath}
\usepackage{tikz}
\usepackage{multirow}
\usepackage[noend]{algpseudocode}
\usepackage{algorithm}
\usepackage{algorithmicx}
\usetikzlibrary{shapes,arrows}
%usepackage{fancyvrb}
%\usepackage{minted}
%\usepackage{verbments}


\title[]{YaccConstructor}
\subtitle[YaccConstructor]{Задачи на осенний семестр 2018}
% То, что в квадратных скобках, отображается в левом нижнем углу. 
\institute[]{
Лаборатория языковых инструментов JetBrains \\
Санкт-Петербургский государственный университет \\
Математико-механический факультет }

% То, что в квадратных скобках, отображается в левом нижнем углу.
\author[Семён Григорьев]{Семён Григорьев}

\date{7 сентября 2018г.}

\definecolor{orange}{RGB}{179,36,31}

\begin{document}
{
\begin{frame}[fragile]
  \begin{tabular}{p{2.5cm} p{5.5cm} p{2cm}}
   \begin{center}
      \includegraphics[height=1.5cm]{pictures/JBLogo3.pdf}
    \end{center}
    &
    \begin{center}
      \includegraphics[height=1.5cm]{pictures/SPbGU_Logo.png}
    \end{center}
    &
    \begin{center}
      \includegraphics[height=1.5cm]{pictures/YC_logo.pdf}
    \end{center} 
  \end{tabular}
  \titlepage
\end{frame}
}

\begin{frame}[fragile]
  \transwipe[direction=90]
  \frametitle{YaccConstructor}
  \begin{itemize}
    \item Исследования в области теории формальных языков/алгоритмов синтаксического 
    анализа/графовых баз данных
    \item Исследовательская группа лаборатории языковых инстументов JetBrains Research
    \begin{itemize}
      \item \url{https://github.com/YaccConstructor}
    \end{itemize}
    \item Открытый исходный код
    \begin{itemize}
      \item \url{https://github.com/YaccConstructor}
    \end{itemize}
  \end{itemize}
\end{frame}

\begin{frame}[plain,c]
 \transwipe[direction=90]
 \begin{center}
  \Huge Реализация высокопроизводительных алгоритмов выполнения запросов к графовым БД
 \end{center}
\end{frame}

\begin{frame}[fragile]
\transwipe[direction=90]
\frametitle{Задачи}
  \begin{itemize}
    \item Высокопроизводительный алгоритм на CPU
    \begin{itemize}
       \item Изучить алгоритм !!!
       \item Расширить существуюую библиотеку быстрого перемножения булевых матриц недостающими 
       операциями (сложение, подсчёт количества ненулевых элементов и т.д.)
       \item Реализовать алгоритм !!! на основе расширенной библиотеки
       \item Провести экспериментальное исследование полученного решения
    \end{itemize}
    \item Высокопроизводительный алгоритм на GPGPU
    \begin{itemize}
       \item Изучить алгоритм !!!
       \item Найти готовую библиотеку для работы с булевыми матрицами для GPGPU или реализовать 
       свою со всеми необходимыми операциями (умножение, сложение, подсчёт количества ненулевых элементов и т.д.)
       \item Реализовать алгоритм !!! на основе полученной библиотеки
       \item Провести экспериментальное исследование полученного решения
    \end{itemize}

  \end{itemize}  
\end{frame}

\begin{frame}[fragile]
\transwipe[direction=90]
\frametitle{Требования к кандидатам}
  \begin{itemize}
    \item Хорошее знание C, OpenCL C, GPGPU
    \item Навыки низкоуровневого программирования, низкоуровневых оптимизаций
    \item Навыки написания высокопроизводительных алгоритмических решений
  \end{itemize}  
\end{frame}

\begin{frame}[fragile]
\transwipe[direction=90]
\frametitle{Информация}
  \begin{itemize}
    \item Два (2) человека (по одному на задачу)
    \item Подробное описание на GitHub
    \item Перспективы: хорошая курсовая/семестровая задача, может быть превращена в техническую бакалаврскую работу 
    \item !!!
  \end{itemize}  
\end{frame}


\begin{frame}[plain,c]
 \transwipe[direction=90]
 \begin{center}
  \Huge Исследование и реализация алгоритмов выполнения запросов к графовым БД \\ Практика
 \end{center}
\end{frame}

\begin{frame}[fragile]
\transwipe[direction=90]
\frametitle{Задачи}
  \begin{itemize}
    \item Реализация алгоритма поиска путей с ограничениями в в виде конъюнктивных грамматик
    \begin{itemize}
       \item Перенести существующий алгоритм поиска путей с КС ограничениями с алгоритма RNGLR 
       на BRNGLR
       \item Расширить результат предыдущего шага до конъюнктивных грамматик
       \item Провести экспериментальное исследование полученного решения
       \item Доказать корректность полученного решения
       \item Провести теоретическое исследование свойств полученного решения (оценка сложности, качество апроксимации, и т.д.)
    \end{itemize}
    \item Реализация алгоритма поиска путей с контекстно-свободными ограничениями на основе 
    производные (!!! derivatives)
    \begin{itemize}
       \item Реализовать алгоритм
       \item Реализовать распределённую версию алгоритма (Google Pregel)
       \item Провести экспериментальное исследование полученного решения
       \item Написать статью
    \end{itemize}

  \end{itemize}  
\end{frame}

\begin{frame}[fragile]
\transwipe[direction=90]
\frametitle{Требования к кандидатам}
  \begin{itemize}
    \item \textbf{Уверенное владение теорией формальных языков}
    \item \textbf{Хорошие знания алгоритмов синтаксического анализа}
    \item Навыки чтения и написания ``академических'' статей
    \item Знание F\# будет плюсом
  \end{itemize}  
\end{frame}

\begin{frame}[fragile]
\transwipe[direction=90]
\frametitle{Информация}
  \begin{itemize}
    \item Два (2) человека (по одному на задачу)
    \item Подробное описание на GitHub
    \item Перспективы: хорошая бакалаврская, может быть превращена в магистерскую
    \item !!!
  \end{itemize}  
\end{frame}
            

\begin{frame}[plain,c]
 \transwipe[direction=90]
 \begin{center}
  \Huge Разработка алгоритмов выполнения запросов к графовым БД \\ Теория
 \end{center}
\end{frame}

\begin{frame}[fragile]
\transwipe[direction=90]
\frametitle{Задачи}
  \begin{itemize}
    \item Поиск кратчайших путей с контекстно-свободными ограничениями
    \begin{itemize}
       \item Изучить алгоритм !!! поиска путей с контекстно-свободными ограничениями
       \item Расширить расширить его до алгоритмапоиска \textbf{кратчайших} путей с контекстно-свободными 
       ограничениями
       \item Доказать корректность полученного алгоритма
       \item Получить оценки сложности алгоритма
       \item Провести экспериментальное исследование
       \item Написать статью
    \end{itemize}
    \item Запросы с контекстно-свободными ограничениями к контекстно-свободно сжатым данным
    \begin{itemize}
       \item Изучить существующие алгоритмы поиска в конекстно-свободно сжатых данных
       \item Изучить применимость алгоритма абстрактного LR для выполнения запросов с КС 
       ограничеиями к КС-сжатым данным
       \item ...
    \end{itemize}

  \end{itemize}  
\end{frame}

\begin{frame}[fragile]
\transwipe[direction=90]
\frametitle{Требования к кандидатам}
  \begin{itemize}
    \item \textbf{Уверенное владение теорией формальных языков}
    \item \textbf{Глубокие знания алгоритмов синтаксического анализа}
    \item \textbf{Хорошие знания теории алгоритмов (анализ сложности, доказательство корректности и т.д.)}
    \item \textbf{Навыки чтения и написания ``академических'' статей}
    \item Знание F\# будет плюсом
  \end{itemize}  
\end{frame}

\begin{frame}[fragile]
\transwipe[direction=90]
\frametitle{Информация}
  \begin{itemize}
    \item Два (2) человека (по одному на задачу)
    \item Подробное описание на GitHub
    \item Перспективы: 
    \begin{itemize}
      \item Превая задача --- хорошая курсовая/семестровая
      \item Вторая задача --- бакалаврская/магистерская 
      \item !!!
    \end{itemize}  

    \item !!!
  \end{itemize}  
\end{frame}


\begin{frame}
\transwipe[direction=90]
\frametitle{Контакты}
\begin{itemize}
  \item Почта: \url{rsdpisuy@gmail.com}
  \item Исходный код YaccConstructor: \url{https://github.com/YaccConstructor}
\end{itemize}
\end{frame}
\end{document}
