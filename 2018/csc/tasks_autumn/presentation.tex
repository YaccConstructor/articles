\documentclass{beamer}
\usepackage{beamerthemesplit}
\usepackage{wrapfig}
\usetheme{SPbGU}
\usepackage{pdfpages}
\usepackage{amsmath}
\usepackage{cmap} 
\usepackage[T2A]{fontenc} 
\usepackage[utf8]{inputenc}
\usepackage[english,russian]{babel}
\usepackage{indentfirst}
\usepackage{amsmath}
\usepackage{tikz}
\usepackage{multirow}
\usepackage[noend]{algpseudocode}
\usepackage{algorithm}
\usepackage{algorithmicx}
\usetikzlibrary{shapes,arrows}
%usepackage{fancyvrb}
%\usepackage{minted}
%\usepackage{verbments}


\title[]{YaccConstructor}
\subtitle[YaccConstructor]{Задачи на осенний семестр 2018}
% То, что в квадратных скобках, отображается в левом нижнем углу. 
\institute[]{
Лаборатория языковых инструментов JetBrains \\
Санкт-Петербургский государственный университет \\
Математико-механический факультет }

% То, что в квадратных скобках, отображается в левом нижнем углу.
\author[Семён Григорьев]{Семён Григорьев}

\date{7 сентября 2018г.}

\definecolor{orange}{RGB}{179,36,31}

\begin{document}
{
\begin{frame}[fragile]
  \begin{tabular}{p{2.5cm} p{5.5cm} p{2cm}}
   \begin{center}
      \includegraphics[height=1.5cm]{pictures/JBLogo3.pdf}
    \end{center}
    &
    \begin{center}
      \includegraphics[height=1.5cm]{pictures/SPbGU_Logo.png}
    \end{center}
    &
    \begin{center}
      \includegraphics[height=1.5cm]{pictures/YC_logo.pdf}
    \end{center} 
  \end{tabular}
  \titlepage
\end{frame}
}

\begin{frame}[fragile]
  \transwipe[direction=90]
  \frametitle{YaccConstructor}
  \begin{itemize}
    \item Исследования в области теории формальных языков, алгоритмов синтаксического 
    анализа, графовых баз данных
    \item Исследовательская группа лаборатории языковых инстументов JetBrains Research
    \begin{itemize}
      \item \url{https://research.jetbrains.org/groups/plt_lab}
    \end{itemize}
    \item Открытый исходный код
    \begin{itemize}
      \item \url{https://github.com/YaccConstructor}
    \end{itemize}
  \end{itemize}
\end{frame}


\begin{frame}[fragile]
\transwipe[direction=90]
\frametitle{Инженерные задачи}
  \begin{itemize}
    \item QuickGraph $\Rightarrow$ проект с открытым исходным кодом
    \begin{itemize}
       \item \footnotesize{\url{https://github.com/YaccConstructor/QuickGraph/issues/181}}
       \item .NET, GitHub
       \item 1 человек 
       \item Курсовая
    \end{itemize}
    \item Высокопроизводительная реализация алгоритма выполнения запросов к графам на CPU
    \begin{itemize}
       \item \footnotesize{\url{https://github.com/YaccConstructor/YaccConstructor/issues/320}}
       \item C, высокопроизводительные алгоритмические решения
       \item 1 человек
       \item Курсовая
    \end{itemize}
    \item Высокопроизводительная реализация алгоритма выполнения запросов к графам на GPGPU
    \begin{itemize}
       \item \footnotesize{\url{https://github.com/YaccConstructor/YaccConstructor/issues/321}}
       \item OpenCL C, GPGPU, высокопроизводительные алгоритмические решения
       \item 1 человек
       \item Курсовая (бакалаврский диплом)
    \end{itemize}
  \end{itemize}  
\end{frame}

\begin{frame}[fragile]
\transwipe[direction=90]
\frametitle{Прикладные исследования}
  \begin{itemize}
    \item Поиск путей с ограничениями в форме конъюнктивных граммтик
    \begin{itemize}
       \item \footnotesize{\url{https://github.com/YaccConstructor/YaccConstructor/issues/311}}
       \item Теория формальных языков, алгоритмы синтаксического анализа, F\#
       \item 1 человек 
       \item Бакалаврский диплом
    \end{itemize}
    \item Производные (Brzozowski’s derivatives) для поиска путей с контекстно-свободными 
    ограничениям
    \begin{itemize}
       \item \footnotesize{\url{https://github.com/YaccConstructor/YaccConstructor/issues/306}}
       \item Теория формальных языков, алгоритмы синтаксического анализа
       \item 1 человек
       \item Бакалаврский диплом
    \end{itemize}
    \item Вывод типов для Ruby-библиотек на основе контекстно-свободной достижимости
    \begin{itemize}
       \item \footnotesize{\url{https://github.com/YaccConstructor/YaccConstructor/issues/322}}
       \item Статический анализ кода, вывод типов, теория формальных языков, Java, Ruby, F\#
       \item 2 человека
       \item Бакалаврский диплом
    \end{itemize}
  \end{itemize}  
\end{frame}

\begin{frame}[fragile]
\transwipe[direction=90]
\frametitle{Теоритические исследования}
  \begin{itemize}
    \item Поиск \textbf{кратчайших} путей с контекстно-свободными ограничениями
    \begin{itemize}
       \item \footnotesize{\url{https://github.com/YaccConstructor/YaccConstructor/issues/293}}
       \item Теория графов, теория алгоритмов, теория формальных языков 
       \item 1 человек 
       \item Курсовая
    \end{itemize}
    \item Контекстно-свободные запросы к контекстно-свободно сжатым данным
    \begin{itemize}
       \item \footnotesize{\url{https://github.com/YaccConstructor/YaccConstructor/issues/287}}
       \item Теория формальных языков, теория алгоритмов 
       \item 1 человек
       \item Бакалаврский диплом
    \end{itemize}
  \end{itemize}  
\end{frame}

\begin{frame}[fragile]
\transwipe[direction=90]
\frametitle{Суперкомпиляция\footnote{На самом деле, специализация}}
  \begin{itemize}
    \item \footnotesize{\url{https://github.com/YaccConstructor/Brahma.FSharp/issues/99}}
    \item Разработать спецализатор (подмножетсва) OpenCL C
    \begin{itemize}
       \item Статический анализ кода, принципы построения трансляторов
       \item От одного человека
       \item Бакалаврский диплом
    \end{itemize}
    \item Интегрировать специализатор в Brahma.FSharp
    \begin{itemize}
       \item Хорошее знание F\#: code quotations, особенности системы вывода типов
       \item 1 человек
       \item Бакалаврский диплом
    \end{itemize}
    \item Подготовить окружение для экспериментальных исследований
    \begin{itemize}
       \item Знание F\# 
       \item От одного человека
       \item Курсовая
    \end{itemize}
  \end{itemize}  
\end{frame}


\begin{frame}
\transwipe[direction=90]
\frametitle{Контакты}
\begin{itemize}
  \item Почта (она же google hangout для оперативной связи): \url{rsdpisuy@gmail.com}
  \item GitHub: \url{https://github.com/YaccConstructor}
  \item Профиль на JetBrains Research: \url{https://research.jetbrains.org/researchers/gsv}
\end{itemize}
\end{frame}
\end{document}
