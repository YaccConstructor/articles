%%%%%%%%%%%%%%%%%%%%%%%%%%%%%%%%%%%%%%%%%
% NIWeek 2014 Poster by T. Reveyrand
% www.microwave.fr
% http://www.microwave.fr/LaTeX.html
% ---------------------------------------
% 
% Original template created by:
% Brian Amberg (baposter@brian-amberg.de)
%
% This template has been downloaded from:
% http://www.LaTeXTemplates.com
%
% License:
% CC BY-NC-SA 3.0 (http://creativecommons.org/licenses/by-nc-sa/3.0/)
%
%%%%%%%%%%%%%%%%%%%%%%%%%%%%%%%%%%%%%%%%%

%----------------------------------------------------------------------------------------
%   PACKAGES AND OTHER DOCUMENT CONFIGURATIONS
%----------------------------------------------------------------------------------------

\documentclass[b1paper,portrait]{baposter}

\usepackage[font=small,labelfont=bf]{caption} % Required for specifying captions to tables and figures
\usepackage{booktabs} % Horizontal rules in tables
\usepackage{relsize} % Used for making text smaller in some places
\usepackage{amsmath,amsfonts,amssymb,amsthm, mathtools} % Math packages
\usepackage{eqparbox}
\usepackage{textcomp}
\usepackage{caption}
\usepackage{subcaption}
\usepackage{graphicx}
\usepackage{listings}
\usepackage{enumitem}
\usepackage{pgf}
\usepackage{tikz}
\usetikzlibrary{positioning}
\usepackage[utf8]{inputenc}
\usetikzlibrary{arrows,automata}
\usetikzlibrary{positioning}
\usepackage{hyperref}


\graphicspath{{figures/}} % Directory in which figures are stored

 \definecolor{bordercol}{RGB}{40,40,40} % Border color of content boxes
 \definecolor{headercol1}{RGB}{210,235,250} % Background color for the header in the content boxes (left side)
 \definecolor{headercol2}{RGB}{210,235,250} % Background color for the header in the content boxes (right side)
 \definecolor{headerfontcol}{RGB}{0,0,0} % Text color for the header text in the content boxes
 \definecolor{boxcolor}{RGB}{240,255,255} % Background color for the content in the content boxes


\begin{document}

\background{ % Set the background to an image (background.pdf)
\begin{tikzpicture}[remember picture,overlay]
\draw (current page.north west)+(-2em,2em) node[anchor=north west]
{\includegraphics[height=1.1\textheight]{background}};
\end{tikzpicture}
}

\begin{poster}{
grid=false,
columns=4, % because reasons
borderColor=bordercol, % Border color of content boxes
headerColorOne=headercol1, % Background color for the header in the content boxes (left side)
headerColorTwo=headercol2, % Background color for the header in the content boxes (right side)
headerFontColor=headerfontcol, % Text color for the header text in the content boxes
boxColorOne=boxcolor, % Background color for the content in the content boxes
headershape=rectangle, % Specify the rounded corner in the content box headers
headerfont=\Large\sf\bf, % Font modifiers for the text in the content box headers
textborder=none,
background=none,
headerborder=none, % Change to closed for a line under the content box headers
boxshade=plain
}
%
%----------------------------------------------------------------------------------------
%   TITLE AND AUTHOR NAME
%----------------------------------------------------------------------------------------
%
{\includegraphics[width=2.5cm]{jetbrains.png}}
{\\  \vspace{20pt}  \huge \bf Context-Free Path Querying via Matrix Equations} % Poster title
{\vspace{0.6em} \smaller \textbf{Yuliya Susanina} \\  % Author names
\smaller \it {JetBrains Research, Saint Petersburg University, Russia } \\ % Author email addresses
\smaller  {jsusanina@gmail.com}}
{\includegraphics[width=2cm]{SPbGU_Logo.png}} % University/lab logo

%----------------------------------------------------------------------------------------
%   INTRODUCTION
%----------------------------------------------------------------------------------------
\headerbox{Context-Free Path Querying}{name=introduction,column=0,row=0.02, span=2}{
CFPQ---a way to specify path constraints in terms of context-free grammars---is gaining popularity in many areas, such as bioinformatics, graph databases or static code analysis. It is crucial to develop highly efficient algorithms for CFPQ since the size of the input data is typically large. We propose a new way to reduce CFPQ to a problem with available high-performance solutions: CFPQ can be reduced to solving the systems of equations over real numbers $\mathbb{R}$. So, we can use numerical linear algebra and computational mathematics to improve the performance of query evaluation.
}

\headerbox{Equation-Based Approach}{name=approach,column=0,row=0,span=2, below=introduction}{ 
We reduce CFPQ to solving the equations similarly to the way Sato reduces Datalog program evaluation to linear algebra~\cite{sato2017linear}.

\begin{center}
\textbf{Reduction from solving Boolean matrix equations}
\end{center}
$$
A = \begin{psmallmatrix}
	0 & 1 & 0 & 0       \\
	0 & 0 & 1 & 0       \\
	1 & 0 & 0 & 0       \\
	0 & 0 & 0 & 0      
\end{psmallmatrix} \quad
B = \begin{psmallmatrix}
	0 & 0 & 0 & 1       \\
	0 & 0 & 0 & 0       \\
	0 & 0 & 0 & 0       \\
	1 & 0 & 0 & 0      
\end{psmallmatrix}
$$

$$
T_S^* = \lim_{k \to \infty} T_S^k= \lim_{k \to \infty} AT_S^{k-1}B + AB = 
\begin{psmallmatrix}
	1 & 0 & 0 & 1       \\
	1 & 0 & 0 & 1       \\
	1 & 0 & 0 & 1       \\
	0 & 0 & 0 & 0      
\end{psmallmatrix}
$$

$$T_S^* \text{--- the least solution of } T_S = AT_SB + AB$$
\begin{center}
\textbf{Reduction from solving matrix equations over $\mathbb{R}$}
\end{center}
$$
\begin{array}{cl}
0 < \epsilon \leq \frac{1}{1 + \| T_B^T \otimes T_A\|} 
\Rightarrow &\mathcal{T}_S = \epsilon(A\mathcal{T}_SB + AB) \\
            &\text{has a unique solution } \mathcal{T}_S^*
\end{array}
$$

$$(\mathcal{T}_S^*)_{ij} > 0 \Leftrightarrow (T_S^*)_{ij} = 1$$


In real-world cases, we deal with the systems of matrix equations because grammars contain more than one nonterminal. There are different methods to solve these systems:
\begin{itemize}
    \item Naive iterative process
    \item Methods for linear systems of form $Ax =b$
    \item Efficient algorithms for special cases (for example, Sylvester equations)
    \item Approximate methods (Newton's method and its modifications)
\end{itemize}
}



\headerbox{Example}{name=example,span=2,column=2,row=0.02}{
    \begin{center}
	\textbf{Labeled directed graph \boldmath$D$:}	
	\includegraphics[width=6cm]{example_graph_transparent.png}
	
	\textbf{CF-grammar \boldmath$G$ for the language \boldmath$\mathcal{L}(G_S) = \{a^n b^n\}$:}	
    $\begin{array}{cc}
    & S \ \rightarrow \ A \ S \ B \ | \ A \ B; \  A \ \rightarrow \ a ; \ B \ \rightarrow \ b \
    \end{array}$

	\textbf{The result of CFPQ is Boolean matrix \boldmath$T_S$:} \\
	$\begin{array}{cl}
    (T_S)_{ij} = 1 \Leftrightarrow \exists &\text{path from } i \text{ to } j \text{ in } D \\
                                           &\text{and its labeling is in } \mathcal{L}(G_S)
	\end{array}$
    \end{center}
}


\headerbox{Evaluation}{name=eval,column=2,row=0, span=2, below=example}{
    \begin{center}
    \textbf{Query:} 
    $\begin{array}{rccl}
       S \ \rightarrow \ A \ S \ B \ | \ B; \  A \ \rightarrow \ a ; \ B \ \rightarrow \ b \
    \end{array}$
    
    \textbf{Graphs:} classical ontologies (RDFs)
    
    \textbf{CPU-based implementations using \textit{scipy} library:} \\
    \textrm{[sSLV]} solve equation as sparse linear systems \\
    \textrm{[dNWT]} find roots of $F(T_S) = \mathcal{T}_S - \epsilon(A\mathcal{T}_SB + B) = 0$
    \end{center}
    
    \textbf{Comparative analysis with matrix-based approach~\cite{azimov2018context}} \\
    \resizebox{0.93\textwidth}{!}{\begin{minipage}{\textwidth}
    \begin{tabular}{ |c | c || c | c || c | c | c |}
    \hline
    Ontology    & |V| & dNWT & sSLV & dGPU & sCPU &  sGPU  \\
    \hline
    \hline
    bio-meas    & 341 &  284 & 35   & 276  & 91  & 24\\
    people-pets & 337 &  73  & 49   & 144  & 38  & 6\\
    funding     & 778 &  502 & 184  & 1246 & 344 & 27\\
    wine        & 733 &  791 & 171  & 722  & 179 & 6\\
    pizza       & 671 &  334 & 161  & 943  & 256 & 23\\
    \hline
    \end{tabular}
    \end{minipage} }
}

\headerbox{Results}{name=results,column=2,below=eval, span=2}{
  \begin{itemize} 
    \item Equation-based approach for CFPQ was proposed
    \item The feasibility of using both accurate and approximate methods of computational mathematics was assessed
    \item The evaluation on a set of conventional benchmarks showed that it is comparable with the matrix-based approach and applicable for real-world data processing
  \end{itemize}
}

\headerbox{Future Research}{name=future,column=2,span=2,below=results}{
    \begin{itemize}
        \item Employ high-performance solvers which utilize sparse matrices, GPGPU and distributed computations 
        \item Determine the subclasses of polynomial equations the solution of which can be reduced to CFPQ and try to construct a bidirectional reduction between them, thereby finding efficient solutions for both problems
    \end{itemize}
}


\headerbox{Acknowledgments}{name=acks,column=0,below=approach, span=2}{
The research was supported by the Russian Science Foundation grant 18-11-00100 and a grant from JetBrains Research.
}

%----------------------------------------------------------------------------------------
%   REFERENCES
%----------------------------------------------------------------------------------------


\headerbox{References}{name=ref, span=2,column=2,row=0,below=future}{
	
	\smaller % Reduce the font size in this block
	\renewcommand{\section}[2]{\vskip 0.05em} % Get rid of the default "References" section title
	%\nocite{*} % Insert publications even if they are not cited in the poster
	
	\bibliographystyle{unsrt}
	\bibliographystyle{IEEEtran}
	\bibliography{main} % Use biblio.bib as the bibliography file
}

{\hspace{5pt}\includegraphics[width=0.47\textwidth]{2020Sigmod-LOGO.png}}


\end{poster}

\end{document}