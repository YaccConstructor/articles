\documentclass{beamer}
\usepackage{beamerthemesplit}
\usepackage{wrapfig}
\usetheme{SPbGU}
\usepackage{pdfpages}
\usepackage{amsmath}
\usepackage{cmap} 
\usepackage[T2A]{fontenc} 
\usepackage[utf8]{inputenc}
\usepackage[english,russian]{babel}
\usepackage{indentfirst}
\usepackage{amsmath}
\usepackage{tikz}
\usepackage{multirow}
\usepackage[noend]{algpseudocode}
\usepackage{algorithm}
\usepackage{algorithmicx}
\usetikzlibrary{shapes,arrows}
%usepackage{fancyvrb}
%\usepackage{minted}
%\usepackage{verbments}


\title[]{YaccConstructor}
\subtitle[YaccConstructor]{Задачи на 2020/2021 учебный год}
% То, что в квадратных скобках, отображается в левом нижнем углу. 
\institute[]{
Лаборатория языковых инструментов JetBrains \\
Санкт-Петербургский государственный университет \\
Математико-механический факультет }

% То, что в квадратных скобках, отображается в левом нижнем углу.
\author[Семён Григорьев]{Семён Григорьев}

\date{16 сентября 2020г.}

\definecolor{orange}{RGB}{179,36,31}

\begin{document}
{
\begin{frame}[fragile]
  \begin{tabular}{p{2.5cm} p{5.5cm} p{2cm}}
   \begin{center}
      \includegraphics[height=1.5cm]{pictures/JBLogo3.pdf}
    \end{center}
    &
    \begin{center}
      \includegraphics[height=1.5cm]{pictures/SPbGU_Logo.png}
    \end{center}
    &
    \begin{center}
      \includegraphics[height=1.5cm]{pictures/YC_logo.pdf}
    \end{center} 
  \end{tabular}
  \titlepage
\end{frame}
}

\begin{frame}[fragile]
  \frametitle{YaccConstructor}
  \begin{itemize}
    \item Исследования в области теории формальных языков, алгоритмов синтаксического 
    анализа, графовых баз данных
    \item Исследовательская группа лаборатории языковых инстументов JetBrains Research
    \begin{itemize}
      \item \url{https://research.jetbrains.org/groups/plt_lab}
      \item \url{https://vk.com/pltlab}
    \end{itemize}
    \item Открытый исходный код
    \begin{itemize}
      \item \url{https://github.com/YaccConstructor}
      \item \url{https://github.com/JetBrains-Research}
    \end{itemize}
  \end{itemize}
\end{frame}

\begin{frame}[fragile]
\frametitle{Задачи}
  Общее требование: хорошие знания в теории формальных языков и алгоритмах синтаксического анализа
  \begin{itemize}
    \item Реализация алгоритма контекстно-свободной достижимости на OpenCL
    \item Реализация операций линейной алгебры на графическом процесcоре с использованием F\# 
    \item Расширение набора данных для экспериментального исследования запросов с контекстно-свободными ограничениями
    \item Вычислительная теория групп
  \end{itemize}
\end{frame}


\begin{frame}
\frametitle{Контакты}
\begin{itemize}
  \item Почта (она же google hangout или Телеграм для оперативной связи): \url{rsdpisuy@gmail.com}
  \item GitHub: \url{https://github.com/YaccConstructor}
  \item Профиль на JetBrains Research: \url{https://research.jetbrains.org/researchers/gsv}
\end{itemize}
\end{frame}
\end{document}
