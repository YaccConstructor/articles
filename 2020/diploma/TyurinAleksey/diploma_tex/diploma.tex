% Тут используется класс, установленный на сервере Papeeria. На случай, если
% текст понадобится редактировать где-то в другом месте, рядом лежит файл matmex-diploma-custom.cls
% который в момент своего создания был идентичен классу, установленному на сервере.
% Для того, чтобы им воспользоваться, замените matmex-diploma на matmex-diploma-custom
% Если вы работаете исключительно в Papeeria то мы настоятельно рекомендуем пользоваться
% классом matmex-diploma, поскольку он будет автоматически обновляться по мере внесения корректив
%

% По умолчанию используется шрифт 14 размера. Если нужен 12-й шрифт, уберите опцию [14pt]
%\documentclass[14pt]{matmex-diploma}
\documentclass[14pt]{matmex-diploma-custom}
\usepackage{graphicx}

%code highlight
% \usepackage{listings}
\usepackage{listings, listings-rust}
\usepackage{lipsum}
\usepackage{xcolor}

\usepackage{tikz}
\usetikzlibrary{decorations.pathreplacing,calc,shapes,positioning}

\newcommand\tbox[1]{\tikz[overlay]\node[inner sep=2pt, draw=red, ultra thick, anchor=text, rectangle] {#1};\phantom{#1}}
 
\definecolor{codegreen}{rgb}{0,0.6,0}
\definecolor{codegray}{rgb}{0.5,0.5,0.5}
\definecolor{codepurple}{rgb}{0.58,0,0.82}
\definecolor{backcolour}{rgb}{1,1,1}
 
\lstdefinestyle{mystyle}{
    backgroundcolor=\color{backcolour},   
    commentstyle=\color{codegreen},
    keywordstyle=\color{magenta},
    numberstyle=\color{codegray},
    stringstyle=\color{codepurple},
    basicstyle=\ttfamily\small,
    breakatwhitespace=false,         
    breaklines=true,                 
    captionpos=b,                    
    keepspaces=true,                 
    numbers=left,                    
    numbersep=5pt,                  
    showspaces=false,                
    showstringspaces=false,
    showtabs=false,                  
    tabsize=2
    % xleftmargin=.2\textwidth,
    % xrightmargin=.2\textwidth
}
 
\lstset{style=mystyle}

\lstnewenvironment{code}[1][]%
{
   \noindent
   \minipage{\linewidth} 
   \vspace{0.5\baselineskip}
   \lstset{basicstyle=\ttfamily\footnotesize,frame=single,#1}}
{\endminipage}

\begin{document}
% Год, город, название университета и факультета предопределены,
% но можно и поменять.
% Если англоязычная титульная страница не нужна, то ее можно просто удалить.
\filltitle{ru}{
    chair              = {Программная инженерия\\ \vspace{5mm}Системное программирование},
    title              = {Экспериментальное исследование специализатора GPGPU-программ AnyDSL},
    % Здесь указывается тип работы. Возможные значения:
    %   coursework - Курсовая работа
    %   diploma - Диплом специалиста
    %   master - Диплом магистра
    %   bachelor - Диплом бакалавра
    type               = {bachelor},
    position           = {студента},
    group              = 471,
    author             = {Тюрин Алексей Валерьевич},
    supervisorPosition = {к.\,ф.-м.\,н., доцент},
    supervisor         = {С.\,В.~Григорьев},
    consultantPosition = {программист ООО <<Интеллиджей Лабс>>\\к.\,ф.-м.\,н.,},
    consultant         = {Д.\,А.~Березун},
    reviewerPosition   = {стажёр-исследователь\, <<ИСП\,им.\,В.\,П.~Иванникова РАН>>\\},
    reviewer           = {Е.\,Ю.\,Шарыгин},
    chairHeadPosition  = {TBD},
    chairHead          = {TBD},
%   university         = {Санкт-Петербургский Государственный Университет},
%   faculty            = {Математико-механический факультет},
%   city               = {Санкт-Петербург},
%   year               = {2013}
}
\filltitle{en}{
    chair              = {Software Engineering},
    title              = {Practical study of AnyDSL GPGPU program partial evaluator},
    author             = {Aleksey Tyurin},
    supervisorPosition = {Associate professor, Ph.\,D.},
    supervisor         = {Semyon Grigorev},
    consultantPosition = {IntelliJ Labs Co. Ltd developer\\Ph.\,D.},
    consultant         = {Daniil Berezun},
    reviewerPosition   = {Research assistant at Ivannikov Institute for System Programming of the RAS\\},
    reviewer           = {Eugene Sharygin},
    chairHeadPosition  = {TBD},
    chairHead          = {TBD},
}

\newcommand\todo[1]{{\color{violet}#1}}
\newcommand\db[1]{{\color{red}#1}}
\newcommand\question[1]{{\color{cyan}#1}}

\maketitle
\tableofcontents
% У введения нет номера главы
\section{Introduction}

Scalable high-performance graph analysis is an actual challenge.
There is a big number of ways to attack this challenge~\cite{Coimbra2021} and the first promising idea is to utilize general-purpose graphic processing units (GPGPU).
Such existing solutions, as CuSha~\cite{10.1145/2600212.2600227} and Gunrock~\cite{7967137} show that utilization of GPUs can improve the performance of graph analysis, moreover it is shown that solutions may be scaled to multi-GPU systems.
But low flexibility and high complexity of API are problems of these solutions.

The second promising thing which provides a user-friendly API for high-performance graph analysis algorithms creation is a GraphBLAS API~\cite{7761646} which provides linear algebra based building blocks to create graph analysis algorithms.
The idea of GraphBLAS is based on a well-known fact that linear algebra operations can be efficiently implemented on parallel hardware.
Along with that, a graph can be natively represented using matrices: adjacency matrix, incidence matrix, etc.
While reference CPU-based implementation of GraphBLAS, SuiteSparse:GraphBLAS~\cite{10.1145/3322125}, demonstrates good performance in real-world tasks, GPU-based implementation is challenging.

One of the challenges in this way is that real data are often sparse, thus underlying matrices and vectors are also sparse, and, as a result, classical dense data structures and respective algorithms are inefficient. 
So, it is necessary to use advanced data structures and procedures to implement sparse linear algebra, but the efficient implementation of them on GPU is hard due to the irregularity of workload and data access patterns.
Though such well-known libraries as cuSPARSE show that sparse linear algebra operations can be efficiently implemented for GPGPU, it is not so trivial to implement GraphBLAS on GPGPU. 
First of all, it requires \textit{generic} sparse linear algebra, thus it is impossible just to reuse existing libraries which are almost all specified for operations over floats.
The second problem is specific optimizations, such as masking fusion, which can not be natively implemented on top of existing kernels.
Nevertheless, there is a number of implementations of GraphBLAS on GPGPU, such as GraphBLAST~\cite{yang2019graphblast}, GBTL~\cite{7529957}, which show that GPGPUs utilization can improve the performance of GraphBLAS-based graph analysis solutions.
But these solutions are not portable because they are based on Nvidia Cuda stack.
Moreover, the scalability problem is not solved: all these solutions support only single-GPU, not multi-GPU computations.

To provide portable GPU implementation of GraphBLAS API we developed a \textit{SPLA} library\footnote{Source code available at: \url{https://github.com/JetBrains-Research/spla}}.
This library utilizes OpenCL for GPGPU computing to be portable across devices of different vendors.
Moreover, it is initially designed to utilize multiple GPGPUs to be scalable.
To sum up, the contribution of this work is the following.
\begin{itemize}
    \item Design of portable GPU GraphBLAS implementation proposed. The design involves the utilization of multiple GPUS. Additionally, the proposed design is aimed to simplify library tuning and wrappers for different high-level platforms and languages creation. 
    \item Subset of GraphBLAS API, including such operations as masking, matrix-matrix multiplication, matrix-matrix e-wise addition, is implemented. The current implementation is limited by COO and CSR matrix representation format and uses basic algorithms for some operations, but work in progress and more data formats will be supported and advanced algorithms will be implemented in the future.
    \item Preliminary evaluation on such algorithms as breadth-first search (BFS) and triangles counting (TC), and real-world graphs shows portability across different vendors and promising performance: for some problems Spla is comparable with GraphBLAST. Surprisingly, for some problems, the proposed solution on embedded Intel graphic card shows better performance than SuiteSparse:GraphBLAS on the respective CPU. At the same time, the evaluation shows that further optimization is required.
\end{itemize} 
\section*{Problem statement}\label{ps}
The aim of the work is the practical evaluation of whether any performance enhancement could be brought %by partial evaluation technique
by partially evaluating memory accesses through the utilization of AnyDSL framework, compared to CUDA implementations, considering GPU microarchitecture details that affect the result. In order to achieve the aim, the following objectives have been set.
\begin{itemize}
    \item Implement experimental scenarios in both AnyDSL and CUDA.
    \item Collect relevant datasets for the evaluation to be more practical.
    \item Perform the evaluation and analyze the results.
\end{itemize}

More specifically, the work performs the evaluation on string matching and convolutional filtering scenarios, providing some relevant CUDA assembly examples to ground the effects being observed.

% Firstly, the hypothesis of whether GPU-based string pattern matching program performance speed up could be achieved via partially evaluating memory accesses, should have been verified. A modern GPU has different types of memory, varying by access latency. In order to achieve maximal performance every type of memory should be utilized carefully, satisfying alignment and access patterns requirements. Moreover, several cache levels are extensively used to mitigate the latency, and, to some extent, caches could keep up the performance of an application, even if a proper access pattern is hard to achieve. There is a partial evaluator, being developed as part of AnyDSL framework~\cite{LeiBa}, with the support for generation of specialized Nvidia CUDA C code. It has been utilized to verify the hypothesis.   

% Next, the bottlenecks that arise during string pattern matching program specialization should be identified. Particular program transformations could potentially hurt further parallelization or simply not achieve expected effects. Thus, string pattern matching algorithms specialization should be examined using available partial evaluators.

% Then the partial evaluator should be implemented considering the identified bottlenecks either from scratch or as an extension for available ones.

% Finally, the obtained partial evaluator efficiency should be evaluated through performance comparison between specialized programs and manually fine-tuned ones.   
\section{Related work \& background}
This section includes basic notation and definitions in graph theory and formal language theory which are used in this work. Also, the further description of both the theoretical part of the GLL-based CFPQ algorithm and its implementation are provided.

\subsection{Basic Definitions of Formal Languages}

In this work, the context-free grammars are used as path constraints, thus context-free languages and grammars are defined in this subsection.

\begin{rudefinition}A \emph{context-free grammar} is a tuple $G= \langle N, \Sigma, P, S \rangle$, where
\begin{itemize}
    \item $N$ is a finite set of nonterminals
    \item $\Sigma$ is a finite set of terminals, $N \cap \Sigma = \varnothing$
    \item $P$ is a finite set of productions of the form $A \to \alpha$, where $A \in N,\ \alpha \in (N \cup \Sigma)^*$
    \item $S$ $\in$ $N$.
\end{itemize} \qed
\end{rudefinition}

We use the conventional notation $A \Rightarrow^* w$ to denote, that a
word $w \in \Sigma^*$ can be derived from a non-terminal $A$ using some sequence of production rules from $P$.

\begin{rudefinition} A \emph{context-free language} is a language generated by a con-text-free grammar $G$:
\begin{align*}
     L(G) = \{w \in \Sigma^* \mid S \Rightarrow^* w \}
\end{align*}
\end{rudefinition}

% \begin{rudefinition} A \emph{context-free language with a specified starting non-terminal $S$} is a set of strings that can be generated from $S$ by a context-free grammar $G$:
% \begin{align*}
%      L(G_S) = \{w \in \Sigma^* \mid S \Rightarrow^* w \}
% \end{align*}
% \end{rudefinition}

\subsection{Basic Definitions of Graph Theory}
In a simplified way, the Neo4j graph database uses a labeled directed graph as a data model. It can be defined as follows.

\begin{rudefinition} \emph{Labeled directed graph} is a tuple $D = \langle V, E, T \rangle$, where
\begin{itemize}
    \item $V$ is a finite set of vertices. For simplicity, we assume that the vertices are natural numbers from $0$ to $|V|-1$.
    \item $T$ is a set of labels on edges.
    \item $E \subseteq V \times T \times V$ is a set of edges.
\end{itemize} \qed
\end{rudefinition}

\begin{rudefinition}
Path $\pi$ in the graph $D = \langle V, E, T \rangle$ is a finite sequence of edges $(e_0, e_1, ..., e_{n-1})$, where $\forall~ j,~ 0 \leq j \leq n - 1: e_j=(v_j,t_j,v_{j+1}) \in E$.

We denote the set of all paths in the graph $D$ as $\pi(D)$. \qed
\end{rudefinition}

\subsection{Context-free Path Querying}
Now, we can define context-free path querying problems. Let be:
\begin{itemize}
      \item a context-free grammar $G=\langle N, \Sigma, P, S \rangle$;
      \item a directed graph $D=\langle V, E, T \rangle$, where $V$ is the set of vertices of the graph, $ E \subseteq V \times T \times V $ is the set of edges, $ T \subseteq \Sigma $ is the set of labels on edges, where each label is a terminal symbol of the grammar $G$;
\item a set of start vertices $V_S \subseteq V$ and final vertices \mbox {$V_F \subseteq V$.}
\end{itemize}

Consider a path in the graph $D$: $$\pi = (e_0, e_1, \cdots, e_{n - 1}), $$ where $ e_k = (v_{k}, t_k , v_{k+1}), ~ \forall~k,~ 0 \leq k \leq n - 1 ~e_k \in E$.
To path in the graph the word $ l(\pi) = t_0t_1 \cdots t_{n_1} $ is associated --- the concatenation of the labels on the edges of this path.

In the introduced notation, the following problems can be formulated.

\begin{itemize}
     \item \textbf{The problem of a path querying in a graph with context-free constraints} consists in finding all paths in the graph such that $l(\pi) \in L (G)$ and $v_0 \in V_S, ~v_n \in V_F$.
    
     \item \textbf{The problem of reachability in a graph with context-free constraints} consists in finding a set of pairs of vertices for which there is a path with a beginning and an end at these vertices, such that the word composed of labels of the edges of the path  belongs to the given language: $ \{(v_i, v_j) ~ | ~ \exists ~ l (\pi) \in L (G) $ and $ v_0 \in V_S, ~ v_n \in V_F \} $.
\end{itemize}

It should be noted that it is often necessary to identify complex dependencies in a graph data model. So, according to the context and application area, both variants of the above problems are of practical importance. 

For each problem there are two variants of set of starting vertices: the set may consist of all vertices of a graph or may consist only a particular vertices of interest. The first variant is called all-pairs context-free path querying problem and the second is called a multiple-source (and a single-source as a partial case) context-free path querying problem.

\subsection{Generalized LL Parsing Algorithm}
One of the common parsing techniques is the LL(k) algorithm~\cite{10.5555/1076440}, that performs top-down analysis with a lookahead. It means that the decision about which production of the grammar should be applied is based on looking at the $ k $ following character from the current one. To choose the right production rule at this step algorithm supports a parsing table, where the information for parsing the current non-terminal is stored. However, it can be applied only to a subset of the context-free grammars class and does not support ambiguous context-free grammars or grammars with left recursion in derivation.

Top-down analysis algorithms are relatively easier to implement and debug, because it fully matches the structure of the grammar. For this reason, to extend the parsing power of above-mentioned technique there was proposed~\cite{SCOTT2010177} the generalized LL (GLL) algorithm. Also GLL can handle ambiguous grammars.
In case of LL(k) algorithm may arise the situation when it is impossible to determine which production should be applied in the current state of the parsing process ~\cite{10.1145/800105.803402}. To solve this issue the GLL algorithm maintains a queue of descriptors. Each descriptor is a structure that describes the current state of the analyzer. Thus, using a queue of descriptors allows one to consider all possible transitions during the operation of the parser.

The parsing table for the generalized GLL algorithm can store multiple alternatives for parsing the current non-terminal. In this case, descriptor duplication can occur. For efficient storage and reuse of many different descriptors, GLL uses a specific structure --- Graph Structured Stack (GSS)~\cite{10.5555/1623611.1623625}.

To represent the result, GLL provides the Shared Packed Parse Forest (SPPF) structure~\cite{SCOTT20131828}, which contains all derivation trees for all paths satisfying the specified language.

\subsection{GLL-based CFPQ Algorithm}
As it was showed, classical GLL parsing technique can be used to solve context-free language constrained path problem. It means that such technique can be used to proceed graph input. Previously, the algorithm was generalized from linear input to graph processing, as was described in \cite{10.1145/3166094.3166104}.

To do this, the following modifications were proposed.
\begin{itemize}
\item A query has became a triple: a set of initial vertices, a set of final vertices, and a grammar.
\item An initial set of descriptors must include all the start vertices of the graph.
\item At the step of transition to the next character, it is necessary to support all possible transition options that correspond to all outgoing edges of the vertex.
\item If parsing is completed, it is necessary to check whether the final vertex in the parsing belongs to the set of final vertices of the graph.
\end{itemize}

The described principles of the generalized GLL algorithm are important for understanding the features of its implementation, which will be described below.

The implementation of the algorithm is based on the Iguana project which is written in Java. This library provides the modified GLL algorithm. The advantage of  Iguana project is that it uses a more efficient GSS for GLL parsing. In addition, it does not affect the worst-case cubic run time and space complexities of GLL parsing.
 
Under this work, it is important to pay attention to the following changes that were made to the workflow of the GLL algorithm to unable graph processing.

\begin{itemize}
    \item In order to support graph processing, the abstraction of an input data was changed. The new implementation of the $Input$ interface has been added. Now it is represented as a graph adjacency list, a set of start and final vertices of the resulting paths.
    \item There can be multiple start vertices for a graph input, unlike a linear input. So, also the initialization of the descriptor queue was modified. In case of processing a descriptor with slot $(N \rightarrow \alpha.x\beta)$, where $x$ is a terminal, the nextSymbols method was used. It took an index $i$ in the input string and returns an index $j$ such that the substring of the input string from $i$ to $j - 1$ matches $x$. Thus, $ j $ is the index in the input string from which the parsing  should continue by going to the slot $(N \rightarrow \alpha x.\beta)$. Considering the graph input there can be several similar positions. Therefore, the signature of this method has been changed. Now it returns a list of identifiers.
\end{itemize}

As far as the original GLL is aimed to handle arbitrary context-free grammars, this solution can handle arbitrary grammars too. It makes the solution less restrictive with regard to a query specification language, thus being more user-friendly.

%As a storage for graphs,the Neo4j graph database was used. This is the most commonly used graph DBMS. Neo4j supports Cypher query language and represents data as nodes (vertices) and relations between them (edges). Vertices and edges can be labeled. Neo4j is an open source project and, like Iguana, implemented in Java. The modified algorithm has been integrated with Neo4j using the Native Java API.


\section{Experimental setup}
In this section, the scheme of runtime partial evaluation is presented as well as the evaluation configuration.

\subsection{Runtime partial evaluation}

\begin{figure}[b!]
    \centering
    \includegraphics[width=\linewidth]{figures/SeqDiagram.pdf}
    \caption{Runtime partial evaluation diagram}
    \label{fig:seq_pe}
\end{figure}

In practice, it is infeasible to compile a new kernel for each static input value, which is often known in runtime. Thus the partial evaluation of the kernel should be performed in runtime as well as the kernel compilation to a specific GPU target.
Each specialized benchmark scenario corresponds to the sequence in figure~\ref{fig:seq_pe}. The device kernel in Impala is included in the target application using \lstinline{xxd} tool during the compilation. When static data becomes known at runtime, the kernel wrapper is constructed, that supplies the static data to the included kernel, creating partially applied kernel. Then AnyDSL JIT compiler is invoked, which specializes the kernel according to the annotations provided, and static arguments supplied, generating CUDA C code, which is then passed to NVRTC\,\footnote{\url{https://docs.nvidia.com/cuda/nvrtc/index.html} (last accessed date: 30.05.2020)} and got eventually compiled to GPU assembly and invoked.

The evaluation aim is to show whether a device kernel could benefit from data embedding performed by partial evaluation and possible reduction of static computations. For the data to be embedded, the accesses should be static, corresponding to constant memory to be a good fit in CUDA code, thus constant memory being a baseline in several scenarios. Further, only the execution time of a device kernel should be measured, since, for example, overhead for partial evaluation and JIT compilation for a device could be hidden by GPU data transferring or other workarounds.

Since NVRTC is internal to AnyDSL framework, the benchmarking results could be obtained via nvprof\,\footnote{\url{https://docs.nvidia.com/cuda/profiler-users-guide/index.html} \\ (last accessed date: 30.05.2020)} or by utilizing a specially recompiled version of the framework runtime with CUDA events. The latter option is used since it allows to perform warm-up runs of the kernel to make the benchmarking more reliable. The whole system is implemented in C++ and Python, and enclosed into a Docker container with the datasets indexed in Git LFS\footnote{\url{https://git-lfs.github.com/} (last accessed date: 30.05.2020)} for benchmarks to be easily built and run on any system with NVIDIA GPU\,\footnote{\url{https://github.com/Tiltedprogrammer/spec} (last accessed date: 30.05.2020)}. The following GPGPU scenarios have been implemented, which are fit under the described in~\ref{PEsurvey} pipeline.
\begin{itemize}
    \item Na\'ive single substring matching.
    \item Na\'ive multiple substring matching.
    \item Aho--Corasick matching.
    \item 2-D convolution filter.
\end{itemize}The following system has been used to run the benchmarks: Ubuntu 18.04 with CUDA Toolkit 10.2. hosted in Google Cloud bundled with 4 cores of Intel Xeon and NVIDIA Tesla T4 GPU.

\section{Evaluation}

This section describes the methodology and answers the following research questions.

\begin{enumerate}
    \item Does fusion via distillation give any benefits at the software and hardware levels?
    \item What are the properties of the generated hardware?
    \item Does the generated hardware outperform software implementations?
\end{enumerate}

\subsection{Methodology}

Our focus is on creating a basis for future research and experiments, thus we make our experiments as much reproducible as possible\footnote{\url{https://github.com/sedwards-lab/fhw/tree/sparse-linear-algebra-distillation/examples/QTreeBenchmarks/diploma/verilog-bool-no-nnz-inlined} (online; accessed:
2022-06-07) Here one could find all the results. For each benchmark all statistics are specified: matrix names, their sizes, collected metrics for both hardware and software benchmarks.}. We benchmarked the following list of chained functions. The choice is prompted by the current state of the distiller: at the moment, it does not successfully distill matrix multiplication. However, the functions are still practical enough, for example, chained addition could be seen in Luby's maximal independent set algorithm and clearly describe the applicability of the proposed approach.

\begin{itemize}
    \item \mintinline{Haskell}{mAdd (==False) (||) (mAdd (==False) (||) m1 m2) m3}
    \item \mintinline{Haskell}{mask (mAdd (== False) (||) m2 m3) (m1)}
    \item \mintinline{Haskell}{map (==Zero) (to_nat) (mAdd (==False) (||) m1 m2}
    \item \mintinline{Haskell}{map (==Zero) (to_nat) (kron (==False) (&&) m1 m2}
\end{itemize}

Above, \mintinline{Haskell}{Zero} and \texttt{to\_nat} are corresponding definitions for Peano arithmetics, since the \texttt{.pot} language does not have any primitives. For the same reason, we operated with boolean matrices. Such functions could be abstracted with free variables and then instantiated in the emitted Haskell code. However, to get maximum from distillation, we provided all the information about the functions. 

For these functions, we compared the execution time of distilled and not distilled hardware generated circuits, execution time of original and distilled Haskell code and reference \textit{Suite Sparse}\footnote{\url{https://github.com/DrTimothyAldenDavis/GraphBLAS} (online; accessed:
2022-06-07), Suite Sparse library sources.}\textsuperscript{,}\footnote{The library also uses different variations of coordinate formats (opaque to the user) and not a quadtree representation.} variants of these functions in C\texttt{++}. Note that SuiteSparse does not support recursive data types, thus only the first two function chains were implemented in SuiteSparse (since Peano number is essentially a linked list). We did not replace Peano numbers with integers, so our experiments could be interpreted easier. For hardware experiments we collected execution time and the number of memory writes and reads, to access how well fusion is performed. For software experiments we only measured the execution time. Also note that we measured only the time, required to execute the lines above, not including any IO, required to get and evaluate function arguments. But in hardware benchmarks we measured the time required to pass arguments into the circuit's memory, because such IO is inevitable. It is tricky to make such measures in Haskell due to laziness, thus the programs were compiled with \texttt{--fno-full-laziness} to turn off memoization. Also all the arguments were forced to normal form via \texttt{force} and \texttt{evaluate}. Haskell programs were compiled\footnote{GHC 8.10.4.} with \texttt{-O2 --fno-full-laziness} and Suite Sparse was compiled with default flags and linked as a shared library to C\texttt{++} code.

We took matrices from SuiteSparse matrix collection with sizes ranging from \texttt{64x64} to \texttt{512x512}. We limited ourselves with such sizes due to the fact that this is the maximum sizes that fit into \texttt{bram} with $2^{16}$ address space. Such number of \texttt{bram} blocks is available only on really expensive FPGA boards, thus in practice sizes would be smaller to achieve better utilization. Once again, it models the situation when data fits into the cache, since \texttt{bram} in our circuits will operate as a cache in real application.

\subsection{Experiments}

Table~\ref{tab:bench_results} shows the results of all execution time benchmarks. To evaluate execution time for hardware simulation, implementation stage was performed to assess the maximum frequency of FPGA device used for synthesis and implementation, and the number of execution cycles was multiplied by the number of nanoseconds a clock cycle takes. The frequencies were equal within the same benchamark set, i.e., frequency was not affected by distillation. We used \texttt{xcu250figd2104-2L} device\footnote{\url{https://www.xilinx.com/products/boards-and-kits/alveo/u250.html}  (online; accessed:
2022-06-07)} for synthesis and implementation stages. It is not really a casual and affordable chip, but it contains enough \texttt{bram} for our evaluation to see scalability. In the table a median across several benchmarks is shown. 

As it could be seen, distillation steadily increases performance: up to 2x speedup for hardware simulation and up to 3x for software benchmarks. The results are maintained within the borders of the corresponding confidence interval and the borders are not shown for brevity. Hardware speedup is lower due to the different execution semantics, dataflow is not reduction-based and distillation is a reduction-based transformation. Note that generated hardware appears to be less performant than both Haskell and C\texttt{++}, which a bit contradicts the results from~\cite{oldfhw}. For hardware benchmarks \texttt{time (IO)} shows the execution time including the time needed to transfer the data though the arguments, \texttt{time (no IO)} does not include it in its turn. It could be seen that not all the benchmarks are computationally extensive enough to cover memory transferring costs, but for more complex examples the ratio would be better. Since we basically transfer the matrices node by node from a file in the testbench, we have probably the lowest possible latency, and in practice it would be higher if reading from DDR, however the bandwidth could be increased. Noticeably, running times for \texttt{mMaskAdd} for C\texttt{++} and distilled Haskell are similar, which shows that fusion really provides some extra performance: SuiteSparse at the moment does not implement any fusion.

Table~\ref{tab:mem_results} summarizes the ratios between distilled and not distilled hardware circuits memory reads and writes. Since in general case distillation removes extra pattern matching, essentially it saves memory reads and writes. The eventual number of memory reads and writes is implementation dependent, thus the table shows what share of speedup is prompted by saving memory operations. Distillation successfully reduces the number of memory accesses, about 15\% on average. \texttt{mMapKron} has a bit higher ratio due to the fact that \texttt{Nat} numbers require additional memory accesses, since the type is recursive. It could also be seen that a major part of speedups is attributed to saved memory accesses. 

Finally, table~\ref{tab:resource_util} shows device resources utilization ratios between distilled and not distilled hardware circuits and frequencies. Columns are device primitives: registers, lookup tables, \texttt{bram} blocks or multiplexers. Utilization for both types of circuits is below 1\% of available resources on the device, except for the memory. Memory blocks utilization is about 30\% (since we choose larger \texttt{brams} to store larger matrices). Apart from that, distilled circuits could have both higher and lower utilization. Since the hardware generation is primarily syntax-directed it follows from the distilled program structure. For example, distillation might glue two recursive functions into one (in that case, memory utilization would be lower, because each cluster of mutually recursive functions possesses its own heap) or make more recursive functions than in the original program. The frequencies are the same, however, they possibly could be made better with more intelligent buffer allocation.

\subsection{Discussion}
Answering the research questions above.

\begin{enumerate}
    \item Fusion gives significant benefits, however at the hardware level the benefits are a bit smaller since hardware semantics is not reduction based. The benefits at the hardware level are mostly determined by the reduced number of memory accesses (each access takes 2 clock cycles). Notably, distilled Haskell implementation of \texttt{mMaskAdd} has similar performance with C\texttt{++}. 
    \item Device utilization is low, but such circuits could be copied on the same device to provide better utilization and higher parallelism. Resource utilization might be both better and worse after distillation, depending on the transformed program itself since translation is syntax-directed. Frequency could be increased by more intelligent buffering strategy.
    \item Although we use low-latency design with \texttt{bram}s that take 2 clock cycles per request and transfer matrices from files, which does not have any latency in simulation, we get slower execution time than Haskell and C\texttt{++} counterparts. It could be partly due to excessive buffering performed by FHW at the moment. Also there is no pipelining for recursive calls, i.e. only one set
of function argument tokens are allowed to enter a tail-recursive function call until a result is finally generated. Further CPS transformation hinders parallelization, which could be made more explicit with SSA. Some other optimizations exist that may significantly influence the performance. Also, since device utilization is about 1\%, such circuits could be copied on one device and provide more parallelism. A more detailed discussion could be found at~\cite{Edwards2019FHWP}.
\end{enumerate}

Distillation clearly showed its applicability to optimization of sparse linear algebra routines and notably it still could be combined with other techniques, like rewrite rules to achieve better results. High-level synthesis has a room for improvements by increasing pipelining, parallelism and frequency and the generated hardware could be improved from usability perspective: a support for arbitrary sized matrices is desirable. Thus we will focus on these directions. Probably a better solution would be to embed \texttt{.pot} language into e.g. Haskell to leverage its type system (to be able to use some rewrite rules as well), and add support for primitive types and parallel primitives to be able to conduct a more scalable comparison with SuiteSparse (since SuiteSparse is multithreaded). For such embedding different execution models could be implemented, including hardware synthesis, for which SSA form of GRIN looks promising, as well as extra optimizations shipped with GRIN. For hardware synthesis, an interesting direction is achieving predictable results in hardware from certain modifications in software. This property partly holds for the current approach, since the translation is syntax- directed. More information on this could be found at~\cite{predict}.

\pagebreak

\begin{table}[t]
\scriptsize
\centering
\caption*{mAddAdd}
\begin{tabular}{|c|c|c|c|c|c|c|c|c|c|} 
\hline
\rowcolor{LightBlue}
\multicolumn{3}{|c|}{Matrices dimensions} & Haskell & Haskell (distilled) & \multicolumn{2}{c|}{FHW} & \multicolumn{2}{c|}{FHW (distilled)} & {C\texttt{++}}\\
% \rowcolor{LightBlue}
\hline
m1 & m2 & m3 & time & time & time (no IO) & time (IO) & time (no IO) & time (IO) & time \\ 
\hline
64 & 64 & 64 & 29 us & 20 us & 76 us & 170 us & 64 us & 158 us & 14 us\\ 
128 & 128 & 128 & 94 & 79 & 146 & 476 & 134 & 469 & 30 \\
256 & 256 & 256 & 123 & 103 & 202 &  681 & 168 & 662 & 44\\
512 & 512 & 512 & 219 & 143 & 474 & 1192 & 375 & 1093 & 49\\
\hline
\end{tabular}

\caption*{mMaskAdd}
\begin{tabular}{|c|c|c|c|c|c|c|c|c|c|} 
\hline
\rowcolor{LightBlue}
\multicolumn{3}{|c|}{Matrices dimensions} & Haskell & Haskell (distilled) & \multicolumn{2}{c|}{FHW} & \multicolumn{2}{c|}{FHW (distilled)} & {C\texttt{++}}\\
% \rowcolor{LightBlue}
\hline
m1 & m2 & m3 & time & time & time (no IO) & time (IO) & time (no IO) & time (IO) & time \\ 
\hline
64 & 64 & 64 & 10 us & 7 us & 64 us & 133 us & 46 us & 111 us & 18 us\\ 
128 & 128 & 128 & 38 & 30 & 118 & 322 & 75 & 292 & 33 \\
256 & 256 & 256 & 48 & 42 & 168 &  498 & 104 & 456 & 46\\
512 & 512 & 512 & 126 & 76 & 400 & 762 & 300 & 729 & 65\\
\hline
\end{tabular}

\caption*{mMapAdd}
\begin{tabular}{|c|c|c|c|c|c|c|c|c|c|} 
\hline
\rowcolor{LightBlue}
\multicolumn{3}{|c|}{Matrices dimensions} & Haskell & Haskell (distilled) & \multicolumn{2}{c|}{FHW} & \multicolumn{2}{c|}{FHW (distilled)} & {C\texttt{++}}\\
% \rowcolor{LightBlue}
\hline
m1 & m2 & m3 & time & time & time (no IO) & time (IO) & time (no IO) & time (IO) & time \\ 
\hline
64 & 64 & --- & 45 us & 37 us & 189 us & 253 us & 137 us & 202 us & ---\\ 
128 & 128 & --- & 162 & 105 & 524 & 685 & 397 & 579 & --- \\
256 & 256 & --- & 312 & 216 & 1047 &  1360 & 680 & 986 & ---\\
512 & 512 & --- & 436 & 273 & 1346 & 1776 & 900 & 1330 & ---\\
\hline
\end{tabular}

\caption*{mMapKron}
\begin{tabular}{|c|c|c|c|c|c|c|c|c|c|} 
\hline
\rowcolor{LightBlue}
\multicolumn{3}{|c|}{Matrices dimensions} & Haskell & Haskell (distilled) & \multicolumn{2}{c|}{FHW} & \multicolumn{2}{c|}{FHW (distilled)} & {C\texttt{++}}\\
% \rowcolor{LightBlue}
\hline
m1 & m2 & m3 & time & time & time (no IO) & time (IO) & time (no IO) & time (IO) & time \\ 
\hline
2 & 64 & --- & 64 us & 36 us & 212 us & 242 us & 94 us & 125 us & ---\\ 
2 & 128 & --- & 137 & 68 & 434 & 502 & 199 & 266 & --- \\
2 & 256 & --- & 364 & 126 & 1004 &  1188 & 449 & 636 & ---\\
4 & 128 & --- & 302 & 94 & 694 & 763 & 330 & 401 & ---\\
\hline
\end{tabular}



\caption{Execution time}
\label{tab:bench_results}

\end{table}
\begin{table}[h]
\scriptsize
\begin{minipage}{0.45\linewidth}
\centering
\caption*{mAddAdd}
\begin{tabular}{|c|c|c|c|c|c|c|} 
\hline
\rowcolor{LightBlue}
\multicolumn{3}{|c|}{Matrices dimensions} & \multicolumn{2}{c|}{Ratio ($\frac{FHW}{FHW_{distilled}}$)}\\
% \rowcolor{LightBlue}
\hline
m1 & m2 & m3 & writes & reads\\ 
\hline
64 & 64 & 64 & 1.10 & 1.15\\ 
128 & 128 & 128 & 1.02 & 1.05\\
256 & 256 & 256 & 1.03 & 1.06\\
512 & 512 & 512 & 1.10 & 1.16\\
\hline
\end{tabular}
\end{minipage}
\begin{minipage}{0.45\linewidth}
\centering
\caption*{mMaskAdd}
\begin{tabular}{|c|c|c|c|c|c|c|} 
\hline
\rowcolor{LightBlue}
\multicolumn{3}{|c|}{Matrices dimensions} & \multicolumn{2}{c|}{Ratio ($\frac{FHW}{FHW_{distilled}}$)}\\
% \rowcolor{LightBlue}
\hline
m1 & m2 & m3 & writes & reads\\ 
\hline
64 & 64 & 64 & 1.13 & 1.26\\ 
128 & 128 & 128 & 1.06 & 1.11\\
256 & 256 & 256 & 1.08 & 1.09\\
512 & 512 & 512 & 1.10 & 1.16\\
\hline
\end{tabular}
\end{minipage}
\begin{minipage}{0.45\linewidth}
\centering
\caption*{mMapAdd}
\begin{tabular}{|c|c|c|c|c|c|c|} 
\hline
\rowcolor{LightBlue}
\multicolumn{3}{|c|}{Matrices dimensions} & \multicolumn{2}{c|}{Ratio ($\frac{FHW}{FHW_{distilled}}$)}\\
% \rowcolor{LightBlue}
\hline
m1 & m2 & m3 & writes & reads\\ 
\hline
64 & 64 & --- & 1.10 & 1.21\\ 
128 & 128 & --- & 1.07 & 1.14\\
256 & 256 & --- & 1.07 & 1.19\\
512 & 512 & --- & 1.10 & 1.21\\
\hline
\end{tabular}
\end{minipage}
\hfill
\begin{minipage}{0.45\linewidth}
\centering
\caption*{mMapKron}
\begin{tabular}{|c|c|c|c|c|c|c|} 
\hline
\rowcolor{LightBlue}
\multicolumn{3}{|c|}{Matrices dimensions} & \multicolumn{2}{c|}{Ratio ($\frac{FHW}{FHW_{distilled}}$)}\\
% \rowcolor{LightBlue}
\hline
m1 & m2 & m3 & writes & reads\\ 
\hline
2 & 64 & --- & 1.71 & 1.88\\ 
2 & 128 & --- & 1.72 & 1.87\\
2 & 256 & --- & 1.65 & 1.83\\
4 & 128 & --- & 1.81 & 1.91\\
\hline
\end{tabular}
\end{minipage}

\caption{Memory accesses}
\label{tab:mem_results}
\end{table}

\begin{table}[h]
\scriptsize
\centering
\begin{tabular}{|l|c|c|c|c|c|c|c|c|c|} 
\hline
\rowcolor{LightBlue}

{Benchmark} & \multicolumn{8}{c|}{Ratio (${\frac{FHW}{FHW_{distilled}}}$)} & {Frequency}\\
\hline
{} & FDRE & LUT3 & LUT6 & LUT5 & LUT4 & LUT2 & RAMB36E2 & MUXF7 & {} \\
% \rowcolor{LightBlue}
\hline
mAddAdd & 0.3 & 0.3 & 0.3 & 0.5 & 0.3 & 0.3 & 0.5 & 0.5 & 200 MHz\\ 
mMaskAdd & 0.5 & 0.5 & 0.7 & 0.4 & 0.7 & 0.5 & 0.7 & 0.6 & 200 MHz\\
mMapAdd & 1 & 0.9 & 0.9 & 1.2 & 1 & 1.1 & 1.1 & 1.2 & 200 MHz\\
mMapKron & 1.5 & 1.5 & 1.3 & 2 & 2 & 1.8 & 1.4 & 1.7 & 200 MHz\\
\hline
\end{tabular}
\caption{Resource utilization}
\label{tab:resource_util}
\end{table}
\pagebreak

\section{Conclusion and Future Work}

We present !!!

Our evaluation shows that !!!

First direction for future research is a more detailed CFPQ algorithms investigation.
We should do More evaluation on sparse matrices on GPGPUs.

Also it is nesessary to implement and evaluate solutions for graphs which is not fit in RAM.
There is a set of technics for huge matrices multiplication.
Is it possible to dopt it for CFPQ

Another direcion is a dataset improvement.
More data.
More grammars/queries.

 
\setmonofont[Mapping=tex-text]{CMU Typewriter Text}
\bibliographystyle{ugost2008ls}
\bibliography{diploma.bib}
\end{document}
