% Тут используется класс, установленный на сервере Papeeria. На случай, если
% текст понадобится редактировать где-то в другом месте, рядом лежит файл matmex-diploma-custom.cls
% который в момент своего создания был идентичен классу, установленному на сервере.
% Для того, чтобы им воспользоваться, замените matmex-diploma на matmex-diploma-custom
% Если вы работаете исключительно в Papeeria то мы настоятельно рекомендуем пользоваться
% классом matmex-diploma, поскольку он будет автоматически обновляться по мере внесения корректив
%

% По умолчанию используется шрифт 14 размера. Если нужен 12-й шрифт, уберите опцию [14pt]
%\documentclass[14pt]{matmex-diploma}

\documentclass[14pt]{matmex-diploma-custom}
\usepackage{lipsum}
\usepackage{listings}
\usepackage{array}
\usepackage{graphicx}
\graphicspath{ {./} }
\usepackage{amsmath}

\usepackage{subcaption}
\newcommand{\red}[1]{\textcolor{red}{#1}}
\usepackage{scalerel,amssymb}
\def\mcirc{\mathbin{\scalerel*{\circ}{j}}}
\def\msquare{\mathord{\scalerel*{\Box}{gX}}}
\usepackage{algorithm}
\usepackage{algpseudocode}
\usepackage{mathtools}

\floatname{algorithm}{Алгоритм}

\begin{document}

% Год, город, название университета и факультета предопределены,
% но можно и поменять.
% Если англоязычная титульная страница не нужна, то ее можно просто удалить.
\filltitle{ru}{
    chair              = {Кафедра системного программирования\\Программная инженерия},
    title              = {Реализация и применение строковых алгоритмов к задаче поиска
повторов в документации программного обеспечения},
    % Здесь указывается тип работы. Возможные значения:
    %   coursework - Курсовая работа
    %   diploma - Диплом специалиста
    %   master - Диплом магистра
    %   bachelor - Диплом бакалавра
    type               = {bachelor},
    position           = {студента},
    group              = 16.Б11-мм,
    % group              = 16.Б11-мм,
    author             = {Мишин Никита Матвеевич},
    supervisorPosition = {к.\,ф.-м.\,н., доцент кафедры информатики СПбГУ},
    faculty = {Математико-механический факультет},
    supervisor         = {Григорьев С.\,В.},
    consultantPosition  = {программист ООО ”Интеллиджей Лабс”\\ к.\,ф.-м.\,н.,},
    consultant      =     {Березун Д.\,А.},
    reviewerPosition = {д.\,ф.\,н., доцент},
    reviewer = {Тискин А.\,В.},
%   university         = {Санкт-Петербургский Государственный Университет},
%   faculty            = {Математико-механический факультет},
%   city               = {Санкт-Петербург},
%   year               = {2013}
}
\filltitle{en}{
    chair              = {Software Engineering},
    title              = {String algorithms: implementation and application to clone detection in software documentation},
    author             = {Mishin Nikita},
    supervisorPosition = {Associate professor, Ph.\,D.},
    supervisor         = {Semyon Grigorev},
    type               = {bachelor},
    consultantPosition  = {IntelliJ Labs Co. Ltd developer\\ Ph.\,D.},
    consultant      =     {Daniil Berezun},
    reviewerPosition   = {Associate professor, Ph.\,D.},
    reviewer           = {Alexander Tiskin}
}

\maketitle
\tableofcontents
% У введения нет номера главы

\paragraph{}
Графы являются одной из основных структур в информатике, а алгоритмы над ними чрезвычайно важны в анализе компьютерных и социальных сетей, статическом анализе кода, биоинформатике~\cite{gb_math}. Графы, возникающие в данных областях, могут содержать миллионы узлов и ребер, поэтому распараллеливание алгоритмов для работы с ними оказывается необходимым условием достижения высокой производительности. Однако такое улучшение их эффективности происходит за счет более сложной модели программирования~\cite{blast}. Результатом этого становится, в том числе, несоответствие между языками высокого уровня, на которых пользователи и разработчики графовых алгоритмов предпочли бы программировать (например, Python), и языками программирования для параллельного оборудования~\cite{blast}. Таким образом, существующие решения либо просты в использовании (networkx\footnote{Репозиторий библиотеки networkx: \url{https://github.com/networkx/networkx}. Дата посещения: 13.12.2020}), либо высокопроизводительны (gunrock\footnote{Репозиторий библиотеки gunrock: \url{https://github.com/gunrock/gunrock} Дата посещения: 13.12.2020}). Проблемой является и то, что традиционные параллельные алгоритмы для анализа графов тяжело реализовывать и оптимизировать, а прирост производительности от роста числа параллельных процессов снижается~\cite{gb_math}. 

С появлением эффективных структур и алгоритмов для разреженных матриц становится возможным использование подхода к вычислению над графами, основанного на линейной алгебре. Матрица смежности может представлять широкий спектр графов, включая ориентированные, взвешенные, двудольные. Ключевым свойством такого подхода является способность оперировать богатым набором графов различных типов с помощью небольшого набора матричных операций над полукольцами. Например, транспонирование матрицы смежности изменяет направления ребер в ориентированном графе, а умножение матрицы на вектор, как показано на рисунке~\ref{fig:bfs_step}, является шагом в алгоритме поиска в ширину.

\begin{figure}[h!]
    \centering
    \includegraphics[width=0.9\linewidth]{pictures/MatrixBFS.png}
    \caption{Вычисление одного шага в алгоритме поиска в ширину\footnotemark}
    \label{fig:bfs_step}
\end{figure}

Спецификация GraphBLAS~\cite{gb_math} определяет базовые примитивы для построения графовых алгоритмов в терминах линейной алгебры. Разработчики считают, что эта область достаточно зрела, чтобы иметь потребность в стандартизации. Стандартизация позволяет сконцентрировать усилия исследователей на разработке инновационных алгоритмов для анализа и обработки графов, а не придумывать все новые, во многом пересекающиеся, низкоуровневые решения. Кроме того, благодаря такому подходу, по словам авторов\cite{sevengr}, возможно решение некоторых проблем, описанных ниже.
\begin{enumerate}
    \item \textbf{Переносимость.} Алгоритмы не требуют модификаций для достижения высокой производительности на конкретном устройстве.
    \item \textbf{Лаконичность.} Алгоритмы выражаются гораздо меньшим числом строчек кода.
    \item \textbf{Производительность.} Алгоритмы остаются высокопроизводительными.
    \item \textbf{Масштабируемость.} Алгоритмы эффективны как на небольших, так и на огромных данных.
\end{enumerate}

\footnotetext{GraphBLAS [Электронный ресурс] // Википедия. Свободная энциклопедия. – URL: \url{https://en.wikipedia.org/wiki/GraphBLAS} (дата обращения: 13.12.2020).}

GraphBLAS описывает небольшое множество математических операций, которые необходимы для реализации широкого спектра операций над графами. В стандарте описаны следующие объекты:
\begin{itemize}
    \item абстрактные структуры для хранения дынных (матрицы, векторы)
    \item алгебраические структуры (моноиды, полукольца, бинарные и унарные операторы)
    \item операции линейной алгебры над произвольными алгебраическими структурами (произведение матриц, поэлементное сложение и умножение, взятие подматрицы и т.д.)
    \item объекты управления (маски и дескрипторы)
\end{itemize}

На данный момент стандарт GraphBLAS уже имеет несколько полноценных реализаций\footnote{Форум, посвященный стандарту GraphBLAS: \url{https://graphblas.github.io/}. Дата посещения: 04.06.2021}, однако все они в основном ориентированы на исполнение на CPU. В то же время разработка инструмента с поддержкой исполнения на графических процессорах общего назначения является перспективным направлением исследований, так как их использование может существенно повысить производительность такого рода решений\cite{gbtl}\cite{blast}. На текущий момент нет стандартного подхода к реализации спецификации GraphBLAS на GPU --- разработчики сталкиваются не только с проблемами, связанными с реализацией обобщенных операций на графических процессорах с помощью стандартных инструментов языка C++, но и с переносимостью решений, основанных на программно-аппаратной платформе CUDA. Одним из возможных подходов к реализации GraphBLAS на GPU является использование языка высокого уровня, а также библиотек, динамически транслирующих конструкции и объекты данного языка в низкоуровневый код, способный исполнятся на графическом процессоре видеокарты. 


\section{Постановка задачи}
Целью данной дипломной работы является адаптация алгоритмов решения полулокальных задач  поиска наибольшей общей подпоследовательности и выравнивая строк к задачам поиска повторов в документации ПО. 
Для достижения этой цели были сформулированы следующие задачи.
\begin{itemize}
    \item Исследовать существующие теоретические алгоритмы решения задач полулокального поиска наибольшей общей подпоследовательности и выравнивания строк и реализовать их на практике в виде \emph{библиотеки алгоритмов}.
    % \item Реализация алгоритмов решения полулокальных задач поиска \emph{lcs} и \emph{sa}, предложенных и только теоретически описанных Тискиным в книге (или лучше статьях?).
    \item Адаптировать алгоритмы решения полулокальных задач  поиска \emph{lcs} и \emph{sa} к задаче поиска повторов в \emph{JavaDoc} документации и реализовать соответствующее приложение на их основе. 
    % \item Реализация приложения для поиска повторов в \emph{JavaDoc} документации с применением алгоритмов решения полулокальных задач.
    \item Провести экспериментальное исследование реализованных алгоритмов  и анализ результатов.
\end{itemize}

\section{Related work \& background}
This section includes basic notation and definitions in graph theory and formal language theory which are used in this work. Also, the further description of both the theoretical part of the GLL-based CFPQ algorithm and its implementation are provided.

\subsection{Basic Definitions of Formal Languages}

In this work, the context-free grammars are used as path constraints, thus context-free languages and grammars are defined in this subsection.

\begin{rudefinition}A \emph{context-free grammar} is a tuple $G= \langle N, \Sigma, P, S \rangle$, where
\begin{itemize}
    \item $N$ is a finite set of nonterminals
    \item $\Sigma$ is a finite set of terminals, $N \cap \Sigma = \varnothing$
    \item $P$ is a finite set of productions of the form $A \to \alpha$, where $A \in N,\ \alpha \in (N \cup \Sigma)^*$
    \item $S$ $\in$ $N$.
\end{itemize} \qed
\end{rudefinition}

We use the conventional notation $A \Rightarrow^* w$ to denote, that a
word $w \in \Sigma^*$ can be derived from a non-terminal $A$ using some sequence of production rules from $P$.

\begin{rudefinition} A \emph{context-free language} is a language generated by a con-text-free grammar $G$:
\begin{align*}
     L(G) = \{w \in \Sigma^* \mid S \Rightarrow^* w \}
\end{align*}
\end{rudefinition}

% \begin{rudefinition} A \emph{context-free language with a specified starting non-terminal $S$} is a set of strings that can be generated from $S$ by a context-free grammar $G$:
% \begin{align*}
%      L(G_S) = \{w \in \Sigma^* \mid S \Rightarrow^* w \}
% \end{align*}
% \end{rudefinition}

\subsection{Basic Definitions of Graph Theory}
In a simplified way, the Neo4j graph database uses a labeled directed graph as a data model. It can be defined as follows.

\begin{rudefinition} \emph{Labeled directed graph} is a tuple $D = \langle V, E, T \rangle$, where
\begin{itemize}
    \item $V$ is a finite set of vertices. For simplicity, we assume that the vertices are natural numbers from $0$ to $|V|-1$.
    \item $T$ is a set of labels on edges.
    \item $E \subseteq V \times T \times V$ is a set of edges.
\end{itemize} \qed
\end{rudefinition}

\begin{rudefinition}
Path $\pi$ in the graph $D = \langle V, E, T \rangle$ is a finite sequence of edges $(e_0, e_1, ..., e_{n-1})$, where $\forall~ j,~ 0 \leq j \leq n - 1: e_j=(v_j,t_j,v_{j+1}) \in E$.

We denote the set of all paths in the graph $D$ as $\pi(D)$. \qed
\end{rudefinition}

\subsection{Context-free Path Querying}
Now, we can define context-free path querying problems. Let be:
\begin{itemize}
      \item a context-free grammar $G=\langle N, \Sigma, P, S \rangle$;
      \item a directed graph $D=\langle V, E, T \rangle$, where $V$ is the set of vertices of the graph, $ E \subseteq V \times T \times V $ is the set of edges, $ T \subseteq \Sigma $ is the set of labels on edges, where each label is a terminal symbol of the grammar $G$;
\item a set of start vertices $V_S \subseteq V$ and final vertices \mbox {$V_F \subseteq V$.}
\end{itemize}

Consider a path in the graph $D$: $$\pi = (e_0, e_1, \cdots, e_{n - 1}), $$ where $ e_k = (v_{k}, t_k , v_{k+1}), ~ \forall~k,~ 0 \leq k \leq n - 1 ~e_k \in E$.
To path in the graph the word $ l(\pi) = t_0t_1 \cdots t_{n_1} $ is associated --- the concatenation of the labels on the edges of this path.

In the introduced notation, the following problems can be formulated.

\begin{itemize}
     \item \textbf{The problem of a path querying in a graph with context-free constraints} consists in finding all paths in the graph such that $l(\pi) \in L (G)$ and $v_0 \in V_S, ~v_n \in V_F$.
    
     \item \textbf{The problem of reachability in a graph with context-free constraints} consists in finding a set of pairs of vertices for which there is a path with a beginning and an end at these vertices, such that the word composed of labels of the edges of the path  belongs to the given language: $ \{(v_i, v_j) ~ | ~ \exists ~ l (\pi) \in L (G) $ and $ v_0 \in V_S, ~ v_n \in V_F \} $.
\end{itemize}

It should be noted that it is often necessary to identify complex dependencies in a graph data model. So, according to the context and application area, both variants of the above problems are of practical importance. 

For each problem there are two variants of set of starting vertices: the set may consist of all vertices of a graph or may consist only a particular vertices of interest. The first variant is called all-pairs context-free path querying problem and the second is called a multiple-source (and a single-source as a partial case) context-free path querying problem.

\subsection{Generalized LL Parsing Algorithm}
One of the common parsing techniques is the LL(k) algorithm~\cite{10.5555/1076440}, that performs top-down analysis with a lookahead. It means that the decision about which production of the grammar should be applied is based on looking at the $ k $ following character from the current one. To choose the right production rule at this step algorithm supports a parsing table, where the information for parsing the current non-terminal is stored. However, it can be applied only to a subset of the context-free grammars class and does not support ambiguous context-free grammars or grammars with left recursion in derivation.

Top-down analysis algorithms are relatively easier to implement and debug, because it fully matches the structure of the grammar. For this reason, to extend the parsing power of above-mentioned technique there was proposed~\cite{SCOTT2010177} the generalized LL (GLL) algorithm. Also GLL can handle ambiguous grammars.
In case of LL(k) algorithm may arise the situation when it is impossible to determine which production should be applied in the current state of the parsing process ~\cite{10.1145/800105.803402}. To solve this issue the GLL algorithm maintains a queue of descriptors. Each descriptor is a structure that describes the current state of the analyzer. Thus, using a queue of descriptors allows one to consider all possible transitions during the operation of the parser.

The parsing table for the generalized GLL algorithm can store multiple alternatives for parsing the current non-terminal. In this case, descriptor duplication can occur. For efficient storage and reuse of many different descriptors, GLL uses a specific structure --- Graph Structured Stack (GSS)~\cite{10.5555/1623611.1623625}.

To represent the result, GLL provides the Shared Packed Parse Forest (SPPF) structure~\cite{SCOTT20131828}, which contains all derivation trees for all paths satisfying the specified language.

\subsection{GLL-based CFPQ Algorithm}
As it was showed, classical GLL parsing technique can be used to solve context-free language constrained path problem. It means that such technique can be used to proceed graph input. Previously, the algorithm was generalized from linear input to graph processing, as was described in \cite{10.1145/3166094.3166104}.

To do this, the following modifications were proposed.
\begin{itemize}
\item A query has became a triple: a set of initial vertices, a set of final vertices, and a grammar.
\item An initial set of descriptors must include all the start vertices of the graph.
\item At the step of transition to the next character, it is necessary to support all possible transition options that correspond to all outgoing edges of the vertex.
\item If parsing is completed, it is necessary to check whether the final vertex in the parsing belongs to the set of final vertices of the graph.
\end{itemize}

The described principles of the generalized GLL algorithm are important for understanding the features of its implementation, which will be described below.

The implementation of the algorithm is based on the Iguana project which is written in Java. This library provides the modified GLL algorithm. The advantage of  Iguana project is that it uses a more efficient GSS for GLL parsing. In addition, it does not affect the worst-case cubic run time and space complexities of GLL parsing.
 
Under this work, it is important to pay attention to the following changes that were made to the workflow of the GLL algorithm to unable graph processing.

\begin{itemize}
    \item In order to support graph processing, the abstraction of an input data was changed. The new implementation of the $Input$ interface has been added. Now it is represented as a graph adjacency list, a set of start and final vertices of the resulting paths.
    \item There can be multiple start vertices for a graph input, unlike a linear input. So, also the initialization of the descriptor queue was modified. In case of processing a descriptor with slot $(N \rightarrow \alpha.x\beta)$, where $x$ is a terminal, the nextSymbols method was used. It took an index $i$ in the input string and returns an index $j$ such that the substring of the input string from $i$ to $j - 1$ matches $x$. Thus, $ j $ is the index in the input string from which the parsing  should continue by going to the slot $(N \rightarrow \alpha x.\beta)$. Considering the graph input there can be several similar positions. Therefore, the signature of this method has been changed. Now it returns a list of identifiers.
\end{itemize}

As far as the original GLL is aimed to handle arbitrary context-free grammars, this solution can handle arbitrary grammars too. It makes the solution less restrictive with regard to a query specification language, thus being more user-friendly.

%As a storage for graphs,the Neo4j graph database was used. This is the most commonly used graph DBMS. Neo4j supports Cypher query language and represents data as nodes (vertices) and relations between them (edges). Vertices and edges can be labeled. Neo4j is an open source project and, like Iguana, implemented in Java. The modified algorithm has been integrated with Neo4j using the Native Java API.


\section{Реализация библиотеки алгоритмов для полулокальных задач}\label{librarySection}
Как было отмечено в обзоре полулокальных задач,
существует необходимость в реализации большей части алгоритмов.
% В данном разделе будет описана реализация и архитектура библиотеки алгоритмов, связанных с полулокальными задачами.

% В качестве языка программирования был выбран \emph{Kotlin}

\subsection{Архитектура библиотеки}
В качестве языка программирования для написания библиотеки\footnote{https://github.com/NikitaMishin/SemiLocalLcs, дата обращения 26.05.2020} был выбран язык \emph{Kotlin}.


Код библиотеки можно  разделить на два логических модуля:
\begin{itemize}
    \item модуль \emph{semilocalProblem}  (рис.~\ref{fig:libraryProblem});
    \item модуль \emph{semilocalApplication} (рис.~\ref{fig:libraryApplication}).
\end{itemize}



\subsubsection{Модуль semilocalProblem}
В данном модуле реализована вся логика, отвечающая за задачи \emph{semi-local}.

% умножение муравья
Интерфейс \emph{IBraidMultiplication} отвечает за алгоритмы, реализующие операцию $\odot$, в частности, алгоритм \emph{муравья}.
В ходе работы алгоритма происходит последовательный доступ к перестановочным матрицам, к которым применен оператор $^{\nearrow}$ и $^{\swarrow}$\footnote{Оператор $^{\nearrow}$ ($^{\swarrow}$)  подсчитывает сумму элементов, которые лежат левее (правее) и ниже (выше) выбранного узла в матрице.}.
Для достижения быстрого времени доступа к элементам  матрицы ($O(1)$) используется
класс \emph{CountingQuery}, инкапсулирующий эту логику\footnote{Теорема 1 в \cite{tiskin2015fast}.}.
% 
Хранение перестановочных матриц реализовано с помощью хранения двух перестановок, отвечающим строчкам и столбцам матрицы.
Для реализации иной логики хранения необходимо реализовать методы абстрактного класса $AbstractPermuattionMatrix$

Классы, реализующие интерфейс \emph{ISemiLocalCombined}, инкапсулируют всю логику, связанную с задачей $semi-local$.
В частности, \emph{ImplicitSe-\\miLocalSA} инкапсулирует логику по неявному хранению матрицы, отвечающей задаче \emph{semi-local}, в виде \emph{ядра $P$} и соответствующими алгоритмами над ним.
На данный момент такими алгоритмами являются описанные в обзоре рекурсивный алгоритм с операцией $\odot$ и итеративный алгоритм распутывания кос.
Стоит отметить, что для быстрого доступа к произвольному элементу матрицы  используется интервальное двумерное дерево (\emph{range tree}), построенное над ненулевыми элементами ядра.
Это позволяет сократить объем требующейся памяти, но требует неконстантного времени доступа к элементам.
В случае если доступ последовательный, используется  \emph{CountingQuery}, который позволяет добиться константной асимптотической сложности.

\begin{figure}
    \includegraphics[height=0.72\columnwidth,angle=90]{figures/Library.png}
    \caption{Диаграмма классов UML части библиотеки, относящейся к различным задачам, основанным на  \emph{semi-local} задачах. Часть деталей и классов опущена.}\label{fig:libraryProblem}
\end{figure}

\emph{ExplicitSemiLocalSA} хранит матрицу $H_{a,b}$ в явном виде.
В данном случае нет экономии памяти, но доступ к произвольному элементу матрицы константный $O(1)$. 
Для решения задачи \emph{semi-local} используется алгоритм \emph{smawk}~\cite{aggarwal1987geometric}, отвечающий операции  $\otimes$.

% Стоит отметить, что недавние исследования~\cite{gawrychowski2020submatrix}, позволяют добиться 
% Нужно ли описать scorinSCheme



% \emph{ISemiLocalProvider}, \emph{ISemiLocalCombined} === паттерн фабричный метод
% набор классов и интерфейсов связанных с \emph{}
% Strategy --- стратегия




\subsubsection{Модуль semilocalApplication}
В данном модуле реализованы алгоритмы\footnote{Детальное описание алгоритмов и их доказательств находятся в ~\cite{tiskin2006all}}, для следующих задач:
\begin{enumerate}
    \item \emph{CompleteAMatch}
    \item \emph{Minimal-inclusive ThresholdAMatch}
    \item \emph{WindowAMatch}
    \item \emph{WindowSubstring}
    \item \emph{FragmentSubstring}
    \item \emph{BoundedLengthSmithWatermanAlignment}
\end{enumerate}

Первая задача относится к нахождению значения максимального выравнивания заданного шаблона $p$ и всех префиксов текста $t$ из всевозможных суффиксов из данного префикса:
\begin{equation}
    h[j] = \max _{i \in 0 ..j} sa(p,t[i,j]), j \in 0..|t|
\end{equation}

В рамках второй задачи ставится задача нахождения всех непересекающихся повторов шаблона $p$ в тексте $t$, чьи длины минимальны, а значение выравнивания выше заданного порога похожести $h$.

В третьей задаче необходимо найти все подстроки текста $t$ (уже могут пересекаться) длины $w$, чье выравнивание с шаблоном $p$ больше заданного порога похожести $h$.

Все эти три задачи сводятся к анализу \emph{подматрицы} задачи 
\emph{semi-local}, отвечающей \emph{srting-substring}.
И, соответственно, в зависимости от выбранного алгоритма решения задачи \emph{semi-local} асимптотика алгоритмов решения этих задач $O(n \times m \times \log n)$ и $O(v \times  m \times n)$.


\begin{figure}
    \includegraphics[width=\columnwidth]{figures/semiLocalApplication.png}
    \caption{Диаграмма классов UML части библиотеки, относящейся к \emph{semi-local} задачам. Часть деталей и классов опущена}\label{fig:libraryApplication}
\end{figure}


Задача \emph{FragmentSubstring} формулируется следующим образом: Для заданного множества интервалов (подстрок) $r$ из текста $t$ размера $m$ и текста $b$ размера $n$ необходимо вычислить \emph{semi-local} матрицу для каждого фрагмента из $r$ против $b$.
Имеет асимптотическую сложность $O(v^2 \times r \times  n \times \log m \times \log mv)$ \footnote{Существует возможность улучшить асиптотику до $O(v \times r \times  n \times \log^{2} m)$ } и $O(r \times n \times m  \times  log m)$.

Задача \emph{WindowSubstring}  является частным случаем   \emph{FragmentSubst-\\ring}, в которой размер фрагментов фиксирован.
Асимптотическая сложность решения уже будет $O(n \times m \times \log n)$ и $O(v^2 \times  m \times n)$.

Обе задачи основаны на двоичном разложении числа и предподсчете \emph{semi-local} решений, отвечающих данным разложениям.

Задача \emph{SmithWatermanAlignment} относится к локальному выравниванию.
В рамках данной задачи необходимо вычислить значение максимального локального выравнивания между $a$ и $b$, т.е найти пару подстрок, на которых достигается максимальное выравнивание.
Очень часто данная задача представляет интерес с различными ограничениями~\cite{arslan2004dynamic}.
Например, ограничения на минимальную длину подстрок.
Данное ограничение реализовано с помощью алгоритма из ~\cite{tiskin2019bounded} и инкапсулировано в соответствующем классе  \emph{BoundedLengthSmithWaterman-\\Alignment}.
Как отметил автор статьи, можно реализовать нормализованную версию данной задачи с применением \emph{BoundedLengthSmithWa-\\terman Alignment} к алгоритму из \cite{arslan2004dynamic}.

Нетрудно заметить, что часть алгоритмов в той или иной степени может быть адаптирована к задачам поиска повторов в документации.
Для задачи поиска по образцу такими кандидатами являются алгоритмы для решения задачи \emph{Minimal-inclusive ThresholdAMatch} и \emph{WindowSubstring}.
Для задачи поиска групп повторов могут быть применены  алгоритмы для \emph{WindowAMatch},
\emph{Minimal-inclusive ThresholdAMatch},\\\emph{BoundedLengthSmithWatermanAlignment} и \emph{semi-local}.

Адаптация алгоритмов к поиску повторов описана в следующем разделе.
Экспериментальная проверка асимптотики части алгоритмов, а так же их потенциальная возможность применения к большим данным даны в главе \ref{appob}.

% Соответствующая адаптация части алгоритмов описана в следующем разделе.

\section{Приложение для поиска повторов в документации ПО}\label{searchPO}
В данной главе  описано приложение для поиска повторов в документации ПО, в рамках которого будет производится экспериментальное исследование применимости алгоритмов, решающих полулокальные задачи.
Также описаны  основные технические решения и архитектура.
Описан подходы для поиска повтор.
Для каждой из \emph{задач поиска повторов} описано решение, основанное на использовании \emph{библиотеки алгоритмов} для полулокальных задач (см. главу \ref{librarySection}).

% Реализация приложения для поиска повторов в JavaDoc докумен-тации с применением алгоритмов решения полулокальных задач


\subsection{Общая архитектура приложения}
На рисунке \ref{fig:application} представлена архитектура приложения.
Оно реализовано в виде двух крупных компонент.

\begin{figure}[H]
    \includegraphics[width=\columnwidth]{figures/arhitecture.png}
    \caption{Диаграмма компонентов системы}\label{fig:application}
\end{figure}


Первая компонента (клиентская часть) --- это пользовательский интерфейс (\emph{UI}), который  отвечает за визуализацию и взаимодействие с пользователем.
Пользователь настраивает параметры поиска повторов, тип поиска, указывает файлы, в которых необходимо произвести анализ на дубликаты, или путь к проекту на \emph{Github} в интернете (рис. \ref{fig:startApp}).
Клиентская часть написана на \emph{python} и \emph{java script}.
Детальное описание клиентской части дано в секции \ref{clinet}.

Вторая компонента (серверная часть) отвечает за поиск повторов согласно заданным настройкам.
Данная часть написана на языке \emph{Kotlin}.
В секции \ref{server} детально описана функциональность данной части системы.

Взаимодействие между компонентами осуществляется посредством \emph{JSON}-формата.
Приложение реализовано в виде вэб-приложения, которое запускается в докер-контейнере, что минимизирует пользовательские требования для запуска программы\footnote{Достаточно иметь \emph{Docker} и \emph{Браузер}.}.

% Приложение реализовано в виде вэб-сервиса, который запускается в докер-контейнере на пользовательском компьюетере.



Стоит отметить, что в этой работе сделан основной акцент на поиск повторов в \emph{JavaDoc} документации в силу её актуальности для данного формата (см. главу \ref{duplicateReport}).

\begin{figure}[H]
    \includegraphics[width=\columnwidth]{figures/startApp.png}
    \caption{Интерфейс пользователя перед запуском анализатора для случая поиска групп повторов}\label{fig:startApp}
\end{figure}

\subsection{Клиентская часть}\label{clinet}
Как было отмечено выше, клиентская часть реализована в виде веб-приложения, которое написано посредством \emph{python}-фреймворка \emph{flask}\footnote{https://flask.palletsprojects.com/en/1.1.x/, микро-фреймворк для написания веб-приложений, дата обращения 26.05.2020}.
На основной странице вэб-приложения пользователь выставляет тип решаемой задачи, параметры запуска, указывает исходники и запускает вычисления.
Пользователь имеет возможность выбрать задачу \emph{"Поиск повтора по шаблону"} или \emph{"Поиск всех групп повторов"}.

Для визуализации найденных повторов в случае с  \emph{"Поиском повтора по шаблону"} результаты отображаются на странице ответа с цветовой расцветкой (см. рис. \ref{fig:pattViz}).


\begin{figure}[H]
    \includegraphics[width=\columnwidth]{figures/outputExampleAmatch.png}
    \caption{Визуализация найденных повторов для поиска по шаблону}\label{fig:pattViz}
\end{figure}

Для групп повторов механизм визуализации более комплексный (см. рис \ref{fig:groupViz}): используется три окна для интерпретации результатов.

\begin{figure}[H]
    \includegraphics[width=\columnwidth]{figures/outputGroup.png}
    \caption{Визуализация групп повторов}\label{fig:groupViz}
\end{figure}


Первое окно отвечает за отображение отношений между фрагментами, которые состоят в отношении (имеют похожие части).
Интерпретация следующая: для текущего узла "родителем" являются все те фрагменты, в которые была взята часть информации с текущего фрагмента и, возможно, видоизменена.
"Детьми" являются те фрагменты, c которых, вероятнее всего, произошло дублирование информации.

Второе окно отвечает за представление каждого повтора в виде иерархической структуры.
Представление реализовано в виде \emph{tree control} --- для каждого повтора (вершины) строится ориентированное дерево вывода из этой вершины.
Такая интерпретация показывает, откуда вероятнее всего "произошел повтор", т.е откуда произошло дублирование данных.

Третье окно отвечает за визуализацию соответствующей группы повторов в  виде ориентированного графа.
Это позволяет увидеть структуру найденной группы. Все три окна синхронизированы между собой.

Описание используемых алгоритмов для нахождения повторов даны в секциях \ref{fics} и \ref{grouppa}  соответственно.    


% Рассмотрим пример работы  на рис \ref{TODO}. TODO описать пример. 

\subsection{Серверная часть}\label{server}
Как было указано выше, эта часть приложения отвечает за поиск повторов в документации.

На рис. \ref{fig:application} выделены компоненты серверной части (\emph{Kotlin application}).
Общий подход (\emph{pipeline}) поиска повторов заключается в следующем.

Сперва происходит синтаксический анализ с целью нахождения тех фрагментов, которые будут анализироваться согласно выбранным параметрам, заданными пользователем.
В данной работе анализируются \emph{JavaDoc} документация, а именно \emph{JavaDoc}-комментарии методов, классов и интерфейсов.
Это осуществлено с использованием библиотеки \emph{JavaParser}\footnote{https://javaparser.org/, дата обращения 26.05.2020}.
Для анализа иного вида документации (например, обычный текст, как в случае с поиском по образцу) достаточно в модуле \emph{Parser} реализовать необходимые интерфейсы.
Далее, комментарии обрабатываются различными фильтрами --- происходит токенизация, лемматизация, убираются стоп-слова и пр.


Данная обработка происходит с частичным использованием функциональности стэнфордского  фреймворка \emph{core nlp}\footnote{https://stanfordnlp.github.io/CoreNLP/, дата обращения 26.05.2020} для обработки естественного языка.

После этапа предобработки фрагментов происходит конвертация \\слов  в промежуточное представление в виде чисел с целью экономии памяти и ускорения работы алгоритмов.

Затем  происходит запуск соответствующих алгоритмов поиска, о которых пойдет речь дальше.


\subsection{Алгоритмы для решения задачи поиска повторов}\label{fics}
% fix

В данной секции будут описаны алгоритмы для решения задачи поиска по образцу, согласующиеся с определенной моделью в главе \ref{Model}.
Алгоритмы основаны на использовании \emph{библиотеки алгоритмов} (см. главу \ref{librarySection}).

\subsubsection{Улучшенный алгоритм интерактивного поиска}
В данной секции описана улучшенная версия алгоритма из \cite{luciv2019interactive}. Псевдокод алгоритма представлен ниже (алгоритм  \ref{alg:patternMathing1}).
\begin{algorithm}[H]
\caption{Нечеткий поиск по шаблону с использованием semi-local}\label{alg:patternMathing1}
Вход: шаблона поиска $p$, текст $t$, пороговое значение похожести $k$\\
Выход: множество непересекающихся повторов шаблона $p$\\
Комментарии:
\begin{equation}
    k_{di}=|p|*(\frac{1}{k}+1)(1-k^2)
\end{equation}
Псевдокод:
\begin{algorithmic}[1]
\State $W = semilocalSa(p,t)$
\State $W_2 = \emptyset$
\For{$w \in W$}
   \If{ $sa(p,w) \geq -k_{di}$}
   \State \emph{continue}
   \EndIf
   \State $maximums = FindMaxForColumnsBySmawk(w)$
   \State $max = FindMaxWithLenghtConstraint(maximums)$
   \If{$max \geq -k_{di}$}\Comment{В исходном был минимум, поэтому минус} 
    \State add substring associated with max to $W_{2}$ 
    \EndIf
\EndFor
\State $W_3 = UNIQUE(W_2)$\Comment{3 фаза без изменений}
\For{$w \in W_3$}
\If{$\exists w^{'} \in W_3:w \subset w^{'} $}
\State $remove$ $w$ $from$ $W_3$
\EndIf
\EndFor
\State $return$ $W_3$

\end{algorithmic}
\end{algorithm}

В строке $1$ вычисляется решение задачи \emph{semi-local sa}, в частности подзадачи \emph{string-substring}.
В строках ${3-12}$ внутри каждого окна размера $|w_{s}| \approx |p|$ сперва проверяется, что значение выравнивания шаблона $p$ и окна $w$ больше заданного порога похожести $k_{di}$, а затем внутри окна находится такая подстрока, что её выравнивания максимально среди всех друг подстрок данного окна. При одинаковом значении выравнивания будет выбираться наиболее длинная строка.
Строки ${13-18}$ отвечают фильтрации, в рамках которой происходит удаление одинаковых и пересекающихся повторов, в результате чего остаются только непересекающиеся повторы.

\paragraph*{Корректность улучшения}\mbox{}

Нетрудно заметить, что данная версия имеет лучшую асимптотическую сложность, чем исходный алгоритм. Более того, все свойства алгоритма сохраняются.

Исходный алгоритм разбит на 3 фазы: 'сканирование', 'усушка' и 'фильтрация'.

На первой фазе исходный текст $t$ анализируется скользящим окном размером $w_{s} = \frac{|p|}{k}$, где $k \in [\frac{1}{\sqrt{3}},1]$ параметр алгоритма, с шагом в один символ, а именно вычисляется редакционного расстояние\footnote{Метрика, минимальное количество операций вставки, удаления и замены одного символа на другой для превращения одной строки в другую.} между каждым окном и заданным шаблоном $p$.
Асимптотическая сложность данного шага $O(|p|^2 \times |t|)$ согласно \cite{luciv2019interactive}.

Заметим, что редакционное расстояние может быть выражено через выравнивание последовательностей.
Конкретнее, редакционное расстояние для данного случая выражается через следующую схему оценки:
\begin{equation}\label{weightAppr}
    (w_{+},w_{0},w_{-}) = (0,-2,-1)
\end{equation}
Соответственно, редакционное расстояние можно заменить на выравнивание последовательностей с весами $(0,-2,-1)$ и искать максимальное выравнивание без потери свойств алгоритма в силу эквивалентности двух задач.
Из этого следует, что можно применить алгоритмы для решения \emph{semi-local sa}.
В силу  формулы (\ref{weightNormalization}), значение нормализованной схемы будет следующее:
\begin{equation}
    (0, -2, -1) \rightarrow (1,\frac{\mu=0}{v=1}, 0)
\end{equation}
Таким образом, используя нормализацию, задачу можно свести к \emph{semi-local lcs}.
И, следовательно, асимптотическая сложность первой фазы алгоритма  из \cite{luciv2019interactive} может быть улучшена. Тогда её асимптотическая сложность будет $O(|t| \times |p|)$ вместо $O(|t| \times |p|^2)$.
Данная стадия эмулируется в строке $1$.

Во второй фазе происходит так называемая 'усушка' --- для каждого окна происходит вычисление минимальной подстроки, на которой достигается минимальное значение редакционного расстояния (в случае выравнивания последовательностей это относится к максимальному значению).
При равенстве расстояний выбирается подстрока наибольшая по длине.
Асимптотическая сложность данной фазы выражается как $O(|p|^4$).

Для улучшения данной фазы применяется следующее.
Матрица решений $H_{p,t}$ \emph{semi-local} содержит подматрицу \emph{string-substring}, которая содержит значения выравниваний шаблона $p$ со всеми подстроками текста $t$. 
Как было отмечено выше, эта матрица является анти-матрицей Монжа, следовательно, в ней можно быстро искать максимум в столбцах (строках) через алгоритм \emph{smawk}, имеющий асимптотическую сложность $O($\emph{размер матрицы}$\times$\emph{время доступа к элементу матрицы} = $\gamma$\footnote{Далее будет обозначать асимптотическую сложность доступа к произвольному элементу матрицы через символ $\gamma$}$)$.
Заметим, что этот алгоритм устойчив, в том плане, что он выдает первую позицию, на которой достигается максимум.
Например, если для текущего столбца $j$  максимум достигает в позициях $i$ и $i^{'}$, $i<i^{'}<=j$, то в результате работы \emph{smawk}, для столбца $j$ алгоритм выдаст $i$, что соответствует тому, что при равенстве значений будет выбираться наиболее длинная подстрока.
Имея максимумы для каждого столбца (каждого суффикса префикса), можно найти все максимумы и среди них выбрать подстроку наибольшую по длине.

Для нахождения максимума в столбце воспользуемся следующими соображениями.
\begin{itemize}
    \item  Если $H_{p,t}$ является \emph{анти-матрицей Монжа}, то $-H_{p,t}$ является \emph{матицей Монжа}.
    \item В результате транспонирования, матрица не перестает быть \emph{(анти)-матрицей Монжа}.
\end{itemize}
Иными словами, нахождение минимума в  строке в $-H_{p,t}^{T}$ будет соответствовать нахождению максимума в столбце в матрице $H_{p,t}$. 
В улучшенном алгоритме это отвечает строкам $7-10$.

Таким образом, асимптотическая сложность для каждого окна будет соответственно равна $O(|w_{s}| \times \gamma) $.
В худшем случае, таких окон будет $O(|t|)$.
Следовательно, вторая фаза алгоритма будет иметь асимптотическую сложность  $O(|w_{s}| \times |t| \times \gamma )$.
Учитывая то, что $k \in [\frac{1}{\sqrt{3}},1]$, то $|w_{s}| \approx |p|$ и $O(|w_{s}| \times |t| \times \gamma)=O(|p| \times |t| \times \gamma)$.
Значит, асимптотическая сложность второй фазы $O(|p| \times |t| \times \gamma )$, все свойства алгоритма сохранены.

Третья фаза, отвечающая за фильтрацию фрагментов, остается без изменений. Ее асимптотическая сложность $O(|p| \times \log |p|)$ согласно \cite{luciv2019interactive}.

Соответственно, алгоритм сохранит все свои свойства и будет уже иметь асимптотику $O(|p| \times |t|)$\footnote{В то время как исходный алгоритм оценивается как $\max (O(|p|^2 \times |t|, O(|p|^4)$, что при $|p| \approx |t|$ дает 4 степень.}.
Заметим, что алгоритм имеет оптимальную асимптотику при явном хранении матрицы $H_{p,t}$ (время доступа к произвольному элементу константное) согласно \cite{abboud2015tight}.
% Как отмечено выше, псевдокод алгоритма представлен на  \ref{alg:patternMathing1}.

\subsubsection{Алгоритм нечеткого поиска шаблона с использованием ThresholdAMatch}
Более простое решение относится к алгоритму \ref{alg:patternMathing2}, реализация которого уже содержится в \emph{библиотеке алгоритмов}. 
Он позволяет найти все непересекающиеся повторы шаблона $p$ в тексте $t$.

\begin{algorithm}[h]
\caption{Нечеткий поиск по шаблону с использованием Min-inclusive ThresholdAMatch}\label{alg:patternMathing2}
Вход: шаблона поиска $p$, текст $t$, пороговое значение похожести $h$\\
Выход: множество непересекающихся повторов шаблона $p$\\
Комментарии: в реализации $h$ высчитывается исходя из схемы оценки и процента похожести, выраженного через число из отрезка $[0,1]$\\
Замечание: При использовании интервального дерева данный алгоритм можно адаптировать таким образом, что на каждом шаге будет выбираться максимальный интервал из текста.
Тогда асимптотика возрастет в $O(log|t|)$ раз\\
Псевдокод:
\begin{algorithmic}[1]
\State $maxSuffixes= CompleteAMatch(p,t)$
\State $reverse(maxSuffixes)$
\State $result = \emptyset$
\For{$(i,j,score) \in maxSuffixes$}
   \If{$score \geq h \& j \leq res.last().i $} 
    \State $res.add((i,j,score))$ 
    \EndIf
\EndFor
\State $return$ $result$

\end{algorithmic}
\end{algorithm}

Сперва решается задача \emph{CompleteAMatch}, в рамках которой для каждого столбца $j$ подматрицы \emph{string-substring} находится максимальное значение и позиция $(i,j')$, в которой оно  достигается, т.е находится суффикс префикса, который больше всего похож на шаблон.
Далее, над полученным результатом производится фильтрация, начиная с конца.
В результате чего остаются только непересекающиеся повторы минимальной длины, которые больше заданного порога похожести.
Асимптотическая сложность данного решения зависит от выбранного алгоритма  решения \emph{semi-local}.
Соответственно, $O(|t| \times |p| \times \log |t|)$, $O(|t| \times |p| \times v^2)$ или $O(|t| \times |p| \times v)$.

Данный алгоритм рассматривается как альтернатива алгоритму \ref{alg:patternMathing1}.

\subsubsection{Алгоритм нечеткого поиска шаблона с использованием Разреза}

Еще одним решением на основе \emph{semi-local} является следующий алгоритм.
Во-первых, задачу поиска по шаблону можно сформулировать с иной точки зрения: 
необходимо найти все максимальные по выравниванию непересекающиеся повторы шаблона $p$ в тексте $t$, т.е получить такую цепочку непересекающихся интервалов $(i_1,j_1),...,(i_n,j_n)$, что на $(i_k,j_k)$ достигается максимальная похожесть на еще непокрытой найденными интервалами части текста $t$.
В рамках задач \emph{semi-local} это означает разбитие матрицы \emph{string-substring} на непересекающиеся подматрицы, с учетом максимумов в подматрицах.
Последнее относится к быстрому поиску максимума в матрице (\emph{range maximum query}).
В силу того, что матрица \emph{string-substring} является матрицей Монжа, можно применить результат из статьи  \cite{gawrychowski2020submatrix}. 
Для этого необходимо реализовать структуру данных, асимптотическая сложность построения которой равна $O(|t|\times \log |t|)$ (размер структуры выражается как $O(|t|)$), которая позволяет делать запросы на поиск максимума в произвольной подматрице, имеющие асимптотическую сложность $O(\log \log|t|)$.
Учитывая, что непересекающихся повторов в тексте $t$ может быть в худшем случае $|t|$ штук,
для их нахождения необходимо осуществить $|t|$ запросов на поиск максимуму.
Следовательно, конечная асимптотика алгоритма $O(|t| \times \log \log t) +O($\emph{сложность подсчета 
semi-local}$)=O($\emph{сложность подсчета 
semi-local}$)$, так как $\log \log |t| \leq |p|$ для достаточно больших значений $|t|$. 
Псевдокод алгоритма представлен на листинге \ref{alg:patternMathing3}.

\begin{algorithm}[h]
\caption{Нечеткий поиск по шаблону с использованием maxRangeQuery}\label{alg:patternMathing3}
Вход: шаблон поиска $p$, текст $t$, пороговое значение похожести $h$\\
Выход: множество непересекающихся повторов шаблона $p$\\
Комментарии: в реализации $h$ высчитывается исходя из схемы оценки и процента похожести, выраженного через число из отрезка $[0,1]$\\
Псевдокод:
\begin{algorithmic}[1]
\State $S = SolveSemiLocalSA(p,t)$
\State $W = BuildStructForRangeQuery(S)$
\State $IntervalsToSearch = \emptyset $
\State$IntervalsToSearch.add((0,|t|))$
\State $result = \emptyset$
\While{$IntervalsToSearch.isNotEmpty()$}
    \State $(i,j) = IntervalsToSearch.pop()$
    \State $score,i^{'},j^{'} = W.query(i,j)$
    \If{$score \geq h $} 
    \State $result.add(( i^{'},j^{'},score ))$
    \State $IntervalsToSearch.add(i,i^{'})$        \State $IntervalsToSearch.add(j^{'},j)$
    \EndIf
\EndWhile
\State $return$ $result$

\end{algorithmic}
\end{algorithm}


% Для апробации применимости \emph{semi-local} было решено реализовать алгоритм \ref{alg:patternMathing2} т.к
% для \ref{alg:patternMathing1} была произведена апробация в статье \cite{luciv2019interactive}, а \ref{alg:patternMathing3} на данный момент не имеет доказанных теоретических свойств и требует использования сложной структуры данных из \cite{gawrychowski2020submatrix}.

% Для задачи поиска по образцу постпроцессинг не нужен.

\subsection{Алгоритмы для решения задачи поиска групп повторов}\label{grouppa}
В данной главе описаны алгоритмы решения задачи поиска групп повторов на основе использования \emph{библиотеки алгоритмов} и применения графовых алгоритмов.

Согласно определенной в секции \ref{Model} модели, для решения задачи \emph{поиска групп повторов}  необходимо
задать функцию похожести $g$ и выбрать предикат $\gamma$.
Заметим, что для рассматриваемого случая, текстовыми фрагментами, в которых ищутся повторы, являются цельные \emph{JavaDoc}-комментарии.
Соответственно, в данной работе в отношении поиска групп повторов \emph{повторами} будут служить семантически замкнутые куски текста, как и в \cite{soto2015similarity}\footnote{В \cite{soto2015similarity} это были топики текста в \emph{Dita} документации.}, т.е \emph{JavaDoc} комментарии.

Соответственно, весь набор комментариев образует граф.
Он может быть как ориентированным, так и неориентированным.
Это зависит  от того, является ли $g$ симметричной по отношению к своим аргументам.
Таким образом, в полученном графе можно выделить группы повторов согласно предикату $\gamma$.
На \ref{alg:groupDuplicate} представлен псевдокод алгоритма.
Отметим, что асимптотика алгоритма выражается, как $\max (O(|t|^2*g), O(s))$.

Функция $g$ может быть определена через локальное, полулокальное и глобальное выравнивание, соответственно.
В данной работе в качестве $g$ выбраны следующие алгоритмы из \emph{библиотеки алгоритмов}.
\begin{itemize}
    \item \emph{BoundedLengthSmithWaterman} --- локальное выравнивание.
    % , симметричная функция.
    \item \emph{Semi-local sa} --- полулокальное выравнивание.
    % не симметричная функция.
    % \item \emph{ThrehsoldAMatch} --- поиск по шаблону. 
\end{itemize}

\begin{algorithm}[h]
\caption{Алгоритм поиска групп повторов для JavaDoc-комментариев}\label{alg:groupDuplicate}
Вход: набор комментариев $t_{i}$, функция $g$, которая меряет похожесть между двумя комментариями, функция $s$, которая согласно выбранному предикату $\gamma$ строит группы, пороговое значение похожести $h$\\
Выход: группы непересекающихся повторов\\
Комментарии: в реализации $h$ высчитывается исходя из схемы оценки и процента похожести, выраженного через число из отрезка $[0,1]$\\
Псевдокод:
\begin{algorithmic}[1]
\State $graph = Graph(vertices=t)$
\For{$t_{i} \in t $}
\For{$t_{j} \in t,t_{i} \neq t_{j} $}
\If{$g(t_{i},t_{j})\geq h $}
\State $addEdge(t_{i},t_{j},g(t_{i},t_{j}))$
\EndIf
\If{$g(t_{j},t_{i}) \geq h$}
\State $addEdge(t_{j},t_{i},g(t_{j},t_{i}))$
\EndIf
\EndFor
\EndFor
\State $groups = s(graph)$
\State $return$ $groups$
\end{algorithmic}
\end{algorithm}

% На \ref{1,2,3} представлены алгоритмы, определяющие функцию $s$

Следующие эвристические соображения помогают определить функции $s$, которые подходят для нахождения групп.

Во-первых, в силу того, что мы рассматриваем граф, естественным образом задача сводится к  кластеризации графа/выделению компонент (сильной) связности.

Во-вторых, ориентированное ребро $a \xrightarrow{g(a,b)} b$ в графе можно естественным образом интерпретировать так: часть текста из $b$ была скопирована в фрагмент $a$ или текст $b$ был скопирован, и в новом фрагменте произведена модификация этой копии и получено $a$.
При существовании обратного ребра будем считать, что при условии $g(a,b)\geq g(b,a)$, $a$ является потомком $b$ (помним, что в общем случае $g(a,b)\neq g(b,a)$ и наоборот.

В-третьих, очень часто бывает, что повторы практически не отличаются друг от друга или же в точности совпадают друг с другом.
Такие повторы хочется различать от обычных.
Если рассматривать граф, такое состояние для части вершины выражается через термин \emph{клика}\footnote{Полный граф на заданных вершинах.} и относится к задаче поиска клики.
Соответственно, новый граф, в котором присутствуют клики строится, из исходного обновлением весов тех ребер, которые входят в клики или находятся внутри клик.


В-четвертых, ассоциированный с группой ориентированный граф должен быть деревом.
Эта эвристика основана на том, что вершина не может быть потомком сама себе (наличие циклов) и не может иметь двух одинаковых потомков (проблема множественного наследования).
Соответственно, граф является деревом.  

Исходя из описанных выше эвристик, были разработаны алгоритмы \ref{alg:cluster1}, \ref{alg:clusterMcl}.
Также применен алгоритм из статьи \cite{tofigh2009optimum}.

% 
В \ref{alg:cluster1} используется идея иерархической кластеризации с тем изменением, что добавляется новый вид вершины, который олицетворяет клики.
Как известно, поиск клики --- это \emph{NP}-полная задача.
Поэтому в данном алгоритме произведена аппроксимация поиска клик:
листовая вершина принадлежит кластерному узлу, если её  расстояние до клики больше заданного порога похожести для клик.
% \red{Для подсчета расстояний будет использоваться \emph{минимальное расстояние между вершинами}.} 
% Существует разные варианты подсчета расстояний, о них подробно  будет описано в главе TODOапробации.
В ходе алгоритма в цикле \emph{while} происходит нахождение двух ближайших вершин согласно выбранной метрике, их объединение согласно правилам и пересчет матрицы расстояний, которая отвечает уже новому графу.
В общем случае\footnote{Некоторые метрики позволяют считать асимптотически быстрее.} пересчет матрицы требует $O(n^2)$ времени, где $n$ --- количество вершин.
В худшем случае  цикл \emph{while} будет исполняться $O(n)$ раз,
тогда асимптотика алгоритма составит $O(n^3)$.
% песос пример нужен лучше напиши епта

% иерархичпская кластеризация
\begin{algorithm}[h]
\caption{Алгоритм выделения групп на основе Иерархической кластеризации}\label{alg:cluster1}
Вход: граф $G$ с матрицей расстояний, функция  $f$, которая считает дистанцию между вершинами, пороговое значение похожести $h_{clique}$, при котором вершины образуют очередной уровень в иерархии, $h_{group}$ --- пороговое значение похожести\\
Выход: иерархические группы повторов \\
Псевдокод:
\begin{algorithmic}[1]
\State $roots = G.vertices()$
\While{$roots.isNotEmpty()$}
\State $(from, to,score) = closestVertices(root)$
\State $newVertex = switch \{$
\State $score\geq h_{clique} , from,to \in Leaf \rightarrow Clique(from,to) $
\State $score\geq h_{clique} , to \in Clique,from \in Leaf \rightarrow to.add(from);to $
\State $score\geq h_{clique} , from,to \in Clique \rightarrow from.addAll(to);from$
\State $score\geq h_{group}, \rightarrow ClusterNode(from,to) $
\State $else \rightarrow break$ 
\State $\}$
\State $G.recalcualateDistance()$
\State $roots.remove(from)$
\State $roots.remove(to)$
\State $roots.add(newVertex)$
\EndWhile
\State $return$ $roots$
\end{algorithmic}
\end{algorithm}

В алгоритме \ref{alg:clusterMcl} использована идея кластеризации на основе марковских моделей~\cite{dongen2000cluster} и дальнейшего построения минимальных остовных деревьев внутри каждого кластера.
 Асимптотическая сложность первого шага реализации в данной работе $O(n^3)$ в худшем случае.
 Нахождение минимального остовного дерева реализовано с помощью алгоритма Крускала с использованием системы непересекающихся множеств. Сложность второго шаге $O(n^2 \times \log n)$.
 Соответственно, общая сложность $O(n^3)$.

% mcl clustering
\begin{algorithm}
\caption{Алгоритм выделения групп на основе Марковских моделей}\label{alg:clusterMcl}
Вход: граф $G$ с матрицей расстояний\\
Выход: группы повторов с структурой группы в виде дерева\\
Псевдокод:
\begin{algorithmic}[1]
\State $trees = \emptyset$
\State $ clusters = mclClustering(G)$
\For{$cluster \in clusters$}
\State $tree =  BuildMaximumSpanningTree()$
\State $trees.add(tree)$
\EndFor
\State
\State $return$ $trees$
\end{algorithmic}
\end{algorithm}


Алгоритм \cite{tofigh2009optimum}  решает задачу построения такого ориентированного подграфа $G_{branch}$ из исходного $G$, что:
\begin{itemize}
    \item В нем нет циклов
    \item Ни в какую вершину не входит больше одного ребра
\end{itemize}
Причем среди всех таких подграфов он оптимален:
\begin{equation}
\sum_{w \in G_{branch}} w \geq \sum_{w^{'} \in G_{branch^{'}}} w^{'}
% \forall G_{branch^{'}} 
\end{equation}
Это соотносится с последней эвристикой о том, что граф должен быть деревом.
Алгоритм из \cite{tofigh2009optimum} обладает асимптотической сложностью $O(n^2)$.

% вероятностная кластеризация
% разбить компоненты сильной связности-> построить ориентированное дерево
%  

\vspace{10 mm}
Результаты применимости описанных алгоритмов из  данной главы к поиску повторов в документации ПО представлены в разделе \ref{appob}.
\section{Evaluation}

This section describes the methodology and answers the following research questions.

\begin{enumerate}
    \item Does fusion via distillation give any benefits at the software and hardware levels?
    \item What are the properties of the generated hardware?
    \item Does the generated hardware outperform software implementations?
\end{enumerate}

\subsection{Methodology}

Our focus is on creating a basis for future research and experiments, thus we make our experiments as much reproducible as possible\footnote{\url{https://github.com/sedwards-lab/fhw/tree/sparse-linear-algebra-distillation/examples/QTreeBenchmarks/diploma/verilog-bool-no-nnz-inlined} (online; accessed:
2022-06-07) Here one could find all the results. For each benchmark all statistics are specified: matrix names, their sizes, collected metrics for both hardware and software benchmarks.}. We benchmarked the following list of chained functions. The choice is prompted by the current state of the distiller: at the moment, it does not successfully distill matrix multiplication. However, the functions are still practical enough, for example, chained addition could be seen in Luby's maximal independent set algorithm and clearly describe the applicability of the proposed approach.

\begin{itemize}
    \item \mintinline{Haskell}{mAdd (==False) (||) (mAdd (==False) (||) m1 m2) m3}
    \item \mintinline{Haskell}{mask (mAdd (== False) (||) m2 m3) (m1)}
    \item \mintinline{Haskell}{map (==Zero) (to_nat) (mAdd (==False) (||) m1 m2}
    \item \mintinline{Haskell}{map (==Zero) (to_nat) (kron (==False) (&&) m1 m2}
\end{itemize}

Above, \mintinline{Haskell}{Zero} and \texttt{to\_nat} are corresponding definitions for Peano arithmetics, since the \texttt{.pot} language does not have any primitives. For the same reason, we operated with boolean matrices. Such functions could be abstracted with free variables and then instantiated in the emitted Haskell code. However, to get maximum from distillation, we provided all the information about the functions. 

For these functions, we compared the execution time of distilled and not distilled hardware generated circuits, execution time of original and distilled Haskell code and reference \textit{Suite Sparse}\footnote{\url{https://github.com/DrTimothyAldenDavis/GraphBLAS} (online; accessed:
2022-06-07), Suite Sparse library sources.}\textsuperscript{,}\footnote{The library also uses different variations of coordinate formats (opaque to the user) and not a quadtree representation.} variants of these functions in C\texttt{++}. Note that SuiteSparse does not support recursive data types, thus only the first two function chains were implemented in SuiteSparse (since Peano number is essentially a linked list). We did not replace Peano numbers with integers, so our experiments could be interpreted easier. For hardware experiments we collected execution time and the number of memory writes and reads, to access how well fusion is performed. For software experiments we only measured the execution time. Also note that we measured only the time, required to execute the lines above, not including any IO, required to get and evaluate function arguments. But in hardware benchmarks we measured the time required to pass arguments into the circuit's memory, because such IO is inevitable. It is tricky to make such measures in Haskell due to laziness, thus the programs were compiled with \texttt{--fno-full-laziness} to turn off memoization. Also all the arguments were forced to normal form via \texttt{force} and \texttt{evaluate}. Haskell programs were compiled\footnote{GHC 8.10.4.} with \texttt{-O2 --fno-full-laziness} and Suite Sparse was compiled with default flags and linked as a shared library to C\texttt{++} code.

We took matrices from SuiteSparse matrix collection with sizes ranging from \texttt{64x64} to \texttt{512x512}. We limited ourselves with such sizes due to the fact that this is the maximum sizes that fit into \texttt{bram} with $2^{16}$ address space. Such number of \texttt{bram} blocks is available only on really expensive FPGA boards, thus in practice sizes would be smaller to achieve better utilization. Once again, it models the situation when data fits into the cache, since \texttt{bram} in our circuits will operate as a cache in real application.

\subsection{Experiments}

Table~\ref{tab:bench_results} shows the results of all execution time benchmarks. To evaluate execution time for hardware simulation, implementation stage was performed to assess the maximum frequency of FPGA device used for synthesis and implementation, and the number of execution cycles was multiplied by the number of nanoseconds a clock cycle takes. The frequencies were equal within the same benchamark set, i.e., frequency was not affected by distillation. We used \texttt{xcu250figd2104-2L} device\footnote{\url{https://www.xilinx.com/products/boards-and-kits/alveo/u250.html}  (online; accessed:
2022-06-07)} for synthesis and implementation stages. It is not really a casual and affordable chip, but it contains enough \texttt{bram} for our evaluation to see scalability. In the table a median across several benchmarks is shown. 

As it could be seen, distillation steadily increases performance: up to 2x speedup for hardware simulation and up to 3x for software benchmarks. The results are maintained within the borders of the corresponding confidence interval and the borders are not shown for brevity. Hardware speedup is lower due to the different execution semantics, dataflow is not reduction-based and distillation is a reduction-based transformation. Note that generated hardware appears to be less performant than both Haskell and C\texttt{++}, which a bit contradicts the results from~\cite{oldfhw}. For hardware benchmarks \texttt{time (IO)} shows the execution time including the time needed to transfer the data though the arguments, \texttt{time (no IO)} does not include it in its turn. It could be seen that not all the benchmarks are computationally extensive enough to cover memory transferring costs, but for more complex examples the ratio would be better. Since we basically transfer the matrices node by node from a file in the testbench, we have probably the lowest possible latency, and in practice it would be higher if reading from DDR, however the bandwidth could be increased. Noticeably, running times for \texttt{mMaskAdd} for C\texttt{++} and distilled Haskell are similar, which shows that fusion really provides some extra performance: SuiteSparse at the moment does not implement any fusion.

Table~\ref{tab:mem_results} summarizes the ratios between distilled and not distilled hardware circuits memory reads and writes. Since in general case distillation removes extra pattern matching, essentially it saves memory reads and writes. The eventual number of memory reads and writes is implementation dependent, thus the table shows what share of speedup is prompted by saving memory operations. Distillation successfully reduces the number of memory accesses, about 15\% on average. \texttt{mMapKron} has a bit higher ratio due to the fact that \texttt{Nat} numbers require additional memory accesses, since the type is recursive. It could also be seen that a major part of speedups is attributed to saved memory accesses. 

Finally, table~\ref{tab:resource_util} shows device resources utilization ratios between distilled and not distilled hardware circuits and frequencies. Columns are device primitives: registers, lookup tables, \texttt{bram} blocks or multiplexers. Utilization for both types of circuits is below 1\% of available resources on the device, except for the memory. Memory blocks utilization is about 30\% (since we choose larger \texttt{brams} to store larger matrices). Apart from that, distilled circuits could have both higher and lower utilization. Since the hardware generation is primarily syntax-directed it follows from the distilled program structure. For example, distillation might glue two recursive functions into one (in that case, memory utilization would be lower, because each cluster of mutually recursive functions possesses its own heap) or make more recursive functions than in the original program. The frequencies are the same, however, they possibly could be made better with more intelligent buffer allocation.

\subsection{Discussion}
Answering the research questions above.

\begin{enumerate}
    \item Fusion gives significant benefits, however at the hardware level the benefits are a bit smaller since hardware semantics is not reduction based. The benefits at the hardware level are mostly determined by the reduced number of memory accesses (each access takes 2 clock cycles). Notably, distilled Haskell implementation of \texttt{mMaskAdd} has similar performance with C\texttt{++}. 
    \item Device utilization is low, but such circuits could be copied on the same device to provide better utilization and higher parallelism. Resource utilization might be both better and worse after distillation, depending on the transformed program itself since translation is syntax-directed. Frequency could be increased by more intelligent buffering strategy.
    \item Although we use low-latency design with \texttt{bram}s that take 2 clock cycles per request and transfer matrices from files, which does not have any latency in simulation, we get slower execution time than Haskell and C\texttt{++} counterparts. It could be partly due to excessive buffering performed by FHW at the moment. Also there is no pipelining for recursive calls, i.e. only one set
of function argument tokens are allowed to enter a tail-recursive function call until a result is finally generated. Further CPS transformation hinders parallelization, which could be made more explicit with SSA. Some other optimizations exist that may significantly influence the performance. Also, since device utilization is about 1\%, such circuits could be copied on one device and provide more parallelism. A more detailed discussion could be found at~\cite{Edwards2019FHWP}.
\end{enumerate}

Distillation clearly showed its applicability to optimization of sparse linear algebra routines and notably it still could be combined with other techniques, like rewrite rules to achieve better results. High-level synthesis has a room for improvements by increasing pipelining, parallelism and frequency and the generated hardware could be improved from usability perspective: a support for arbitrary sized matrices is desirable. Thus we will focus on these directions. Probably a better solution would be to embed \texttt{.pot} language into e.g. Haskell to leverage its type system (to be able to use some rewrite rules as well), and add support for primitive types and parallel primitives to be able to conduct a more scalable comparison with SuiteSparse (since SuiteSparse is multithreaded). For such embedding different execution models could be implemented, including hardware synthesis, for which SSA form of GRIN looks promising, as well as extra optimizations shipped with GRIN. For hardware synthesis, an interesting direction is achieving predictable results in hardware from certain modifications in software. This property partly holds for the current approach, since the translation is syntax- directed. More information on this could be found at~\cite{predict}.

\pagebreak

\begin{table}[t]
\scriptsize
\centering
\caption*{mAddAdd}
\begin{tabular}{|c|c|c|c|c|c|c|c|c|c|} 
\hline
\rowcolor{LightBlue}
\multicolumn{3}{|c|}{Matrices dimensions} & Haskell & Haskell (distilled) & \multicolumn{2}{c|}{FHW} & \multicolumn{2}{c|}{FHW (distilled)} & {C\texttt{++}}\\
% \rowcolor{LightBlue}
\hline
m1 & m2 & m3 & time & time & time (no IO) & time (IO) & time (no IO) & time (IO) & time \\ 
\hline
64 & 64 & 64 & 29 us & 20 us & 76 us & 170 us & 64 us & 158 us & 14 us\\ 
128 & 128 & 128 & 94 & 79 & 146 & 476 & 134 & 469 & 30 \\
256 & 256 & 256 & 123 & 103 & 202 &  681 & 168 & 662 & 44\\
512 & 512 & 512 & 219 & 143 & 474 & 1192 & 375 & 1093 & 49\\
\hline
\end{tabular}

\caption*{mMaskAdd}
\begin{tabular}{|c|c|c|c|c|c|c|c|c|c|} 
\hline
\rowcolor{LightBlue}
\multicolumn{3}{|c|}{Matrices dimensions} & Haskell & Haskell (distilled) & \multicolumn{2}{c|}{FHW} & \multicolumn{2}{c|}{FHW (distilled)} & {C\texttt{++}}\\
% \rowcolor{LightBlue}
\hline
m1 & m2 & m3 & time & time & time (no IO) & time (IO) & time (no IO) & time (IO) & time \\ 
\hline
64 & 64 & 64 & 10 us & 7 us & 64 us & 133 us & 46 us & 111 us & 18 us\\ 
128 & 128 & 128 & 38 & 30 & 118 & 322 & 75 & 292 & 33 \\
256 & 256 & 256 & 48 & 42 & 168 &  498 & 104 & 456 & 46\\
512 & 512 & 512 & 126 & 76 & 400 & 762 & 300 & 729 & 65\\
\hline
\end{tabular}

\caption*{mMapAdd}
\begin{tabular}{|c|c|c|c|c|c|c|c|c|c|} 
\hline
\rowcolor{LightBlue}
\multicolumn{3}{|c|}{Matrices dimensions} & Haskell & Haskell (distilled) & \multicolumn{2}{c|}{FHW} & \multicolumn{2}{c|}{FHW (distilled)} & {C\texttt{++}}\\
% \rowcolor{LightBlue}
\hline
m1 & m2 & m3 & time & time & time (no IO) & time (IO) & time (no IO) & time (IO) & time \\ 
\hline
64 & 64 & --- & 45 us & 37 us & 189 us & 253 us & 137 us & 202 us & ---\\ 
128 & 128 & --- & 162 & 105 & 524 & 685 & 397 & 579 & --- \\
256 & 256 & --- & 312 & 216 & 1047 &  1360 & 680 & 986 & ---\\
512 & 512 & --- & 436 & 273 & 1346 & 1776 & 900 & 1330 & ---\\
\hline
\end{tabular}

\caption*{mMapKron}
\begin{tabular}{|c|c|c|c|c|c|c|c|c|c|} 
\hline
\rowcolor{LightBlue}
\multicolumn{3}{|c|}{Matrices dimensions} & Haskell & Haskell (distilled) & \multicolumn{2}{c|}{FHW} & \multicolumn{2}{c|}{FHW (distilled)} & {C\texttt{++}}\\
% \rowcolor{LightBlue}
\hline
m1 & m2 & m3 & time & time & time (no IO) & time (IO) & time (no IO) & time (IO) & time \\ 
\hline
2 & 64 & --- & 64 us & 36 us & 212 us & 242 us & 94 us & 125 us & ---\\ 
2 & 128 & --- & 137 & 68 & 434 & 502 & 199 & 266 & --- \\
2 & 256 & --- & 364 & 126 & 1004 &  1188 & 449 & 636 & ---\\
4 & 128 & --- & 302 & 94 & 694 & 763 & 330 & 401 & ---\\
\hline
\end{tabular}



\caption{Execution time}
\label{tab:bench_results}

\end{table}
\begin{table}[h]
\scriptsize
\begin{minipage}{0.45\linewidth}
\centering
\caption*{mAddAdd}
\begin{tabular}{|c|c|c|c|c|c|c|} 
\hline
\rowcolor{LightBlue}
\multicolumn{3}{|c|}{Matrices dimensions} & \multicolumn{2}{c|}{Ratio ($\frac{FHW}{FHW_{distilled}}$)}\\
% \rowcolor{LightBlue}
\hline
m1 & m2 & m3 & writes & reads\\ 
\hline
64 & 64 & 64 & 1.10 & 1.15\\ 
128 & 128 & 128 & 1.02 & 1.05\\
256 & 256 & 256 & 1.03 & 1.06\\
512 & 512 & 512 & 1.10 & 1.16\\
\hline
\end{tabular}
\end{minipage}
\begin{minipage}{0.45\linewidth}
\centering
\caption*{mMaskAdd}
\begin{tabular}{|c|c|c|c|c|c|c|} 
\hline
\rowcolor{LightBlue}
\multicolumn{3}{|c|}{Matrices dimensions} & \multicolumn{2}{c|}{Ratio ($\frac{FHW}{FHW_{distilled}}$)}\\
% \rowcolor{LightBlue}
\hline
m1 & m2 & m3 & writes & reads\\ 
\hline
64 & 64 & 64 & 1.13 & 1.26\\ 
128 & 128 & 128 & 1.06 & 1.11\\
256 & 256 & 256 & 1.08 & 1.09\\
512 & 512 & 512 & 1.10 & 1.16\\
\hline
\end{tabular}
\end{minipage}
\begin{minipage}{0.45\linewidth}
\centering
\caption*{mMapAdd}
\begin{tabular}{|c|c|c|c|c|c|c|} 
\hline
\rowcolor{LightBlue}
\multicolumn{3}{|c|}{Matrices dimensions} & \multicolumn{2}{c|}{Ratio ($\frac{FHW}{FHW_{distilled}}$)}\\
% \rowcolor{LightBlue}
\hline
m1 & m2 & m3 & writes & reads\\ 
\hline
64 & 64 & --- & 1.10 & 1.21\\ 
128 & 128 & --- & 1.07 & 1.14\\
256 & 256 & --- & 1.07 & 1.19\\
512 & 512 & --- & 1.10 & 1.21\\
\hline
\end{tabular}
\end{minipage}
\hfill
\begin{minipage}{0.45\linewidth}
\centering
\caption*{mMapKron}
\begin{tabular}{|c|c|c|c|c|c|c|} 
\hline
\rowcolor{LightBlue}
\multicolumn{3}{|c|}{Matrices dimensions} & \multicolumn{2}{c|}{Ratio ($\frac{FHW}{FHW_{distilled}}$)}\\
% \rowcolor{LightBlue}
\hline
m1 & m2 & m3 & writes & reads\\ 
\hline
2 & 64 & --- & 1.71 & 1.88\\ 
2 & 128 & --- & 1.72 & 1.87\\
2 & 256 & --- & 1.65 & 1.83\\
4 & 128 & --- & 1.81 & 1.91\\
\hline
\end{tabular}
\end{minipage}

\caption{Memory accesses}
\label{tab:mem_results}
\end{table}

\begin{table}[h]
\scriptsize
\centering
\begin{tabular}{|l|c|c|c|c|c|c|c|c|c|} 
\hline
\rowcolor{LightBlue}

{Benchmark} & \multicolumn{8}{c|}{Ratio (${\frac{FHW}{FHW_{distilled}}}$)} & {Frequency}\\
\hline
{} & FDRE & LUT3 & LUT6 & LUT5 & LUT4 & LUT2 & RAMB36E2 & MUXF7 & {} \\
% \rowcolor{LightBlue}
\hline
mAddAdd & 0.3 & 0.3 & 0.3 & 0.5 & 0.3 & 0.3 & 0.5 & 0.5 & 200 MHz\\ 
mMaskAdd & 0.5 & 0.5 & 0.7 & 0.4 & 0.7 & 0.5 & 0.7 & 0.6 & 200 MHz\\
mMapAdd & 1 & 0.9 & 0.9 & 1.2 & 1 & 1.1 & 1.1 & 1.2 & 200 MHz\\
mMapKron & 1.5 & 1.5 & 1.3 & 2 & 2 & 1.8 & 1.4 & 1.7 & 200 MHz\\
\hline
\end{tabular}
\caption{Resource utilization}
\label{tab:resource_util}
\end{table}
\pagebreak

\section{Conclusion and Future Work}

We present !!!

Our evaluation shows that !!!

First direction for future research is a more detailed CFPQ algorithms investigation.
We should do More evaluation on sparse matrices on GPGPUs.

Also it is nesessary to implement and evaluate solutions for graphs which is not fit in RAM.
There is a set of technics for huge matrices multiplication.
Is it possible to dopt it for CFPQ

Another direcion is a dataset improvement.
More data.
More grammars/queries.


\setmonofont[Mapping=tex-text]{CMU Typewriter Text}
\bibliographystyle{ugost2008ls}
\bibliography{diploma.bib}
\end{document}
