%20 min preso!
\documentclass[xcolor=table]{beamer}
\usepackage{beamerthemesplit}
\usepackage{wrapfig}
\usetheme{SPbGU}
\usepackage{pdfpages}
\usepackage{amsmath}
\usepackage{cmap}
\usepackage[T2A]{fontenc}
\usepackage[utf8]{inputenc}
\usepackage[english]{babel}
\usepackage{indentfirst}
\usepackage{amsmath}
\usepackage{tikz}
\usepackage{multirow}
\usepackage[noend]{algpseudocode}
\usepackage{algorithm}
\usepackage{algorithmicx}
\usepackage{fancyvrb}
\usetikzlibrary{calc}
\usetikzlibrary{shapes,arrows}
\usetikzlibrary{arrows,automata}
\usetikzlibrary{positioning}

\usepackage{tabularx}
\newcolumntype{Y}{>{\raggedleft\arraybackslash}X}

\renewcommand{\thealgorithm}{}

\newtheorem{mytheorem}{Theorem}
\renewcommand{\thealgorithm}{}

\newcommand{\tikzmark}[1]{\tikz[overlay,remember picture] \node (#1) {};}
\def\Put(#1,#2)#3{\leavevmode\makebox(0,0){\put(#1,#2){#3}}}

\newcommand{\ltz}{$< 1$}


\tikzset{
    state/.style={
           rectangle,
           rounded corners,
           draw=black, very thick,
           minimum height=2em,
           inner sep=2pt,
           text centered,
           },
}

\beamertemplatenavigationsymbolsempty

\title[Static Code Analysis and Linear Algebra]{Fast and Scable Static Code Analysis Requires \\ Fast and Scable Linear Algebra}
\institute[SPbSU]{
Saint Petersburg State University
}

% То, что в квадратных скобках, отображается в левом нижнем углу.
\author[Semyon Grigorev]{Semyon Grigorev}

\date{October 28, 2020}

\begin{document}
{
\begin{frame}[fragile]
  \begin{table}
  \centering
  \begin{tabularx}{\linewidth}{XcX}
    \includegraphics[height=1.5cm]{pictures/YC_logo.pdf} \hfill
    & \begin{minipage}[t]{0.3\textwidth}\center \vspace{-1cm}  Huawei-SPbSU Open Day 2020
      \end{minipage}
    & \hfill \includegraphics[height=1.5cm]{pictures/SPbGU_Logo.png}
  \end{tabularx}
  \end{table}
  \titlepage
\end{frame}
}


\begin{frame}[fragile]
  \frametitle{Static Code Analysis}
  \begin{itemize}
  \item It plays important role in the development workflow
  \item \textbf{Interprocedural} code analysis is the most important 
  \begin{itemize}
    \item Computationally hard problem
    \item Can be expressed in terms of \textbf{language constrained path querying} 
  \end{itemize}
  \end{itemize}
\end{frame}

\begin{frame}[fragile]
  \frametitle{Language Constrained Path Querying\footnote{Or Formal Language Reachability problem}}
  \begin{itemize}
  \item $\Sigma$ is a set of terminals 
  \item $L(\Sigma)$ is a language over $\Sigma$
  \pause
  \item $G = (V,E,D)$ is a directed graph, $E \subseteq V\times D \times V$, $D\subseteq \Sigma$
  \pause
  \item $p = v_0 \xrightarrow{l_0} v_1 \xrightarrow{l_1} \cdots v_{n-1}\xrightarrow{l_{n-1}}v_n$ is a path in $G$
  \item $w(p) = w(v_0 \xrightarrow{l_0} v_1 \xrightarrow{l_1} \cdots v_{n-1}\xrightarrow{l_{n-1}}v_n) = l_0 l_1 \cdots l_{n-1}$
  \pause
  \item $R =\{ p \ | \ w(p) \in L(\Sigma)\}$
  \begin{itemize}
    \item $R$ can be an infinite set
  \end{itemize}
  \pause
  \item Alternative formulation: $Q =\{ (v_0,v_n) \ | \ \exists p = v_0 \xrightarrow{l_0} \cdots \xrightarrow{l_{n-1}}v_n \ (w(p) \in 
  L(\Sigma))\}$
  \end{itemize}
\end{frame}


\begin{frame} \frametitle{Context-Free Path Querying (CFPQ)}
  CFPQ is applicable for static code analysis % (Language Reachability Framework)
  \bigskip
    \begin{itemize}
        \item \emph{Thomas Reps et al.} ``Precise interprocedural dataflow analysis via graph reachability.'' 1995 
        \item \emph{Jakob Rehof and Manuel Fahndrich.} ``Type-base flow analysis: from polymorphic subtyping to CFL-reachability.'' 2001
        \item \emph{Dacong Yan et al.} ``Demand-driven context-sensitive alias analysis for Java.'' 2011
        \item \emph{Qirun Zhang et al.}  ``Efficient subcubic alias analysis for C.'' 2014
        \item ...
    \end{itemize}
\end{frame}


\begin{frame}[fragile]
\frametitle{Example: Field-Sensitivity}
\begin{columns}[c] % The "c" option specifies centered vertical alignment while the "t" option is used for top vertical alignment

\column{.25\textwidth} % Left column and width
\begin{Verbatim}[numbers=left]
  v.h = u;
  ...
  p = v;
  ...
  q.g = p;
  ...
  w = q;
  ...
  x = w.g;
  if (...) {
    y = w.f;
  }
  else {
    y = x.h;
  }    
\end{Verbatim}

\column{.6\textwidth} % Right column and width
\begin{figure}[h]
    \centering        
    \begin{tikzpicture}[shorten >=1pt,auto]
       \node[state] (q_0)                      {$u$};
       \node[state] (q_1) [right=of q_0] {$v$};
       \node[state] (q_2)  [above=of q_1]  {$p$};
       \node[state] (q_3) [right=of q_2] {$q$};
     \node[state] (q_4) [below=of q_3] {$w$};
     \node[state] (q_5) [right=of q_4] {$x$};
    \node[state] (q_6) [below=of q_5] {$y$};

        \path[->]
        (q_0) edge  node {$h$} (q_1)
        (q_1) edge[thick]  node {$$} (q_2)
        (q_2) edge  node {$g$} (q_3)
       (q_3) edge [thick] node {$$} (q_4)
       (q_4) edge  node {$\bar{g}$} (q_5)
      (q_5) edge  node {$\bar{h}$} (q_6)
       (q_4) edge[bend right, below]  node {$\bar{f}$} (q_6);

    \end{tikzpicture}
\end{figure}
Correct path: \textcolor{green}{$hg\bar{g}\bar{h}$}

Incorrect path: \textcolor{red}{$hg\bar{f}$}

\end{columns}
\end{frame}


\begin{frame}
  \frametitle{Applicability of CFPQ}

% \begin{itemize} 
 Interprocedural static nullability analysis\footnote{\emph{Kai Wang et. al.} Graspan: a single-machine disk-based graph system for interprocedural 
static analyses of large-scale systems code. 2017}
 
\bigskip

\begin{itemize} 
   \item ``We have identified a total of 1127 unnecessary NULL tests in Linux, 149 in PostgreSQL, 
   32 in httpd.''
   \item ``Our analyses reported 108 new NULL pointer dereference bugs in Linux, among which 23 are false positives''
   \item ``For PostgreSQL and httpd, we detected 33 and 14 new NULL pointer bugs; our manual 
   validation did not find any false positives among them.''
\end{itemize}

%\end{itemize}

\end{frame}


\begin{frame}
  \frametitle{CFPQ and Linear Algebra}

  \begin{itemize}
    \item CFQP can be formulated in terms of linear algebra: \emph{Rustam Azimov, Semyon Grigorev} ``Context-free path querying by matrix multiplication.'' 2018 
    \item CFQP can be efficiently implemented using sparse linear algebra and modern parallel hardware: \emph{Arseniy Terekhov, Artyom Khoroshev, Rustam Azimov, Semyon Grigorev} ``Context-Free Path Querying with Single-Path Semantics by Matrix Multiplication.'' 2020 
  \end{itemize}
\end{frame}



\begin{frame} \frametitle{Evaluation}
\begin{itemize}
  \item PC
  \begin{itemize}
   \item OS: Ubuntu 18.04
   \item CPU: Intel core i7 8700k 3,4GHz
   \item RAM: DDR4 64 Gb
  \end{itemize}
  \pause
  \item Points-to analysis
  \item Graphs are generated by LLVM for submodules of Linux core
  \item Implementation is based on SuiteSparse:GraphBLAS\footnote{\url{https://people.engr.tamu.edu/davis/GraphBLAS.html}}
\end{itemize}
\end{frame}

\begin{frame}[fragile] \frametitle{Evaluation: Results\footnote{We include only time spent analyzing graph. Graph building time is not included}}
\begin{center}
  {
  \rowcolors{1}{}{lightgray}
  \begin{tabular}{ c | c | c | c }
      %\hline
      
      Name           & \#V        & \#E        & Time (min) \\ 
      \hline
      \hline
      arch           & 3 448 422  & 5 940 484  & 3.25       \\
      crypto         & 3 464 970  & 5 976 774  & 3.25       \\
      drivers        & 4 273 803  & 7 415 538  & 17.5       \\
      fs             & 4 177 416  & 7 218 746  & 6.2        \\  
      %\hline
    \end{tabular}
    }
 \end{center}
\end{frame}

\begin{frame}[fragile] \frametitle{Conclusion}
  \begin{itemize}
    \item Many interprocedural static code analysis tasks can be expressed in~terms of linear algebra
    \item Sparse linear algebra can be efficiently implemented on modern parallel hardware
    \item Database querying, social network analysis and other problems can be reduced to sparse linear algebra
  \end{itemize}
\end{frame}

\end{document}
