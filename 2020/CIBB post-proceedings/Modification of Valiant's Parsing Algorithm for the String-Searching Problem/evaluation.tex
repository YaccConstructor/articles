\section{Evaluation}

In this section we present the results of experiments whose purpose is to demonstrate the practical applicability of the proposed algorithm. All tests were run on a PC with the following characteristics: OS: Linux Mint 19.1, CPU: Intel i5-8250U, 3400 Mhz, RAM: 8 GB, GPU: NVIDIA GeForce GTX 1050 MAX-Q.

We implement two different versions of Valiant's algorithm and its modification in C++ programming language\footnote{The source code is available on GitHub: \url{https://github.com/SusaninaJulia/PBMM}.}:

\begin{itemize}
    \item \textbf{CPU-based solutions} (\textsf{valCPU} and \textsf{modCPU}).
    One of the most efficient implementations of the “Method of the Four Russians”~\cite{arlazarov1970economical} from the library M4RI~\cite{Albrecht_2010} is used for Boolean matrix multiplication. 
    \item \textbf{GPU-based solutions} (\textsf{valGPU} and \textsf{modGPU}).
    A naive Boolean matrix multiplication in CUDA C with Boolean values treated as bits and packed into \texttt{uint\_32} is implemented.
\end{itemize}


We evaluate these implementations on context-free grammars $D_2$: 
$$
s  \rightarrow  s \ s \ | \ A \ s \ U \ | \ C \ s \ G \ | \ \varepsilon \\
$$ and $BIO$: \[
    \begin{array}{rcl}
            s & \rightarrow & \ \ \text{stem}\langle s0 \rangle \\
            \text{any\_str} & \rightarrow & \ \ \text{any\_smb}*[2..10] \\
            s0 & \rightarrow & \ \ \text{any\_str} \ | \ \text{any\_str} \ \text{stem}\langle s0 \rangle \ s0 \\
            \text{any\_smb} & \rightarrow &  \ \ A \ | \ U \ | \ C \ | \  G \\
            \text{stem1}\langle s1 \rangle & \rightarrow & \ \ A \ s1 \  U \ | \ G \ s1 \ C \ | \ U \ s1 \ A \ | \ C \ s1 \ G \\
            \text{stem2}\langle s1 \rangle & \rightarrow & \ \ \text{stem1}\langle \text{stem1}\langle s1 \rangle \rangle \\
            \text{stem}\langle s1 \rangle & \rightarrow & \ \ A \ \text{stem}\langle s1 \rangle \ U
              \ | \ U \ \text{stem}\langle s1 \rangle \ A \ | \ C \ \text{stem}\langle s1 \rangle \ G \\
              & & | \ G \ \text{stem}\langle s1 \rangle \ C 
              \  | \ \text{stem1}\langle \text{stem2}\langle s1 \rangle \rangle \\
    \end{array}.
    \]
Grammar $D_2$ generates Dyck language on two types of parentheses and is chosen because grammars that describe well-balanced sequences of brackets are often used in string analysis in bioinformatics. 
Grammar $BIO$ applies to the tRNA classification problem in paper~\cite{bioinformatics19}.
We test various strings with length $n$ up to 8191 and search substrings with length $subs$ up to 2040. The results of the evaluation are summarized in the tables~\ref{tbl1} and~\ref{tbl3}. Time is measured in milliseconds.



\begin{table*}[h]
\caption{Comparison of the Valiant's algorithm and the modification}
\label{tbl1}
\centering
\resizebox{\columnwidth}{!}{%
\begin{tabular}{|| c||c|c|c|c || c|c|c|c ||} 
\hline
\multirow{2}{0.7em}{n} & \multicolumn{4}{c||}{Grammar $D_2$} & \multicolumn{4}{c||}{Grammar $BIO$} \\
& valCPU & modCPU & valGPU & modGPU & valCPU & modCPU & valGPU & modGPU \\
\hline
127 & 78 & 76 & 195 & 105 & 1345 & 1339 & 193 & 106 \\
255 & 289 & 292 & 523 & 130 & 5408 & 5488 & 525 & 140 \\
511 & 1212 & 1177 & 1909 & 250 & 21969 & 22347 & 1994 & 256 \\
1023 & 4858 & 4779 & 7878 & 540 & 88698 & 90318 & 7890 & 598 \\
2047 & 19613 & 19379 & 33508 & 1500 & 363324 & 374204 & 34010 & 1701 \\
4095 & 78361 & 78279 & 140473 & 4453 & 1467675 & 1480594 & 141104 & 5472 \\
8191 & 315677 & 315088 & - & 13650 & - & - & - & 18039 \\
\hline
\end{tabular}
}
\end{table*}

The comparative analysis (table~\ref{tbl1}) shows that the performance of Valiant's and modified algorithms is the same for the CPU-based solutions.
The GPU-based implementation of Valiant algorithm is slower for grammar $D_2$ than the CPU-based one. 
It is probably because of processing a large amount of small matrix multiplication which cannot be computed concurrently. 
In other cases, GPU-based solution provides significant performance improvement, especially for our modification, where utilization of parallelism facilitates parallel multiplication of the matrices itself and matrices in a layer.

To adapt our algorithm for the string-matching problem the, $main()$ function takes an additional argument $sub$ --- the maximum length of strings we want to find, so the modification has no need to compute all layers as shown in section 3.3.
The corresponding implementations are named as \textsf{adpCPU} and \textsf{adpGPU}.

\begin{table*}

\begin{center}
\caption{Modified algorithm evaluation on the string-searching for the $BIO$ grammar}
\label{tbl3}
    \begin{tabular}{ ||c||c||c|c|| } 
    \hline 
    %\multirow{2}{1.5em}{sub} & \multirow{2}{0.7em}{n} & \multicolumn{2}{c||}{Grammar $BIO$} \\
     sub & n & adpCPU &  adpGPU \\
    \hline
    \multirow{4}{1.5em}{250} & 1023 & 2996 & 242 \\ 
    & 2047 & 6647 & 255\\ 
    & 4095 & 13825 & 320\\ 
    & 8191 & 28904 & 456\\ 
    \hline
    \multirow{3}{1.5em}{510} & 2047 & 12178 & 583\\
    & 4095 & 26576 & 653\\
    & 8191 & 56703 & 884\\ 
    \hline
    \multirow{2}{2em}{1020} & 4095 & 48314 & 1590 \\
    & 8191 & 108382 & 1953\\ 
    \hline
    2040 & 4095 & 197324 & 5100\\ 
    \hline
    \end{tabular}
\end{center}

\end{table*}

The results of the second evaluation (table~\ref{tbl3}) shows that the modified version of algorithm can find all derivable substrings much faster than the Valiant's algorithm, thus it can be efficiently applied to the string-searching problem.