%15 min preso!
\documentclass[xcolor=table,aspectratio=169]{beamer}
\usepackage{beamerthemesplit}
\usepackage{wrapfig}
\usetheme{SPbGU}
\usepackage{pdfpages}
\usepackage{amsmath}
\usepackage{cmap}
\usepackage[T2A]{fontenc}
\usepackage[utf8]{inputenc}
\usepackage[english]{babel}
\usepackage{indentfirst}
\usepackage{amsmath}
\usepackage{tikz}
\usepackage{multirow}
\usepackage[noend]{algpseudocode}
\usepackage{algorithm}
\usepackage{algorithmicx}
\usepackage{fancyvrb}
\usepackage{hyperref} 
\usetikzlibrary{calc}
\usetikzlibrary{shapes, backgrounds}
\usetikzlibrary{arrows,automata}
\usetikzlibrary{positioning}
\usetikzlibrary{fit}
\usetikzlibrary{shapes.callouts}
\usetikzlibrary{shapes.misc}
\usepackage{xparse}
\usepackage{fontawesome}

\usepackage{etoolbox,refcount}
\usepackage{multicol}

\usepackage{tabularx}
\newcolumntype{Y}{>{\raggedleft\arraybackslash}X}

\renewcommand{\thealgorithm}{}

\newtheorem{mytheorem}{Theorem}
\renewcommand{\thealgorithm}{}

\newcommand{\tikzmark}[1]{\tikz[overlay,remember picture] \node (#1) {};}
\def\Put(#1,#2)#3{\leavevmode\makebox(0,0){\put(#1,#2){#3}}}

\newcommand{\ltz}{$< 1$}

\definecolor{links}{HTML}{2A1B81}
\hypersetup{colorlinks,linkcolor=,urlcolor=links}


\tikzset{
    state/.style={
           rectangle,
           rounded corners,
           draw=black, very thick,
           minimum height=2em,
           inner sep=2pt,
           text centered,
           },
}

\tikzset{
    invisible/.style={opacity=0,text opacity=0},
    visible on/.style={alt=#1{}{invisible}},
    alt/.code args={<#1>#2#3}{%
      \alt<#1>{\pgfkeysalso{#2}}{\pgfkeysalso{#3}} % \pgfkeysalso doesn't change the path
    },
}

\tikzset{cross/.style={cross out, draw=black, minimum size=2*(#1-\pgflinewidth), inner sep=0pt, outer sep=0pt, ultra thick},
%default radius will be 1pt. 
cross/.default={1pt}}

\NewDocumentCommand{\mycallout}{r<> O{opacity=0.8,text opacity=1} m m m}{%
\tikz[remember picture, overlay]\node[align=center, fill=cyan!20, text width=#5cm,
#2,visible on=<#1>, rounded corners,
draw,rectangle callout,anchor=pointer,callout relative pointer={(290:0.5cm)}]
at (#3) {#4};
}

\NewDocumentCommand{\mycalloutR}{r<> O{opacity=0.8,text opacity=1} m m m}{%
\tikz[remember picture, overlay]\node[align=center, fill=cyan!20, text width=#5cm,
#2,visible on=<#1>, rounded corners,
draw,rectangle callout,anchor=pointer,callout relative pointer={(30:0.8cm)}]
at (#3) {#4};
}


%callout relative pointer={(230:0.5cm)}]

\newcounter{countitems}
\newcounter{nextitemizecount}
\newcommand{\setupcountitems}{%
  \stepcounter{nextitemizecount}%
  \setcounter{countitems}{0}%
  \preto\item{\stepcounter{countitems}}%
}
\makeatletter
\newcommand{\computecountitems}{%
  \edef\@currentlabel{\number\c@countitems}%
  \label{countitems@\number\numexpr\value{nextitemizecount}-1\relax}%
}
\newcommand{\nextitemizecount}{%
  \getrefnumber{countitems@\number\c@nextitemizecount}%
}
\newcommand{\previtemizecount}{%
  \getrefnumber{countitems@\number\numexpr\value{nextitemizecount}-1\relax}%
}
\makeatother    
\newenvironment{AutoMultiColItemize}{%
\ifnumcomp{\nextitemizecount}{>}{3}{\begin{multicols}{2}}{}%
\setupcountitems\begin{itemize}}%
{\end{itemize}%
\unskip\computecountitems\ifnumcomp{\previtemizecount}{>}{3}{\end{multicols}}{}}


\beamertemplatenavigationsymbolsempty

\title[Лаборатория YADRO, Мат-Мех СПбГУ]{Лаборатория YADRO в СПбГУ}
\subtitle{на Математико-Механическом факультете}
\institute[СПбГУ]{
Санкт-Петербургский Государственный Университет
}

% То, что в квадратных скобках, отображается в левом нижнем углу.
\author[Семён Григорьев]{Семён Григорьев}

\date{25 июня 2025}


\begin{document}
{
\begin{frame}[fragile]
  \begin{table}
  \centering
  %\includegraphics[height=1.5cm]{pictures/SPbGU_Logo.png}
  \begin{tabularx}{\linewidth}{XcX}
    \includegraphics[height=1.4cm]{pictures/YADRO_Logo.png} 
    \hfill
    & 
    & \hfill \includegraphics[height=1.6cm]{pictures/SPbGU_Logo.png}
  \end{tabularx}
  \end{table}
  \titlepage
\end{frame}
}

\begin{frame}[fragile]
  \frametitle{Основные направления}
    \begin{minipage}{0.48\textwidth}
        \underline{\textbf{Алгоритмическая статистика}}
        \small{
        \begin{itemize}\setlength\itemsep{-0.4em}
          \item Вячеслав Гориховский, Владимир Кутуев
          \item \href{https://github.com/PySATL/}{PySATL}, cеминар Statistical Software
          \item Python, математическая статистика
        \end{itemize}
        }
        \underline{\textbf{ Системы хранения данных}}
        \small{
        \begin{itemize}\setlength\itemsep{-0.4em}
          \item Анна Васенина, Вячеслав Гориховский
          \item SPDK, ChunkFS
          \item Linux Kernel, дедупликация, Rust
        \end{itemize}
        }
        \underline{\textbf{Компьютерные сети}}
        \small{
        \begin{itemize}\setlength\itemsep{-0.4em}
          \item Илья Зеленчук
          \item \href{https://github.com/mimi-net/miminet}{Miminet}, курс на Stepik, отбор на <<Импульс>>
          \item Эмуляция сетей, VLAN, STP/RSTP, IPIP
        \end{itemize}
        }
    \end{minipage}~  
    \begin{minipage}{0.48\textwidth}
      \underline{\textbf{Системное программирование}}
      \small{
        \begin{itemize}\setlength\itemsep{-0.4em}
          \item Кирилл Смирнов, Дмитрий Косарев
          \item Ghidra, LLVM, Sail, Instrew
          \item C, C++, OCaml, ассемблер
        \end{itemize}
      }
        \underline{\textbf{Аппаратное обеспечение}}
        \small{
        \begin{itemize}\setlength\itemsep{-0.4em}
          \item Кирилл Смирнов, Семён Григорьев
          \item SCR1, \href{https://github.com/Lamagraph}{LamaGraph}
          \item HDL, System Verilog, Clash, FPGA
        \end{itemize}
        }
        \underline{\textbf{\href{https://github.com/SparseLinearAlgebra}{Разреженная линейная алгебра}}}
        \small{
        \begin{itemize}\setlength\itemsep{-0.4em}
          \item Семён Григорьев, Владимир Кутуев
          \item GraphBLAS, Spla, LAGraph, CuBool
          \item C, C++, Cuda, OpenCL, GPGPU, параллельные вычисления
        \end{itemize}
        }      
    \end{minipage}
    \noindent\makebox[\linewidth]{\rule{\paperwidth}{0.4pt}}
    \begin{center}
      Курсовые, дипломы, магистерские (в сумме больше 50), публикации (больше 20) \\
      Курсы, модули в курсах, отдельные лекции
    \end{center}
\end{frame}


%\begin{frame}[fragile]
%  \frametitle{Основные направления}
%      
%    \begin{minipage}{0.48\textwidth}
%      \begin{itemize}
%        \item Алгоритмическая статистика
%        \begin{itemize}
%          \item Вячеслав Гориховский, Владимир Кутуев
%          \item PySATL 
%          \item Семинар Statistical Software, 
%        \end{itemize}
%        \item Системы хранения данных
%        \begin{itemize}
%          \item Анна Васенина, Вячеслав Гориховский
%          \item SPDK, ChunkFS
%          \item Лекции, семинары
%        \end{itemize}
%        \item Компьютерные сети
%        \begin{itemize}
%          \item Илья Зеленчук
%          \item Miminet
%          \item Курс на Stepik, отбор на <<Импульс>>
%        \end{itemize}
%      \end{itemize}
%    \end{minipage}~  
%    \begin{minipage}{0.48\textwidth}
%      \begin{itemize} 
%        \item Системное программирование
%        \begin{itemize}
%          \item Кирилл Смирнов, Дмитрий Косарев
%          \item Ghidra, LLVM, Sail, Instrew
%          \item Курсы, учебные модули
%        \end{itemize}
%        \item Аппаратное обеспечение
%        \begin{itemize}
%          \item Кирилл Смирнов, Семён Григорьев
%          \item SCR1, LamaGraph
%          \item Семинары, лекции
%        \end{itemize}
%        \item Обобщённая разреженная линейная алгебра
%        \begin{itemize}
%          \item Семён Григорьев, Владимир Кутуев
%          \item GraphBLAS, Spla, LAGraph, CuBool
%          \item Лекции, модули в курсах,
%        \end{itemize}
%      \end{itemize}
%    \end{minipage}
%    \noindent\makebox[\linewidth]{\rule{\paperwidth}{0.4pt}}
%    \begin{center}
%      Курсовые, дипломы, магистерские (в сумме больше 50), публикации (больше 20)
%    \end{center}
%\end{frame}

\begin{frame}[fragile]
  \frametitle{<<Три кита>>}
  \begin{minipage}[t]{0.48\textwidth}
    \begin{center}
    \underline{\textbf{Образование}}
    \begin{itemize}
      \item Летняя Школа\footnotemark, Зимняя Школа, кружки, семинары
      \item Подготовка учебных материалов, курсов
      \item Ревью учебных материалов
    \end{itemize}
    \end{center}
  \end{minipage}
  %\pause
  \begin{minipage}[t]{0.48\textwidth}
    \begin{center}
    \underline{\textbf{Индустрия}}
    \begin{itemize}
      \item Прикладные исследования
      \item Подготовка специалистов с необходимыми знаниями и навыками
      \item Трансфер знаний, рабочие группы, \ldots
    \end{itemize}
    \end{center}
  \end{minipage}
  %\pause
  \begin{center}
  \underline{\textbf{Наука}}
  \begin{itemize}
      \item Фундаментальные исследования
      \item Доклады на академических конференциях, публикации
      \item Участие в международном академическом сообществе
  \end{itemize}
\end{center}
  \footnotetext{\url{https://t.me/+YW4sRNVvsn81YTcy}}
\end{frame}


\begin{frame}[fragile]
  \frametitle{Пример: обобщённая разреженная линейная алгебра}
  \begin{itemize}
    \item Вклад студентов 2--4 курсов
      \begin{itemize}
        \item Добавление кросс-сборки в CI под различные архитектуры для SuiteSparse\footnote{\url{https://github.com/DrTimothyAldenDavis/SuiteSparse/pull/955}}
        \item Оптимизация умножения матриц с использованием векторного расширения RVV 1.0\footnote{\url{https://github.com/DrTimothyAldenDavis/GraphBLAS/pull/381}}
        \item Интеграция высокопроизводительных алгоритмов анализа графов в LAGraph\footnote{\url{https://github.com/GraphBLAS/LAGraph/pull/265}}\textsuperscript{,}\footnote{\url{https://github.com/GraphBLAS/LAGraph/pull/261}}
      \end{itemize}
      %\pause
    \item Публикации студентов 2--4 курсов и магистров
    \begin{itemize}
      \item Оптимизация функции умножения матриц библиотеки SuiteSparse:GraphBLAS с использованием векторного расширения RISC-V, Суворов Р., Григорьев С., Кутуев В. 
      \item Разработка инфраструктуры для создания специализированных ускорителей на основе Interaction Nets, Кубышкин Е., Пономарев Н.
      %\pause
      \item Universal High-Performance CFL-Reachability via Matrix Multiplication, Ilia Muravev, Semyon Grigorev, SOAP@PLDI 2025
    \end{itemize}
    %\pause
    \item Участие в рабочей группе <<Развитие экосистемы ПО>> Альянса RISC-V
  \end{itemize}
\end{frame}

\end{document}
