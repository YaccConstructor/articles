\section{Conclusion and Future Work}

Brahma.FSharp---a tool for developing cross-platform, GPGPU-accelerated .NET applications---is presented.
We demonstrated applications portability by evaluating performance across multiple platforms, including RISC-V with PowerVR GPGPU and embedded Intel GPUs.

While the work remains in progress, Brahma.FSharp already enables creation of linear algebra kernels sufficiently performant for integration into libraries like Math.NET Numerics, allowing transparent offloading of generic linear algebra operations to GPGPUs.
Such integration is planned for the near future.

%Also, within the translator improvements, it is necessary to improve performance of data transferring between managed and native memory for complex types such as discriminated unions.
%For translator improvements, optimizing data transfer performance between managed and native memory for complex types (e.g., discriminated unions) requires further development.

Although the agent-based communication approach aligns naturally with both OpenCL and F\#, MailboxProcessor may not be optimal for high-frequency CPU-GPU communication.
Alternative solutions like Hopac\footnote{Hopac and MailboxProcessor performance comparison: \url{https://vasily-kirichenko.github.io/fsharpblog/actors}} or lightweight command queue wrappers could provide better performance for latency-critical code.

A significant challenge for future research involves automatic memory management.
Currently, GPGPU memory requires manual cleanup despite .NET's garbage collector.
Developing a hybrid approach that leverages automatic garbage collection while retaining manual control when needed remains an open problem.
