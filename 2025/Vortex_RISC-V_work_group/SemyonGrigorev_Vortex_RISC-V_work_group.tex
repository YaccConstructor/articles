%15 min preso!
\documentclass[xcolor=table,aspectratio=169]{beamer}
\usepackage{beamerthemesplit}
\usepackage{wrapfig}
\usetheme{SPbGU}
\usepackage{pdfpages}
\usepackage{amsmath}
\usepackage{cmap}
\usepackage[T2A]{fontenc}
\usepackage[utf8]{inputenc}
\usepackage[english]{babel}
\usepackage{indentfirst}
\usepackage{amsmath}
\usepackage{tikz}
\usepackage{multirow}
\usepackage[noend]{algpseudocode}
\usepackage{algorithm}
\usepackage{algorithmicx}
\usepackage{fancyvrb}
\usepackage{hyperref} 
\definecolor{links}{HTML}{2A1B81}
\hypersetup{colorlinks,linkcolor=,urlcolor=links}
\usetikzlibrary{calc}
\usetikzlibrary{shapes, backgrounds}
\usetikzlibrary{arrows,automata}
\usetikzlibrary{positioning}
\usetikzlibrary{fit}
\usetikzlibrary{shapes.callouts}
\usetikzlibrary{shapes.misc}
\usepackage{xparse}
\usepackage{fontawesome}

\usepackage{etoolbox,refcount}
\usepackage{multicol}

\usepackage{tabularx}
\newcolumntype{Y}{>{\raggedleft\arraybackslash}X}

\renewcommand{\thealgorithm}{}

\newtheorem{mytheorem}{Theorem}
\renewcommand{\thealgorithm}{}

\newcommand{\tikzmark}[1]{\tikz[overlay,remember picture] \node (#1) {};}
\def\Put(#1,#2)#3{\leavevmode\makebox(0,0){\put(#1,#2){#3}}}

\newcommand{\ltz}{$< 1$}

\tikzset{
    state/.style={
           rectangle,
           rounded corners,
           draw=black, very thick,
           minimum height=2em,
           inner sep=2pt,
           text centered,
           },
}

\tikzset{
    invisible/.style={opacity=0,text opacity=0},
    visible on/.style={alt=#1{}{invisible}},
    alt/.code args={<#1>#2#3}{%
      \alt<#1>{\pgfkeysalso{#2}}{\pgfkeysalso{#3}} % \pgfkeysalso doesn't change the path
    },
}

\tikzset{cross/.style={cross out, draw=black, minimum size=2*(#1-\pgflinewidth), inner sep=0pt, outer sep=0pt, ultra thick},
%default radius will be 1pt. 
cross/.default={1pt}}

\NewDocumentCommand{\mycallout}{r<> O{opacity=0.8,text opacity=1} m m m}{%
\tikz[remember picture, overlay]\node[align=center, fill=cyan!20, text width=#5cm,
#2,visible on=<#1>, rounded corners,
draw,rectangle callout,anchor=pointer,callout relative pointer={(290:0.5cm)}]
at (#3) {#4};
}

\NewDocumentCommand{\mycalloutR}{r<> O{opacity=0.8,text opacity=1} m m m}{%
\tikz[remember picture, overlay]\node[align=center, fill=cyan!20, text width=#5cm,
#2,visible on=<#1>, rounded corners,
draw,rectangle callout,anchor=pointer,callout relative pointer={(30:0.8cm)}]
at (#3) {#4};
}

\newcommand\colR{\cellcolor{red!20}}
\newcommand\colB{\cellcolor{blue!20}}
\newcommand\colG{\cellcolor{green!20}}
\definecolor{Gray}{gray}{0.8}

%callout relative pointer={(230:0.5cm)}]

\newcounter{countitems}
\newcounter{nextitemizecount}
\newcommand{\setupcountitems}{%
  \stepcounter{nextitemizecount}%
  \setcounter{countitems}{0}%
  \preto\item{\stepcounter{countitems}}%
}
\makeatletter
\newcommand{\computecountitems}{%
  \edef\@currentlabel{\number\c@countitems}%
  \label{countitems@\number\numexpr\value{nextitemizecount}-1\relax}%
}
\newcommand{\nextitemizecount}{%
  \getrefnumber{countitems@\number\c@nextitemizecount}%
}
\newcommand{\previtemizecount}{%
  \getrefnumber{countitems@\number\numexpr\value{nextitemizecount}-1\relax}%
}
\makeatother    
\newenvironment{AutoMultiColItemize}{%
\ifnumcomp{\nextitemizecount}{>}{3}{\begin{multicols}{2}}{}%
\setupcountitems\begin{itemize}}%
{\end{itemize}%
\unskip\computecountitems\ifnumcomp{\previtemizecount}{>}{3}{\end{multicols}}{}}


\beamertemplatenavigationsymbolsempty

\title[Vortex RISC-V GPGPU]{О состоянии проекта RISC-V GPGPU Vortex}
\subtitle{Рабочая группа ``Развитие экосистемы ПО на RISC-V''}
\institute[СПбГУ]{
Санкт-Петербургский Государственный Университет
}

\author[Семён Григорьев]{Семён Григорьев}

\date{13 ноября 2025}

\begin{document}
{
\begin{frame}[fragile]
  \begin{table}
  \centering  
  \begin{tabularx}{\linewidth}{XcX}
    \hfill
    & 
    & \hfill \includegraphics[height=1.4cm]{pictures/SPbSU_Logo.pdf}
  \end{tabularx}
  \end{table}
  \titlepage
\end{frame}
}

\begin{frame}[fragile]
  \frametitle{Контекст}
  \begin{itemize}
    \item \textbf{Spla\footnote{\url{https://github.com/SparseLinearAlgebra/spla}}}: библиотека анализа графов 
    \begin{itemize}
      \item На основе разреженной линейной алгебры
      \item Использует графические ускорители через OpenCL
    \end{itemize} 
    \item \textbf{Vortex\footnote{\url{https://github.com/vortexgpgpu}}}: открытый проект графического ускорителя
    \begin{itemize} 
      \item Набор инструкций, основанный на RISC-V ISA
      \item Поддержка OpenCL через POCL\footnote{\url{https://portablecl.org/}}
      \item Конфигурируемая архитектура: количество кластеров, ядер, потоков, наличие/отсутствие кэшей разного уровня и т.д.
    \end{itemize}
  \end{itemize}
\end{frame}

\begin{frame}[fragile]
  \frametitle{Как Spla ведёт себя на Vortex?}
  \begin{itemize}
    \item[\faQuestion] Работоспособность
    \begin{itemize}
      \item Проходят тесты
      \item Работают прикладные алгоритмы на основе Spla
    \end{itemize} 
    \item[\faQuestion] Масштабируемость
    \begin{itemize} 
      \item Зависимость производительности на конкретных прикладных задачах от параметров архитектуры
      \item Потребление ресурсов ПЛИС
    \end{itemize}
    \item[\faQuestion] Реальная производительность на ПЛИС
  \end{itemize}
\end{frame}

\begin{frame}[fragile]
  \frametitle{Проблемы со сбросом регистров}
    \begin{itemize}  
    \item Регистры сбрасываются в глобальную память
    \begin{itemize}
      \item У <<взрослых>> ГПУ для этого специальные сегменты в районе локальной памяти
    \end{itemize}
    \item Типичные оптимизации\footnote{Например, в умножении плотных матриц} не работают\footnote{\url{https://github.com/vortexgpgpu/vortex/issues/251}}$^,$\footnote{\url{https://github.com/vortexgpgpu/vortex/issues/205}}
    \begin{itemize}
      \item Не просто не улучшают производительность, а заметно её ухудшают
    \end{itemize}
    \item В целом, есть подозрение, что мало регистров на нитку
    \begin{itemize}      
      \item Типичное для RISC-V $32 + 32$
      \item У <<взрослых>> ГПУ порядка 256
      \item Ну и в целом с дизайном не всё гладко\footnote{\url{https://github.com/vortexgpgpu/vortex/issues/286}}
    \end{itemize}
  \end{itemize}

\end{frame}

\begin{frame}[fragile]
  \frametitle{Не самая плная поддержка OpenCL}
  \begin{itemize}
    \item Базовые функции драйвера
    \begin{itemize}
      \item Не было поддержки работы с несколькими ядрами\footnote{Теперь есть, мы поправили: \url{https://github.com/vortexgpgpu/pocl/pull/6}}
      \item Не работает получение информации о скомпилированном ядре (\texttt{clGetProgramInfo})\footnote{\url{https://github.com/vortexgpgpu/vortex/issues/258}}$^,$\footnote{\url{https://github.com/vortexgpgpu/vortex/issues/287}}
    \end{itemize}
    \item Не все функции работы с памятью
    \begin{itemize}
      \item Например, не работает \texttt{copyBuffer}\footnote{\url{https://github.com/vortexgpgpu/vortex/issues/283}}
    \end{itemize}
    \item Плохо поддержаны атомарные операции (хотя заявлена поддержка A-расширения)
    \begin{itemize}
      \item[\faCheck] Есть поддержка в функциональном симуляторе SimX 
      \item[\faGears] Работаем над поддержкой в компиляторе\footnote{\url{https://github.com/vortexgpgpu/pocl/pull/8}} 
      \item[\faTimes] Нет поддержки на уровне HDL\footnote{\url{https://github.com/vortexgpgpu/vortex/issues/285}}
    \end{itemize}
  \end{itemize}
\end{frame}

\begin{frame}[fragile]
  \frametitle{Результаты и планы}
  \begin{itemize}
    \item[\faCheck] Существующие прикладные алгоритмы с использованием Spla запускаются в SimX (BFS, SSSP, PageRank, подсчёт треугольников), тесты проходят
    \item[\faCheck] Подсчёт треугольников работает на уровне RTL симуляции\footnote{Так как не требует атомарных операций}
    \item[\faGears] Запустить подсчёт треугольников на ПЛИС
    \item[\faGears] Подобрать оптимальную конфигурацию для прикладных алгоритмов, оценить масштабируемость
    \item[\faHourglassStart] Поддержка атомарных операций
    \begin{itemize}
      \item Мы, скорее всего, на стороне компиляторной инфраструктуры
      \item Команда Vortex говорит, что работает над поддержкой на уровне RTL, но пока без конкретных сроков
    \end{itemize}
  \end{itemize}
\end{frame}


\end{document}
