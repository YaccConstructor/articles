\documentclass{beamer}
\usepackage{beamerthemesplit}
\usetheme{SPbGU}
\usepackage{pdfpages}
\usepackage{amsmath}
\usepackage{cmap}
\usepackage[T2A]{fontenc} 
\usepackage[utf8]{inputenc}
\usepackage[english,russian]{babel}
\usepackage{indentfirst}
\usepackage{amsmath}
\usepackage{dot2texi}
\usepackage{tikz}
\usepackage{graphicx}
\usepackage{mathtools}


\usetikzlibrary{shapes,arrows}
\title[]{Введение в синтаксический анализ}
\subtitle[Лекция 3]{$LL$-грамматики}
\institute[СПбГУ]{
Санкт-Петербургский государственный университет \\
Математико-Механический факультет \\
Кафедра системного программирования }

\author[Григорьев Семён]{Григорьев Семён}

\date{22 февраля 2012г.}

\begin{document}
{
    \begin{frame}
        \begin{center}
        {\includegraphics[width=1cm]{SPbGU_Logo.png}}
        \end{center}

        \titlepage

    \end{frame}
}

\begin{frame}
	\transwipe[direction=90]
	\frametitle{LL(k)-грамматики}
	\begin{block}{Definition}
		Пусть $G = (V_N, V_T, P, S)$ -- контекстно-свободная грамматика. Определим функцию $FIRST_k^G (\alpha) = \{w \in V_T | $ либо $ |w| < k $ и $ \alpha \xRightarrow[G]{*}w, $ либо $ |w| = k$ и $\alpha \xRightarrow[G]{*} wx$ для некоторой цепочки $ x \in V_T \}$. \\ Здесь $k \geq 0$ -- целое, $\alpha \in  (V_N \cup V_T )^*$.
	\end{block}
\end{frame}

\begin{frame}
	\transwipe[direction=90]
	\frametitle{LL(k)-грамматики}
	\begin{block}{Definition}
		Пусть $G = (V_N , V_T , P, S )$ -- контекстно-свободная грамматика. Говорят, что $G$ есть $LL(k )$-грамматика для некоторого
фиксированного $k$, если для любых двух левосторонних выводов вида 
    \begin {enumerate}
        \item $S \xRightarrow[lm]{*} wA\alpha \xRightarrow[lm]{} w \beta \alpha \xRightarrow[lm]{*} wx$
        \item $ S \xRightarrow[lm]{*} wA\alpha \xRightarrow[lm]{} w \gamma\alpha \xRightarrow[lm]{*} wy $
    \end{enumerate}
в которых $FIRST_k^G (x) = FIRST_k^G (y)$, имеет место равенство $\beta = \gamma$.
	\end{block}
	\begin{block}{Definition}
		Говорят, что контекстно-свободная грамматика $G$ есть $LL$-грамматика, если она $LL(k )$ для некоторого $k \geq 0$.
	\end{block}
\end{frame}

\begin{frame}
	\transwipe[direction=90]
	\frametitle{Простые LL(1)-грамматики}
	\begin{block}{Definition}
		Говорят, что контекстно-свободная грамматика $G$ является простой $LL(1)$-грамматикой, если в ней нет $\varepsilon$ -правил, и все
альтернативы для каждого нетерминала начинаются с терминалов и притом различных.
	\end{block}
\end{frame}

\begin{frame}
	\transwipe[direction=90]
	\frametitle{Свойства $LL(k)$-грамматик}
	\begin{block}{Theorem}
		Чтобы контекстно-свободная грамматика $G = (V_N , V_T , P , S )$ была $LL(k )$-грамматикой, необходимо и достаточно, чтобы
		\begin{center}
            $FIRST_k^G (\beta \alpha) \cap FIRST_k^G (\gamma \alpha) = \varnothing$
        \end{center}

для всех $\alpha, \beta , \gamma$ , таких, что существуют правила $A \rightarrow \beta , A \rightarrow \gamma \in P, \beta \neq \gamma$ и существует вывод $S \xRightarrow[lm]{*} wA\alpha$ .
	\end{block}
\end{frame}

\begin{frame}
	\transwipe[direction=90]
	\frametitle{$FOLLOW_k^G$}
	\begin{block}{Definition}
		Пусть $G = (V_N , V_T , P , S )$ -- контекстно-свободная грамматика, и $\beta \in V^{*}$ . Определим функцию 
		\begin{center}
		    $FOLLOW_k^G (\beta ) = \{w \in V_T |S \xRightarrow[G]{*} \gamma \beta \alpha, w \in FIRST_k^G (\alpha)\}$
	    \end{center}
	    Здесь $k \geq 0$ -- целое.
	\end{block}
\end{frame}

\begin{frame}
	\transwipe[direction=90]
	\frametitle{Свойство $LL(1)$-грамматик}
	\begin{block}{Theorem}
		Чтобы контекстно-свободная грамматика $G = (V_N , V_T , P , S )$ была $LL(1)$-грамматикой, необходимо и достаточно, чтобы
        \begin{center}
            $FIRST_1^G (\beta FOLLOW_1^G (A)) \cap FIRST_1^G (\gamma FOLLOW_1^G (A)) = \varnothing$
        \end{center}
        для всех $A \in V_N , \beta , \gamma \in (V_N \cup V_T )^{*}$ , таких, что существуют правила $A \rightarrow \beta , A \rightarrow \gamma \in P, \beta \neq \gamma$ и существует вывод $S \xRightarrow[lm]{*} wA\alpha$ .\\
        Пояснение: $FIRST_1^G (W) =\bigcup_{ s \in W} FIRST_1^G (s)$.
	\end{block}
\end{frame}

\begin{frame}
	\transwipe[direction=90]
	\transwipe[direction=90]
	\frametitle{Свойство $LL(1)$-грамматик}
	\begin{block}{Consequence}
		Эту теорему можно переформулировать следующим образом: КС-грамматика $G$ является $LL(1)$-грамматикой тогда и только тогда, когда для каждого множества $A$-правил: $A \rightarrow \alpha_1 |\alpha_2 |...|\alpha_n$ -- выполняются следующие условия:
        \begin{enumerate}
            \item $FIRST_1^G (\alpha_i ) \cap FIRST_1^G (\alpha_j ) = \varnothing$, при $i \neq j , 0 \leq i , j \leq n$;
            \item если $\alpha_i = \varepsilon$ , то $FIRST_1^G (\alpha_j ) \cap FIRST_1^G (A) = 0$, для всех $j : 0 \leq j \leq n, j \neq i$.
        \end{enumerate}
	\end{block}	
\end{frame}

\begin{frame}
	\transwipe[direction=90]
	\frametitle{Леворекурсивные и $LL(k )$-грамматики}
	\begin{block}{Theorem}
		Если $G = (V_N , V_T , P , S )$ -- контекстно-свободная грамматика, и $G$ – леворекурсивна, то $G$ -- не $LL(k )$-грамматика ни при каком $k$.
	\end{block}
	\pause
	\begin{block}{Consequence}
		Мы имеем два достаточных признака для того, чтобы считать КС-грамматику не $LL$-грамматикой. Это -- неоднозначность и леворекурсивность.
	\end{block}
\end{frame}

\end{document}
