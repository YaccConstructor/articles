\documentclass{beamer}
\usepackage{beamerthemesplit}
\usepackage{wrapfig}
\usetheme{SPbGU}
\usepackage{pdfpages}
\usepackage{amsmath}
\usepackage{cmap} 
\usepackage[T2A]{fontenc} 
\usepackage[utf8]{inputenc}
\usepackage[english,russian]{babel}
\usepackage{indentfirst}
\usepackage{amsmath}
\usepackage{tikz}
\usepackage{multirow}
\usepackage[noend]{algpseudocode}
\usepackage{algorithm}
\usepackage{algorithmicx}
\usepackage{listings}
\usepackage{pifont}% http://ctan.org/pkg/pifont
\usepackage{xcolor}
\usepackage{mdframed}
\usepackage{multicol}
\newcommand{\cmark}{\ding{51}}%
\newcommand{\xmark}{\ding{55}}%

\usepackage{epstopdf}
\usepackage{forest}
\usetikzlibrary{shapes,arrows}
\usepackage{fancyvrb}
\newtheorem{rutheorem}{Theorem}
\beamertemplatenavigationsymbolsempty
\setbeamertemplate{itemize items}[circle]

\lstdefinelanguage{ocaml}{
%keywords={fresh, let, begin, end, in, match, type, and, fun, function, try, with, when, class, 
%object, method, of, rec, %repeat, 
%until, while, not, do, done, as, val, inherit, 
%new, module, sig, deriving, datatype, struct, if, then, else, open, private, virtual, ostap},
sensitive=true,
basicstyle=\small,
commentstyle=\small\itshape\ttfamily,
keywordstyle=\ttfamily\underbar,
identifierstyle=\ttfamily,
basewidth={0.5em,0.5em},
columns=fixed,
fontadjust=true,
literate={->}{{$\;\;\to\;\;$}}1
         {===}{{$\equiv$}}1
         {&&&}{{$\wedge$}}1
         {|||}{{$\vee$}}1
         {fresh}{{$\exists$}}1,
morecomment=[s]{(*}{*)}
}

\lstset{
basicstyle=\small,
identifierstyle=\ttfamily,
keywordstyle=\bfseries,
commentstyle=\scriptsize\rmfamily,
basewidth={0.5em,0.5em},
fontadjust=true,
%escapechar=~,
language=ocaml,
mathescape=true,
moredelim=[is][\bfseries]{[*}{*]}
}

\definecolor{SadRed}{RGB}{255, 180, 180}
\definecolor{MehYellow}{RGB}{255, 255, 180}
\definecolor{HappyGreen}{RGB}{180, 255, 180}

\newmdenv[backgroundcolor=SadRed, innertopmargin=8,innerbottommargin=1, linecolor=SadRed]{badcode}

\newmdenv[backgroundcolor=MehYellow, innertopmargin=8,innerbottommargin=1, linecolor=MehYellow]{mehcode}

\newmdenv[backgroundcolor=HappyGreen, innertopmargin=8,innerbottommargin=1, linecolor=HappyGreen]{goodcode}

\newcommand{\miniKanren}{\texttt{miniKanren}}

\title[]{Специализация \miniKanren}
\institute[]{
Лаборатория языковых инструментов JetBrains
}

\author[Екатерина Вербицкая]{Екатерина Вербицкая}

\date{14 декабря 2019}

\definecolor{orange}{RGB}{179,36,31}

\begin{document}
{

\begin{frame}
      \begin{center} 
        {\includegraphics[width=1.5cm]{pics/jb.png}} 
      \end{center}

  \titlepage
\end{frame}
}

\begin{frame}[fragile]
  \transwipe[direction=90]
  \frametitle{Реляционное программирование}
 
 \begin{center}
    Программа --- отношение 
 \end{center}

 \vfill
  
  \begin{center}
    \begin{minipage}{4.5cm}
    \begin{lstlisting}[frame=single]  
append$^o$ x y z = 
  (x === [] &&& z === y)
  ||| (fresh h t r
       ( x === h : t 
       &&& z === h : r 
       &&& append$^o$ t y r))
    \end{lstlisting}
    \end{minipage}
    \end{center}
\end{frame}


\begin{frame}[fragile]
  \transwipe[direction=90]
  \frametitle{Направление вычислений}

\begin{center}
  $foo^o \subseteq A \times B$
\end{center}

\vfill

\begin{itemize}
  \item $foo^o \ \alpha \ ? : A \rightarrow [B]$
  \item $foo^o \ ? \ \beta  : B \rightarrow [A]$ --- в ``обратном'' направлении
  \item $foo^o \ ? \ ?      : () \rightarrow [(A \times B)] $ 
\end{itemize}  

\end{frame}

\begin{frame}[fragile]
  \transwipe[direction=90]
  \frametitle{Реляционные интерпретаторы для синтеза программ}

  \begin{center}
    \begin{minipage}{6.2cm}
    \begin{lstlisting}[frame=single]  
eval$^o$ $\subseteq$ Program $\times$ Input $\times$ Output 

eval$^o$ q 1 1 &&&  eval$^o$ q 2 1 &&& 
eval$^o$ q 3 2 &&&  eval$^o$ q 4 3 &&& 
eval$^o$ q 5 5 &&&  eval$^o$ q 6 8
    \end{lstlisting}
    \end{minipage}
    \end{center}
\end{frame}


\begin{frame}[fragile]
  \transwipe[direction=90]
  \frametitle{Реализация реляционных 
  интерпретаторов} 

\begin{center}    
    Lozov, P., Verbitskaia, E. and Boulytchev, D., 2019.
    
    Relational Interpreters for Search Problems. 
\end{center}

\vfill 

  \begin{itemize}
    \item Реализовать функциональный интерпретатор
    \item Транслировать функциональный интерпрератор на \miniKanren
    \item Запустить реляционный интерпретатор в обратном направлении 
  \end{itemize}

  \vfill

  \begin{itemize}
    \item Транслятор генерирует неэффективный код (в некоторых направлениях) 
    \item Специализация помогает избавиться от неэффективности
  \end{itemize}

\end{frame}

\begin{frame}[fragile]
  \transwipe[direction=90]
  \frametitle{Цель и задачи}

\begin{center}
  Разработать методы специализации \miniKanren, которые обеспечивают применимость реляционных интерпретаторов для синтеза
\end{center}

\begin{itemize}
  \item Адаптировать и реализовать конъюнктивную частичную дедукцию для \miniKanren
  \item Адаптировать и реализовать суперкомпилятор для \miniKanren
  \item Исследовать другие стратегии для специализации \miniKanren
  \item Реализовать трансляцию реляционных программ в функциональные 
  \item Сравнить и выбрать самую адекватную стратегию
\end{itemize}
\end{frame}

\begin{frame}[fragile]
  \transwipe[direction=90]
  \frametitle{Конъюнктивная частичная дедукция\footnote{De Schreye, Danny, et al. ``Conjunctive partial deduction: Foundations, control, algorithms, and experiments.'' } для \miniKanren}

\begin{center}
  Иногда не терминируется

  Почти всегда ускоряет программы
\end{center}

  \begin{table}
    \footnotesize
    \centering
    \begin{tabular}{c|c|c|c|c|c|c}
    Path length                   & 5      & 7     & 9      & 11      & 13     & 15        \\
    \hline\hline
    Only conversion               & 0.01  & 1.39 &  82.13 & >300     & ---      & ---     \\
    \hline
    Backward oriented conversion  & 0.01 & 0.37 &  2.68 & 2.91   & 4.88    & 10.63   \\
    \hline
    Conversion and CPD            & 0.01  & 0.06 &  0.34 & 2.66   & 3.65    & 6.22  
    \end{tabular}
    
     \caption{Searching for paths in the graph (seconds)}
        \label{tab:is_path}
    \end{table}

    \begin{table}
      \tiny
      \centering
      \begin{tabular}{c|c|c|c}
      \multirow{ 2}{*}{Terms} &
      f(X, a) & f(a \% b \% nil, c \% d \% nil, L) & f(X, X, g(Z, t))  \\
      \cline{2-4} &
      f(a, X) & f(X \% XS, YS, X \% ZS) & f(g(p, L), Y, Y)  \\
      \hline\hline
      Only conversion               & 0.01  &  >300 & >300 \\
      \hline
      Backward oriented conversion  & 0.01  &  0.11 & 2.26  \\
      \hline
      Conversion and CPD            & 0.01  &  0.07 & 0.90  \\
      \end{tabular}
       \caption{Searching for a unifier of two terms (seconds)}
          \label{tab:uni}
      \end{table}
\end{frame}

\begin{frame}[fragile]
  \transwipe[direction=90]
  \frametitle{Суперкомпиляция для \miniKanren \ (Мария Куклина)}

\begin{itemize}
  \item Full unfold 
  \item Sequential unfold
  \item Random unfold 
  \item Характеристические деревья для частичной дедукции (в процессе)
\end{itemize}

\vfill

\begin{center}
  Иногда не терминируется
  
  Иногда работает быстрее CPD 
\end{center}

\end{frame}

\begin{frame}[fragile]
  \transwipe[direction=90]
  \frametitle{Трансляция из \miniKanren в функциональный язык \ (Ирина Артемьева)}

\begin{itemize}
  \item Учитывается направление вычисления
  \item Унификации превращаются в сопоставления с образцами или let-связывания, вызовы --- в let-связывания для правильного направления
  \item В случае недетерминированных вычислений возвращаем списки
\end{itemize}

\begin{center}
  \begin{minipage}{4.2cm}
  \begin{lstlisting}[frame=single]  
appendo [] y = return y 
appendo (h:t) y = do 
  ty $\leftarrow$ appendo t y
  return (h:ty)
  \end{lstlisting}
  \end{minipage}
  \end{center}

\begin{center}
  На простых программах работает

  Требуется доработка реализации
\end{center}
\end{frame}

\begin{frame}[fragile]
  \transwipe[direction=90]
  \frametitle{Текущие результаты}
\begin{itemize}
  \item Реализована конъюнктивная частичная дедукция для \miniKanren 
  \item Реализован суперкомпилятор для \miniKanren, рассмотрены несколько стратегий анфолдинга
  \item Реализован прототип транслятора реляционных программ в функциональные 
\end{itemize}
\end{frame}

\begin{frame}[fragile]
  \transwipe[direction=90]
  \frametitle{Доклады и преподавание}
  \begin{itemize}
    \item Публикация
    \begin{itemize}
      \item Lozov, P., Verbitskaia, E. and Boulytchev, D., 2019. Relational Interpreters for Search Problems (\miniKanren \ workshop при ICFP)
    \end{itemize} 
    \item Преподавание
    \begin{itemize}
      \item Лекции и практика по формальным языкам (ИТМО, ВШЭ, ЛЭТИ)
      \item Помощь с курсом по компиляторам (CSCenter)
    \end{itemize} 
    \item Открытая лекция в CSClub
    \begin{itemize}
      \item Реляционное программирование
    \end{itemize}
  \end{itemize}


\end{frame}



\end{document}