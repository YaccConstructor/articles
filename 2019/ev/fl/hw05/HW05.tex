\documentclass[12pt]{article}
\usepackage[left=2cm,right=2cm,top=2cm,bottom=2cm,bindingoffset=0cm]{geometry}
\usepackage[utf8x]{inputenc}
\usepackage[english,russian]{babel}
\usepackage{cmap}
\usepackage{amssymb}
\usepackage{amsmath}
\usepackage{url}
\usepackage{hyperref}
\usepackage{pifont}
\usepackage{tikz}
\usepackage{verbatim}

\usetikzlibrary{shapes,arrows}
\usetikzlibrary{positioning,automata}
\tikzset{
  every state/.style={minimum size=0.2cm},
  initial text={}
}

\newenvironment{myauto}[1][3]
{
  \begin{center}
    \begin{tikzpicture}[> = stealth,node distance=#1cm, on grid, very thick]
}
{
    \end{tikzpicture}
  \end{center}
}


\begin{document}
\begin{center} {\LARGE Формальные языки} \end{center}

\begin{center} \Large домашнее задание до 23:59 30.10 \end{center}
\bigskip

\begin{enumerate}
  \item Привести контекстно-свободную грамматику, задающую язык арифметических операций над целыми числами в десятичной записи с операциями $+, -, *, /$ естественными приоритетом операций и ассоциативностью. Выражения могут содержать скобки. Примеры корректных выражений:
  \begin{itemize}
    \item 1 + 2 * 3
    \item (1 - 2) / 3 
    \item 123 
  \end{itemize}
  \item Привести грамматику из предыдущего задания в Нормальную Форму Хомского. В отчете привести грамматику, полученную на каждом из шагов.
  \item Провести синтаксический анализ алгоритмом CYK трех корректных цепочек длины не меньше 7 и одной некорректной цепочки длины не меньше 5, построить деревья вывода. В отчете привести заполненные таблицы и деревья вывода. 
  \end{enumerate}
\end{document}
