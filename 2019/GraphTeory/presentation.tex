%20 min preso!
\documentclass[xcolor=table]{beamer}
\usepackage{beamerthemesplit}
\usepackage{wrapfig}
\usetheme{SPbGU}
\usepackage{pdfpages}
\usepackage{amsmath}
\usepackage{cmap}
\usepackage[T2A]{fontenc}
\usepackage[utf8]{inputenc}
\usepackage[english]{babel}
\usepackage{indentfirst}
\usepackage{amsmath}
\usepackage{tikz}
\usepackage{multirow}
\usepackage[noend]{algpseudocode}
\usepackage{algorithm}
\usepackage{algorithmicx}
\usepackage{fancyvrb}
\usetikzlibrary{calc}
\usetikzlibrary{shapes,arrows}
\usetikzlibrary{arrows,automata}
\usetikzlibrary{positioning}

\usepackage{tabularx}
\newcolumntype{Y}{>{\raggedleft\arraybackslash}X}

\renewcommand{\thealgorithm}{}

\newtheorem{mytheorem}{Theorem}
\renewcommand{\thealgorithm}{}

\newcommand{\tikzmark}[1]{\tikz[overlay,remember picture] \node (#1) {};}
\def\Put(#1,#2)#3{\leavevmode\makebox(0,0){\put(#1,#2){#3}}}

\newcommand{\ltz}{$< 1$}


\tikzset{
    state/.style={
           rectangle,
           rounded corners,
           draw=black, very thick,
           minimum height=2em,
           inner sep=2pt,
           text centered,
           },
}


\tikzset{
    between/.style args={#1 and #2}{
         at = ($(#1)!0.5!(#2)$)
    }
}


\beamertemplatenavigationsymbolsempty

\title[Графы]{Теория графов}
%\subtitle[YaccConstructor]{Parsing techniques for graph analysis}
% То, что в квадратных скобках, отображается в левом нижнем углу.
\institute[СПбГУ]{
Санкт-Петербургский Государственный Университет
}

% То, что в квадратных скобках, отображается в левом нижнем углу.
\author[Семён Григорьев]{Семён Григорьев}

\date{2 ноября, 2019}

\begin{document}
{
\begin{frame}[fragile]
  \centering
  \includegraphics[height=1.5cm]{pictures/SPbGU_Logo.png}
  \titlepage
\end{frame}
}

\begin{frame} \frametitle{Эйлеровость}
  
\begin{definition}[Эйлеров цикл]
Эйлеров цикл --- цикл, содержащий все рёбра графа.
\end{definition}

\begin{definition}[Эйлеров граф]
Эйлеров граф --- связанный граф, содержащий эйлеров цикл.
\end{definition}


\begin{center}
    \begin{tikzpicture}[node distance=2cm, shorten >=1pt,on grid,auto]
    \node[state] (q_1)   {$3$};
    \node[state] (q_0) [below=of q_1] {$0$};
    \node[state] (q_3) [right=of q_0]  {$1$};
    \node[state] (q_2) [below=of q_3]  {$2$};
    \node[state] (q_4) [right=of q_3]  {$4$};
    \path[-]
    (q_0) edge   (q_1)
    (q_0) edge   (q_2)
    (q_1) edge   (q_3)
    (q_1) edge   (q_2)
    (q_1) edge   (q_4)
    (q_3) edge  (q_2)
    (q_2) edge  (q_4)
    ;
    
    \end{tikzpicture}
\end{center}


\end{frame}


\begin{frame} \frametitle{Как искать эйлеров цикл}
  
\begin{theorem}
Связаный граф эйлеров $\iff$ степени всех вершин чётные.
\end{theorem}
\pause
\begin{definition}[Алгоритм Флёри]
Начинаем из произвльной вершины. На каждом шаге выбираем ребро, проходим по нему, удаляем его из графа. Мосты выбираем в последнюю очередь.
\end{definition}
Проверять, не мост ли это --- долго. В итоге $O(|E|^2)$  при наивной реализации.
\pause
\begin{definition}[Алгоритм через поиск циклов]
Эйлеров цикл --- объединение всех простых циклов графа.

Ваш любимый алгоритм поиска циклов.
\end{definition}

\end{frame}


\begin{frame} \frametitle{Гамильтоновость}
  
\begin{definition}[Гамильтонов цикл]
Гамильтонов цикл --- простой цикл, содержащий все вершины графа.
\end{definition}


\begin{definition}[Гамильнонов граф]
Гамильтонов граф --- связанный граф, содержащий гамильтонов цикл.
\end{definition}

Найти гамильтонов цикл на кубе.

\end{frame}


\begin{frame} \frametitle{Как проверять на гамильтоновость}
  
\begin{theorem}[Хватала]
Пусть дан граф $G, \ |V| = n$ и его степенная последовательность $d_1 \leq \ldots \leq d_n$. $G$ --- гамильтонов, если $\forall k: 1 \leq k \leq n/2 : (d_k \leq k) \Rightarrow (d_{n-k} \geq n-k)$.
\end{theorem}
\pause

\begin{theorem}[Оре]
Пусть дан граф $G$, $|V|\geq 3$. Если для любых несмежных вершин $v$ и $u$ $deg(u) + deg(v) \geq |V|$, то $G$ гамильтонов.
\end{theorem}
\pause

\begin{theorem}[Дирака]
Пусть дан граф $G$, $|V|\geq 3$. Если для $\forall v \ deg(v) \geq |V|/2$, то $G$ гамильтонов.
\end{theorem}


\end{frame}


\begin{frame} \frametitle{Как искать гамильтонов цикл}

Задача из NP.
  
\begin{itemize}
\item Перебор с откатами.
\pause
\item Алгебра. $A$ --- матрица смежности, $B$ --- модифицированная матрица смежности: $b[i,j] = x_j$ если есть ребро $(x_i,x_j)$, 0 иначе.
Вычисляем $A*B*B*\ldots *B$ --- матрица гамильтоновых циклов.
\end{itemize}

Сколько раз надо перемножать матрицы?

Найти гамильтоновы циклы в кубе через матричный алгоритм.



\end{frame}


\begin{frame} \frametitle{Эйлеровость и гамильтоновость вместе}

\begin{theorem}
Пусть $G$ --- граф, $L(G)$ --- рёберный граф.
\begin{itemize}
\item Если $G$ --- эйлеров, то $L(G)$ --- эйлеров и гамильтонов
\item Если $G$ --- гамильтонов, то $L(G)$ --- гамильтонов
\end{itemize}
\end{theorem}

\pause

Обратное не верно.

\begin{minipage}{0.49\textwidth}
G: 
\begin{center}
    \begin{tikzpicture}[node distance=2cm, shorten >=1pt,on grid,auto]
    \node[state] (q_1)   {$3$};
    \node[state] (q_0) [below=of q_1] {$0$};
    \node[state] (q_3) [right=of q_0]  {$1$};
    \node[state] (q_2) [right=of q_1]  {$2$};
    \path[-]
    (q_0) edge   (q_1)
    (q_0) edge   (q_2)
    (q_0) edge   (q_3)
    (q_1) edge   (q_2)
    (q_1) edge   (q_3)
    (q_2) edge   (q_3)
    ;
    
    \end{tikzpicture}
\end{center}

\end{minipage}
~\begin{minipage}{0.49\textwidth}
L(G):
\begin{center}
    \begin{tikzpicture}[node distance=1cm, shorten >=1pt,on grid,auto]
    \node[state] (q_0)   {$3$};
    \node[state] (q_5) [below=of q_0] {$0$};
    \node[state,draw=none] (q__0) [right=of q_0] {};
    \node[state] (q_2) [right=of q__0]  {$1$};
    \node[state,draw=none] (q__1) [right=of q_5] {};
    \node[state] (q_3) [right=of q__1]  {$2$};
    \node[state] (q_1) [above=of q__0]  {$4$};
    \node[state] (q_4) [below=of q__1]  {$5$};
    
    \path[-]
    (q_0) edge   (q_1)
    (q_1) edge   (q_2)
    (q_2) edge   (q_3)
    (q_3) edge   (q_4)
    (q_4) edge   (q_5)
    (q_5) edge   (q_0)

    (q_0) edge   (q_2)
    (q_0) edge   (q_4) 

    (q_1) edge   (q_3)
    (q_1) edge   (q_5) 

    (q_2) edge   (q_4)

    (q_3) edge   (q_5) 
    ;
    
    \end{tikzpicture}
\end{center}


\end{minipage}

\end{frame}

%\begin{frame} \frametitle{Подграфы}

%\begin{definition}[Подграф]
%$G_1$ --- подграф $G$, если все вершины и рёбра $G_1$ принадлежат $G$.
%\end{definition}
%\pause

%\begin{definition}[Остовный подграф]
%$G_1$ --- остовный подграф $G$, если $G_1$ --- подграф $G$, содержащий все вершины $G$.
%\end{definition}
%\end{frame}


\begin{frame} \frametitle{Раскраски}

\begin{definition}[Раскраска]
Раскраска --- назначение цветов вершинам.
\end{definition}
\pause

\begin{definition}[Правильная раскраска]
Раскраска называется правильной, если любые две смежные вершины имеют разные цвета.
\end{definition}
\pause

\begin{definition}[Хроматическое число]
Хроматическое число графа $G = X(G)$ --- минимальное число красок, достаточное для того, чтобы правильно раскрасить граф.
\end{definition}
\pause

\begin{definition}
Граф $G$ является $n$-раскрашиваемым, если $X(G) \leq n$.
Граф $G$ является $n$-хроматическим, если $X(G) = n$.
\end{definition}

\end{frame}

\begin{frame} \frametitle{Проверка раскрашиваемости}

Проверить, можно ли граф правильно раскрасить в $k\geq 3$ цветов --- NP-полная задача.
\pause

Как найти минимальное $k$?

\pause

\begin{theorem}[Теорема о 5 красках]
Любой планарный граф 5-раскрашиваем.
\end{theorem}
\pause

\begin{theorem}[Теорема о 4 красках]
Любой планарный граф 4-раскрашиваем.
\end{theorem}
\pause

\begin{theorem}[Грёти]
Любой планарный граф без треугольников 3-раскрашиваем.
\end{theorem}

\end{frame}


\begin{frame} \frametitle{Хроматическая функция}
Считаем, что граф помечен (вершины имеют уникальные метки)
\begin{definition}
Две раскраски различны, если хотя бы одной вершине они сопоставляют разные цвета.
\end{definition}
\pause

\begin{definition}
Раскраска графа $t$ цветами --- раскраска, использующая не более $t$ цветов.
\end{definition}
\pause

\begin{definition}[Хроматическая функция]
Хроматическая функция графа $f(G,t)$ --- число различных раскрасок $G$ $t$ цветами.
\end{definition}
\pause
Хроматическая функция для полного графа?
\end{frame}


\begin{frame} \frametitle{Построение хроматической функции}

\begin{definition}
Элементарный гомоморфизм $\varepsilon$ на графе $G$: отождествляет любые две вершины (стягивает их в одну).
$\varepsilon(G,u,v)$.
\end{definition}
\pause

\begin{theorem}
$f(G,t) = f(G + (u,v), t) + f(\varepsilon(G,u,v),t)$
\end{theorem}
\pause
Этот процесс можно продолжать до полных графов. 

\end{frame}

\begin{frame} \frametitle{Про контекстно-свободные языки и граммтики}


\end{frame}


\begin{frame} \frametitle{Про поиск путей с контекстно-свободными ограничениями}


\end{frame}


\begin{frame} \frametitle{Задачи}

\begin{enumerate}
\item Реализуйте алгоритм для решения задачи достижимости с КС ограничениями через тензорное произведение.
\item Какова временная сложность алгоритма для решения задачи достижимости с КС ограничениями через тензорное произведение относительно размера входного графа и автомата, построенного по грамматике? 
\item Предложите алгоритм преобразования произвольной контекстно-свободной граммтики в нормальную форму Хомского. Оцените увеличение размера грамматики при таком преобразовании.
\item Реализуйте алгоритм минимизации автомата, построенного по контекстно-свободной грамматике. Чем он отличается от алгоритма минимизации обычного конечного автомата?
\end{enumerate}

\end{frame}


\end{document}
