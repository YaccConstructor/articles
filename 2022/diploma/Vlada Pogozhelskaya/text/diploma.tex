% Тут используется класс, установленный на сервере Papeeria. На случай, если
% текст понадобится редактировать где-то в другом месте, рядом лежит файл matmex-diploma-custom.cls
% который в момент своего создания был идентичен классу, установленному на сервере.
% Для того, чтобы им воспользоваться, замените matmex-diploma на matmex-diploma-custom
% Если вы работаете исключительно в Papeeria то мы настоятельно рекомендуем пользоваться
% классом matmex-diploma, поскольку он будет автоматически обновляться по мере внесения корректив
%

% По умолчанию используется шрифт 14 размера. Если нужен 12-й шрифт, уберите опцию [14pt]
%\documentclass[14pt]{matmex-diploma}
\documentclass[14pt]{matmex-diploma-custom}
\usepackage{graphicx}

\usepackage{fontspec}
\usepackage{polyglossia}
\usepackage{amsmath}
\usepackage{amsfonts}
\usepackage{amssymb}

% пакеты из презентации
\usepackage{algpseudocode}
\usepackage{algorithm}
\usepackage{algorithmicx}
\usepackage{pdfpages}

\usepackage{subcaption}
\usepackage{geometry}
\usepackage{amsfonts,latexsym}
\usepackage{amsthm}
\usepackage{amssymb}
\usepackage[utf8]{inputenc} % Кодировка
\usepackage{mathtools}
\usepackage{hyperref}
\usepackage{tikz}
\usepackage{dsfont}
\usepackage{multicol}
\usetikzlibrary{fit,calc,automata,positioning}

\theoremstyle{definition}
\newtheorem{rudefinition}{Definition}[section]
\newtheorem{example}{Example}[section]
\newtheorem{theorem}{Theorem}[section]
\newtheorem{proposition}[theorem]{Proposition}
\newtheorem{lemma}[theorem]{Lemma}
\newtheorem{corollary}[theorem]{Corollary}
\newtheorem{conjecture}[theorem]{Conjecture}
\newtheorem{note}[theorem]{Утверждение}

%code highlight
% \usepackage{listings}
\usepackage{listings, listings-rust}
\usepackage{lipsum}
\usepackage{xcolor}

\usepackage{tikz}
\usetikzlibrary{decorations.pathreplacing,calc,shapes,positioning}

\newcommand\tbox[1]{\tikz[overlay]\node[inner sep=2pt, draw=red, ultra thick, anchor=text, rectangle] {#1};\phantom{#1}}

\definecolor{codegreen}{rgb}{0,0.6,0}
\definecolor{codegray}{rgb}{0.5,0.5,0.5}
\definecolor{codepurple}{rgb}{0.58,0,0.82}
\definecolor{backcolour}{rgb}{1,1,1}
 
\lstdefinestyle{mystyle}{
    backgroundcolor=\color{backcolour},   
    commentstyle=\color{codegreen},
    keywordstyle=\color{magenta},
    numberstyle=\color{codegray},
    stringstyle=\color{codepurple},
    basicstyle=\ttfamily\small,
    breakatwhitespace=false,         
    breaklines=true,                 
    captionpos=b,                    
    keepspaces=true,                 
    numbers=left,                    
    numbersep=5pt,                  
    showspaces=false,                
    showstringspaces=false,
    showtabs=false,                  
    tabsize=2
    % xleftmargin=.2\textwidth,
    % xrightmargin=.2\textwidth
}
 
\lstset{style=mystyle}

\lstnewenvironment{code}[1][]%
{
   \noindent
   \minipage{\linewidth} 
   \vspace{0.5\baselineskip}
   \lstset{basicstyle=\ttfamily\footnotesize,frame=single,#1}}
{\endminipage}

\begin{document}
\filltitle{ru}{
    %% Актуально только для курсовых/практик. ВКР защищаются не на кафедре а в ГЭК по направлению, 
    %%   и к моменту защиты вы будете уже не в группе.
    chair              = {Кафедра, на которой работает научник},
    group              = {18.Б11-мм},
    %
    %% Макрос filltitle ненавидит пустые строки, поэтому обязателен хотя бы символ комментария на строке
    %% Актуально всем.
    title              = {Реализация и экспериментальное исследование алгоритма поиска путей с контекстно-свободными ограничениями в графовой базе данных Neo4j},
    % 
    %% Здесь указывается тип работы. Возможные значения:
    %%   coursework - отчёт по курсовой работе;
    %%   practice - отчёт по учебной практике;
    %%   prediploma - отчёт по преддипломной практике;
    %%   master - ВКР магистра;
    %%   bachelor - ВКР бакалавра.
    type               = {bachelor},
    %
    %% Здесь указывается вид работы. От вида работы зависят критерии оценивания.
    %%   solution - <<Решение>>. Обучающемуся поручили найти способ решения проблемы в области разработки программного обеспечения или теоретической информатики с учётом набора ограничений.
    %%   experiment - <<Эксперимент>>. Обучающемуся поручили изучить возможности, достоинства и недостатки новой технологии, платформы, языка и т. д. на примере какой-то задачи.
    %%   production - <<Производственное задание>>. Автору поручили реализовать потенциально полезное программное обеспечение.
    %%   comparison - <<Сравнение>>. Обучающемуся поручили сравнить несколько существующих продуктов и/или подходов.
    %%   theoretical - <<Теоретическое исследование>>. Автору поручили доказать какое-то утверждение, исследовать свойства алгоритма и т.п., при этом не требуя написания кода.
    kind               = {experiment},
    %
    author             = {ПОГОЖЕЛЬСКАЯ Влада Владимировна},
    % 
    %% Актуально только для ВКР. Указывается код и название направления подготовки. Типичные примеры:
    %%   02.03.03 <<Математическое обеспечение и администрирование информационных систем>>
    %%   02.04.03 <<Математическое обеспечение и администрирование информационных систем>>
    %%   09.03.04 <<Программная инженерия>>
    %%   09.04.04 <<Программная инженерия>>
    %% Те, что с 03 в середине --- бакалавриат, с 04 --- магистратура.
    specialty          = {09.03.04 <<Программная инженерия>>},
    % 
    %% Актуально только для ВКР. Указывается шифр и название образовательной программы. Типичные примеры:
    %%   СВ.5006.2017 <<Математическое обеспечение и администрирование информационных систем>>
    %%   СВ.5162.2020 <<Технологии программирования>>
    %%   СВ.5080.2017 <<Программная инженерия>>
    %%   ВМ.5665.2019 <<Математическое обеспечение и администрирование информационных систем>>
    %%   ВМ.5666.2019 <<Программная инженерия>>
    %% Шифр и название программы можно посмотреть в учебном плане, по которому вы учитесь. 
    %% СВ.* --- бакалавриат, ВМ.* --- магистратура. В конце --- год поступления (не обязательно ваш, если вы были в академе/вылетали).
    programme          = {СВ.5080.2018 <<Программная инженерия>>},
    % 
    %% Актуально только для ВКР, только для матобеса и только 2017-2018 годов поступления. Указывается профиль подготовки, на котором вы учитесь.
    %% Названия профилей можно найти в учебном плане в списке дисциплин по выбору. На каком именно вы, вам должны были сказать после второго курса (можно уточнить в студотделе).
    %% Вот возможные вариканты:
    %%   Математические основы информатики
    %%   Информационные системы и базы данных
    %%   Параллельное программирование
    %%   Системное программирование
    %%   Технология программирования
    %%   Администрирование информационных систем
    %%   Реинжиниринг программного обеспечения
    % profile            = {Системное программирование},
    % 
    %% Актуально всем.
    %supervisorPosition = {проф. каф. СП, д.ф.-м.н., проф.}, % Терехов А.Н.
    supervisorPosition = {доцент кафедры информатики, к.ф.-м.н.,}, % Григорьев С.В.
    supervisor         = {Григорьев С.В.},  
    % 
    %% Актуально только для практик и курсовых. Если консультанта нет, закомментировать или удалить вовсе.
    % consultantPosition = {должность ООО <<Место работы>> степень},
    % consultant         = {К.К. Консультант},
    %
    %% Актуально только для ВКР.
    reviewerPosition   = {Программист ООО "Интеллиджей Лабс"},
    reviewer           = {Вербицкая Е.А.},
}

\filltitle{en}{
    chair              = {Software Engineering},
    group              = {18.B11-mm},
    title              = {Implementation and experimental study of GLL algorithm with Neo4j graph database},
    type               = {bachelor},
    author             = {Vlada Pogozhelskaya},
    % 
    %% Possible choices:
    %%   02.03.03 <<Software and Administration of Information Systems>>
    %%   02.04.03 <<Software and Administration of Information Systems>>
    %%   09.03.04 <<Software Engineering>>
    %%   09.04.04 <<Software Engineering>>
    %% Те, что с 03 в середине --- бакалавриат, с 04 --- магистратура.
    specialty          = {09.03.04 <<Software Engineering>>},
    % 
    %% Possible choices:
    %%   СВ.5006.2017 <<Software and Administration of Information Systems>>
    %%   СВ.5162.2020 <<Programming Technologies>>
    %%   СВ.5080.2017 <<Software Engineering>>
    %%   ВМ.5665.2019 <<Software and Administration of Information Systems>>
    %%   ВМ.5666.2019 <<Software Engineering>>
    programme          = {СВ.5080.2018 <<Software Engineering>>},
    % 
    %% Possible choices:
    %%   Mathematical Foundations of Informatics
    %%   Information Systems and Databases
    %%   Parallel Programming
    %%   System Programming
    %%   Programming Technology
    %%   Information Systems Administration
    %%   Software Reengineering
    % profile            = {Software Engineering},
    % 
    %% Note that common title translations are:
    %%   кандидат наук --- C.Sc. (NOT Ph.D.)
    %%   доктор ... наук --- Sc.D.
    %%   доцент --- docent (NOT assistant/associate prof.)
    %%   профессор --- prof.
    supervisorPosition = {C.Sc., docent},
    supervisor         = {Semyon Grigorev},
    % 
    % consultantPosition = {position at ``Company'', degree if present},
    % consultant         = {C.C. Consultant},
    %
    reviewerPosition   = {Software Engineer at IntelliJ Labs Co. Ltd},
    reviewer           = {Ekaterina Verbitskaia},
}
% Год, город, название университета и факультета предопределены,
% но можно и поменять.
% Если англоязычная титульная страница не нужна, то ее можно просто удалить.
% \filltitle{ru}{
%     chair              = {Программная инженерия\\ \vspace{5mm}Системное программирование},
%     title              = {Реализация и экспериментальное исследование алгоритма поиска путей с контекстно-свободными ограничениями в графовой базе данных Neo4j},
%     % Здесь указывается тип работы. Возможные значения:
%     %   coursework - Курсовая работа
%     %   diploma - Диплом специалиста
%     %   master - Диплом магистра
%     %   bachelor - Диплом бакалавра
%     type               = {bachelor},
%     position           = {студента},
%     group              = 471,
%     author             = {Погожельская Влада Владимировна},
%     supervisorPosition = {к.\,ф.-м.\,н., доцент кафедры информатики},
%     supervisor         = {Григорьев Семен Вячеславович},
%     consultantPosition = {},
%     consultant         = {},
%     reviewerPosition   = {Программист в ООО "Интеллиджей Лабс"},
%     reviewer           = {Вербицкая Екатерина Андреевна},
%     chairHeadPosition  = {...},
%     chairHead          = {...},
% %   university         = {Санкт-Петербургский Государственный Университет},
% %   faculty            = {Математико-механический факультет},
% %   city               = {Санкт-Петербург},
% %   year               = {2013}
% }
% \filltitle{en}{
%     chair              = {Software Engineering},
%     title              = {Implementation and experimental study of GLL algorithm with Neo4j graph database},
%     author             = {Vlada Pogozhelskaya},
%     supervisorPosition = {Associate professor, Ph.\,D.},
%     supervisor         = {Semyon Vyacheslavovich Grigorev},
%     consultantPosition = {},
%     consultant         = {},
%     reviewerPosition   = {Software engineer at IntelliJ Labs Co. Ltd},
%     reviewer           = {Ekaterina Andreyevna Verbitskaia},
%     chairHeadPosition  = {...},
%     chairHead          = {...},
% }

\newcommand\todo[1]{{\color{violet}#1}}
\newcommand\db[1]{{\color{red}#1}}
\newcommand\question[1]{{\color{cyan}#1}}

\maketitle
\tableofcontents
% У введения нет номера главы
\section{Introduction}

Scalable high-performance graph analysis is an actual challenge.
There is a big number of ways to attack this challenge~\cite{Coimbra2021} and the first promising idea is to utilize general-purpose graphic processing units (GPGPU).
Such existing solutions, as CuSha~\cite{10.1145/2600212.2600227} and Gunrock~\cite{7967137} show that utilization of GPUs can improve the performance of graph analysis, moreover it is shown that solutions may be scaled to multi-GPU systems.
But low flexibility and high complexity of API are problems of these solutions.

The second promising thing which provides a user-friendly API for high-performance graph analysis algorithms creation is a GraphBLAS API~\cite{7761646} which provides linear algebra based building blocks to create graph analysis algorithms.
The idea of GraphBLAS is based on a well-known fact that linear algebra operations can be efficiently implemented on parallel hardware.
Along with that, a graph can be natively represented using matrices: adjacency matrix, incidence matrix, etc.
While reference CPU-based implementation of GraphBLAS, SuiteSparse:GraphBLAS~\cite{10.1145/3322125}, demonstrates good performance in real-world tasks, GPU-based implementation is challenging.

One of the challenges in this way is that real data are often sparse, thus underlying matrices and vectors are also sparse, and, as a result, classical dense data structures and respective algorithms are inefficient. 
So, it is necessary to use advanced data structures and procedures to implement sparse linear algebra, but the efficient implementation of them on GPU is hard due to the irregularity of workload and data access patterns.
Though such well-known libraries as cuSPARSE show that sparse linear algebra operations can be efficiently implemented for GPGPU, it is not so trivial to implement GraphBLAS on GPGPU. 
First of all, it requires \textit{generic} sparse linear algebra, thus it is impossible just to reuse existing libraries which are almost all specified for operations over floats.
The second problem is specific optimizations, such as masking fusion, which can not be natively implemented on top of existing kernels.
Nevertheless, there is a number of implementations of GraphBLAS on GPGPU, such as GraphBLAST~\cite{yang2019graphblast}, GBTL~\cite{7529957}, which show that GPGPUs utilization can improve the performance of GraphBLAS-based graph analysis solutions.
But these solutions are not portable because they are based on Nvidia Cuda stack.
Moreover, the scalability problem is not solved: all these solutions support only single-GPU, not multi-GPU computations.

To provide portable GPU implementation of GraphBLAS API we developed a \textit{SPLA} library\footnote{Source code available at: \url{https://github.com/JetBrains-Research/spla}}.
This library utilizes OpenCL for GPGPU computing to be portable across devices of different vendors.
Moreover, it is initially designed to utilize multiple GPGPUs to be scalable.
To sum up, the contribution of this work is the following.
\begin{itemize}
    \item Design of portable GPU GraphBLAS implementation proposed. The design involves the utilization of multiple GPUS. Additionally, the proposed design is aimed to simplify library tuning and wrappers for different high-level platforms and languages creation. 
    \item Subset of GraphBLAS API, including such operations as masking, matrix-matrix multiplication, matrix-matrix e-wise addition, is implemented. The current implementation is limited by COO and CSR matrix representation format and uses basic algorithms for some operations, but work in progress and more data formats will be supported and advanced algorithms will be implemented in the future.
    \item Preliminary evaluation on such algorithms as breadth-first search (BFS) and triangles counting (TC), and real-world graphs shows portability across different vendors and promising performance: for some problems Spla is comparable with GraphBLAST. Surprisingly, for some problems, the proposed solution on embedded Intel graphic card shows better performance than SuiteSparse:GraphBLAS on the respective CPU. At the same time, the evaluation shows that further optimization is required.
\end{itemize} 
\section*{Problem statement}\label{ps}
The aim of the work is the practical evaluation of whether any performance enhancement could be brought %by partial evaluation technique
by partially evaluating memory accesses through the utilization of AnyDSL framework, compared to CUDA implementations, considering GPU microarchitecture details that affect the result. In order to achieve the aim, the following objectives have been set.
\begin{itemize}
    \item Implement experimental scenarios in both AnyDSL and CUDA.
    \item Collect relevant datasets for the evaluation to be more practical.
    \item Perform the evaluation and analyze the results.
\end{itemize}

More specifically, the work performs the evaluation on string matching and convolutional filtering scenarios, providing some relevant CUDA assembly examples to ground the effects being observed.

% Firstly, the hypothesis of whether GPU-based string pattern matching program performance speed up could be achieved via partially evaluating memory accesses, should have been verified. A modern GPU has different types of memory, varying by access latency. In order to achieve maximal performance every type of memory should be utilized carefully, satisfying alignment and access patterns requirements. Moreover, several cache levels are extensively used to mitigate the latency, and, to some extent, caches could keep up the performance of an application, even if a proper access pattern is hard to achieve. There is a partial evaluator, being developed as part of AnyDSL framework~\cite{LeiBa}, with the support for generation of specialized Nvidia CUDA C code. It has been utilized to verify the hypothesis.   

% Next, the bottlenecks that arise during string pattern matching program specialization should be identified. Particular program transformations could potentially hurt further parallelization or simply not achieve expected effects. Thus, string pattern matching algorithms specialization should be examined using available partial evaluators.

% Then the partial evaluator should be implemented considering the identified bottlenecks either from scratch or as an extension for available ones.

% Finally, the obtained partial evaluator efficiency should be evaluated through performance comparison between specialized programs and manually fine-tuned ones.   
\section{Related work \& background}
This section includes basic notation and definitions in graph theory and formal language theory which are used in this work. Also, the further description of both the theoretical part of the GLL-based CFPQ algorithm and its implementation are provided.

\subsection{Basic Definitions of Formal Languages}

In this work, the context-free grammars are used as path constraints, thus context-free languages and grammars are defined in this subsection.

\begin{rudefinition}A \emph{context-free grammar} is a tuple $G= \langle N, \Sigma, P, S \rangle$, where
\begin{itemize}
    \item $N$ is a finite set of nonterminals
    \item $\Sigma$ is a finite set of terminals, $N \cap \Sigma = \varnothing$
    \item $P$ is a finite set of productions of the form $A \to \alpha$, where $A \in N,\ \alpha \in (N \cup \Sigma)^*$
    \item $S$ $\in$ $N$.
\end{itemize} \qed
\end{rudefinition}

We use the conventional notation $A \Rightarrow^* w$ to denote, that a
word $w \in \Sigma^*$ can be derived from a non-terminal $A$ using some sequence of production rules from $P$.

\begin{rudefinition} A \emph{context-free language} is a language generated by a con-text-free grammar $G$:
\begin{align*}
     L(G) = \{w \in \Sigma^* \mid S \Rightarrow^* w \}
\end{align*}
\end{rudefinition}

% \begin{rudefinition} A \emph{context-free language with a specified starting non-terminal $S$} is a set of strings that can be generated from $S$ by a context-free grammar $G$:
% \begin{align*}
%      L(G_S) = \{w \in \Sigma^* \mid S \Rightarrow^* w \}
% \end{align*}
% \end{rudefinition}

\subsection{Basic Definitions of Graph Theory}
In a simplified way, the Neo4j graph database uses a labeled directed graph as a data model. It can be defined as follows.

\begin{rudefinition} \emph{Labeled directed graph} is a tuple $D = \langle V, E, T \rangle$, where
\begin{itemize}
    \item $V$ is a finite set of vertices. For simplicity, we assume that the vertices are natural numbers from $0$ to $|V|-1$.
    \item $T$ is a set of labels on edges.
    \item $E \subseteq V \times T \times V$ is a set of edges.
\end{itemize} \qed
\end{rudefinition}

\begin{rudefinition}
Path $\pi$ in the graph $D = \langle V, E, T \rangle$ is a finite sequence of edges $(e_0, e_1, ..., e_{n-1})$, where $\forall~ j,~ 0 \leq j \leq n - 1: e_j=(v_j,t_j,v_{j+1}) \in E$.

We denote the set of all paths in the graph $D$ as $\pi(D)$. \qed
\end{rudefinition}

\subsection{Context-free Path Querying}
Now, we can define context-free path querying problems. Let be:
\begin{itemize}
      \item a context-free grammar $G=\langle N, \Sigma, P, S \rangle$;
      \item a directed graph $D=\langle V, E, T \rangle$, where $V$ is the set of vertices of the graph, $ E \subseteq V \times T \times V $ is the set of edges, $ T \subseteq \Sigma $ is the set of labels on edges, where each label is a terminal symbol of the grammar $G$;
\item a set of start vertices $V_S \subseteq V$ and final vertices \mbox {$V_F \subseteq V$.}
\end{itemize}

Consider a path in the graph $D$: $$\pi = (e_0, e_1, \cdots, e_{n - 1}), $$ where $ e_k = (v_{k}, t_k , v_{k+1}), ~ \forall~k,~ 0 \leq k \leq n - 1 ~e_k \in E$.
To path in the graph the word $ l(\pi) = t_0t_1 \cdots t_{n_1} $ is associated --- the concatenation of the labels on the edges of this path.

In the introduced notation, the following problems can be formulated.

\begin{itemize}
     \item \textbf{The problem of a path querying in a graph with context-free constraints} consists in finding all paths in the graph such that $l(\pi) \in L (G)$ and $v_0 \in V_S, ~v_n \in V_F$.
    
     \item \textbf{The problem of reachability in a graph with context-free constraints} consists in finding a set of pairs of vertices for which there is a path with a beginning and an end at these vertices, such that the word composed of labels of the edges of the path  belongs to the given language: $ \{(v_i, v_j) ~ | ~ \exists ~ l (\pi) \in L (G) $ and $ v_0 \in V_S, ~ v_n \in V_F \} $.
\end{itemize}

It should be noted that it is often necessary to identify complex dependencies in a graph data model. So, according to the context and application area, both variants of the above problems are of practical importance. 

For each problem there are two variants of set of starting vertices: the set may consist of all vertices of a graph or may consist only a particular vertices of interest. The first variant is called all-pairs context-free path querying problem and the second is called a multiple-source (and a single-source as a partial case) context-free path querying problem.

\subsection{Generalized LL Parsing Algorithm}
One of the common parsing techniques is the LL(k) algorithm~\cite{10.5555/1076440}, that performs top-down analysis with a lookahead. It means that the decision about which production of the grammar should be applied is based on looking at the $ k $ following character from the current one. To choose the right production rule at this step algorithm supports a parsing table, where the information for parsing the current non-terminal is stored. However, it can be applied only to a subset of the context-free grammars class and does not support ambiguous context-free grammars or grammars with left recursion in derivation.

Top-down analysis algorithms are relatively easier to implement and debug, because it fully matches the structure of the grammar. For this reason, to extend the parsing power of above-mentioned technique there was proposed~\cite{SCOTT2010177} the generalized LL (GLL) algorithm. Also GLL can handle ambiguous grammars.
In case of LL(k) algorithm may arise the situation when it is impossible to determine which production should be applied in the current state of the parsing process ~\cite{10.1145/800105.803402}. To solve this issue the GLL algorithm maintains a queue of descriptors. Each descriptor is a structure that describes the current state of the analyzer. Thus, using a queue of descriptors allows one to consider all possible transitions during the operation of the parser.

The parsing table for the generalized GLL algorithm can store multiple alternatives for parsing the current non-terminal. In this case, descriptor duplication can occur. For efficient storage and reuse of many different descriptors, GLL uses a specific structure --- Graph Structured Stack (GSS)~\cite{10.5555/1623611.1623625}.

To represent the result, GLL provides the Shared Packed Parse Forest (SPPF) structure~\cite{SCOTT20131828}, which contains all derivation trees for all paths satisfying the specified language.

\subsection{GLL-based CFPQ Algorithm}
As it was showed, classical GLL parsing technique can be used to solve context-free language constrained path problem. It means that such technique can be used to proceed graph input. Previously, the algorithm was generalized from linear input to graph processing, as was described in \cite{10.1145/3166094.3166104}.

To do this, the following modifications were proposed.
\begin{itemize}
\item A query has became a triple: a set of initial vertices, a set of final vertices, and a grammar.
\item An initial set of descriptors must include all the start vertices of the graph.
\item At the step of transition to the next character, it is necessary to support all possible transition options that correspond to all outgoing edges of the vertex.
\item If parsing is completed, it is necessary to check whether the final vertex in the parsing belongs to the set of final vertices of the graph.
\end{itemize}

The described principles of the generalized GLL algorithm are important for understanding the features of its implementation, which will be described below.

The implementation of the algorithm is based on the Iguana project which is written in Java. This library provides the modified GLL algorithm. The advantage of  Iguana project is that it uses a more efficient GSS for GLL parsing. In addition, it does not affect the worst-case cubic run time and space complexities of GLL parsing.
 
Under this work, it is important to pay attention to the following changes that were made to the workflow of the GLL algorithm to unable graph processing.

\begin{itemize}
    \item In order to support graph processing, the abstraction of an input data was changed. The new implementation of the $Input$ interface has been added. Now it is represented as a graph adjacency list, a set of start and final vertices of the resulting paths.
    \item There can be multiple start vertices for a graph input, unlike a linear input. So, also the initialization of the descriptor queue was modified. In case of processing a descriptor with slot $(N \rightarrow \alpha.x\beta)$, where $x$ is a terminal, the nextSymbols method was used. It took an index $i$ in the input string and returns an index $j$ such that the substring of the input string from $i$ to $j - 1$ matches $x$. Thus, $ j $ is the index in the input string from which the parsing  should continue by going to the slot $(N \rightarrow \alpha x.\beta)$. Considering the graph input there can be several similar positions. Therefore, the signature of this method has been changed. Now it returns a list of identifiers.
\end{itemize}

As far as the original GLL is aimed to handle arbitrary context-free grammars, this solution can handle arbitrary grammars too. It makes the solution less restrictive with regard to a query specification language, thus being more user-friendly.

%As a storage for graphs,the Neo4j graph database was used. This is the most commonly used graph DBMS. Neo4j supports Cypher query language and represents data as nodes (vertices) and relations between them (edges). Vertices and edges can be labeled. Neo4j is an open source project and, like Iguana, implemented in Java. The modified algorithm has been integrated with Neo4j using the Native Java API.


\section{Modified Valiant's algorithm}

In this section, we propose how to rearrange the order in which submatrices are processed in the algorithm.
The different order improves the independence of submatrices handling and facilitates the implementation of parallel submatrix processing.

\subsection{Layered submatrices processing}

We propose to divide the parsing table into layers of disjoint submatrices of the same size (see Figure~\ref{fig2}).
Such division is possible because the derivation of a substring of the fixed length does not depend on either left or right contexts.
An appropriate order of substrings processing guarantees the disjointness of submatrices which form a layer.
Each layer consists of square matrices which size is a power of 2.
The layers are computed successively in the bottom-up order.
Each matrix in the layer can be handled independently, which facilitates parallelization of layer processing.

\begin{figure}[h]
\vspace{3mm}
 \begin{center}
 \includegraphics[width=12cm]{pictures/modivis2.pdf}
    \caption{An example of the modification of Valiant's algorithm}
    \label{fig4}
 \end{center}
\vspace{-8mm}
\end{figure}

Figure~\ref{fig4} demonstrates the modified algorithm.
The lowest layer (submatrices of size 1) has already been computed.
The second layer is filled in in the steps 1-2.
Although the original algorithm computes the same matrix, the modified one needs only two steps using parallel computation of submatrix products.

The modified version of Valiant's algorithm is presented in Listing~\ref{algo:modified}.
The procedure \textit{main()} computes the lowest layer $(T_{l, l+1})$, and then divides the table into layers, and computes them with the \textit{completeVLayer()} function.
Thus, \textit{main()} computes all elements of parsing table $T$.

We define \textit{left(subm)}, \textit{right(subm)}, \textit{top(subm)}, \textit{bottom(subm)}, \textit{rightgrounded(subm)} and \textit{leftgrounded(subm)} functions which return the submatrices for matrix $\textit{subm} = (l, m, l', m')$ according to the original Valiant's algorithm (Figure~\ref{fig2}).


\begin{algorithm}[!h]
\SetAlgoNoLine
\KwIn{$G = (\Sigma, N, R, S), w = a_{1} \dots a_{n}, n \geq 1, n + 1 = 2^p, a_{i} \in \Sigma$ }
\underline{main()}{:}{

 \lFor {$l \in \{1, \ldots, n \}$}{$T_{l, l + 1} = \{A | A \rightarrow a_{l + 1} \in R\}$}
 \For{$1 \le i < p - 1 $}{
 \textit{layer = constructLayer($i$)}\;
 \textit{completeVLayer(layer)}
 }
 accept if and only if $S \in T_{0, n}$
 \BlankLine
 }

\underline{constructLayer(i)}{:}{
 \BlankLine
 $\{(k2^i, (k+1)2^i, (k + 1)2^i, (k+2)2^i) \, |\, 0 \le k < 2^{p - i} - 1\}$
 \BlankLine
    }
\underline{completeLayer(M)}{:}{
\BlankLine
\If {$\forall (l, m, l', m') \in M \quad (m - l = 1)$}{\lFor{$ (l, m, l', m') \in M$}{$T_{l, l'} = f(P_{l, l'})$}}
\Else{
\textit{completeLayer($\{\textit{bottom(subm)}\, |\,\textit{subm} \in M \})$}\;
\textit{completeVLayer(M)}
}
\BlankLine
}

\underline{completeVLayer(M)}{:}{
 \BlankLine
 $\textit{multiplicationTasks}_1 = \linebreak
    \{\textit{left(subm)}, \textit{leftgrounded(subm)}, \textit{bottom(subm)}\, 
    |\,\textit{subm} \in M \} \cup \linebreak  \{\textit{right(subm)}, \textit{bottom(subm)}, \textit{rightgrounded(subm)}\, |\,\textit{subm} \in M\}$\;
 \BlankLine
 \textit{multiplicationTask$_2$} = $\{\textit{top(subm)}, \textit{leftgrounded(subm)}, \textit{right(subm)}\, |\,\textit{subm} \in M\}$\;
 \BlankLine
 \textit{multiplicationTask$_3$} = $\{\textit{top(subm)}, \textit{left(subm)}, \textit{rightgrounded(subm)}\, |\,\textit{subm} \in M\}$\;
 \BlankLine
 \textit{performMultiplications(multiplicationTask$_1$)}\;
 \textit{completeLayer($\{\textit{left(subm)}\, |\,subm \in M \} \cup \{\textit{right(subm)}\, |\,\textit{subm} \in M \}$)}\;
 \textit{performMultiplications(multiplicationTask$_2$)}\;
 \textit{performMultiplications(multiplicationTask$_3$)}\;
 \textit{completeLayer($\{top(subm)\, |\,subm \in M \}$)}

 }
 \BlankLine

 \underline{performMultiplication(tasks)}{:}{\\
 \lFor{$ (m, m1, m2) \in \textit{tasks}$}{$P_{m} = P_{m} \cup (T_{m1} \times T_{m2})$}
 }

\caption{Parsing by Matrix Multiplication: Modified Version}
\label{algo:modified}
\end{algorithm}


The procedure \textit{completeVLayer(M)} takes an array of disjoint submatrices $M$ which represents a layer.
For each \textit{subm = (l, m, l', m') $\in M$} this procedure computes \textit{left(subm), right(subm), top(subm)}.
The procedure assumes that the elements of \textit{bottom(subm)} and $T_{i, j}$ for all $i$ and $j$ such that $l \leq i < j < m$ and $  l' \leq i < j < m'$ are already constructed.
Also it is assumed that the current value of
$P_{i, j} =  \{ (B, C) | \exists k, (m \le k < l'), a_{i + 1} \dots a_{k} \in L_G(B), a_{k + 1} \dots a_{j} \in L_G(C)\} $ for all $i$ and $j$ such that $l \leq i < m$ and $l' \leq j < m'$.

The procedure \textit{completeLayer(M)} also takes an array of disjoint submatrices $M$, but unlike the previous one, it computes $T_{i, j}$ for all $(i, j) \in subm$.
This procedure requires exactly the same assumptions on $T_{i, j}$  and $P_{i, j}$  as in the previous case.

In other words, \textit{completeVLayer(M)} computes the entire layer \textit{M} \linebreak and \textit{completeLayer($M_{2}$)} is a helper function which is necessary for computation of smaller square submatrices $subm_{2} \in M_{2}$ inside of \textit{M}.

Finally, the procedure \textit{performMultiplication(tasks)}, where \textit{tasks} is an array of triples of submatrices, performs the basic step of the algorithm: matrix multiplication.
It is worth mentioning that $|tasks| \ge 1$ and each task can be computed independently, while the original algorithm handles one \textit{task} per step sequentially.
So, the practical implementation of this procedure can easily utilize different techniques of parallel array processing.

\subsection{Correctness and complexity}

We provide the proof of correctness and time complexity for the proposed modification in this section.
To do it we should prove correctness of subprocedure \textit{completeLayer}.

\begin{lemma}
Let $M$ be a layer. If for all $(l, m, l', m') \in M$:
\begin{enumerate}
  \item $T_{i, j} = \{ A |  a_{i + 1} \dots a_{j} \in L_G(A)\}$ for all $i$ and $j$ such that $l \leq i < j < m$ and $l' \leq i < j < m'$;
  \item $P_{i, j} =  \{ (B, C) |\exists k, (m \le k < l'): a_{i + 1} \dots a_{k} \in L_G(B), a_{k + 1} \dots a_{j} \in L_G(C)\}$ for all $l \leq i < m$ and $l' \leq j < m'$.
\end{enumerate}

Then the procedure \textit{completeLayer(M)}, returns correctly computed sets of $T_{i, j}$ for all $l \leq i \le m$ and $l' \leq j \le m'$ for all $(l, m, l', m') \in M$.
\end{lemma}

\begin{proof}
Proof by induction on $m - l$.
\end{proof}

\begin{theorem}
Algorithm from listing~\ref{algo:modified} correctly computes $T_{i, j}$ for all i and j, thus an input string $a = a_{1}a_{2} \dots a_{n} \in L_{G}(S)$ if and only if $S \in T_{0, n}$.
\end{theorem}

\begin{proof}
Primarily to prove the theorem, we show by induction that all layers of the parsing table T are computed correctly.

\underline{\textbf{Basis:}} layer of size $1 \times 1$.
Parsing table \textit{T} consists of one layer of size 1 and its elements are correctly computed in lines 2-3 in listing~\ref{algo:modified}.

\underline{\textbf{Inductive step:}} assume any layer of size less than or equal to $2^{p - 2} \times 2^{p - 2}$ are computed correctly. 

Define layer of size $2^{p - 1} \times 2^{p - 1}$ as M. 
Hereinafter \textit{subm = (l, m, l', m')} is a typical element of layer M.

Consider \textit{completeVLayer(M)} call. 

Firstly, \textit{performMultiplications(multiplicationTask$_1$)} adds to each P$_{i,j}$ all pairs 
$(B, C)$ such that $\exists k$, $(\frac{l+m}{2} \le k < l')$, $a_{i + 1} \dots a_{k} \in L_{G}(B)$, $a_{k + 1} \dots a_{j} \in L_{G}(C)$ for all $(i, j)$ $\in leftsublayer(M)$
and
$(B, C)$ such that $\exists k$, $(m \le k < \frac{l'+m'}{2})$, $a_{i + 1} \dots a_{k} \in L_{G}(B)$, $a_{k + 1} \dots a_{j} \in L_{G}(C)$ for all $(i, j)$ $\in rightsublayer(M)$.
Now \textit{completeLayer(leftsublayer(M) $\cup$ rightsublayer(M))} can be called and it returns correctly computed \textit{leftsublayer(M) $\cup$ rightsublayer(M)}.

Then \textit{performMultiplications} called with arguments 
\textit{multiplicationTask$_2$} and \textit{multiplicationTask$_3$} adds pairs 
$(B, C)$ such that $\exists k$, $(\frac{l+m}{2} \le k < m)$, $a_{i + 1} \dots a_{k} \in L_{G}(B)$, $a_{k + 1} \dots a_{j} \in L_{G}(C)$ 
and 
$(B, C)$ such that $\exists k$, $(l' \le k < \frac{l'+m'}{2})$, $a_{i + 1} \dots a_{k} \in L_{G}(B)$, $a_{k + 1} \dots a_{j} \in L_{G}(C)$
to each P$_{i,j}$ for all $(i, j)$ $\in topsublayer(M)$. 
So as $m = l'$ (from the construction of the layer), condition for elements of matrix $P$ are fulfilled.
Now \textit{completeLayer(topsublayer(M))} can be called and it returns correctly computed \textit{topsublayer(M)}.

All $T[i, j]$ $\forall (i, j) \in M$ are computed correctly.

Thus, \textit{completeVLayer(M)} returns correct $T_{i, j}$ for all $(i, j)$ $\in M$ for any layer M of parsing table T and lines 4-6 in listing~\ref{algo:modified} return all $T_{i, j} =  \{ A | A \in N, a_{i + 1} \dots a_{j} \in L_{G}(A)\}$.
\end{proof}


\begin{lemma}
Let \textit{calls$_{i}$} is a number of the calls of \textit{completeVLayer(M)} where for all $(l, m, l', m') \in M$ with $m - l = 2^{p - i}$.
\begin{itemize}
 \item for all $i \in \{ 1, .., p - 1\}$  $\sum_{n=1}^{calls_i}{|M|}$ is exactly $2^{2i - 1} - 2^{i - 1}$;
 \item for all $ i \in \{ 1, .., p - 1\}$ products of submatrices of size $2^{p - i} \times 2^{p - i}$ are calculated exactly $2^{2i - 1} - 2^{i}$ times.
\end{itemize}
\end{lemma}

\begin{proof}

Prove the first statement by induction on i.

\underline{\textbf{Basis:}} i = 1. \textit{calls$_{1}$} and $|M| = 1$. So, $2^{2i - 1} - 2^{i - 1} = 2^1 - 2^0 = 1$.

\underline{\textbf{Inductive step:}} assume that $\sum_{n=1}^{calls_i}{|M|}$ is exactly $2^{2i - 1} - 2^{i - 1}$ for all $i \in \{ 1, .., j\}$.

Let us consider $i = j + 1$.

Firstly, note that function $\textit{costructLayer(i)}$ returns $2^{p - i} - 1$ matrices of size $2^i$, so in the call of \textit{completeVLayer(costructLayer(k - i))}  \textit{costructLayer(k - i)} returns $2^i - 1$ matrices of size $2^{p - i}$. 
Secondly, \textit{completeVLayer(M)} is called 3 times for the left, right and top submatrices of size $2^{p - (i - 1)}$. Finally, \textit{completeVLayer(M)} is called 4 times for the bottom, left, right and top submatrices of size $2^{p - (i - 2)}$, except $2^{i - 2} - 1$ matrices which were already computed.

Then, $\sum_{n=1}^{calls_i}{|M|} = 2^{i} - 1 + 3 \times (2^{2(i - 1) - 1} - 2^{(i - 1) - 1}) + 4 \times (2^{2(i - 2) - 1} - 2^{(i - 2) - 1}) - (2^{i - 2} - 1) = 2^{2i - 1} - 2^{i - 1}$.

Now we know that $\sum_{n=1}^{calls_{i-1}}{|M|}$  is $2^{2(i - 1) - 1} - 2^{(i - 1) - 1}$ and we can calculate the number of products of submatrices of size $2^{p - i} \times 2^{p - i}$. 
During these calls \textit{performMultiplications} run 3 times, $|multiplicationTask1| = 2 \times 2^{2(i - 1) - 1} - 2^{(i - 1) - 1}$ and $|multiplicationTask2|$ = $|multiplicationTask3| = 2^{2(i - 1) - 1} - 2^{(i - 1) - 1}$. So, the number of products of submatrices of size $2^{p - i} \times 2^{p - i}$ is $ 4 \times (2^{2(i - 1) - 1} - 2^{(i - 1) - 1}) = 2^{2i - 1} - 2^{i}$.
\end{proof}

\begin{theorem}
Let $|G|$ be a length of the description of the grammar G and let n be a length of an input string. Then algorithm from listing~\ref{algo:modified} calculates matrix \textit{T} in $\mathcal{O}(|G|BMM(n)\log{n})$ where BMM(n) is the number of operations needed to multiply two Boolean matrices of size $n \times n$.
\end{theorem}

\begin{proof}
The proof is almost identical with the proof of theorem 1 given by Okhotin~\cite{Okhotin:2014:PMM:2565359.2565379}, because, as shown in the last lemma, the Algorithm 1 has the same number of products of submatrices.
\end{proof}

To summarize, the correctness of the modification was proved and it was shown that the time complexity remained the same as in Valiant's version.


\subsection{Algorithm for substrings}

Next, we show how our modification can be applied to the string-matching problem.

To find all substrings of size $s$, which can be derived from a start symbol for an input string of size $n = 2^p$, we need to compute layers with submatrices of size not greater than $2^{l'}$, where $2^{l' - 2} < s \le 2^{l' - 1}$.

Let $l' = p - (m - 2)$ and consequently $(m - 2) = p - l'$.
For any  $m \le i \le p$ products of submatrices of size $2^{p - i}$ are calculated exactly $2^{2i - 1} - 2^{i}$ times and each of them imply multiplying $\mathcal{O}(|G|)$ Boolean submatrices.
Now we estimate the number of operations needed to find all substrings:

\begin{equation*}
\begin{array}{c}
C \cdot \sum\limits_{i=m}^p 2^{2i - 1} \cdot 2^{\omega(p - i)} \cdot f(2^{p - i}) =
C \cdot 2^{\omega l'}\sum\limits_{i=2}^{l'} 2^{(2 - \omega)i} \cdot 2^{2(p - l') - 1} \cdot f(2^{l' - i}) \le \\
C \cdot 2^{\omega l'} f(2^{l'}) \cdot 2^{2(p - l') - 1} \sum\limits_{i=2}^{l'} 2^{(2 - \omega)i} =
\mathrm{BMM}(2^{l'}) \cdot 2^{2(p - l') - 1} \sum\limits_{i=2}^{l'} 2^{(2 - \omega)i}
\end{array}
\end{equation*}

Thus, time complexity for searching all substrings is  $O(|G|\mathrm{BMM}(2^{l'})(l' - 1))$, while time complexity for the full input string is $O(|G|\mathrm{BMM}(2^p)(p - 1))$. 
The Valiant's algorithm completely calculate at least 2 triangle submatrices of size $\frac{n}{2}$, as shown in Figure~\ref{fig5}, thus the minimum asymptotic complexity is $O(|G|\mathrm{BMM}(2^{p - 1})(p - 2))$.
Thus we can conclude that the modification is asymptotically faster than the original algorithm for substrings of size $s \ll n$.

\begin{figure}
\vspace{3mm}
 \begin{center}
 \includegraphics[width=12cm]{pictures/valsubstring.pdf}
    \caption{The number of elements necessary to compute in Valiant's algorithm. It is necessary to calculate at least 2 triangle submatrices of size $\frac{n}{2}$.}
    \label{fig5}
 \end{center}
\vspace{-8mm}
\end{figure}

Для экспериментальных исследований были необходимы данные двух типов: последовательности РНК для подачи на вход синтаксическому анализатору и эталонные вторичные структуры для этих последовательностей --- и то, и другое было получено из популярной в исследовательских работах базы данных RNAstrand~\cite{andronescu2008rna}. Эта база представляет собой сборку тщательно отобранных и приведенных к единому формату данных сразу из нескольких надежных баз, содержащих цепочки РНК вместе с полученными методами лабораторного эксперимента или эволюционного анализа вторичными структурами. Из выгруженных данных были удалены дубликаты и образцы с неточностями в нуклеотидной цепи или же вторичной структуре, а также было выставлено ограничение на максимальную длину последовательности --- таким образом была получена выборка из 801 последовательности длин от 8 до 100, для которой были сгенерированы матрицы разбора и матрицы контактов, переведенные в черно-белые изображения. Для цепочки длины $n$ и входное, и эталонное изображения имеют размер $n \times n$, поэтому для корректной обработки изображений разного размера перед каждой эпохой обучения нейросети данные группировались по батчам, в каждом из которых присутствовали изображения только одного размера. Распределение длин последовательностей в итоговой выборке продемонстрировано на рис.~\ref{plot_distr}, при этом медианным значением является 44, а средним --- 47, что говорит о практически одинаковой представленности коротких и длинных цепочек среди исследуемых данных.

\begin{figure}[h]
\begin{center}
\centering
\includegraphics[width=16cm]{pics/plot_distr.png}
\caption{Распределение длин последовательностей РНК в выборке}
\label{plot_distr}
\end{center}
\end{figure} 

Для оценки качества работы обученных на данных изображениях нейронных сетей были выбраны следующие метрики, посчитанные относительно попиксельной разницы между предсказанным и эталонным изображениями. Далее $TP$ (true positive), $FP$ (false positive) и $FN$ (false negative), где под positive и negative  понимаются белые и черные пиксели изображений соответственно, --- информация о том, сколько раз нейронная сеть приняла верное и сколько раз неверное решение по каждому пикселю (кроме диагональных) каждого изображения тестовой выборки.
\begin{itemize} 
    \item $Precision = \frac{TP}{TP + FP}$ (доля предсказанных контактов, которые действительно являются контактами в эталонном изображении).
    \item $Recall = \frac{TP}{TP + FN}$ (доля найденных нейронной сетью контактов среди всех искомых).
    \item $F1 = 2 * \frac{Precision * Recall}{Precision + Recall}$ (гармоническое среднее $Precision$ и $Recall$, используется как удобная объединяющая метрика).
\end{itemize}

При обучении нейросети была использована функция потерь, в основе построения которой лежит идея о максимизации метрики $F1$ с несколькими уточнениями. Во-первых, $F1$ дискретна, а функция ошибки должна быть дифференцируема вследствие вычисления на ней градиента. Во-вторых, передача среднего по выборке значения $1 - F1$ в качестве функции ошибки не гарантирует отсутствие большого разброса $Precision$ и $Recall$ как в пределах отдельно взятого изображения, так и в масштабах всей выборки, следствием чего будет нестабильность качества работы модели и высокая вероятность появления очень низкой точности результата для случайно взятого тестового образца. На основании данных соображений была реализована функция $F1\_loss$, представленная на рис.~\ref{loss}. Здесь дифференцируемость обеспечивается заменой сумм дискретных целочисленных значений на непрерывную сумму значений вероятности, а поддержка баланса между $Precision$ и $Recall$ для каждого изображения и для выборки в целом --- двумя пропорциональными величине разброса штрафными коэффициентами $k1$ и $k2$, накладываемыми на метрику $F1$.

\begin{figure}[h]
\begin{center}
\centering
\begin{python}
from keras import backend as K

def f1_loss(y_true, y_pred):
    #normalize pixels values to [0, 1]
    y_true, y_pred = K.minimum(y_true / 255, 1), K.minimum(y_pred / 255, 1)
    #calculate differentiable versions of TW, FW and FB
    tw = K.sum(K.cast(y_true * y_pred, 'float32'), axis=[1, 2, 3])
    fw = K.sum(K.cast((1 - y_true) * y_pred, 'float32'), axis=[1, 2, 3])
    fb = K.sum(K.cast(y_true * (1 - y_pred), 'float32'), axis=[1, 2, 3])
    #calculate precision and recall secure from zero division error
    precision = tw / (tw + fw + K.epsilon())
    recall = tw / (tw + fb + K.epsilon())
    #penalty coefficients for huge difference between precision and recall 
    #calculated for each image and whole dataset respectively
    k1 = 1 -  K.abs(precision - recall)
    k2 = 1 -  K.abs(K.mean(precision) - K.mean(recall))
    #calculate upgraded f1 score
    f1 = k1 * k2 * 2 * precision * recall / (precision + recall + K.epsilon()) 
    return 1 - K.mean(f1)
\end{python}
\caption{Функция потерь нейронной сети}
\label{loss}
\end{center}
\end{figure} 

Вследствие того, что количество обучаемых параметром используемой модели является достаточно большим относительно размера обучающей выборки, после каждого остаточного блока был добавлен слой Dropout, исключающий заданный процент случайных нейронов во время обучения. Кроме того, во всех сверточных слоях была применена регуляризация L2, которая, помимо уменьшения переобучения нейросети, оказывает положительное влияние на процесс поиска сложных закономерностей в данных. В качестве оптимизатора был использован адаптивный градиентный спуск (Adagrad)~\cite{duchi2011adaptive}, удобный для работы с разреженными данными, а также автоматически настраивающий скорость обучения.

Для сравнения результатов работы обученной модели с существующими в области аналогами был проведен анализ различных инструментов, предсказывающих вторичную структуру РНК, по следующим критериям: заявленная высокая точность результатов, возможность предсказания псевдоузлов, удобство использования и адекватное время работы. На основании данных соображений были отобраны шесть инструментов, основанных на различных подходах.
\begin{itemize}
    \item HotKnots --- минимизации свободной энергии через эвристический алгоритм~\cite{ren2005hotknots}.
    \item SPOT-RNA --- глубокое обучение, основанное на технике transfer learning~\cite{singh2019rna}.
    \item PknotsRG --- минимизация свободной энергии с использованием Turner energy rules~\cite{reeder2007pknotsrg}.
    \item RNAstructure --- минимизация свободной энергии с помощью динамического программирования~\cite{bellaousov2013rnastructure}.
    \item Ipknot --- поиск оптимальной вторичной структуры методом целочисленного программирования~\cite{sato2011ipknot}.
    \item Knotty --- алгоритм для минимизация свободной энергии, основанный на разреженном динамическом программировании~\cite{jabbari2018knotty}.
\end{itemize}

\subsection{Результаты}
Все тестовые запуски проводились на рабочей станции со следующими характеристиками.
\begin{itemize}
    \item Операционная система: Ubuntu 20.04.2 LTS.
    \item Центральный процессор: Intel Core i5-10210U CPU 1.60GHz.
    \item Графический процессор: NVIDIA GeForce MX250.
    \item Объем оперативной памяти: 7.5 GB.
\end{itemize}

На рис.~\ref{plot_f1} представлены значения метрики $F1$, показанные шестью вышеописанными инструментами на всей выборке из 801 образца, а разработанной моделью (New-model) --- для различных разделений данных на обучающую и тестовую выборки (10\%:90\%, ..., 90\%:10\%). На графике видно, что при малых размерах обучающей выборки новая модель демонстрирует достаточно низкую точность, однако при увеличении выборки до 40\% результаты становятся сравнимыми с остальными подходами, а при максимальном объеме выборки (90\%) ---  лучшими в приведенном сравнении.

На рис.~\ref{plot_pr} показаны результаты аналогичного тестирования всех моделей по метрикам $Precision$ и $Recall$; здесь черная прямая $y=x$ символизирует оптимальное для рассматриваемой задачи положение этих метрик --- их равенство, --- а фиолетовая пунктирная линия указывает направление увеличения размера обучающей выборки для нашей модели от 10\% до 90\% с шагом в 10\%. Значения метрик для New-model расположены достаточно близко к желаемой прямой, что говорит о сбалансированности предсказаний разработанной нейросети. Кроме того, реализованный в данной работе алгоритм --- единственный на данном графике, имеющий $Recall$, больший, чем $Precision$: это произошло из-за того, что парсер находит значительную часть требуемых контактов, поэтому нейронная сеть, владея этой информацией еще до начала обучения, основной своей задачей имеет улучшение точности, а не полноты системы. Это делает наш подход несколько нетрадиционным относительно аналогов, которые, по всей видимости, сталкиваются с рядом проблем в процессе поиска контактов во вторичной структуре.

\begin{figure}[h]
\centering
\begin{subfigure}{.5\textwidth}
  \centering
  \fbox{\includegraphics[width=.95\linewidth]{pics/plot_f1.png}}
  \caption{Значения метрики $F1$}
  \label{plot_f1}
\end{subfigure}%
\begin{subfigure}{.5\textwidth}
  \centering
  \fbox{\includegraphics[width=.95\linewidth]{pics/plot_pr.png}}
  \caption{Значения метрик $Precision$ и $Recall$}
  \label{plot_pr}
\end{subfigure}
\caption{Сравнение разработанного подхода с аналогами}
\label{plot}
\end{figure}

Помимо точности, важной характеристикой алгоритма в области биоинформатики является время его работы, так как исследователям часто приходится работать с достаточно большими биологическими базами данных. В таблице~\ref{time} приведены замеры времени, потраченного всеми инструментами на обработку 100 цепочек РНК различных длин из рассматриваемого промежутка от 8 до 100. Несмотря на то, что разные подходы могут предполагать разные сценарии использования (обработка одной или нескольких последовательностей, вывод ответа через интерфейс командной строки или в специальный файл, а также сохранение результатов в различных форматах), одним из традиционных вариантов является обработка файла в формате fasta, содержащего набор последовательностей с метаданными, и последующее сохранение результата в одном из общепринятых форматов (например, dot-bracket или bpseq). Для данного сценария и был произведен сравнительный анализ производительности подходов: файл с последовательности был преобразован в необходимые для всех инструментов входные форматы, выходные же форматы были оставлены без изменений. В таблице~\ref{time} представлены средние значения для десяти прогонов в секундах, упорядоченные по возрастанию времени. Инструменты Ipknot, Hotknots, PknotsRG, RNAstructure и Knotty работают только на CPU, SPOT-RNA имеет и CPU, и GPU-реализации, а для нашего подхода как алгоритм синтаксического анализа (PA), так и нейронная сеть (NN) используют GPU. Можно увидеть, что New-model значительно проигрывает по времени большинству аналогов и наиболее времязатратной операцией здесь является синтаксический анализ, занимающий почти 80\% от общего времени работы.

\begin{table}[]
\centering
\caption{Time measurements for 100 sequences processing}
\begin{tabular}{|p{2cm}||p{2cm}|p{2cm}|p{2cm}|p{2cm}|}
\hline
\multirow{2}{*}{Step} & \multicolumn{2}{l|}{Vector based approach} & \multicolumn{2}{l|}{Image based approach} \\ \cline{2-5} 
 & \begin{tabular}[c]{@{}l@{}}Base \end{tabular} & \begin{tabular}[c]{@{}l@{}}Extended \end{tabular} & \begin{tabular}[c]{@{}l@{}}Base \end{tabular} & \begin{tabular}[c]{@{}l@{}}Extended \end{tabular} \\ \hline \hline
Parse & 307.6s & --- & 310.5s & --- \\ \hline
Load weights & 0.2s & 0.2s & 0.1s & 0.3s \\ \hline
Predict class & 0.2s & 0.2s & 0.2s & 0.3s \\ \hline
Total & 308.0s & 0.4s & 310.8s & 0.6s \\ \hline
\end{tabular}
\label{time}
\end{table}

Подводя итоги, экспериментальные исследования показали работоспособность разработанного подхода применительно к задаче предсказания вторичной структуры РНК даже в сравнении с лучшими инструментами в области. Высокая точность уже полученных результатов вместе с общей гибкостью подхода и обширными возможностями для дальнейших экспериментов позволяют полагать, что предложенные в данной работе идеи имеют значительный потенциал. Однако на данный момент наш проект по большей части исследовательский --- для создания полноценного инструмента требуется тщательный анализ качества всех обученных на различного размера выборках моделей с целью выбора оптимальной, а также, несомненно, повышение производительности подхода, в частности, ускорение синтаксического анализатора.
\section{\bf Modified Valiant's algorithm}

In this section we describe the reorganization of submatrices processing order in the Valiant's algorithm which simplify independent handling of submatrices. As a result, proposed modification can facilitate implementation of parallel submatrix processing.

\subsection{\bf \it Layered submatrices processing}

The main change of this modification is the possibility to divide the parsing table into layers of disjoint submatrices of the same size.
The idea of division we have made from the reorganization of the matrix multiplication order is presented in figure~\ref{fig2}.
Each layer consists of square matrices which size is power of 2.
The layers are computed successively in the bottom-up order.
Each matrix in the layer can be handled independently, which can help to implement parallel version of layer processing function.

\begin{figure}[h]
\vspace{3mm}
 \begin{center}
 \includegraphics[width=12cm]{pictures/modivis2.pdf}
    \caption{An example of the modification of Valiant's algorithm}
    \label{fig4}
 \end{center}
\vspace{-8mm}
\end{figure}

A simple example of the modification is shown in figure~\ref{fig4}.
The lowest layer (submatrices which size is 1) is already computed and filling of the matrix starts with the second layer (subfigures 1-2).
Note that the same process is presented in figure~\ref{fig3}, but here it can be done only in two steps using parallel computation of submatrix products.

The modified version of Valiant's algorithm is presented in listing~\ref{algo:modified}.
The procedure \textit{main()} computes the lowest layer $(T_{l, l+1})$, and then divide the table into layers, described earlier, and computes them through the \textit{completeVLayer()} call.
Thus, \textit{main()} computes all elements of parsing table $T$.
(Hereinafter, we use layer to mean set of submatrices.)

For brevity, we define \textit{left(subm), right(subm), top(subm), bottom(subm), \linebreak rightgrounded(subm)} and \textit{leftgrounded(subm)} functions which returns the submatrices for matrix $subm = (l, m, l', m')$ according to the original Valiant's algorithm (figure~\ref{fig2}).

Also denote some subsidiary functions for matrix layer $M$:
\begin{itemize}[noitemsep, nolistsep]
    \item[$-$] \textit{bottomsublayer(M)} $ = \{bottom(subm)\, |\,subm \in M \}$,
    \item[$-$] \textit{leftsublayer(M)} $ = \{\textit{left(subm)}\, |\,subm \in M \}$,
    \item[$-$] \textit{rightsublayer(M)} $ =\{\textit{right(subm)}\, |\,subm \in M \}$,
    \item[$-$] \textit{topsublayer(M)} $ = \{top(subm)\, |\,subm \in M \}$.
\end{itemize}

\begin{algorithm}[!h]
\SetAlgoNoLine
\KwIn{$G = (\Sigma, N, R, S), w = a_{1} \dots a_{n}, n \geq 1, n + 1 = 2^p, a_{i} \in \Sigma$ }
\underline{main()}{:}{

 \For {$l \in \{1, \ldots, n \}$}{$T_{l, l + 1} = \{A | A \rightarrow a_{l + 1} \in R\}$}
 \For{$1 \le i < p - 1 $}{
 layer = \textit{constructLayer(i)}\;
 \textit{completeVLayer(layer)}
 }
 accept if and only if $S \in T_{0, n}$
 \BlankLine
 }

\underline{constructLayer(i)}{:}{
 \BlankLine
 $\{(k2^i, (k+1)2^i, (k + 1)2^i, (k+2)2^i) \, |\, 0 \le k < 2^{p - i} - 1\}$
 \BlankLine
    }
\underline{completeLayer(M)}{:}{
\BlankLine
\If {$\forall (l, m, l', m') \in M \quad (m - l = 1)$}{\For{$ (l, m, l', m') \in M$}{$T_{l, l'} = f(P_{l, l'})$\;}}
\Else{
\textit{completeLayer(bottomsublayer(M))}\;
\textit{completeVLayer(M)}
}
\BlankLine
}

\underline{comleteVLayer(M)}{:}{
 \BlankLine
 \textit{multiplicationTasks$_1$ = \linebreak
    \{$left(subm)$, $leftgrounded(subm)$, $bottom(subm)\, |\,subm \in M \} \cup \linebreak  \{right(subm), bottom(subm), rightgrounded(subm)\, |\,subm \in M\}$\;}
 \BlankLine
 multiplicationTask$_2$ = $\{top(subm), leftgrounded(subm), right(subm)\, |\,subm \in M\}$\;
 \BlankLine
 multiplicationTask$_3$ = $\{top(subm), left(subm), rightgrounded\, |\,subm \in M\}$\;
 \BlankLine
 \textit{performMultiplications(multiplicationTask$_1$)}\;
 \textit{completeLayer(leftsublayer(M) $\cup$ rightsublayer(M))}\;
 \textit{performMultiplications(multiplicationTask$_2$)}\;
 \textit{performMultiplications(multiplicationTask$_3$)}\;
 \textit{completeLayer(topsublayer(M))}

 }
 \BlankLine

 \underline{performMultiplication(tasks)}{:}{\\
 \For{$ (m, m1, m2) \in \textit{tasks}$}{$P_{m} = P_{m} \cup (T_{m1} \times T_{m2})$\;}
 }

\caption{Parsing by matrix multiplication: Modified Version}
\label{algo:modified}
\end{algorithm}


The procedure \textit{completeVLayer(M)} takes an array of disjoint submatrices $M$ which represents a layer.
For each \textit{subm = (l, m, l', m') $\in M$} this procedure computes \textit{left(subm), right(subm), top(subm)}.
The procedure assumes that the elements of \textit{bottom(subm)} and $T_{i, j}$ for all $i$ and $j$ such that $l \leq i < j < m$ and $  l' \leq i < j < m'$ are already constructed.
Also it is assumed that the current value of
$P_{i, j} =  \{ (B, C) | \exists k, (m \le k < l'), a_{i + 1} \dots a_{k} \in L_G(B), a_{k + 1} \dots a_{j} \in L_G(C)\} $ for all $i$ and $j$ such that $l \leq i < m$ and $l' \leq j < m'$.

The procedure \textit{completeLayer(M)} also takes an array of disjoint submatrices $M$, but unlike the previous one, it computes $T_{i, j}$ for all $(i, j) \in subm$.
This procedure requires exactly same assumptions on $T_{i, j}$  and $P_{i, j}$  as in the previous case.

In the other words, \textit{completeVLayer(M)} computes the entire layer \textit{M} \linebreak and \textit{completeLayer($M_{2}$)} is a support function which is necessary for computation of smaller square submatrices $subm_{2} \in M_{2}$ inside of \textit{M}.

Finally, the procedure \textit{performMultiplication(tasks)}, where \textit{tasks} is an array of a triple of submatrices, perform basic step of algorithm: matrix multiplication. It is worth mentioning that, as distinct from the original algorithm, here $|tasks| \ge 1$ and each task can be computed independently.
So, practical implementation of this procedure can easily involve different techniques of parallel array processing, such as OpenMP.

\subsection{\bf \it Algorithm for substrings}

Next we show how our modification can be applied to the string-matching problem.

So if we want to find all substrings of size $s$ which can be derived from a start symbol for an input string of size $n = 2^p$, we need to compute layers with submatrices of size not greater than $2^{l'}$, where $2^{l' - 2} < s \le 2^{l' - 1}$.

Let $l' = p - (m - 2)$ and consequently $(m - 2) = p - l'$.

For any  $m \le i \le p$ products of submatrices of size $2^{p - i}$ are calculated exactly $2^{2i - 1} - 2^{i}$ times and each of them imply multiplying $\mathcal{O}(|G|)$ Boolean submatrices.

\begin{equation}
\begin{array}{c}
C \sum\limits_{i=m}^p 2^{2i - 1} \cdot 2^{\omega(p - i)} \cdot f(2^{p - i}) =
C \cdot 2^{\omega l'}\sum\limits_{i=2}^{l'} 2^{(2 - \omega)i} \cdot 2^{2(p - l') - 1} \cdot f(2^{l' - i}) \le \\
C \cdot 2^{\omega l'} f(2^{l'}) \cdot 2^{2(p - l') - 1} \sum\limits_{i=2}^{l'} 2^{(2 - \omega)i} =
BMM(2^{l'}) \cdot 2^{2(p - l') - 1} \sum\limits_{i=2}^{l'} 2^{(2 - \omega)i}
\end{array}
\end{equation}

Thus, time complexity for searching all substrings is  $O(|G|BMM(2^{l'})(l' - 1))$, while time complexity for the full input string is $O(|G|BMM(2^p)(p - 1))$. In contract to the modification, Valiant's algorithm completely calculate at least 2 triangle submatrices of size $\frac{n}{2}$ (as shown in figure~\ref{fig5}) which mean minimum asymptotic complexity  $O(|G|BMM(2^{p - 1})(p - 2))$. Thus we can conclude that the modification is asymptotically faster for substrings of size $s \ll n$  than the original algorithm.

\begin{figure}[h]
\vspace{3mm}
 \begin{center}
 \includegraphics[width=12cm]{pictures/valsubstring.pdf}
    \caption{The number of elements necessary to compute in Valiant's algorithm. That means it is nessesary to calculate at least 2 triangle submatrices of size $\frac{n}{2}$.}
    \label{fig5}
 \end{center}
\vspace{-8mm}
\end{figure}


\setmonofont[Mapping=tex-text]{CMU Typewriter Text}
\bibliographystyle{ugost2008ls}
\bibliography{diploma.bib}
\end{document}
