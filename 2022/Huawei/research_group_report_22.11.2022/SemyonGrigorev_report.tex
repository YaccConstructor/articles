%15 min preso!
\documentclass[xcolor=table,aspectratio=169]{beamer}
\usepackage{beamerthemesplit}
\usepackage{wrapfig}
\usetheme{SPbGU}
\usepackage{pdfpages}
\usepackage{amsmath}
\usepackage{cmap}
\usepackage[T2A]{fontenc}
\usepackage[utf8]{inputenc}
\usepackage[english]{babel}
\usepackage{indentfirst}
\usepackage{amsmath}
\usepackage{tikz}
\usepackage{multirow}
\usepackage[noend]{algpseudocode}
\usepackage{algorithm}
\usepackage{algorithmicx}
\usepackage{fancyvrb}
\usepackage{hyperref} 
\usetikzlibrary{calc}
\usetikzlibrary{shapes, backgrounds}
\usetikzlibrary{arrows,automata}
\usetikzlibrary{positioning}
\usetikzlibrary{fit}
\usetikzlibrary{shapes.callouts}
\usetikzlibrary{shapes.misc}
\usepackage{xparse}
\usepackage{fontawesome}

\usepackage{etoolbox,refcount}
\usepackage{multicol}

\usepackage{tabularx}
\newcolumntype{Y}{>{\raggedleft\arraybackslash}X}

\renewcommand{\thealgorithm}{}

\newtheorem{mytheorem}{Theorem}
\renewcommand{\thealgorithm}{}

\newcommand{\tikzmark}[1]{\tikz[overlay,remember picture] \node (#1) {};}
\def\Put(#1,#2)#3{\leavevmode\makebox(0,0){\put(#1,#2){#3}}}

\newcommand{\ltz}{$< 1$}

\tikzset{
    state/.style={
           rectangle,
           rounded corners,
           draw=black, very thick,
           minimum height=2em,
           inner sep=2pt,
           text centered,
           },
}

\tikzset{
    invisible/.style={opacity=0,text opacity=0},
    visible on/.style={alt=#1{}{invisible}},
    alt/.code args={<#1>#2#3}{%
      \alt<#1>{\pgfkeysalso{#2}}{\pgfkeysalso{#3}} % \pgfkeysalso doesn't change the path
    },
}

\tikzset{cross/.style={cross out, draw=black, minimum size=2*(#1-\pgflinewidth), inner sep=0pt, outer sep=0pt, ultra thick},
%default radius will be 1pt. 
cross/.default={1pt}}

\NewDocumentCommand{\mycallout}{r<> O{opacity=0.8,text opacity=1} m m m}{%
\tikz[remember picture, overlay]\node[align=center, fill=cyan!20, text width=#5cm,
#2,visible on=<#1>, rounded corners,
draw,rectangle callout,anchor=pointer,callout relative pointer={(290:0.5cm)}]
at (#3) {#4};
}

\NewDocumentCommand{\mycalloutR}{r<> O{opacity=0.8,text opacity=1} m m m}{%
\tikz[remember picture, overlay]\node[align=center, fill=cyan!20, text width=#5cm,
#2,visible on=<#1>, rounded corners,
draw,rectangle callout,anchor=pointer,callout relative pointer={(30:0.8cm)}]
at (#3) {#4};
}


%callout relative pointer={(230:0.5cm)}]

\newcounter{countitems}
\newcounter{nextitemizecount}
\newcommand{\setupcountitems}{%
  \stepcounter{nextitemizecount}%
  \setcounter{countitems}{0}%
  \preto\item{\stepcounter{countitems}}%
}
\makeatletter
\newcommand{\computecountitems}{%
  \edef\@currentlabel{\number\c@countitems}%
  \label{countitems@\number\numexpr\value{nextitemizecount}-1\relax}%
}
\newcommand{\nextitemizecount}{%
  \getrefnumber{countitems@\number\c@nextitemizecount}%
}
\newcommand{\previtemizecount}{%
  \getrefnumber{countitems@\number\numexpr\value{nextitemizecount}-1\relax}%
}
\makeatother    
\newenvironment{AutoMultiColItemize}{%
\ifnumcomp{\nextitemizecount}{>}{3}{\begin{multicols}{2}}{}%
\setupcountitems\begin{itemize}}%
{\end{itemize}%
\unskip\computecountitems\ifnumcomp{\previtemizecount}{>}{3}{\end{multicols}}{}}


\beamertemplatenavigationsymbolsempty

\title[FLDDA Research Group Report]{Formal Language Driven Data Analysis Research Group Report}
\institute[SPbSU]{
Saint Petersburg State University
}

% То, что в квадратных скобках, отображается в левом нижнем углу.
\author[Semyon Grigorev]{Semyon Grigorev}

\date{Novenber 22, 2022}


%Я предлагаю сначала в общем рассказать интересы и компетенции группы, 
%что научная группа сделала за 3 месяца, 
%потом потратить 1 страницу презентации на каждого из студентов (что сделал, почему важно для Yiming, какие планы). 
%В конце перейти к планам на год.

\begin{document}
{
\begin{frame}[fragile]
  \begin{table}
  \centering
  %\includegraphics[height=1.5cm]{pictures/SPbGU_Logo.png}
  \begin{tabularx}{\linewidth}{XcX}
    \includegraphics[height=0.9cm]{pictures/hu_logo.jpeg} \hfill
    & 
    & \hfill \includegraphics[height=1.6cm]{pictures/SPbGU_Logo.png}
  \end{tabularx}
  \end{table}
  \titlepage
\end{frame}
}

\begin{frame}[fragile]
  \frametitle{Projects Status}  
  \begin{center}
    %\begin{tabularx}{\textwidth}{>{\setlength\hsize{0.5\hsize}\setlength\linewidth{\hsize}}X>{\setlength\hsize{1.1\hsize}\setlength\linewidth{\hsize}}X}      
      \begin{tabularx}{\textwidth}{>{\hsize=.3\hsize\linewidth=\hsize}X
                                   >{\hsize=0.9\hsize\linewidth=\hsize}X}      

Graph analysis
\scriptsize{
  \begin{itemize}
    \item Semyon Grigorev
    \item Rustam Azimov
    \item Ekaterina Shemetova
    \item Alexandra Istomina
    \item Denis Porsev
  \end{itemize}
}
&
\vspace{-10pt}
\begin{itemize}
\item[\faCheck] Interprocedural graph for symbolic execution engine
%\item[\faCheck] Participating in discussions on graph database utilization for code analysis
\item[\faGears] Advice on graph-based declarative code analysis techniques
\item[\faGears] Dataset for path querying algorithms evaluation
\item[\faGears] Evaluation of graph querying algorithms
\item[\faGears] Submission to GRADES-NDA 2023 (at SIGMOD)
\end{itemize}\\
\hline 
\onslide<2->{
  \vspace{0pt}
Linear algebra
\scriptsize{
  \begin{itemize}
    \item Egor Orachev
    \item Kirill Garbar
    \item Igor Erin (December)
  \end{itemize}
}
&
\begin{itemize}
\item[\faCheck] Implementation of basic graph analysis algorithms
\item[\faGears] Performance tuning
\item[\faGears] Submission to GrAPL 2023 (at IPDPS)
\end{itemize}\\
\hline
}
\onslide<3->{
\vspace{0pt}
Parsing algorithms
\scriptsize{
  \begin{itemize}
    \item Semyon Grigorev
  \end{itemize}
}
&
\begin{itemize}
\item[\faCheck] Overview of modern parsing technique
\item[\faGears] Precise error recovery algorithm for generalized LL
\end{itemize}\\
}
\end{tabularx}
\end{center} 
\end{frame}

\begin{frame}[fragile]
  \frametitle{Graph Analysis: Details} 
  \begin{itemize}
    \item[\faCheck] Interprocedural graph for symbolic execution engine (in collaboration with Dmitriy Mordvinov)
    \begin{itemize}
      \item[\faHourglassHalf] Interprocedural navigation for symbolic execution engine (6 months)
    \end{itemize}
    \pause
    \item[\faGears] Advice on graph-based declarative code analysis techniques
    \begin{itemize}
      \item[\faCheck] Participating in discussions on graph database utilization for code analysis
      \item[\faCheck] Collaboration with Dmitriy Ivanov and Evgeniy Koshkin on graph-based code analysis engines
      \item[\faGears] Overview of graph-based declarative code analysis techniques (few weeks)
    \end{itemize}
    \pause
    \item[\faGears] Different graph-based code analysis algorithms comparison
    \begin{itemize}
      \item[\faGears] Dataset for path querying algorithms evaluation
      \item[\faGears] Evaluation of graph querying algorithms
      \item[\faGears] Paper submission in 3--4 months 
    \end{itemize} 
    %\pause
    %\item[\faGears] Papers submission 
    %\begin{itemize}
    %  \item GRADES-NDA 2023 (at SIGMOD)
    %  \item (Springer journal) Theory of Computing Systems (ToCS)
    %  \item GrAPL 2023 (at IPDPS)
    %\end{itemize}
    \end{itemize}
\end{frame}

\end{document}
