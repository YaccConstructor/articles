%15 min preso!
\documentclass[xcolor=table,aspectratio=169]{beamer}
\usepackage{beamerthemesplit}
\usepackage{wrapfig}
\usetheme{SPbGU}
\usepackage{pdfpages}
\usepackage{amsmath}
\usepackage{cmap}
\usepackage[T2A]{fontenc}
\usepackage[utf8]{inputenc}
\usepackage[english]{babel}
\usepackage{indentfirst}
\usepackage{amsmath}
\usepackage{tikz}
\usepackage{multirow}
\usepackage[noend]{algpseudocode}
\usepackage{algorithm}
\usepackage{algorithmicx}
\usepackage{fancyvrb}
\usepackage{hyperref} 
\usetikzlibrary{calc}
\usetikzlibrary{shapes}
\usetikzlibrary{arrows,automata}
\usetikzlibrary{positioning}
\usetikzlibrary{fit}
\usetikzlibrary{shapes.callouts}
\usetikzlibrary{shapes.misc}
\usepackage{xparse}

\usepackage{etoolbox,refcount}
\usepackage{multicol}

\usepackage{tabularx}
\newcolumntype{Y}{>{\raggedleft\arraybackslash}X}

\renewcommand{\thealgorithm}{}

\newtheorem{mytheorem}{Theorem}
\renewcommand{\thealgorithm}{}

\newcommand{\tikzmark}[1]{\tikz[overlay,remember picture] \node (#1) {};}
\def\Put(#1,#2)#3{\leavevmode\makebox(0,0){\put(#1,#2){#3}}}

\newcommand{\ltz}{$< 1$}

\tikzset{
    state/.style={
           rectangle,
           rounded corners,
           draw=black, very thick,
           minimum height=2em,
           inner sep=2pt,
           text centered,
           },
}

\tikzset{
    invisible/.style={opacity=0,text opacity=0},
    visible on/.style={alt=#1{}{invisible}},
    alt/.code args={<#1>#2#3}{%
      \alt<#1>{\pgfkeysalso{#2}}{\pgfkeysalso{#3}} % \pgfkeysalso doesn't change the path
    },
}

\tikzset{cross/.style={cross out, draw=black, minimum size=2*(#1-\pgflinewidth), inner sep=0pt, outer sep=0pt, ultra thick},
%default radius will be 1pt. 
cross/.default={1pt}}

\NewDocumentCommand{\mycallout}{r<> O{opacity=0.8,text opacity=1} m m m}{%
\tikz[remember picture, overlay]\node[align=center, fill=cyan!20, text width=#5cm,
#2,visible on=<#1>, rounded corners,
draw,rectangle callout,anchor=pointer,callout relative pointer={(290:0.5cm)}]
at (#3) {#4};
}

\NewDocumentCommand{\mycalloutR}{r<> O{opacity=0.8,text opacity=1} m m m}{%
\tikz[remember picture, overlay]\node[align=center, fill=cyan!20, text width=#5cm,
#2,visible on=<#1>, rounded corners,
draw,rectangle callout,anchor=pointer,callout relative pointer={(30:0.8cm)}]
at (#3) {#4};
}


%callout relative pointer={(230:0.5cm)}]

\newcounter{countitems}
\newcounter{nextitemizecount}
\newcommand{\setupcountitems}{%
  \stepcounter{nextitemizecount}%
  \setcounter{countitems}{0}%
  \preto\item{\stepcounter{countitems}}%
}
\makeatletter
\newcommand{\computecountitems}{%
  \edef\@currentlabel{\number\c@countitems}%
  \label{countitems@\number\numexpr\value{nextitemizecount}-1\relax}%
}
\newcommand{\nextitemizecount}{%
  \getrefnumber{countitems@\number\c@nextitemizecount}%
}
\newcommand{\previtemizecount}{%
  \getrefnumber{countitems@\number\numexpr\value{nextitemizecount}-1\relax}%
}
\makeatother    
\newenvironment{AutoMultiColItemize}{%
\ifnumcomp{\nextitemizecount}{>}{3}{\begin{multicols}{2}}{}%
\setupcountitems\begin{itemize}}%
{\end{itemize}%
\unskip\computecountitems\ifnumcomp{\previtemizecount}{>}{3}{\end{multicols}}{}}


\beamertemplatenavigationsymbolsempty

\title[Algebraic Path Problems \& GraphBLAS]{Algebraic Path Problems and GraphBLAS: a Way to High-Performance Network Analysis}
\institute[PL\&T@SPbSU]{
Saint Petersburg State University
}

% То, что в квадратных скобках, отображается в левом нижнем углу.
\author[Semyon Grigorev]{Semyon Grigorev}

\date{April 28, 2022}

\begin{document}
{
\begin{frame}[fragile]
  \begin{table}
  \centering
  \includegraphics[height=1.5cm]{pictures/SPbGU_Logo.png}
  %\begin{tabularx}{\linewidth}{XcX}
    %\includegraphics[height=1.5cm]{pictures/YC_logo.pdf} \hfill
    %& \begin{minipage}[t]{0.3\textwidth}\center \vspace{-1cm}  %Huawei-SPbSU Open Day 2021
    %  \end{minipage}
    %& \hfill \includegraphics[height=1.5cm]{pictures/SPbGU_Logo.png}
  %\end{tabularx}
  \end{table}
  \titlepage
\end{frame}
}

\begin{frame}[fragile]
  \frametitle{Agenda}  
  \begin{itemize}
    \item Algebraic Path Problems
    \item GraphBLAS API
    \item Our team
  \end{itemize}
\end{frame}

\begin{frame}[fragile]
  \frametitle{Algebraic Path Problems}  
    \begin{itemize}
      \item Semiring-like structures to specify constraints on paths
      \begin{itemize}
        \item Reachability --- boolean semiring
        \item Shortest paths --- tropical semiring
        \item \ldots
      \end{itemize}
      \pause
      \item Linear algebra friendly algorithms 
      \begin{itemize}
        \item Transitive closure using matrix-matrix multiplication
        \item APSP using matrix-matrix multiplication
        \item BFS-like traversals using matrix-vector multiplication
        \item \ldots
      \end{itemize}
      \pause
      \item Compositionality
      \begin{itemize}
        \item Having two semirings one can create a new one
        \item Single solution for similar problems
        \begin{itemize}
          \item Generic solution
          \item Configurable solution
        \end{itemize}
      \end{itemize}
    \end{itemize}     
\end{frame}

\begin{frame}[fragile]
  \frametitle{Expressivity of the Framework}      
    \begin{itemize}                  
        \item Semiring with typical associativity and distributivity laws 
        \begin{itemize}
          \item Path Problems in Networks\footnote{\url{https://www.morganclaypool.com/doi/abs/10.2200/S00245ED1V01Y201001CNT003}}
          \begin{itemize}
            \item Most reliable path
            \item $k$-shortest paths
            \item Reachability under edge failures
            \item \ldots
          \end{itemize}
        \end{itemize}
        \item Less restrictive structures
        \begin{itemize}
          \item Without associativity and/or distributivity laws
          \item Nonassociative, nonmonotonic, partially ordered, not antisymmetric
          \item Negative cycles
          \begin{itemize}
            \item Unstructured path problems and the making of semirings\footnote{\url{https://link.springer.com/chapter/10.1007/BFb0028261}}
        \item Efficient Algorithms for Path Problems with Gernal Cost Citeria\footnote{\url{https://www.semanticscholar.org/paper/Efficient-Algorithms-for-Path-Problems-with-Gernal-Lengauer-Theune/3fd320d97db0a581952d2919587b112f8df57c0b}}        
          \end{itemize}
        \end{itemize}         
    \end{itemize}
\end{frame}

\begin{frame}[fragile]
  \frametitle{Why Associativity Matters}
  \begin{center}
    $$
    f(M_{n\times n})= M^n = \underbrace{M\cdot M \ldots \cdot M}_{n}
    $$
  \end{center}      
  \begin{columns}[t]
    \begin{column}{0.5\textwidth}
      \begin{center}
        Repeated squaring
        $$
        \underbrace{(\underbrace{(M\cdot\ldots)}_{\frac{n}{4}}\cdot\underbrace{(M\cdot\ldots)}_{\frac{n}{4}}}_{\frac{n}{2}})\cdot\underbrace{(\underbrace{(M\cdot\ldots)}_{\frac{n}{4}}\cdot\underbrace{(M\cdot\ldots)}_{\frac{n}{4}}}_{\frac{n}{2}})
        $$
        $$
        O(n^3)\log{n}
        $$
      \end{center}
    \end{column}
    \begin{column}{0.5\textwidth}
      \begin{center}
        No ways to optimize
        $$
        (\ldots((M \cdot M) \cdot M)\ldots \cdot M)
        $$
        $$
        O(n^4)
        $$
      \end{center}
    \end{column}
  \end{columns}
\end{frame}

\begin{frame}[fragile]
  \frametitle{GraphBLAS API}
  \begin{columns}[t]
    \begin{column}{0.45\textwidth}
  \begin{itemize}
    \item Graph-matrix duality
    \item Operations over matrices and vectors
    \begin{itemize}
      \item Parametrized by semiring-like structures
      \item Based on sparse data structures
      \item Highly parallel
    \end{itemize}    
  \end{itemize}
\end{column}
\pause
\begin{column}{0.55\textwidth}
  \begin{itemize}
  \item High-performance implementations
    \begin{itemize}
      \item SuiteSparse:GraphBLAS\footnotemark[1]: pure C
      \item GraphBLAST\footnotemark[2]: GPGPU, Cuda C
      \item Huawei's GraphBLAS\footnotemark[3]: C++
      \item \ldots
    \end{itemize}
  \end{itemize}
\end{column}
\end{columns}
\pause
\begin{itemize}
  \item More information on GraphBLAS
  \begin{itemize}
    \item Home page: \url{https://graphblas.org/}
    \item GraphBLAS-related resources: \url{https://graphblas.org/GraphBLAS-Pointers/}
    \item Introduction to GraphBLAS: \url{http://mit.bme.hu/~szarnyas/grb/graphblas-introduction.pdf}     
  \end{itemize}    
\end{itemize}
\footnotetext[1]{\url{https://github.com/DrTimothyAldenDavis/GraphBLAS}}
\footnotetext[2]{\url{https://github.com/gunrock/graphblast}}
\footnotetext[3]{\url{https://gitee.com/CSL-ALP/graphblas}}
\end{frame}


\begin{frame}[fragile]
  \frametitle{GraphBLAS Applications}  
  \begin{columns}[t]
    \begin{column}{0.5\textwidth}
  \begin{itemize}
    \item BFS-like algorithms
    \begin{itemize}
      \item BFS: levels, parents, multiple sources
      \item SSSP
      \item \ldots
    \end{itemize}
    \item Graph clustering
  \end{itemize}
 \end{column}
\begin{column}{0.5\textwidth}
  \begin{itemize}
    \item Transitive closure based algorithms
    \begin{itemize}
      \item APSP      
      \item \ldots
    \end{itemize}
    \item Triangle counting    
    \item \ldots
  \end{itemize}
\end{column}
\end{columns}
\pause
\vspace{0.5cm}
\begin{itemize}
    \item LAGraph: collection of GraphBLAS-based algorithms  
  \begin{itemize}
    \item GitHub: \url{https://github.com/GraphBLAS/LAGraph}
    \item Latest report: \url{https://arxiv.org/pdf/2104.01661.pdf}
  \end{itemize}
\end{itemize}
\end{frame}

\newcommand\colR{\cellcolor{red!20}}
\newcommand\colB{\cellcolor{blue!20}}
\newcommand\colG{\cellcolor{green!20}}


\begin{frame}[fragile]
  \frametitle{BFS-like Skeleton}

  
  \begin{minipage}{0.2\textwidth}
  \begin{tikzpicture}[shorten >=1pt,auto]
    \node[state] (q_0)                      {$0$};
    \node[state] (q_1) [above right=of q_0] {$1$};
    \node[state,fill=red!20] (q_2) [right=of q_0]       {$2$};
    \node[state] (q_3) [right=of q_2]       {$3$};
    \path[->]
    (q_0) edge  node {} (q_1)
    (q_1) edge  node {} (q_2)
    (q_2) edge  node {} (q_0)
    (q_2) edge[bend left, above]  node {} (q_3)
    (q_3) edge[bend left, below]  node {} (q_2);
    \end{tikzpicture}
  \end{minipage}~\pause
  \tikzmark{xPos}{}
  \begin{minipage}{0.75\textwidth}    
    \begin{equation*}
      \left(\begin{array}{cccc}        
        0  & 0  & \colR 1 & 0 \\        
      \end{array}\right)
      \times    
      \left(\begin{array}{cccc}        
        0 & 1 & 0 & 0 \\
        0 & 0 & 1 & 0 \\
        \rowcolor{red!20}
        1 & 0 & 0 & 1 \\
        0 & 0 & 1 & 0 \\        
      \end{array}\right)
      =      
        \left(\begin{array}{cccc}        
          \colB 1 & 0  & 0 & \colB 1 \\        
        \end{array}\right)
    \end{equation*}
    \mycallout<2-4>[opacity=1]{$(xPos) + (2.9,0.4)$}{Current front}{2.5}
    \mycallout<2-4>[opacity=1]{$(xPos) + (5.9,1.1)$}{Adjacency matrix}{3.5}
    \mycallout<2-4>[opacity=1]{$(xPos) + (9.2,0.4)$}{New front}{2.5}
    \mycalloutR<2-4>[opacity=1]{$(xPos) + (4.3,0.2)$}{Semiring}{1.5}
  \end{minipage}

  \pause

  \begin{minipage}{0.2\textwidth}
    \begin{tikzpicture}[shorten >=1pt,auto]
      \node[state, fill=blue!20] (q_0)                      {$0$};
      \node[state] (q_1) [above right=of q_0] {$1$};
      \node[state, fill=red!20] (q_2) [right=of q_0]       {$2$};
      \node[state, fill=blue!20] (q_3) [right=of q_2]       {$3$};
      \path[->]
      (q_0) edge  node {} (q_1)
      (q_1) edge  node {} (q_2)
      (q_2) edge  node {} (q_0)
      (q_2) edge[bend left, above]  node {} (q_3)
      (q_3) edge[bend left, below]  node {} (q_2);
      \end{tikzpicture}
    \end{minipage}~
    \begin{minipage}{0.75\textwidth}
    \begin{equation*}
      \left(\begin{array}{cccc}        
        \colB 1 & 0  & 0 & \colB 1 \\        
      \end{array}\right)
      \times
      \left(\begin{array}{cccc}        
        \rowcolor{blue!20}
        0 & 1 & 0 & 0 \\
        0 & 0 & 1 & 0 \\        
        1 & 0 & 0 & 1 \\
        \rowcolor{blue!20}
        0 & 0 & 1 & 0 \\        
      \end{array}\right)
      =      
        \left(\begin{array}{cccc}        
          0 & \colG 1  & \colG 1 & 0 \\        
        \end{array}\right)
    \end{equation*}
    \pause     
    \begin{tikzpicture}[overlay,remember picture,auto]
        \draw (9.16, 1.21) node[cross=10pt, color=red] {};
    \end{tikzpicture}
  \end{minipage}

\end{frame}

\newcommand\minf{+\infty}

\begin{frame}[fragile]
  \frametitle{Shortest Paths}

  \setlength\arraycolsep{3pt}

  \begin{minipage}{0.2\textwidth}
  \begin{tikzpicture}[shorten >=1pt,auto]
    \node[state] (q_0)                      {$0$};
    \node[state] (q_1) [above right=of q_0] {$1$};
    \node[state,fill=red!20] (q_2) [right=of q_0]       {$2$};
    \node[state] (q_3) [right=of q_2]       {$3$};
    \path[->]
    (q_0) edge  node {0.5} (q_1)
    (q_1) edge  node {1.2} (q_2)
    (q_2) edge  node {0.3} (q_0)
    (q_2) edge[bend left, above]  node {1.1} (q_3)
    (q_3) edge[bend left, below]  node {0.4} (q_2);
    \end{tikzpicture}
  \end{minipage}~\pause
  \tikzmark{yPos}{}
  \begin{minipage}{0.75\textwidth}    
    \begin{equation*}
      \left(\begin{array}{cccc}        
        \minf  & \minf  & \colR 0 & \minf \\        
      \end{array}\right)
      \times    
      \left(\begin{array}{cccc}        
        0 & 0.5 & \minf & \minf \\
        \minf & 0 & 1.2 & \minf \\
        \rowcolor{red!20}
        0.3   & \minf & 0 & 1.1 \\
        \minf & \minf & 0.4 & 0 \\        
      \end{array}\right)
      =      
        \left(\begin{array}{cccc}        
          \colB 0.3 & \minf  & \colB 0 & \colB 1.1 \\        
        \end{array}\right)
    \end{equation*}
    %\mycallout<2-3>[opacity=1]{$(xPos) + (2.9,0.4)$}{Current front}{2.5}
    %\mycallout<2-3>[opacity=1]{$(xPos) + (5.9,1.1)$}{Adjacency matrix}{3.5}
    %\mycallout<2-3>[opacity=1]{$(xPos) + (9.2,0.4)$}{New front}{2.5}
    \mycalloutR<1-3>[opacity=1]{$(yPos) + (3.9,0.1)$}{$\{min,+\}$}{1.5}
  \end{minipage}

  \pause

  \begin{minipage}{0.2\textwidth}
    \begin{tikzpicture}[shorten >=1pt,auto]
      \node[state, fill=blue!20] (q_0)                      {$0$};
      \node[state] (q_1) [above right=of q_0] {$1$};
      \node[state, fill=red!20] (q_2) [right=of q_0]       {$2$};
      \node[state, fill=blue!20] (q_3) [right=of q_2]       {$3$};
      \path[->]
      (q_0) edge  node {0.5} (q_1)
      (q_1) edge  node {1.2} (q_2)
      (q_2) edge  node {0.3} (q_0)
      (q_2) edge[bend left, above]  node {1.1} (q_3)
      (q_3) edge[bend left, below]  node {0.4} (q_2);
      \end{tikzpicture}
    \end{minipage}~
    \begin{minipage}{0.75\textwidth}
    \begin{equation*}
      \left(\begin{array}{cccc}        
        \colB 0.3 & \minf  & 0 & \colB 1.1 \\        
      \end{array}\right)
      \times
      \left(\begin{array}{cccc}        
        \rowcolor{blue!20}
        \minf & 0.5 & \minf & \minf \\
        \minf & \minf & 1.2 & \minf \\
        0.3   & \minf & \minf & 1.1 \\
        \rowcolor{blue!20}
        \minf & \minf & 0.4 & \minf \\        
      \end{array}\right)
      =      
        \left(\begin{array}{cccc}        
          0.3 & \colG 0.8  & 0 & 1.1 \\        
        \end{array}\right)
    \end{equation*}

  \end{minipage}

\end{frame}

\newcommand\colO{\cellcolor{orange!20}}
\newcommand\colC{\cellcolor{cyan!20}}
\newcommand\colY{\cellcolor{yellow!40}}

\begin{frame}[fragile]
  \frametitle{Multiple Sources Traversion Skeleton}

  
  \begin{minipage}{0.2\textwidth}
  \begin{tikzpicture}[shorten >=1pt,auto]
    \node[state, fill=red!20] (q_0)                      {$0$};
    \node[state] (q_1) [above right=of q_0] {$1$};
    \node[state] (q_2) [right=of q_0]       {$2$};
    \node[state, fill=orange!20] (q_3) [right=of q_2]       {$3$};
    \path[->]
    (q_0) edge  node {} (q_1)
    (q_1) edge  node {} (q_2)
    (q_2) edge  node {} (q_0)
    (q_2) edge[bend left, above]  node {} (q_3)
    (q_3) edge[bend left, below]  node {} (q_2);
    \end{tikzpicture}
  \end{minipage}~\pause
  \tikzmark{xPos}{}
  \begin{minipage}{0.75\textwidth}    
    \begin{equation*}
      \left(\begin{array}{cccc}        
        \colR 1  & 0  & 0 & 0 \\
        0  & 0  & 0 & \colO 1 \\
      \end{array}\right)
      \times    
      \left(\begin{array}{cccc}        
        \rowcolor{red!20}
        0 & 1 & 0 & 0 \\
        0 & 0 & 1 & 0 \\        
        1 & 0 & 0 & 1 \\
        \rowcolor{orange!20}
        0 & 0 & 1 & 0 \\        
      \end{array}\right)
      =      
        \left(\begin{array}{cccc}        
          0 & \colB 1  & 0 & 0 \\
          0 & 0  & \colC 1 & 0 \\
        \end{array}\right)
    \end{equation*}
    \mycallout<2-4>[opacity=1]{$(xPos) + (2.2,0.7)$}{Current fronts for independent tracking}{3.5}
    %\mycallout<2-3>[opacity=1]{$(xPos) + (5.9,1.1)$}{Adjacency matrix}{3.2}
    \mycallout<2-4>[opacity=1]{$(xPos) + (9.3,0.7)$}{New independent fronts}{2.5}
    %\mycalloutR<2-3>[opacity=1]{$(xPos) + (4.3,0.2)$}{Semiring}{1.5}
  \end{minipage}

  \pause

  \begin{minipage}{0.2\textwidth}
    \begin{tikzpicture}[shorten >=1pt,auto]
      \node[state, fill=red!20] (q_0)                      {$0$};
      \node[state, fill=blue!20] (q_1) [above right=of q_0] {$1$};
      \node[state, fill=cyan!20] (q_2) [right=of q_0]       {$2$};
      \node[state, fill=orange!20] (q_3) [right=of q_2]       {$3$};
      \path[->]
      (q_0) edge  node {} (q_1)
      (q_1) edge  node {} (q_2)
      (q_2) edge  node {} (q_0)
      (q_2) edge[bend left, above]  node {} (q_3)
      (q_3) edge[bend left, below]  node {} (q_2);
      \end{tikzpicture}
    \end{minipage}~
    \begin{minipage}{0.75\textwidth}
    \begin{equation*}
      \left(\begin{array}{cccc}        
        0 & \colB 1  & 0 & 0 \\
        0 & 0  & \colC 1 & 0 \\
      \end{array}\right)
      \times
      \left(\begin{array}{cccc}        
        0 & 1 & 0 & 0 \\
        \rowcolor{blue!20}
        0 & 0 & 1 & 0 \\
        \rowcolor{cyan!20}        
        1 & 0 & 0 & 1 \\        
        0 & 0 & 1 & 0 \\        
      \end{array}\right)
      =      
        \left(\begin{array}{cccc}        
          0 & 0  & \colG 1 & 0 \\
          \colY 1 & 0  & 0 & \colY 1 \\
        \end{array}\right)
    \end{equation*}

  \end{minipage}
  \pause     
  \begin{tikzpicture}[overlay,remember picture]
       \draw (-1.75,-0.15) node[cross=10pt, color=red] {};
  \end{tikzpicture}

\end{frame}


\begin{frame}[fragile]
  \frametitle{Team}  
  \begin{itemize}
      \item Semyon Grigorev (Lead)
      \vspace{-0.4cm}
      \begin{AutoMultiColItemize}
        \item PhD (2016)
        \item Associate professor (2016, SPbSU)        
        \item \href{s.v.grigoriev@spbu.ru}{s.v.grigoriev@spbu.ru}
        \item High-performance graph analysis
        \item Graph databases
        \item dblp: \href{https://dblp.org/pid/181/9903.html}{https://dblp.org/pid/181/9903.html}
      \end{AutoMultiColItemize}
      
      \vspace{-0.2cm}
      \item Ekaterina Shemetova
      \vspace{-0.4cm}
      \begin{AutoMultiColItemize}
        \item PhD student 
        \item Path problems with constraints
        \item Fine-grained complexity
        \item Dynamic graph problems
      \end{AutoMultiColItemize}

      \vspace{-0.2cm}
      \item Rustam Azimov
      \vspace{-0.4cm}
      \begin{AutoMultiColItemize}
        \item PhD student 
        \item Linear algebra based graph analysis
        \item GraphBLAS API
        \item Algebraic path problem
      \end{AutoMultiColItemize}

    \end{itemize}
\end{frame}

\begin{frame}[fragile]
  \frametitle{Team: Master Students}  
  \begin{itemize}
      \vspace{-0.2cm}
      \item Alexandra Istomina
      \vspace{-0.4cm}
      \begin{AutoMultiColItemize}
        \item Master student
        \item Fine-grained complexity
        \item Path problems with constraints
        \item Algebraic path problem
      \end{AutoMultiColItemize}

      \vspace{-0.2cm}
      \item Egor Orachev
      \vspace{-0.4cm}
      \begin{AutoMultiColItemize}
        \item Master student
        \item Linear algebra based graph analysis
        \item GraphBLAS API
        \item GPGPU programming
      \end{AutoMultiColItemize}

      \vspace{-0.2cm}
      \item Vladimir Kutuev
      \vspace{-0.4cm}
      \begin{AutoMultiColItemize}
        \item Master student
        \item Linear algebra based graph analysis
        \item GraphBLAS API
        \item Parallel programming
      \end{AutoMultiColItemize}

      \vspace{-0.2cm}
      \item Julia Susanina
      \vspace{-0.4cm}
      \begin{AutoMultiColItemize}
        \item Master student 
        \item Linear algebra based graph analysis
        \item Probabilistic graph analysis
        \item GPGPU programming
      \end{AutoMultiColItemize}

    \end{itemize}
\end{frame}

\begin{frame}[fragile]
  \frametitle{Research areas}
  \begin{itemize}
    \item Linear algebra based algorithms for graph analysis
    \begin{itemize}
      \item GraphBLAS-based algorithms design, implementation and evaluation
      \item Portable multi-GPGPU implementation of GraphBALS-like API
      \item GraphBLAS API analysis
    \end{itemize} 
    \pause
    \item Path problems with constraints
    \begin{itemize}
      \item Formal Language Constrained Path Querying
      \begin{itemize}
        \item New algorithms development
        \item Complexity analysis
        \item New classes of languages investigation
        \item High performance algorithms implementation and evaluation 
      \end{itemize}
    \end{itemize}
  \end{itemize}
\end{frame}

\begin{frame}[fragile]
  \frametitle{Formal Language Constrained Path Querying}
    \begin{itemize}
      \item Particular case of algebraic path problem
      \begin{itemize}
        \item Multiplication is not associative
        \item Multiplication is not commutative
        \item \ldots
      \end{itemize}
      \pause
      \item Examples
      \begin {itemize}
        \item Regular path querying (RPQ)
        \item Context-free path querying (CFPQ)
      \end{itemize}
    \pause    
    \item Applications 
    \begin{itemize}
      \item Graph analysis
      \item Interprocedural static code analysis
      \item Graph database querying
    \end{itemize}    
  \end{itemize}
\end{frame}

\begin{frame}[fragile]
  \frametitle{Our Results}
    \begin{itemize}
      \item Tools
      \begin{itemize}
        \item \href{https://github.com/JetBrains-Research/spla}{Spla: sparse linear algebra framework for multi-GPU computations based on OpenCL}
        \item \href{https://github.com/JetBrains-Research/spbla}{SPbLA: library of GPGPU-powered sparse boolean linear algebra operations}    
        \pause
        \item \href{https://github.com/JetBrains-Research/CFPQ_PyAlgo}{CFPQ\_PyAlgo: set of GraphBLAS-based FLPQ algorithms}
        \item \href{https://github.com/JetBrains-Research/ldbc_graphalytics}{LDBC Graphalytics extension for evaluation of formal language constrained path querying}            
        \pause
        \item \href{https://github.com/JetBrains-Research/GLL4Graph}{GLL4Graph: CFPQ for Neo4j}
        \item \href{https://github.com/YaccConstructor/RedisGraph}{CFPQ for RedisGraph}
      \end{itemize}
      \pause
      \item Papers (> 10)
      \begin{itemize}
        \item SPbLA: The Library of GPGPU-Powered Sparse Boolean Linear Algebra Operations (GrAPL@IPDPS)
        \item Evaluation of the context-free path querying algorithm based on matrix multiplication (GRADES-NDA@SIGMOD)
        \item Multiple-Source Context-Free Path Querying in Terms of Linear Algebra (EDBT, Core A)
        \item Context-free path querying by matrix multiplication (GRADES-NDA@SIGMOD)
      \end{itemize} 
    \end{itemize}
\end{frame}


\begin{frame}[fragile]
  \frametitle{Possible Ways for Collaboration}
    \begin{itemize}
      \item Algebraic Path Problem framework applicability for network analysis
      \begin{itemize}        
        \item Which constraints can be specified in terms of semirings?
        \begin{itemize}
          \item Length minimality
          \item Nodes to visit
          \item \ldots
        \end{itemize}
        \item Is it flexible enough?
      \end{itemize}
      \item High-performance network analysis
      \begin{itemize}        
        \item GraphBLAS-based solution
        \item Algorithms development and analysis
        \item Algorithms implementation and evaluation
      \end{itemize} 
    \end{itemize}
\end{frame}






%How it (may) works



\end{document}
