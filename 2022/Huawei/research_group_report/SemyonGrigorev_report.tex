%15 min preso!
\documentclass[xcolor=table,aspectratio=169]{beamer}
\usepackage{beamerthemesplit}
\usepackage{wrapfig}
\usetheme{SPbGU}
\usepackage{pdfpages}
\usepackage{amsmath}
\usepackage{cmap}
\usepackage[T2A]{fontenc}
\usepackage[utf8]{inputenc}
\usepackage[english]{babel}
\usepackage{indentfirst}
\usepackage{amsmath}
\usepackage{tikz}
\usepackage{multirow}
\usepackage[noend]{algpseudocode}
\usepackage{algorithm}
\usepackage{algorithmicx}
\usepackage{fancyvrb}
\usepackage{hyperref} 
\usetikzlibrary{calc}
\usetikzlibrary{shapes, backgrounds}
\usetikzlibrary{arrows,automata}
\usetikzlibrary{positioning}
\usetikzlibrary{fit}
\usetikzlibrary{shapes.callouts}
\usetikzlibrary{shapes.misc}
\usepackage{xparse}

\usepackage{etoolbox,refcount}
\usepackage{multicol}

\usepackage{tabularx}
\newcolumntype{Y}{>{\raggedleft\arraybackslash}X}

\renewcommand{\thealgorithm}{}

\newtheorem{mytheorem}{Theorem}
\renewcommand{\thealgorithm}{}

\newcommand{\tikzmark}[1]{\tikz[overlay,remember picture] \node (#1) {};}
\def\Put(#1,#2)#3{\leavevmode\makebox(0,0){\put(#1,#2){#3}}}

\newcommand{\ltz}{$< 1$}

\tikzset{
    state/.style={
           rectangle,
           rounded corners,
           draw=black, very thick,
           minimum height=2em,
           inner sep=2pt,
           text centered,
           },
}

\tikzset{
    invisible/.style={opacity=0,text opacity=0},
    visible on/.style={alt=#1{}{invisible}},
    alt/.code args={<#1>#2#3}{%
      \alt<#1>{\pgfkeysalso{#2}}{\pgfkeysalso{#3}} % \pgfkeysalso doesn't change the path
    },
}

\tikzset{cross/.style={cross out, draw=black, minimum size=2*(#1-\pgflinewidth), inner sep=0pt, outer sep=0pt, ultra thick},
%default radius will be 1pt. 
cross/.default={1pt}}

\NewDocumentCommand{\mycallout}{r<> O{opacity=0.8,text opacity=1} m m m}{%
\tikz[remember picture, overlay]\node[align=center, fill=cyan!20, text width=#5cm,
#2,visible on=<#1>, rounded corners,
draw,rectangle callout,anchor=pointer,callout relative pointer={(290:0.5cm)}]
at (#3) {#4};
}

\NewDocumentCommand{\mycalloutR}{r<> O{opacity=0.8,text opacity=1} m m m}{%
\tikz[remember picture, overlay]\node[align=center, fill=cyan!20, text width=#5cm,
#2,visible on=<#1>, rounded corners,
draw,rectangle callout,anchor=pointer,callout relative pointer={(30:0.8cm)}]
at (#3) {#4};
}


%callout relative pointer={(230:0.5cm)}]

\newcounter{countitems}
\newcounter{nextitemizecount}
\newcommand{\setupcountitems}{%
  \stepcounter{nextitemizecount}%
  \setcounter{countitems}{0}%
  \preto\item{\stepcounter{countitems}}%
}
\makeatletter
\newcommand{\computecountitems}{%
  \edef\@currentlabel{\number\c@countitems}%
  \label{countitems@\number\numexpr\value{nextitemizecount}-1\relax}%
}
\newcommand{\nextitemizecount}{%
  \getrefnumber{countitems@\number\c@nextitemizecount}%
}
\newcommand{\previtemizecount}{%
  \getrefnumber{countitems@\number\numexpr\value{nextitemizecount}-1\relax}%
}
\makeatother    
\newenvironment{AutoMultiColItemize}{%
\ifnumcomp{\nextitemizecount}{>}{3}{\begin{multicols}{2}}{}%
\setupcountitems\begin{itemize}}%
{\end{itemize}%
\unskip\computecountitems\ifnumcomp{\previtemizecount}{>}{3}{\end{multicols}}{}}


\beamertemplatenavigationsymbolsempty

\title[FLDDA Research Group Report]{Formal Language Driven Data Analysis Research Group Report}
\institute[SPbSU]{
Saint Petersburg State University
}

% То, что в квадратных скобках, отображается в левом нижнем углу.
\author[Semyon Grigorev]{Semyon Grigorev}

\date{September 14, 2022}


%Я предлагаю сначала в общем рассказать интересы и компетенции группы, 
%что научная группа сделала за 3 месяца, 
%потом потратить 1 страницу презентации на каждого из студентов (что сделал, почему важно для Yiming, какие планы). 
%В конце перейти к планам на год.

\begin{document}
{
\begin{frame}[fragile]
  \begin{table}
  \centering
  %\includegraphics[height=1.5cm]{pictures/SPbGU_Logo.png}
  \begin{tabularx}{\linewidth}{XcX}
    \includegraphics[height=0.9cm]{pictures/hu_logo.jpeg} \hfill
    & 
    & \hfill \includegraphics[height=1.6cm]{pictures/SPbGU_Logo.png}
  \end{tabularx}
  \end{table}
  \titlepage
\end{frame}
}

\begin{frame}[fragile]
  \frametitle{Research Area}  
  \def\firstcircle{(0,0) circle (2cm)}
  \def\secondcircle{(60:2.8cm) circle (2cm)}
  \def\thirdcircle{(0:2.8cm) circle (2cm)}
  \begin{minipage}{0.2\textwidth}
    \onslide<3->{
    Applications
    \begin{itemize}
      \item Code analysis
      \item Code querying
      \item Code parsing
    \end{itemize}}
  \end{minipage}
  \begin{minipage}{0.48\textwidth}
  \onslide<1->{
  \begin{center}  
  \begin{tikzpicture}
    \tikzmark{xPos}{}
      \begin{scope}[shift={(3cm,-5cm)}, fill opacity=0.5]
          \fill[red] \firstcircle;
          \fill[green] \secondcircle;
          \fill[blue] \thirdcircle;
          \draw \firstcircle node[left, text width=1cm] {Graph databases};
          \draw \secondcircle node [above] {Graph analysis};
          \draw \thirdcircle node [right, text width=1.8cm] {Parsing algorithms};
      \end{scope}
  \end{tikzpicture}}  
  \mycallout<2->[opacity=1]{$(xPos) + (4.5,-4.4)$}{We are here}{2.5}
\end{center}
\end{minipage}
\begin{minipage}{0.28\textwidth}
  \onslide<4->{
  Research directions
    \begin{itemize}
      \item Graph algorithms
      \begin{itemize}
        \item Dynamic graphs
        \item Linear algebra
        \item Path querying 
      \end{itemize}
      \item Formal languages
      \begin{itemize}
        \item Languages classes and properties
        \item Parsing algorithms  
        \item \textbf{Formal language constrained path querying}
      \end{itemize}
    \end{itemize}}
\end{minipage}
\end{frame}

\begin{frame}[fragile]
  \frametitle{Code Analysis and Querying} 
  \tikzset{dbl/.style={double,double distance=2pt}} 
  \begin{tikzpicture}[%
    >=stealth,
    node distance=5.5cm,
    on grid,
    auto
  ]
    \node (A) [text width=4cm]             
      {Huge software projects
        \begin{itemize} 
          \item millions LOC 
          \item complex structure
          \item dynamic
        \end{itemize}};
    \node (B) [right of=A, text width=4.3cm] 
      {Huge graphs for analysis
      \begin{itemize}
        \item millions of vertices
        \item dynamic
      \end{itemize}
      };
    \node (C) [right of=B, text width=4cm] 
      {Graph storage
      \begin{itemize}
        \item graph querying
        \item graph querying
      \end{itemize}
      };
      \node (D) [below = 2.5cm of C, text width=6cm] 
      {Graph analysis
      \begin{itemize}
        \item Performance-critical
        \item Nontrival (esp. for dynamic graphs)
      \end{itemize}
      };
      \onslide<2->{
      \node (E) [left = 6.5cm of D, text width=5cm] 
      {Linear algebra
      \begin{itemize}
        \item parallel (multicore CPU, GPGPU)
        \item Flexible, expressive 
      \end{itemize}
      };}
    \path[->] (A) edge (B);
    \path[->] (B) edge (C);
    \path[->] (B) edge (D);
    \onslide<2->{\path[->] (D) edge (E);}
  \end{tikzpicture}
\end{frame}

\begin{frame}[fragile]
  \frametitle{Code parsing (for IDE)}  
    \begin{itemize}
      \item Dynamic      
      \item Error recovery
    \end{itemize}     
\end{frame}


\begin{frame}[fragile]
  \frametitle{Results}  
    \begin{itemize}
      \item Graph analysis layer for symbolic execution engine
      \item Formal language constrained path querying
      \begin{itemize}
        \item Evaluation
        \item Implementation        
      \end{itemize}
      \pause
      \item Linear algebra
      \begin{itemize}
        \item Brahma
        \item SPLA        
      \end{itemize}
      \pause
      \item Graph analysis algorithms
      \begin{itemize}
        \item Complexity        
      \end{itemize}
    \end{itemize}     
\end{frame}

\begin{frame}[fragile]
  \frametitle{Rustam Azimov}
    \begin{itemize}
      \item !!! 
    \end{itemize}
\end{frame}

\begin{frame}[fragile]
  \frametitle{Ekaterina Shemetova}
    \begin{itemize}
      \item !!! 
    \end{itemize}
\end{frame}

\begin{frame}[fragile]
  \frametitle{Vladimir Kutuev}
    \begin{itemize}
      \item !!! 
    \end{itemize}
\end{frame}

\begin{frame}[fragile]
  \frametitle{Egor Orachev}
    \begin{itemize}
      \item !!! 
    \end{itemize}
\end{frame}

\begin{frame}[fragile]
  \frametitle{Vlada Pogozhelskaya}
    \begin{itemize}
      \item !!! 
    \end{itemize}
\end{frame}

\begin{frame}[fragile]
  \frametitle{Ilya Epelbaum}
    \begin{itemize}
      \item !!! 
    \end{itemize}
\end{frame}

\begin{frame}[fragile]
  \frametitle{Dmitriy Panfilenok}
    \begin{itemize}
      \item !!! 
    \end{itemize}
\end{frame}

\begin{frame}[fragile]
  \frametitle{Kirill Garbar}
    \begin{itemize}
      \item !!! 
    \end{itemize}
\end{frame}

\begin{frame}[fragile]
  \frametitle{Artyom Chernikov}
    \begin{itemize}
      \item !!! 
    \end{itemize}
\end{frame}

\begin{frame}[fragile]
  \frametitle{Denis Porsev}
    \begin{itemize}
      \item !!! 
    \end{itemize}
\end{frame}




\begin{frame}[fragile]
  \frametitle{Possible Ways for Collaboration}
    \begin{itemize}
      \item Algebraic Path Problem framework applicability for network analysis
      \begin{itemize}        
        \item Which constraints can be specified in terms of semirings?
        \begin{itemize}
          \item Length minimality
          \item Nodes to visit
          \item \ldots
        \end{itemize}
        \item Is it flexible enough?
      \end{itemize}
      \item High-performance network analysis
      \begin{itemize}        
        \item GraphBLAS-based solution
        \item Algorithms development and analysis
        \item Algorithms implementation and evaluation
      \end{itemize} 
    \end{itemize}
\end{frame}


\end{document}
