%15 min preso!
\documentclass[xcolor=table,aspectratio=169]{beamer}
\usepackage{beamerthemesplit}
\usepackage{wrapfig}
\usetheme{SPbGU}
\usepackage{pdfpages}
\usepackage{amsmath}
\usepackage{cmap}
\usepackage[T2A]{fontenc}
\usepackage[utf8]{inputenc}
\usepackage[english]{babel}
\usepackage{indentfirst}
\usepackage{amsmath}
\usepackage{tikz}
\usepackage{multirow}
\usepackage[noend]{algpseudocode}
\usepackage{algorithm}
\usepackage{algorithmicx}
\usepackage{fancyvrb}
\usepackage{hyperref} 
\usetikzlibrary{calc}
\usetikzlibrary{shapes}
\usetikzlibrary{arrows,automata}
\usetikzlibrary{positioning}
\usetikzlibrary{fit}
\usetikzlibrary{shapes.callouts}
\usetikzlibrary{shapes.misc}
\usepackage{xparse}

\usepackage{etoolbox,refcount}
\usepackage{multicol}

\usepackage{fontawesome5}
\usepackage{fontawesome}

\usepackage{tabularx}
\newcolumntype{Y}{>{\raggedleft\arraybackslash}X}

\renewcommand{\thealgorithm}{}

\newtheorem{mytheorem}{Theorem}
\renewcommand{\thealgorithm}{}

\newcommand{\tikzmark}[1]{\tikz[overlay,remember picture] \node (#1) {};}
\def\Put(#1,#2)#3{\leavevmode\makebox(0,0){\put(#1,#2){#3}}}

\newcommand{\ltz}{$< 1$}

\tikzset{
    state/.style={
           rectangle,
           rounded corners,
           draw=black, very thick,
           minimum height=2em,
           inner sep=2pt,
           text centered,
           },
}

\tikzset{
    invisible/.style={opacity=0,text opacity=0},
    visible on/.style={alt=#1{}{invisible}},
    alt/.code args={<#1>#2#3}{%
      \alt<#1>{\pgfkeysalso{#2}}{\pgfkeysalso{#3}} % \pgfkeysalso doesn't change the path
    },
}

\tikzset{cross/.style={cross out, draw=black, minimum size=2*(#1-\pgflinewidth), inner sep=0pt, outer sep=0pt, ultra thick},
%default radius will be 1pt. 
cross/.default={1pt}}

\NewDocumentCommand{\mycallout}{r<> O{opacity=0.8,text opacity=1} m m m}{%
\tikz[remember picture, overlay]\node[align=center, fill=cyan!20, text width=#5cm,
#2,visible on=<#1>, rounded corners,
draw,rectangle callout,anchor=pointer,callout relative pointer={(290:0.5cm)}]
at (#3) {#4};
}

\NewDocumentCommand{\mycalloutR}{r<> O{opacity=0.8,text opacity=1} m m m}{%
\tikz[remember picture, overlay]\node[align=center, fill=cyan!20, text width=#5cm,
#2,visible on=<#1>, rounded corners,
draw,rectangle callout,anchor=pointer,callout relative pointer={(30:0.8cm)}]
at (#3) {#4};
}


%callout relative pointer={(230:0.5cm)}]

\newcounter{countitems}
\newcounter{nextitemizecount}
\newcommand{\setupcountitems}{%
  \stepcounter{nextitemizecount}%
  \setcounter{countitems}{0}%
  \preto\item{\stepcounter{countitems}}%
}
\makeatletter
\newcommand{\computecountitems}{%
  \edef\@currentlabel{\number\c@countitems}%
  \label{countitems@\number\numexpr\value{nextitemizecount}-1\relax}%
}
\newcommand{\nextitemizecount}{%
  \getrefnumber{countitems@\number\c@nextitemizecount}%
}
\newcommand{\previtemizecount}{%
  \getrefnumber{countitems@\number\numexpr\value{nextitemizecount}-1\relax}%
}
\makeatother    
\newenvironment{AutoMultiColItemize}{%
\ifnumcomp{\nextitemizecount}{>}{3}{\begin{multicols}{2}}{}%
\setupcountitems\begin{itemize}}%
{\end{itemize}%
\unskip\computecountitems\ifnumcomp{\previtemizecount}{>}{3}{\end{multicols}}{}}


\beamertemplatenavigationsymbolsempty

\title[Declarative Code Analysis]{Declarative Code Analysis}
\subtitle{Existing Solutions, Challenges and Research Directions}
%\institute[PL\&T@SPbSU]{
%Saint Petersburg State University
%}

% То, что в квадратных скобках, отображается в левом нижнем углу.
\author[Semyon Grigorev]{Semyon Grigorev}

\date{September 19, 2022}

\begin{document}
{
\begin{frame}[fragile]
  %\begin{table}
  %\centering
  %\includegraphics[height=1.5cm]{pictures/SPbGU_Logo.png}  
  %\end{table}
  \titlepage
\end{frame}
}

\begin{frame}[fragile]
  \frametitle{Declarative Code Analysis}  
  \tikzset{dbl/.style={double,double distance=2pt}} 
  \begin{tikzpicture}[%
    >=stealth,
    node distance=8cm,
    on grid,
    auto
  ]
    \node (A) [text width=7cm]             
      { {\Large \textbf{What}} is the goal of analysis?
        \begin{itemize} 
          \item[\faQuestion] Analytics
          \item[\faQuestion] Vulnerability detection
          \item[\faQuestion] Code smells detection
          \item[\faQuestion] \ldots 
        \end{itemize}
        };
    \onslide<2->{
    \node (B) [right of=A, text width=7cm] 
    { 
      {\Large \textbf{Where}} the place of developed tool in software development process?
    \begin{itemize} 
      \item[\faQuestion] Part of CI
      \item[\faQuestion] IDE-level analysis
      \item[\faQuestion] Standalone server-side analysis
      \item[\faQuestion] \ldots 
    \end{itemize}
    };
    }
    \onslide<3->{
    \node (C) [below = 4cm of A, text width=7cm] 
      {
        {\Large \textbf{Who}} is a user?
    \begin{itemize} 
      \item[\faQuestion] Software architect/analyst
      \item[\faQuestion] Regular developer
      \item[\faQuestion] Advanced developer
      \item[\faQuestion] \ldots 
    \end{itemize}
      };
      }
      \onslide<4->{
      \node (D) [right of = C, text width=7cm] 
      {
        {\Large \textbf{How}} it should be done?
        \begin{itemize} 
          \item[\faQuestion] Information storage
          \item[\faQuestion] Analysis specification language
          \item[\faQuestion] Advanced topics
          \item[\faQuestion] \ldots 
        \end{itemize}
      };
      }
      
    \onslide<4->{
      \path[->] (A) edge (D);
      \path[->] (B) edge (D);
      \path[->] (C) edge (D);}
  \end{tikzpicture}
\end{frame}

\begin{frame}[fragile]
  \frametitle{Declarative Code Analysis: How}  
  \tikzset{dbl/.style={double,double distance=2pt}} 
  \begin{tikzpicture}[%
    >=stealth,
    node distance=8cm,
    on grid,
    auto
  ]
    \node (A) [text width=7cm]             
      { 
        {\Large \textbf{How}} it should be done?
        \begin{itemize} 
          \item[\faQuestion] Information storage
          \item[\faQuestion] Analysis specification language
          \item[\faQuestion] Advanced topics
          \item[\faQuestion] \ldots 
        \end{itemize}
      };
    \onslide<2->{
    \node (B) [right of=A, text width=7cm] 
    { 
      {\Large \textbf{Information storage}}
      \begin{itemize} 
        \item Relational database
        \item Graph database
        \item Custom problem-specific storage
      \end{itemize}
    };
    }
    \onslide<3->{
    \node (C) [below = 4cm of A, text width=7.5cm] 
      {
        {\Large \textbf{Analysis specification language}}
        \begin{itemize} 
          \item Cypher/GQL-like language
          \item Datalog-like language
          \item Custom domain-specific language
        \end{itemize}
      };
      }
      \onslide<4->{
      \node (D) [right of = C, text width=7.5cm] 
      {
        {\Large \textbf{Advanced topics}}
        \begin{itemize} 
          \item Dynamic data analysis (incremental analysis)
          \item Results analysis
          \item Query debugging
          \item \ldots
        \end{itemize}
      };
      }
      
    \onslide<2->{\path[->] (A) edge (B);}
    \onslide<3->{\path[->] (A) edge (C);}
    \onslide<4->{\path[->] (A) edge (D);}
  \end{tikzpicture}
\end{frame}

%\begin{frame}[fragile]
%  \frametitle{Solutions}
  % IncA
  % ShiftLeft (Joern, Ocular)
  % 2020 on Neo4j
%  \begin{center}
%    \begin{tabularx}
%      {\textwidth}
%      {>{\setlength\hsize{0.8\hsize}\setlength\linewidth{\hsize}}
%      X>{\setlength\hsize{1.2\hsize}\setlength\linewidth{\hsize}}
%      X>{\setlength\hsize{1.0\hsize}\setlength\linewidth{\hsize}}
%      X>{\setlength\hsize{1.0\hsize}\setlength\linewidth{\hsize}}
%      X>{\setlength\hsize{1.0\hsize}\setlength\linewidth{\hsize}}X}      

%\href{https://fbinfer.com/}{Infer (Facebook)}
%&
%\begin{itemize}
%  \item Predefined set of analysis
%  \item Program API (Ocaml)
%\end{itemize}
%&
%\vspace{-10pt}
%\begin{itemize}
%\item Standalone analyzer 
%\item Part of CI
%\end{itemize}
%& Bug detection 
%& Separation logic and bi-abuction
%\\
%\hline 
%\onslide<2->{
%  \vspace{0pt}
%CodeQL (GitHub/Microsoft)
%&
%\vspace{0pt}
%Research prototype
%&
%\begin{itemize}
%\item New algorithms
%\item Complexity analysis
%\item Performance analysis
%\end{itemize}
%& !
%\\
%\hline
%}
%\onslide<3->{
%\vspace{0pt}
%Souffle
%&
%\vspace{0pt}
%Research prototype
%&
%\begin{itemize}
%\item Operations implementation
%\item Optimizations
%\item Performance analysis
%\end{itemize}
%& 1
%\\
%}
%\end{tabularx}
%\end{center} 
%
%\end{frame}

\begin{frame}[fragile]
  \frametitle{Infer (Facebook)}  
  \begin{itemize}
    \item \url{https://fbinfer.com/}
    \item General-purpose static code analysis
    \item Separation logic + abstract interpretation
    \begin{itemize}
      \item Modular engine
      \item Program API (OCaml)
      \item Predefined analysis
    \end{itemize} 
  \end{itemize}
\end{frame}

\begin{frame}[fragile]
  \frametitle{CodeQL (GitHub/Microsoft)}
  \begin{itemize}
    \item \url{https://codeql.github.com/}
  \end{itemize}
\end{frame}

\begin{frame}[fragile]
  \frametitle{NG SAST (ShiftLeft)} 
  \begin{itemize}
    \item \url{https://www.shiftleft.io/}
    \item Static application security testing (vulnerability detection)
    \item Ocular (\href{https://joern.io/}{Joern}) as a graph storage and query engine
    \begin{itemize}
      \item Custom graph database
      \item Custom graph query language
    \end{itemize}
  \end{itemize}
\end{frame}

\begin{frame}[fragile]
  \frametitle{Souffl\'e (Oracle Labs/The University of Sydney)} 
  \begin{itemize}
    \item \url{https://souffle-lang.github.io/index.html}
    \item General-purpose static code analysis
    \item Logic programming language inspired by Datalog
    \begin{itemize}
      \item Translation to C++
      \item Can use external storages for relations
    \end{itemize}
    \pause
    \item[\faGears] Query debugging and results analysis (provenance)
    \item[\faGears] Incrementalization
    \item[\faGears] Cloud infrastructure
  \end{itemize}
\end{frame}

\begin{frame}[fragile]
  \frametitle{IncA (Johannes Gutenberg University Mainz)} 
  \begin{itemize}
    \item \url{https://github.com/szabta89/IncA}
    \item Incremental static code analysis framework
    \item Datalog-like DSL
    \item Aimed to provide IDE-level incremental analysis
  \end{itemize}
\end{frame}

\begin{frame}[fragile]
  \frametitle{ProgQuery}
  \begin{itemize}
    \item \url{https://github.com/OscarRodriguezPrieto/ProgQuery}
    \item \href{https://ieeexplore.ieee.org/stamp/stamp.jsp?tp=&arnumber=9064792}{An Efficient and Scalable Platform for Java Source Code Analysis Using Overlaid Graph Representations (2020)}
    \item Neo4j-based
    \begin{itemize}
      \item Cypher query language
      \item Gremlin API
      \item Java native API
    \end{itemize}
    \item Evaluation shows (see paper above)
    \begin{itemize}
      \item Can be more expressive than CodeQL and other tools
      \item Can demonstrates better performance than CodeQL and other tools
    \end{itemize} 
  \end{itemize}
\end{frame}

\begin{frame}[fragile]
  \frametitle{Conclusion}   
  \begin{itemize}
    \item Cypher can be expressive enough against custom and Datalog-like DSLs
    \pause
    \item Graph database can be an appropriate storage (even Neo4j)
    \pause
    \item There is no production ready solutions for IDE-level declarative code analysis
    \pause
    \item Incremental analysis is a nontrivial challenge
    \pause
    \item Query debugging and results analysis is a nontrivial challenge 
  \end{itemize}
\end{frame}

\begin{frame}[fragile]
  \frametitle{Challenges/Research Directions}  
  \begin{columns}[t]
    \begin{column}{0.5\textwidth}
  \begin{itemize}
    \item Graph databases evaluation
    \begin{itemize}
      \item Code analysis related scenarios
      \item Graph representations comparison
      \item Low-level API comparison  
    \end{itemize}
    \pause
    \item Query languages evaluation
    \begin{itemize}
      \pause
      \item Whether advanced DSL needed?
      \pause
      \item Can GQL be an appropriate language?
      %\pause
      \item GQL is SQL for graphs: \textbf{ISO standard} for graph query language
      %\pause
      \item Cypher-like 
      %\pause
      \item Friendly to non-advanced users, widely used
    \end{itemize}
  \end{itemize}  
 \end{column}
 \pause
\begin{column}{0.5\textwidth}
  \begin{itemize}
    \item Dynamic data analysis
    \begin{itemize}
      \item Incremental view maintenance
      \item Incremental static code analysis
      \item Persistent queries
      \item \ldots
    \end{itemize}
    \pause
    \item Query debugging and results analysis 
    \begin{itemize}
      \item Appropriate data structures
      \item Quick fixes
      \item \ldots
    \end{itemize}
  \end{itemize}
\end{column}
\end{columns}
\end{frame}

\end{document}
