%15 min preso!
\documentclass[xcolor=table,aspectratio=169]{beamer}
\usepackage{beamerthemesplit}
\usepackage{wrapfig}
\usetheme{SPbGU}
\usepackage{pdfpages}
\usepackage{amsmath}
\usepackage{cmap}
\usepackage[T2A]{fontenc}
\usepackage[utf8]{inputenc}
\usepackage[english]{babel}
\usepackage{indentfirst}
\usepackage{amsmath}
\usepackage{tikz}
\usepackage{multirow}
\usepackage[noend]{algpseudocode}
\usepackage{algorithm}
\usepackage{algorithmicx}
\usepackage{fancyvrb}
\usepackage{hyperref} 
\usetikzlibrary{calc}
\usetikzlibrary{shapes,arrows}
\usetikzlibrary{arrows,automata}
\usetikzlibrary{positioning}

\usepackage{tabularx}
\newcolumntype{Y}{>{\raggedleft\arraybackslash}X}

\renewcommand{\thealgorithm}{}

\newtheorem{mytheorem}{Theorem}
\renewcommand{\thealgorithm}{}

\newcommand{\tikzmark}[1]{\tikz[overlay,remember picture] \node (#1) {};}
\def\Put(#1,#2)#3{\leavevmode\makebox(0,0){\put(#1,#2){#3}}}

\newcommand{\ltz}{$< 1$}


\tikzset{
    state/.style={
           rectangle,
           rounded corners,
           draw=black, very thick,
           minimum height=2em,
           inner sep=2pt,
           text centered,
           },
}

\beamertemplatenavigationsymbolsempty

\title[HLL for HPC]{High-Level Languages for High-Performance Computing}
\institute[PL\&T@SPbSU]{
Saint Petersburg State University
}

% То, что в квадратных скобках, отображается в левом нижнем углу.
\author[Semyon Grigorev]{Semyon Grigorev}

\date{April 14, 2022}

\begin{document}
{
\begin{frame}[fragile]
  \begin{table}
  \centering
  \includegraphics[height=1.5cm]{pictures/SPbGU_Logo.png}
  \end{table}
  \titlepage
\end{frame}
}

\begin{frame}[fragile]
  \frametitle{High-Level Languages For High-Performance Computing (HLL for HPC)}
  
  \begin{itemize}
    \item Functional, functional-first programming languages for 
    \begin{itemize}
      \item GPGPU programming
      \item FPGA programming (program specific processors)
      \item Hardware synthesis
    \end{itemize}  
    \pause  
    \item Expressivity, high-level composable primitives 
    \item Type safety, static code checks
    \item Specific optimizations
    \begin{itemize}
      \item Fusion (stream fusion)
      \item Partial evaluation
      \item Deforestation
    \end{itemize}   
    \pause
    \item Specific hardware
  \end{itemize}
\end{frame}


\begin{frame}[fragile]
  \frametitle{HLL for HPC: Projects}

  \begin{itemize}
    \item \href{https://www.lift-project.org/}{LIFT}: high-level functional data parallel language for portable HPC 
    \begin{itemize}
      \item University of Edinburgh, University of Glasgow      
      \item Supported by HIRP FLAGSHIP
    \end{itemize}
    \item \href{https://haflang.github.io/}{Haflang}: special purpose processor for accelerating functional programming languages 
    \begin{itemize}
      \item Heriot Watt University
      \item Supported by Xilinx and QBayLogic
    \end{itemize}    
    \item \href{https://anydsl.github.io/}{AnyDSL}: A partial evaluation framework for programming high-performance libraries 
    \begin{itemize}
      \item Saarland University, German Research Center for Artificial Intelligence (DFKI)      
    \end{itemize}
    \item \href{https://futhark-lang.org/}{Futhark}: high-performance purely functional data-parallel array programming 
    \begin{itemize}
      \item University of Copenhagen      
    \end{itemize}
    \item \ldots
  \end{itemize}
\end{frame}

\begin{frame}[fragile]
  \frametitle{Our projects}
  \begin{itemize}
    \item \href{https://github.com/YaccConstructor/Brahma.FSharp}{Brahma.FSharp: F\# to OpenCL C translator and respective runtime}
    \pause
    \item Software-hardware platform for functional programming language
    \begin{itemize}
      \item Powerful fusion-like optimization (distillation)
      \item Special hardware for functional programming language
    \end{itemize}
  \end{itemize}
 
\end{frame}

\begin{frame}[fragile]
  \frametitle{Brahma.FSharp\footnote{\href{https://github.com/YaccConstructor/Brahma.FSharp}{https://github.com/YaccConstructor/Brahma.FSharp}}}
  \begin{itemize}
    \item Transparent integration GPGPU computations to .NET applications
    \pause
    \item Runtime translation and compilation of F\# functions to OpenCL kernels
    \item Data transfer
    \begin{itemize}
      \item Primitive types: int, float, bool, \ldots
      \item Structures
      \item Discriminated Unions
    \end{itemize}
    \item Runtime for kernels execution
  \end{itemize}
 
\end{frame}

\begin{frame}[fragile]
  \frametitle{Software-Hardware Platform for Functional Programming Languages}  
  \begin{itemize}
  \item Final goal: high-performance sparse linear algebra  
  \item Problems 
  \begin{itemize}
    \item Intermediate data structures $\to$ memory traffic
    \item Sparsity $\to$ irregular parallelism 
  \end{itemize}
  \end{itemize}
\end{frame}


\begin{frame}[fragile]
  \frametitle{Solution (Work in Progress)}  
  \begin{itemize}
  \item Distillation\footnote{\href{https://github.com/YaccConstructor/Distiller}{https://github.com/YaccConstructor/Distiller}} 
  \begin{itemize}
    \item High-level program transformation technique
    \item Includes fusion-like optimization
  \end{itemize} 
  \pause
  \item Special hardware
  \begin{itemize}
    \item Reduceron\footnote{\href{https://github.com/tommythorn/Reduceron}{https://github.com/tommythorn/Reduceron}}
    \begin{itemize}
      \item Lambda-processor
      \item Migration to Haflang
    \end{itemize}
    \item FHW\footnote{\href{https://github.com/sedwards-lab/fhw}{https://github.com/sedwards-lab/fhw}}
    \begin{itemize}
      \item Functional program to hardware translator
      \item Program-specific accelerator
    \end{itemize}
  \end{itemize}
  \end{itemize}
\end{frame}

\begin{frame}[fragile]
  \frametitle{Preliminary Evaluation: Input}
  A set of functions for sparse matrices manipulation  
    \begin{itemize}
      \item \verb|addMask m1 m2 m3 = mask (mtxAdd m1 m2) m3|
      \item \verb|kronMask m1 m2 m3 = mask (kron m1 m2) m3 |   
      \item \verb|addMap m1 m2 = map f (mtxAdd m1 m2)|
      \item \verb|kronMap m1 m2 = map f (kron m1 m2)|
      \item \verb|seqAdd m1 m2 m3 m4 = mtxAdd (mtxAdd (mtxAdd m1 m2) m3) m4|              
    \end{itemize}
\end{frame}

\begin{frame}[fragile]
  \frametitle{Preliminary Evaluation: Results}
  In emulator
  \begin{table}
    \centering    
    \begin{tabular}{|c|c|c|c|c||c|c|c|c|}
        \hline
        \multirow{2}{*}{Function} &  \multicolumn{4}{c||}{Matrix size}  & \multicolumn{2}{c|}{Interpreter}            & Reduceron & FHW\\
        \cline{2-9}
                                  &   m1 & m2 & m3 & m4                & Red-s & Reads                               & Ticks     & Ticks \\
        \hline
        seqAdd   & $64 \times 64$ & $64 \times 64$ & $64 \times 64$ & $64 \times 64$ & 2.7          & 1.9        & 1.8 & 1.4  \\ 
        addMask  & $64 \times 64$ & $64 \times 64$ & $64 \times 64$ & --             & 2.1          & 1.8        & 1.4 & 1.4  \\ 
        kronMask & $64 \times 64$ & $2 \times 2$   &$128 \times 128$& --             & 2.2          & 1.9        & 1.4 & 2.7  \\ 
        addMap   & $64 \times 64$ & $64 \times 64$ & --             & --             & 2.5          & 1.7        & 1.7 & 1.5  \\
        kronMap  & $64 \times 64$ & $2 \times 2$   & --             & --             & 2.9          & 2.2        & 1.8 & 2.0  \\ 
        \hline
        
    \end{tabular}
    \caption{Evaluation results: original program to distilled one ratio of measured metrics}
    \label{tbl:evaluationResults}
  \end{table} 
\end{frame}

\begin{frame}[fragile]
  \frametitle{Publications}
    \begin{itemize}
        \item Optimizing GPU programs by partial evaluation (PPoPP, Core A)
        \item Distilling Sparse Linear Algebra (SRC@ICFP)
    \end{itemize}
\end{frame}

\begin{frame}[fragile]
  \frametitle{Contact Info}  
  \begin{itemize}      
      \item Lead: Semyon Grigorev      
      \begin{itemize}
        \item PhD (2016), Associate professor (2016, SPbSU)
        \item dblp: \href{https://dblp.org/pid/181/9903.html}{https://dblp.org/pid/181/9903.html}
        \item h-index (scopus): 5
        \item \href{s.v.grigoriev@spbu.ru}{s.v.grigoriev@spbu.ru}
      \end{itemize}
      \item Collaboration with Daniil Berezun      
    \end{itemize}
\end{frame}


\end{document}
