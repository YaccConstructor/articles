\documentclass[12pt]{article}  % standard LaTeX, 12 point type
\usepackage{amsfonts,latexsym}
\usepackage{amsthm}
\usepackage{amssymb}
\usepackage[utf8]{inputenc} % Кодировка
\usepackage[english]{babel} % Многоязычность

\newtheorem{theorem}{Theorem}[section]
\newtheorem{proposition}[theorem]{Proposition}
\newtheorem{lemma}[theorem]{Lemma}
\newtheorem{corollary}[theorem]{Corollary}
\newtheorem{conjecture}[theorem]{Conjecture}

\theoremstyle{definition}
\newtheorem{definition}{Определение}[section]
\newtheorem{example}{Example}[section]

% unnumbered environments:

\theoremstyle{remark}
\newtheorem*{remark}{Remark}
\newtheorem*{notation}{Notation}
\newtheorem*{note}{Note}

\setlength{\parskip}{5pt plus 2pt minus 1pt}
%\setlength{\parindent}{0pt}

\usepackage{color}
\usepackage{listings}
\usepackage{caption}
\usepackage{graphicx}
\usepackage{ucs}

\newcommand{\tab}[1][0.3cm]{\ensuremath{\hspace*{#1}}}
% A generalized view on parsing and translation
% http://dl.acm.org/citation.cfm?id=2206331
\title{Generalized LL Parsing Generalization}
\author{Semyon Grigorev}
%\date{\today}

\begin{document}

\maketitle

Today data for parsing is not only linear string, and context-free grammar is not only programming language specification.
Classical example is a graph parsing where input is a graph and grammar is path constraint specification.
Also you can see such generalizations of parsing like Multi-variant Lexical presented ap Parsing@SLE-2016, Abstract parsing, ETC.
All of them are separated solutions (except lexing).

Goal of our work is an abstract framework for parsing based on geteralization of GLL parsing algorithm which proposed by Scott and J.
We propose not omly regular input parsing (graph parsing(DB, bio), as special case --- error recovery as graph parsing), but also CF-compressed input processing which is actual for metagenomic assembly precessing. 
Sequitur compression algorithm. 
Our GLL-basd graph-aprsing algorithm is faster then presented at WWW!!!.  

\end{document}