\section{Обзор}

Для того чтобы упростить ход рассуждений, касающихся контекстно-свободных грамматик, в данной работе используется понятие \textit{рекурсивного автомата} --- абстракции, позволяющей задавать произвольную КС-грамматику.
Ее описание приводится в первом параграфе обзора. 
Второй параграф посвящен вопросу о разрешимости задачи синтаксического анализа контекстно-свободного представления данных и его связи с фундаментальными проблемами теории формальных языков.

Предлагаемый в данной работе алгоритм основан на алгоритме синтаксического анализа регулярных множеств, который, в свою очередь, явлется модификацией алгоритма обобщенного синтаксического анализа Generalized LL. 
Об этих алгоритмах, а также о проекте, в рамках которого проведена разработка предложенного решения, также будет рассказано в обзоре.

\subsection{Рекурсивные автоматы и КС-грамматики}
Введем понятие рекурсивного автомата, которое потребуется для дальнейшего изложения

\begin{defn}
	Рекурсивный автомат R --- это пятерка $(\Sigma, Q, \delta, q_0, q_f)$, где $\Sigma$ --- конечное множество терминальных символов, $Q$ --- конечное множество состояний автомата, $\delta : Q \times (\Sigma \cup Q) \rightarrow 2^Q$ --- функция переходов, $q_0 \in Q$ --- начальное состояние, $q_f$ --- конечное состояние. 
\end{defn}

Можно заметить, что данное определение практически идентично определению стандартного конечного автомата. 
Единственное отличие состоит в том, что метками на ребрах рекурсивного автомата могут как терминальные символы (терминальные переходы), так и состояния (нетерминальные переходы).
Класс рекурсивных автоматов обладает такой же выразительностью, как и контекстно-свободные грамматики, т.е. позволяет описать любой контекстно-свободный язык. 
Более того, грамматика тривиальным образом может быть преобразована в рекурсивный автомат (обратное тоже верно) \cite{tellier2006ra}. 
Пример рекурсивного автомата, построенного по грамматике, можно увидеть на рисунке \ref{fig:ra_ex}.

\begin{figure}[h]
	\centering
	\begin{subfigure}[b]{0.45\textwidth}
		\centering
		$$
		\begin{array}{crcl}
		&S' & ::= & S \\
		&S  & ::= & \texttt{[ } S \texttt{ ]}\\
		&S  & ::= & \mbox{\texttt{a}}
		\end{array}
		$$
		\caption{Грамматика $G_1$}
	\end{subfigure}
	~
	\begin{subfigure}[b]{0.45\textwidth}
		\centering
		\includegraphics[width=4cm]{pictures/ra_example.pdf}
		\caption{Рекурсивный автомат для $G_1$}
	\end{subfigure}
	\caption{Преобразование между грамматикой и рекурсивным автоматом}
	\label{fig:ra_ex}
\end{figure}


\subsection{Разрешимость задачи синтаксического анализа контекстно-свободного представления}
Как было сказано ранее, задачу поиска шаблона, при условии, что и шаблон, и данные, в которых осуществляется поиск, представлены контекстно-свободными грамматиками, мы назовем синтаксическим анализом контекстно-свободного представления. 

Для доказательства предложений, сформулированных далее, будет использоваться следующая теорема \cite{Nederhof}.

\begin{theorem}[Nederhof, Satta]
	Пусть $G_1$ --- произвольная контекстно-свободная грамматика, $G_2$ --- грамматика, которая не содержит непосредственной или скрытой рекурсий. Тогда проблема проверки пустоты пересечения языков, порождаемых данными грамматиками, относится к классу PSPACE-complete.
\end{theorem}

Рассмотрим случай, когда грамматика данных задает ровно одну строку. Пусть $G_t$ --- произвольная КС-грамматика, задающая шаблоны для поиска, а $G_d$ --- КС-грамматика, которая не содержит непосредственной или скрытой рекурсий. $L(G_t)$ и $L(G_d)$ --- языки, порождаемые грамматиками, при этом $L(G_d) = \{\omega\}$, где $\omega$ --- исходные данные, к которым был применен алгоритм сжатия. 
Необходимо определить, существуют ли такие строки $\omega'$, что $\omega' \in L(G_t)$ и $\omega'$ --- подстрока $\omega$.

%т.е. $\omega' \in L_{sub}(\omega)$, где $L_{sub}(s)$ --- обозначение для языка, состоящего из всех подстрок заданной строки $s$. Отметим, что $L_{sub}$ относится к классу регулярных языков, так как множество всех подстрок конечно. 

\begin{prop}
	При выполнении описанных условий задача синтаксического анализа КС-представления разрешима.
\end{prop}

\begin{proof}
Пользуясь эквивалентностью представлений, можно записать грамматику $G_d$ в виде рекурсивного автомата $R_d$. Рассмотрим рекурсивный автомат $R_{i,\,j}$, полученный из $R_d$ путем замены стартового состояния на $i \in Q(R_d)$ и назначения терминирующего (финального, из которого не может быть совершено переходов) состояния $j \in Q(R_d)$. Такой автомат описывает грамматику, которая является представленим некоторой подстроки $\omega$. 
%В таком случае, рекурсивный автомат $R_{i,\,j}$, полученный из $R_d$ путем замены стартового и конечного состояний на $i, j \in Q(R_d)$ соответственно, описывает грамматику, которая является представленим некоторой подстроки $\omega$. 
Рассмотрев все возможные пары $i$ и $j$, получаем конечное множество грамматик, для каждой из которых необходимо проверить, содержится ли строка, порождаемая ей, в языке $L(G_t)$. 
Согласно теореме 1, такая проверка является разрешимой задачей и принадлежит к классу PSPACE-complete.
\end{proof}

Отдельно отметим, что для описанных процедур используется лишь исходный автомат, эквивалентный грамматике $G_d$. 
Условия задачи поиска шаблонов непосредственно в контекстно-свободном представлении, таким образом, выполняются. 
Верна также разрешимость более общей задачи.

\begin{prop}
	Пусть грамматика $G_d$ задает конечное множество строк $L(G_d) = \{\omega_1, \, \dots \, , \omega_n \}$. Необходимо определить, существуют ли строки $\omega'$, для которых верно: $\omega' \in L(G_t)$ и $\omega'$ --- подстрока одной из строк $\omega_i \in L(G_d)$. Данная задача разрешима и принадлежит классу PSPACE-complete.
\end{prop}

\begin{proof}
	Как и в предыдущем доказательстве, используем запись грамматики в виде рекурсивного автомата $R_d$ и рассмотрим автоматы $R_{i, j}$. В данном случае каждый из этих автоматов представляет собой грамматику, которая порождает некоторое конечное множество подстрок исходных строк из $L(G_d)$. Проверка пустоты пересечения такой грамматики с $G_t$ также соответствует условиям теоремы 1.
\end{proof}

В случае, когда грамматика $G_d$ представляет собой бесконечный регулярный язык (т.е. содержит левую и/или правую рекурсию), разрешимость задачи поиска шаблонов установить не удается. Подход, использованный ранее в доказательстве предложений, не может быть применен, так как части рекурсивного автомата, представляющего грамматику $G_d$, также могут содержать рекурсивные переходы, что выходит за рамки условия теоремы 1. Проверка разрешимости и определение класса сложности задачи проверки пустоты пересечения произвольной и регулярной КС-грамматик в настоящее время остаются открытыми проблемами \cite{Nederhof}.

%Пусть $G$ --- произвольная КС-грамматика, $M$ --- конечный автомат. Тогда задача проверки
%\begin{itemize}
%	\item включения языков ($L(M) \subseteq L(G)$) --- неразрешима
%	\item пустоты пересечения ($L(M) \cap L(G) = \emptyset$) --- разрешима (т.к. в пересечении не более чем КС-язык) за полиномиальное время \cite{Hunt}
%	\item регулярности языка $L(G)$ --- неразрешима \cite{Greibach1968}
%\end{itemize} 
%
%Если использовать представление регулярного языка $L(M)$ в виде КС-грамматики $G_r$, то задача проверки пустоты пересечения ($L(G_r) \, \cap \, L(G) = \emptyset$) становится немного интереснее: если $G_r$ 
%\begin{itemize}
%	\item нерекурсивная --- задача из PSPACE \cite{Nederhof} (точнее результата нет (я не нашел, по крайней мере))
%	\item лево- или праволинейная --- ничего не известно (см. последний абзац заключения из \cite{Nederhof})
%	\item принадлежит еще более широкому классу --- тем более ничего не известно
%\end{itemize}

%Еще немного про вложенную рекурсию и регулярность языка. Грамматика без вложенной рекурсии (NSE) порождает регулярный язык \cite{Chomsky} (обратное тоже верно, для регулярного языка можно построить NSE грамматику, т.к. праволинейная, например, --- частный случай NSE). Существует алгоритм, который позволяет проверять грамматику на наличие вложенной рекурсии за полином \cite{Anselmo}. Однако, грамматика с вложенной рекурсией тоже может порождать регулярный язык \cite{Andrei2004}, поэтому задача о проверке регулярности языка, порождаемого КС-грамматикой, остается неразрешимой. 

\subsection{GLL-алгоритм и его модификации}

Классические алгоритмы нисходящего и восходящего синтаксического анализа предполагают использование грамматики, которая является в достаточной мере однозначной. 
В противном случае, управляющие таблицы анализаторов содержат конфликты, из-за чего нельзя гарантировать корректное поведение на любых входных данных. 
Для работы с сильно неоднозначными грамматикам используются алгоритмы \textit{обобщенного синтаксического анализа}, которые позволяют рассмотреть все возможные пути разбора строки и построить соответствующие деревья вывода.
Поиск шаблонов не требует наличия деревьев вывода, поэтому в дальнейшем алгоритмы синтаксического анализа рассматриваются только как механизм, позволяющий определить принадлежность строки языку.

\subsubsection{Оригинальный GLL-алгоритм}

Generalized LL (GLL) \cite{gll} --- алгоритм, обобщающий идеи нисходящего синтаксического анализа. GLL, в отличие от стандартных LL-алгоритмов, позволяет использовать для анализа произвольную \linebreak контекстно-свободную грамматику, в том числе содержащую леворекурсивные правила. Вместе с тем, GLL наследует такие полезные свойства алгоритмов нисходящего анализа, как непосредственная связь с грамматикой и простота отладки и диагностики ошибок.

Для обработки неоднозначностей GLL разделяет стек анализатора на несколько ветвей, каждая из которых соответствует возможному пути разбора. При таком подходе необходимо компактное представление множества стеков, в качестве которого выступает Graph Structured Stack (GSS). В работе \cite{Afroozeh2015gss} была представлена модификация GSS, которая позволяет увеличить эффективность GLL-анализа. Вершины такого представления хранят в себе номер нетерминала и позицию в строке, с которой начался разбор подстроки, соответствующей ему. На ребрах хранятся позиции в грамматике (вида $X \rightarrow \alpha A \cdot \beta$), на которые необходимо вернуться после завершения разбора нетерминала. 
%При помощи GSS также решается проблема бесконечного роста стеков при обработке левой рекурсии: при попытке создать вершину, которая уже существует, в граф добавится

Основной идеей GLL является использование дескрипторов, позволяющих полностью описывать состояние анализатора в текущий момент времени.

\begin{defn}
	Дескриптор --- это тройка (L, u, i), где
	\begin{itemize}
		\setlength\itemsep{0em}
		\item L --- текущая позиция в грамматике вида $A \rightarrow \alpha \cdot \beta$
		\item u --- текущая вершина GSS
		\item i --- позиция во входном потоке 
	\end{itemize}
\end{defn}  

В процессе работы поддерживается глобальная очередь дескрипторов. В начале каждого шага исполнения алгоритм берет следующий в очереди дескриптор и производит действия в зависимости от позиции в грамматике и текущего входного символа, передвигая соответствующие указатели. 
При наличии конфликтов в грамматике алгоритм добавляет дескрипторы для каждого возможного пути анализа в конец очереди.

\subsubsection{Поддержка грамматик в EBNF}

В работе Артема Горохова \cite{Gorokhov2017ebnf} была описана модификация GLL, которая позволяет использовать грамматики, записанные в расширенной форме Бэкуса-Наура (EBNF). Грамматика такого вида трансформируется в соответствующий рекурсивный автомат, в котором затем минимизируется количество состояний. Синтаксический анализ производится без построения управляющих таблиц: алгоритм обходит рекурсивный автомат в соответствии со входным потоком символов. При обработке текущего дескриптора $(C_S, C_U, i)$, где $C_S$ --- вершина автомата (эквивалент позиции в грамматике), $C_U$ --- вершина GSS, $i$ --- позиция в строке, могут возникать следующие ситуации.

\begin{itemize}
	\item $C_S$ --- финальное состояние. Показывает, что разбор текущего нетерминала был завершен. Необходимо осуществить возврат из $C_U$ по меткам на исходящих из нее ребрах.
	\item Присутствует нетерминальный переход из $C_S$. В данном случае необходимо начать разбор указанного нетерминала $X$. Для этого в GSS должна быть создана новая вершина $(X, i)$, если она не создавалась ранее, а текущая вершина автомата изменена на стартовую для $X$.
	\item Присутствует терминальный переход из $C_S$. Необходимо сравнить терминал на ребре автомата с текущим входным символом. Если они совпадают, то осуществить переход в вершину автомата, на которую указывает ребро, и передвинуть указатель в строке.
\end{itemize}

За счет уменьшения количества состояния в автомате удается достичь прироста в производительности по сравнению со стандартным GLL-алгоритмом. 

\subsubsection{Синтаксический анализ графов}

Стандартными входными данными для алгоритмов синтаксического анализа являются линейные последовательности токенов. На основе GLL был разработан алгоритм, который позволяет производить синтаксический анализ регулярных множеств строк, представленных в виде конечного автомата (который, в свою очередь, является ориентированным графом с токенами на ребрах).

Поддержка нелинейного входа не потребовала существенных изменений в оригинальном алгоритме. Дескрипторы модифицированного алгоритма хранят номер вершины входного графа вместо позиции в строке. Также, на шаге исполнения просматривается не единственный текущий символ, а множество символов на ребрах, исходящих из текущей вершины.

Производительность данного алгоритма, как и обычного GLL, может быть увеличена при помощи представления входной грамматики в виде рекурсивного автомата. В таком случае, алгоритм будет производить обход двух автоматов --- рекурсивного и конечного. Ситуации, возникающие при обработке дексрипторов, не отличаются от описанных ранее ситуаций для линейного входа. Псевдокод данной модификации приведен в приложении.

\subsection{Проект YaccConstructor}

YaccConstructor --- исследовательский проект лаборатории языковых инструментов JetBrains на математико-механическом факультете СПбГУ, направленный на исследования в области лексического и синтаксического анализа. Проект включает в себя одноименную модульную платформу для разработки лексических и синтаксических анализаторов, содержащую большое количество компонент: язык описания грамматик YARD, преобразования над грамматиками и др. Основным языком разработки является F$\#$.

Ранее в рамках YaccConstructor были реализованы генераторы GLL-анализаторов, описание которых было приведено в данном обзоре. 
