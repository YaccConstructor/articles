% Тут используется класс, установленный на сервере Papeeria. На случай, если
% текст понадобится редактировать где-то в другом месте, рядом лежит файл matmex-diploma-custom.cls
% который в момент своего создания был идентичен классу, установленному на сервере.
% Для того, чтобы им воспользоваться, замените matmex-diploma на matmex-diploma-custom
% Если вы работаете исключительно в Papeeria то мы настоятельно рекомендуем пользоваться
% классом matmex-diploma, поскольку он будет автоматически обновляться по мере внесения корректив
%

% По умолчанию используется шрифт 14 размера. Если нужен 12-й шрифт, уберите опцию [14pt]
\documentclass[14pt]{matmex-diploma}
%\documentclass[14pt]{matmex-diploma-custom}

\begin{document}
\filltitle{ru}{
    chair              = {Кафедра Системного программирования},
    title              = {Синтаксический анализ графов с помеченными вершинами и ребрами},
    type               = {coursework},
    position           = {студента},
    group              = 344,
    author             = {Ершов Кирилл Максимович},
    supervisorPosition = {ст. преп., к.\,ф.-м.\,н.},
    supervisor         = {Григорьев С.\,В.},
}
\maketitle
\tableofcontents
\section*{Введение}
    Помеченные графы являются удобным способом представления различных структурированных данных. Такие графы используются, например, в биоинформатике, логистике, графовых базах данных.
    
    Иногда для представления данных с использованием графов обходятся только метками на рёбрах. Но в некоторых случаях метки на вершинах позволяют более наглядно отображать зависимости между сущностями. К примеру, в биоинформатике существует большое количество данных, содержащих взаимосвязь между генами и белками. Такие данные удобно представлять в виде графа, вершины которого помечены определенными генами и белками, а ребра показывают их отношение (например, ген кодирует белок).
     
     Часто возникает необходимость извлекать из графа пути, удовлетворяющие какому-либо запросу. Одним из способов задавать классы путей являются КС-грамматики. Пути рассматриваются как строки, состоящие из меток на рёбрах и вершинах. Остаётся проверить, принадлежит ли эта строка данному КС-языку. Для всех КС-грамматик существует эффективный алгоритм синтаксического GLL \cite{gll}, основанный на идее рекурсивного спуска. На основе этого алгоритма и планируется реализовать возможность выполнять запросы к помеченным графам.

\section{Постановка задачи}
\begin{itemize}
    \item В рамках проекта YaccConstructor \cite{YaccConstructorPage} реализовать алгоритм на основе GLL, выполняющий поиск путей в графе с помеченными вершинами и рёбрами по заданной КС-грамматике
    \item протестировать алгоритм на реальных данных и сравнить производительность с существующими решениями
\end{itemize}

\section{Обзор}
 Для реализации запросов к помеченным графам широко используются регулярные грамматики. Однако их возможностей бывает недостаточно для формулирования нужного запроса. Поэтому хотелось бы иметь возможность писать более выразительные запросы к графам, используя КС-грамматики.
    \subsection{Синтаксический анализ КС-грамматик}
    Для синтаксического анализа строки по произвольной КС-грамматике существуют различные алгоритмы. Например, Early parser \cite{Early} осуществляет разбор входной последовательности сверху-вниз и использует принцип динамического программирования. Этот алгоритм для произвольных КС-грамматик работает за время $O(n^3)$ , для однозначных за  $O(n^2)$ и за линейное время для большинства LR(k) грамматик. Динамический алгоритм CYK \cite{CYK} также работает за время $O(n^3)$, однако требует приведения КС-грамматики к нормальной форме Хомского. Алгоритмы GLR \cite{glr} и GLL \cite{gll} являются обобщенными версиями анализаторов LR и LL соответственно (поддерживают неоднозначные грамматики), позволяют использовать произвольные КС-грамматики и работают в худшем случае за время  $O(n^3)$. В основе GLL лежит нисходящий анализ, а значит он более прост в реализации. Алгоритм GLL для LL грамматик работает за линейное время.
    \subsection{Синтаксический анализ графов}
    Нахождение путей в графе с помеченными вершинами и рёбрами по КС-грамматике можно свести к задаче поиска путей в графе без помеченных вершин. Это можно сделать просто заменив вершины графа на ребро и две вершины, расположив метку с вершины на новом ребре. Это приведёт к увеличению числа вершин графа в 2 раза, а ребер прибавится на число вершин исходного графа. При достаточно больших входных данных это плохо скажется на производительности. Например, в работе \cite{subgraph} была поставлена задача поиска связного подграфа в графе с помеченными вершинами и рёбрами по заданной КС-грамматике. Для проверки принадлежности пути КС-языку использовался алгоритм Early, а граф предварительно сводился к графу с метками только на рёбрах. Алгоритм тестировался на реальных данных, и для 300 пар вершин с максимальной длиной пути 8 время работы достигало 240 секунд, что делает этот алгоритм мало применимым на практике.
    
    На кафедре СП Артёмом Гороховым в YaccConstructor был реализован алгоритм \cite{conjCF} на основе GLL, выполняющий запросы к графам с помеченными рёбрами по конъюктивной грамматике. Такие грамматики расширяют класс КС-грамматик. При описании продукций грамматики используется операция конъюнкции, что даёт возможность отсеивать ненужные цепочки. Однако тесты показали, что с конъюктивными грамматиками алгоритм работает в несколько раз медленнее, чем с контекстно-свободными.
    \subsection{YaccConstructor}
    На кафедре Системного программирования в лаборатории языковых инструментов разрабатывается проект YaccConstructor. Это платформа для исследований в области синтаксического анализа, написанная на языке F\#. YaccConstructor позволяет создавать синтаксические анализаторы и имеет модульную архитектуру. Для построения анализатора  выбирается фронтенд для обработки грамматик, выполняются необходимые преобразования и по указанному генератору строится нужный результат. В рамках этого проекта реализованы различные варианты алгоритма GLL.

\section{Заключение}
Результаты, достигнутые на данный момент:
\begin{itemize}
    \item Написан обзор предметной области
    \item Реализован прототип, который находит начало и конец пути в графе с помеченными вершинами и рёбрами по заданной КС-грамматике
\end{itemize}
В дальнейшем планируется добавить в алгоритм поддержку SPPF (shared packed parse forest) \cite{SPPF} для вывода найденных путей, а также  протестировать алгоритм на реальных данных и сравнить с существующими решениями.
\setmonofont[Mapping=tex-text]{CMU Typewriter Text}
\bibliographystyle{ugost2008ls}
\bibliography{diploma.bib}
\end{document}
