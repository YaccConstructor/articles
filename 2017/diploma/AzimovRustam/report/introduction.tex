% У введения нет номера главы
\section*{Введение}

Во многих областях возникают задачи поиска путей в графах, удовлетворяющих определенным условиям. Например, на искомые пути могут быть наложены ограничения на длину, или производится поиск лишь простых путей. Но при работе со сложными системами зачастую таких ограничений бывает недостаточно. Поэтому широко распространено использование ограничений на метки ребер/вершин путей помеченного графа. Для заданного алфавита $\Sigma$ и ориентированного графа $G$, ребра которого помечены символами из $\Sigma$, естественно выбрать в качестве ограничений на пути --- формальный язык $L$, где для искомых путей $p$ графа $G$ выполняется $l(p) \in L$. Здесь $l(p)$ означает слово $w \in \Sigma^*$, полученное последовательной конкатенацией меток пути $p$. Задачи поиска путей в графе, которые используют такие ограничения с формальными языками, являются задачами синтаксического анализа графов.

Задачи с различным сочетанием описанных ограничений используются в таких областях, как графовые базы данных~\cite{graphDB}, биоинформатика~\cite{Anderson} и др. Кроме того, в контексте задачи бывает необходимо отвечать на различного рода вопросы, связанные с искомыми в графе путями. Если рассматривать процесс поиска ответа на такие вопросы, как вычисление запроса для графовой структуры данных, то типы вопросов, на которые отвечает задача принято называть семантикой запроса. Например, в задачах, использующих реляционную семантику запроса, для каждой пары вершин $m, n$ графа $G$ и нетерминала $N$ входной грамматики необходимо ответить, существует ли путь $p$ из вершины $m$ в вершину $n$, такой что $l(p)$ принадлежит языку, порожденному нетерминалом $N$.

В качестве формальных грамматик, описывающих язык в задачах синтаксического анализа графа, широко используются регулярные и контекстно-свободные грамматики. Но существуют грамматики, обладающие большей выразительной мощностью. Например, конъюнктивные грамматики~\cite{conjunctiveGrammar}, которые уже нашли свое применение в биоинформатике при работе с РНК~\cite{ConjunctiveRNA}. Чтобы исследовать применимость ранее не используемых грамматик в задачах синтаксического анализа графов, необходимо структурировать и привести к общей терминологии существующие результаты этой области.

В данной работе будет произведен обзор существующих результатов для различных вариаций задачи синтаксического анализа графов.



