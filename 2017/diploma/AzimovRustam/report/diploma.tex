% Тут используется класс, установленный на сервере Papeeria. На случай, если
% текст понадобится редактировать где-то в другом месте, рядом лежит файл matmex-diploma-custom.cls
% который в момент своего создания был идентичен классу, установленному на сервере.
% Для того, чтобы им воспользоваться, замените matmex-diploma на matmex-diploma-custom
% Если вы работаете исключительно в Papeeria то мы настоятельно рекомендуем пользоваться
% классом matmex-diploma, поскольку он будет автоматически обновляться по мере внесения корректив
%

% По умолчанию используется шрифт 14 размера. Если нужен 12-й шрифт, уберите опцию [14pt]
\documentclass{matmex-diploma}
%\documentclass[14pt]{matmex-diploma-custom}
\usepackage{algpseudocode}
\usepackage{algorithm}
\usepackage{caption}
\usepackage{algorithmicx}
\usepackage{amssymb}
\usepackage{listings}
\usepackage{graphicx} 
\usepackage{subcaption}
\usepackage[flushleft]{threeparttable}
\usepackage{longtable}
\usepackage{epstopdf}
\usepackage{chngcntr}
%\counterwithin{listing}{chapter}
%\counterwithout{figure}{chapter}
%\counterwithout{table}{chapter}

\usepackage{minted}
\usepackage{verbments}

\begin{document}

\algtext*{EndWhile}% Remove "end while" text
\algtext*{EndIf}% Remove "end if" text
\algtext*{EndFor}% Remove "end for" text
\algtext*{EndFunction}% Remove "end function" text

\renewcommand{\lstlistingname}{Листинг}
\renewcommand\listingscaption{Листинг}

% Год, город, название университета и факультета предопределены,
% но можно и поменять.
% Если англоязычная титульная страница не нужна, то ее можно просто удалить.
\filltitle{ru}{
    chair              = {Математическое обеспечение и администрирование \\ информационных систем \\ \vspace{5mm} Системное Программирование},
    title              = {Обзор задач синтаксического анализа графов},
    % Здесь указывается тип работы. Возможные значения:
    %   coursework - Курсовая работа
    %   diploma - Диплом специалиста
    %   master - Диплом магистра
    %   bachelor - Диплом бакалавра
    type               = {coursework},
    position           = {студента},
    group              = 546,
    author             = {Азимов Рустам Шухратуллович},
    supervisorPosition = {к.\,ф.-м.\,н., ст.\,преп.},
    supervisor         = {Григорьев C.\,В.},
%    reviewerPosition   = {},
%    reviewer           = {},
    chairHeadPosition  = {д.\,ф.-м.\,н., профессор},
    chairHead          = {Терехов А.\,Н.},
%   university         = {Санкт-Петербургский Государственный Университет},
%   faculty            = {Математико-механический факультет},
%   city               = {Санкт-Петербург},
%   year               = {2013}
}
\filltitle{en}{
    type               = {coursework},
    chair              = {Software and Administration of Information Systems \\ \vspace{5mm} Software Engineering},
    title              = {Survey of graph parsing problems},
    author             = {Rustam Azimov},
    supervisorPosition = {Senior Lecturer},
    supervisor         = {Semen Grigorev},
%    reviewerPosition   = {},
%    reviewer           = {},
    chairHeadPosition  = {Professor},
    chairHead          = {Andrey Terekhov},
}
\maketitle
\tableofcontents

\section{Introduction}

Scalable high-performance graph analysis is an actual challenge.
There is a big number of ways to attack this challenge~\cite{Coimbra2021} and the first promising idea is to utilize general-purpose graphic processing units (GPGPU).
Such existing solutions, as CuSha~\cite{10.1145/2600212.2600227} and Gunrock~\cite{7967137} show that utilization of GPUs can improve the performance of graph analysis, moreover it is shown that solutions may be scaled to multi-GPU systems.
But low flexibility and high complexity of API are problems of these solutions.

The second promising thing which provides a user-friendly API for high-performance graph analysis algorithms creation is a GraphBLAS API~\cite{7761646} which provides linear algebra based building blocks to create graph analysis algorithms.
The idea of GraphBLAS is based on a well-known fact that linear algebra operations can be efficiently implemented on parallel hardware.
Along with that, a graph can be natively represented using matrices: adjacency matrix, incidence matrix, etc.
While reference CPU-based implementation of GraphBLAS, SuiteSparse:GraphBLAS~\cite{10.1145/3322125}, demonstrates good performance in real-world tasks, GPU-based implementation is challenging.

One of the challenges in this way is that real data are often sparse, thus underlying matrices and vectors are also sparse, and, as a result, classical dense data structures and respective algorithms are inefficient. 
So, it is necessary to use advanced data structures and procedures to implement sparse linear algebra, but the efficient implementation of them on GPU is hard due to the irregularity of workload and data access patterns.
Though such well-known libraries as cuSPARSE show that sparse linear algebra operations can be efficiently implemented for GPGPU, it is not so trivial to implement GraphBLAS on GPGPU. 
First of all, it requires \textit{generic} sparse linear algebra, thus it is impossible just to reuse existing libraries which are almost all specified for operations over floats.
The second problem is specific optimizations, such as masking fusion, which can not be natively implemented on top of existing kernels.
Nevertheless, there is a number of implementations of GraphBLAS on GPGPU, such as GraphBLAST~\cite{yang2019graphblast}, GBTL~\cite{7529957}, which show that GPGPUs utilization can improve the performance of GraphBLAS-based graph analysis solutions.
But these solutions are not portable because they are based on Nvidia Cuda stack.
Moreover, the scalability problem is not solved: all these solutions support only single-GPU, not multi-GPU computations.

To provide portable GPU implementation of GraphBLAS API we developed a \textit{SPLA} library\footnote{Source code available at: \url{https://github.com/JetBrains-Research/spla}}.
This library utilizes OpenCL for GPGPU computing to be portable across devices of different vendors.
Moreover, it is initially designed to utilize multiple GPGPUs to be scalable.
To sum up, the contribution of this work is the following.
\begin{itemize}
    \item Design of portable GPU GraphBLAS implementation proposed. The design involves the utilization of multiple GPUS. Additionally, the proposed design is aimed to simplify library tuning and wrappers for different high-level platforms and languages creation. 
    \item Subset of GraphBLAS API, including such operations as masking, matrix-matrix multiplication, matrix-matrix e-wise addition, is implemented. The current implementation is limited by COO and CSR matrix representation format and uses basic algorithms for some operations, but work in progress and more data formats will be supported and advanced algorithms will be implemented in the future.
    \item Preliminary evaluation on such algorithms as breadth-first search (BFS) and triangles counting (TC), and real-world graphs shows portability across different vendors and promising performance: for some problems Spla is comparable with GraphBLAST. Surprisingly, for some problems, the proposed solution on embedded Intel graphic card shows better performance than SuiteSparse:GraphBLAS on the respective CPU. At the same time, the evaluation shows that further optimization is required.
\end{itemize} 
\section{Постановка задачи}
Целью данной работы является разработка алгоритма синтаксического анализа данных, представленных в виде контекстно-свободной грамматики. Для ее достижения были поставлены следующие задачи.
\begin{itemize}
	\item Определить ограничения, при которых синтаксический анализ \linebreak контекстно-свободного представления является разрешимой задачей.
	\item Разработать алгоритм синтаксического анализа КС-представления данных с учетом поставленных ограничений.
	\item Реализовать предложенный алгоритм.
	\item Провести экспериментальное исследование.
\end{itemize}
%\section{Related Works}

Our parsing algorithm is based on a RNGLR-algorithm presented by Elizabeth Scott 
and Adrian Johnstone in~\cite{RNGLR}. In order to better understand the paper, a reader 
should be familiar to its principles of work, so we briefly describe RNGLR-algorithm 
in this section.  Also we point out differences between our approach and existing
tools which operate with regular approximation of string-embedded language since
we use such type of approximation as input for our algorithm.

% Abstract parsing -- Doh

\subsection{Regular Approximation of Sting-Embedded Language}
Some tools are aimed to build high quality regular approximation. For example, 
Stranger~\cite{Stranger} which use forward reachability analysis to compute 
over-approximation of all string values for program. Further analysis in Stranger 
is based on patterns detection in approximation or generation finite subset of 
strings for analyzing with standalone tools. Implementation of our algorithm may 
use such tools as input generator.

Paper~\cite{JSA} presents Java String Analyzer (JSA) � tool for static syntax 
correctness checking of embedded SQL statements.  This tool build regular approximation 
with Mohri-Nederhof~\cite{MohriNederhof} algorithm and then check its inclusion into reference grammar 
without parsing and forest construction.
 
Our algorithm is inspired by Alvor~\cite{Alvor} which apply GLR-based technique 
for syntax correctness checking of regular approximation. Key difference of our 
algorithm is building of parse forest finite representation. 

\subsection{RNGLR}
Generalized LR parsing algorithm was presented by Masaru Tomita~\cite{Tomita}
as a solution for natural language processing and was intended to handle ambiguous
context-free grammars. Ambiguities of grammar produce Shift/Reduce and 
Reduce/Reduce conflicts. The algorithm uses parser tables similar to classical LR-tables,
each cell of which can contain multiple actions in case of conflicts. The general approach of 
the algorithm is to carry out all possible actions in these situations. 

However, Tomita's algorithm failed to process general context-free grammars.  
Elizabeth Scott and Adrian Johnstone presented Right-Nulled Generalized LR algorithm~\cite{RNGLR}
which extends and corrects Tomita's GLR parsing methods by
specific way of handling \emph{right nullable} rules (i.e. rules of the form 
$\mathrm{A} \rightarrow \alpha \beta$, where $\beta$ reduces to the empty string). 
That is, not only reductions for items $\mathrm{A} \rightarrow \alpha \cdot$ are 
applied, but also for the items of the form  $\mathrm{A} \rightarrow \alpha \cdot 
\beta$, where $\beta \Rightarrow \epsilon$. Thus, reduction length -- the number of 
symbols to be reduced to a nonterminal -- may be less than or equal to the length 
of righthand side of the rule. There are also possible reductions of 0-length, 
also called as $\epsilon$-reductions, corresponding to items of the form $\mathrm{A} 
\rightarrow \cdot$. 

To represent the set of stacks produced during conflict processing efficiently,
RNGLR algorithm uses Graph Structured Stack. GSS is an ordered graph, 
vertices of which corresponds to elements of classical stack and edges link sequential 
elements together. Each vertex can have multiple incoming edges and by means of 
it be shared between several stacks. Vertex is a pair $(s, l)$, where $s$ is a 
parser state and $l$ is a level -- position in an input string. Vertices in GSS 
are unique and there is no multiple edges. GSS construction routine is illustrated with 
\emph{addVertex} and \emph{addEdge} functions in Algorithm~\ref{rnglr}.

RNGLR-algorithm reads an input from left to right, one token at a time, and 
constructs levels of GSS sequentially for each position in the input. In the 
main loop of the algorithm for each token from the input, firstly, all possible 
reductions are applied (see \emph{reduce} function in Algorithm~\ref{rnglr}), and then the next token 
is shifted (see \emph{push} function in Algorithm~\ref{rnglr}).
\begin{algorithm}[!ht]
\begin{algorithmic}[1]
\caption{RNGLR algorithm}
\label{rnglr}
  
\Function{addVertex}{$level, state$}
  \If{GSS does not contain vertex $v = (level, state)$}
    \State{add new vertex $v = (level, state)$ to GSS}
    \State{calculate the set of shifts by $v$ and the next token and add them to $\mathcal{Q}$}
    \State{calculate the set of zero-reductions by $v$ and the next token and add them to $\mathcal{R}$}
  \EndIf
  \State{\Return{$v$}}
\EndFunction

\Function{addEdge}{$v_{h}, level_{t}, state_{t}, isZeroReduction$}
  \State{$v_{t} \gets$ \Call{addVertex}{$level_{t}, state_{t}$}}
  \If{GSS does not contain edge from $v_{t}$ to $v_{h}$}
    \State{add new edge from $v_{t}$ to $v_{h}$ to GSS}
    \If{not $isZeroReduction$}
      \State{calculate the set of reductions by $v$ and the next token and add them to $\mathcal{R}$}
    \EndIf
  \EndIf
\EndFunction

\Function{reduce}{}
  \While{$\mathcal{R}$ is not empty}
    \State{$(v, N, l) \gets \mathcal{R}.Dequeue()$}
    \State{find the set $\mathcal{X}$ of vertices reachable from $v$ along the path of length $(l-1)$, or length $0$ if $l=0$}
    \ForAll{$v_{h} = (level_{h}, state_{h})$ in $\mathcal{X}$}
      \State{$state_{t} \gets$ calculate new state by $state_{h}$ and nonterminal $N$}
      \State{\Call{addEdge}{$v_{h}, v.level, state_{tail}, (l=0)$}}
    \EndFor
  \EndWhile
\EndFunction

\Function{push}{}
  \State{$\mathcal{Q^{'}} \gets$ copy $\mathcal{Q}$}
  \While{$\mathcal{Q^{'}}$ is not empty}
    \State{$(v, state) \gets \mathcal{Q}.Dequeue()$}
    \State{\Call{addEdge}{$v, v.level + 1, state, false$}}
  \EndWhile
\EndFunction

\end{algorithmic}
\end{algorithm}

\clearpage
\setmonofont[Mapping=tex-text]{CMU Typewriter Text}
\section{Общая таблица для классов задач синтаксического анализа графов}

В данном разделе приведена таблица, содержащая существующие результаты для различных классов задач в области синтаксического анализа графов. Данные результаты были приведены к общей терминологии для упрощения таблицы.

Рассматриваемые классы задач отличаются:
\begin{itemize}
    \item семантикой запроса (поиск кратчайшего пути, простого пути и т.д.);
    \item типом входного графа (без циклов, планарный и т.д.);
    \item классом входной формальной грамматики (регулярная грамматика, контекстно-свободная и т.д.).
\end{itemize}

\begin{table}[h!]
 \centering
 \includegraphics[width=17cm]{pictures/table_short.png}
 \caption{Общая таблица, отражающая связь классов задач синтаксического анализа графов и их характеристик. $CFG$ и $CSG$ означают контекстно-свободную и контекстно-зависимую грамматики соответственно. $FP$ означает, что искомый путь может быть найден на детерминированной машине Тьюринга за полиномиальное время, даже если спецификация формального языка является частью входных данных. $m$ означает максимальную длину пути в графе, а $f = |N||V|^{2}((|N||V|^{2}) log(|N||V|^{2}) + |P||V|^{3} + min(|N|,|P|) |E|) + 2^{|N||V|^{2} - 1}$.}
 \label{table}
\end{table}

\clearpage
\setmonofont[Mapping=tex-text]{CMU Typewriter Text}
\bibliographystyle{ugost2008ls}
\bibliography{diploma.bib}
\end{document}
