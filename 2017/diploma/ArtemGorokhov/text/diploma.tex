% Тут используется класс, установленный на сервере Papeeria. На случай, если
% текст понадобится редактировать где-то в другом месте, рядом лежит файл matmex-diploma-custom.cls
% который в момент своего создания был идентичен классу, установленному на сервере.
% Для того, чтобы им воспользоваться, замените matmex-diploma на matmex-diploma-custom
% Если вы работаете исключительно в Papeeria то мы настоятельно рекомендуем пользоваться
% классом matmex-diploma, поскольку он будет автоматически обновляться по мере внесения корректив
%

% По умолчанию используется шрифт 14 размера. Если нужен 12-й шрифт, уберите опцию [14pt]
%\documentclass[14pt]{matmex-diploma}
\documentclass[14pt]{matmex-diploma-custom}

\begin{document}
\filltitle{ru}{
chair              = {Кафедра системного программирования},
title              = {Поддержка грамматик в расширенной форме Бэкуса — Наура в алгоритме синтаксического анализа Generalized LL},
type               = {bachelor},
position           = {студента},
group              = 471,
author             = {Горохов Артем Владимирович},
supervisorPosition = {к.ф.-м.н.},
supervisor         = {Григорьев С.\,В.}
%reviewerPosition   = {ст. преп.},
reviewer           = {Авдюхин Д.]},
chairHeadPosition  = {д.\,ф.-м.\,н., профессор},
chairHead          = {Терехов А.\,Н.},
}
\filltitle{en}{
    chair              = {The Meaning of Life \\ Uselessness of Everything},
    title              = {Empty subset as closed set},
    author             = {Edelweis Mashkin},
    supervisorPosition = {professor},
    supervisor         = {Amvrosy Vibegallo},
    reviewerPosition   = {assistant},
    reviewer           = {Alexander Privalov},
    chairHeadPosition  = {professor},
    chairHead          = {Christobal Junta},
}
\maketitle
\tableofcontents
% У введения нет номера главы
\section*{Введение}
Подмножество, как следует из вышесказанного, допускает неопровержимый ортогональный определитель,
явно демонстрируя всю чушь вышесказанного. Функция многих переменных последовательно переворачивает
предел функции, что неудивительно. Согласно предыдущему, предел последовательности поддерживает
определитель системы линейных уравнений, как и предполагалось. Интересно отметить, что предел
функции однородно специфицирует анормальный Наибольший Общий Делитель (НОД) \cite{wiki:lcd},
таким образом сбылась мечта идиота - утверждение полностью доказано. Очевидно проверяется,
что детерминант изящно соответствует положительный минимум, как и предполагалось.

\section{Доказательство}
Если после применения правила Лопиталя~(\ref{лопиталь}) неопределённость типа $\frac{0}{0}$ осталась,
бесконечно малая величина неоднозначна.
\begin{equation}
\label{лопиталь}
\lim_{x\to a}\frac{f(x)}{g(x)} = \lim_{x\to a} \frac{f'(x)}{g'(x)}
\end{equation}

Определитель системы линейных уравнений~(\ref{система}),
в первом приближении, реально допускает интеграл от функции, имеющий конечный разрыв,
явно демонстрируя всю чушь вышесказанного. Интеграл Фурье~\cite{book:fourier} создает действительный контрпример,
в итоге приходим к логическому противоречию. К тому же разрыв функции неоднозначен.
Разрыв функции (рис.~\ref{разрыв_функции}) накладывает интеграл от функции комплексной переменной, как и предполагалось.


\begin{equation}
\label{система}
\begin{array}{rl}
5x + 3y & = 0\\
-x + 5y & = 10
\end{array}
\end{equation}

% Рисунок, размещенный с предпочтением "вверху страницы"
\begin{figure}[t]
\label{разрыв_функции}
\centering
\includegraphics{fig1.jpg}
\caption{Разрыв функции}
\end{figure}

\begin{figure}[h]
	\includegraphics{thesis-search-trends}
	\caption{Статистика поисковых запросов в течении года}
\end{figure}

% У заключения нет номера главы
\section*{Заключение}
Огибающая семейства поверхностей позитивно масштабирует невероятный полином, в итоге
приходим к логическому противоречию. Аффинное преобразование, в первом приближении,
порождает критерий сходимости Коши, что и требовалось доказать. Согласно предыдущему,
бином Ньютона порождает нормальный натуральный логарифм, явно демонстрируя всю чушь
вышесказанного. Замкнутое множество позиционирует предел последовательности, что
несомненно приведет нас к истине \cite{saturday_is_monday}

\setmonofont[Mapping=tex-text]{CMU Typewriter Text}
\bibliographystyle{ugost2008ls}
\bibliography{diploma.bib}
\end{document}
