\documentclass[12pt]{article}  % standard LaTeX, 12 point type
\usepackage{amsfonts,latexsym}
\usepackage{amsthm}
\usepackage{amssymb}
\usepackage[utf8]{inputenc} % Кодировка
\usepackage[english,russian]{babel} % Многоязычность

\newtheorem{theorem}{Theorem}[section]
\newtheorem{proposition}[theorem]{Proposition}
\newtheorem{lemma}[theorem]{Lemma}
\newtheorem{corollary}[theorem]{Corollary}
\newtheorem{conjecture}[theorem]{Conjecture}

\theoremstyle{definition}
\newtheorem{definition}{Определение}[section]
\newtheorem{example}{Example}[section]

% unnumbered environments:

\theoremstyle{remark}
\newtheorem*{remark}{Remark}
\newtheorem*{notation}{Notation}
\newtheorem*{note}{Note}

\setlength{\parskip}{5pt plus 2pt minus 1pt}
%\setlength{\parindent}{0pt}

\usepackage{color}
\usepackage{listings}
\usepackage{caption}
\usepackage{graphicx}
\usepackage{ucs}

\graphicspath{{pics/}}


\lstnewenvironment{algorithm}[1][]
{   
    \lstset{ 
        frame=tB,
        numbers=left, 
        mathescape=true,
        numberstyle=\small,
        basicstyle=\small, 
        inputencoding=utf8, 
        extendedchars=\true,
        keywordstyle=\color{black}\bfseries,
        keywords={,function, procedure, return, datatype, function, in, if, else, for, foreach, while, denote, do, and, then, assert,} 
        numbers=left,
        xleftmargin=.04\textwidth,
        #1 % this is to add specific settings to an usage of this environment (for instnce, the caption and referable label)
    }
}
{}

\newcommand{\tab}[1][0.3cm]{\ensuremath{\hspace*{#1}}}





\title{О странных деревьях}
\author{Semyon Grigorev}
%\date{\today}

\begin{document}

\maketitle

Все знают про контекстно-свободные грамматики и некоторые даже любят их.
Некоторые что-то слышали про конъюнктивные и даже булевы грамматики и побаиваются их.
И есть за что. 
Давайте рассмотрим простой язык $L_0 = \{\omega \omega \ | \ \omega \in \{a,b\}^*\}$. Данный язык не является контекстно-свободным. Но для него в работе Александра Охотина “Boolean Grammars” предложена замечательная булева грамматика (обозначим её G0):
$$
\begin{array}{l}
S \rightarrow  -A B \ \& \ -B A \ \& \ C \\
A \rightarrow  X A X \ | \ a  \\
B \rightarrow  X B X \ | \ b  \\
C \rightarrow  X X C \ | \ \varepsilon \\
X \rightarrow  a \ | \ b  \\
\end{array}
$$

Очень просто, правда? 
Задание для самопроверки — убедитесь, что данная грамматика действительно описывает язык $L_0$.
Теперь давайте рассмотрим язык $L_1 = \{\omega c \omega | \omega \in \{a,b\}^*\}$, который получается из языка $L_0$ простым добавлением разделителя `c`, что наводит на мысль, что грамматики этих языков должны быть очень похожи.
Давайте попробуем получить грамматику для $L_1$ из грамматики $G_0$.
Не уверен, что это очень простая задача, поэтому предъявим грамматику для этого языка, приводимую в работе “”. 
Любопытный читатель может попробовать построить дерево вывода цепочки aabcaab в данной грамматике, а так же получить из неё грамматику для языка $L_0$. 

$$
\begin{array}{l}
S \rightarrow C \ \& \ D \\
C \rightarrow a C a \ | \ a C b \ | \ b C a \ | \ b C b \ | \ c  \\
D \rightarrow a A \ \& \ a D \ | \ b B \ \& \ b D \ | \ c E  \\
A \rightarrow a A a \ | \ a A b \ | \ b A a \ | \ b A b \ | \ c E a  \\
B \rightarrow a B a \ | \ a B b \ | \ b B a \ | \ b B b \ | \ c E b \\
E \rightarrow a E \ | \ b E \ | \ \varepsilon                \\
\end{array}
$$

Не самое очевидный результат, не правда ли?
Некоторые трудности в работе с булевыми грамматиками отмечает и Andrew Stevenson в своей диссертации в разделе “Managing Grammar Complexity”. 

Давайте вместо букв оперировать более сложными объектами (предпосылки к этому есть в гиперграмматиках, где оперируют множествами). Деревья. 
Строгое расширение КС. Видна лемма о накачке. 
Структура грамматики более простая. Примеры для пиведённых ранее языков. Ещё для некоторых. 

\begin{thebibliography}{9}

\bibitem{okhotin13}
  Alexander Okhotin,
  ``Parsing by matrix multiplication generalized to Boolean grammars'',
  \emph{Theoretical Computer Science},
  V. 516,
  p. 101--120,
  January 2014

\end{thebibliography}

\end{document}