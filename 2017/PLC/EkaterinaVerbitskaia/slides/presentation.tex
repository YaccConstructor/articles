\documentclass{beamer}
\usepackage{beamerthemesplit}
\usepackage{wrapfig}
\usetheme{SPbGU}
\usepackage{pdfpages}
\usepackage{amsmath}
\usepackage{cmap} 
\usepackage[T2A]{fontenc} 
\usepackage[utf8]{inputenc}
\usepackage[english,russian]{babel}
\usepackage{indentfirst}
\usepackage{amsmath}
\usepackage{tikz}
\usepackage{multirow}
\usepackage[noend]{algpseudocode}
\usepackage{algorithm}
\usepackage{algorithmicx}
\usepackage{listings}

\usepackage{epstopdf}
\usepackage{forest}
\usetikzlibrary{shapes,arrows}
\usepackage{fancyvrb}
\newtheorem{rutheorem}{Theorem}
\beamertemplatenavigationsymbolsempty

\lstdefinelanguage{ocaml}{
keywords={fresh, let, begin, end, in, match, type, and, fun, function, try, with, when, class, 
object, method, of, rec, repeat, until, while, not, do, done, as, val, inherit, 
new, module, sig, deriving, datatype, struct, if, then, else, open, private, virtual, ostap},
sensitive=true,
basicstyle=\small,
commentstyle=\small\itshape\ttfamily,
keywordstyle=\ttfamily\underbar,
identifierstyle=\ttfamily,
basewidth={0.5em,0.5em},
columns=fixed,
fontadjust=true,
literate={->}{{$\;\;\to\;\;$}}1,
morecomment=[s]{(*}{*)}
}

\lstset{
basicstyle=\small,
identifierstyle=\ttfamily,
keywordstyle=\bfseries,
commentstyle=\scriptsize\rmfamily,
basewidth={0.5em,0.5em},
fontadjust=true,
%escapechar=~,
language=ocaml,
mathescape=true
}

\title[]{Ostap: синтаксическое расширение OCaml для создания парсер-комбинаторов с поддержкой левой рекурсии}
\institute[СПбГУ]{
Санкт-Петербургский государственный университет\\
Лаборатория языковых инструментов JetBrains }

\author[Екатерина Вербицкая]{Екатерина Вербицкая}

\date{04 апреля 2017}

\definecolor{orange}{RGB}{179,36,31}

\begin{document}
{

\begin{frame}
  \begin{columns} 
    \begin{column}{4.1cm}
      \begin{center} 
        {\includegraphics[width=1.5cm]{pics/jb.png}} 
      \end{center}
    \end{column}
    \begin{column}{4.1cm}
      \begin{center} 
        \colorbox{black}{\includegraphics[width=3.5cm]{pics/logo-plc-plain.png}}
      \end{center}
    \end{column}
    \begin{column}{4.1cm}
      \begin{center} 
        {\includegraphics[width=1.5cm]{pics/SPbGU_Logo.png}} 
      \end{center}
    \end{column}
  \end{columns}

  \titlepage
\end{frame}
}

\begin{frame}[fragile]
  \transwipe[direction=90]
  \frametitle{Синтаксический анализ}
  Сопоставление последовательности лексем с грамматикой языка
  
  \begin{columns}
    \begin{column}{8cm}
  \begin{minipage}[t]{8cm}

\begin{center}
 $1 + 2 * 3$
\end{center}

\medskip


$$
\begin{array}{crcl}
&e & : & e + t \ | \ t \\
&t & : & t * p \ | \ p \\
&p & : & 0 \ | \ 1 \ | \ 2 \ | \dots | \ 9
\end{array}
$$
\end{minipage}
\end{column}
\begin{column}{5cm}
\begin{forest}
  [e[e[t[p[$1$]]]][$+$][t[t[p[$2$]]][$*$][p[$3$]]]]
\end{forest}
\end{column}
\end{columns}
\end{frame}

\begin{frame}
  \transwipe[direction=90]
  \frametitle{Парсер-комбинаторы}  
  \begin{itemize}
  \item Подход к реализации синтаксического анализа в парадигме функционального программирования
  \item Реализуют нисходящий разбор
  \item Синтаксический анализатор: функция высшего порядка
  \item Позволяет считать семантику ``на лету'', без явного построения деревьев разбора
  \item Анализатор произвольной сложности можно получить путем комбинирования нескольких простых базовых парсеров
  \item Позволяют разбирать некоторые контекстно-зависимые языки
  \end{itemize}
\end{frame}

\begin{frame}[containsverbatim]
  \transwipe[direction=90]
  \frametitle{Тип парсеров}
  
\begin{lstlisting}[frame=single]  
type $\alpha$ tag = Parsed of $\alpha$ | Failed

type ($\sigma$, $\rho$) result = ($\rho$ * $\sigma$) tag

and  ($\sigma$, $\rho$) parse  = $\sigma$ ->   ($\sigma$, $\rho$) result
\end{lstlisting}
\end{frame}


\begin{frame}[fragile]
  \transwipe[direction=90]
  \frametitle{Базовые парсер-комбинаторы}  
\begin{lstlisting}[frame=single]  
let empty s = Parsed ((), s)

let fail  s = Failed 

let return x s = Parsed (x, s)
\end{lstlisting}
\end{frame}

\begin{frame}[fragile]
  \transwipe[direction=90]
  \frametitle{Комбинатор последовательности}  
\begin{lstlisting}[frame=single]  
let seq x y s =
  match x s with
  | Parsed (b, s') -> y b s'
  | Failed -> Failed

let ($\vartriangleright$) = seq
\end{lstlisting}
\end{frame}


\begin{frame}[fragile]
  \transwipe[direction=90]
  \frametitle{Комбинатор альтернативы}  
\begin{lstlisting}[frame=single]  
let alt x y s =
  match x s with
  | Failed -> y s
  | x -> x

let ($\lozenge$) = alt
\end{lstlisting}
\end{frame}


\begin{frame}[fragile]
  \transwipe[direction=90]
  \frametitle{Пример парсера для языка $\{ a^n \ | \ n \geq 0 \}$}  
\begin{lstlisting}[frame=single]  
(* A : $\varepsilon$ | 'a' A *)
let rec a s = empty $\lozenge$ (terminal 'a' $\vartriangleright$ fun _ ->    a) s
\end{lstlisting}
\end{frame}


\begin{frame}[fragile]
  \transwipe[direction=90]
  \frametitle{Семантические действия}  
\begin{lstlisting}[frame=single]  
let map f p s = 
  match p s with
  | Parsed (b, s') -> Parsed (f b, s')
  | x -> x
\end{lstlisting}
\end{frame}

\begin{frame}[fragile]
  \transwipe[direction=90]
  \frametitle{Вспомогательные парсер-комбинаторы}  
\begin{lstlisting}[frame=single]  
let opt p = map (fun x -> Some x) p $\lozenge$ return None
    
let rec many p =
  p $\vartriangleright$ (fun h ->   map (fun t -> h :: t) (many p)) $\lozenge$ 
  return []
    
\end{lstlisting}

\begin{lstlisting}[frame=single]  
let a = many (terminal 'a')
\end{lstlisting}
\end{frame}

\begin{frame}[fragile]
  \transwipe[direction=90]
  \frametitle{Синтаксическое расширение}  
\begin{lstlisting}[frame=single]  
ostap (
  e : x:e "+" y:t {Add (x, y)} 
    | t; 
  t : x:t "*" y:p {Mul (x, y)} 
    | p;
  p : (* parse integer *)
)
\end{lstlisting}
\end{frame}

\begin{frame}[fragile]
  \transwipe[direction=90]
  \frametitle{Парсеры высшего порядка: переиспользование кода}  
\begin{lstlisting}[frame=single]  
ostap (
  e : x:e "+" y:t {Add (x, y)} 
    | t; 
  t : x:t "*" y:p {Mul (x, y)} 
    | p;
  p : (* parse integer *)
)
\end{lstlisting}
  
\begin{lstlisting}[frame=single]  
ostap (
  e : x:e "+" y:t {Add (x, y)} 
    | t; 
  t : x:t "*" y:p {Mul (x, y)} 
    | p;
  p : (* parse double *)
)
\end{lstlisting}	  
  
\end{frame}


\begin{frame}[fragile]
  \transwipe[direction=90]
  \frametitle{Парсеры высшего порядка}  
\begin{lstlisting}[frame=single]  
ostap (
  e[p] : x:e[p] "+" y:t[p] {Add (x, y)} 
       | t[p]; 
  t[p] : x:t[p] "*" y:p    {Mul (x, y)} 
       | p
)
\end{lstlisting}
  
\begin{lstlisting}[frame=single]  
ostap (
  e$_{integer}$: e[(* parse integer *)]
)
\end{lstlisting}
  
\begin{lstlisting}[frame=single] 
ostap (
  e$_{double}$ : e[(* parse double  *)]
)
\end{lstlisting}
\end{frame}


\begin{frame}[fragile]
  \transwipe[direction=90]
  \frametitle{Парсер-комбинаторы и левая рекурсия}  
Являясь реализацией нисходящего анализа, парсер-комбинаторы не способны обрабатывать леворекурсивные правила

\begin{lstlisting}[frame=single]  
ostap (
  e[p] : e[p] "+" t[p] (* to apply e, *)
       | t[p];         (* one needs to apply e... *)
  t[p] : t[p] "*" p    
       | p
)
\end{lstlisting}

Удаление левой рекурсии значительно усложняет спецификацию языка и ухудшает композициональность анализаторов
\end{frame}


\begin{frame}[fragile]
  \transwipe[direction=90]
  \frametitle{Поддержка леворекурсивных анализаторов}
  \begin{itemize} % СНАЧАЛА СУТЬ, ПОТОМ КРЕДИТ, ПРИ ЭТОМ БЕЗ НАЗВАНИЯ, ТОЛЬКО ГОД
    \item Ограничение количества леворекурсивных вызовов длиной непрочитанной строки
    \begin{itemize}
      \item Frost R. A., Hafiz R., Callaghan P. [2008] % Parser combinators for ambiguous leftrecursive grammars
    \end{itemize}
    \item Использование мемоизации для обеспечения завершаемости
    \begin{itemize}
      \item Warth A., Douglass J. R., Millstein T. D. [2008] % Packrat parsers can support left recursion
    \end{itemize}

    \item Требуют, чтобы парсер был первого порядка
  \end{itemize}

  \begin{itemize}
    \item Использование техники CPS для обеспечения завершаемости
    \begin{itemize}
      \item Izmaylova	A., Afroozeh A., van der Storm T. [2016] % Practical, general parser combinators
    \end{itemize}

    \item Фиксируют конкретную семантику
  \end{itemize}  

\end{frame}

%\begin{frame}[fragile]
%  \transwipe[direction=90]
%  \frametitle{Полиморфность относительно типа входного потока}
%\begin{itemize}
%  \item Позволяет инкорпорировать в себя сканер (лексер)
%  \item Повышает композиционность еще сильнее
%  \item Позволяет реализовывать в абстракции входного потока другие полезные фичи
%\end{itemize}

%\end{frame}


\begin{frame}
  \transwipe[direction=90]
  \frametitle{Поддержка левой рекурсии в PEG}
  \begin{itemize}
    \item Medeiros S., Mascarenhas F., Ierusalimschy R. [2014] % Left Recursion in Parsing Expression Grammars 
    \item Динамический поиск наилучшего количества леворекурсивных вызовов
    \item Использует мемоизацию
    \item Поддерживает явную, неявную, взаимную рекурсию
    \item Требуют, чтобы парсер был первого порядка
  \end{itemize}
\end{frame}

\begin{frame}[fragile]
  \transwipe[direction=90]
  \frametitle{Поддержка левой рекурсии в Ostap}
  \begin{itemize} 
    \item Используется идея Medeiros et al
    \item Специальный комбинатор \verb!fix! для поддержки леворекурсивных парсеров высшего порядка
  \end{itemize}
\begin{lstlisting}[frame=single]  
ostap (
  e[p] : e[p] "+" e[p] | p
)
\end{lstlisting}

\begin{lstlisting}[frame=single]  
ostap (
  let e p s = fix (fun e -> ostap (e "+" e | p)) s
)
\end{lstlisting}
\end{frame}


\begin{frame}[fragile]
  \transwipe[direction=90]
  \frametitle{Комбинатор $fix$}

\begin{lstlisting}[frame=single]  
let fix p s = 
  let x' = ref None in  
  let rec fix p s = p (fix p) s in
  let help cur prev =
    match cur, prev with 
    | Failed, _ -> prev
    | Parsed (_,s), Parsed (_,s') when s#pos < s'#pos -> prev
    | _, _ -> cur
  in
  let p x s = 
    match !x' with
    | None -> x' := Some (fun s -> memo x s); help (p x s) (x s)
    | Some x -> help (p x s) (x s)
  in
  fix p s
\end{lstlisting}
\end{frame}

\begin{frame}[fragile]
  \transwipe[direction=90]
  \frametitle{Заключение}
  \begin{itemize} 
    \item Представлена библиотека парсер-комбинаторов и синтаксическое расширение для языка OCaml
    \item Реализована поддержка леворекурсивных спецификаций синтаксических анализаторов
    \begin{itemize}
      \item Позволяет использование парсеров высшего порядка
      \item Сложность растет экспоненциально в зависимости от глубины вложенности рекурсии
    \end{itemize}
  \end{itemize}
\end{frame}


\end{document}