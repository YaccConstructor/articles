\section{Синтаксический анализ контекстно-свободной аппроксимации}

Наш алгоритм принимает на вход управляющие таблицы LL-анализа, построенные по эталонной грамматике, и GFG, который является представлением КС-грамматики --- аппроксимации множества значений выражения. 
Мы переиспользуем основные структуры данных и функции описанного ранее алгоритма анализа регулярной аппроксимации на основе GLL, расширяя данный подход для корректной обработки GFG.

Алгоритм последовательно обходит узлы GFG, производя синтаксический анализ порождаемых им строк. 
Для правильного построения таких строк, согласно определению выводимости строки в GFG-грамматике, для каждого просматриваемого пути необходимо поддерживать баланс call- и return-узлов. 
То есть, при прохождении пути алгоритм должен манипулировать дополнительным стеком (назовем его \textit{CR-стеком}). 
При достижении call-узла в стек добавляется номер return-узла, соответствующего ему; при достижении end-узла необходимо снять со стека номер return-узла и продолжить обход из него. 

Для экономии памяти мы не храним CR-стек для каждой из текущих ветвей работы алгоритма (напомним, что GLL-алгоритм может одновременно рассматривать несколько вариантов разбора строки). 
Вместо этого множество CR-стеков, по аналогии с основным стеков GLL-анализатора, представляется в виде GSS. 
Пример можно увидеть на рисунке \ref{fig:gss}. GSS позволяет хранить только одну копию общих префиксов нескольких стеков, каждый путь в нем соответствует отдельному CR-стеку.

\begin{figure}[h]
	\centering
	\includegraphics[width=6cm]{pictures/gss_cr}
	\caption{Структурированный в виде графа стек}
	\label{fig:gss}
\end{figure}

Для хранения указателя на текущую вершину стека мы добавили в дескрипторы дополнительное поле. 
Таким образом, дескриптором в нашем алгоритме называется пятерка вида $(L, u, i, w, s)$, где $i$ --- номер вершины GFG, $s$ --- указатель на вершину CR-стека в GSS, остальные поля аналогичны тем, которые представлены в дескрипторах оригинального GLL-алгоритма.

Другой особенностью работы с GFG является то, что он, в отличие от регулярной аппроксимации --- детерминированного конечного автомата, допускает возможность неоднозначного выбора пути обхода. 
Подобная ситуация возникает при наличии в исходной грамматике нескольких продукций, содержащих в левой части одинаковый нетерминал. 
Например, GFG на рисунке \ref{fig:gfg} содержит start-узел под номером 3, из которого выходит два ребра с пустой меткой (аналог $\epsilon$-переходов в конечном автомате).

Механизм дескрипторов позволяет решать проблему недетерминированного выбора пути --- для каждого из возможных вариантов создается отдельный дескриптор, который добавляется в очередь исполнения. 
Вернемся к примеру с узлом 3 на рисунке \ref{fig:gfg}. Пусть в текущий момент времени мы имеем дескриптор $(L_1, u_1, i_1, w_1, s_1)$. 
При рассмотрении ребер, выходящих из узла 3, будут созданы дескрипторы $(L_1, u_1, 4, w_1, s_1)$ и $(L_1, u_1, 5, w_1, s_1)$. Если ранее такие дескрипторы не создавались (для контроля за этим в GLL поддерживается глобальное множество создаваемых дескрипторов), они будут добавлены в очередь.

Функции \textbf{dispatcher} и \textbf{add} (проверка и добавление в очередь дескриптора) алгоритма анализа регулярной аппроксимации были незначительно изменены нами для работы с расширенными дескрипторами. 
Функция \textbf{processing} и методы для работы с основным стеком и построения SPPF переиспользованы без изменений.
Обработка start/call/exit-узлов и контроль за состоянием CR-стеков были реализованы во вспомогательной функции \textbf{closure}, псевдокод которой приведен ниже.
Она исполняется перед вызовами \textbf{dispatcher} и \textbf{processing},  %отвечающих за извлечение дескрипторов из очереди и проведение синтаксического анализа соотвественно.
производя рекурсивный обход GFG до тех пор, пока не встретит start- или scan-узел. 
При достижении start-узла создаются дескрипторы для каждого из возможных путей, и управление переходит к \textbf{dispatcher};
scan-узел обрабатывается функцией \textbf{processing} так же, как и вершина конечного автомата в оригинальном алгоритме.
%мы переиспользуем \textbf{processing} вместе с функциями построения SPPF и основного стека анализатора.

%\footnote{Отметим, что $\epsilon$-замыкание для start-узлов в общем случае не решает проблемы недетерминированного выбора, т.к. продукции, например, могут начинаться с одинакового терминального символа. Кроме того, замыкание в случае левой рекурсии приводит к появлению петель и усложняет логику работы алгоритма}

\section{Closure properties of languages with polynomial rational indices}
\label{sec:closure}
Given a context-free language $L$ with the polynomial rational index, it is interesting to find which language operations preserve this property.  Boasson et al. \cite{RatBasic} give the following useful relations for polynomial indices of two languages $L$ and $L'$.
\begin{lemma}[\cite{RatBasic}]
\label{lem:closure}
Context-free languages with polynomial rational indices are closed under intersection with a regular language, union, concatenation, the Kleene star, homomorphism and inverse homomorphism. More precisely,
\begin{itemize}
\item $\rho_{L \cup L'}(n) \le  \max{(\rho_L(n), \rho_{L'}(n))} $
\item $\rho_{LL'}(n) \le \rho_L(n) + \rho_{L'}(n)$
\item $\rho_{L^{*}}(n) \le n(\rho_L(n))$
\item $\rho_{L \cap R}(n) \le \rho_L(nm)$, where $R$ is a regular language recognised by an $m$-state automaton
\item $\rho_{h(L)}(n) \le \rho_L(n)$ and $\rho_{h^{-1}(L)}(n) < n(\rho_L(n) +1)$, where $h: \Sigma^* \rightarrow \Delta^*$ is a homomorphism
\item $\rho_{\tau(L)}(n) \le (mn + 1)\rho_L(mn)$, where $\tau$ is a rational transduction and $m$ is some integer.
\end{itemize}
\end{lemma}
 From the relations above it is easy to see that the family of context-free languages with polynomial rational indices is a full trio. Every full trio is closed under left and right quotient with regular languages, prefix, suffix, infix, and outfix \cite{GinsburgAlgebraic}. Obviously, CFLs with the polynomial rational indices languages are closed under reversal.  Next we show that context-free languages with the polynomial rational indices are closed under insertion of a regular language.
\begin{theorem}
Context-free languages with the polynomial rational indices are closed under the insertion of a regular language. 
\\Particularly, $\rho_{L_{INSERT(K)}}(n) \le (mn + 1)\rho_L(mn)$, where $m$ is the number of states in the NFA accepting $K$.
\end{theorem}
\begin{proof}
 Let $L$ be a language with the polynomial rational index over an alphabet $\Sigma$ and $K$ be a regular language over an alphabet $\Delta$, where an NFA $M(K)$ with $m$ states is an NFA accepting $K$.  Define a homomorphism $h: \Delta^*  \rightarrow \bar{\Delta}^{*}$, such that $h(a)=\bar{a}$,  $\forall a \in \Delta$. In simple words, $h$ makes all symbols from $\Delta$ ``marked''. Then, by defining a homomorphism $g$, such that $g(a) = \varepsilon$, $\forall a \in \bar{\Delta}$ ($g$ erases the symbols of $\bar{\Delta}$), one can insert an arbitrary number of symbols from $\bar{\Delta}$ into strings in $L$ using an inverse homomorphism $g^{-1}$. To obtain a string from $L_{INSERT(K)}$ it is left to intersect $g^{-1}(L)$ with a regular set $K'$ containing strings in the form $xyz$, where $x, z \subseteq \Sigma^{*}$ and $y \in h(K)$. Then ``marked'' symbols from  $\bar{\Delta}$ is unmarked by a homomorphism $\phi:  \bar{\Delta^*}  \rightarrow \Delta^{*}$, where $\phi(\bar{a}) = a$, $\forall \bar{a} \in \bar{\Delta}$. Finally, every word $w' \in L_{INSERT(K)}$ can be written as $\phi(g^{-1}(w) \cap K') = \tau(w)$, where $w \in L$ and $\tau$ is a rational transduction. By Lemma~\ref{lem:closure}, languages with the polynomial rational indices are closed under rational trunsductions, so $L_{INSERT(K)}$ has the polynomial rational index. An NFA $M(K')$ can be easily constructed from $M(K)$ and has $O(m)$ states. Then the value of the rational index $\rho_{L_{INSERT(K)}}(n) \le (mn + 1)\rho_L(mn)$.
\end{proof}



Using closure properties, it is easier to find new subclasses of context-free languages for which the CFL-reachability problem is in NC.
\begin{example}[Metalinear languages \cite{metalinear}.]
Let $G = (\Sigma, N, P, S)$ be a context-free grammar. $G$ is \textit{metalinear} if all productions of $P$ are of the following forms:
\begin{enumerate}
\item $S \rightarrow A_1A_2...A_k$, where $A_i \in N \setminus \{S\}$
\item $A \rightarrow u$, where $A \in N \setminus \{S\}$ and $u \in (\Sigma^*((N \setminus \{S\}) \cup {\varepsilon})\Sigma^*)$
\end{enumerate}


The width of a metalinear grammar is $max\{k\vert S \rightarrow A_1A_2...A_k \}$. Metalinear languages of width 1 are obviously linear languages. It is easy to see that every metalinear language is a union of concatenations of $k$ linear languages. Linear languages have polynomial rational index,  CFLs with the polynomial rational index are closed under concatenation and union, so metalinear languages have the polynomial rational index and, hence, is in NC.
\end{example}



Результатом работы алгоритма является SPPF, представляющий множество деревьев разбора всех строк, порождаемых GFG и одновременно с этим выводимых в эталонной грамматике.