\begin{abstract}
Многие программы в процессе работы формируют из строк исходный код на некотором языке программирования и передают его для исполнения в соответствующее окружение (пример --- dynamic SQL). Для статической проверки корректности динамически формируемого строкового выражения используются различные методы, одним из которых является синтаксический анализ регулярной аппроксимации множества значений такого выражения. Так как большинство языков программирования относится к классу контекстно-свободных, регулярная аппроксимация влечет за собой появление синтаксических ошибок, которые отсутствуют в исходном множестве значений. Использование контекстно-свободной аппроксимации позволяет повысить точность статического анализа. В данной статье будет описан алгоритм синтаксического анализа  контекстно-свободной аппроксимации динамически формируемого кода. ПЕРЕПИСАТЬ
\\
\\
\textbf{Ключевые слова:} синтаксический анализ, динамически формируемый код, контекстно-свободные грамматики, GLL, GFG, dynamic SQL, DSQL 
\end{abstract}
