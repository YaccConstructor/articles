\documentclass[xcolor=table,aspectratio=169]{beamer}
\usepackage{beamerthemesplit}
\usepackage{wrapfig}
\usetheme{SPbGU}
\usepackage{pdfpages}
\usepackage{amsmath}
\usepackage{cmap}
\usepackage[T2A]{fontenc}
\usepackage[utf8]{inputenc}
\usepackage[english]{babel}
\usepackage{indentfirst}
\usepackage{mathtools}
\usepackage{tikz}
\usepackage{multirow}
\usepackage[noend]{algpseudocode}
\usepackage{algorithm}
\usepackage{algorithmicx}
\usepackage{fancyvrb}

\usepackage{minted}

\usetikzlibrary{calc}
\usetikzlibrary{shapes,arrows}
\usetikzlibrary{arrows,automata}
\usetikzlibrary{positioning}

\usepackage{fontawesome}

\usetikzlibrary{shapes.callouts}

\usepackage{xparse}

%for [[ ]]
\usepackage{stmaryrd}


\tikzset{
    invisible/.style={opacity=0,text opacity=0},
    visible on/.style={alt=#1{}{invisible}},
    alt/.code args={<#1>#2#3}{%
      \alt<#1>{\pgfkeysalso{#2}}{\pgfkeysalso{#3}} % \pgfkeysalso doesn't change the path
    },
}

\NewDocumentCommand{\mycallout}{r<> O{opacity=0.8,text opacity=1} m m m}{%
\tikz[remember picture, overlay]\node[align=center, fill=cyan!20, text width=#5cm,
#2,visible on=<#1>, rounded corners,
draw,rectangle callout,anchor=pointer,callout relative pointer={(230:1cm)}]
at (#3) {#4};
}

%\newcommand{\tikzmark}[1]{\tikz[overlay,remember picture,baseline=-0.5ex] \node (#1) {};}



\usepackage{tabularx}
\newcolumntype{Y}{>{\raggedleft\arraybackslash}X}

\renewcommand{\thealgorithm}{}

\newtheorem{mytheorem}{Theorem}
\renewcommand{\thealgorithm}{}

\newcommand{\tikzmark}[1]{\tikz[overlay,remember picture] \node (#1) {};}
\def\Put(#1,#2)#3{\leavevmode\makebox(0,0){\put(#1,#2){#3}}}

\newcommand{\ltz}{$< 1$}


\tikzset{
    state/.style={
           rectangle,
           rounded corners,
           draw=black, very thick,
           minimum height=2em,
           inner sep=2pt,
           text centered,
           },
}

\beamertemplatenavigationsymbolsempty

\title[Multiple-Source CFPQ]{Multiple-Source Context-Free Path Querying in Terms of Linear Algebra}
%\subtitle[YaccConstructor]{Parsing techniques for graph analysis}
\institute[JB Research, SPbSU]{
JetBrains Research, Programming Languages and Tools Lab  \\
Saint Petersburg University
}


\author[Semyon Grigorev]{Arseniy Terekhov, Vlada Pogozhelskaya, Vadim Abzalov, Timur Zinnatulin, \textbf{Semyon Grigorev}}

\date{March 24, 2021}

\begin{document}
{
\begin{frame}[fragile]
  \begin{table}
  \centering
  \begin{tabularx}{\linewidth}{XcX}
    \includegraphics[height=1.5cm]{pictures/JB_logo_RGB_research_vert.pdf} \hfill
    & \begin{minipage}[t]{0.3\textwidth}\center \includegraphics[height=1.5cm]{pictures/EDBT.png} \hfill
      \end{minipage}
    & \hfill \includegraphics[height=1.5cm]{pictures/SPbGU_Logo.png}
  \end{tabularx}
  \end{table}
  \titlepage
\end{frame}
}

\begin{frame}[fragile] \frametitle{Formal Language Constrained Path Querying}
      \begin{minipage}[m]{0.45\linewidth}
  \raisebox{-0.5\totalheight}{\includegraphics[width=\textwidth]{pictures/hierarchical.pdf}}
  \end{minipage}\hfill
  \begin{minipage}[m]{0.5\linewidth}
  Navigation through an edge-labeled graph
  \begin{itemize}
        \item \textbf{Path} specifies a \textbf{word} formed by labels of edges
        \item \textbf{Paths constraint} is a \textbf{language}: path is good if related word in the give language
        \item Constraints expressiveness is related to \textbf{formal languages classes}
  \end{itemize}
  \pause
  Regular path queries (RPQ)
  \begin{itemize}
        \item \textbf{Regular} languages are used as constraints
        \item Which nodes are reachable from \textbf{C} by arbitrary number of $\textbf{Up} \text{ and } \textbf{Down}$ edges?
        \item $\mathcal{L} = (\text{Up} \mid \text{Down})^*$
  \end{itemize}

  \end{minipage}

\end{frame}

\begin{frame}[fragile] \frametitle{Context-Free Path Querying (CFPQ)}
      \begin{minipage}[m]{0.45\linewidth}
  \raisebox{-0.5\totalheight}{\includegraphics[width=\textwidth]{pictures/hierarchical.pdf}}
  \end{minipage}\hfill
  \begin{minipage}[m]{0.5\linewidth}  
  The constraint is a \textbf{context-free language}
  \begin{itemize}
        \item Are nodes A and B on the same level of hierarchy?
        \item Is there a path of form $\textbf{Up}^n \, \textbf{Down}^n$?
        \item Find all paths of form $\textbf{Up}^n \, \textbf{Down}^n$ which start from the~node A
  \end{itemize}
  \pause
  Applications
    \begin{itemize}
      \item Static code analysis [\href{https://dl.acm.org/doi/10.1145/199448.199462}{T. Reps, et al, 1995}]
      \item Graph segmentation [\href{https://dblp.org/rec/conf/icde/0001D19.html}{H. Miao, et al, 2019}]
      \item Biological data analysis [\href{https://pubmed.ncbi.nlm.nih.gov/20134073/}{P. Sevon, et al, 2008}]
      \item ...      
    \end{itemize}
    
  \end{minipage}

  \end{frame}

\begin{frame}[fragile] \frametitle{The Problem}
%\begin{center}
   There is no support of of CFPQ in real-world graph analysis systems (graph databases)
%\end{center}
\pause
   \begin{itemize}     
      \item J. Kuijpers, et al\footnote{Jochem Kuijpers, George Fletcher, Nikolay Yakovets, and Tobias Lindaaker. 2019. An Experimental Study of Context-Free Path Query Evaluation Methods.}: existing algorithms are too slow to be practical (in the context of Neo4j)
      \pause    
      \item A. Terekhov, et al\footnote{Arseniy Terekhov, Artyom Khoroshev, Rustam Azimov, and Semyon Grigorev. 2020. Context-Free Path Querying with Single-Path Semantics by Matrix Multiplication.}: linear algebra based CFPQ algorithm can be performant enough
      \pause
      \item There is no full-stack support of CFPQ
      \begin{itemize}
        \item Grammars instead of full-featured queries
        \item Custom graph storage instead of real-world graph database        
      \end{itemize}
    \end{itemize}
  
\end{frame}


\begin{frame}[fragile] \frametitle{Proposed Solution}
  \begin{itemize}
      \item \textbf{Multiple-Source CFPQ} allows one to reduce computations in real-world cases
      \pause
      \item \textbf{Cypher} extended with \textbf{path patterns}\footnote{Tobias Lindaaker, Path Patterns for Cypher, 2017, \scriptsize \url{https://github.com/thobe/openCypher/blob/rpq/cip/1.accepted/CIP2017-02-06-Path-Patterns.adoc}} allows one to express context-free constraints 
      \pause
      \item \textbf{RedisGraph} database 
      \begin{itemize}
        \item Provides graph storage with matrix-based representation
        \item Contains linear algebra based query engine (SuitSparse:GraphBLAS\footnote{!!!} is used)
        \item Allows one to use Cypher for querying (libcypher.parser\footnote{!!!} is used)
      \end{itemize}
      
      
  \end{itemize}
  
\end{frame}


\begin{frame}[fragile] \frametitle{Multiple-Source CFPQ}
\vspace{-0.5cm}
\tikzmark{zzz}{ }
\small
\begin{algorithmic}[1]
\Function{MultiSrcCFPQ}{$D = (V, E, \Sigma_V, \Sigma_E, \lambda_V, \lambda_E)$, $G=(N,\Sigma,P,S)$, $Src$}
    \State{$T \gets \{T^A \mid  A \in N, T^A[i,j] \gets \textit{false} \text{, for all $i,j$}\} $}

    \State{$TSrc \gets \{TSrc^A \mid  A \in N, TSrc^A[i,j] \gets \textit{false} \text{, for all $i,j$}\}$}

    \ForAll{$ v \in Src$} $TSrc^S[v,v] \gets true$
    \EndFor

    \State $MSrc \gets TSrc^S$

    \ForAll{$A \to x \in P \mid x \in \Sigma_E$} 
        \ForAll{$(v, to) \in E \mid x \in \lambda_E(v,to)$} $T^A[v,to] \gets true$
        \EndFor
    \EndFor

    \ForAll{$A \to x \in P~\mid x \in \Sigma_V$}
        \ForAll{$v \in V~|~ x \in \lambda_V(v)$} $T^A[v,v] \gets true$
        \EndFor
    \EndFor

    \While{$T\ or\ TSrc\ is\ changing$}
        \ForAll{$A \to B C \in P$}
            \State{$M \gets TSrc^A*T^B$}
            \State{$T^A \gets T^A + M*T^C$}
            \State{$TSrc^B \gets TSrc^B + TSrc^A$}
            \State{$TSrc^C \gets TSrc^C + $ \Call{getDst}{$M$}}
        \EndFor
    \EndWhile
    \State \Return $MSrc * T^S$
\EndFunction

\end{algorithmic}


  %}
  \pause
  \onslide<2>{\tikz[overlay,remember picture]{\draw[draw=red,thick,double,fill opacity=0.2] ($ (zzz) + (4.7,-0.1)$) rectangle ($ (zzz) + (13.0,-0.6)$);}}

  \onslide<3>{\tikz[overlay,remember picture]{\draw[draw=red,thick,double,fill opacity=0.2] ($ (zzz) + (0.8,-0.6)$) rectangle ($ (zzz) + (9.8,-4.23)$);}}

  \onslide<4>{\tikz[overlay,remember picture]{\draw[draw=red,thick,double,fill opacity=0.2] ($ (zzz) + (0.8,-4.23)$) rectangle ($ (zzz) + (9.8,-6.76)$);}}

  \onslide<5>{\tikz[overlay,remember picture]{\draw[draw=red,thick,double,fill opacity=0.2] ($ (zzz) + (0.8,-6.76)$) rectangle ($ (zzz) + (9.8,-7.26)$);}}

  %\onslide<5>{\tikz[overlay,remember picture]{\draw[draw=red,thick,double,fill opacity=0.2] ($ (zzz) + (7.8,5.1)$) rectangle ($ (zzz) + (11.8,0.35)$);}}

  %\onslide<6>{\tikz[overlay,remember picture]{\draw[draw=red,thick,double,fill opacity=0.2] ($ (zzz) + (3.9,5.2)$) rectangle ($ (zzz) + (7.8,0.2)$);}}

  %\onslide<7>{\tikz[overlay,remember picture]{\draw[draw=red,thick,double,fill opacity=0.2] ($ (zzz) + (-0.1,4.8)$) rectangle ($ (zzz) + (3.8,4.0)$);}}


%\Function{getDst}{$M$}
%    \State{$A[i,j] \gets \textit{false}$}
%    \ForAll{$(v,to) \in V^2 \mid M[v,to] = true$}
%        \State{$A[to,to] \gets true$}
%    \EndFor
%    \State \Return A
%\EndFunction


\end{frame}


\begin{frame}[fragile] \frametitle{Cypher Extension}
        \begin{minipage}[m]{0.43\linewidth}
  \raisebox{-0.5\totalheight}{\includegraphics[width=\textwidth]{pictures/hierarchical.pdf}}
  \end{minipage}\hfill
  \begin{minipage}[m]{0.55\linewidth}
  \begin{minted}{cypher}
MATCH (a)-[(Down | Up)*]->(b)
RETURN a.name, b.name
  \end{minted}
  
  \pause
  \vspace{1cm}
  \tikzmark{xxx}{}
  \begin{minted}{cypher}
PATH PATTERN SameLvl = 
  ()-/ <:Down [~SameLvl | ()] :Down> /->()
MATCH (a)-/ ~SameLvl /->(b)
RETURN a.name, b.name
  \end{minted}
  
    {\tikz[overlay,remember picture]{\draw[draw=red, fill opacity=0.2, double, line width=0.25mm] ($ (xxx) + (-0.1,-0.2)$) rectangle ($  (xxx) + (8.6,-1.2)$);}
    \mycallout<2>[opacity=1]{$ (xxx) + (1.15,-0.3)$}{Named path pattern}{3.5}
    %\tikz[overlay,remember picture]{\draw[draw=red, fill opacity=0.2, line width=0.25mm] ($ (xxx) + (8.9,-0.8)$) rectangle ($  (xxx) + (10.8,-1.3)$);}
    \mycallout<2>[opacity=1]{$ (xxx) + (4.9,-1.0)$}{\small $\text{SameLvl} \to \overline{\text{Down}} \ \text{SameLvl} \ \text{Down} \mid \varepsilon$ }{5.5}
    }


  \end{minipage}


\end{frame}


\begin{frame}[fragile] \frametitle{Implementation Details}
  \begin{itemize}
  \item Linear algebra based multiple-source CFPQ is implemented as part of RedisGraph query engine
  \item Cypher parser is extended to support path patterns
  \item Path patterns are supported\footnote{Partially. Full support is a nontrivial challenge: formal description of the extension is required} in RedisGreaph query execution workflow
  \end{itemize}
  
\end{frame}




\begin{frame}[fragile] \frametitle{Evaluation Setup}

\begin{minipage}[t]{0.51\textwidth}
\vspace{-2cm}
\begin{itemize}
  \item Ubuntu 18.04, Intel Core i7-6700 CPU, 3.4GHz, DDR4 64Gb RAM
  \item Graphs stored in RedisGraph with our extensions
  \item Queries are generated with template for given size of start set
  \item The union of all start sets is a $V$ 
\end{itemize}

\end{minipage}
\pause
\begin{minipage}[t]{0.44\textwidth}
{
\rowcolors{2}{black!2}{black!10}
\begin{tabular}{|l|c|c|c|}
\hline
Graph                  & \#V                  & \#E                  & Q     \\
              
\hline
\hline
core                   & 1323                 & 4342                 & $g_1$ \\
pathways               & 6238                 & 18 598               & $g_1$ \\
gohierarchy            & 45 007               & 980 218              & $g_1$ \\
enzyme                 & 48 815               & 109 695              & $g_1$ \\
eclass\_514en          & 239 111              & 523 727              & $g_1$ \\
geospecies             & 450 609              & 2 311 461            & $geo$ \\
go                     & 272 770              & 534 311              & $g_1$ \\
\hline
\end{tabular}
}

\end{minipage}

\vspace{1cm}
\pause
\begin{minted}{cypher}
PATH PATTERN S = 
  ()-/ [<:SubClassOf [~S | ()] :SubClassOf] | [<:Type [~S | ()] :Type] /->()
MATCH (src)-/ ~S /->()
WHERE {id_from} <= src.id and src.id <= {id_to}
RETURN count(*)

\end{minted}


\end{frame}

\begin{frame}[fragile] \frametitle{Evaluation Results}  
  \begin{center}
  \begin{minipage}[t]{0.15\textwidth}
  \vspace{1.5cm}
  eclass\_514en\\  
  Query: g1
  \vspace{2.5cm}\\
  geospecies
  Query: geo
  \end{minipage}
  \begin{minipage}[t]{0.4\textwidth}
    \begin{center} 
    %\vspace{-0.5cm}
    Time\\
  \tikzmark{y1}{\includegraphics[width=.85\textwidth]{pictures/eclass_514en_time.pdf}}\\
  \tikzmark{y2}{\includegraphics[width=.85\textwidth]{pictures/geospecies_time.pdf}}
\end{center}
\end{minipage}
\begin{minipage}[t]{0.4\textwidth}
  \begin{center}
  %\vspace{-0.5cm}
    Memory\\
  \tikzmark{z1}{\includegraphics[width=.85\textwidth]{pictures/eclass_514en_mem.pdf}}\\
  \tikzmark{z2}{\includegraphics[width=.85\textwidth]{pictures/geospecies_mem.pdf}}
\end{center}
\end{minipage}
\end{center}
\end{frame}


\begin{frame}[fragile] \frametitle{Summary}
  \begin{itemize}
      \item Full-stack support for CFPQ in real-world graph query language (Cypher) on the top of real-world graph database (RedisGraph)
      \begin{itemize}
         \item No more ugly context-free grammars
         \item No more custom graph formats and storages
      \end{itemize}
      \item Reasonable performance of context-free path queries
      \begin{itemize}
         \item Multiple-source scenario
         \item Space-time ratio can be tuned
      \end{itemize}
      \item Context-free path queries can be used in applications with well-established tools
  \end{itemize}  
\end{frame}



\begin{frame}[fragile] \frametitle{Future Research}
  \begin{itemize}
    \item Mechanization of Cypher semantics in Coq
    \begin{itemize}
      \item Including path patterns
      \item Correctness of translation to linear algebra
    \end{itemize}
    \pause
    \item Integration of tensor-based CFPQ algorithm\footnote{Egor Orachev, Ilya Epelbaum, R. Azimov and S. Grigorev. “Context-Free Path Querying by Kronecker Product.” ADBIS (2020).} to RedisGraph
    \begin{itemize}
      \item Allows one to construct paths, not only reachability facts
      \item Should be modified to get multiple-source version
    \end{itemize}
    \pause
    \item Detailed evaluation
    \begin{itemize}
      \item More graphs and queries, including RPQs      
      \item Scalability of the solution
      \item Comparison with other graph query engines
    \end{itemize}
  \end{itemize}
\end{frame}

\begin{frame}
\frametitle{Contact Information}
\begin{minipage}[t]{0.8\textwidth}
\begin{itemize}
  \item Try it out (Docker image with all included): \url{https://hub.docker.com/r/simpletondl/redisgraph}
  \item Sources of RedisGraph extended with CFPQ: \url{https://github.com/YaccConstructor/RedisGraph}
  \item Sources of Cypher parser extended with path patterns: \url{https://github.com/YaccConstructor/libcypher-parser}

  \vspace{0.5cm}  
  \pause
  \item Semyon Grigorev: \href{mailto:s.v.grigoriev@spbu.ru}{s.v.grigoriev@spbu.ru}    
  \item Arseniy Terekhov: \href{mailto:simpletondl@yandex.ru}{simpletondl@yandex.ru}
  \item Vlada Pogozhelskaya: \href{mailto:pogozhelskaya@gmail.com}{pogozhelskaya@gmail.com}
  \item Vadim Abzalov: \href{mailto:vadim.i.abzalov@gmail.com}{vadim.i.abzalov@gmail.com}
  \item Timur Zinnatulin: \href{mailto:teemychteemych@gmail.com}{teemychteemych@gmail.com}
\end{itemize}
\end{minipage}~
\begin{minipage}[t]{0.19\textwidth}
\pause
\vspace{2.5cm}
\center{\huge{Thanks!}}
\end{minipage}
\end{frame}
\end{document}
