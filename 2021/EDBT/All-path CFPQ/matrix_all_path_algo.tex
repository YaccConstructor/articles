\section{Matrix-based CFPQ algorithm for all-path query semantics}
\label{sec:all-path-algo}
In this section, we propose the matrix-based algorithm for CFPQ w.r.t. the all-path query semantics (see Listing~\ref{lst:algo1}). This algorithm is a modification of Azimov's matrix-based algorithm for CFPQ and it constructs the set of matrices $T$ with AllPathIndexes as elements.

Let $G = (N, \Sigma, P, S)$ be the input context-free grammar, $D = (V, E, \Sigma)$ be the input graph.
The result of the algorithm is a set of matrices $T$ which stores information about all paths in the graph $D$ that form a word derivable from some nonterminal of the context-free grammar $G$. Note that in line 4 we add the special value $n$ to the $T^{A_i}_{k,l}.middles$ to specify that this path is a single-edge path or an empty path $\pi_{\varepsilon}$.

\begin{algorithm}
\small
\begin{algorithmic}[1]
\floatname{algorithm}{Listing}
\caption{CFPQ algorithm for all-path query semantics}
\label{lst:algo1}
\Function{AllPathCFPQ}{\par
\hskip\algorithmicindent $D = (V, E, \Sigma)$, \par
\hskip\algorithmicindent $G=(N,\Sigma,P,S)$} \Comment{Grammar in WCNF}\par
\State{$n \gets$ |V|}
\State{$T \gets \{T^{A} \mid A \in N, T^{A}$ is a matrix $n \times n$, $T^{A}_{i,j} \gets \bot$ \} }
\ForAll{$(i,x,j) \in E$, $A \mid A \to x \in P$}
%\Comment{Matrices initialization}
%\For{$A_k \mid A_k \to x \in P$}
{$T^{A}_{i,j} \gets (i,j,\{n\})$}
%\EndFor
\EndFor
\ForAll{$A \mid A \to \varepsilon \in P$}
{$T^{A}_{i,i} \gets (i,i,\{n\})$}
\EndFor

\While{any matrix in $T$ is changing}
%\Comment{Transitive closure calculation}
\ForAll{$A \to B C \in P$ where $T^{B}$ or $T^{C}$ are changed}
\State{ $T^{A} \gets T^{A} + (T^{B} \odot T^{C})$ } 
\EndFor
\EndWhile
\State \Return $T$
\EndFunction

\end{algorithmic}
\end{algorithm}

After constructing a set of matrices $T$ or so-called \textit{index}, we can construct a set of all paths $\pi$ between specified vertex pair $(i, j)$ and a non-terminal $A$ such that $A \xLongrightarrow[G]{*} l(\pi)$. The index $T$ already stores data about all paths derivable from each nonterminal. However, the
set of such paths can be infinite. From a practical perspective, it is necessary
to use lazy evaluation or limit the resulting set of paths in some other way.
For example, one can try to query some fixed number of paths or query paths
of fixed maximum length.

We propose the algorithm (see Listing~\ref{lst:algo2}) for extracting these paths. Our algorithm returns a set with the empty path $\pi_{\varepsilon}$ only if $i = j$ and $A \to \varepsilon \in P$. If the AllPathIndex for given $i,j,A$ is equal to $\bot$ then our algorithm returns $\emptyset$ since such paths do not exist. Note that in line 19 we use
the operation $\cdot$ which naturally generalizes the path concatenation operation
by constructing all possible concatenations of path pairs from the given two
sets. It is assumed that the sets are computed lazily, so as to ensure the termination in case of an infinite number of paths.

\begin{algorithm}
	\small
	\begin{algorithmic}[1]
		\floatname{algorithm}{Listing}
		\caption{All paths extraction algorithm}
		\label{lst:algo2}		
		\Function{extractAllPaths}{$i, j, A, T=\{T^{A_i}\}, G=(N,\Sigma,P,S)$}
		\State{$index \gets T^{A}_{i,j}$ }
		
		\If{$index = \bot$}
		\State \Return $\emptyset$
		\Comment{Such paths do not exist}
		\EndIf
		
		\State{$n \gets $ size of the square matrix $T^{A}$}
		\State{$resultPaths \gets \emptyset$}
		
		\ForAll{$middle \in index.middles$}		
		\If{$middle = n$}  \Comment{Add single-edge or empty paths}
		\ForAll{$x \mid A \to x \in P$}
		\If{$(i,x,j) \in E$}
		\State{$resultPaths \gets resultPaths \cup \{((i,x,j))\}$}
		\EndIf
		\EndFor
		\If{$(i = j) \wedge (A \to \varepsilon \in P)$}
		\State{$resultPaths \gets resultPaths \cup \{\pi_{\varepsilon}\}$}
		\EndIf
		\Else \Comment{Add to result the concatenated paths from $i$ to $middle$ and from $middle$ to $j$}
		\ForAll{$A \to B C \in P$}
		\State{$index_B \gets T^{B}_{i,middle}$ }
		\State{$index_C \gets T^{C}_{middle,j}$ }
		\If{$(index_B \neq \bot) \wedge (index_C \neq \bot)$}
		\State{$lPaths \gets$ \Call{extractAllPaths}{$i, middle, B, T, G$}}
		\State{$rPaths \gets$ \Call{extractAllPaths}{$middle, j, C, T, G$}}
		\State{$resultPaths \gets resultPaths \cup lPaths \cdot rPaths$}
		\EndIf
		\EndFor
		\EndIf
		\EndFor
		\State \Return $resultPaths$
		\EndFunction
	\end{algorithmic}
\end{algorithm}

\subsection{Correctness}

The following correctness theorem holds.

\begin{mytheorem}\label{thm:correct}
Let $G = (N, \Sigma, P, S)$ be the input context-free grammar, $D = (V, E, \Sigma)$ be the input graph, and $T$ be a set of matrices returned by the algorithm in Listing~\ref{lst:algo1}. Then for any $i, j$ and for any non-terminal $A \in N$, $index = T^A_{i,j}$ and $index = (i,j,middles) \neq \bot$ iff $(i,j) \in R_{G_A, D}$ and there is a path $\pi$ from vertex $i$ to $j$ such that $l(\pi) \in G_A = (N,\Sigma,P,A)$.
\end{mytheorem}
\begin{proof}[Proof sketch]
	At each iteration of the main cycle in the lines 6-8 of the algorithm, the new paths corresponding to nonterminals $A \in N$ are considered using the rules $A \to B C \in P$. These new paths are obtained by the concatenation of two smaller paths corresponding to the nonterminals $B$ and $C$. At the initialization step of the algorithm in lines 3-5, we consider all single-edge or empty paths corresponding to the derivation tree of height 1. Thus, it can be shown that at iteration $l$ of the main cycle we consider all paths $\pi$ such that there is a derivation tree of the height $h \leq l + 1$ for the string $l(\pi)$ and a context-free grammar $G_A$. Therefore, the theorem can be proved using the induction on the height of such derivation trees.
	
\end{proof}

Now, using the theorem~\ref{thm:correct} and induction on the length of the path, it can be easily shown that the following theorem holds.

\begin{mytheorem}\label{thm:correct_extraction}
Let $G = (N, \Sigma, P, S)$ be the input context-free grammar, $D = (V, E, \Sigma)$ be the input graph, and $T$ be a set of matrices returned by the algorithm in Listing~\ref{lst:algo1}. Then for any $i, j$ and for any non-terminal $A \in N$ such that $index = T^A_{i,j}$ and $index = (i,j,middles) \neq \bot$, the algorithm in Listing~\ref{lst:algo2} for these parameters will return a set of all paths $\pi$ from vertex $i$ to $j$ such that $l(\pi) \in G_A = (N,\Sigma,P,A)$.
\end{mytheorem}

We can, therefore, determine whether $(i,j) \in R_{G, D}$ by asking whether $T^S_{i,j} = \bot$. Also, we can extract all paths which forms a word from the context-free language $L(G)$ by using our algorithm in Listing~\ref{lst:algo2}. Thus, we show how the context-free path query evaluation w.r.t. the all-path query semantics can be solved in terms of matrix operations.

%\subsection{Complexity}

%Denote the number of elementary operations executed by the algorithm of multiplying two $n \times n$ matrices with PathIndexes as $MM(n)$. Also, denote the number of elementary operations, executed by the matrix element-wise + operation of two $n \times n$ matrices with PathIndexes as $MA(n)$. Since the line \textbf{7} of the algorithm in listing~\ref{lst:algo2} is executed no more than $|V|^2|N|$ times (for the same reasons as in the original paper~\cite{Azimov:2018:CPQ:3210259.3210264} of the matrix-based CFPQ algorithm), the following theorem holds.

%\begin{myproposition}\label{thm:time}
%	Let $D = (V,E)$ be a graph and let $G =(N,\Sigma,P)$ be a grammar. The algorithm in listing~\ref{lst:algo2} calculates the set of matrices $T$ in $O(|V|^2|N|^3(MM(|V|) + MA(|V|)))$.
%\end{myproposition}

%Also, denote the time complexity of the access to the PathIndex in the $n \times n$ matrix as $Access(n)$. Then the following theorem on the time complexity of the path extraction algorithm holds.

%\begin{myproposition}\label{thm:time_extraction}
%	Let $D = (V,E)$ be a graph, let $G =(N,\Sigma,P)$ be a grammar and $T$ be a set of matrices returned by the algorithm in listing~\ref{lst:algo2}. Then for any $i, j$ and for any non-terminal $A \in N$ such that $index = T^A_{i,j}$ and $index = (i,j,k,h,l) \neq \bot$, the algorithm in listing~\ref{lst:algo3} for these parameters calculates a path $i \pi j$ in $O(l \times N \times Access(|V|))$.
%\end{myproposition}

\subsection{An Example}
In this section, we provide a step-by-step demonstration of the proposed algorithms. %For this, we consider the example with the worst-case time complexity.

We run the query on a graph $D_1$, presented in Figure~\ref{fig:example_input_graph}. We provide a step-by-step demonstration of the work of algorithm in Listing~\ref{lst:algo1} with the given graph $D$ and grammar $G_1^{\text{wcnf}}$ from section~\ref{sec:preliminaries}. After the matrix initialization in lines \textbf{3-5} of this algorithm, we have a set of matrices $T^{(1)}$, presented in Figure~\ref{ExampleQueryInitMatrix}.

{\footnotesize
	\begin{figure}[h]
		\[
		T^{(1),A} = \begin{pmatrix}
			\bot & (0,1,\{4\})       & \bot & \bot       \\
			\bot & \bot & (1,2,\{4\})       & \bot \\
			(2,0,\{4\})       & \bot & \bot & \bot \\
			\bot       & \bot & \bot & \bot \\
		\end{pmatrix}
		\]
		\[
		T^{(1),B} = \begin{pmatrix}
			\bot & \bot       & \bot & (0,3,\{4\})       \\
			\bot & \bot & \bot       & \bot \\
			\bot       & \bot & \bot & \bot \\
			(3,0,\{4\})      & \bot & \bot & \bot \\
		\end{pmatrix}
		\]
		\caption{The initial matrices for the example query. The PathIndexes $T^{(1),S_1}_{i,j}$ and $T^{(1),S}_{i,j}$ are equal to $\bot$ for every $i,j$}
		\label{ExampleQueryInitMatrix}
	\end{figure}
}

After the initialization, the only matrices which will be updated are $T^{S_1}$ and $T^{S}$. These matrices obtained after the first loop iteration is shown in Figure~\ref{ExampleQueryFirstIteration}.

{\footnotesize
	\begin{figure}[h]
		\[
		T^{(2),S} = \begin{pmatrix}
			\bot & \bot       & \bot & \bot       \\
			\bot & \bot & \bot       & \bot \\
			\bot       & \bot & \bot & (2,3,\{0\}) \\
			\bot       & \bot & \bot & \bot \\
		\end{pmatrix}
		\]
		\caption{The first iteration of computing the transitive closure for the example query. The PathIndexes $T^{(1),S_1}_{i,j}$ are equal to $\bot$ for every $i,j$}
		\label{ExampleQueryFirstIteration}
	\end{figure}
}

When the algorithm at some iteration finds new paths for some non-terminal in the graph $D_1$, then it adds corresponding AllPathIndexes to the matrix for this non-terminal. For example, after the first loop iteration, AllPathIndex $(2,3,\{0\})$ is added to the matrix $T^{S}$. This AllPathIndex is added to the element with a row index $i = 2$ and a column index $j = 3$. This means, that there is a path $\pi$ from the vertex 2 to the vertex 3, such that $S \xLongrightarrow[G_1^{\text{wcnf}}]{*} l(\pi)$ and this path obtained by concatenation of two smaller paths via vertex 0.

The calculation of the index $T$ is completed after $k$ iterations, when a fixpoint is reached: $T^{(k)} = T^{(k-1)}$. For the example query, $k = 14$ since $T_{14} = T_{13}$. The resulted matrix for non-terminal $S$ is presented in Figure~\ref{ExampleQueryFinalMatrices}.

{\footnotesize
	\begin{figure}[h]
		\[
		T^{(14),S} = \begin{pmatrix}
			(0,0,\{1\}) & \bot       & \bot & (0,3,\{1\})       \\
			(1,0,\{2\}) & \bot & \bot       & (1,3,\{2\}) \\
			(2,0,\{0\})       & \bot & \bot & (2,3,\{0\}) \\
			\bot       & \bot & \bot & \bot \\
		\end{pmatrix}
		\]
		\caption{The final matrix for non-terminal $S$ after computing the index}
		\label{ExampleQueryFinalMatrices}
	\end{figure}
}

Now, after constructing the index, we can construct the context-free relation $$R_{G_1^{\text{wcnf}}, D_1}=\{(0,0),(0,3),(1,0),(1,3),(2,0),(2,3)\}.$$

In the relation $R_{G_1^{\text{wcnf}}, D_1}$, we have all vertex pairs corresponding to paths, whose labeling is in the language $L(G_1^{\text{wcnf}}) = \{a^n b^n \mid n \geq 1\}$. Using the algorithm in Listing~\ref{lst:algo2} we can restore paths for each vertex pair from the context-free relation. For example, given $i=j=0$, non-terminal $S$, set of resulted matrices $T$, and context-free grammar $G_1^{\text{wcnf}}$, the algorithm in Listing~\ref{lst:algo2} returns an infinite set of all paths from vertex 0 to vertex 0 whose labeling form words from the following set $\{a^6 b^6, a^{12} b^{12}, a^{18} b^{18}, \ldots \}$. Following the path corresponding to the word $a^{6m} b^{6m}$, we will go through the cycle with $a$ labels $2m$ times and through the cycle with $b$ labels $3m$ times for all $m \geq 1$.