\section{Preliminaries}
\label{sec:preliminaries}

In this section, we introduce common definitions in graph theory and formal language theory which are used in this paper.
Also, we provide a brief description of CFPQ problems and AllPathIndex structure which is used as a base of our solution for all-path query semantics.

\subsection{Basic Definitions of Graph Theory}

In this paper, we use a labeled directed graph as a data model and define it as follows.
\begin{definition} \emph{Labeled directed graph} is a tuple $D = (V, E, \Sigma)$, where
\begin{itemize}
    \item $V$ is a finite set of vertices. For simplicity, we assume that the vertices are natural numbers ranging from $0$ to $|V|-1$,
    \item $E \subseteq V \times \Sigma \times V$ is a set of labeled edges,
    \item $\Sigma$ is a set of edge labels.
\end{itemize} 
\end{definition}
%~\cite{Angles2018ThePG}

An example of the labeled directed graph $D_1$ is presented in Figure~\ref{fig:example_input_graph}. Here the set of labels $\Sigma = \{a, b\}$.

\begin{figure}[h]
	\[
	\includegraphics[width=5cm]{pictures/example_graph.pdf}
	\]
	\caption{The input graph $D_1$}
	\label{fig:example_input_graph}
\end{figure}

\begin{definition}
The \emph{path} $\pi$ in the graph $D=(V, E, \Sigma)$ is a finite sequence of labeled edges $(e_1, ..., e_{n})$, where $\forall i, 1 \leq j \leq n: e_i=(v_{i-1},l_i, v_i) \in E$.
\end{definition}

\begin{definition}
	The \emph{word} $l(\pi) \in \Sigma^*$ in the graph $D=(V, E, \Sigma)$ is the unique word $l_1 ... l_n$, obtained by concatenating the labels of the edges along the	path $\pi = (e_1 = (v_0, l_1, v_1), \ldots, e_n = (v_{n-1}, l_n, v_n))$ in the graph $D$.
\end{definition}


\subsection{Basic Definitions of Formal Languages}
We use context-free grammars as path constraints, thus in this subsection, we define context-free languages and grammars.

\begin{definition}A \emph{context-free grammar} $G$ is a tuple $(N, \Sigma, P, S)$, where
\begin{itemize}
    \item $N$ is a finite set of nonterminals
    \item $\Sigma$ is a finite set of terminals, $N \cap \Sigma = \varnothing$
    \item $P$ is a finite set of productions of the form $A \to \alpha$, where $A \in N,\ \alpha \in (N \cup \Sigma)^*$
    \item $S$ is the start nonterminal
\end{itemize} 
\end{definition}

We use the conventional notation $A \xLongrightarrow[G]{*} w$ to denote, that a
string $w \in \Sigma$ can be derived from a non-terminal $A$ by some sequence of production rule applications from $P$ in grammar $G$.

\begin{definition} A \emph{context-free language} is a language generated by a context-free grammar $G=(N, \Sigma, P, S)$:
\begin{align*}
    L(G) &= \{w \in \Sigma^* \mid S \xLongrightarrow[G]{*} w \}.
\end{align*}
\end{definition}

\begin{definition} A context-free grammar $G = (N, \Sigma, P, S)$ is in \emph{weak Chomsky normal form} (WCNF) if every production in $P$ has one of the following forms:
    \begin{itemize}
        \item $A \rightarrow BC$, where $A, B, C \in N$
        \item  $A \rightarrow a$, where $A \in N, a \in \Sigma$
        \item $A \rightarrow \varepsilon$, where $A \in N$
    \end{itemize}
\end{definition}

Note that weak Chomsky normal form differs from Chomsky normal form in the following:
\begin{itemize}
    \item $\varepsilon$ can be derived from any non-terminal;
    \item $S$ can occur on the right-hand side of productions.
\end{itemize}

The matrix-based CFPQ algorithms process grammars only in weak Chomsky normal form, but every context-free grammar can be transformed into the equivalent grammar in this form.

Consider the context-free grammar $G_1=(\{S\},\{a, b\}, P, S)$, where $P$ contains two rules:
$S \rightarrow a \ S \ B; \ \ \ 
S \rightarrow a \ b
$.

This grammar generates the context-free language: $$L(G_1) = \{a^nb^n, n \in \mathbb{N}\}.$$
The following production rules of the grammar $G_1^{\text{wcnf}}$ is a result of the transformation of $G_1$ to weak Chomsky normal form:
\begin{align*}
S& \to A \ B   & S_1& \to S \ B   & B& \to b  \\
S& \to A \ S_1 & A& \to a &&  \\
\end{align*}


\subsection{Context-Free Path Querying}

\begin{definition}
Let $D = (V, E, \Sigma)$ be a labeled graph, $G = (N, \Sigma, P, S)$ be a context free grammar. Then a \emph{context-free relation} with grammar $G$ on the labeled graph $D$ is the relation $R_{G, D} \subseteq V \times V$:
\begin{equation*} \label{eq1}
\begin{split}
R_{G, D} = \{ &(v_0, v_n) \in V \times V  \mid \\ &\ \exists \pi = (e_1 = (v_0, l_1, v_1), \ldots, e_n = (v_{n-1}, l_n, v_n)) \in \pi(D): \\
      &\ l(\pi) \in L(G) \}.
\end{split}
\end{equation*}
\end{definition}

For example, the vertex pair $(0,0) \in R_{G_1, D_1}$, since there is a path in the labeled graph $D_1$ presented in Figure~\ref{fig:example_input_graph} from the vertex $0$ to the vertex $0$, whose labeling forms a word $$w = aaaaaabbbbbb = a^6b^6 \in L(G_1).$$

Finally, we can define context-free path querying problems.
\begin{definition}
    \emph{Context-free path querying problem with relational query semantics} is the problem of finding context-free relation $R_{G, D}$ for a given directed labeled graph $D$ and a context-free grammar $G$.
\end{definition}

In other words, the result of context-free path query evaluation is a set of vertex pairs such that there is a path between them that forms a word from the language generated by the given context-free grammar.

Using this definition, we can also define context-free path querying problems with single-path and all-path query semantics.

\begin{definition}
	\emph{Context-free path querying problem with single-path query semantics} for a given directed labeled graph $D$ and a context-free grammar $G$ is the problem of finding context-free relation $R_{G, D}$ and finding for each vertex pair $(v_0, v_n) \in R_{G, D}$ the one example of path $\pi$ between these vertices such that $l(\pi) \in L(G)$.
\end{definition}

\begin{definition}
	\emph{Context-free path querying problem with all-path query semantics} for a given directed labeled graph $D$ and a context-free grammar $G$ is the problem of finding context-free relation $R_{G, D}$ and finding for each vertex pair $(v_0, v_n) \in R_{G, D}$ all paths $\pi$ between these vertices such that $l(\pi) \in L(G)$.
\end{definition}


\subsection{Matrix-Based Algorithm}
Our algorithm is based on Azimov's CFPQ algorithm~\cite{Azimov:2018:CPQ:3210259.3210264} which is based on matrix operations.
This algorithm reduces CFPQ to operations over Boolean matrices and as a result, allows one to use high-performance linear algebra libraries and utilize modern parallel hardware for CFPQ.

Note, that the algorithm computes not only the context-free relation $R_{G,D}$ but also a set of context-free relations $R_{G_A,D} \subseteq V \times V$ for every $A \in N$ where $G_A = (N, \Sigma, P, A)$.
Thus it provides information about paths that form words derivable from any nonterminal $A \in N$.

We use an idea similar to one that was used for the CFPQ with single-path query semantics in~\cite{10.1145/3398682.3399163}. We store additional information in matrices to be able to restore all paths which form words derivable from any nonterminal in the given grammar.

In order to do this, we introduce the 
$$\textit{AllPathIndex} = (\textit{left},\textit{right},\textit{middles})$$ 
--- the elements of matrices that describe the found paths as concatenations of two smaller paths and help to restore each path after the index creation. Here \textit{left} and \textit{right} stand for the indexes of starting and ending vertices in the founded path, \textit{middles} --- the set of indexes of intermediate vertices used in the concatenation of two smaller paths. When we do not find the path for some vertex pair $i,j$, we use the $\textit{AllPathIndex} = \bot = (0,0,\emptyset)$.

Additionally, we will use the notation of \textit{proper matrix} which means that for every element of the matrix with indexes $i,j$ it either $\textit{AllPathIndex} = (i,j,\_)$ or $\bot$.

For proper matrices we use a binary operation $\otimes$ defined for AllPathIndexes \mbox{$AP_1, AP_2$} which are not equal to $\bot$ and with $AP_1.\textit{right} = AP_2.\textit{left}$ as
\begin{align*}
	AP_1 \otimes AP_2 = (&AP_1.left, AP_2.right, \{AP_1.right\}).
\end{align*}

And if at least one operand is equal to $\bot$ then $AP_1 \otimes AP_2 = \bot$.

For proper matrices we also use a binary operation $\oplus$ defined for AllPathIndexes \mbox{$AP_1, AP_2$} which are not equal to $\bot$ with $AP_1.\textit{left} = AP_2.\textit{left}$ and $AP_1.\textit{right} = AP_2.\textit{right}$ as
\begin{align*}
	AP_1 \otimes AP_2 = (&AP_1.left, AP_1.right, AP_1.middles \cup AP_2.middles).
\end{align*}
If only one operand is equal to $\bot$ then $AP_1 \oplus AP_2$ equal to another operand. If both operands are equal to $\bot$ then $AP_1 \oplus AP_2 = \bot$.

Using $\otimes$ as multiplication of AllPathIndexes, and $\oplus$ as an addition, we can define a \emph{matrix multiplication}, \mbox{$a \odot b = c$}, where $a$ and $b$ are matrices of a suitable size, that have AllPathIndexes as elements, as $c_{i,j} = \bigoplus^{n}_{k=1}{a_{i,k} \otimes b_{k,j}}.$

Also, we use the element-wise $+$ operation on matrices $a$ and $b$ with the same size: \mbox{$a + b = c$}, where $c_{i,j} = a_{i,j} \oplus b_{i,j}.$
