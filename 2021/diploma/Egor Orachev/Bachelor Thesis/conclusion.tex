\section{Заключение}

В рамках выполнения данной работы были получены следующие результаты.

\begin{itemize}
    \item Спроектирована библиотека примитивов разреженной линейной булевой алгебры cuBool для вычислений на GPGPU.
    Данная библиотека экспортирует С-совместимый интерфейс, имеет поддержку различных вычислительных модулей, а также предоставляет Python-пакет для работы конечного пользователя с примитивами библиотеки в высокоуровневой среде вычислений с управляемыми ресурсами.

    \item Реализована библиотека cuBool в соответствии с разработанной архитектурой. 
    Ядро библиотеки написано на языке С++, а математические операции, выполняющиеся на GPU, реализованы на языке Cuda C/C++. Библиотека предоставляет модуль CPU вычислений для компьютеров без Cuda-устройства. 
    Также создан Python-пакет pycubool, который доступен для скачивания через пакетный менеджер PyPI.
    
    \item С использованием pycubool реализован алгоритм поиска путей с КС ограничениями через тензорное произведение. 
    Данный алгоритм использует операции матричного умножения, сложения и произведение Кронекера в булевом полукольце, а также различные операции для манипуляций над значениями матриц. 
   
    % На вход алгоритм получается представление графа и КС грамматики в виде набора матриц, а на выходе --- возвращает матрицу смежности графа достижимости, а также индекс, который позволяет восстанавливать все пути в графе, в соответствии с входной грамматикой.
    
    \item Выполнено экспериментальное исследование реализованной библиотеки и алгоритма.
    Матричное умножение показывает ускорение до 5 раз по сравнению с существующими аналогами, 
    матричное сложение сравнимо по времени с существующими аналогами, 
    однако операции потребляют до 3 раз меньше видеопамяти. 
    Реализация алгоритма поиска путей с КС ограничениями на основе cuBool показывает ускорение до 6 раз по сравнению с CPU версией, что делает ее более применимой для анализа реальных данных.
\end{itemize}

Результаты, полученные в данном исследовании, были представлены на конференции GrAPL 2021\footnote{GrAPL 2021: Workshop on Graphs, Architectures, Programming, and Learning. Дата обращения: 1.04.2021. Сайт конференции: \url{https://hpc.pnl.gov/grapl/}.}. Соответствующая статья опубликована в сборнике материалов конференции.

Библиотека cuBool и Python-пакет pycubool для работы с данной библиотекой доступны для скачивания через следующие онлайн ресурсы: \url{https://github.com/JetBrains-Research/cuBool} и \url{https://pypi.org/project/pycubool/}.

\subsection*{Благодарности}

Хотелось бы выразить благодарность компании JetBrains и лаборатории языковых инструментов на базе кафедры системного программирования за предоставленную рабочую станцию для осуществления работы и за предоставление доступа к серверам для постановки экспериментов.

Отдельно хотелось бы выразить благодарность И.В. Эпельбауму за помощь в постановке экспериментов для исследования алгоритмов КС достижимости.

Также хотелось бы выразить благодарность Д.В. Кознову за помощь в работе над текстом в рамках данного документа.

В заключение, хотелось бы выразить особую благодарность научному руководителю, С.В. Григорьеву, за чуткое, своевременное, грамотное и полное трудолюбия научное руководство, а также за помощь в исследовании и в работе над данным документом. 

