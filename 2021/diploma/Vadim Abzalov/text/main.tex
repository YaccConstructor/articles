% Тут используется класс, установленный на сервере Papeeria. На случай, если
% текст понадобится редактировать где-то в другом месте, рядом лежит файл matmex-diploma-custom.cls
% который в момент своего создания был идентичен классу, установленному на сервере.
% Для того, чтобы им воспользоваться, замените matmex-diploma на matmex-diploma-custom
% Если вы работаете исключительно в Papeeria то мы настоятельно рекомендуем пользоваться
% классом matmex-diploma, поскольку он будет автоматически обновляться по мере внесения корректив
%

% По умолчанию используется шрифт 14 размера. Если нужен 12-й шрифт, уберите опцию [14pt]
%\documentclass[14pt]{matmex-diploma}
\documentclass[14pt]{matmex-diploma-custom}
\usepackage{fontspec}
\usepackage{polyglossia}
\usepackage{amsmath}
\usepackage{amsfonts}
\usepackage{amssymb}

% пакеты из презентации
\usepackage{algpseudocode}
\usepackage{algorithm}
\usepackage{algorithmicx}
\usepackage{pdfpages}


\usepackage{geometry}
\usepackage{amsfonts,latexsym}
\usepackage{amsthm}
\usepackage{amssymb}
\usepackage[utf8]{inputenc} % Кодировка
\usepackage{mathtools}
\usepackage{hyperref}
\usepackage{tikz}
\usepackage{dsfont}
\usepackage{multicol}
\usetikzlibrary{fit,calc,automata,positioning}

\usepackage{fontawesome}

\theoremstyle{definition}
\newtheorem{definition}{Определение}[section]
\newtheorem{example}{Пример}[section]
\newtheorem{theorem}{Теорема}[section]
\newtheorem{proposition}[theorem]{Proposition}
\newtheorem{lemma}[theorem]{Лемма}
\newtheorem{corollary}[theorem]{Corollary}
\newtheorem{conjecture}[theorem]{Conjecture}
\newtheorem{note}[theorem]{Утверждение}

\begin{document}
% Год, город, название университета и факультета предопределены,
% но можно и поменять.
% Если англоязычная титульная страница не нужна, то ее можно просто удалить.
\filltitle{ru}{
    chair              = {Программная инженерия},
    title              = {Модернизация набора данных \textsc{CFPQ\_Data}},
    % Здесь указывается тип работы. Возможные значения:
    %   coursework - Курсовая работа
    %   diploma - Диплом специалиста
    %   master - Диплом магистра
    %   bachelor - Диплом бакалавра
    type               = {coursework},
    position           = {студента},
    group              = 371,
    author             = {Абзалов Вадим Игоревич},
    supervisorPosition = {к.\,ф.-м.\,н., доцент кафедры информатики СПбГУ},
    supervisor         = {С.\,В. Григорьев},
%   university         = {Санкт-Петербургский Государственный Университет},
%   faculty            = {Математико-механический факультет},
%   city               = {Санкт-Петербург},
%   year               = {2019}
}

\maketitle
\tableofcontents

\section{Introduction}

Scalable high-performance graph analysis is an actual challenge.
There is a big number of ways to attack this challenge~\cite{Coimbra2021} and the first promising idea is to utilize general-purpose graphic processing units (GPGPU).
Such existing solutions, as CuSha~\cite{10.1145/2600212.2600227} and Gunrock~\cite{7967137} show that utilization of GPUs can improve the performance of graph analysis, moreover it is shown that solutions may be scaled to multi-GPU systems.
But low flexibility and high complexity of API are problems of these solutions.

The second promising thing which provides a user-friendly API for high-performance graph analysis algorithms creation is a GraphBLAS API~\cite{7761646} which provides linear algebra based building blocks to create graph analysis algorithms.
The idea of GraphBLAS is based on a well-known fact that linear algebra operations can be efficiently implemented on parallel hardware.
Along with that, a graph can be natively represented using matrices: adjacency matrix, incidence matrix, etc.
While reference CPU-based implementation of GraphBLAS, SuiteSparse:GraphBLAS~\cite{10.1145/3322125}, demonstrates good performance in real-world tasks, GPU-based implementation is challenging.

One of the challenges in this way is that real data are often sparse, thus underlying matrices and vectors are also sparse, and, as a result, classical dense data structures and respective algorithms are inefficient. 
So, it is necessary to use advanced data structures and procedures to implement sparse linear algebra, but the efficient implementation of them on GPU is hard due to the irregularity of workload and data access patterns.
Though such well-known libraries as cuSPARSE show that sparse linear algebra operations can be efficiently implemented for GPGPU, it is not so trivial to implement GraphBLAS on GPGPU. 
First of all, it requires \textit{generic} sparse linear algebra, thus it is impossible just to reuse existing libraries which are almost all specified for operations over floats.
The second problem is specific optimizations, such as masking fusion, which can not be natively implemented on top of existing kernels.
Nevertheless, there is a number of implementations of GraphBLAS on GPGPU, such as GraphBLAST~\cite{yang2019graphblast}, GBTL~\cite{7529957}, which show that GPGPUs utilization can improve the performance of GraphBLAS-based graph analysis solutions.
But these solutions are not portable because they are based on Nvidia Cuda stack.
Moreover, the scalability problem is not solved: all these solutions support only single-GPU, not multi-GPU computations.

To provide portable GPU implementation of GraphBLAS API we developed a \textit{SPLA} library\footnote{Source code available at: \url{https://github.com/JetBrains-Research/spla}}.
This library utilizes OpenCL for GPGPU computing to be portable across devices of different vendors.
Moreover, it is initially designed to utilize multiple GPGPUs to be scalable.
To sum up, the contribution of this work is the following.
\begin{itemize}
    \item Design of portable GPU GraphBLAS implementation proposed. The design involves the utilization of multiple GPUS. Additionally, the proposed design is aimed to simplify library tuning and wrappers for different high-level platforms and languages creation. 
    \item Subset of GraphBLAS API, including such operations as masking, matrix-matrix multiplication, matrix-matrix e-wise addition, is implemented. The current implementation is limited by COO and CSR matrix representation format and uses basic algorithms for some operations, but work in progress and more data formats will be supported and advanced algorithms will be implemented in the future.
    \item Preliminary evaluation on such algorithms as breadth-first search (BFS) and triangles counting (TC), and real-world graphs shows portability across different vendors and promising performance: for some problems Spla is comparable with GraphBLAST. Surprisingly, for some problems, the proposed solution on embedded Intel graphic card shows better performance than SuiteSparse:GraphBLAS on the respective CPU. At the same time, the evaluation shows that further optimization is required.
\end{itemize} 
% Обязательный слайд: четкая формулировка цели данной работы и постановка задачи
% Описание выносимых на защиту результатов, процесса или особенностей их достижения и т.д.
\begin{frame}
	\transwipe[direction=90]
	\frametitle{\faThumbTack\ Цель и задачи}
	\textbf{Цель}: модернизация существующего набора данных \textsc{CFPQ\_Data} для создания унифицированного средства подготовки проведения экспериментального исследования CFPQ алгоритмов
	  
	~\
	  
	\textbf{Задачи}
	
	\newline
	
	\begin{itemize}
		\item[\bullet] Модернизация архитектуры набора данных
		\item[\bullet] Добавление новых возможностей работы с данными
		      \begin{itemize}
		      	\item[\bullet] Загрузка конкретных графов из набора данных
		      	\item[\bullet] Преобразование графов в другие форматы
		      	\item[\bullet] Получение информации о графе
		      	\item[\bullet] Трансформация графов
		      \end{itemize}
% 		\item[\bullet] Разработка и реализация протокола версионирования данных
		\item[\bullet] Публикация Python пакета для работы с набором данных и документации к нему
	\end{itemize}
	
\end{frame}


\section{Обзор}

Для того чтобы упростить ход рассуждений, касающихся контекстно-свободных грамматик, в данной работе используется понятие \textit{рекурсивного автомата} --- абстракции, позволяющей задавать произвольную КС-грамматику.
Ее описание приводится в первом параграфе обзора. 
Второй параграф посвящен вопросу о разрешимости задачи синтаксического анализа контекстно-свободного представления данных и его связи с фундаментальными проблемами теории формальных языков.

Предлагаемый в данной работе алгоритм основан на алгоритме синтаксического анализа регулярных множеств, который, в свою очередь, явлется модификацией алгоритма обобщенного синтаксического анализа Generalized LL. 
Об этих алгоритмах, а также о проекте, в рамках которого проведена разработка предложенного решения, также будет рассказано в обзоре.

\subsection{Рекурсивные автоматы и КС-грамматики}
Введем понятие рекурсивного автомата, которое потребуется для дальнейшего изложения

\begin{defn}
	Рекурсивный автомат R --- это пятерка $(\Sigma, Q, \delta, q_0, q_f)$, где $\Sigma$ --- конечное множество терминальных символов, $Q$ --- конечное множество состояний автомата, $\delta : Q \times (\Sigma \cup Q) \rightarrow 2^Q$ --- функция переходов, $q_0 \in Q$ --- начальное состояние, $q_f$ --- конечное состояние. 
\end{defn}

Можно заметить, что данное определение практически идентично определению стандартного конечного автомата. 
Единственное отличие состоит в том, что метками на ребрах рекурсивного автомата могут как терминальные символы (терминальные переходы), так и состояния (нетерминальные переходы).
Класс рекурсивных автоматов обладает такой же выразительностью, как и контекстно-свободные грамматики, т.е. позволяет описать любой контекстно-свободный язык. 
Более того, грамматика тривиальным образом может быть преобразована в рекурсивный автомат (обратное тоже верно) \cite{tellier2006ra}. 
Пример рекурсивного автомата, построенного по грамматике, можно увидеть на рисунке \ref{fig:ra_ex}.

\begin{figure}[h]
	\centering
	\begin{subfigure}[b]{0.45\textwidth}
		\centering
		$$
		\begin{array}{crcl}
		&S' & ::= & S \\
		&S  & ::= & \texttt{[ } S \texttt{ ]}\\
		&S  & ::= & \mbox{\texttt{a}}
		\end{array}
		$$
		\caption{Грамматика $G_1$}
	\end{subfigure}
	~
	\begin{subfigure}[b]{0.45\textwidth}
		\centering
		\includegraphics[width=4cm]{pictures/ra_example.pdf}
		\caption{Рекурсивный автомат для $G_1$}
	\end{subfigure}
	\caption{Преобразование между грамматикой и рекурсивным автоматом}
	\label{fig:ra_ex}
\end{figure}


\subsection{Разрешимость задачи синтаксического анализа контекстно-свободного представления}
Как было сказано ранее, задачу поиска шаблона, при условии, что и шаблон, и данные, в которых осуществляется поиск, представлены контекстно-свободными грамматиками, мы назовем синтаксическим анализом контекстно-свободного представления. 

Для доказательства предложений, сформулированных далее, будет использоваться следующая теорема \cite{Nederhof}.

\begin{theorem}[Nederhof, Satta]
	Пусть $G_1$ --- произвольная контекстно-свободная грамматика, $G_2$ --- грамматика, которая не содержит непосредственной или скрытой рекурсий. Тогда проблема проверки пустоты пересечения языков, порождаемых данными грамматиками, относится к классу PSPACE-complete.
\end{theorem}

Рассмотрим случай, когда грамматика данных задает ровно одну строку. Пусть $G_t$ --- произвольная КС-грамматика, задающая шаблоны для поиска, а $G_d$ --- КС-грамматика, которая не содержит непосредственной или скрытой рекурсий. $L(G_t)$ и $L(G_d)$ --- языки, порождаемые грамматиками, при этом $L(G_d) = \{\omega\}$, где $\omega$ --- исходные данные, к которым был применен алгоритм сжатия. 
Необходимо определить, существуют ли такие строки $\omega'$, что $\omega' \in L(G_t)$ и $\omega'$ --- подстрока $\omega$.

%т.е. $\omega' \in L_{sub}(\omega)$, где $L_{sub}(s)$ --- обозначение для языка, состоящего из всех подстрок заданной строки $s$. Отметим, что $L_{sub}$ относится к классу регулярных языков, так как множество всех подстрок конечно. 

\begin{prop}
	При выполнении описанных условий задача синтаксического анализа КС-представления разрешима.
\end{prop}

\begin{proof}
Пользуясь эквивалентностью представлений, можно записать грамматику $G_d$ в виде рекурсивного автомата $R_d$. Рассмотрим рекурсивный автомат $R_{i,\,j}$, полученный из $R_d$ путем замены стартового состояния на $i \in Q(R_d)$ и назначения терминирующего (финального, из которого не может быть совершено переходов) состояния $j \in Q(R_d)$. Такой автомат описывает грамматику, которая является представленим некоторой подстроки $\omega$. 
%В таком случае, рекурсивный автомат $R_{i,\,j}$, полученный из $R_d$ путем замены стартового и конечного состояний на $i, j \in Q(R_d)$ соответственно, описывает грамматику, которая является представленим некоторой подстроки $\omega$. 
Рассмотрев все возможные пары $i$ и $j$, получаем конечное множество грамматик, для каждой из которых необходимо проверить, содержится ли строка, порождаемая ей, в языке $L(G_t)$. 
Согласно теореме 1, такая проверка является разрешимой задачей и принадлежит к классу PSPACE-complete.
\end{proof}

Отдельно отметим, что для описанных процедур используется лишь исходный автомат, эквивалентный грамматике $G_d$. 
Условия задачи поиска шаблонов непосредственно в контекстно-свободном представлении, таким образом, выполняются. 
Верна также разрешимость более общей задачи.

\begin{prop}
	Пусть грамматика $G_d$ задает конечное множество строк $L(G_d) = \{\omega_1, \, \dots \, , \omega_n \}$. Необходимо определить, существуют ли строки $\omega'$, для которых верно: $\omega' \in L(G_t)$ и $\omega'$ --- подстрока одной из строк $\omega_i \in L(G_d)$. Данная задача разрешима и принадлежит классу PSPACE-complete.
\end{prop}

\begin{proof}
	Как и в предыдущем доказательстве, используем запись грамматики в виде рекурсивного автомата $R_d$ и рассмотрим автоматы $R_{i, j}$. В данном случае каждый из этих автоматов представляет собой грамматику, которая порождает некоторое конечное множество подстрок исходных строк из $L(G_d)$. Проверка пустоты пересечения такой грамматики с $G_t$ также соответствует условиям теоремы 1.
\end{proof}

В случае, когда грамматика $G_d$ представляет собой бесконечный регулярный язык (т.е. содержит левую и/или правую рекурсию), разрешимость задачи поиска шаблонов установить не удается. Подход, использованный ранее в доказательстве предложений, не может быть применен, так как части рекурсивного автомата, представляющего грамматику $G_d$, также могут содержать рекурсивные переходы, что выходит за рамки условия теоремы 1. Проверка разрешимости и определение класса сложности задачи проверки пустоты пересечения произвольной и регулярной КС-грамматик в настоящее время остаются открытыми проблемами \cite{Nederhof}.

%Пусть $G$ --- произвольная КС-грамматика, $M$ --- конечный автомат. Тогда задача проверки
%\begin{itemize}
%	\item включения языков ($L(M) \subseteq L(G)$) --- неразрешима
%	\item пустоты пересечения ($L(M) \cap L(G) = \emptyset$) --- разрешима (т.к. в пересечении не более чем КС-язык) за полиномиальное время \cite{Hunt}
%	\item регулярности языка $L(G)$ --- неразрешима \cite{Greibach1968}
%\end{itemize} 
%
%Если использовать представление регулярного языка $L(M)$ в виде КС-грамматики $G_r$, то задача проверки пустоты пересечения ($L(G_r) \, \cap \, L(G) = \emptyset$) становится немного интереснее: если $G_r$ 
%\begin{itemize}
%	\item нерекурсивная --- задача из PSPACE \cite{Nederhof} (точнее результата нет (я не нашел, по крайней мере))
%	\item лево- или праволинейная --- ничего не известно (см. последний абзац заключения из \cite{Nederhof})
%	\item принадлежит еще более широкому классу --- тем более ничего не известно
%\end{itemize}

%Еще немного про вложенную рекурсию и регулярность языка. Грамматика без вложенной рекурсии (NSE) порождает регулярный язык \cite{Chomsky} (обратное тоже верно, для регулярного языка можно построить NSE грамматику, т.к. праволинейная, например, --- частный случай NSE). Существует алгоритм, который позволяет проверять грамматику на наличие вложенной рекурсии за полином \cite{Anselmo}. Однако, грамматика с вложенной рекурсией тоже может порождать регулярный язык \cite{Andrei2004}, поэтому задача о проверке регулярности языка, порождаемого КС-грамматикой, остается неразрешимой. 

\subsection{GLL-алгоритм и его модификации}

Классические алгоритмы нисходящего и восходящего синтаксического анализа предполагают использование грамматики, которая является в достаточной мере однозначной. 
В противном случае, управляющие таблицы анализаторов содержат конфликты, из-за чего нельзя гарантировать корректное поведение на любых входных данных. 
Для работы с сильно неоднозначными грамматикам используются алгоритмы \textit{обобщенного синтаксического анализа}, которые позволяют рассмотреть все возможные пути разбора строки и построить соответствующие деревья вывода.
Поиск шаблонов не требует наличия деревьев вывода, поэтому в дальнейшем алгоритмы синтаксического анализа рассматриваются только как механизм, позволяющий определить принадлежность строки языку.

\subsubsection{Оригинальный GLL-алгоритм}

Generalized LL (GLL) \cite{gll} --- алгоритм, обобщающий идеи нисходящего синтаксического анализа. GLL, в отличие от стандартных LL-алгоритмов, позволяет использовать для анализа произвольную \linebreak контекстно-свободную грамматику, в том числе содержащую леворекурсивные правила. Вместе с тем, GLL наследует такие полезные свойства алгоритмов нисходящего анализа, как непосредственная связь с грамматикой и простота отладки и диагностики ошибок.

Для обработки неоднозначностей GLL разделяет стек анализатора на несколько ветвей, каждая из которых соответствует возможному пути разбора. При таком подходе необходимо компактное представление множества стеков, в качестве которого выступает Graph Structured Stack (GSS). В работе \cite{Afroozeh2015gss} была представлена модификация GSS, которая позволяет увеличить эффективность GLL-анализа. Вершины такого представления хранят в себе номер нетерминала и позицию в строке, с которой начался разбор подстроки, соответствующей ему. На ребрах хранятся позиции в грамматике (вида $X \rightarrow \alpha A \cdot \beta$), на которые необходимо вернуться после завершения разбора нетерминала. 
%При помощи GSS также решается проблема бесконечного роста стеков при обработке левой рекурсии: при попытке создать вершину, которая уже существует, в граф добавится

Основной идеей GLL является использование дескрипторов, позволяющих полностью описывать состояние анализатора в текущий момент времени.

\begin{defn}
	Дескриптор --- это тройка (L, u, i), где
	\begin{itemize}
		\setlength\itemsep{0em}
		\item L --- текущая позиция в грамматике вида $A \rightarrow \alpha \cdot \beta$
		\item u --- текущая вершина GSS
		\item i --- позиция во входном потоке 
	\end{itemize}
\end{defn}  

В процессе работы поддерживается глобальная очередь дескрипторов. В начале каждого шага исполнения алгоритм берет следующий в очереди дескриптор и производит действия в зависимости от позиции в грамматике и текущего входного символа, передвигая соответствующие указатели. 
При наличии конфликтов в грамматике алгоритм добавляет дескрипторы для каждого возможного пути анализа в конец очереди.

\subsubsection{Поддержка грамматик в EBNF}

В работе Артема Горохова \cite{Gorokhov2017ebnf} была описана модификация GLL, которая позволяет использовать грамматики, записанные в расширенной форме Бэкуса-Наура (EBNF). Грамматика такого вида трансформируется в соответствующий рекурсивный автомат, в котором затем минимизируется количество состояний. Синтаксический анализ производится без построения управляющих таблиц: алгоритм обходит рекурсивный автомат в соответствии со входным потоком символов. При обработке текущего дескриптора $(C_S, C_U, i)$, где $C_S$ --- вершина автомата (эквивалент позиции в грамматике), $C_U$ --- вершина GSS, $i$ --- позиция в строке, могут возникать следующие ситуации.

\begin{itemize}
	\item $C_S$ --- финальное состояние. Показывает, что разбор текущего нетерминала был завершен. Необходимо осуществить возврат из $C_U$ по меткам на исходящих из нее ребрах.
	\item Присутствует нетерминальный переход из $C_S$. В данном случае необходимо начать разбор указанного нетерминала $X$. Для этого в GSS должна быть создана новая вершина $(X, i)$, если она не создавалась ранее, а текущая вершина автомата изменена на стартовую для $X$.
	\item Присутствует терминальный переход из $C_S$. Необходимо сравнить терминал на ребре автомата с текущим входным символом. Если они совпадают, то осуществить переход в вершину автомата, на которую указывает ребро, и передвинуть указатель в строке.
\end{itemize}

За счет уменьшения количества состояния в автомате удается достичь прироста в производительности по сравнению со стандартным GLL-алгоритмом. 

\subsubsection{Синтаксический анализ графов}

Стандартными входными данными для алгоритмов синтаксического анализа являются линейные последовательности токенов. На основе GLL был разработан алгоритм, который позволяет производить синтаксический анализ регулярных множеств строк, представленных в виде конечного автомата (который, в свою очередь, является ориентированным графом с токенами на ребрах).

Поддержка нелинейного входа не потребовала существенных изменений в оригинальном алгоритме. Дескрипторы модифицированного алгоритма хранят номер вершины входного графа вместо позиции в строке. Также, на шаге исполнения просматривается не единственный текущий символ, а множество символов на ребрах, исходящих из текущей вершины.

Производительность данного алгоритма, как и обычного GLL, может быть увеличена при помощи представления входной грамматики в виде рекурсивного автомата. В таком случае, алгоритм будет производить обход двух автоматов --- рекурсивного и конечного. Ситуации, возникающие при обработке дексрипторов, не отличаются от описанных ранее ситуаций для линейного входа. Псевдокод данной модификации приведен в приложении.

\subsection{Проект YaccConstructor}

YaccConstructor --- исследовательский проект лаборатории языковых инструментов JetBrains на математико-механическом факультете СПбГУ, направленный на исследования в области лексического и синтаксического анализа. Проект включает в себя одноименную модульную платформу для разработки лексических и синтаксических анализаторов, содержащую большое количество компонент: язык описания грамматик YARD, преобразования над грамматиками и др. Основным языком разработки является F$\#$.

Ранее в рамках YaccConstructor были реализованы генераторы GLL-анализаторов, описание которых было приведено в данном обзоре. 

\section{\bf Modified Valiant's algorithm}

In this section we describe the reorganization of submatrices processing order in the Valiant's algorithm which simplify independent handling of submatrices. As a result, proposed modification can facilitate implementation of parallel submatrix processing.

\subsection{\bf \it Layered submatrices processing}

The main change of this modification is the possibility to divide the parsing table into layers of disjoint submatrices of the same size.
The idea of division we have made from the reorganization of the matrix multiplication order is presented in figure~\ref{fig2}.
Each layer consists of square matrices which size is power of 2.
The layers are computed successively in the bottom-up order.
Each matrix in the layer can be handled independently, which can help to implement parallel version of layer processing function.

\begin{figure}[h]
\vspace{3mm}
 \begin{center}
 \includegraphics[width=12cm]{pictures/modivis2.pdf}
    \caption{An example of the modification of Valiant's algorithm}
    \label{fig4}
 \end{center}
\vspace{-8mm}
\end{figure}

A simple example of the modification is shown in figure~\ref{fig4}.
The lowest layer (submatrices which size is 1) is already computed and filling of the matrix starts with the second layer (subfigures 1-2).
Note that the same process is presented in figure~\ref{fig3}, but here it can be done only in two steps using parallel computation of submatrix products.

The modified version of Valiant's algorithm is presented in listing~\ref{algo:modified}.
The procedure \textit{main()} computes the lowest layer $(T_{l, l+1})$, and then divide the table into layers, described earlier, and computes them through the \textit{completeVLayer()} call.
Thus, \textit{main()} computes all elements of parsing table $T$.
(Hereinafter, we use layer to mean set of submatrices.)

For brevity, we define \textit{left(subm), right(subm), top(subm), bottom(subm), \linebreak rightgrounded(subm)} and \textit{leftgrounded(subm)} functions which returns the submatrices for matrix $subm = (l, m, l', m')$ according to the original Valiant's algorithm (figure~\ref{fig2}).

Also denote some subsidiary functions for matrix layer $M$:
\begin{itemize}[noitemsep, nolistsep]
    \item[$-$] \textit{bottomsublayer(M)} $ = \{bottom(subm)\, |\,subm \in M \}$,
    \item[$-$] \textit{leftsublayer(M)} $ = \{\textit{left(subm)}\, |\,subm \in M \}$,
    \item[$-$] \textit{rightsublayer(M)} $ =\{\textit{right(subm)}\, |\,subm \in M \}$,
    \item[$-$] \textit{topsublayer(M)} $ = \{top(subm)\, |\,subm \in M \}$.
\end{itemize}

\begin{algorithm}[!h]
\SetAlgoNoLine
\KwIn{$G = (\Sigma, N, R, S), w = a_{1} \dots a_{n}, n \geq 1, n + 1 = 2^p, a_{i} \in \Sigma$ }
\underline{main()}{:}{

 \For {$l \in \{1, \ldots, n \}$}{$T_{l, l + 1} = \{A | A \rightarrow a_{l + 1} \in R\}$}
 \For{$1 \le i < p - 1 $}{
 layer = \textit{constructLayer(i)}\;
 \textit{completeVLayer(layer)}
 }
 accept if and only if $S \in T_{0, n}$
 \BlankLine
 }

\underline{constructLayer(i)}{:}{
 \BlankLine
 $\{(k2^i, (k+1)2^i, (k + 1)2^i, (k+2)2^i) \, |\, 0 \le k < 2^{p - i} - 1\}$
 \BlankLine
    }
\underline{completeLayer(M)}{:}{
\BlankLine
\If {$\forall (l, m, l', m') \in M \quad (m - l = 1)$}{\For{$ (l, m, l', m') \in M$}{$T_{l, l'} = f(P_{l, l'})$\;}}
\Else{
\textit{completeLayer(bottomsublayer(M))}\;
\textit{completeVLayer(M)}
}
\BlankLine
}

\underline{comleteVLayer(M)}{:}{
 \BlankLine
 \textit{multiplicationTasks$_1$ = \linebreak
    \{$left(subm)$, $leftgrounded(subm)$, $bottom(subm)\, |\,subm \in M \} \cup \linebreak  \{right(subm), bottom(subm), rightgrounded(subm)\, |\,subm \in M\}$\;}
 \BlankLine
 multiplicationTask$_2$ = $\{top(subm), leftgrounded(subm), right(subm)\, |\,subm \in M\}$\;
 \BlankLine
 multiplicationTask$_3$ = $\{top(subm), left(subm), rightgrounded\, |\,subm \in M\}$\;
 \BlankLine
 \textit{performMultiplications(multiplicationTask$_1$)}\;
 \textit{completeLayer(leftsublayer(M) $\cup$ rightsublayer(M))}\;
 \textit{performMultiplications(multiplicationTask$_2$)}\;
 \textit{performMultiplications(multiplicationTask$_3$)}\;
 \textit{completeLayer(topsublayer(M))}

 }
 \BlankLine

 \underline{performMultiplication(tasks)}{:}{\\
 \For{$ (m, m1, m2) \in \textit{tasks}$}{$P_{m} = P_{m} \cup (T_{m1} \times T_{m2})$\;}
 }

\caption{Parsing by matrix multiplication: Modified Version}
\label{algo:modified}
\end{algorithm}


The procedure \textit{completeVLayer(M)} takes an array of disjoint submatrices $M$ which represents a layer.
For each \textit{subm = (l, m, l', m') $\in M$} this procedure computes \textit{left(subm), right(subm), top(subm)}.
The procedure assumes that the elements of \textit{bottom(subm)} and $T_{i, j}$ for all $i$ and $j$ such that $l \leq i < j < m$ and $  l' \leq i < j < m'$ are already constructed.
Also it is assumed that the current value of
$P_{i, j} =  \{ (B, C) | \exists k, (m \le k < l'), a_{i + 1} \dots a_{k} \in L_G(B), a_{k + 1} \dots a_{j} \in L_G(C)\} $ for all $i$ and $j$ such that $l \leq i < m$ and $l' \leq j < m'$.

The procedure \textit{completeLayer(M)} also takes an array of disjoint submatrices $M$, but unlike the previous one, it computes $T_{i, j}$ for all $(i, j) \in subm$.
This procedure requires exactly same assumptions on $T_{i, j}$  and $P_{i, j}$  as in the previous case.

In the other words, \textit{completeVLayer(M)} computes the entire layer \textit{M} \linebreak and \textit{completeLayer($M_{2}$)} is a support function which is necessary for computation of smaller square submatrices $subm_{2} \in M_{2}$ inside of \textit{M}.

Finally, the procedure \textit{performMultiplication(tasks)}, where \textit{tasks} is an array of a triple of submatrices, perform basic step of algorithm: matrix multiplication. It is worth mentioning that, as distinct from the original algorithm, here $|tasks| \ge 1$ and each task can be computed independently.
So, practical implementation of this procedure can easily involve different techniques of parallel array processing, such as OpenMP.

\subsection{\bf \it Algorithm for substrings}

Next we show how our modification can be applied to the string-matching problem.

So if we want to find all substrings of size $s$ which can be derived from a start symbol for an input string of size $n = 2^p$, we need to compute layers with submatrices of size not greater than $2^{l'}$, where $2^{l' - 2} < s \le 2^{l' - 1}$.

Let $l' = p - (m - 2)$ and consequently $(m - 2) = p - l'$.

For any  $m \le i \le p$ products of submatrices of size $2^{p - i}$ are calculated exactly $2^{2i - 1} - 2^{i}$ times and each of them imply multiplying $\mathcal{O}(|G|)$ Boolean submatrices.

\begin{equation}
\begin{array}{c}
C \sum\limits_{i=m}^p 2^{2i - 1} \cdot 2^{\omega(p - i)} \cdot f(2^{p - i}) =
C \cdot 2^{\omega l'}\sum\limits_{i=2}^{l'} 2^{(2 - \omega)i} \cdot 2^{2(p - l') - 1} \cdot f(2^{l' - i}) \le \\
C \cdot 2^{\omega l'} f(2^{l'}) \cdot 2^{2(p - l') - 1} \sum\limits_{i=2}^{l'} 2^{(2 - \omega)i} =
BMM(2^{l'}) \cdot 2^{2(p - l') - 1} \sum\limits_{i=2}^{l'} 2^{(2 - \omega)i}
\end{array}
\end{equation}

Thus, time complexity for searching all substrings is  $O(|G|BMM(2^{l'})(l' - 1))$, while time complexity for the full input string is $O(|G|BMM(2^p)(p - 1))$. In contract to the modification, Valiant's algorithm completely calculate at least 2 triangle submatrices of size $\frac{n}{2}$ (as shown in figure~\ref{fig5}) which mean minimum asymptotic complexity  $O(|G|BMM(2^{p - 1})(p - 2))$. Thus we can conclude that the modification is asymptotically faster for substrings of size $s \ll n$  than the original algorithm.

\begin{figure}[h]
\vspace{3mm}
 \begin{center}
 \includegraphics[width=12cm]{pictures/valsubstring.pdf}
    \caption{The number of elements necessary to compute in Valiant's algorithm. That means it is nessesary to calculate at least 2 triangle submatrices of size $\frac{n}{2}$.}
    \label{fig5}
 \end{center}
\vspace{-8mm}
\end{figure}

\section{Conclusion and Future Work}

We present !!!

Our evaluation shows that !!!

First direction for future research is a more detailed CFPQ algorithms investigation.
We should do More evaluation on sparse matrices on GPGPUs.

Also it is nesessary to implement and evaluate solutions for graphs which is not fit in RAM.
There is a set of technics for huge matrices multiplication.
Is it possible to dopt it for CFPQ

Another direcion is a dataset improvement.
More data.
More grammars/queries.


\setmonofont[Mapping=tex-text]{CMU Typewriter Text}
\bibliographystyle{ugost2008ls}
\bibliography{diploma.bib}
\end{document}