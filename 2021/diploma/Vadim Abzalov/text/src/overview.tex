\section{Обзор}

Прежде чем приступать к модернизации \textsc{CFPQ\_Data} необходимо разобраться, какие стандарты оформления наборов данных приняты в современном мире.

\subsection{Наборы графовых данных}

Стоит отметить, что существует множество различных наборов графовых данных~\cite{BoVWFI, BRSLLP, SNAPDATESETS}.
Так, например, проект <<SNAP: Stanford Network Analysis Project>>~\cite{SNAPDATESETS}, который начал активно развиваться в 2004 году в результате исследований по анализу крупных социальных и информационных сетей.
Крупнейшей сетью, которая была проанализирована с помощью библиотеки, была сеть <<Micro\-soft Instant Messenger>> 2006 года, содержащая 240 миллионов вершин и 1,3 миллиарда ребер.
Наборы данных~\cite{SNAPDATESETS}, доступные на веб-сайте библиотеки WebGraph\footnote{Веб-сайт библиотеки <<SNAP: Stanford Network Analysis Project>>: \url{https://snap.stanford.edu/}, дата последнего доступа --- 04.06.2021}, были собраны для целей этих исследований.
Сам набор данных оформлен в виде нескольких таблиц, отвечающих различным прикладным областям, из которых были извлечены графы.
При этом каждая таблица содержит: ссылку на страницу с описанием графа, тип абстракции графа (ориентированный / неориентированный, с весами / без весов и т.п.), количество вершин, количество рёбер и описание того, откуда был извлечён граф.

В работах <<The webgraph framework I: compression techniques>>~\cite{BoVWFI} и <<Layered Label Propagation: A MultiResolution Coordinate-Free Ordering for Compressing Social Networks>>~\cite{BRSLLP} предлагаются новые методы сжатия графов социальных и информационных сетей.
Это важно, поскольку изучение таких графов часто затруднено из-за их большого размера.
На основе этих работ был разработан фреймворк <<WebGraph>> --- набор алгоритмов и инструментов, направленных на упрощение манипулирования большими графами.
С помощью это фреймворка были получены компактные представления различных графов реальных социальных и информационных сетей.
Все эти наборы данных представлены на веб-сайте проекта\footnote{Веб-сайт проекта <<WebGraph>>: \url{http://law.di.unimi.it/datasets.php}, дата последнего доступа --- 04.06.2021}.
Они также оформлены в виде нескольких таблиц.
Каждая таблица содержит: ссылку на страницу с описанием графа, дату загрузки графа, количество вершин и рёбер.

Все эти проекты, собирающие наборы данных для исследований в своих прикладных областях, так или иначе выделяют некоторую общую информацию о каждом графе: описание графа, количество вершин и рёбер.
Подобные данные обязательно должны быть включены в \textsc{CFPQ\_Data}.

Для CFPQ алгоритмов ключевую роль играют метки на рёбрах, которые представляют различные отношения между вершинами графа.
Именно поэтому, указанные выше наборы данных, не подходят для подготовки экспериментального исследования CFPQ алгоритмов, поскольку представляют собой наборы непомеченных графов.
Попытки же синтетического добавления меток могут привести к полной потере всей практической ценности этих графов.

\subsubsection{Наборы графовых данных для задачи с регулярными ограничениями}

Существует довольно много различных наборов данных для экспериментального исследования алгоритмов, реализующих регулярные запросы \cite{RBench, GSCALER, gMark}.
Например, проект <<RBench>>~\cite{RBench} для создания масштабируемых синтетических наборов графовых данных по данному набору графов, представленных в формате RDF.
Однако регулярные запросы представляют более узкий класс, чем контекстно-свободные, что не позволяет в полной мере использовать такие данные для экспериментального исследования CFPQ алгоритмов.

Формат RDF был выбран в качестве основной модели представления графов консорциумом <<W3C>>~\cite{SEMANTICWEB} и, благодаря этому, имеет широкую поддержку.
Он позволяет описывать отношения между ресурсами в виде <<объект, предикат, субъект>>, что идеально соответствует абстракции помеченного графа.
Именно по этим причинам данный формат был выбран в качестве стандартного представления графов, собранных в \textsc{CFPQ\_Data}.

\subsubsection{Наборы графовых данных для задачи с кон\-текст\-но-свобод\-ны\-ми ограничениями}

Графы и грамматики, представляющие наборы данных для подготовки экспериментального исследования CFPQ алгоритмов представлены весьма разрозненно, что вызвано отсутствием единого набора данных и проблемой создания помеченных графов исключительно под соответствующие экспериментальные нужды.

Например, набор популярных онтологий, связанных с концепцией семантической паутины~\cite{SEMANTICWEB}, который можно найти в работе <<Context-Free Path Queries on RDF Graphs>>~\cite{CFPQORDFG}.
Графы именно оттуда наиболее часто использовались для подготовки экспериментального исследования CFPQ алгоритмов.
К сожалению, они достаточно небольшие (несколько сотен вершин), что не позволяет использовать их для исследования практической применимости CFPQ алгоритмов.
Однако, для простой проверки того, что CFPQ алгоритм работает, такие данных отлично подходят. 

Недавно появилась работа <<An Experimental Study of Context-Free Path Query Evaluation Methods>>~\cite{ANESOFCFPQEM}, в которой представлены графы гораздо большего размера (от нескольких тысяч до первых миллионов вершин), что уже позволяет использовать их для исследования практической применимости CFPQ алгоритмов.
Поскольку такие графы по своим размерам гораздо лучше соответствуют тем помеченным графам, которые извлекались из различных практических областей в других наборах графовых данных~\cite{BoVWFI, BRSLLP, SNAPDATESETS}.

 В работах <<Batch alias analysis>>~\cite{BAA} и <<Demand-driven alias analysis for C>>~\cite{DDAAFORC} используются помеченные графы, представляющие данные для задачи поиска объектов ссылающихся на одни и те же места в памяти.
 Так как эта задача сводится к поиску путей с кон\-текст\-но-свобод\-ными ограничениями, то граф, построенный для её решения, однозначно соответствует абстракции помеченного графа, используемой в CFPQ алгоритмах.

Графы из представленных выше работ~\cite{CFPQORDFG, ANESOFCFPQEM, BAA, DDAAFORC} уже добавлены в \textsc{CFPQ\_Data}.
Поскольку они идеально соответствуют абстракции помеченного графа и представляют реальные данные из различных прикладных областей, что позволяет полноценно использовать их для подготовки экспериментального исследования CFPQ алгоритмов.

В работе <<Subgraph Queries by Context-free Grammars>>~\cite{SQBYCFG} для экспериментального исследования нового CFPQ алгоритма синтезирован граф на примерно 1 миллион вершин и примерно 5.7 миллионов рёбер путём объединения набора общедоступных источников данных: UniProt (белки), Entrez Gene (гены), Gene Ontology (функции белков, биологические процессы и клеточные местоположения), InterPro (семейства белков и консервативные домены), KEGG (биохимические пути), OMIM (отношения ген-фенотип), HomoloGene (группы гомологии генов) и STRING (взаимодействия белков).
Подобный подход к подготовке экспериментального исследования с одной стороны, является весьма перспективным, поскольку позволяет синтезировать помеченные графы любых размеров, отвечающие реальным данным, но, с другой стороны, требует весьма глубокого понимания структуры самих данных, которые будут использованы для построения графа.
Именно поэтому данный способ не применяется в \textsc{CFPQ\_Data}.

\subsection{Проект \textsc{CFPQ\_Data}}

Из-за проблемы разрозненности наборов графовых данных, подходящих для использования в экспериментальном исследовании CFPQ алгоритмов, графы из работ <<Context-Free Path Queries on RDF Graphs>>~\cite{CFPQORDFG}, <<An Experimental Study of Context-Free Path Query Evaluation Methods>>~\cite{ANESOFCFPQEM}, <<Batch alias analysis>>~\cite{BAA} и <<Demand-driven alias analysis for C>>~\cite{DDAAFORC} были собраны в единый набор данных, который получил название \textsc{CFPQ\_Data}.

Также в него были добавлены функции для генерации синтетических графов для особых случаев: теоретически доказанный худший случай запроса в виде языка правильных скобочных последовательностей на графе, состоящем из двух циклов~\cite{QFORPINGUCFPQ}; разреженные графы для симуляции реальных данных; графы, результат вычисления запроса на которых является теоретически максимальным; случайные безмасштабные сети, для генерации которых применяется модель Барабаши-Альберта~\cite{SMOFCN}.
А также функции для преобразования кон\-текстно-свобод\-ной грамматики выбранного формата в нормальную форму Хомского.

Но проект \textsc{CFPQ\_Data} имеет ряд технических проблем, не позволяющих в полной мере насладиться процессом подготовки экспериментального исследования CFPQ алгоритмов.
Так, вместо того, чтобы предоставить исследователям возможность загружать конкретный граф, набор данных загружается целиком, что становится критической проблемой при увеличении количества графов в наборе.
При этом в самом наборе данных имеется информация лишь о названиях графов в нем содержащихся, что не соответствует принятым в сообществе стандартам оформления наборов графовых данных.
Кроме того, все функции, предоставляемые проектом \textsc{CFPQ\_Data}, доступны пользователю через единственный интерфейс командной строки, что крайне, крайне радикально ограничивает возможности по взаимодействию с набором данных и подготовке экспериментального исследования замечательных CFPQ алгоритмов.
