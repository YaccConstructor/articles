Среди множества направлений научных исследований, которое охватывает вычислительная биология, особое место занимают различные прикладные задачи, связанные с анализом последовательностей, входящих в состав важнейших для всех живых организмов макромолекул --- ДНК, РНК и белков. Процесс разработки и оптимизации алгоритмов для решения целого ряда задач, например, классификации организмов, расшифровки геномов, предсказания функций белков и других, не прекращается уже много лет, и за это время были выработаны некоторые общие принципы работы с биологическими данными.

Во-первых, линейная (первичная) и пространственная (вторичная) структуры этих макромолекул содержат важную информацию о клеточных функциях и эволюционном происхождении организмов и могут быть формально описаны с помощью различных математических моделей. В частности, первичная структура молекулы РНК представляет собой цепочку особых веществ --- нуклеотидов, --- и в том случае, когда два фрагмента этой цепи соединяются друг с другом, перегибаясь и образуя на конце неспаренный участок в форме петли, формируется элемент, называемый в биологии стемом. Совокупность вложенных стемов разных размеров составляет сложную и стабильную вторичную структуру. Известно, что вторичная структура играет важную роль в регуляции клеточных процессов~\cite{vandivier2016conservation}, поэтому во многих геномных исследованиях требуется учитывать или предсказывать ее конфигурацию. Существуют различные методы формального описания вторичной структуры, например, скрытые марковские модели~\cite{yoon2004hmm}, ковариационные модели~\cite{sippl1999biological} и формальные грамматики~\cite{dowell2004evaluation,knudsen1999rna,rivas2000language}.

Во-вторых, при работе с биологическими данными важно учитывать их потенциальную зашумленность, т.е. присутствие различных неточностей, мутаций и случайных всплесков, и, кроме того, законы образования пространственных молекулярных структур сами по себе имеют стохастическую природу. Поэтому в данной области у точных алгоритмов зачастую выигрывают те, что предполагают некоторого рода вероятностную оценку. Популярным способом обработки зашумленных данных являются методы машинного обучения, в частности, нейронные сети, которые в настоящее время успешно используются в том числе и в биоинформатике~\cite{higashi2009bacteria,sherman2017humidor}.

В рамках предыдущей дипломной работы был разработан подход для решения задач обработки последовательностей, обладающих некоторой синтаксической структурой. Данный подход основан на комбинировании методов синтаксического анализа и машинного обучения и может быть применен в совершенно разных предметных областях. Предлагается использовать формальную грамматику для кодирования характерных элементов синтаксической структуры, алгоритм синтаксического анализа --- для их поиска во входных данных, а обработку информации о наличии и расположении этих элементов в цепочке и вероятностную оценку провести с помощью нейронной сети, которая некоторым образом обрабатывает сгенерированные парсером данные. Анализ геномных последовательностей является одной из потенциальных областей применения этого подхода и, если говорить непосредственно об исследовании РНК, то входными данными являются нуклеотидные цепочки, под синтаксической структурой следует понимать вторичную структуру РНК, а под искомыми характерными элементами --- составляющие ее стемы. 

Направлением исследования, представленного в данной работе, является предсказание вторичной структуры РНК с использованием описанного выше подхода. Правила контекстно-свободной грамматики описывают определенный по некоторым эмпирическим критериям общий вид стемов вторичной структуры, а синтаксический анализатор выполняет задачу поиска подстроки в строке, что с теоретической точки зрения означает получение всех выводимых по правилам грамматики подстрок, а с практической --- всех потенциально возможных в данной строке стемов. Однако в контексте реальной вторичной структуры РНК живого организма эта информация является избыточной, так как из всех возможных комбинаций стемов будет присутствовать только какая-то одна, а иногда и недостаточной, потому что грамматика не может не содержать определенные ограничения, например, на максимальный размер петли внутри стема. Поэтому для генерации чистой вторичной структуры из результата работы парсера в рамках рассматриваемого подхода предлагается использовать нейронную сеть, задача которой в данном случае --- отфильтровать лишние стемы и достроить невыразимые в грамматике элементы.