\input{header.tex}
\input{header2.tex}

\usepackage{caption}
\usepackage{listings}
\usepackage{graphicx}
\usepackage{multirow}
\usepackage{fancyvrb}
\usepackage[misc,geometry]{ifsym} 
\usepackage{subcaption}
\usepackage{multirow}
\usepackage{array}
\newcolumntype{P}[1]{>{\centering\arraybackslash}p{#1}}
\usepackage{tikz}
\usepackage{bbding}
\usepackage{pifont}
\usepackage{wasysym}
\usepackage{amssymb}
\usepackage{listings}
\usepackage{pythonhighlight}
\newtheorem{definition}{Определение}

\DeclareCaptionFont{white}{ \color{white} }
\DeclareCaptionFormat{listing}{
    \parbox{\textwidth}{\hspace{15pt}#1#2#3}
}
\captionsetup[lstlisting]{ format=listing
  %, labelfont=white, textfont=white
  , singlelinecheck=false, margin=0pt, font={bf}
}

\begin{document}
%% Если что-то забыли, при компиляции будут ошибки Undefined control sequence \my@title@<что забыли>@ru
%% Если англоязычная титульная страница не нужна, то ее можно просто удалить.
\filltitle{ru}{
    %% Актуально только для курсовых/практик. ВКР защищаются не на кафедре а в ГЭК по направлению, 
    %%   и к моменту защиты вы будете уже не в группе.
    chair              = {Кафедра системного программирования},
    group              = {17Б.11-мм},
    %% Макрос filltitle ненавидит пустые строки, поэтому обязателен хотя бы символ комментария на строке
    %% Актуально всем.
    title              = {Реализация алгоритма поиска путей в графовых базах данных через тензорное произведение на GPGPU},
    % 
    %% Здесь указывается тип работы. Возможные значения:
    %%   coursework - отчёт по курсовой работе;
    %%   practice - отчёт по учебной практике;
    %%   prediploma - отчёт по преддипломной практике;
    %%   master - ВКР магистра;
    %%   bachelor - ВКР бакалавра.
    type               = {bachelor},
    author             = {Орачев Егор Станиславович},
    % 
    %% Актуально только для ВКР. Указывается код и название направления подготовки. Типичные примеры:
    %%   02.03.03 <<Математическое обеспечение и администрирование информационных систем>>
    %%   02.04.03 <<Математическое обеспечение и администрирование информационных систем>>
    %%   09.03.04 <<Программная инженерия>>
    %%   09.04.04 <<Программная инженерия>>
    %% Те, что с 03 в середине --- бакалавриат, с 04 --- магистратура.
    specialty          = {09.03.04 <<Программная инженерия>>},
    % 
    %% Актуально только для ВКР. Указывается шифр и название образовательной программы. Типичные примеры:
    %%   СВ.5006.2017 <<Математическое обеспечение и администрирование информационных систем>>
    %%   СВ.5162.2020 <<Технологии программирования>>
    %%   СВ.5080.2017 <<Программная инженерия>>
    %%   ВМ.5665.2019 <<Математическое обеспечение и администрирование информационных систем>>
    %%   ВМ.5666.2019 <<Программная инженерия>>
    %% Шифр и название программы можно посмотреть в учебном плане, по которому вы учитесь. 
    %% СВ.* --- бакалавриат, ВМ.* --- магистратура. В конце --- год поступления (не обязательно ваш, если вы были в академе/вылетали).
    programme          = {СВ.5080.2017 <<Программная инженерия>>},
    % 
    %% Актуально только для ВКР, только для матобеса и только 2017-2018 годов поступления. Указывается профиль подготовки, на котором вы учитесь.
    %% Названия профилей можно найти в учебном плане в списке дисциплин по выбору. На каком именно вы, вам должны были сказать после второго курса (можно уточнить в студотделе).
    %% Вот возможные вариканты:
    %%   Математические основы информатики
    %%   Информационные системы и базы данных
    %%   Параллельное программирование
    %%   Системное программирование
    %%   Технология программирования
    %%   Администрирование информационных систем
    %%   Реинжиниринг программного обеспечения
    % profile            = {Системное программирование},
    % 
    %% Актуально всем.
    supervisorPosition = {Доцент кафедры информатики, к.\,ф.-м.\,н.},
    supervisor         = {С. В. Григорьев},
    % 
    %% Актуально только для практик и курсовых. Если консультанта нет, закомментировать или удалить вовсе.
    % consultantPosition = {должность ООО <<Место работы>> степень},
    % consultant         = {К.К. Консультант},
    %
    %% Актуально только для ВКР.
    reviewerPosition   = {Разработчик биоинформатического ПО, ЗАО “БИОКАД”},
    reviewer           = {А.С. Хорошев},
}

\filltitle{en}{
    chair              = {Software Engineering},
    group              = {17B.11-mm},
    title              = {Context-Free path querying by tensor product for graph databases on GPGPU},
    type               = {bachelor},
    author             = {Egor Orachev},
    % 
    %% Possible choices:
    %%   02.03.03 <<Software and Administration of Information Systems>>
    %%   02.04.03 <<Software and Administration of Information Systems>>
    %%   09.03.04 <<Software Engineering>>
    %%   09.04.04 <<Software Engineering>>
    %% Те, что с 03 в середине --- бакалавриат, с 04 --- магистратура.
    specialty          = {09.03.04 <<Software Engineering>>},
    % 
    %% Possible choices:
    %%   СВ.5006.2017 <<Software and Administration of Information Systems>>
    %%   СВ.5162.2020 <<Programming Technologies>>
    %%   СВ.5080.2017 <<Software Engineering>>
    %%   ВМ.5665.2019 <<Software and Administration of Information Systems>>
    %%   ВМ.5666.2019 <<Software Engineering>>
    programme          = {СВ.5080.2017 <<Software Engineering>>},
    % 
    %% Possible choices:
    %%   Mathematical Foundations of Informatics
    %%   Information Systems and Databases
    %%   Parallel Programming
    %%   System Programming
    %%   Programming Technology
    %%   Information Systems Administration
    %%   Software Reengineering
    % profile            = {Software Engineering},
    % 
    %% Note that common title translations are:
    %%   кандидат наук --- C.Sc. (NOT Ph.D.)
    %%   доктор ... наук --- Sc.D.
    %%   доцент --- docent (NOT assistant/associate prof.)
    %%   профессор --- prof.
    supervisorPosition = {C.Sc., docent},
    supervisor         = {S.V. Grigorev},
    % 
    % consultantPosition = {position at ``Company'', degree if present},
    % consultant         = {C.C. Consultant},
    % %
    reviewerPosition   = {Bioinformatics Software Engineer, BIOCAD},
    reviewer           = {A.S. Khoroshev},
}
\maketitle
\setcounter{tocdepth}{2}
\tableofcontents

% \begin{abstract}
%   В курсаче не нужен
% \end{abstract}

\section{Введение}
\paragraph{}
Графы являются одной из основных структур в информатике, а алгоритмы над ними чрезвычайно важны в анализе компьютерных и социальных сетей, статическом анализе кода, биоинформатике~\cite{gb_math}. Графы, возникающие в данных областях, могут содержать миллионы узлов и ребер, поэтому распараллеливание алгоритмов для работы с ними оказывается необходимым условием достижения высокой производительности. Однако такое улучшение их эффективности происходит за счет более сложной модели программирования~\cite{blast}. Результатом этого становится, в том числе, несоответствие между языками высокого уровня, на которых пользователи и разработчики графовых алгоритмов предпочли бы программировать (например, Python), и языками программирования для параллельного оборудования~\cite{blast}. Таким образом, существующие решения либо просты в использовании (networkx\footnote{Репозиторий библиотеки networkx: \url{https://github.com/networkx/networkx}. Дата посещения: 13.12.2020}), либо высокопроизводительны (gunrock\footnote{Репозиторий библиотеки gunrock: \url{https://github.com/gunrock/gunrock} Дата посещения: 13.12.2020}). Проблемой является и то, что традиционные параллельные алгоритмы для анализа графов тяжело реализовывать и оптимизировать, а прирост производительности от роста числа параллельных процессов снижается~\cite{gb_math}. 

С появлением эффективных структур и алгоритмов для разреженных матриц становится возможным использование подхода к вычислению над графами, основанного на линейной алгебре. Матрица смежности может представлять широкий спектр графов, включая ориентированные, взвешенные, двудольные. Ключевым свойством такого подхода является способность оперировать богатым набором графов различных типов с помощью небольшого набора матричных операций над полукольцами. Например, транспонирование матрицы смежности изменяет направления ребер в ориентированном графе, а умножение матрицы на вектор, как показано на рисунке~\ref{fig:bfs_step}, является шагом в алгоритме поиска в ширину.

\begin{figure}[h!]
    \centering
    \includegraphics[width=0.9\linewidth]{pictures/MatrixBFS.png}
    \caption{Вычисление одного шага в алгоритме поиска в ширину\footnotemark}
    \label{fig:bfs_step}
\end{figure}

Спецификация GraphBLAS~\cite{gb_math} определяет базовые примитивы для построения графовых алгоритмов в терминах линейной алгебры. Разработчики считают, что эта область достаточно зрела, чтобы иметь потребность в стандартизации. Стандартизация позволяет сконцентрировать усилия исследователей на разработке инновационных алгоритмов для анализа и обработки графов, а не придумывать все новые, во многом пересекающиеся, низкоуровневые решения. Кроме того, благодаря такому подходу, по словам авторов\cite{sevengr}, возможно решение некоторых проблем, описанных ниже.
\begin{enumerate}
    \item \textbf{Переносимость.} Алгоритмы не требуют модификаций для достижения высокой производительности на конкретном устройстве.
    \item \textbf{Лаконичность.} Алгоритмы выражаются гораздо меньшим числом строчек кода.
    \item \textbf{Производительность.} Алгоритмы остаются высокопроизводительными.
    \item \textbf{Масштабируемость.} Алгоритмы эффективны как на небольших, так и на огромных данных.
\end{enumerate}

\footnotetext{GraphBLAS [Электронный ресурс] // Википедия. Свободная энциклопедия. – URL: \url{https://en.wikipedia.org/wiki/GraphBLAS} (дата обращения: 13.12.2020).}

GraphBLAS описывает небольшое множество математических операций, которые необходимы для реализации широкого спектра операций над графами. В стандарте описаны следующие объекты:
\begin{itemize}
    \item абстрактные структуры для хранения дынных (матрицы, векторы)
    \item алгебраические структуры (моноиды, полукольца, бинарные и унарные операторы)
    \item операции линейной алгебры над произвольными алгебраическими структурами (произведение матриц, поэлементное сложение и умножение, взятие подматрицы и т.д.)
    \item объекты управления (маски и дескрипторы)
\end{itemize}

На данный момент стандарт GraphBLAS уже имеет несколько полноценных реализаций\footnote{Форум, посвященный стандарту GraphBLAS: \url{https://graphblas.github.io/}. Дата посещения: 04.06.2021}, однако все они в основном ориентированы на исполнение на CPU. В то же время разработка инструмента с поддержкой исполнения на графических процессорах общего назначения является перспективным направлением исследований, так как их использование может существенно повысить производительность такого рода решений\cite{gbtl}\cite{blast}. На текущий момент нет стандартного подхода к реализации спецификации GraphBLAS на GPU --- разработчики сталкиваются не только с проблемами, связанными с реализацией обобщенных операций на графических процессорах с помощью стандартных инструментов языка C++, но и с переносимостью решений, основанных на программно-аппаратной платформе CUDA. Одним из возможных подходов к реализации GraphBLAS на GPU является использование языка высокого уровня, а также библиотек, динамически транслирующих конструкции и объекты данного языка в низкоуровневый код, способный исполнятся на графическом процессоре видеокарты. 


\section{Постановка задачи}
\label{sec:task}
% !TeX spellcheck = ru_RU
Целью данной работы является улучшение производительности алгоритма поиска путей с контекстно-свободными ограничениями, основанного на произведении Кронекера. Для достижения цели были поставлены приведенные ниже задачи.

\begin{enumerate}
    \item Улучшить алгоритм построения индекса и реализовать полученное улучшение.
    \item Разработать алгоритм извлечения путей по результату работы алгоритма построения индекса и реализовать полученный алгоритм.
    \item Разработать и реализовать алгоритм обнаружения путей, исходящих из заданного набора стартовых вершин.
    \item Провести экспериментальное исследование производительности реализаций. Для демонстрации затраченного времени работы реализованных алгоритмов над исходными данными использовать набор данных \textit{CFPQ\_Data}.
\end{enumerate}


\section{Обзор}
\label{sec:relatedworks}
\paragraph{} 
В таблице~\ref{tab:libs_comparison} представлено сравнение текущих реализаций спецификации GraphBLAS, а также некоторых других популярных библиотек для анализа графов на GPU, основанных на иных подходах к построению алгоритмов.

Одним из критериев оценки существующих решений стало наличие возможности использования графических процессоров видеокарты в качестве платформы для исполнения вычислений. Такая возможность может существенно повысить производительность решений, основанных на линейной алгебре, так как операции с разреженными матрицами поддаются массовому параллелизму~\cite{blast}. Однако реализация GraphBLAS на GPU является открытой проблемой --- как показано в таблице~\ref{tab:libs_comparison}, на данный момент нет решений, которые бы полностью поддерживали вычисления на графическом процессоре. Это связано, в том числе с тем, что библиотеки, использующее для этого языки C и С++, сталкиваются с рядом проблем при попытке реализовать обобщенные операции на GPU. Кроме того, все существующие решения, которые разрабатываются для исполнения на GPU, используют программно-аппаратную архитектуру CUDA. Это оз\-на\-ча\-ет, что они могут исполняться только на графических устройствах от NVIDIA, что негативно сказывается на переносимости таких библиотек. Решение, описываемое в данной работе, предлагает использовать открытую платформу OpenCL, которая позволяет осуществлять вычисления на любом устройстве с поддержкой данного стандарта, будь то GPU, CPU или FPGA. Такой подход не только позволит использовать всю вычислительную силу современных графических систем, но и облегчит разработку и тестирование алгоритмов широкому кругу исследовательских групп и независимых разработчиков.

Другим критерием оценки существующих библиотек стал язык программирования, на котором она написаны. Как показано в таблице~\ref{tab:libs_comparison}, на текущий момент большинство реализаций написаны для языков C и C++. В то же время использование языков более высокого уровня имеет свои преимущества. Среди них можно выделить возможность дополнительных проверок во время компиляции, богатую экосистему некоторых платформ, абстрактную систему типов, благодаря которой можно более полно описывать абстракции линейной алгебры. Для функциональных языков программирования могут быть доступны некоторые техники суперкомпиляции и оптимизации, например kernel fusion\cite{fusion} и deforestation\cite{deforset}.

\begin{table}
    % \centering
    \begin{minipage}{\linewidth}
    \begin{tabular}{l|l|l}
        Реализация & Язык & Поддержка GPU \\
        \hline \hline
        SuiteSparse~\cite{suitesparse} & C & Нет (поддержка CUDA в разработке) \\
        IBM GraphBLAS\footnote{Репозиторий библиотеки IBM GraphBLAS: \url{https://github.com/IBM/ibmgraphblas}. Дата посещения: 13.12.2020} & C & Нет \\
        GBTL~\cite{gbtl} & C++ & Нет (версия с CUDA устарела) \\
        GraphBLAST~\cite{blast} & C++ & CUDA (в разработке) \\
        CombBLAS~\cite{combblas} & C++ & Частичная, CUDA \\
        Graphulo\footnote{Репозиторий библиотеки Graphulo: \url{https://github.com/Accla/graphulo}. Дата посещения: 13.12.2020} & Java & Нет \\ 
        GraphMat\footnote{Репозиторий библиотеки GraphMat: \url{https://github.com/narayanan2004/GraphMat}. Дата посещения: 13.12.2020} & С++ & Нет \\
        \hline
        Gunrock~\cite{gunrock} & C++ & CUDA \\
        CuSha\footnote{Репозиторий библиотеки CuSha: \url{https://github.com/farkhor/CuSha}. Дата посещения: 13.12.2020} & C++ & CUDA \\
        \hline
    \end{tabular}
    \end{minipage}
    \caption{Сравнение текущих реализаций GraphBLAS (верхний сегмент) и некоторых популярных библиотек для анализа графов на GPU, основанных на иных подходах к построению алгоритмов (нижний сегмент).}
    \label{tab:libs_comparison}
\end{table}

В таблице~\ref{tab:opencl_comparison} приведены инструменты для работы с OpenCL на платформах .NET и Java. Эти платформы достаточно популярны, чтобы предлагаемое решение было более востребовано. Библиотеки \\ OpenCL.NET, Cloo, JavaCL и JOCL предоставляют высокоуровневый интерфейс для работы с ядрами, написанными на C. Эти библиотеки не позволяют естественным образом работать со сложными типами высокоуровневого языка, поэтому не подходят для данной работы. Среди оставшихся инструментов была выбрана библиотека Brahma.FSharp для языка программирования F\#. На данный момент она активную разрабатывается на кафедре системного программирования СПбГУ, поэтому в случае возникновения ошибок можно ожидать их своевременного исправления.

\begin{table}
    % \centering
    \begin{minipage}{\linewidth}
    \begin{tabularx}{\textwidth}{X|X}
    .NET & Java \\
    \hline
    OpenCL.NET\footnote{Репозиторий библиотеки OpenCL.NET: \url{https://github.com/dgsantana/OpenCL.NET}. Дата посещения: 01.06.2021} & JavaCL\footnote{Репозиторий библиотеки JavaCL: \url{https://github.com/nativelibs4java/JavaCL}. Дата посещения: 01.06.2021} \\
    Cloo\footnote{Репозиторий библиотеки Cloo: \url{https://github.com/clSharp/Cloo}. Дата посещения: 01.06.2021} & JOCL\footnote{Репозиторий библиотеки JOCL: \url{https://github.com/gpu/JOCL}. Дата посещения: 01.06.2021} \\
    FSCL\footnote{Репозиторий библиотеки FSCL: \url{https://github.com/FSCL/FSCL.Compiler}. Дата посещения: 01.06.2021} & Aparapi\footnote{Репозиторий библиотеки Aparapi: \url{https://github.com/aparapi/aparapi}. Дата посещения: 01.06.2021} \\
    Brahma.FSharp\footnote{Репозиторий библиотеки Brahma.FSharp: \url{https://github.com/YaccConstructor/Brahma.FSharp}. Дата посещения: 01.06.2021} & ScalaCL\footnote{Репозиторий библиотеки ScalaCL: \url{https://github.com/nativelibs4java/ScalaCL}. Дата посещения: 01.06.2021} \\
    \hline
  \end{tabularx}
  \end{minipage}
  \caption{Инструменты для работы с OpenCL на платформах .NET и Java}
  \label{tab:opencl_comparison}
\end{table}


\section{Архитектура решения}
\chapter{Инструментальный пакет} \label{relWorks}

В данной главе описан инструментальный пакет (Software Development Kit, SDK) \textbf{YC.SEL.SDK}, предназначенный для разработки различных решений по статическому анализу динамически формируемых выражений. Представлена архитектура разработанного SDK, а также архитектура надстройки \textbf{YC.SEL.SDK.ReSharper}, позволяющей создавать расширения для ReSharper, предоставляющие поддержку встроенных языков. Изложенный выше алгоритм синтаксического анализа реализован в рамках одной из компонент SDK. YC.SEL.SDK и YC.SEL.SDK.ReSharper являются \textbf{платформами} для разработки инструментов статического анализа динамически формируемого кода.

\section{Архитектура}

Практически любой язык программирования может использоваться как встроенный. Даже если рассматривать только SQL, то окажется, что у него множество различных диалектов, каждый из которых имеет свои особенности. Внешним языком также может быть любой язык программирования. Трудность заключается в том, что  любое из сочетаний внешнего и встроенного языка может встретиться на практике, и задачи, которые необходимо решать в этой ситуации, могут быть различными (поиск ошибок, подсчёт метрик, автоматизация трансформаций и т.д.). Реализовать универсальный инструмент, решающий все задачи для всех языков, не представляется возможным. Более целесообразно создать набор инструментов, упрощающий создание конечных решений для конкретных языков и конкретных задач. В качестве примера можно рассмотреть инструменты для разработки компиляторов, которые включают в себя генераторы лексических, синтаксических анализаторов и набор библиотек с вспомогательными функциями, и тем самым упрощают создание конкретного компилятора для выбранного языка и целевой платформы.

Требуемый набор инструментов для работы со встроенными языками должен поддерживать весь процесс обработки кода, который может выглядеть так, как представлено на рисунке~\ref{fig:SeqSelProcessing}. Можно выделить следующие основные шаги.
\begin{itemize}
    \item Анализ основного кода, который выполняется сторонним инструментом. Результат этого шага --- это дерево разбора с информацией, достаточной для выполнения дальнейших шагов.
    \item Построение аппроксимации множества возможных значений динамически формируемых выражений.
    \item Лексический анализ построенной на предыдущем шаге аппроксимации.
    \item Синтаксический анализ, результатом которого является лес разбора, пригодный для дальнейшей обработки.
    \item Обработка леса разбора, вычисление семантики.
\end{itemize}

На каждом шаге может быть получена информация, полезная для пользователя, такая как список ошибок, и её необходимо отобразить для него соответствующим образом.

\begin{figure}[h!]
\begin{center}
\includegraphics[width=0.9\textwidth]{pics/Activ_SEL_Processing}
\caption{Диаграмма последовательности обработки встроенных языков}
\label{fig:SeqSelProcessing} 
\end{center}
\end{figure}


Существующие инструменты для работы со встроенными языками обычно реализуют поддержку какого-то фиксированного набора языков. При этом поддержка нового языка, как правило, требует нетривиальнй доработки инструмента. Чтобы получить поддержку встроенного языка без изменений в исходном коде базового инструмента, необходимо предоставить соответствующий механизм. 

Для того чтобы упростить процесс создания конечных инструментов, создан SDK, одной из компонент которого является генератор синтаксических анализаторов на основе предложенного в данной работе алгоритма. Также в него входит генератор лексических анализаторов, библиотека построения регулярной аппроксимации, набор вспомогательных функций. Подробное описание компонент приведено далее.

Так как анализ внешнего языка является сложной самостоятельной задачей, то он не включён в разработанный SDK. На вход созданному на основе SDK инструменту должно подаваться дерево разбора внешнего языка с информацией, достаточной для решения поставленных в данной работе задач. 


\subsection{Архитектура YS.SEL.SDK}

Разработанный SDK включает компоненты, необходимые для реализации шагов, представленных на рисунке ~\ref{fig:SeqSelProcessing} и описанных ранее, за исключением анализа внешнего языка. Архитектура SDK изображена на рисунке~\ref{fig:SDKHLArch} и включает в себя генераторы анализаторов и различные библиотеки времени исполнения.

\begin{figure}[h!]
\begin{center}
\includegraphics[width=0.9\textwidth]{pics/HighLevelArch}
\caption{Архитектура SDK целиком}
\label{fig:SDKHLArch} 
\end{center}
\end{figure}

Так как анализ внешнего языка не входит в задачи разработанного SDK, то первый шаг, выполнение которого необходимо обеспечить, --- это построение аппроксимации. В нашем случае строится регулярное приближение множества значений динамически формируемого выражения.

Построение регулярной аппроксимации основано на алгоритме, изложенном в работе~\cite{RegOverApprox}, который позволяет строить приближение сверху для множества значений выражений. То есть $L_R$, задаваемый регулярным приближением, не меньше, чем $L_1$, задаваемый программой (выполняется включение $L_R \in L_1$). Это позволяет говорить о надёжности дальнейших анализов в том смысле, что они не теряют информации о $L_1$. Например, это важно при поиске ошибок. Если в $L_R$ не обнаружено ошибок (то есть $L_R \in L_2$, где $L_2$ --- эталонный), значит и в $L_1$ ошибок нет. При этом могут быть найдены ошибки в $L_R$, которых нет в $L_1$, то есть будут ложные срабатывания. Однако наличие ложных срабатываний лучше, чем пропущенные ошибки, и их количество может быть уменьшено путём повышения точности аппроксимации. 

Для того чтобы сделать построение приближения независимым от внешнего языка, реализовано обобщённое представление графа потока управления (Control Flow Graph, CFG)~\cite{Dragon}, которое содержит всю необходимую для дальнейшей работы информацию. Таким образом, разработчику необходимо реализовать построение обобщённого представления CFG для конкретного внешнего языка. В результате компонента строит конечный автомат, являющийся приближением множества значений динамически формируемых выражений.

Компонента, отвечающая за лексический анализ, состоит из двух частей: генератора лексических анализаторов, который по описанию лексики обрабатываемого языка строит соответствующий конечный преобразователь, и интерпретатора, который производит анализ входной структуры данных на основе построенного генератором преобразователя. Архитектура компоненты представлена на рисунке~\ref{fig:LexArch}. На вход принимается конечный автомат над символами, результатом работы является конечный автомат над алфавитом токенов анализируемого языка. Входной конечный автомат может быть построен с помощью компоненты построения регулярной аппроксимации. Основные структуры данных --- конечный автомат и конечный преобразователь --- и функции работы с ними описаны в соответствующей библиотеке.

\begin{figure}[h!]
\begin{center}
\includegraphics[width=0.95\textwidth]{pics/LexerDiagram}
\caption{Архитектура лексического анализатора}
\label{fig:LexArch} 
\end{center}
\end{figure}

Лексический анализатор реализован на основе инструмента FsLex, который является стандартным генератором лексических анализаторов для языка F\#. При реализации был переиспользован язык описания лексики и некоторые структуры данных.

Реализованный генератор лексических анализаторов обладает следующими особенностями.
\begin{itemize}
    \item Поддерживаются разрывные токены, то есть токены формируемые из нескольких строковых литералов.
    \item Сохраняется привязка лексических единиц к исходному коду: сохраняется информация о строковом литерале, из которого породился токен и координаты его внутри этой строки. Так как одна лексическая единица может формироваться из нескольких строковых литералов, то привязка сохраняется отдельно для каждой части.
    \item Поддерживается обработка входных конечных автоматов, содержащих циклы.
    \item Так как значение токена может формироваться с помощью цикла и, как следствие, быть бесконечным, то каждый токен содержит конечный автомат, порождающий все возможные значения для данного токена, а не единственное значение, как это реализовано в классическом лексическом анализе.
\end{itemize}

\textbf{Генератор синтаксических анализаторов}, названный ARNGLR, реализован на основе алгоритма, описанного в разделе~\ref{AlgoDescr}. Его архитектура представлена на рисунке~\ref{fig:ParsArch}.  

\begin{figure}[h!]
\begin{center}
\includegraphics[width=0.95\textwidth]{pics/ARNGLRArch}
\caption{Архитектура синтаксического анализатора}
\label{fig:ParsArch} 
\end{center}
\end{figure}

Генератор реализован как один из модулей YC и может принимать на вход внутреннее представление грамматики (IL), которое может быть получено с помощью различных фронтендов (Frontends), однако в рамках YC.SEL.SDK 
основным фронтендом является YARD, так как он предоставляет наиболее развитые средства для описания грамматик. По грамматике обрабатываемого языка строятся управляющие таблицы анализатора, которые сохраняются в файле Parser.fs. 
Построенные таблицы должны быть включены в разрабатываемое приложение UserApplication. Интерпретатор, предназначенный для синтаксического разбора конечного автомата, полученный после лексического анализа, реализован 
в виде отдельной библиотеки ARNGLR.Runtime, которая также должна быть подключена к разрабатываемому приложению. В результате работы интерпретатора будет получен SPPF, который может быть использован для дальнейшей 
обработки (например, подсчёта метрик). Для упрощения работы с SPPF реализован ряд вспомогательных функций.

\subsection{Архитектура YC.SEL.SDK.ReSharper}

ReSharper --- это расширение к Microsoft Visual Studio IDE, предоставляющее широкий спектр  дополнительной функциональности по анализу и рефакторингу кода. ReSharper поддерживает несколько языков, например C\#, Visual Basic .NET, JavaScript, и этот список может быть расширен благодаря наличию свободно распространяемого ReSharper SDK, описание которого было представлено ранее в разделе~\ref{ReSharperSDKDescr}. ReSharper.SDK позволяет получить деревья разбора для поддерживаемых языков, предоставляет набор готовых анализов и упрощает взаимодействие с Microsoft Visual Studio IDE и её компонентами. Более того, предоставляется возможность разработки собственных расширений для ReSharper на основе ReSharper.SDK.

Microsoft Visual Studio является достаточно распространённой средой разработки, но не поддерживает встроенные языки, поэтому было решено разработать ряд расширений к ReSharper с использованием разработанного инструментария, которые будут устранять данный недостаток. Стоит отметить, что не ставилось задачи поддержать все встроенные языки, так как встроенным может быть любой язык программирования. Также не было необходимости поддержать все внешние языки программирования. Необходимо на базе разработанного YC.SEL.SDK создать инфраструктуру, позволяющую реализовывать поддержку новых встроенных языков в Microsoft Visual Studio через расширения к ReSharper и реализовать несколько расширений, демонстрирующих возможности созданной инфраструктуры. 

Так как необходимо поддерживать различные языки, то необходимо обеспечить расширяемость по новыми языками. Классический подход к решению такой задачи для интегрированных сред разработки заключается в том, что поддержка нового языка реализуется в виде независимой компоненты. Если пользователь хочет получить поддержку какого-либо языка в своей среде разработки, то он должен установить соответствующий пакет. При этом поддержка различных языков осуществляется независимо, однако часто выделяется общая функциональность, которая может быть оформлена в виде отдельного пакета.

Для предоставления описанных выше возможностей была реализована надстройка над YC.SEL.SDK, упрощающая создание расширений для ReSharper, названная  YC.SEL.SDK.ReSharper. В неё включены компоненты, реализующие функции, которые упрощающают взаимодействие YC.SEL.SDK и ReSharper.SDK. Назначени основных из них описаны ниже.

\begin{itemize}
  \item Общая точка расширения, необходимая для подключения функциональности для различных встроенных языков, которая может быть реализована в различных расширениях, к ReSharper через общий интерфейс. Также общая точка расширения позволяет использовать общую функциональность, необходимую для работы со встроенными языками.
  \item Отображение в IDE информации, полученной в ходе анализа. Например, подсветка синтаксиса и ошибок. Вывод диагностических сообщений с информацией об ошибках.
  \item Преобразование данных, из формата, используемого в ReSharper.SDK, в формат для YC.SEL.SDK. Например, преобразование графа потока управления внешнего языка, построенного ReSharper.SDK, в формат, пригодный для построения регулярной аппроксимации средствами YC.SEL.SDK.
  \item Управление работой анализаторов, необходимое, с одной стороны, для обеспечения своевременной реакции на изменения в коде, совершённые пользователем, а с другой, для прекращения вычислений, результаты которых уже не актуальны. Управление построено на основе общего для ReSharper механизма, обеспечивающего асинхронную работу анализов. При этом вычисления могут быть прерваны, если, например, пользователь внёс в код изменения, делающие анализ или его результаты некорректными. 
\end{itemize}

Как уже говорилось, встроенными могут быть различные языки и учесть заранее все их особенности не представляется возможным. Кроме того, даже при использовании одного встроенного языка могут использоваться различные способы выполнения сформированного запроса. Таким образом, необходимо предоставлять возможность настройки расширений конечным пользователем. Для этого в рамках YC.SEL.SDK.ReSharper была реализована возможность управления следующими основными параметрами расширений. 

\begin{itemize}
    \item Подсветка синтаксиса для каждого языка. Предоставлена возможность указать цвет для каждого типа токена.
    \item Указание парных элементов. Для каждого языка можно указать, какие лексические единицы считать парными: для каждой пары указывается ``левый'' (открывающая скобка) и ``правый'' (закрывающая скобка) элементы. При расположении курсора в тексте рядом с одним из элементов пары будут подсвечены соответствующие элементы. Пример подсветки парных элементов приведён на рисунке~\ref{fig:braces}.
    \item Точки интереса или хотспоты (hotspot) --- это места, в которых должно быть сформировано финальное выражение. Необходимо знать, какой хотспот какому языку соответствует. При этом нужно учитывать, что одному языку может соответствовать несколько хотспотов. Например, динамически сформированный SQL-запрос в программе на языке программирования C\# может быть выполнен с помощью метода \verb|ExecuteQuery| класса \verb|DataContext|~\cite{ExecuteQuery}
     или же текст запроса может быть передан как аргумент конструктора класса \verb|SqlCommand|~\cite{SqlCommand} с последующим выполнением с помощью метода \verb|ExecuteReader|.

\end{itemize}

Настройка указанных выше параметров хранится в соответствующих конфигурационных файлах в формате XML, которые на данный момент редактируются вручную. Настройка подсветки синтаксиса и парных элементов совмещена в одном файле и для каждого языка создаётся отдельный такой файл. Конфигурационный файл с точками интереса является общим для всех языков и, соответственно, для всех установленных расширений для поддержки встроенных языков.

В листинге~\ref{lst:codeHighlighting} приведён пример конфигурации подсветки синтаксиса и парных скобок для языка Calc. Для указания цвета используются имена, принятые в ReSharper (например, \verb|"CONSTANT_IDENTIFIER_ATTRIBUTE"|), что должно сделать настройку цветов более единообразной. В xml-тэге \verb|Matched| содержится описание парных элементов. Каждая пара описывается в xml-тэге \verb|Pair| и для одного языка может быть указано более одной такой пары.

\fvset{frame=lines,framesep=5pt}
\begin{listing}[H]
    \begin{pyglist}[language=xml,numbers=left,numbersep=5pt]
<?xml version="1.0" encoding="utf-8"?>
<SyntaxDefinition name="CalcHighlighting">
    <Colors>
        <Tokens color="CONSTANT_IDENTIFIER_ATTRIBUTE">
            <Token> DIV </Token>
            <Token> LBRACE </Token>
            <Token> MINUS </Token>
            <Token> MULT </Token>
            <Token> NUMBER </Token>
            <Token> PLUS </Token>
            <Token> POW </Token>
            <Token> RBRACE </Token>
        </Tokens>
    </Colors>
<!-- Dynamic highlighting: -->
    <Matched>
        <Pair>
            <Left> LBRACE </Left>
            <Right> RBRACE </Right>
        </Pair>
<!-- You can specify more then one pair:        
        <Pair>
            <Left> LEFT_SQUARE_BRACKET </Left>
            <Right> RIGHT_SQUARE_BRACKET </Right>
        </Pair>
        <Pair>
            <Left> LEFT_FIGURE_BRACKET </Left>
            <Right> LEFT_FIGURE_BRACKET </Right>
        </Pair>
-->        
    </Matched>
</SyntaxDefinition>
    \end{pyglist}
\caption{Пример конфигурационного файла для настройки подсветки синтаксиса}
\label{lst:codeHighlighting}
\end{listing}

Листинг~\ref{lst:hotspots} содержит пример описания точек интереса. Для каждой точки интереса должна быть указана следующая информация.
\begin{itemize}
    \item Какому встроенному языку соответствует точка. Информация хранится в Xml-тэге \verb|Language|.
    \item Полное имя метода, являющегося точкой интереса. Информация хранится в Xml-тэге \verb|Method|.
    \item Порядковый номер аргумента данного метода, являющегося выражением на встроенном языке. Нумерация начинается с нуля. Информация хранится в Xml-тэге \verb|ArgumentPosition|. 
    \item Возвращаемый тип метода.  Информация хранится в Xml-тэге \verb|ReturnType|. 
\end{itemize}

\fvset{frame=lines,framesep=5pt}
\begin{listing}[H]
    \begin{pyglist}[language=xml,numbers=left,numbersep=5pt]
<?xml version="1.0" encoding="utf-8"?>
<!-- comment about body -->
<Body>
    <!-- comment about hotspot -->
  <Hotspot>
      <!-- comment about tsql -->
      <Language> TSQL </Language>
      <!-- comment about fullName -->
      <Method>Program.ExecuteImmediate</Method>
      <!-- zero-based -->
      <ArgumentPosition> 0 </ArgumentPosition>
      <!-- comment about return type -->
      <ReturnType> void </ReturnType>
  </Hotspot>
  <Hotspot>
      <Language> Calc </Language>
      <Method>Program.Eval</Method>
      <ArgumentPosition> 0 </ArgumentPosition>
      <ReturnType> int </ReturnType>
  </Hotspot>
</Body>
    \end{pyglist}
\caption{Пример конфигурационного файла для настройки точек интереса}
\label{lst:hotspots}
\end{listing}


\section{Области и способы применения YC.SEL.SDK}

Разработанный SDK предназначен для создания инструментов статического анализа динамически формируемых строковых выражений. Решения, созданные с его помощью, могут применяться для работы с проектами, активно использующими динамически формируемые строковые выражения. Необходимость работать с такими проектами может возникнуть, например, в следующих областях.

\begin{itemize}
    \item Реинжиниринг программного обеспечения.
    \item Поддержка встроенных языков в средах разработки.
    \item Оценка качества и сложности кода.
\end{itemize}

Общим для всех этих областей является то, что для решения многих задач необходимо структурное представление динамически формируемого кода. При этом анализируемые языки могут быть различными и процесс их анализа часто тесно связан с анализом внешнего языка.

Отметим, что встроенные языки используются всё менее активно в молодых проектах и системах. На смену им приходят более надёжные способы композиции языков и метапрограммирования. Например LINQ или ORM-технологии. Однако это не всегда так. Использование строковых выражений для взаимодействия с базами данных и генерации WEB-страниц в приложениях на PHP всё ещё широко распространено~\cite{DSQLInActiveUse}. Это необходимо учитывать при поддержке встроенных языков в средах разработки. Для каких-то языков на первый план выходят возможности по изучению и модификации уже созданного кода, а для каких-то --- возможность быстро и удобно создавать новый код. Во втором случае могут возникнуть дополнительные требования к скорости работы инструмента, так как подразумевается выполнение некоторых операций ``на лету'', что может послужить ограничением на использование SDK, так как многие механизмы, реализованные в нём, не предусматривают возможности уменьшения точности в пользу увеличения быстродействия. Оценка качества и сложности  кода часто может выполняться в рамках комплекса задач по реинжинирингу системы, однако может быть и самостоятельной задачей, например, при оценке сложности работ по поддержке и сопровождению информационной системы.

Детали применения SDK могут варьироваться в зависимости от решаемых задач и контекста использования. Например, механизм построения регулярной аппроксимации может быть реализован независимо в рамках внешнего инструмента. Однако основной сценарий использования аналогичен использованию инструментариев для разработки компиляторов. Последовательность шагов, представленная ниже, может быть изменена в зависимости от особенностей задачи.

\textbf{Шаг 1.} Создание грамматики обрабатываемого языка. Грамматика может быть создана на основе документации соответствующего языка или переиспользована готовая, что оправданно, например, при создании анализатора для динамического SQL, когда внешний и встроенный языки совпадают.

\textbf{Шаг 2.} Генерация синтаксического анализатора по созданной грамматике. Для этого используется генератор синтаксических анализаторов, присутствующий в SDK. Результатом работы генератора является файл с исходным кодом на языке программирования F\#, который должен быть включён в разрабатываемый код. Файл содержит описание типов для лексических единиц, управляющие таблицы анализатора и функцию, которая по конечному автомату над алфавитом токенов анализируемого языка построит SPPF, содержащий деревья вывода всех корректных цепочек.

\textbf{Шаг 3.} Создание лексической спецификации обрабатываемого языка. Спецификация может быть извлечена из документации или заимствована из других проектов. При обработке динамически формируемого SQL возможно переиспользовать спецификацию, созданную для основного языка, которым также является SQL. При этом необходимо обратить внимание на то, что типы лексических единиц определяются на основе созданного на предыдущих шагах синтаксического анализатора.

\textbf{Шаг 4.} Генерация лексера по созданной спецификации. Для этого применяется генератор лексических анализаторов, входящий в состав SDK. В результате его применения получается файл с исходным кодом на языке F\#, который должен быть подключён к разрабатываемому решению. 

\textbf{Шаг 5.} Реализация механизма построения регулярной аппроксимации, результатом которого является функция, строящая конечный автомат над алфавитом символов. Данный механизм может быть реализован либо на основе предоставляемого в рамках SDK, либо независимо. В первом случае от разработчика требуется построить обобщённый CFG для внешнего языка. Во втором случае необходимо только гарантировать правильность возвращаемого конечного автомата. Второй подход может быть использован, например, при наличии реализованного механизма протягивания констант для внешнего языка. Это позволит создать возможно менее точное, но, скорее всего, более быстрое построение аппроксимации. Такой подход применим при автоматизированном реинжиниринге, когда ручная доработка кода является обязательным шагом и абсолютная точность автоматической обработки не требуется. Ещё одна возможная область применения второго подхода --- это поддержка встроенных языков в средах разработки. Здесь также часто не требуется высокая точность для подсказок пользователю, однако производительность крайне важна. Поэтому иногда приходится жертвовать точностью анализа для достижения нужной скорости работы.

\textbf{Шаг 6.} Реализация работы с SPPF. Синтаксический анализатор возвращает SPPF --- конечное представление леса разбора всех корректных цепочек из аппроксимации. Дальнейшая работа с ним может строиться по двум основным сценариям.

Превый сценарий --- непосредственная обработка SPPF. В этом случае все вычисления происходят над SPPF без извлечения отдельных деревьев. Это позволит ускорить обработку результатов разбора, так как количество деревьев может быть бесконечным, а SPPF является конечной структурой данных. Однако существует несколько проблем, связанных с таким подходом. Во-первых, требуется создание новых процедур обработки, так как классические, как правило, ориентированы на работу с деревьями. Во-вторых, могут возникнуть трудности при выполнении некоторых анализов, вызванные тем, что в SPPF хранятся ``бесконечные'' деревья. Например, необходимо вычислить максимальную глубину вложенности конструкции \verb|if|, являющуюся одной из стандартных метрик сложности кода. SPPF может содержать 
циклы и может оказаться так, что конструкция \verb|if| встречается в цикле таким образом, что потенциальная глубина вложенности может быть бесконечной. Такая ситуация не является стандартной при 
обработке деревьев разбора и её надо отслеживать отдельно.

Второй сценарий --- извлечение отдельных деревьев из SPPF и их обработка. Данный подход может оказаться удобным, если уже существуют процедуры обработки синтаксических деревьев для языка, который оказался встроенным. Это помогает избежать затрат на создание новой функциональности. Такое может произойти при работе с динамическим SQL. В этом случае для работы с деревом разбора внешнего языка и деревьями, извлечёнными из SPPF, можно использовать одни и те же процедуры, так как языки идентичны.

Недостатком второго подхода является то, что конечность числа деревьев не гарантирована. Это значит, что не удастся обработать все деревья. Стоит отметить, что даже в случае конечности числа деревьев, перебор и обработка всех деревьев разбора может потребовать значительных ресурсов.


\textbf{Шаг 7.} Реализация механизмов сбора, обработки и отображения информации, такой как сообщения об ошибках или любой другой, полученной в процессе анализа. Необходимо для предоставления пользователю информации, ожидаемой в рамках решаемой задачи.

На рисунке~\ref{fig:activMethod} изображён один из возможных сценариев использования SDK. Особенностью является цикличность процесса, характерная, например для реинжиниринга программного обеспечения.

\begin{figure}[h!]
\begin{center}
\includegraphics[width=0.9\textwidth]{pics/ActivMethodology}
\caption{Один из возможных вариантов использования SDK в проектах по реинжинирингу}
\label{fig:activMethod} 
\end{center}
\end{figure}

Встраивание анализа строковых выражений в последовательность обработки кода всей системы зависит от решаемых задач. Первыми шагами идут действия, необходимые для того, чтобы получить входные данные для анализа. Для этого необходимо провести лексический и синтаксический анализ внешнего языка, построить граф потока управления. После этого возможно построение аппроксимации и дальнейший анализ встроенных языков. Параллельно с этим может проводиться дальнейшая обработка внешнего языка. Степень параллельности зависит от независимости решаемых задач. Например, некоторые метрики сложности для основного кода и для динамически формируемого можно считать независимо и выводить отдельно. С другой стороны, может возникнуть необходимость вычислить некую комплексную метрику, учитывающую параметры и внешнего и динамически формируемого кода, что приведёт к необходимости синхронизации.

\section{Особенности реализации}

Разработка инструментального пакета с описанной выше архитектурой и плагинов для ReSharper велась в рамках исследовательского проекта YaccConstructor (YC), описанного в разделе~\ref{YCDescr}. 
Разработка велась на платформе .NET и основной язык реализации --- F\#~\cite{FSharp}. Весь исходный код опубликован на GitHub~\cite{YCUrl}. 
Большинство компонент опубликовано под ``открытой'' лицензией Apache License Version~2.0~\cite{ApacheV2}. 

За основу алгоритма синтаксического анализа динамически формируемых выражений был взят реализованный в YC алгоритм синтаксического анализа RNGLR. Генератор управляющих таблиц был использован практически без 
модификаций, а интерпретатор был реализован отдельный. Кроме этого, общими являются некоторые структуры данных и вспомогательные функции, такие как представление леса разбора и его печать в формате 
DOT~\footnote{DOT --- текстовый язык описания графов~\cite{DOT}.}, представление GSS. 

Лексичекий анализ реализован на основе инструмента FsLex, который потребовал значительных доработок для того, чтобы обеспечить обработку конечного автомата, а не линейного входа. Все остальные компоненты, 
необходимые для статического анализа динамически формируемых выражений, такие как построение аппроксимации, вспомогательные функции для упрощения построения целевых инструментов были реализованы ``с нуля'' 
в рамках проекта YC.

Бинарные пакеты, содержащие основную функциональность, опубликованы в сети интернет на NuGet~\footnote{NuGet --- менеджер пакетов для платформы .NET и одноимённый ресурс для их публикации. Позволяет публиковать 
и устанавливать пакеты, автоматически отслеживать зависимости между ними~\cite{NuGet}.}.


\section{Эксперименты}
Для экспериментальных исследований были необходимы данные двух типов: последовательности РНК для подачи на вход синтаксическому анализатору и эталонные вторичные структуры для этих последовательностей --- и то, и другое было получено из популярной в исследовательских работах базы данных RNAstrand~\cite{andronescu2008rna}. Эта база представляет собой сборку тщательно отобранных и приведенных к единому формату данных сразу из нескольких надежных баз, содержащих цепочки РНК вместе с полученными методами лабораторного эксперимента или эволюционного анализа вторичными структурами. Из выгруженных данных были удалены дубликаты и образцы с неточностями в нуклеотидной цепи или же вторичной структуре, а также было выставлено ограничение на максимальную длину последовательности --- таким образом была получена выборка из 801 последовательности длин от 8 до 100, для которой были сгенерированы матрицы разбора и матрицы контактов, переведенные в черно-белые изображения. Для цепочки длины $n$ и входное, и эталонное изображения имеют размер $n \times n$, поэтому для корректной обработки изображений разного размера перед каждой эпохой обучения нейросети данные группировались по батчам, в каждом из которых присутствовали изображения только одного размера. Распределение длин последовательностей в итоговой выборке продемонстрировано на рис.~\ref{plot_distr}, при этом медианным значением является 44, а средним --- 47, что говорит о практически одинаковой представленности коротких и длинных цепочек среди исследуемых данных.

\begin{figure}[h]
\begin{center}
\centering
\includegraphics[width=16cm]{pics/plot_distr.png}
\caption{Распределение длин последовательностей РНК в выборке}
\label{plot_distr}
\end{center}
\end{figure} 

Для оценки качества работы обученных на данных изображениях нейронных сетей были выбраны следующие метрики, посчитанные относительно попиксельной разницы между предсказанным и эталонным изображениями. Далее $TP$ (true positive), $FP$ (false positive) и $FN$ (false negative), где под positive и negative  понимаются белые и черные пиксели изображений соответственно, --- информация о том, сколько раз нейронная сеть приняла верное и сколько раз неверное решение по каждому пикселю (кроме диагональных) каждого изображения тестовой выборки.
\begin{itemize} 
    \item $Precision = \frac{TP}{TP + FP}$ (доля предсказанных контактов, которые действительно являются контактами в эталонном изображении).
    \item $Recall = \frac{TP}{TP + FN}$ (доля найденных нейронной сетью контактов среди всех искомых).
    \item $F1 = 2 * \frac{Precision * Recall}{Precision + Recall}$ (гармоническое среднее $Precision$ и $Recall$, используется как удобная объединяющая метрика).
\end{itemize}

При обучении нейросети была использована функция потерь, в основе построения которой лежит идея о максимизации метрики $F1$ с несколькими уточнениями. Во-первых, $F1$ дискретна, а функция ошибки должна быть дифференцируема вследствие вычисления на ней градиента. Во-вторых, передача среднего по выборке значения $1 - F1$ в качестве функции ошибки не гарантирует отсутствие большого разброса $Precision$ и $Recall$ как в пределах отдельно взятого изображения, так и в масштабах всей выборки, следствием чего будет нестабильность качества работы модели и высокая вероятность появления очень низкой точности результата для случайно взятого тестового образца. На основании данных соображений была реализована функция $F1\_loss$, представленная на рис.~\ref{loss}. Здесь дифференцируемость обеспечивается заменой сумм дискретных целочисленных значений на непрерывную сумму значений вероятности, а поддержка баланса между $Precision$ и $Recall$ для каждого изображения и для выборки в целом --- двумя пропорциональными величине разброса штрафными коэффициентами $k1$ и $k2$, накладываемыми на метрику $F1$.

\begin{figure}[h]
\begin{center}
\centering
\begin{python}
from keras import backend as K

def f1_loss(y_true, y_pred):
    #normalize pixels values to [0, 1]
    y_true, y_pred = K.minimum(y_true / 255, 1), K.minimum(y_pred / 255, 1)
    #calculate differentiable versions of TW, FW and FB
    tw = K.sum(K.cast(y_true * y_pred, 'float32'), axis=[1, 2, 3])
    fw = K.sum(K.cast((1 - y_true) * y_pred, 'float32'), axis=[1, 2, 3])
    fb = K.sum(K.cast(y_true * (1 - y_pred), 'float32'), axis=[1, 2, 3])
    #calculate precision and recall secure from zero division error
    precision = tw / (tw + fw + K.epsilon())
    recall = tw / (tw + fb + K.epsilon())
    #penalty coefficients for huge difference between precision and recall 
    #calculated for each image and whole dataset respectively
    k1 = 1 -  K.abs(precision - recall)
    k2 = 1 -  K.abs(K.mean(precision) - K.mean(recall))
    #calculate upgraded f1 score
    f1 = k1 * k2 * 2 * precision * recall / (precision + recall + K.epsilon()) 
    return 1 - K.mean(f1)
\end{python}
\caption{Функция потерь нейронной сети}
\label{loss}
\end{center}
\end{figure} 

Вследствие того, что количество обучаемых параметром используемой модели является достаточно большим относительно размера обучающей выборки, после каждого остаточного блока был добавлен слой Dropout, исключающий заданный процент случайных нейронов во время обучения. Кроме того, во всех сверточных слоях была применена регуляризация L2, которая, помимо уменьшения переобучения нейросети, оказывает положительное влияние на процесс поиска сложных закономерностей в данных. В качестве оптимизатора был использован адаптивный градиентный спуск (Adagrad)~\cite{duchi2011adaptive}, удобный для работы с разреженными данными, а также автоматически настраивающий скорость обучения.

Для сравнения результатов работы обученной модели с существующими в области аналогами был проведен анализ различных инструментов, предсказывающих вторичную структуру РНК, по следующим критериям: заявленная высокая точность результатов, возможность предсказания псевдоузлов, удобство использования и адекватное время работы. На основании данных соображений были отобраны шесть инструментов, основанных на различных подходах.
\begin{itemize}
    \item HotKnots --- минимизации свободной энергии через эвристический алгоритм~\cite{ren2005hotknots}.
    \item SPOT-RNA --- глубокое обучение, основанное на технике transfer learning~\cite{singh2019rna}.
    \item PknotsRG --- минимизация свободной энергии с использованием Turner energy rules~\cite{reeder2007pknotsrg}.
    \item RNAstructure --- минимизация свободной энергии с помощью динамического программирования~\cite{bellaousov2013rnastructure}.
    \item Ipknot --- поиск оптимальной вторичной структуры методом целочисленного программирования~\cite{sato2011ipknot}.
    \item Knotty --- алгоритм для минимизация свободной энергии, основанный на разреженном динамическом программировании~\cite{jabbari2018knotty}.
\end{itemize}

\subsection{Результаты}
Все тестовые запуски проводились на рабочей станции со следующими характеристиками.
\begin{itemize}
    \item Операционная система: Ubuntu 20.04.2 LTS.
    \item Центральный процессор: Intel Core i5-10210U CPU 1.60GHz.
    \item Графический процессор: NVIDIA GeForce MX250.
    \item Объем оперативной памяти: 7.5 GB.
\end{itemize}

На рис.~\ref{plot_f1} представлены значения метрики $F1$, показанные шестью вышеописанными инструментами на всей выборке из 801 образца, а разработанной моделью (New-model) --- для различных разделений данных на обучающую и тестовую выборки (10\%:90\%, ..., 90\%:10\%). На графике видно, что при малых размерах обучающей выборки новая модель демонстрирует достаточно низкую точность, однако при увеличении выборки до 40\% результаты становятся сравнимыми с остальными подходами, а при максимальном объеме выборки (90\%) ---  лучшими в приведенном сравнении.

На рис.~\ref{plot_pr} показаны результаты аналогичного тестирования всех моделей по метрикам $Precision$ и $Recall$; здесь черная прямая $y=x$ символизирует оптимальное для рассматриваемой задачи положение этих метрик --- их равенство, --- а фиолетовая пунктирная линия указывает направление увеличения размера обучающей выборки для нашей модели от 10\% до 90\% с шагом в 10\%. Значения метрик для New-model расположены достаточно близко к желаемой прямой, что говорит о сбалансированности предсказаний разработанной нейросети. Кроме того, реализованный в данной работе алгоритм --- единственный на данном графике, имеющий $Recall$, больший, чем $Precision$: это произошло из-за того, что парсер находит значительную часть требуемых контактов, поэтому нейронная сеть, владея этой информацией еще до начала обучения, основной своей задачей имеет улучшение точности, а не полноты системы. Это делает наш подход несколько нетрадиционным относительно аналогов, которые, по всей видимости, сталкиваются с рядом проблем в процессе поиска контактов во вторичной структуре.

\begin{figure}[h]
\centering
\begin{subfigure}{.5\textwidth}
  \centering
  \fbox{\includegraphics[width=.95\linewidth]{pics/plot_f1.png}}
  \caption{Значения метрики $F1$}
  \label{plot_f1}
\end{subfigure}%
\begin{subfigure}{.5\textwidth}
  \centering
  \fbox{\includegraphics[width=.95\linewidth]{pics/plot_pr.png}}
  \caption{Значения метрик $Precision$ и $Recall$}
  \label{plot_pr}
\end{subfigure}
\caption{Сравнение разработанного подхода с аналогами}
\label{plot}
\end{figure}

Помимо точности, важной характеристикой алгоритма в области биоинформатики является время его работы, так как исследователям часто приходится работать с достаточно большими биологическими базами данных. В таблице~\ref{time} приведены замеры времени, потраченного всеми инструментами на обработку 100 цепочек РНК различных длин из рассматриваемого промежутка от 8 до 100. Несмотря на то, что разные подходы могут предполагать разные сценарии использования (обработка одной или нескольких последовательностей, вывод ответа через интерфейс командной строки или в специальный файл, а также сохранение результатов в различных форматах), одним из традиционных вариантов является обработка файла в формате fasta, содержащего набор последовательностей с метаданными, и последующее сохранение результата в одном из общепринятых форматов (например, dot-bracket или bpseq). Для данного сценария и был произведен сравнительный анализ производительности подходов: файл с последовательности был преобразован в необходимые для всех инструментов входные форматы, выходные же форматы были оставлены без изменений. В таблице~\ref{time} представлены средние значения для десяти прогонов в секундах, упорядоченные по возрастанию времени. Инструменты Ipknot, Hotknots, PknotsRG, RNAstructure и Knotty работают только на CPU, SPOT-RNA имеет и CPU, и GPU-реализации, а для нашего подхода как алгоритм синтаксического анализа (PA), так и нейронная сеть (NN) используют GPU. Можно увидеть, что New-model значительно проигрывает по времени большинству аналогов и наиболее времязатратной операцией здесь является синтаксический анализ, занимающий почти 80\% от общего времени работы.

\begin{table}[]
\centering
\caption{Time measurements for 100 sequences processing}
\begin{tabular}{|p{2cm}||p{2cm}|p{2cm}|p{2cm}|p{2cm}|}
\hline
\multirow{2}{*}{Step} & \multicolumn{2}{l|}{Vector based approach} & \multicolumn{2}{l|}{Image based approach} \\ \cline{2-5} 
 & \begin{tabular}[c]{@{}l@{}}Base \end{tabular} & \begin{tabular}[c]{@{}l@{}}Extended \end{tabular} & \begin{tabular}[c]{@{}l@{}}Base \end{tabular} & \begin{tabular}[c]{@{}l@{}}Extended \end{tabular} \\ \hline \hline
Parse & 307.6s & --- & 310.5s & --- \\ \hline
Load weights & 0.2s & 0.2s & 0.1s & 0.3s \\ \hline
Predict class & 0.2s & 0.2s & 0.2s & 0.3s \\ \hline
Total & 308.0s & 0.4s & 310.8s & 0.6s \\ \hline
\end{tabular}
\label{time}
\end{table}

Подводя итоги, экспериментальные исследования показали работоспособность разработанного подхода применительно к задаче предсказания вторичной структуры РНК даже в сравнении с лучшими инструментами в области. Высокая точность уже полученных результатов вместе с общей гибкостью подхода и обширными возможностями для дальнейших экспериментов позволяют полагать, что предложенные в данной работе идеи имеют значительный потенциал. Однако на данный момент наш проект по большей части исследовательский --- для создания полноценного инструмента требуется тщательный анализ качества всех обученных на различного размера выборках моделей с целью выбора оптимальной, а также, несомненно, повышение производительности подхода, в частности, ускорение синтаксического анализатора.

\section{Заключение}
\section{Conclusion and Future Work}

We present !!!

Our evaluation shows that !!!

First direction for future research is a more detailed CFPQ algorithms investigation.
We should do More evaluation on sparse matrices on GPGPUs.

Also it is nesessary to implement and evaluate solutions for graphs which is not fit in RAM.
There is a set of technics for huge matrices multiplication.
Is it possible to dopt it for CFPQ

Another direcion is a dataset improvement.
More data.
More grammars/queries.


% \nocite{*}
\setmonofont[Mapping=tex-text]{CMU Typewriter Text}
\bibliographystyle{ugost2008ls}
\bibliography{vkr}
\end{document}
