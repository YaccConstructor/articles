\documentclass{beamer}
\usepackage{beamerthemesplit}
\usepackage{wrapfig}
\usetheme{SPbGU}
\usepackage{pdfpages}
\usepackage{amsmath}
\usepackage{cmap} 
\usepackage[T2A]{fontenc} 
\usepackage[utf8]{inputenc}
\usepackage[english,russian]{babel}
\usepackage{indentfirst}
\usepackage{amsmath}
\usepackage{tikz}
\usepackage{multirow}
\usepackage[noend]{algpseudocode}
\usepackage{algorithm}
\usepackage{algorithmicx}
\usetikzlibrary{shapes,arrows}
\usepackage{fancyvrb}
%\usepackage{minted}
%\usepackage{verbments}


\newtheorem{rutheorem}{Теорема}
\newtheorem{ruproof}{Доказательство}
\newtheorem{rudefinition}{Определение}
\newtheorem{rulemma}{Лемма}
\beamertemplatenavigationsymbolsempty

\title[]{Синтаксический анализ для поиска в метагеномных сборках}
%\subtitle[YaccConstructor]{Использование вторичной структуры}
% То, что в квадратных скобках, отображается в левом нижнем углу. 
\institute[]{
Лаборатория языковых инструментов JetBrains \\
Санкт-Петербургский государственный университет \\
Математико-механический факультет }

% То, что в квадратных скобках, отображается в левом нижнем углу.
\author[Семён Григорьев]{Семён Григорьев}

\date{11 мая 2016г.}

\definecolor{orange}{RGB}{179,36,31}

\begin{document}
{
\begin{frame}[fragile]
  \begin{tabular}{p{2.5cm} p{5.5cm} p{2cm}}
   \begin{center}
      \includegraphics[width=2cm]{pictures/JBLogo3.pdf}
    \end{center}
    &
    \begin{center}
      \includegraphics[width=1.5cm]{pictures/SPbGU_Logo.png}
    \end{center}
    &
    \begin{center}
      \includegraphics[width=1cm]{pictures/YC_big.jpg}
    \end{center} 
  \end{tabular}
  \titlepage
\end{frame}
}

\begin{frame}[fragile]
  \transwipe[direction=90]
  \frametitle{YaccConstructor}
  \begin{itemize}
    \item Исследования в области лексического и синтаксического анализа
    \item Открытый исходный код
    \begin{itemize}
      \item \url{https://github.com/YaccConstructor}
    \end{itemize}
    \item Основной язык разработки --- F\#
  \end{itemize}
\end{frame}

\begin{frame}
  \transwipe[direction=90]
  \frametitle{Обзор}
  \begin{itemize}
    \item REAGO  --- HMM
    \item Xander --- HMM, 
    \item EMIRGE --- HMM
    \item Infernal --- CM, линейный вход 
  \end{itemize}
\end{frame}


\begin{frame}[fragile]
\transwipe[direction=90]
\frametitle{Вторичная структура РНК}
\begin{itemize}
    \item Несёт полезную информацию для идентификации подпоследовательностей
    \item Может быть задана с помощью граммтики
\end{itemize}
\begin{verbatim}
    stem<s> : 
       A stem<s> U | U stem<s> A 
     | G stem<s> C | C stem<s> G 
     | s

    folded: stem<A*[3..7]> 
\end{verbatim} 
\end{frame}

\begin{frame}[fragile]
  \transwipe[direction=90]
  \frametitle{Грамматики для описания вторичной структуры tRNA}
  \begin{itemize}
    \item Уточнение шаблона
    \begin{itemize}
       \item Возможности языков описания грамматики: метаправила, повторения
       \item Кыход за пределы КС-грамматик
    \end{itemize}
    \item Далее на грамматику можно ``навесить'' вероятности и т.д.
  \end{itemize}
\end{frame}

\begin{frame}[fragile]
  \transwipe[direction=90]
  \frametitle{Синтаксический анализ метагеномной сборки}
  \begin{itemize}
    \item Регулярное множество ($R$) = конечный автомат ($M$) = граф конечного автомата или просто 
    граф ($H$)
    \pause
    \item Грамматика $G$ задаёт для цепочек свойство выводимости: $S \Rightarrow^*_G \omega$
    \item Задача: найти все цепочки $\omega \in R : S \Rightarrow^*_G \omega$ 
    \item Метагеномная сборка представима в виде графа, который можно рассматривать как КА, в 
    котором все вершины стартовые
  \end{itemize}
\end{frame}

\begin{frame}[fragile]
  \transwipe[direction=90]
  \frametitle{Синтаксический анализ регулярных множеств}
  \begin{itemize}
    \item Состояние синтаксического анализатора обнозначно задаётся конечным набором данных --- 
    дескриптором
    \begin{itemize}
       \item $s \rightarrow A b \cdot c$ --- ``ситуация'', ``слот''
       \item Позиция во входе
       \item ...
    \end{itemize}    
    \item Имея дескриптор можно продолжить разбор с описанного им места
    \item Дескрипторы переиспользуются: дальнейший разбор не зависит от истории дескриптора 
    (\textbf{контекстно-свободная} грамматика)
  \end{itemize}
\end{frame}


\begin{frame}
  \transwipe[direction=90]
  \frametitle{Синтаксический анализ регулярных множеств: детали}
  \begin{itemize}
    \item Процесс анализа конечен    
    \item Внутренние структуры аналогичны структурам, используемым в обобщённом синтаксическом 
    анализе (GLR, GLL)
    \begin{itemize}
       \item Структурированный в виде графа стек (GSS)
       \item Сжатое представление леса разбора (SPPF)
       \item Управляющие таблицы
    \end{itemize}
    \item Полиномиальная сложность (?)
  \end{itemize}
\end{frame}

\begin{frame}[fragile]
  \transwipe[direction=90]
  \frametitle{Предлагаемый подход}
  \begin{itemize}
    \item \underline{\bfseries{$F^*$} (\url{https://www.fstar-lang.org/tutorial/})}
    \item Coq
    \item Agda
    \item ...
  \end{itemize}
\end{frame}

\begin{frame}
  \transwipe[direction=90]
  \frametitle{Текущие результаты}
  \begin{itemize}
    \item $F\# + F^* = F\#^*$
    \item Парсер для $F\#^*$
    \item Транслятор из AST $F\#$ в AST $F^*$
    \item Разбивается на подзадачи (до 3 человек)
    \item Диплом (вся задача), публикации
  \end{itemize}
\end{frame}

\begin{frame}
  \transwipe[direction=90]
  \frametitle{Планы}
  \begin{itemize}
    \item Поддержать в модели проекта, редакторе, отладчике
      \begin{itemize}
        \item Создание файлов, шаблоны
        \item Подсветка синтаксиса
        \item Сообщения об ошибках, подсветка ошибок
        \item ....
      \end{itemize}
    \item Разбивается на подзадачи (до 4 человек)
    \item Диплом (вся задача), публикации
  \end{itemize}
\end{frame}
          
\begin{frame}
\transwipe[direction=90]
\frametitle{Контакты}
\begin{itemize}
  \item Почта: \url{rsdpisuy@gmail.com}
  \item Исходный код YaccConstructor: \url{https://github.com/YaccConstructor}
  \item Google+ сообщество: \url{https://goo.gl/DuPWkM}
\end{itemize}
\end{frame}
\end{document}
