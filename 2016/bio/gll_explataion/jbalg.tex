\documentclass[12pt]{article}  % standard LaTeX, 12 point type
\usepackage{amsfonts,latexsym}
\usepackage{amsthm}
\usepackage{amssymb}
\usepackage[utf8]{inputenc} % Кодировка
\usepackage[english,russian]{babel} % Многоязычность

\newtheorem{theorem}{Theorem}[section]
\newtheorem{proposition}[theorem]{Proposition}
\newtheorem{lemma}[theorem]{Lemma}
\newtheorem{corollary}[theorem]{Corollary}
\newtheorem{conjecture}[theorem]{Conjecture}

\theoremstyle{definition}
\newtheorem{definition}{Определение}[section]
\newtheorem{example}{Example}[section]

% unnumbered environments:

\theoremstyle{remark}
\newtheorem*{remark}{Remark}
\newtheorem*{notation}{Notation}
\newtheorem*{note}{Note}

\setlength{\parskip}{5pt plus 2pt minus 1pt}
%\setlength{\parindent}{0pt}

\usepackage{color}
\usepackage{listings}
\usepackage{caption}
\usepackage{graphicx}
\usepackage{ucs}

\graphicspath{{pics/}}


\lstnewenvironment{algorithm}[1][]
{   
    \lstset{ 
        frame=tB,
        numbers=left, 
        mathescape=true,
        numberstyle=\small,
        basicstyle=\small, 
        inputencoding=utf8, 
        extendedchars=\true,
        keywordstyle=\color{black}\bfseries,
        keywords={,function, procedure, return, datatype, function, in, if, else, for, foreach, while, denote, do, and, then, assert,} 
        numbers=left,
        xleftmargin=.04\textwidth,
        #1 % this is to add specific settings to an usage of this environment (for instnce, the caption and referable label)
    }
}
{}

\newcommand{\tab}[1][0.3cm]{\ensuremath{\hspace*{#1}}}





\title{Modification of modificated CYK}
\author{Semyon Grogorev}
\date{\today}

\begin{document}


Выполняется обход грамматики, согласованный со входной строкой --- есть два указателя: в грамматику и позицию во воходе.
Указатель в грамматику --- слот: $X \rightarrow x_0 .. x_k \cdot x_{k+1} .. x_n$

Возможны несколько ситуаций. Предположим, что слот выглядит одним из следующих способов: $X \rightarrow \alpha \cdot x \beta$ или $X \rightarrow \alpha \cdot$. Указатель во входе находится на позиции $i$.
\begin{enumerate}
\item $x = \omega[i+1]$. Указатель в грамматике перемещается в позицию $X \rightarrow \alpha x \cdot \beta$ и указатель во входе сдвигается в позицию $i+1$.
\item $x$ --- это нетерминал $A$. Запоминаем точку возврата: кладём на стек слот $X \rightarrow \alpha x \cdot \beta$. Сдвигаем указатель в грамматике в позицию $A \rightarrow \cdot \gamma$. Здесь можно воспользоваться FIRST и FOLLOW для детерминизации выбора. 
      Позиция во входе неизменна.
\item Указатель в грамматике в позиции $X \rightarrow \alpha \cdot$ и стек непуст. На вершине стека должен быть "адрес возврата" \  вида $Y \rightarrow \delta X \cdot \mu$. 
      Он извлекается из стека и становится текущим указателем в грамматике. 
\item Указатель в грамматике в позиции $S \rightarrow \alpha \cdot$ ($S$ --- стартоый нетерминал) и стек пуст. Успешное завершение разбора. Иначе провал. 
\end{enumerate}

На шаге 2 можно не продожать сразу все возможные варианты, а создать дескриптор --- сущность, позволяющую возобновить обход с места, которое он описывает.
Для каждого полученного слота вида $A \rightarrow \cdot \gamma$ создаём дескриптор, содержащий его, указатель на вход в позицию $i$ и указатель на стек, на адрес возврата, который только что был добавлен ($X \rightarrow \alpha A \cdot \beta$).
Для организации стека можно использовать GSS, а дескриптор будет указывыать на вершину в нём.

Возможны проблемы с бесконечным количеством дескрипторов и "потерей" \ точек возврата.

\begin{itemize}
\item $R$ --- мн-во дескрипторов для обработки
\item $U$ --- мн-во созданных дескрипторов
\item $P$ --- мн-во тех, для кого надо не забыть сделать $pop$
\end{itemize}

\begin{verbatim}
let _add L v i = 
    if (L, v, i) not in U 
    then 
      add (L, v, i) to U
      add (L, v, i) to R

let _pop v i =
    if v <> v_0
    then
      add (v,i) to P
      for each u in v.child do _add v.L u i

let _create L v i =
    if (L,i) not in GSS then add (L,i) to GSS
    let u = GSS.get (L,i)
    if there is not an edge from u to v
    then
      add edge from u to v
      for all (u,k) in P do _add L v k
    u
\end{verbatim}

Нам нужен табличный вариант.


\begin{verbatim}
let rec dispatcher () = 
  if R.Count <> 0 
  then 
    (L,v,i) := R.Get() 
    dispatch := false 
  else 
    stop := true 

and processing () =  
  dispatch := true 
  match L with
  | (X -> a . x b where x = input[i + 1]) ->
     i := i + 1
     L := (X -> a x . b)
     dispatch := false 
  | (X -> a . x b where x is nonterminal) ->
     v := _create (X -> a x . b) v i
     let slots = pTable[x][input[i]]  
     for L in slots do             
         _add L v i
  | (X -> a .) -> _pop v i
  | _ -> final result processing and error notification

let control () = 
    while not !stop do 
       if !dispatch then dispatcher() else processing() 

control() 

\end{verbatim}


Если граф детерминированный, то кроме использования в качестве позиции вершины не требуется иных модификаций, так как в первом случае ровно один вариант совпадения входного символа с ожидаемым в правиле.


Лес разбора ($L$ --- слот, $i,j,k$ --- координаты).
\begin{itemize}
\item Терминальные узлы $(T,i,j)$
\item Нетерминальные узлы $(N,i,j)$. Сыновья --- запакованные узлы вида $N \rightarrow \gamma \cdot , k$
\item Промежуточные узлы $(L,i,j)$. Сыновья --- запакованные узлы с меткой $L, k$, где $ i \leq k \leq j$ 
\item Запакованные узлы $(L,k)$. Один или два сына. Правый --- терминал или нетерминал с меткой $(S,k,j)$. Левый (если есть) --- терминал, нетерминал или промежуточный с меткой $(S, j, k)$ 
\end{itemize}

Теперь в дескрипторе надо запоминать ссылку на узел дерева: $(L, v, i, a)$
Дескрипторы с непустой ссылкой создаются в момент вызова $_pop$. Правый сын создаваемого узла --- только что построенный узел. Левый сын хранится на ребре, исходящем из вершины, которую только что достали со стека.
Метка нового узла --- адрес возврата, полученный из достанного узла.

Теперь функции должны работать ещё и с узлами леса.

\begin{verbatim}
let _add L v i a = 
    if (L, v, i, a) not in U 
    then 
      add (L, v, i, a) to U
      add (L, v, i, a) to R

let _pop v i z =
    if v <> v_0
    then
      add (v,z) to P
      for each edge (a, u) in v.outEdges do 
         let y = getNodeP v.L a z
         _add v.L u i y

let _create L v i a =
    if (L,i) not in GSS then add (L,i) to GSS
    let u = GSS.get (L,i)
    if there is not an edge from u to v labelled a
    then
      add edge from u to v labelled a
      for all (u,z) in P do 
         let y = getNodeP L a z
         let (_,_,k) = z
         _add L v k y
    u
\end{verbatim}

\begin{verbatim}
let getNodeT x i = 
    let h = if x = \eps then i else i + 1
    if there is no SPPF node with label (x,i,h)
    then create node with label (x,i,h) 
    return node with label (x,i,h) 

let getNodeP (X -> w1 . w2) a z =
    if w1 is terminal or non-nullable nonterminal and w2 <> \eps
    then return z
    else 
      let t = if w2 = \eps then X else (X -> w1 . w2)
      let (q,k,i) = z.lbl
      if a <> dummy
      then
       let (s,j,k) = a.lbl
       let y = find_or_create SPPF.nodes (n.lbl = (t,i,j))
       if y does not have a child with label (X -> w1 . w2)
       then
         let y' = new_packed_node(a,z)
         y.chld.add y'
       return y
      else
        let y = find_or_create SPPF.nodes (n.lbl = (t,k,i))   
        if y does not have a child with label (X -> w1 . w2)
        then
          let y' = new_packed_node(z)
          y.chld.add y'
        return y

\end{verbatim}


\begin{verbatim}
let rec dispatcher () = 
  if R.Count <> 0 
  then 
    (L,v,i,cN) := R.Get() 
    cR := dummy
    dispatch := false 
  else 
    stop := true 

and processing () =  
  dispatch := true 
  match L with
  | (X -> a . x b where x = input[i + 1]) ->
     if cN = dummyAST 
     then cN := getNodeT i
     else cR := getNodeT i
     i := i + 1
     L := (X -> a x . b)
     if !cR <> dummy
     then cN := getNodeP L cN cR 
     dispatch := false 
  | (X -> a . x b where x is nonterminal) ->
     v := _create (X -> a x . b) v i cN
     let slots = pTable[x][input[i]]  
     for L in slots do             
         _add L v i dummy
  | (X -> a .) -> 
     _pop v i cN
  | _ -> final result processing and error notification

let control () = 
    while not !stop do 
       if !dispatch then dispatcher() else processing() 

control() 

\end{verbatim}


\end{document}