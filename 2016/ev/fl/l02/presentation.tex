\documentclass{beamer}
\usepackage{beamerthemesplit}
\usepackage{wrapfig}
\usetheme{SPbGU}
\usepackage{pdfpages}
\usepackage{amsmath}
\usepackage{mathtools}
\usepackage{cmap} 
\usepackage[T2A]{fontenc} 
\usepackage[utf8]{inputenc}
\usepackage[english,russian]{babel}
\usepackage{indentfirst}
\usepackage{amsmath}
\usepackage{tikz}
\usepackage{multirow}
\usepackage[noend]{algpseudocode}
\usepackage{algorithm}
\usepackage{algorithmicx}
\usepackage{ stmaryrd }
\usetikzlibrary{shapes,arrows}
\usepackage{fancyvrb}
\newtheorem{rutheorem}{Теорема}
\newtheorem{ruproof}{Доказательство}
\newtheorem{rudefinition}{Определение}
\newtheorem{rulemma}{Лемма}
\beamertemplatenavigationsymbolsempty

\title[]{Теория автоматов и формальных языков}
\subtitle[]{Регулярные языки}
\institute[]{
Санкт-Петербургский государственный электротехнический университет <<ЛЭТИ>>\\
}

\author[]{Екатерина Вербицкая}

\date{13 сентября 2016г.}

\definecolor{orange}{RGB}{179,36,31}

\begin{document}
{
  \begin{frame}
    \titlepage
  \end{frame}
}

\begin{frame}[fragile]
  \transwipe[direction=90]
  \frametitle{В предыдущей серии}
  \begin{itemize}
    \item Формальные языки повсюду. Язык --- множество строк над алфавитом
    \item Существует множество способов описать язык
    \item Задачи теории формальных языков
    \begin{itemize}
      \item Как представить язык?
      \item Какие есть характеристики у разных представлений языка?
      \item Как определить, принадлежит ли строка данному языку?
    \end{itemize}
  \end{itemize}
\end{frame}

\begin{frame}[fragile]
  \transwipe[direction=90]
  \frametitle{В предыдущей серии}
  \begin{itemize}
    \item Формальная грамматика 
    \begin{itemize}
      \item $\langle$Терминалы, Нетерминалы, Правила, Стартовый нетерминал$\rangle$
      \item Однозначность (любая строка имеет единственный вывод) и неоднозначность
    \end{itemize}
   \item Вывод: транзитивное и рефлексивное замыкание отношения выводимости
   \begin{itemize}
     \item Левосторонний (на каждом шаге заменяем самый левый нетерминал) и правосторонний
   \end{itemize}
   \item Дерево вывода
   \begin{itemize}
     \item Дерево: листья соответствуют терминалам, внутренние вершины --- нетерминалам; для каждого внутреннего узла существует правило грамматики, правая часть которого совпадает с метками детей узла
   \end{itemize}
    \item  Контекстно-свободная грамматика
      \begin{itemize}
        \item все правила имеют вид $A \rightarrow \alpha$
        
      \end{itemize}

      \end{itemize}

\end{frame}


\begin{frame}[fragile]
  \transwipe[direction=90]
  \frametitle{Конечный автомат}
  \textbf{Конечный автомат} --- $\langle Q, \Sigma, \delta, q_0, F \rangle$
  \begin{itemize}
    \item $Q \neq \emptyset$ --- конечное множество состояний
    \item $\Sigma$ --- Конечный входной алфавит
    \item $\delta$ --- отображение типа $Q \times \Sigma \rightarrow Q$
    \item $q_0 \in Q$ --- начальное состояние
    \item $F \subseteq Q$ --- множество конечных состояний
  \end{itemize}

  КА называется \textbf{полным}, если существует переход из каждого состояния по каждому символу алфавита
\begin{itemize}
    \item Обычно добавляют "дьявольскую" вершину, она же сток. 
\end{itemize}
\end{frame}


\begin{frame}[fragile]
  \transwipe[direction=90]
  \frametitle{Пример конечного автомата}

 $Q = \{ 0, 1, 2, 3\}, \Sigma = \{ 0, 1, -\}, q_0 = 0, F = \{1, 2\}$
 
 $\delta (0, 0) = 1; \delta (0, 1) = 2; \delta (0, -) = 3; \delta (3, 1) = 2; \delta (2, 0) = 2; \delta (2, 1) = 2 $ 

  \begin{center}
     \includegraphics[width=0.55\textwidth]{pics/automaton.png}  
   \end{center}
\end{frame}

\begin{frame}[fragile]
  \transwipe[direction=90]
  \frametitle{Пример полного конечного автомата}
  \begin{center}
     \includegraphics[width=0.55\textwidth]{pics/FA_DN.PNG}  
   \end{center}
\end{frame}

\begin{frame}[fragile]
  \transwipe[direction=90]
  \frametitle{Путь в конечном автомате}
  \begin{itemize}
    \item \textbf{Путь} --- кортеж $\langle q_0, e_1, q_1, \dots, e_n, q_n\rangle$
    \begin{itemize}
      \item $n \geq 0$
      \item $\forall i. e_i = \langle q_{i-1}, w_i, q_i\rangle \in \delta$
      \item $q_0$ --- \textbf{начало} пути
      \item $q_n$ --- \textbf{конец} пути
      \item $w_1, w_2, \dots, w_n$ --- \textbf{метка} пути
      \item $n$ --- \textbf{длина} пути
    \end{itemize}
    \item Путь \textbf{успешен}, если $q_0$ --- начальное состояние, а $q_n \in F$
    \item Состояние $q$ \textbf{достижимо} из состояния $p$, если существует путь из состояния $p$ в состояние $q$
  \end{itemize}
\end{frame}

\begin{frame}[fragile]
  \transwipe[direction=90]
  \frametitle{Пример пути}

  $\langle 0, \langle 0, '-', 3 \rangle, 3, \langle 3, '1', 2 \rangle, 2, \langle 2, '1', 2 \rangle, 2, \langle 2, '0', 2 \rangle, 2\rangle$ --- успешный путь с меткой ``-110'' длины 4

  \begin{center}
     \includegraphics[width=0.55\textwidth]{pics/automaton.png}  
   \end{center}
\end{frame}

\begin{frame}[fragile]
  \transwipe[direction=90]
  \frametitle{Такт работы КА (шаг)}
  \begin{itemize}
    \item \textbf{Конфигурация (Мгновенное описание)} КА --- $\langle q, \omega \rangle$, где $q \in Q, \omega \in \Sigma^*$
    \item \textbf{Такт работы} --- бинарное отношение $\vdash$: если $\langle p , x , q \rangle \in \Delta$ и $\omega \in \Sigma ^*$, то $\langle p , x \omega \rangle \vdash \langle q , \omega \rangle$
    \item Бинарное отношение $\vdash^*$ --- рефлексивное, транзитивное замыкание $\vdash$
  \end{itemize}
\end{frame}

\begin{frame}[fragile]
  \transwipe[direction=90]
  \frametitle{Распознавание слова конечным автоматом}
  \begin{itemize}
    \item Цепочка $\omega$ \textbf{распознается} КА, если $\exists$ успешный путь с меткой $\omega$

    \item \textbf{Язык, распознаваемый конечным автоматом}: \\ $\{ \omega \in \Sigma^* \, | \, \exists p$ --- успешный путь с меткой $\omega \}$
  \end{itemize}
\end{frame}


\begin{frame}[fragile]
  \transwipe[direction=90]
  \frametitle{Распознавание слова конечным автоматом}
  \begin{rutheorem}[]
   Рассмотрим конечный автомат  $M = \langle Q , \Sigma , \delta , q_0 , F \rangle$. 
   Слово $\omega \in \Sigma ^*$ принадлежит языку $L(M) \Leftrightarrow \exists q \in F.\langle q_0 , \omega \rangle \vdash^* \langle q , \varepsilon \rangle$.
  \end{rutheorem}  
\end{frame} 
  
\begin{frame}[fragile]
  \transwipe[direction=90]
  \frametitle{Распознавание слова конечным автоматом}
   Обобщаем функцию перехода:
 
      \begin{itemize}
        \item $\delta' (q, \varepsilon) = q$
        \item $\delta' (q, xa) = \delta(\delta'(q, x), a)$, где $x \in \Sigma^*, a \in \Sigma$
      \end{itemize}

  \begin{rutheorem}[]   
     Цепочка $\omega$ \textbf{распознается} КА $\langle Q, \Sigma, \delta, q_0, F \rangle \Leftrightarrow \exists p \in F. \delta'(q, \omega) = p$
  \end{rutheorem}  
   
   \textbf{Язык, распознаваемый конечным автоматом}: $\{ \omega \in \Sigma^* \, | \, \exists p \in F.\delta'(q_0, \omega) = p \}$
\end{frame}


\begin{frame}[fragile]
  \transwipe[direction=90]
  \frametitle{Свойство конкатенации строк}
  \begin{rutheorem}[]   
    $\langle q_1 , \alpha \rangle \vdash^* \langle q_2 , \varepsilon \rangle, \langle q_2 , \beta \rangle \vdash^* \langle q_3 , \varepsilon \rangle \Rightarrow \langle q_1 , \alpha \beta \rangle \vdash^* \langle q_3 , \varepsilon \rangle$
  \end{rutheorem}
\end{frame}


\begin{frame}[fragile]
  \transwipe[direction=90]
  \frametitle{Эквивалентность конечных автоматов}
  \begin{itemize}
    \item Конечные автоматы $A_1$ и $A_2$ \textbf{эквивалентны}, если распознают один и тот же язык
    \item Как проверить что автоматы эквиваленты?
  \end{itemize}
\end{frame}

\begin{frame}[fragile]
  \transwipe[direction=90]
  \frametitle{Проверка на эквивалентность автоматов}
  \begin{itemize}
    \item Запустить одновременный обход в ширину двух автоматов
    \item Каждый переход должен приводить в терминальные или нетерминальные вершины в обоих автоматах соответственно
  \end{itemize}
\end{frame}

\begin{frame}[fragile]
  \transwipe[direction=90]
  \frametitle{Минимальный конечный автомат}
  \begin{itemize}
    \item \textbf{Минимальный конечный автомат} --- автомат, имеющий наименьшее число состояний, распознающий тот же язык, что и данный
  \end{itemize}
\end{frame}

\begin{frame}[fragile]
  \transwipe[direction=90]
  \frametitle{Классы эквивалентности}
    \textbf{Отношение эквивалентности} --- рефлексивное, симметричное, транзитивное отношение
    \begin{itemize}
      \item $xRx; xRy \Leftrightarrow yRx; xRy, yRz \Rightarrow xRz$
    \end{itemize}
    
    \begin{rutheorem}
       $\forall R$ --- отношение эквивалентности на множестве $S$
      
      Можно разбить $S$ на $k$ непересекающихся подмножеств $I_1 \dots I_k$, т.ч. $aRb \Leftrightarrow a, b \in I_j$
    \end{rutheorem}
    
    Множества $I_1 \dots I_k$ называются \textbf{классами эквивалентности}
\end{frame}

\begin{frame}[fragile]
  \transwipe[direction=90]
  \frametitle{Эквивалентные состояния}
  \begin{itemize}
    \item $\omega \in \Sigma^*$ различает состояния $q_i$ и $q_j$, если $\delta' (q_i, \omega) = t_1, \delta' (q_j, \omega) = t_2 \Rightarrow (t_1 \notin F \Leftrightarrow t_2 \in F)$
    \item $q_i$ и $q_j$ \textbf{эквивалентны} $(q_i \sim q_j)$, если $\forall \omega \in \Sigma^*. \delta' (q_i, \omega) = t_1, \delta' (q_j, \omega) = t_2 \Rightarrow (t_1 \in F \Leftrightarrow t_2 \in F)$
    \begin{itemize}
      \item Является отношением эквивалентности
    \end{itemize}
  \end{itemize}
   \begin{rulemma}
      $\mathcal{A} = \langle Q, \Sigma, \delta, q_0, F \rangle, p_1, p_2, q_1, q_2 \in Q, q_i = \delta(p_i, c)$
      
      $\omega \in \Sigma^*$ различает $q_1$ и $q_2$. Тогда $c \omega$ различает $p_1$ и $p_2$
   \end{rulemma}
   
   \begin{ruproof}
     $\delta' (p_i, c \omega) = \delta' (\delta (p_i, c), \omega) = \delta' (q_i, \omega) = t_i$
   \end{ruproof}
\end{frame}

\begin{frame}[fragile]
  \transwipe[direction=90]
  \frametitle{Алгоритм минимизации КА}
    TLDR: разбиваем состояния на классы эквивалентности, которые делаем новыми состояниями

\end{frame}

\begin{frame}[fragile]
  \transwipe[direction=90]
  \frametitle{Алгоритм минимизации КА}
   $Q$ --- очередь; $marked$ --- таблица размером $n \times n$ (n --- количество состояний КА).
   
   ~\\
    Помечаем в таблице пары неэквивалентных состояний и кладем их в очередь
\end{frame}
      
\begin{frame}[fragile]
  \transwipe[direction=90]
  \frametitle{Алгоритм минимизации КА}   
    \begin{itemize}
      \item Если автомат не полный --- дополнить дьявольской вершиной
      \item Строим отображение $\delta^{-1}$ --- обратные ребра
      \item Находим все достижимые из стартового состояния
      \item Добавляем в $Q$ и отмечаем в $marked$ пары состояний, различимые $\varepsilon$
      \item Можем пометить пару $(u, v)$, если $\exists c \in \Sigma. (\delta(u, c), \delta(v, c)) помечена$. Для этого, пока $Q \neq \emptyset$:
      \begin{itemize}
        \item Извлекаем $(u, v)$ из $Q$
        \item $\forall c \in \Sigma$ перебираем $(\delta^{-1}(u, c), \delta^{-1}(v, c))$ --- если пара не помечена, помечаем и кладем в очередь
      \end{itemize}
      \item В момент опустошения $Q$ непомеченные пары являются эквивалентными
      \item За проход по таблице выделяем классы эквивалентности
      \item За проход по таблице формируем новые состояния и переходы	
    \end{itemize}
    
\end{frame}

\begin{frame}[fragile]
  \transwipe[direction=90]
  \frametitle{Алгоритм минимизации КА}   
    \begin{itemize}
      \item Стартовое состояние --- класс эквивалентности, которому принадлежит стартовое состояние исходного КА
      \item Конечные состояния --- классы эквивалентности, которым принадлежат конечные состояния исходного КА
    \end{itemize}
    
\end{frame}

\begin{frame}[fragile]
  \transwipe[direction=90]
  \frametitle{Алгоритм минимизации КА: корректность}   
    \begin{itemize}
      \item Пусть в результате применения алгоритма к КА $A$ получили КА $A_{min}$. Покажем, что этот автомат минимальный и единственный с точностью до изоморфизма
      \item Пусть $\exists A'. A'$ и $A$ эквивалентны, но количество состояний $A'$ меньше, чем у $A_{min}$
      \item Стартовые состояния $s \in A_{min}$ и $s' \in A'$ эквивалентны (КА допускают один язык)
      \item $\sphericalangle \alpha = a_1 a_2 \dots a_k, a_i \in \Sigma: \langle s, \alpha \rangle \vdash^* \langle u, \varepsilon \rangle; \langle s', \alpha \rangle \vdash^* \langle u', \varepsilon \rangle$
      \item $\sphericalangle \langle s, a_1 \rangle \vdash^* \langle l, \varepsilon \rangle; \langle s', a_1 \rangle \vdash^* \langle l', \varepsilon \rangle$. $s, s'$ эквивалентны $\Rightarrow l, l'$ эквивалентны
      \item Аналогично для всех $a_i \Mapsto u, u'$ эквивалентны 
      \item $\Mapsto \forall q$ --- состояние $A_{min} \exists q'$ --- эквивалентное состояние $A'$
      \item Состояний $A'$ меньше, чем состояний $A_{min} \Rightarrow$ 2 состояниям $A_{min}$ соответствует 1 состояние $A' \Rightarrow$ они эквивалентны. Но по построению $A_{min}$ в нем не может быть эквивалентных состояний. Противоречие
    \end{itemize}
    
\end{frame}

\begin{frame}[fragile]
  \transwipe[direction=90]
  \frametitle{Недетерминированный КА}   
 \textbf{Недетерминированный конечный автомат} --- $\langle Q, \Sigma, \delta, q_0, F \rangle$
  \begin{itemize}
    \item $Q \neq \emptyset$ --- конечное множество состояний
    \item $\Sigma$ --- Конечный входной алфавит
    \item $\delta$ --- отображение типа $Q \times \Sigma \rightarrow 2^Q$
    \item $q_0 \in Q$ --- начальное состояние
    \item $F \subseteq Q$ --- множество конечных состояний
  \end{itemize}
    
\end{frame}

\begin{frame}[fragile]
  \transwipe[direction=90]
  \frametitle{Распознавание слова НКА}
  \begin{itemize}
    \item \textbf{Конфигурация (Мгновенное описание)} КА --- $\langle q, \omega \rangle$, где $q \in Q, \omega \in \Sigma^*$
    \item \textbf{Такт работы} --- бинарное отношение $\vdash$: если $q \in \delta(p, x)$ и $\omega \in \Sigma ^*$, то $\langle p , x \omega \rangle \vdash \langle q , \omega \rangle$
    \item Бинарное отношение $\vdash^*$ --- рефлексивное, транзитивное замыкание $\vdash$
    \item НКА \textbf{допускает} слово $\alpha$, если $\exists t \in F. \langle s, \alpha \rangle \vdash^* \langle t, \varepsilon \rangle$
    \item \textbf{Язык НКА} $L(A) = \{ \omega \in \Sigma^* \, | \, \exists t \in F. \langle s, \omega \rangle \vdash^* \langle t, \varepsilon \rangle \}$ 
  \end{itemize}
  \begin{itemize}
    \item ДКА --- частный случай НКА
  \end{itemize}
\end{frame}

\begin{frame}[fragile]
  \transwipe[direction=90]
  \frametitle{Алгоритм, определяющий допустимость слова}
  \begin{itemize}
    \item $R(\alpha) = \{ p | \langle s, \alpha \rangle \vdash^* \langle p, \varepsilon \rangle\}$
    \item $R(\varepsilon) = \{q_0\}$
    \item $R(\alpha c) = \{q | q \in \delta (p, c), p \in R(\alpha)\}$
    \item НКА допускает слово $\alpha \Leftrightarrow \exists t \in F. t \in R(\alpha) $
  \end{itemize}
\end{frame}

\begin{frame}[fragile]
  \transwipe[direction=90]
  \frametitle{Построение ДКА по НКА: алгоритм Томпсона}
  \begin{itemize}
    \item Помещаем в $Queue$ множество $\{ q_0 \}$
    \item Пока очередь не пуста, выполняем:
    \begin{itemize}
      \item $q = Queue.pop()$
      \item Строим множество $q' = \{t = \delta (s, c) | s \in q, c \in \Sigma\}$. Если $q' \notin Queue$, добавить его в очередь. Каждое такое множество --- новая вершина ДКА; добавляем переходы по соответствующим символам
      \item Если во множестве есть хотя бы одна вершина, являющаяся терминальной в данном НКА, то соответствующая вершина ДКА будет конечной
    \end{itemize}
  \item Результат: $\langle \Sigma, Q_d, q_{d_0} \in Q_d, F_d \subset Q_d, \delta_d: Q_d \times \Sigma \rightarrow {Q_d} \rangle$
  \begin{itemize}
    \item $Q_d = \{q_d \mid q_d \subset 2^Q \}$
    \item $q_{d_0} = \{q_0\}$
    \item $F_d = \{q \in Q_d \mid \exists p \in F : p \in q\}$
    \item $\delta_d(q, c) = \{ \delta(a, c) \mid a \in q \}$
  \end{itemize}
  \end{itemize}
\end{frame}

\begin{frame}[fragile]
  \transwipe[direction=90]
  \frametitle{Эквивалентность языков, распознаваемых ДКА и НКА}
  \begin{rutheorem}
  ДКА и НКА распознают один и тот же класс языков
  \end{rutheorem}
  \begin{proof}
  $\Rightarrow: $ очевидно
  
  $\Leftarrow: $ Рассмотрим произвольный НКА и покажем, что алгоритм Томпсона строит по нему эквивалентный ДКА.

  $\forall q \in q_d, \forall c \in \Sigma, \forall p \in \delta(q, c). p \in \delta_d (q_d, c)$

  Рассмотрим $\langle q_0, w_1 w_2 \dots w_m \rangle \vdash \langle u_1, w_2 \dots w_m\rangle  \vdash^* \langle u_m, \varepsilon \rangle, u_m \in F $
  
  $\forall i. u_i \in u_{d_i}, $ где $(q_{d_0}, w_1 w_2 \dots w_m) \vdash (u_{d_1}, w_2 \dots w_m) \vdash^* (u_{d_m}, \varepsilon)$

  $\Mapsto u_m \in u_{d_m}$
  \end{proof}
\end{frame}

%\begin{frame}[fragile]
%  \transwipe[direction=90]
%  \frametitle{Регулярная грамматика}
%  \textbf{Праволинейная грамматика} --- грамматика, все правила которой имеют следующий вид:
%  \begin{itemize}
%    \item $A \rightarrow a B$ или $A \rightarrow a$, где $A, B \in V_N, a \in V_T$
%  \end{itemize}
%
%
%  \textbf{Леволинейная грамматика} --- грамматика, все правила которой имеют следующий вид:
%  \begin{itemize}
%    \item $A \rightarrow B a$ или $A \rightarrow a$, где $A, B \in V_N, a \in V_T$
%  \end{itemize}
%
%\pause 
%
%  \begin{rutheorem}[]
%    Пусть L --- формальный язык. 
%
%    $\exists G_r$ --- праволинейная грамматика, т.ч. $L = L(G_r) \Leftrightarrow \exists G_l$ --- леволинейная грамматика, т.ч. $L = L(G_l) $
%  \end{rutheorem}
%\pause
%  \textbf{Регулярная грамматика} --- праволинейная или леволинейная грамматика
%\end{frame}


\end{document}
