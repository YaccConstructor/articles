\documentclass{beamer}
\usepackage{beamerthemesplit}
\usepackage{wrapfig}
\usetheme{SPbGU}
\usepackage{pdfpages}
\usepackage{amsmath}
\usepackage{cmap} 
\usepackage[T2A]{fontenc} 
\usepackage[utf8]{inputenc}
\usepackage[english,russian]{babel}
\usepackage{indentfirst}
\usepackage{amsmath}
\usepackage{tikz}
\usepackage{array}
\usepackage{multirow}
\usepackage[noend]{algpseudocode}
\usepackage{algorithm}
\usepackage{algorithmicx}
\usepackage[absolute,overlay]{textpos}
\usetikzlibrary{shapes,arrows}
\usepackage{fancyvrb}
\newtheorem{rutheorem}{Теорема}
\newtheorem{ruproof}{Доказательство}
\newtheorem{rudefinition}{Определение}
\newtheorem{rulemma}{Лемма}
\beamertemplatenavigationsymbolsempty

\title[]{Оптимизация алгоритма лексического анализа 
динамически формируемого кода}
% То, что в квадратных скобках, отображается в левом нижнем углу. 
\institute[СПбГУ]{
Санкт-Петербургский государственный университет \\
Кафедра системного программирования }

% То, что в квадратных скобках, отображается в левом нижнем углу.
\author[Александр Байгельдин]{Александр Байгельдин, студент СПбГУ\\
  % У научного руководителя должна быть указана научная степень
  \and  
    {\bfseriesНаучный руководитель:} ст.пр. С.В. Григорьев}

\date{26 апреля 2016г.}

\definecolor{orange}{RGB}{179,36,31}

\begin{document}
{
% Лого университета или организации, отображается в шапке титульного листа
\begin{frame}
  \begin{center}
  {\includegraphics[width=1.5cm]{pictures/SPbGU_Logo.png}}
  \end{center}
  \titlepage
\end{frame}
}

\begin{frame}[fragile]
  \transwipe[direction=90]
  \frametitle{Динамически формируемый код}
  \includegraphics[width=11cm]{pictures/intro_code.pdf} \\
  Условные выражения, циклы, строковые операции (replace, concat) 
\end{frame}
            
\begin{frame}[fragile]
  \transwipe[direction=90]
  \frametitle{Лексический анализ}
  В классическом лексическом анализе применяются конечные преобразователи 
  (Finite State Transducers) \\[1\baselineskip]
  \footnotesize
  \begin{Verbatim}[commandchars=\\\{\}]
      rule token = parse
      | [\textcolor{red}{'0'}-\textcolor{red}{'9'}]  \{ \textcolor{green}{Some}(\textcolor{blue}{NUM}(gr)) \}
      | \textcolor{red}{'+'}  \{ \textcolor{green}{Some}(\textcolor{blue}{PLUS}(gr)) \}
  \end{Verbatim}
  \normalsize
  \includegraphics[width=9cm]{pictures/lexer_.pdf}
\end{frame}

\begin{frame}
  \transwipe[direction=90]
  \frametitle{Регулярная аппроксимация}
  Для динамически формируемого кода можно построить регулярную 
  аппроксимацию \\[2\baselineskip]
  \includegraphics[width=11cm]{pictures/approx_code.pdf} \\
  \includegraphics[width=11cm]{pictures/approx_fsa.pdf}
\end{frame}

\begin{frame}
  \transwipe[direction=90]
  \frametitle{Композиция FST}
  \begin{textblock}{12}(0.5, 3)
    Важной частью лексического анализа динамически формируемого
    кода является операция композиции FST (\textbf{T = T1 $\circ$ T2})
  \end{textblock}
  \begin{textblock}{12}(0.5, 5)
    \includegraphics[width=12cm]{pictures/example_.pdf}
  \end{textblock}
  \begin{textblock}{8}(0.5, 7.5)
    \includegraphics[width=8cm]{pictures/lexer_.pdf}
  \end{textblock}
  \begin{textblock}{12}(0.5, 12)
    \includegraphics[width=12cm]{pictures/res_.pdf}
  \end{textblock}
\end{frame}

\begin{frame}
  \transwipe[direction=90]
  \frametitle{YaccConstructor}
  \begin{itemize}
    \item YaccConstructor — исследовательский проект в области 
    лексического и синтаксического анализа
    \item В YaccConstuctor для лексического анализа динамически формируемого кода
    применяется подход, в основе которого лежит построение  регулярной аппроксимации и 
    композиция FST
    \item Реализованный в YaccConstructor алгоритм композиции FST обладает недостаточной
    производительностью
  \end{itemize}
\end{frame}

% Обязательный слайд: четкая формулировка цели данной работы и постановка задачи
% Описание выносимых на защиту результатов, процесса или особенностей их достижения и т.д.
\begin{frame}
  \transwipe[direction=90]
  \frametitle{Постановка задачи}
  \textbf{Целью} работы является исследование возможности улучшения
  производительности лексического анализа динамически 
  формируемого кода\\[2\baselineskip]

  \textbf{Задачи}:
  \begin{itemize}
    \item Исследовать различные алгоритмы композиции FST
    \item Реализовать наиболее оптимальный алгоритм композиции
    \item Сравнить производительность реализаций текущего и выбранного алгоритмов
  \end{itemize}
\end{frame}

\begin{frame}
  \transwipe[direction=90]
  \frametitle{Решение}
  \begin{itemize}
    \item Временная сложность текущего алгоритма: \\
    \[O(V_1 * V_2 * D_1 * D_2)\]
    где E — число ребер, V — число вершин, D — максимальное количество исходящих ребер
    \item В текущем алгоритме образуются недостижимые вершины, которые приходится удалять
    \item Временная сложность выбранного алгоритма: \\
    \[O(V_1 * V_2 * D_1 * (log(D_2) + M_2))\]
    где M — степень недетерминированности
    \item В выбранном алгоритме недостижимых вершин не образуется
  \end{itemize}
\end{frame}

\begin{frame}
  \transwipe[direction=90]
  \frametitle{Измерения}
    \begin{center}
        \begin{tabular}{ | p{1.5cm} | p{1.5cm} | p{2.4cm} | p{2.6cm} | p{1.7cm} | }
        \hline
        Кол-во вершин & Кол-во ребер & 
        Время работы текущего алгоритма (мс) & 
        Время работы выбранного алгоритма (мс) & Ускорение (раз) \\ \hline
        700 & 2652 & 17215 & 4885 & 3.5 \\ \hline
        74 & 4452 & 2964 & 687 & 4.0 \\ \hline
        711 & 1766 & 15451 & 2565 & 6.0 \\ \hline
        250 & 738 & 8708 & 1414 & 6.0 \\ \hline
        128 & 374 & 2976 & 411 & 7.5 \\ \hline
        31 & 195 & 279 & 36 & 8.0 \\ \hline
        %326 & 550 & 11574 & 7070 & 1.5 \\ \hline
        \end{tabular}
    \end{center}
\end{frame}

\begin{frame}
  \transwipe[direction=90]
  \frametitle{Технологии}
  \begin{itemize}
    \item F\# — язык семейства .NET с уклоном в функциональную парадигму
    \item YaccConstructor — исследовательский проект в области лексического и синтаксического анализа
    \item QuickGraph — библиотека .NET для работы с графами
  \end{itemize}
\end{frame}

\begin{frame}
  \transwipe[direction=90]
  \frametitle{Результаты}
  \begin{itemize}
    \item Исследованы различные алгоритмы композиции FST
    \item Найден алгоритм, обладающий наилучшей производительностью
    \item Произведено сравнение производительности реализаций текущего и выбранного алгоритмов
  \end{itemize}
\end{frame}

\end{document}