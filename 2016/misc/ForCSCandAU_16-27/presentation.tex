\documentclass{beamer}
\usepackage{beamerthemesplit}
\usepackage{wrapfig}
\usetheme{SPbGU}
\usepackage{pdfpages}
\usepackage{amsmath}
\usepackage{cmap} 
\usepackage[T2A]{fontenc} 
\usepackage[utf8]{inputenc}
\usepackage[english,russian]{babel}
\usepackage{indentfirst}
\usepackage{amsmath}
\usepackage{tikz}
\usepackage{multirow}
\usepackage[noend]{algpseudocode}
\usepackage{algorithm}
\usepackage{algorithmicx}
\usetikzlibrary{shapes,arrows}
\usepackage{fancyvrb}
\usepackage{minted}
\usepackage{verbments}


\newtheorem{rutheorem}{Теорема}
\newtheorem{ruproof}{Доказательство}
\newtheorem{rudefinition}{Определение}
\newtheorem{rulemma}{Лемма}
\beamertemplatenavigationsymbolsempty

\title[]{YaccConstructor}
\subtitle[YaccConstructor]{Задачи на осенний семестр 2016}
% То, что в квадратных скобках, отображается в левом нижнем углу. 
\institute[]{
Лаборатория языковых инструментов JetBrains \\
Санкт-Петербургский государственный университет \\
Математико-механический факультет }

% То, что в квадратных скобках, отображается в левом нижнем углу.
\author[Семён Григорьев]{Семён Григорьев}

\date{8 февраля 2016г.}

\definecolor{orange}{RGB}{179,36,31}

\begin{document}
{
\begin{frame}[fragile]
  \begin{tabular}{p{2.5cm} p{5.5cm} p{2cm}}
   \begin{center}
      \includegraphics[width=2cm]{pictures/JBLogo3.pdf}
    \end{center}
    &
    \begin{center}
      \includegraphics[width=1cm]{pictures/SPbGU_Logo.png}
    \end{center}
    &
    \begin{center}
%      \includegraphics[width=1cm]{pictures/YC_logo}
    \end{center} 
  \end{tabular}
  \titlepage
\end{frame}
}

\begin{frame}[fragile]
  \transwipe[direction=90]
  \frametitle{YaccConstructor}
  \begin{itemize}
    \item Исследования в области лексического и синтаксического анализа
    \item Открытый исходный код
    \begin{itemize}
      \item \url{https://github.com/YaccConstructor}
    \end{itemize}
    \item Основной язык разработки --- F\#
  \end{itemize}
\end{frame}

\begin{frame}
  \transwipe[direction=90]
  \frametitle{Требования к знаниям и навыкам}
  \begin{itemize}
    \item Знакомство с функциональным программированием (F\# или OCaml)
    \item Умение читать и понимать научные статьи
    \item Умение читать и понимать чужой код
    \item Навыки работы с Git/GitHub, Microsof VisualStudio
  \end{itemize}
\end{frame}

\begin{frame}[plain,c]
 \transwipe[direction=90]
 \begin{center}
  \Huge Применение F\# для создания облачных приложений
 \end{center}
\end{frame}

\begin{frame}[fragile]
\transwipe[direction=90]
\frametitle{Задачи}
\begin{itemize}
\item Изучить средства разработки облачных приложений, предоставляемые F\#
\item Выбрать средство, наиболее подходящее для решения задачи распределённой обработки графов в контексте проекта YaccConstructor
\item Реализовать решение на основе выбранных средста, сравнить с исходным.
\item Облачное решение по распределённому синтаксическому анализу графов;
\item Базовые знаниея F\# или другого функционального языка программирования, знания git, навыки работы с платформой .NET, понимание принципов облачных архитектур.


\item  Использование F\# для программирования GPGPU;
\item  Устранить ряд известных проблем в трансляторе F\# в OpenCL, реализовать возможность использования вызовов готовых OpenCL-процедур в коде на F\#, исследовать возможности генерации кода на лету для ускорения вычислений;
\item  Улучшеный транслятор с возможностью использовать стороние процедуры. По результатам исследований -- приложение с улучшенной производительностью благодаря кодогенерации на лету;
\item  Базовые знаниея F\# или другого функционального языка программирования, знания git, навыки работы с платформой .NET, понимание особенностей архитектуры GPGPU и разработки для неё.

\item Синтаксический анализ графов (graph parsing)
\end{itemize}

\end{frame}

\begin{frame}[fragile]
  \transwipe[direction=90]
  \frametitle{Задача: применение анализа строковых выражений для JavaScript eval}
  \begin{itemize}
    \item Цель: трансляция стандартного \texttt{eval} в "безопасный"
    \begin{itemize}
        \item Martin Lester. Information Flow Analysis for a Dynamically Typed Functional Language with Staged Metaprogramming
        \item Martin Lester. Analysing Eval using Staged Metaprogramming
    \end{itemize} 
    \item Исследовательская задача: диплом, публикации
    \item Разбивается на 2 подзадачи
    \begin{itemize}
        \item Получение аппроксимации
        \item Трансляция SPPF в "безопасный \texttt{eval}"        
    \end{itemize} 
  \end{itemize}
\end{frame}

\begin{frame}
  \transwipe[direction=90]
  \frametitle{Задача: использование SPPF в абстрактном синтаксическом анализе}
  \begin{itemize}
    \item Абстрактный синтаксический анализ --- один из подходов к анализу динамически формируемого кода
    \begin{itemize}
        \item K. G. Doh, H. Kim, D. A. Schmidt. Static Validation of Dynamically Generated HTML Documents Based on Abstract Parsing and Semantic Processing
        \item \href{https://github.com/YaccConstructor/articles/blob/master/2015/PSI/paper/psi_2015.pdf}{E. Verbitskaia, S. Grigorev, D. Avdyukhin. Relaxed Parsing of Regular Approximations of String-Embedded Languages}
    \end{itemize} 
    \item Реализовать вычисление семантики по статьям
    \item Сравнить с нашим подходом
    \item А можно ли использовать SPPF
    \item Диплом, публикации
  \end{itemize}
\end{frame}

\begin{frame}[plain,c]
 \transwipe[direction=90]
 \begin{center}
  \Huge Средства для сертификацонного программирования \\ или \\ $F\# + F^* = \, ?$
 \end{center}
\end{frame}

\begin{frame}[fragile]
  \transwipe[direction=90]
  \frametitle{Сертификационное программирование}
%  \begin{tabular}[t]{p{6cm} p{8cm}}
  \begin{itemize}
    \item \underline{\bfseries{$F^*$} (\url{https://www.fstar-lang.org/tutorial/})}
    \item Coq
    \item Agda
    \item ...
  \end{itemize}

\fvset{frame=lines,framesep=5pt}
\begin{pyglist}[language=ocaml]
val sort: l:list int -> 
          Tot (m:list int{sorted m 
                          /\ (forall i. mem i l = mem i m)})
          (decreases (length l))
let rec sort l = match l with
  | [] -> []
  | pivot::tl ->
    let hi, lo = partition (cmp pivot) tl in
    let l' = append (sort lo) (pivot::sort hi) in
    dedup l' 
\end{pyglist}

\begin{verbatim}
\end{verbatim}

%\end{tabular}
\end{frame}


\begin{frame}
  \transwipe[direction=90]
  \frametitle{Задача: объединение $F\#$ и $F^*$}
  \begin{itemize}
    \item $F\# + F^* = F\#^*$
    \item Парсер для $F\#^*$
    \item Транслятор из AST $F\#$ в AST $F^*$
    \item Разбивается на подзадачи (до 3 человек)
    \item Диплом (вся задача), публикации
  \end{itemize}
\end{frame}

\begin{frame}
  \transwipe[direction=90]
  \frametitle{Задача: поддержка $F\#^*$ в Microsoft Visual Studio}
  \begin{itemize}
    \item Поддержать в модели проекта, редакторе, отладчике
      \begin{itemize}
        \item Создание файлов, шаблоны
        \item Подсветка синтаксиса
        \item Сообщения об ошибках, подсветка ошибок
        \item ....
      \end{itemize}
    \item Разбивается на подзадачи (до 4 человек)
    \item Диплом (вся задача), публикации
  \end{itemize}
\end{frame}

\begin{frame}
  \transwipe[direction=90]
  \frametitle{Задача: межъязыковое взаимодействие $F\#$ и $F^*$}
  \begin{itemize}
    \item Использовать функции, написанные на $F^*$ в $F\#$(.NET)
    \item Использовать функции, написанные на $F\#$(.NET) в $F^*$
    \item Сохранить типизацию/вывод типов
    \item Разбивается на подзадачи (до 2 человек)
    \item Диплом (вся задача), публикации
  \end{itemize}
\end{frame}
            
\begin{frame}
\transwipe[direction=90]
\frametitle{Контакты}
\begin{itemize}
  \item Почта: \url{rsdpisuy@gmail.com}
  \item Исходный код YaccConstructor: \url{https://github.com/YaccConstructor}
  \item Google+ сообщество: \url{https://goo.gl/DuPWkM}
\end{itemize}
\end{frame}
\end{document}
